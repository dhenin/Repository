\addcontentsline{toc}{chapter}{\thechapter ~ Les variables}
\setcounter{section}{0}
\markboth{Variables}{Variables}
\centerline{\Large\bf Les variables}
\label{Variable}
\index{variable}
\index{scalaire}
\index{identificateur}
\index{valeur}
\index{value, (left, right)}
\index{m\'emoire (zone)}
\index{left (value)}
\index{right (value)}
\index{type (de variable)}
\index{entier}
\index{r\'eel}
\index{caract\`ere}
\index{tableau}
\index{agr\'egat}
\index{enregistrement}
\index{structure}
\index{pointeur}
\label{pointeur}
\index{Mammeri}
 
 \noindent\hrulefill  

\begin{multicols}{2}
\section*{Les scalaires}

Il ne faut pas confondre l'{\em identificateur} d'une variable et la
{\em valeur} de la variable. La valeur d'une variable est conserv\'ee dans une
zone m\'emoire. 


Si {\small \tt SOMME} vaut {\tt 10}, la notation algorithmique
{\tt
\begin{quote}
SOMME = SOMME + 5
\end{quote}
}

affecte\footnote{Page~\pageref{Affectation} la propri\'et\'e de l'affectation}
\`a la zone  {\small \tt SOMME}
la {\em valeur pr\'ec\'edente} de {\small \tt SOMME} + 5
(soit {\tt 15}) .

Autre exemple~:
{\tt
\begin{quote}
      SOMME = SOMME + SOMME
\end{quote}
}
fait passer la zone valeur {\small \tt SOMME} de {\tt 10} \`a {\tt 20}.

Les termes de {\it left value} et {\it right value} d\'esignent respectivement
la zone m\'emoire et la valeur qu'elle rec\`ele.

On consid\`ere des objets informatiques et non plus math\'ematiques.
Un objet informatique est caract\'eris\'e par le fait qu'il a non seulement une
valeur mais une identit\'e (ou une place dans la m\'emoire si l'on pr\'ef\`ere).
Ainsi, en math\'ematique il n'existe qu'un objet 392 alors que dans un syst\`eme
informatique on pourra avoir deux objets diff\'erents qui ont cette valeur (deux
places dans la m\'emoire qui contiennent cette valeur). Il s'agira donc de bien
faire la distinction entre identit\'e (le m\^eme objet) et \'egalit\'e (la m\^eme
valeur).

Un ordinateur manipule des {\it repr\'esentations} de valeurs, qui sont des
configurations de bits, d'octets ou de mots de la m\'emoire. Comme les
repr\'esentations physiques varient selon les objets, on est conduit \`a sp\'ecifier
leurs types. 


\subsection* {Les entiers} sont en nombre fini dans l'ordinateur,
      donc certaines op\'erations peuvent cr\'eer un d\'ebordement, {\em i.e.} fournir 
      un r\'esultat hors des limites de l'intervalle. Le langage C ne teste 
      jamais le r\'esultat.

\subsection*{Les r\'eels} L'ensemble est discontinu,
      la norme {\sc ieee 754} d\'ecrit les nombres \`a virgule flottante.

\setbox0=\vbox{\hsize=70mm \begin{minipage}[t]{7cm}
\listing{Prog/max_int.c}
\end{minipage} }


La longueur effective (en {\em octets}), et donc les bornes extr\^emes, d\'ependent
\'evidemment de la machine. L'information est souvent disponible.

\begin{figurette}

\centerline{\begin{tabular}{|l|}
\hline
~\\
\box0\\
~\\
\hline
\end{tabular} }
\end{figurette}
\addcontentsline{lof}{section}{Programme pour afficher les nombres limites}

\subsection*{Les caract\`eres} ne sont pas les m\^emes selon le pays, 
      l'ensemble des caract\`eres est un sous-ensemble des entiers\footnote{
Page \pageref{TableAscii} une table de caract\`eres}. Pour \unix\ un ensemble
de variables d�finit la langue dans laquelle s'affichent les messages et l'ordre
du tri alphab�tique.


\section*{Tableaux}

Il est fr\'equent de manipuler des donn\'ees de m\^eme type formant une collection
que l'on nomme {\em tableau}. Chaque valeur est d\'esign\'ee par le {\em nom} du
tableau et d'un ou plusieurs {\em indice(s)} ; ce nom peut \^etre manipul\'e comme
celui d'une variable. Un tableau \`a une seule dimension est souvent appel\'e {\em
vecteur}. On peut utiliser des tableaux \`a $n$ dimensions.


\section*{Les variables composites}

Il est parfois n\'ecessaire de manipuler des ensembles de donn\'ees formant un
tout ; ces ensembles sont nomm\'es agr\'egats, enregistrements ou
structures\footnote{B. Mammeri, {\em Programmation}, p. 112.}.

\section*{Les pointeurs}

Un pointeur est une variable dont la valeur donne l'acc\`es \`a une {\em autre}
variable (elle contient en quelque sorte l'adresse de celle-ci)\footnote{B.
Mammeri {\em Programmation}, p. 104.}.
\end{multicols}

\newpage

\addtocounter{section}{1}
\addcontentsline{toc}{section}{La norme IEEE 754}
\markboth{La norme IEEE754}{La norme IEEE754}
\centerline{\Large\bf La norme IEEE 754}
\label{IEEE754}
 \noindent\hrulefill  

\begin{multicols}{2} 
La norme IEEE\footnote{%
The {\em Institut of  electrical and Electronics Engineers}.%
}{\it Standard for binary Floating-Point Arithmetic}
(ANSI/IEEE Standard 754 - 1985) a \'et\'e d\'efinie dans le but d'am\'eliorer
la qualit\'e du calcul flottant et la portabilit\'e des applications.
Ce standard est maintenant utilis\'e et respect\'e par tous les acteurs
principaux du calcul scientifique.
 Deux formats principaux 32 et 64 bits (voir
figure ) et quatre modes d'arrondis (vers $\infty$, vers $-\infty$,
vers 0, au plus pr\`es)
sont d\'efinis, ainsi que des formats dits {\it \'etendus}.

\centerline{ \begin{tabular}{|c|c|c|}
\multicolumn{1}{l}{31} &
\multicolumn{1}{c}{30 \hfill 23} &
\multicolumn{1}{c}{22 \hfill 0} \\
\hline
S & E & M \\
signe & exposant & mantisse \\
\hline 
\multicolumn{1}{c}{1} & \multicolumn{1}{c}{8} & \multicolumn{1}{c}{23} \\
\multicolumn{3}{c}{} \\
\multicolumn{3}{c}{Format flottant simple pr\'ecision} \\
\multicolumn{3}{c}{} \\
\multicolumn{1}{l}{63} &
\multicolumn{1}{c}{62 \hfill 52} &
\multicolumn{1}{c}{51 \hfill 0} \\
\hline
S & E & M \\
signe & exposant & mantisse \\
\hline 
\multicolumn{1}{c}{1} & \multicolumn{1}{c}{11} & \multicolumn{1}{c}{52} \\
\multicolumn{3}{c}{} \\
\multicolumn{3}{c}{Format flottant double pr\'ecision} \\
\multicolumn{3}{c}{} \\
\end{tabular} }

Le nombre repr\'esent\'e par le flottant (S,E,M) est \( (-1)^{s} \times
(1 + M) \times 2^{(\mbox{ exposant - biais})} \) 
 o\`u : {\it biais = 127} pour les flottants simple pr\'ecision et
{\it biais = 1023} pour les flottants double pr\'ecision\footnote{
Pour \^etre complet, la repr\'esentation machine des nombres flottants est
l\'eg\`erement diff\'erente en {\sc ieee}. En effet, on s'arrange pour que le nombre
0 puisse \^etre repr\'esent\'e par le mot machine dont tous les bits sont \`a 0, et on
additionne la partie exposant du mot machine flottant de $e_{min}$,
c'est-\`a-dire de 127 en simple pr\'ecision, ou de 1023 en double pr\'ecision.}
; la mantisse (M)
est cod\'ee sur 23 bits pour les flottants simple pr\'ecision et sur 52 bits
pour les flottants double pr\'ecision.
\par 
Les microprocesseurs MIPS R10000 et UltraSPARC supportent les formats
flottants IEEE 32 et 64 bits. De plus, le jeu d'instructions SPARC
V9 int\`egre les flottants 128 bits ; cependant ces op\'erations ne sont
pas support\'ees par mat\'eriel sur l'UltraSPARC, mais \'emul\'ees. L'unit\'e
flottante du PentiumPro comme les autres microprocesseurs xxx86 manipule
seulement des flottants 80 bits dans un format dit $<$$<$\'etendu$>$$>$ :
64 bits de mantisse, 15 bits d'exposant et 1 bit de signe.
\par Source : {\it flloansi@IRISA.irisa.fr Tue Jun  4 09:57 MET DST 1996}

% Pour rendre portables des programmes utilisant les nombres r\'eels
% ({\em ie flottants}), une norme {\sc ieee 754} ({\em
% the Institut of  electrical and Electronics Engineers
% }) a \'et\'e d\'efinie. Non seulement elle
% d\'ecrit les bornes des nombres flottants, mais
La Norme donne une convention pour
 repr\'esenter des valeurs sp\'eciales : $\pm\infty$, {\sc n}a{\sc n} ({\em not a
 number}) qui permettent de donner des valeurs \`a des divisions par z\'ero, ou \`a
 des racines carr\'ees de nombres n\'egatifs par exemple. Les valeurs sp\'eciales
 permettent d'\'ecrire des programmes de calculs de racines de fonctions
 \'eventuellement discontinues.

 La norme {\sc ieee 754} est la suivante :

 \begin{center} \begin{tabular}{lcl}
 & & \\
 \hline
 \multicolumn{1}{c}{\hbox to 2cm {\small\hfil Exposant\hfil}}&
 \multicolumn{1}{c}{\hbox to 2 cm{\small\hfil Mantisse\hfil}} &
 \multicolumn{1}{c}{\hbox to 2 cm{\small\hfil Valeur\hfil}}\\
 \hline
 $e = e_{min} - 1$ & $f = 0$ & $\pm0$ \\
 $e = e_{min} - 1$ & $f \neq 0 $ & $0,f \times 2^{e_{min}}$\\
 $e_{min}\leq e \leq e_{max}$  & & $1,f \times 2^{e}$ \\
 $e = e_{max} + 1$ & $f = 0 $ & $ \pm\infty$ \\
 $e = e_{max} + 1$ & $f \neq 0$ & {\sc n}a{\sc n} \\
 \hline
 & & \\
 \end{tabular} \end{center} 
%
% et les formats en bits (simple et double) pr\'ecision sont
%
% \begin{center} \begin{tabular}{lrr}
% & & \\
% \hline
% \multicolumn{1}{c}{\hbox to 2cm {\small\hfil Param\`etre\hfil}}&
% \multicolumn{1}{c}{\hbox to 2 cm{\small\hfil Simple\hfil}} &
% \multicolumn{1}{c}{\hbox to 2 cm{\small\hfil Double\hfil}}\\
% \hline
% $p$   & 23 & 52 \\
% $q$   & 8  & 11 \\
% $e_{max}$ & +127 & +1023 \\
% $e_{min}$ & -126 & -1022 \\
% {\small Taille totale du mot} & 32 & 64 \\
% \hline
% & & \\
% \end{tabular} \end{center} 
%
%On en d\'eduit donc que la valeur absolue de tout nombre flottant $x$ v\'erifie en
%simple pr\'ecision 
%
%\[ 10^{-45}\simeq 2^{-150}\leq\mid x \mid< 2^{128}\simeq 3\times 10^{28} \]
%

La pr\'ecision d'un nombre flottant est $2^{-23} \simeq 10^{-7}$ 
en simple pr\'ecision et $2^{-52} \simeq 2 \times 10^{-16}$ en double pr\'ecision.
On perd donc 2 \`a 4 chiffres de pr\'ecision par rapport aux op\'erations enti\`eres.
Il faut comprendre aussi que les nombres flottants sont align\'es avant toute
addition ou soustraction, ce qui entra\^\i ne des pertes de pr\'ecision. Par exemple,
l'addition d'un tr\`es petit nombre \`a un grand nombre\footnote{Page
\pageref{Permutation}} va laisser ce dernier inchang\'e. Il y a alors
d\'epassement de capacit\'e vers le bas ({\em underflow}). Un bon exercice est de
montrer que la s\'erie harmonique converge en informatique flottante, ou que
l'addition flottante n'est pas associative ! Il y a aussi des d\'ebordements de
capacit\'e vers le haut ({\em overflow}). Ces derniers sont en g\'en\'eral plus
souvent test\'es que les d\'epassements vers le bas.

\end{multicols}
