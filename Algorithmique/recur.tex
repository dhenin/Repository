\addcontentsline{toc}{section}{La r�curcivit�}
\addcontentsline{lof}{section}{Calcul de $x^{n}$ $1^{\grave ere}$ m\'ethode}
\addcontentsline{lof}{section}{Calcul de $x^{n}$ $2^{\grave eme}$ m\'ethode}
\addcontentsline{lof}{section}{Calcul de $x^{n}$ $3^{\grave eme}$ m\'ethode}
\markboth{R\'ecursivit\'e}{R\'ecursivit\'e}
% \centerline{\Large\bf Sch\'ema fonctionnel (\I\I)}
\index{r\'ecursivit\'e}
\index{puissance (de $x$)}
\index{r\'ecurrence}

 
\noindent\hrulefill  

\begin{multicols}{2} 
%\paragraph{\bf 1. La r\'ecursivit\'e}
\section*{La r\'ecursivit\'e}

Reprenons le calcul de $x^{n}$. On suppose que l'on a fait une partie du
travail, disons on a calcul\'e $x^{i}$. C'est fini si $i= n$. Sinon, en
multipliant le r\'esultat d\'ej\`a calcul\'e par $x$, on obtient $x^{i+1}$ Par
changement de $i$ en $i + 1$, on se ram\`ene \`a l'hypoth\`ese de d\'epart. Pour
d\'emarrer, on peut prendre $i = 1$ et $x^{i} = x$.

  La r\'ecurrence joue donc de la fa\c con suivante:
\begin{itemize}
\item si j'ai pu calculer $ x^{i}$, je peux calculer $x^{i+1}$
\item or je peux calculer $x^{1}$.
\end{itemize}
  Sans rien changer \`a cette forme de raisonnement, nous pouvons construire un
algorithme ``r\'ecursif''.

\begin{verbatim}  expr(x, n) 
    {
    if (n == 1) return x ;
    else return x * expr(x, n-1) ;
    } \end{verbatim} 

  Pour \'ecrire la d\'efinition r\'ecursive de {\tt exp$(x, n)$}, nous avons
utilis\'e la propri\'et\'e bien connue
{\tt \begin{quote} 
     $x^{n} =x\times x^{n-1}$
\end{quote} }


\section*{Construction}

 Reprenons le calcul de $x^{n}$
\begin{quote} \begin{tabular}{l}
{\tt exp$(x, n) = \underbrace{x \times  x \times  x \times  ... \times  x}_{(n x {\rm\ dans\;cette\;formule})}$}\\ 
Encadrons les $n - 1$ $x$ de droite \\
  {\tt exp$(x, n)=x \times$ \fbox{$x\times x\times ... \times x$}} \\
  \multicolumn{1}{r}{($n-1$ $x$ dans la bo\^\i te)} \\
\end{tabular} \end{quote} 
On retrouve la relation connue
{\tt \begin{quote} 
     exp$(x,n) =x \times$ exp$(x,n -1)$
\end{quote} }
Comme cas singulier, on peut prendre {\tt exp$(x, 1) = x$} ou {\tt
exp$(x, 0) = 1 $} (qui a l'avantage d'\'etendre la d\'efinition au cas $n = 0$)
%{\tt \begin{quote} 
% \fbox{exp1$(x, n)\equiv$ SI $n=0$ ALORS $1$ SINON $x \times$ exp1$(x, n-1)$ IS}
%\end{quote} }
\begin{quote} \begin{tabular}{|l|}
\hline
\begin{minipage}[t]{5.5cm}
\begin{verbatim}  
exp(x, n)
    {
    if (n == 0)
        return 1 ;

    return x * exp(x, n-1) ;
    }

\end{verbatim}
\end{minipage} \\
\hline
\end{tabular} \end{quote}  


%Nous en discuterons plus tard la complexit\'e.
On peut grouper autrement
les $x$ dans {\tt exp$(x,n)$}. Supposons $n$ pair. On peut partager les
$x$ en deux paquets \'egaux~:

{\tt
 exp$(x,n)=$ \fbox{$x\times x\times\ldots\times x$} $\times$ \fbox{$x\times x\times\ldots\times x$}
 }
avec $\frac{n}{2}$ $x$ dans chaque paquet.\\
Dans ce cas
{\tt \begin{quote} 
     exp$(x, n) =$ exp$(x,\frac{n}{2})^{2}$
\end{quote} }

%\begin{tabbing}
%ou encore, en posant \= {\tt $y = $exp$(x, \frac{n}{2})$ \\
%                    \> exp$(x, n) = y \times y $} \\
%\end{tabbing}

Si maintenant $n$ est impair, nous pouvons faire de m\^eme, en mettant \`a part le
premier $x$

{\tt 
     exp$(x,n)=x\times$ \fbox{$x\times x\times\ldots\times x$} $\times$ \fbox{$x\times \ldots\times x$}
 }

On obtient ainsi une deuxi\`eme d\'efinition~: 

\begin{quote} \begin{tabular}{|l|}
\hline
\begin{minipage}[t]{5cm}
\begin{verbatim}  
exp2(x, n)
    {
    if (n == 0)
        return 1 ;

    y = expr2(x, n/2) ;
    if (pair(n))
        return y * y ;
    return y * y * x ;
    }

\end{verbatim}
\end{minipage} \\
\hline
\end{tabular} \end{quote}  

%Dans cette d\'efinition, la clause {\tt AVEC $(y =$ exp2$(x, n \div 2))$}
%donne la signification d'une variable interm\'ediaire qui a \'et\'e introduite
%pour la commodit\'e des calculs.

   Nous pouvons faire, quand $n$ est pair, les groupements d'une autre fa\c con, en
enfermant les paires de $x$ dans des bo\^\i tes.
{\tt \begin{quote} 
exp$(x, n) =$ \fbox{$x\times x$} $\times$ $\ldots\times$ \fbox{$x\times x$}

\end{quote} }
Il y a $\frac{n}{2}$ bo\^\i tes ayant chacune deux $x$

\begin{tabular}{|l|}
\hline
\begin{minipage}[t]{6cm}
\begin{verbatim}  
exp3(x, n)
    {
    if (!n) return 0 ;
    if (pair(n))
        return exp3(x*x, n/2) ;
    return x * exp3(x*x, n/2) ;
    }

\end{verbatim}
\end{minipage} \\
\hline
\end{tabular}
\end{multicols}
