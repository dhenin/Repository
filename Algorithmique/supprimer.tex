\addcontentsline{toc}{section}{Suppression/Insertion d'un maillon}
\markboth{Suppression/Insertion d'un maillon}{Suppression/Insertion d'un maillon}
\centerline{\Large\bf Suppression/Insertion d'un maillon de liste}
 \index{liste}
 \index{ins\'erer (un \'el\'ement dans une liste)}
 \index{supprimer (un \'el\'ement dans une liste)}
 
 \noindent\hrulefill  
\begin{multicols}{2}


%\begin{figurette}
%	\input{Figures/espace}
%	\CAPTION{Listes cha\^\i n\'ees correspondant \`a la fig. \ref{lfauxp}}
%	\label{Espace}
%\end{figurette}
%
%\index{maillon} \index{cha\^\i n\'ee (liste)} \index{cha\^\i ne (de maillons)}
%    On peut repr\'esenter les listes du tableau par des cha\^\i nes de maillons
%    li\'es entre eux comme sur la figure \ref{Espace}. 
%
\stepcounter{section}
\stepcounter{subsection}
\subsection*{Suppression d'un \'el\'ement}
\addcontentsline{lof}{section}{Suppression d'un \'el\'ement d'une liste}



On a l'habitude de repr\'esenter une liste cha\^\i n\'ee par une suite de bo\^\i tes
(maillons) \`a 2 compartiments comme dans la figure \ref{suppression}.
Le compartiment de gauche contient
une donn\'ee (ou l'adresse d'une donn\'ee) tandis que le compartiment de droite
contient l'adresse du maillon {\it suivant}. 

Une structure comprenant 3 adresses autorise l'acc\`es imm\'ediat au maillon de
t\^ete, au maillon courant et au dernier maillon. 

\begin{figurette}
        \includegraphics{Figures/suppression.ps}
	\label{suppression}
\medskip
\centerline{{\sc Fig.} \thesubsection -- {\it Suppression de l'\'el\'ement {\tt x}}}
\addcontentsline{lof}{section}{Suppression d'un \'el\'ement}
\end{figurette}


La suppression du maillon qui contient $x$ s'effectue en modifiant la
valeur de $suivant$ contenue dans le pr\'ed\'ecesseur de $x$ : la fonction
$dispose$ restitue \`a la liste $disponible$ la place lib\'er\'ee, celle-ci pourra
\^etre r\'eutilis\'ee lors de la cr\'eation d'un $nouveau$ maillon.

\begin{verbatim}  
static BOOL
LibereObject (List * l, void * ancien)
{   
  Maillon * me = l->premier ;
  if (-1 != me)
  {
    Maillon * precedent ;
    while (-1 != me->suivant)
    {
      precedent = me   ;
      me = me->suivant ;
      /* La comparaison qui suit
       * doit etre adaptee */
      if (me->objet == ancien)
      {
        precedent->suivant = me->suivant ;
        dispose(me) ;
        return VRAI ; /* Succes */
      }
    }
  } /* objet non trouve ou liste vide */
  return FAUX ;
}   \end{verbatim} 


\stepcounter{subsection}
\subsection*{Insertion d'un \'el\'ement}

\addcontentsline{lof}{section} {Insertion d'un \'el\'ement en t\^ete d'une
liste}

      Pour ins\'erer un \'el\'ement $x$ dans une liste $L$, on utilise la
    premi\`ere cellule libre dans la liste disponible et on la place \`a la
    bonne position dans la liste $L$.  L'\'el\'ement $x$ est ensuite plac\'e
    dans le champ \'el\'ement de cette cellule. Figure \ref{Lfauxp2}.

	  Ins\'erer un nouvel \'el\'ement en t\^ete de la liste ${\cal M}$ peut
          s'\'ecrire de fa\c con simplifi\'ee :

\begin{verbatim}  
inserer (M, e)
{
        temp = M ;
           M = premier_disponible() ;
  disponible = suivant(M) ;
  suivant(M) = temp ;
  element(M) = e ;
}
\end{verbatim}

\begin{figurette}
    \setlength{\unitlength}{0.00087500in}%
%
\begingroup\makeatletter\ifx\SetFigFont\undefined%
\gdef\SetFigFont#1#2#3#4#5{%
  \reset@font\fontsize{#1}{#2pt}%
  \fontfamily{#3}\fontseries{#4}\fontshape{#5}%
  \selectfont}%
\fi\endgroup%
\begin{picture}(3312,2646)(751,-2773)
\put(3346,-211){\makebox(0,0)[lb]{\smash{\SetFigFont{10}{12.0}{\rmdefault}{\mddefault}{\itdefault}\special{ps: gsave 0 0 0 setrgbcolor}ESPACE\special{ps: grestore}}}}
\thicklines
\special{ps: gsave 0 0 0 setrgbcolor}\put(1951,-1111){\line( 1, 0){225}}
\put(2176,-1111){\line( 4,-3){600}}
\put(2776,-1561){\vector( 1, 0){375}}
\special{ps: grestore}\special{ps: gsave 0 0 0 setrgbcolor}\put(1876,-1561){\line( 1, 0){300}}
\put(2176,-1561){\line( 3,-5){291.176}}
\put(2476,-2041){\vector( 2,-1){684}}
\special{ps: grestore}\special{ps: gsave 0 0 0 setrgbcolor}\put(1576,-2536){\framebox(450,225){}}
\special{ps: gsave 0 0 0 setrgbcolor}\put(1576,-1711){\framebox(450,225){}}
\special{ps: gsave 0 0 0 setrgbcolor}\put(1576,-1186){\framebox(450,225){}}
\special{ps: gsave 0 0 0 setrgbcolor}\put(3151,-2536){\line( 1, 0){900}}
\special{ps: grestore}\special{ps: gsave 0 0 0 setrgbcolor}\put(3151,-2311){\line( 1, 0){900}}
\special{ps: grestore}\special{ps: gsave 0 0 0 setrgbcolor}\put(3151,-2086){\line( 1, 0){900}}
\special{ps: grestore}\special{ps: gsave 0 0 0 setrgbcolor}\put(3151,-1861){\line( 1, 0){900}}
\special{ps: grestore}\special{ps: gsave 0 0 0 setrgbcolor}\put(3151,-1636){\line( 1, 0){900}}
\special{ps: grestore}\special{ps: gsave 0 0 0 setrgbcolor}\put(3151,-1411){\line( 1, 0){900}}
\special{ps: grestore}\special{ps: gsave 0 0 0 setrgbcolor}\put(3151,-1186){\line( 1, 0){900}}
\special{ps: grestore}\special{ps: gsave 0 0 0 setrgbcolor}\put(3151,-961){\line( 1, 0){900}}
\special{ps: grestore}\special{ps: gsave 0 0 0 setrgbcolor}\put(3151,-736){\line( 1, 0){900}}
\special{ps: grestore}\special{ps: gsave 0 0 0 setrgbcolor}\put(3601,-511){\line( 0,-1){2250}}
\special{ps: grestore}\special{ps: gsave 0 0 0 setrgbcolor}\put(3151,-2761){\framebox(900,2250){}}
\put(751,-2461){\makebox(0,0)[lb]{\smash{\SetFigFont{10}{12.0}{\rmdefault}{\bfdefault}{\updefault}\special{ps: gsave 0 0 0 setrgbcolor}disponible\special{ps: grestore}}}}
\put(1276,-1636){\makebox(0,0)[lb]{\smash{\SetFigFont{10}{12.0}{\rmdefault}{\mddefault}{\itdefault}\special{ps: gsave 0 0 0 setrgbcolor}M\special{ps: grestore}}}}
\put(1276,-1111){\makebox(0,0)[lb]{\smash{\SetFigFont{10}{12.0}{\rmdefault}{\mddefault}{\itdefault}\special{ps: gsave 0 0 0 setrgbcolor}L\special{ps: grestore}}}}
\put(1726,-2461){\makebox(0,0)[lb]{\smash{\SetFigFont{10}{12.0}{\rmdefault}{\bfdefault}{\updefault}\special{ps: gsave 0 0 0 setrgbcolor}10\special{ps: grestore}}}}
\put(1726,-1636){\makebox(0,0)[lb]{\smash{\SetFigFont{10}{12.0}{\rmdefault}{\bfdefault}{\updefault}\special{ps: gsave 0 0 0 setrgbcolor}9\special{ps: grestore}}}}
\put(1726,-1111){\makebox(0,0)[lb]{\smash{\SetFigFont{10}{12.0}{\rmdefault}{\mddefault}{\itdefault}\special{ps: gsave 0 0 0 setrgbcolor}5\special{ps: grestore}}}}
\put(2851,-2686){\makebox(0,0)[lb]{\smash{\SetFigFont{10}{12.0}{\rmdefault}{\mddefault}{\itdefault}\special{ps: gsave 0 0 0 setrgbcolor}10\special{ps: grestore}}}}
\put(2926,-2461){\makebox(0,0)[lb]{\smash{\SetFigFont{10}{12.0}{\rmdefault}{\mddefault}{\itdefault}\special{ps: gsave 0 0 0 setrgbcolor}9\special{ps: grestore}}}}
\put(2926,-2236){\makebox(0,0)[lb]{\smash{\SetFigFont{10}{12.0}{\rmdefault}{\mddefault}{\itdefault}\special{ps: gsave 0 0 0 setrgbcolor}8\special{ps: grestore}}}}
\put(2926,-2011){\makebox(0,0)[lb]{\smash{\SetFigFont{10}{12.0}{\rmdefault}{\mddefault}{\itdefault}\special{ps: gsave 0 0 0 setrgbcolor}7\special{ps: grestore}}}}
\put(2926,-1786){\makebox(0,0)[lb]{\smash{\SetFigFont{10}{12.0}{\rmdefault}{\mddefault}{\itdefault}\special{ps: gsave 0 0 0 setrgbcolor}6\special{ps: grestore}}}}
\put(2926,-1561){\makebox(0,0)[lb]{\smash{\SetFigFont{10}{12.0}{\rmdefault}{\mddefault}{\itdefault}\special{ps: gsave 0 0 0 setrgbcolor}5\special{ps: grestore}}}}
\put(2926,-1336){\makebox(0,0)[lb]{\smash{\SetFigFont{10}{12.0}{\rmdefault}{\mddefault}{\itdefault}\special{ps: gsave 0 0 0 setrgbcolor}4\special{ps: grestore}}}}
\put(2926,-1111){\makebox(0,0)[lb]{\smash{\SetFigFont{10}{12.0}{\rmdefault}{\mddefault}{\itdefault}\special{ps: gsave 0 0 0 setrgbcolor}3\special{ps: grestore}}}}
\put(2926,-886){\makebox(0,0)[lb]{\smash{\SetFigFont{10}{12.0}{\rmdefault}{\mddefault}{\itdefault}\special{ps: gsave 0 0 0 setrgbcolor}2\special{ps: grestore}}}}
\put(2926,-661){\makebox(0,0)[lb]{\smash{\SetFigFont{10}{12.0}{\rmdefault}{\mddefault}{\itdefault}\special{ps: gsave 0 0 0 setrgbcolor}1\special{ps: grestore}}}}
\put(3301,-2236){\makebox(0,0)[lb]{\smash{\SetFigFont{10}{12.0}{\rmdefault}{\mddefault}{\itdefault}\special{ps: gsave 0 0 0 setrgbcolor}b\special{ps: grestore}}}}
\put(3301,-2011){\makebox(0,0)[lb]{\smash{\SetFigFont{10}{12.0}{\rmdefault}{\mddefault}{\itdefault}\special{ps: gsave 0 0 0 setrgbcolor}E\special{ps: grestore}}}}
\put(3301,-1561){\makebox(0,0)[lb]{\smash{\SetFigFont{10}{12.0}{\rmdefault}{\mddefault}{\itdefault}\special{ps: gsave 0 0 0 setrgbcolor}a\special{ps: grestore}}}}
\put(3301,-1111){\makebox(0,0)[lb]{\smash{\SetFigFont{10}{12.0}{\rmdefault}{\mddefault}{\itdefault}\special{ps: gsave 0 0 0 setrgbcolor}c\special{ps: grestore}}}}
\put(3751,-2686){\makebox(0,0)[lb]{\smash{\SetFigFont{10}{12.0}{\rmdefault}{\mddefault}{\itdefault}\special{ps: gsave 0 0 0 setrgbcolor}2\special{ps: grestore}}}}
\put(3751,-2236){\makebox(0,0)[lb]{\smash{\SetFigFont{10}{12.0}{\rmdefault}{\mddefault}{\itdefault}\special{ps: gsave 0 0 0 setrgbcolor}3\special{ps: grestore}}}}
\put(3751,-2011){\makebox(0,0)[lb]{\smash{\SetFigFont{10}{12.0}{\rmdefault}{\mddefault}{\itdefault}\special{ps: gsave 0 0 0 setrgbcolor}-1\special{ps: grestore}}}}
\put(3751,-1786){\makebox(0,0)[lb]{\smash{\SetFigFont{10}{12.0}{\rmdefault}{\mddefault}{\itdefault}\special{ps: gsave 0 0 0 setrgbcolor}-1\special{ps: grestore}}}}
\put(3751,-1561){\makebox(0,0)[lb]{\smash{\SetFigFont{10}{12.0}{\rmdefault}{\mddefault}{\itdefault}\special{ps: gsave 0 0 0 setrgbcolor}8\special{ps: grestore}}}}
\put(3751,-1336){\makebox(0,0)[lb]{\smash{\SetFigFont{10}{12.0}{\rmdefault}{\mddefault}{\itdefault}\special{ps: gsave 0 0 0 setrgbcolor}6\special{ps: grestore}}}}
\put(3751,-1111){\makebox(0,0)[lb]{\smash{\SetFigFont{10}{12.0}{\rmdefault}{\mddefault}{\itdefault}\special{ps: gsave 0 0 0 setrgbcolor}-1\special{ps: grestore}}}}
\put(3751,-886){\makebox(0,0)[lb]{\smash{\SetFigFont{10}{12.0}{\rmdefault}{\mddefault}{\itdefault}\special{ps: gsave 0 0 0 setrgbcolor}4\special{ps: grestore}}}}
\put(3751,-661){\makebox(0,0)[lb]{\smash{\SetFigFont{10}{12.0}{\rmdefault}{\mddefault}{\itdefault}\special{ps: gsave 0 0 0 setrgbcolor}7\special{ps: grestore}}}}
\put(3301,-661){\makebox(0,0)[lb]{\smash{\SetFigFont{10}{12.0}{\rmdefault}{\mddefault}{\itdefault}\special{ps: gsave 0 0 0 setrgbcolor}D\special{ps: grestore}}}}
\put(3766,-2461){\makebox(0,0)[lb]{\smash{\SetFigFont{10}{12.0}{\rmdefault}{\bfdefault}{\updefault}\special{ps: gsave 0 0 0 setrgbcolor}1\special{ps: grestore}}}}
\put(3331,-2491){\makebox(0,0)[lb]{\smash{\SetFigFont{12}{14.4}{\rmdefault}{\bfdefault}{\updefault}F}}}
\put(3151,-421){\makebox(0,0)[lb]{\smash{\SetFigFont{8}{9.6}{\familydefault}{\mddefault}{\updefault}\'el\'ement}}}
\put(3646,-421){\makebox(0,0)[lb]{\smash{\SetFigFont{8}{9.6}{\familydefault}{\mddefault}{\updefault}suivant}}}
\special{ps: gsave 0 0 0 setrgbcolor}\put(1876,-2386){\vector( 4,-1){1256.471}}
\special{ps: grestore}\end{picture}

\medskip
\centerline{{\sc Fig.} \thesubsection -- {\it Insertion en t\^ete de ${\cal M}$}}
\addcontentsline{lof}{section}{Insertion en t\^ete}
	\label{Lfauxp2}
\end{figurette} 

\subsection*{Affichage}

Parcourir la liste pour en faire l'affichage peut se r\'ealiser au moyen de la
fonction {\tt prochain(e)} :

Si l'on inverse les lignes comment\'ees {\tt (1)} et {\tt (2)}, la liste est
affich\'ee en sens inverse.

\begin{verbatim}  
lire(e)
{
    if (-1 == e) return ;
    afficher (e) ;       /* (1) */
    lire (suivant(e)) ;  /* (2) */
}
\end{verbatim}

\newpage
~
\end{multicols}
