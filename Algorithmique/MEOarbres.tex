\addcontentsline{toc}{section}{Mise en \oe uvre des arbres}
\markboth{Arbres (mise en \oe uvre)}{Arbres (mise en \oe uvre)}
\label{TDAarbres}
\centerline{\Large\bf Mise en \oe uvre des arbres}

 \noindent\hrulefill  

%\begin{multicols}{2}
\section*{Le TDA arbre}

Les primitives les plus utiles relatives aux arbres\footnote{Cf. \pageref{Parcours}}~:

\begin{enumerate}

\item {\bf PERE}($n, A$). Cette fonction retourne le p\`ere du n\oe ud $n$ dans
l'arbre $A$. Si $n$ est la racine, le p\`ere n'existe pas et $\Lambda$
est retourn\'e (dans ce contexte, $\Lambda$ est un ``n\oe ud vide'').

\item {\bf PREMIER\_FILS}($n, A$). Retourne le fils le plus \`a gauche du n\oe ud
	$n$ dans l'arbre $A$, ou $\Lambda$ si $n$ est une feuille
	(donc sans enfants).

\item {\bf FRERE\_DROIT}($n, A$). Retourne le fr\`ere droit du n\oe ud $n$ dans
l'arbre $A$, c'est-\`a-dire le n\oe ud de m\^eme p\`ere $p$ que $n$ et situ\'e
imm\'ediatement \`a sa droite dans l'ordre naturel des fils de $p$. Sur
l'arbre de la figure, %\footnote{p.\pageref{arbetiq}} \ref{arbreA},
par exemple, PREMIER FILS($n_{2}$)
= $n_{4}$, FRERE DROIT($n_{4}$) = $n_{5}$ et FRERE DROIT($n_{5}$) =
$\Lambda$.

\item {\bf ETIQUETTE}($n, A$). Retourne l'\'etiquette du n\oe ud $n$ dans l'arbre $A$.
Ceci n'implique pas, malgr\'e tout, que tout arbre soit \'etiquet\'e.

% Modifier cette procedure

\item {\bf CREER}($e, A$). Construit un nouveau n\oe ud  d'\'etiquette $e$.

\item {\bf RACINE}($A$). Retourne la racine de l'arbre $A$, ou bien $\Lambda$ si $A$ est l'arbre
vide.

\item {\bf RAZ}($A$). Transforme l'arbre $A$ en arbre vide.
\end{enumerate}

\section*{Repr\'esentation des arbres par premier fils et fr\`ere droit}
\addcontentsline{lof}{section}{Mise en \oe uvre d'un arbre}


\vspace*{-.8cm}
\begin{figurette}
\begin{psmatrix}
  \[
      \def\arraystretch{3}
      \arraycolsep 5mm
      \begin{array}{ccc}
    &  \rnode{a}{A} &  \\
\rnode{b}{B} &  & \rnode{c}{C}  \\
    &           & \rnode{d}{D} \\ 
      \end{array}
    \psset{arrows=-,nodesep=3pt}
    \pslabelsep 3pt
    \everypsbox{\scriptstyle}
    \ncLine{a}{b}
    \ncLine{a}{c}
    \ncLine{c}{d}
  \] 
\end{psmatrix}
\label{Un_arbre}
\medskip
\centerline{{\sc Fig.} \thesection -- {\it Un arbre}}
\addcontentsline{lof}{section}{Un arbre}
\end{figurette}


\vspace*{-.8cm}
\begin{figurette}
\begin{psmatrix}
  \[
      %\def\arraystretch{3}
      \arraycolsep 5mm
\begin{tabular}{c|c|c|c|}
\multicolumn{1}{c}{} &\multicolumn{1}{c}{\vdots} & \multicolumn{1}{c}{\vdots} & \multicolumn{1}{c}{\vdots} \\
\cline{2-4}
2 & $\bullet$ & D & $\bullet$ \\
\cline{2-4}
\multicolumn{1}{c}{}& \multicolumn{1}{c}{\vdots} & \multicolumn{1}{c}{\vdots} & \multicolumn{1}{c}{\vdots} \\
\cline{2-4}
5 & $\bullet$ & B & \rnode{a}{11} \\
\cline{2-4}
\multicolumn{1}{c}{}& \multicolumn{1}{c}{\vdots} & \multicolumn{1}{c}{\vdots} & \multicolumn{1}{c}{\vdots} \\
\cline{2-4}
10 & 5 & A & $\bullet$ \\
\cline{2-4}
11 & 2 & C & \rnode{b}{$\bullet$} \\
\cline{2-4}
\multicolumn{1}{c}{} & \multicolumn{1}{c}{\vdots} & \multicolumn{1}{c}{\vdots} & \multicolumn{1}{c}{\vdots} \\
\end{tabular} 
\]
\psset{arrows=->}
\ncarc[arcangle=15]{a}{b}
\end{psmatrix}
\medskip
\centerline{{\sc Fig.} \thesection -- {\it Mise en \oe uvre d'un arbre}}
\addcontentsline{lof}{section}{Mise en \oe uvre d'un arbre}
\label{arbre}
\end{figurette}

\section*{Programmation du TDA arbre g\'en\'erique}

La totalit\'e du TDA arbre est donn\'ee en annexe.

\newpage
~
