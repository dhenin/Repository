\stepcounter{section}
\addcontentsline{toc}{section}{Pointeurs de fonctions}
\markboth{Pointeurs de fonctions}{Pointeurs de fonctions}
\centerline{\Large\bf Les pointeurs de fonctions}
\index{Pointeurs de fonctions}
\label{Fonction}
 
\noindent\hrulefill  

% \addtocounter{\thesection}
% \setcounter{section}{0}
% \stepcounter{section}
%\section*{\thesection \hbox to .45cm {}  Quelle importance ?}
\subsection*{Quelle importance ?}

L'\'ecriture de type de donn\'ees abstrait pr\'esuppose une bonne connaissance des 
pointeurs tant pour le maniement des structures\footnote{Voir p.
\ref{structures}.} que pour la mise en place des m\'ethodes.

%\stepcounter{section}
%\section*{\thesection \hbox to .45cm {}  Rappel}
\subsection*{Rappel}
    
	\begin{itemize} 

    \item  En C, une fonction elle-m\^eme n'est pas une variable mais il est
      possible de d\'efinir des pointeurs de fonctions que l'on peut 
      affecter, placer dans des tableaux (de pointeurs de fonctions), 
      passer en argument \`a des fonctions ou faire
	  retourner par des fonctions\ldots

	  \end{itemize} 

\subsection*{D\'eclaration}

	\begin{itemize} 

    \item  La d\'eclaration d'un pointeur de fonction se fait de la fa\c con suivante :

      {\tt int    (*ptf) (int,char *) ;}

      D\'eclaration d'un pointeur de fonction pour une fonction qui retourne
      un entier et a deux arguments, un entier et un pointeur de type char.

    \item Si on \'ecrit :  {\tt int *ptf (int,char *);} on \'ecrit un
	      prototype...  Les parenth\`eses autour de l'identificateur
	      sont l\`a pour pr\'eciser qu'il s'agit d'un pointeur et non
	      d'une fonction bien pr\'ecise.

	 \end{itemize} 

\subsection*{Initialisation et utilisation}

    \begin{itemize} 
    
	\item Un pointeur d\'eclar\'e peut \^etre initialis\'e comme toute variable, mais
      ici avec un nom de fonction.

      Exemple :
\smallskip 
      {\tt int    LireLigne (char *) ;}
{\small
\begin{verbatim}
main ()
{
  char    tab[256]         ;
  int     n                ;
  int     (* ptf) (char *) ;

  ptf = LireLigne ;            /* Initialisation.    */

  n = (*ptf) (tab) ;           /* Appel de LireLigne.    */
}\end{verbatim}
}

\item Dans le module {\tt Liste.c}\footnote{c.f. Annexe},
on initialise les pointeurs de fonction
de la structure {\tt fonctions} d\'efinie dans le fichier {\tt Liste.h} :

{\small
\begin{verbatim}void * lier_liste () 
{
    Liste.creer        = &creer        ;
    Liste.copier       = 0             ;
    Liste.detruire     = &detruire     ;
    ...
    Liste.retablir     = &retablir     ;
    Liste.afficher     = &afficher     ; 
} \end{verbatim}

}
\end{itemize} 

\addcontentsline{lof}{section}{Pointeur sur une fonction}
\newpage
~

%\end{multicols}
