\addcontentsline{toc}{chapter}{\thechapter ~ Les listes}
\markboth{Listes}{Listes}
\centerline{\Large\bf Les listes}
 \index{liste}
 \index{objet (informatique)}
 \index{donn\'ee}
 \index{m\'ethode (pour un objet)}
 \index{vide (liste)}
 \index{longueur (d'une liste)}
 \index{lifo ({\em last in first out})}
 \index{fifo ({\em first in first out})}
 \index{tableau (mise en \oe uvre d'une liste dans un)}
 \index{cha\^\i n\'ee (liste)}
 \index{ins\'erer (un \'el\'ement dans une liste)}
 \index{pointeur (faux)}
 \index{faux (pointeur)}
 
 \noindent\hrulefill  
\begin{multicols}{2}
\stepcounter{section}
\section*{G\'en\'eralit\'es}
  
  Par extension du sens habituel donn\'e au mot, 
  une liste\footnote{Les listes peuvent \^etre vues comme des formes
  simples d'arbres (cf. page \pageref{Arbre}) ; on peut en effet consid\'erer
  qu'une liste est un arbre binaire dans lequel tout fils gauche est une
  feuille.} lin\'eaire\footnote{Knuth distingue les << listes lin\'eaires >>
  des << listes >> au sens large qui d\'esignent 
  les arborescentes. Cependant l'usage ne maintient pas cette distinction.}
  est un {\em type de donn\'ees abstrait} comprenant une suite
  d'informations (donn\'ees) ainsi que les op\'erations (m\'ethodes)
  n\'ecessaires au maniement de ces donn\'ees.  
  La liste lin\'eaire est per\c cue comme une suite d'\'el\'ements ({\it maillons})
  ordonn\'ee, caract\'eris\'ee par le cha\^\i nage\footnote{Au moyen d'un pointeur
  par exemple. (cf. page \pageref{pointeur})} de chaque maillon \`a son suivant.

  La liste est une structure de base de la programmation\footnote{Le langage
{sc lisp}, con\c cu par John MacCarthy en 1960, utilise principalement cette
structure qui se r\'ev\`ele utile pour le calcul symbolique.}. Chaque \'el\'ement
permet l'acc\`es \`a deux valeurs : l'objet associ\'e \`a cet \'el\'ement et l'\'el\'ement
suivant\footnote{Le num\'ero de ligne, lorsqu'il s'agit
d'un tableau, comme dans la figure \ref{lfauxp}, l'adresse en
m\'emoire, lorsqu'il s'agit de {\em pointeurs}.}. La recherche d'un \'el\'ement dans
la liste s'apparente \`a un $<<$ jeu de piste $>>$ dont le but est de retrouver un
objet cach\'e : on commence par avoir des informations sur un lieu o\`u pourrait
se trouver l'objet, en ce lieu on d\'ecouvre des informations sur un autre lieu
o\`u il a une chance de se trouver et ainsi de suite.
  
  Une liste doit pouvoir \^etre modifi\'ee :  on doit \^etre capable de supprimer ou
  d'ajouter des donn\'ees.
  
  
  Pour acc\'eder \`a une donn\'ee appartenant \`a une liste, il convient de
  disposer de la liste mais aussi de la place (position) de cette donn\'ee dans la
  liste.
  
  Une liste peut \^etre vide~;  on peut 
  vider une liste~; on peut acc\'eder au $i$\`eme \'el\'ement, en conna\^\i tre
  le contenu~; on
  peut conna\^\i tre la longueur de la liste (le nombre d'\'el\'ements qu'elle
  contient)~; on peut supprimer un \'el\'ement de la liste, y ins\'erer un
  nouvel \'el\'ement et enfin acc\'eder au successeur d'un \'el\'ement dont on
  conna\^\i t la place.
  
  Enfin, on peut vouloir fabriquer une liste \`a partir de deux
  autres.
  
      
{\bf La pile  (LIFO)} est une liste dans laquelle on ins\`ere et on supprime
seulement \`a une extr\'emit\'e ; 

{\bf  La file  (FIFO)} est une liste dans laquelle on ins\`ere \`a une extr\'emit\'e
et o\`u on supprime \`a l'autre.

Il existe de nombreuses fa\c cons de mettre en \oe uvre les listes. La mise en
\oe uvre dans un tableau facilite la compr\'ehension, mais pr\'esente
l'inconv\'enient de ne contenir qu'un nombre maximum pr\'ed\'efini de cellules.
      
\section*{Mise en \oe uvre des listes  dans un tableau}

      La figure \ref{lfauxp} montre un tableau {\tt ESPACE}
    contenant deux listes ${\cal L} = a, b, c$ et ${\cal M} = D, E$.
    Pour cha\^\i ner les \'el\'ements des listes on utilise un vecteur ``suivant'' 
    de faux-pointeurs, c'est-\`a-dire le num\'ero de ligne du successeur.

	Remarquez que toutes les cellules du tableau n'appartenant \`a
    aucune des deux listes sont cha\^\i n\'ees en une pile\footnote{Cf. page
	\pageref{Pile}.} appel\'ee {\bf disponible}.
	Cette liste suppl\'ementaire sert \`a trouver un espace libre pour ins\'erer
    un nouvel \'e\-l\'e\-ment,
	ou \`a r\'ecup\'erer les espaces lib\'er\'es par la
    suppression d'\'el\'ements pr\'ec\'e\-dem\-ment dans une liste en vue d'une
    utilisation ult\'erieure.

\begin{figurette}
	\input{Figures/lfauxp}
\medskip
\centerline{{\sc Fig.} \thesection -- {\it Une liste cha\^\i n\'ee\\ dans un tablea}}
\addcontentsline{lof}{section}{Implantation d'une liste cha\^\i n\'ee dans un tableau}
	\label{lfauxp}
\end{figurette}

	La variable $L$ contient l'indice de ligne o\`u commence la liste ${\cal L}$
	et la variable $M$ celui de la liste ${\cal M}$ ; la case
	ESPACE[L][element] contient le premier \'el\'ement ($a$) et ESPACE[L][suivant]
	l'indice du successeur (8) c-\`a-d $b$. De la m\^eme fa\c con, ESPACE[8][suivant]
	contient 3.

	$c$ n'ayant pas de successeur, il est le dernier \'el\'ement,
	ESPACE[3][su\-i\-vant] est $-1$.
\end{multicols}
\newpage
\section*{Programmation d'une liste g\'en\'erique}

La totalit\'e des sources du TDA liste est donn\'ee en annexe.

Le fichier d'{\it ent\^ete} du TDA liste ({\tt Liste.h}),
contient la d\'efinition de 3 structures : 

\begin{itemize} 
\item La structure  {\it cellule}, qui contient un pointeur sur la donn\'ee
associ\'ee et un pointeur sur la  cellule suivante,

\item la structure de t\^ete qui donne acc\`es \`a la premi\`ere cellule de la liste,
\`a la derni\`ere et \`a la cellule courante ({\it vue}),

\item la structure donnant acc\`es aux m\'ethodes (fonctions\footnote{Voir les
pointeurs de fonction p. \ref{Fonction}}).
\end{itemize} 
{\small
\listing{Prog/Objet/Liste.h}}
