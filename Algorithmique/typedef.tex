\stepcounter{section}
\addcontentsline{toc}{section}{D\'efinitions de types}
\markboth{D\'efinitions de types}{D\'efinitions de types}
\centerline{\Large\bf Les d\'efinitions de types}
 \index{D\'efinitions de types}
 
 \noindent\hrulefill  
\begin{multicols}{2}

Des d\'eclarations contenant le {\it sp\'ecificateur-d\'eclaration} {\tt typedef} 
nomment des identificateurs pouvant \^etre utilis\'es ensuite pour
la d�signation de
types fondamentaux ou d\'eriv\'es.
Le sp\'ecificateur {\tt typedef} ne peut pas \^etre
utilis\'e dans une {\it d\'efinition-fonction}.

La d\'efinition des prototypes, dans le manuel, conjointement avec les
fichiers d'ent\^ete utilise abondamment le {\tt typedef}. Par exemple

{\small
\begin{verbatim}  
$ man malloc 
...
SYNOPSIS
     #include <stdlib.h>

     void *
     malloc(size_t size)
...

$ more /usr/include/stdlib.h 
...
#include <machine/ansi.h>
...
typedef _BSD_SIZE_T_    size_t;
...

$ more /usr/include/machine/ansi.h
...
#define _BSD_SIZE_T_  unsigned int
...
\end{verbatim} }

ce qui permet de traduire : 

{\small
\tt size\_t} est l'\'equivalent de {\small
\tt unsigned int }.

\stepcounter{subsection}
\subsection*{Utilisation pour les types de donn\'ees abstraits}

L'usage \'el\'ementaire consiste \`a d\'efinir un type {\tt Boolean}\footnote {Mammeri
M. {\it programmation} {\sc ecole centrale de paris} {\tiny 1997-1998} p. 19
et 36}.

{\small
\begin{verbatim}typedef enum { FALSE, TRUE } Boolean ; \end{verbatim} }

L'usage plus \'el\'egant consiste \`a renommer les structures\footnote{{\it ibid.}
p.65} :

{\small
\begin{verbatim}  
struct MAILLON
{
    TypDon           elem ;
    struct MAILLON * suiv ;
} ;

typedef struct MAILLON * File ;
\end{verbatim} }

On pr\'ef\'erera l'\'ecriture en deux \'etapes \`a celle plus propice \`a cr\'eer la
confusion :

{\small
\begin{verbatim}  
typedef struct MAILLON  /* A EVITER */
{
    TypDon           elem ;
    struct MAILLON * suiv ;
} * File ;

\end{verbatim} }

Le fichier {\tt Liste.tda} est un gigantesque {\tt typedef} qui utilise les
possibilit\'es des macros afin de cr\'eer des listes d'objets dont le type est
d\'efini au moment de l'utilisation.

{\footnotesize \begin{verbatim}  

#define nom(a,b) a##b

#define liste(type_objet)                             \
typedef struct nom(type_objet, liste)                 \
{                                                     \
    void *     rep ;                                  \
    type_objet (*copier_objet)   (type_objet) ;       \
    void       (*detruire_objet) (type_objet) ;       \
    void       (*afficher_objet) (type_objet) ;       \
} * nom(type_objet, liste) ;                          \
nom(type_objet, liste) nom(type_objet, liste_creer)   \
    (type_objet (*copier)   (type_objet),             \
     void       (*detruire) (type_objet),             \
     void       (*editer)   (type_objet))             \
{                                                     \
    nom(type_objet, liste) l ;                        \
    lier_liste () ;                                   \
    l = (nom(type_objet, liste)) malloc               \
     (sizeof (struct nom (type_objet, liste))) ;      \
    l->rep = (*Liste.creer) () ;                      \
    l->copier_objet = copier ;                        \
    l->detruire_objet = detruire ;                    \
    l->afficher_objet = editer ;                      \
    return l ;                                        \
}                                                     \
...
void nom(type_objet, liste_afficher)                  \
    (nom(type_objet, liste) l)                        \
{                                                     \
    if (l->afficher_objet)                            \
    (*Liste.afficher) (l->rep, l->afficher_objet) ;   \
    else                                              \
    printf ("ERR : fonction d'affichage  nulle\n") ;  \
}
\end{verbatim} }

Remarquez qu'il s'agit d'une seule longue ligne logique,
les $\backslash$ r\'ealisant la concat\'enation des lignes physiques.
\end{multicols}
\newpage
~
