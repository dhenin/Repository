\addcontentsline{toc}{chapter}{\thechapter ~ Sch\'ema fonctionnel}
%\addcontentsline{toc}{chapter}{\thechapter ~ Sch\'ema fonctionnel  (\I)}
\label{Composition}
\setcounter{section}{0}
\addtocounter{section}{1}
\addcontentsline{toc}{section}{Relation de pr\'ec\'edente}
\addcontentsline{lof}{section}{Calcul de la fr\'equence d'un quartz et de 2 diviseurs}
\markboth{Sch\'ema fonctionnel}{Sch\'ema fonctionnel}
%\centerline{\Large\bf Sch\'emas fonctionnel  (\I)}
\centerline{\Large\bf Sch\'ema fonctionnel (I)}
\index{fonction}
\index{pgcd}
\index{ppcm}
\index{plus grand diviseur}
\index{grand diviseur (plus)}
\index{diviseur (plus grand)}
\index{plus petit multiple}
\index{multiple (plus petit)}
\index{pr\'esupposer}
\index{pr\'ec\'edente (relation de)}
\index{relation (de pr\'ec\'edente)}
\index{ordre (relation d')}
\index{partiel (relation d'ordre)}
\index{variable (relation d'ordre sur les)}

 
 \noindent\hrulefill  


Dans la m\'ethode de construction des algorithmes\footnote{
Page \pageref{Methode}}, nous
partons de l'objectif final, le plus souvent une valeur \`a calculer ; ainsi,
nous faisons appara\^\i tre une variable pour laquelle il faut expliciter
le mode de calcul. Ce calcul {\bf pr\'esuppose} lui-m\^eme le calcul d'autres
variables. On comprend intuitivement que la valeur finale est calcul\'ee 
en {\em fonction} des variables dont elle d\'epend. 
Chaque calcul peut faire l'objet d'un algorithme et 
   on peut utiliser, dans une d\'efinition d'un algorithme, une valeur
r\'esultant de l'ex\'ecution d'un autre algorithme. 

Soit par exemple 
\`a fabriquer une s\'erie d'appareils dont chaque exemplaire sera \'equip\'e
d'un quartz et de deux diviseurs de fr\'equence correspondant \`a la demande de
chaque client. \`A la commande, le client pr\'ecisera les 2 fr\'equences dont il a
besoin. Nous allons \'ecrire le programme qui d\'etermine la valeur du quartz et
les valeurs des deux diviseurs de fr\'equence, sachant que toutes ces
valeurs  sont
 des nombres entiers.

\begin{center}
{\tt
\begin{tabular}{|l|l|c|}   \hline
\multicolumn{3}{|l|}{FREQ : /* Calcul d'une fr\'equence et de deux diviseurs */}\\
\hline
     Lexique             & Actions                               & ordre \\
\hline
$Fq$ : Fr\'equence du quartz& AFFICHER$(Fq$, $div1$, $div2)$ ;     &   5   \\
$F_{1}$ et $F_{2}$ : Fr\'equences client & $Fq =$ PPCM$(F_{1},F_{2})$ ;   &   2   \\
$div1$ : diviseur pour $F_{1}$ & $div1 = \frac{Fq}{F_{1}}$ ;     &   3   \\
                          & $div2 = \frac{Fq}{F_{2}}$ ;      &   4   \\
$div2$ : diviseur pour $F_{2}$ & SAISIR$(F_{1},F_{2})$ ;              &   1   \\
\hline
\multicolumn{3}{c}{} \\
\multicolumn{3}{l}{La valeur $Fq$ sera renvoy\'ee par le calcul du {\sc ppcm} de
$F_{1} \mbox{et} F_{2}$.}\\
\multicolumn{3}{c}{} \\
\hline
\multicolumn{3}{|l|}{PPCM$(a, b)$
                              /* Calcul du Plus Petit Commun Multiple */ }  \\
\hline
		  &                                  &                \\
                  & PPCM $ = \frac{a\times b}{\mbox{PGCD}(a, b)}$ ; &  \\
		  &                                  &                \\
\hline
\multicolumn{3}{c}{} \\
\multicolumn{3}{l}{La valeur du {\sc ppcm} sera renvoy\'ee par le calcul du
{\sc pgcd}\footnote{Le calcul du {\sc pgcd} peut \^etre fait de fa\c con r\'ecursive.
Cf. page \pageref{Pgcd}} de $F_{1} \mbox{et} F_{2}$.}\\
\multicolumn{3}{c}{} \\
\hline
\multicolumn{3}{|l|}{PGCD$(x, y)$
                              /* Calcul du Plus Grand Commun Diviseur */ }  \\
\hline

$m$ : variable temporaire & {\bf while (}$x !=  0${\bf )}  &       \\
                  & \hspace{0mm}{\bf \{}                   &           \\
                  & \hspace{5mm}$m = x$ ;                   &           \\
                  & \hspace{5mm}$x = y$ ;                  &           \\
                  & \hspace{5mm}$y = m$ modulo $y$ ;           &    \\
                  & \hspace{0mm}{\bf \}}                   &           \\
                  & {\bf return } $y$ ;                     &           \\
\hline
\end{tabular} }
\end{center}
On voit donc que l'on est conduit \`a \'ecrire 2 nouvelles fonctions (qui ne
sont pas natives) le {\tt PPCM} et le {\tt PGCD} pour r\'ealiser ce
programme.

\paragraph{Relation de pr\'ec\'edente}


On d\'efinit une relation de pr\'ec\'edente sur les variables d'un algorithme.
Nous dirons que la variable $x$ {\bf pr\'ec\`ede} la variable $y$ {\bf si} $x$ a au
moins {\bf une occurrence dans la d\'efinition}  de $y$.
\`A cause de cela, la valeur
de $y$ d\'epend de la valeur de $x$, et le calcul de $y$ n'est possible
que si $x$ a d\'ej\`a \'et\'e calcul\'e.  La relation ``$x$ pr\'ec\`ede $y$'' peut
donc \^etre lue comme ``le calcul de $x$ doit pr\'ec\'eder le calcul de $y$''.

Cette relation est un ordre partiel sur les variables de l'algorithme.

\newpage 
%----------------------------------------------------------------------
\setcounter{footnote}{0}
\addtocounter{section}{1}
\addcontentsline{toc}{section}{Un calendrier perp\'etuel}
\addcontentsline{lof}{section}{Calendrier perp\'etuel}
\label{Perpetuel}
% \addcontentsline{lof}{section}{Graphe de d\'ependance des variables}
%\markboth{Algorithme}{Algorithme}
\centerline{\Large\bf Un calendrier perp\'etuel\footnote{Mammeri,
{\em Exercices d'algorithmie}, {\sc ecp}.1a 1977-1998}}
\index{quanti\`eme}
\index{jour}
\index{mois}
\index{semaine}
\index{graphe (de d\'ependance)}
\index{d\'ependance (graphe de)}
\index{variable (d\'ependance des)}
 
 \noindent\hrulefill  

\paragraph{D\'efinition du probl\`eme : }
On se propose de d\'eterminer le jour de la semaine d'une date donn\'ee. Sachant
que le $1^{er}$ janvier 1900 \'etait un lundi ; on appellera {\tt d\'ecalage} la
position d'un jour dans la semaine $\mid$ {\tt d\'ecalage} $\in$ [0-6], \mbox{0
$\equiv$ dimanche}.  

%Une date \'etant
%donn\'ee par son quanti\`eme ({\tt jour}), son mois ({\tt mois}), son
%mill\'esime ({\tt an}) (sur 2 chiffres dans l'intervalle [1901--2099]),
%dire quel jour
%de la semaine ({\tt num}  $\in$ [0-6], \mbox{0 $\equiv$ dimanche}) 
% lui correspond.
%
   Pour cela, on va chercher le d\'ecalage entre le 1$^{er}$
   janvier de l'ann\'ee 1900
 et la date consid\'er\'ee.  On se donne une table du nombre
de jours \'ecoul\'es depuis le d\'ebut de l'ann\'ee jusqu'au d\'ebut du mois
{\tt mois}.

\centerline{\tt
\begin{tabular}{|l|ll|c|}
\multicolumn{4}{c}{}\\
\hline
\multicolumn{4}{|l|}{\tt Calper(quanti\`eme, mois, an)}\\
\hline
& \multicolumn{2}{|l|}{{\tt printf ("\%d$\backslash$n",} {\tt Jour}{\bf ) ;}}
& 5 \\
{\tt Jour $\in$ [0-6] }&{\tt Jour }& {\tt = d\'ecalage {\bf modulo} 7 ;} & 4 \\
{\tt d\'ecalage $\in \mathbb{N}$}&{\tt d\'ecalage }& {\tt =  quanti\`eme + DecalAA +
DecalMM ;} & 3 \\
{\tt DecalAA $\in \mathbb{N}$} &{\tt DecalAA }& {\tt = EcoulesAn (ann\'ee) ;} & 1
\\
{\tt DecalMM $\in \mathbb{N}$}&{\tt DecalMM }&
                   {\tt = EcoulesMM (ann\'ee, mois) ;}& 2  \\
\hline
\multicolumn{4}{|l|}{\tt EcoulesAn(a)}\\
\hline
    &\multicolumn{2}{l|}{\tt /* 1 jour de d\'ecalage par an   + 1 par ann\'ee bissextile */}     &  \\
	               & {\bf return} & DecAn ; & 4 \\
{\tt DecAn $\in \mathbb{N}$}
                   &	DecAn    &=  AnsApres1900 + AnsBissextiles ; & 3 \\
{\tt  AnsApres1900 $\in \mathbb{N}$}
                &	AnsApres1900   &= a {\bf modulo} 100 ; & 1 \\
&\multicolumn{2}{l|}{\tt /* On ne compte pas l'ann\'ee courante
                               parmi bissextiles */ } & \\
{\tt  AnsBissextiles $\in \mathbb{N}$}&	AnsBissextiles   &= (a - 1) / 4   ; & 2 \\
\hline
\multicolumn{4}{|l|}{\tt EcoulesMM(a, m)}\\
\hline
&\multicolumn{2}{l|}{\tt /* D\'ecalage de jours entre 1.1.a et 1.m.a */}  &  \\
	               & {\bf return} & DecMois ; & 5 \\
{\tt  DecMois $\in \mathbb{N}$}&  DecMois &= SommeMois[m] + Biss ; & 4 \\
Biss $\in [0,1]$ & Biss &= 0 ; & 2 \\
                 & \multicolumn{2}{l|}{\tt if (a\%4)} &  3 \\
                 & \multicolumn{2}{l|}{\tt ~~~~if (m>2)} &    \\
                 & ~~~~~~~~Biss & = 1 ;  &    \\
{\tt  SommeMois $\in \mathbb{N}$}&  SommeMois[] &= \{0, 0, 31, 59, 90, 120, 151, & \\
 & &~~~~181, 212, 243, 273, 304, 334\} ; & 1 \\
\hline
\end{tabular}}

\begin{center} 
{%\footnotesize \begin{figurette}
\begin{psmatrix}
  \[ \def\arraystretch{3.5}
      \arraycolsep .8cm
      \begin{array}{ccccc}
       & & \rnode{a}{}  & & \rnode{b}{\tt quantieme} \\
       &             &  & \rnode{c}{\tt AnsApres1900} & \\
\rnode{d}{\tt Jour} & \rnode{e}{\tt decalage} & \rnode{f}{\tt DecalAA} & \rnode{h}{\tt
AnsBissextiles} & \rnode{g}{\tt annee} \\
 &  & \rnode{i}{\tt DecalMM} & \rnode{j}{\tt Biss} &  \\
 &  &  & \rnode{k}{\tt SommeMois} & \rnode{l}{\tt mois} \\
      \end{array}
    \pslabelsep 3pt
    \everypsbox{\scriptstyle}
    \psset{arrows=-,nodesep=3pt}
\iffalse
    \ncLine{b}{a}
    \psset{arrows=->,nodesep=3pt}
    \ncLine{a}{e}
    \ncLine{c}{f}
    \ncLine{e}{d}
    \ncLine{f}{e}
    \ncLine{g}{c}
    \ncLine{g}{h}
    \ncLine{g}{j}
    \ncLine{h}{f}
    \ncLine{i}{e}
    \ncLine{j}{i}
    \ncLine{k}{i}
    \ncLine{l}{k}
\fi    
  \] \end{psmatrix}
	\label{Dependances}
\medskip
\centerline{{\sc Fig.} \thesection -- {\it Graphe de d\'ependance des variables}}
\addcontentsline{lof}{section}{Graphe de d\'ependance des variables}
%\end{figurette}
} \end{center} 


On v\'erifie sur le graphe de {\em d\'ependance des variables} de la figure 
 ci-dessus
 que l'objectif ({\tt Jour}) pr\'esuppose le calcul
des variables dont il d\'epend, et que chacune d'elles est explicit\'ee, soit par
un calcul simple, soit par l'{\bf appel d'une fonction}, jusqu'\`a
remonter aux valeurs fournies \`a l'algorithme.
