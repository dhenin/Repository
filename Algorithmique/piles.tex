\addcontentsline{toc}{chapter}{\thechapter ~ Les piles}
\markboth{Traitement des expressions}{Traitement des expressions}
\centerline{\Large\bf Utilisation d'une pile pour le traitement des expressions}
\index{pile}
\label{Pile}
\index{infix\'e}
\index{postfix\'e}
\index{traitement des expressions}
 
\noindent\hrulefill  
\begin{multicols}{2}
\section*{G\'en\'eralit\'es}

    Intuitivement, une pile correspond \`a une pile de livres sur une table.
    La premi\`ere op\'eration consiste \`a poser un livre sur la table vide.
    L'op\'eration suivante place un second livre sur le livre d\'ej\`a pos\'e sur la
    table. Chaque op\'eration de rangement augmente la taille de la pile.
    Prendre un livre sans provoquer l'effondrement de la pile n'est possible
    qu'au sommet de la pile.

\section*{Traitement}

En mode infix\'e, \'ecrire une expression consiste \`a :

\begin{itemize} 
\item \'ecrire un op\'erateur entre 2 op\'erandes,
\item \'ecrire un op\'erateur unaire suivit de son op\'erande,
\item modifier l'ordre des priorit\'es \`a l'aide de parenth\`eses.
\end{itemize} 

Le traitement informatique d'une expression infix\'ee est malais\'e. Il est
pr\'ef\'erable d'effectuer le traitement de l'expression \'equivalente en mode
postfix\'e ou infix\'e.  Par exemple, l'expression $(3 + 4)\times 2$ devient $3 \; 4 + 2
\times$ en notation postfix\'ee.

\medskip 

\begin{tabular}{c|l|l}
\hline
\hline
Symboles & Pile  & Actions \\
\hline
3      &   3   & {\it push} 3 \\
4      &   3, 4   & {\it push} 4 \\
+      &          & {\it pop} 4 ;  {\it pop} 3 \\
       &          & calculer $7 = 3 + 4 $ \\
       &   7      & {\it push} 7 \\
2      &   7, 2   & {\it push} 2 \\
$\times$  &          & {\it pop} 2 ;  {\it pop} 7 \\
       &          & calculer $14 = 7 \times 2 $ \\
       &   14      & {\it push} 14 \\
\hline
\end{tabular} 

\subsubsection*{R\`egles de transformation}
%La totalit\'e des fichiers sources du TDA Pile
%et du traducteur est donn\'ee en annexe.

\begin{enumerate} 
\item Les variables successivement rencontr\'ees dans l'expression infix\'ee sont
rang\'ees directement dans l'expression pr\'efix\'ee.

\item Les op\'erateurs sont empil\'es en tenant compte de leur priorit\'e :
    
    \begin{itemize} 
    \item priorit\'e de l'op\'erateur ds l'expression $>$ priorit\'e de l'op\'erateur
    au sommet de la pile $\Longrightarrow$ empiler l'op\'erateur,
    \item priorit\'e de l'op\'erateur ds l'expression $<=$ priorit\'e de l'op\'erateur
    au sommet de la pile $\Longrightarrow$ d\'epiler et ranger l'op\'erateur
    d\'epil\'e dans l'expression postfix\'ee.
    \end{itemize} 
\item Empiler syst\'ematiquement une parenth\`ese ouvrante car elle d\'elimite une
sous-expression,
\item la parenth\`ese fermante fait sortir tous les \'el\'ements de la pile jusqu'\`a
la rencontre d'une parenth\`ese ouvrante.
\end{enumerate} 

\section*{�criture}


Supposons que nous disposions d'un type de donn\'ees abstrait de
{\it pile},
on peut empiler les termes, les facteurs et les op\'erateurs afin d'obtenir la
traduction de l'expression infix\'ee en expression pr\'efix\'ee :
{\small
\begin{verbatim}struct item
{           
   int type ;
   union   {    
       char op  ; int  val ;
   } contenu ;
    
} ;         
typedef struct item   ITEM ;
typedef struct item * Item ;
...
pile (Item) ; Itempile p ; 

int ValLex ; int Symbole ;
...
Emettre (int Lex, int Val)
{
     ITEM x ; Item y ;

     switch (Lex)
     {
     case '+' : case '-' : case '*' : case '/' :
            x.type = OP  ;
         x.contenu = Lex ;
         y = Item_copier(&x) ;
         Itempile_empiler  (p, y) ;
         break ;

     case NB :  /* C'est un nombre */
             x.type = NB ;
         x.contenu = Val ;
         y = Item_copier(&x) ;
         Itempile_empiler  (p, y) ;
         break ;

     default :
     }
}

main ()
{

    p = Itempile_creer(Item_copier,
            Item_detruire, Item_editer) ;

    Analyse() ;
} \end{verbatim} }
\end{multicols}
\newpage ~
