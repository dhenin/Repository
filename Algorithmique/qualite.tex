\addcontentsline{toc}{chapter}{\thechapter ~ Qualit\'e du logiciel}
\markboth{Qualit\'e du logiciel}{Qualit\'e du logiciel}
\centerline{\Large\bf La qualit\'e du logiciel}
 \index{Qualit\'e du logiciel}
 
 \noindent\hrulefill  
\begin{multicols}{2}

La qualit\'e du logiciel est l'objectif du g\'enie logiciel. Ce n'est pas une id\'ee
simple mais un ensemble de {\it facteurs}. Les recherches de ces derni\`eres 
ann\'ees concernent la rapidit\'e, la fiabilit\'e, la lisibilit\'e et la maintenabilit�. Nous avons, dans
la premi\`ere partie, abord\'ee la notion de programmation structur\'ee. Au cours
des travaux pratiques, nous avons utilis\'e la programmation modulaire.

Du point de vue {\it externe} de l'utilisateur, la qualit\'e d'un logiciel
se manifeste par son ergonomie, son efficacit\'e, sa facilit\'e d'emploi.

Du point de vue {\it interne} du concepteur, la qualit\'e se fonde sur la
facilit\'e de mise au point, la lisibilit\'e du source et la r\'eutilisabilit\'e.

La maintenance du logiciel est chose co\^uteuse\ldots en efforts. (cf. Le bug de
l'an 2000). Il convient donc de s'orienter vers une simplification et une
clarification du travail par :

\begin{description} 
    \item {\bf La modularit\'e.} On d\'ecompose le programme en autant de fichiers
    que n\'ecessaire afin de r\'eduire les difficult\'es. Cette d\'ecomposition 
    logique et sens\'ee vise \`a favoriser la r\'eutilisabilit\'e 
    des modules. En particulier, il est adroit de d\'ecomposer le probl\`eme afin
    qu'un changement de sp\'ecification n'induise la r\'e\'ecriture que d'un seul
    module.

    D'autre part, les donn\'ees d'un module lui sont
    propres\footnote{C'est l'encapsulation. page suivante} et
    prot\'eg\'ees. L'acc\`es aux donn\'ees se fait par {\it interfaces} c'est \`a dire
    des m\'ethodes ({\it i.e.} des fonctions) appartenant aux modules, accessibles
    depuis d'autres modules.

    \item {\bf La r\'eutilisabilit\'e.} Il est agr\'eable de pouvoir utiliser le
    m\^eme module quelque soit le type de donn\'ees trait\'ees. Pourquoi r\'e\'ecrire un
    calcul de moyenne sur des entiers si l'on dispose d\'ej\`a du m\^eme calcul sur
    des r\'eels ? 

    De m\^eme, les travaux r\'ecents sur les interfaces homme/machine
    ({\sc ihm)}, r\'ev\`ele l'importance de disposer de modules
    ind\'ependants de la pr\'esentation et pouvant \^etre utilis\'es
    avec diff\'erentes pr\'esentations ({\sc html, fvwm}
    ou {\sc tk}).

    \item {\bf La certification.} Il n'existe pas (encore) de solution pour
    une r\'ealisation parfaite des logiciels. Il est cependant possible
    d'inclure des \'el\'ements de sp\'ecification parmi les lignes de code. C'est
    notamment le cas en langage C avec la macro {\tt assert}. Nous l'avons
    abondamment utilis\'e dans la premi\`ere partie. En voici un autre exemple : 

\begin{verbatim}  
#include <assert.h>
main()
{
     int a, b, u, x , y ;

     scanf ("%d%d", &x, &y) ;

     a = x ; b = y ;

     u = x + y ;
     u = u + u ;
     x = x + y ; 
     u = u - x ;

     assert ( u == (a + b)) ;
     assert ( x == (a + b)) ;

     printf ("u :%d, x = %d\n", u, x) ;
}
\end{verbatim} 

\end{description} 
\addcontentsline{lof}{section}{Utilisation de {\tt assert}}

On utilisera cette technique afin de garantir que les conditions
d'utilisation des modules sont respect\'ees. Par exemple, la m\'ethode de retrait
d'un \'el\'ement d'une pile s'assurera que la pile n'est pas vide.

Les assertions contribuent \`a r\'ealiser du logiciel correct, aident \`a la
documentation, facilitent le d\'eboguage et la tol\'erance de pannes.
\newpage
~
\end{multicols}
