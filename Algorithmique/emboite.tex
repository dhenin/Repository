\setcounter{footnote}{0}
\addtocounter{section}{1}
\addcontentsline{toc}{section}{Sch\'emas conditionnels embo\^\i t\'es}
% \addcontentsline{lof}{section}{Ventilation}
\addcontentsline{lof}{section}{Fonction de calcul du nombre de jours \'ecoul\'es depuis le premier janvier}
\markboth{Sch\'emas conditionnels embo\^\i t\'es et ventilation}{Sch\'emas conditionnels embo\^\i t\'es et ventilation}
\centerline{\Large\bf Sch\'emas conditionnels embo\^\i t\'es et ventilation }
\label{Emboite}
\label{Ventilation}
\index{ventilation}
\index{choix}
\index{\'ecoul\'es (nb de jours)}
\index{calendrier}
\index{perp\'etuel (calendrier)}
 
 \noindent\hrulefill  



    Le sch\'ema conditionnel\footnote{Page \pageref{If}.}
permet de r\'egler l'alternative ou choix d'un
    ensemble d'actions {\em parmi} 2 ensembles possibles.

    Lorsque l'on doit r\'ealiser le choix d'un ensemble d'actions {\em parmi}
	$n$ ($n > 2 $) on peut :

\begin{tabular}{lr}
\\
\begin{minipage}[t]{6cm}
{\small 
\listing{Prog/2ifsA.c} }
\end{minipage}  & \begin{minipage}[t]{9cm}
\vspace*{2cm}
-- soit utiliser les structures conditionnelles imbriqu\'ees mais la
     lisibilit\'e de la d\'efinition algorithmique devient rapidement
     malais\'ee d\`es que $n$ devient important.
                  \end{minipage} \\
\\
\\
\begin{minipage}[t]{6cm}
{\small
\listing{Prog/2ifsB.c}}
\end{minipage}  & \begin{minipage}[t]{9cm}
\vspace*{1cm}
-- soit utiliser la ventilation qui g\'en\'eralise de mani\`ere beaucoup
     plus satisfaisante le choix de~: 1 {\em parmi} n.
                  \end{minipage} \\
\end{tabular} 

\paragraph{Exemple}


     Soit \`a \'ecrire un programme qui fournit le nombre de jours
	 \'ecoul\'es depuis le
	 premier janvier jusqu'au d\'ebut du mois.
	 Cet algorithme sera utilis\'e dans
	 le calendrier perp\'etuel page \pageref{Perpetuel}. 



\begin{center}
\begin{tabular}{|l|}
\hline
\multicolumn{1}{|c|}{}\\
\begin{minipage}[t]{7cm}
\begin{verbatim}  
switch (mois)
    {
    case  1 : ecoules =   0 ; break ;
    case  2 : ecoules =  31 ; break ;
    case  3 : ecoules =  59 ; break ;
    case  4 : ecoules =  90 ; break ;
    case  5 : ecoules = 120 ; break ;
    case  6 : ecoules = 151 ; break ;
    case  7 : ecoules = 181 ; break ;
    case  8 : ecoules = 212 ; break ;
    case  9 : ecoules = 243 ; break ;
    case 10 : ecoules = 273 ; break ;
    case 11 : ecoules = 304 ; break ;
    case 12 : ecoules = 334 ; break ;
    }
\end{verbatim} 
\end{minipage} \\
\multicolumn{1}{|c|}{}\\
\hline
\end{tabular} \end{center}
