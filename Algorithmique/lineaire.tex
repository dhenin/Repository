\label{Lineaire}
\section*{D\'eroulement lin\'eaire}
   La forme la plus simple de l'algorithme\footnote{Page
   \pageref{Algorithme} la d\'efinition.}
   est le d\'eroulement lin\'eaire (encha\^\i nement).
%   Ce n'est pas tr\`es compliqu\'e \`a comprendre mais c'est tr\`es important.
%   Gr\^ace au d\'eroulement lin\'eaire, il est possible de regrouper
%   diff\'erentes lignes de programme afin qu'une t\^ache sp\'ecifique puisse
%   \^etre ex\'ecut\'ee.
\begin{itemize}
\item      Le d\'eroulement lin\'eaire ne comporte aucune prise de d\'ecision.
       Les lignes de programme qui s'y trouvent seront toujours ex\'ecut\'ees
       dans le m\^eme ordre.

\item      Le d\'eroulement lin\'eaire est le mode implicite d'ex\'ecution d'un
       programme. 
	   %Si vous n'introduisez pas une 
	   Sans instruction qui
       suspend ou modifie le d\'eroulement lin\'eaire, l'ordinateur {\em passe}
       toujours \`a l'instruction  suivante apr\`es l'ex\'ecution
       de l'instruction courante.

\item La caract\'eristique
       d'un algorithme lin\'eaire est de n'utiliser que l'encha\^\i nement
       s\'equentiel d'actions dont la plus simple est l'affectation\footnote{
	   Page \pageref{Affectation}}.

\end{itemize}
\section*{Propri\'et\'e de l'encha\^\i nement}

Une succession d'instructions poursuit un objectif : se rapprocher
d'un r\'esultat pr\'evu ;  
les relations entre les variables sont modifi\'ees. Bien entendu, l'\'etat final
est le r\'esultat de la succession des \'etats interm\'ediaires.

Exemple :

\begin{tabular}{ll}
\parbox[t]{7cm}{
	Supposons les r\'eels $x$ et $y$,  avec\\
	(P) /* $0<x<1$ */\\
	Pour obtenir \\
	(Q) /* $x > y + 2$  */	

	\vspace*{5mm}

	on ex\'ecute\\
	$	x = \frac{1}{x} +y $ ;\\
	$	x = x + 1           $ ;

	\vspace*{5mm}

             } & \parbox[t]{7cm}{
	En effet, nous avons vu ci-dessus que :\\
		/* $0<x<1$ */\\
		$x = \frac{1}{x}+y$ ; \\
		/* $x>y+1$ */

	\vspace*{5mm}

		et que \\
		/* $x>n$ */\\
		$x = x+1$ ; \\
		/* $x> n+1$ */}
\end{tabular} 
\addcontentsline{lof}{section}{Propri\'et\'e de l'encha\^\i nement}

