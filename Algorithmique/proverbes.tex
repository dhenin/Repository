\addcontentsline{toc}{chapter}{\thechapter ~~ Proverbes}
\markboth{Proverbes}{Proverbes}
\centerline{\Large \bf Proverbes de programmation\footnote{ 
B. Mammeri, {\em Programmation}, \'Ecole centrale de Paris, p. 6 et 7.} }
\label{Proverbe}
\index{Mammeri}

\noindent\hrulefill
{\small 
\begin{multicols}{2}

Ne violez pas les r\`egles
avant de les apprendre.


\begin{description}
\item {\normalsize  {\bf \'Etude du programme}}

      pp \pageref{Methode}, \pageref{Repetition},
      \pageref{Composition}, \pageref{Kmag}

\begin{itemize}

\item Un probl\`eme bien pos\'e, est \`a moiti\'e r\'esolu

      {\em D\'efinissez-le aussi compl\`etement que possible ;
	   nous sommes habitu\'es \`a r\'esoudre des probl\`emes, peu (ou pas) 
	   \`a les poser}


\item Sachez ce que vous allez faire avant de le faire

\item Utilisez l'\'etude descendante

\item M\'efiez-vous des autres \'etudes

\end{itemize}

%\end{center}
%\begin{center}
\item {\normalsize  {\bf \'Ecriture du programme}}

\begin{itemize}

\item Construisez votre programme
en unit\'es logiques

\item Utilisez des proc\'edures

\item \'Evitez des branchements inutiles

\item \'Evitez les effets de bord

\item Soignez la syntaxe tout de suite

\item Choisissez bien vos identificateurs

\item Utilisez proprement
les variables interm\'ediaires

\item Ne touchez pas aux param\`etres d'une boucle

\item Ne recalculez pas
de constante dans une boucle

\item \'Evitez les particularit\'es d'une implantation

\item \'Evitez les astuces 

\item Pr\'evoyez des facilit\'es de mise au point

\item Ne supposez jamais que l'ordinateur
suppose quelque chose

\item Employez des commentaires

\item Soignez la pr\'esentation

\item Fournissez une bonne documentation

\end{itemize}

%\end{center}
%\begin{center}
\item {\normalsize  {\bf Ex\'ecution du programme}}

\begin{itemize}

\item Testez le programme \`a la main avant de l'ex\'ecuter

\item Ne vous occupez pas d'une belle
pr\'esentation des r\'esultats
avant que le programme ne soit correct

\item Quand le programme est correct,
soignez la pr\'esentation des r\'esultats

\end{itemize}

\item {\normalsize {\bf De toutes fa\c cons}}

\begin{itemize}

\item Relisez le manuel

\item Consid\'erez un autre langage

\item N'ayez pas peur de tout recommencer

\end{itemize}

\end{description}



\begin{itemize} 
\item Ne compliquez pas inutilement les choses\\
	  S'il y a une marche \`a suivre \'evidente\\
	  {\bf utilisez-la}

\item L'\'evidence se nourrit de connaissances

\item \'Elargissez le champ de l'\'evidence par la connaissance de programmes
courts

\item Pr\'ecisez la forme des relations entre les variables\\
	  Fixez les param\`etres en examinant ``\`a la main'' le cas g\'en\'eral et {\bf
	  les cas limites}

\item \'Ecrivez de nombreux programmes\\
	  travaillez-les soigneusement\\
	  vous enrichirez votre exp\'erience \\
	  {\bf vous d\'evelopperez votre flair}

\item Si vous ne pouvez pr\'eciser les d\'etails\\
	  autrement que par des essais empiriques\\
	  {\bf N'insistez pas}\\
	  il y a d'autres fa\c cons de faire.

\item Traitez d'abord les cas les plus simples

\item S\'eriez les questions\\
	  {\bf Une seule} question \`a la fois.

\item Essayer un programme peut servir \\
	  \`a montrer qu'il contient des erreurs\\
	  {\bf jamais qu'il est juste}

\item Raisonnez pour que votre programme\\
	  soit juste {\bf par construction}

\item Il faut se m\'efier {\bf comme si l'erreur }\\
	  \'etait in\'evitable

\item Pour comprendre un programme\\
      {\bf explicitez les situations} qu'il engendre

\item Pour cr\'eer le programme,\\
	  {\bf il faut partir } des situations

\item Ne vous demandez pas \\
	  {\bf que vais-je faire ?}\\
	  demandez-vous plut\^ot\\
	  {\bf o\`u en suis-je ?}

\end{itemize} 

%\newpage

\begin{itemize} 

\item Pour construire une boucle\\
	  proposez d'abord une situation g\'en\'erale\\
	  Assurez vous que chaque pas \\
	  {\bf rapproche de } la solution

\item Pour obtenir une situation g\'en\'erale\\
	  Supposez qu'on {\bf a fait une partie du travail}

\item D\'eterminez dans quelles conditions\\
	  {\bf le travail est fini}

\item progressez vers la solution\\
	  {\bf et r\'etablissez } la solution g\'en\'erale

\item trouvez des valeurs initiales\\
	  {\bf satisfaisant } la situation g\'en\'erale


\end{itemize} 
\end{multicols}
}
