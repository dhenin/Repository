\addcontentsline{toc}{chapter}{\thechapter ~ Grille d'analyse}
\setcounter{section}{0}
\addtocounter{section}{1}
\addcontentsline{toc}{section}{D\'ependance des variables}
\markboth{Grille d'analyse}{Grille d'analyse}
\centerline{\Large\bf Grille d'analyse}
\index{grille}
\index{analyse (grille d')}
\index{lexique (des variables)}
\index{s\'equence (ordre des)}
\index{conception}
\index{expliciter (une variable)}
\index{pr\'esupposer}
\index{poids-id\'eal (calcul)}
 
 \noindent\hrulefill  
%\def\Rond#1{$\bigcirc\!\!\!\!{\scriptscriptstyle #1}$}
\def\Rond#1{$(#1)$}
\label{Methode}



La grille d'analyse intervient lors de l'\'etape de l'\'ecriture de
l'algorithme.
Cette grille permet d'organiser l'expression de l'algorithme.

\begin{figure}[h]
\centering
\begin{tabular}{lcrl}
\cline{1-3}
\multicolumn{3}{|l|}{}& \\
\multicolumn{3}{|l|}{\large \bf Titre \hfill}& 
						  \begin{minipage}[t]{4cm}
						  Pr\'ecise le r\'e\-sul\-tat attendu de la suite de
						  d\'efinitions \end{minipage}  \\
\cline{1-3}
\multicolumn{3}{|c|}{\large \bf Lexique $|$ \hfill Actions \hfill $|$ S\'equences} & \\
\cline{1-3}
\multicolumn{2}{|c}{\hfil\Rond{2}\hfill\Rond{1}\hfil}&\multicolumn{1}{c|}{} &  \\
\multicolumn{3}{|c|}{\hfil\rightarrowfill\Rond{3}\hfil}&   \\
\cline{1-3}
\multicolumn{3}{c}{\hfil\Big\uparrow \hfill \begin{minipage}[t]{6 cm}
                          D\'ecrit 
					  les actions \`a ex\'ecuter \end{minipage} \hfill \Big\uparrow \hfill} & \\
\begin{minipage}[t]{5cm} Explicite tous les noms symboliques
identificateurs introduits dans les d\'efinitions, 
et pr\'ecise les types manipul\'es. \end{minipage} & & 
						\begin{minipage}[t]{5 cm}
									   Ordonnance les actions pour un
									   traitement au moyen d'un ordinateur
									   \end{minipage} & \\
& & \\
\end{tabular} 
\addcontentsline{lof}{section}{La grille d'analyse}
	
    \label{grille}
\end{figure}

La grille d'analyse est un tableau de 3 colonnes (fig.~\ref{grille})
r\'ealisant le sch\'ema dans lequel la conception de l'algorithme
s'organise et se d\'eveloppe.

Dans la colonne centrale, {\bf on commence par la derni\`ere action}, ce qui
fait appara\^\i tre une variable au moins, que l'on cherche \`a {\bf expliciter}.
Cette variable en
{\bf pr\'esuppose} d'autres que l'on explicite, et ainsi de suite jusqu'\`a ce que
toutes les variables soient explicit\'ees. Dans la colonne {\it Lexique}, on
pr\'ecise pour chaque variable son domaine de d\'efinition. On termine en fixant
dans la colonne de droite l'ordre d'ex\'ecution pour le programme.

Ordre \Rond{1} $\longleftrightarrow$ \Rond{2} puis \`a la fin \Rond{3}

Exemple~: Calcul du poids id\'eal d'une personne.

\centerline{\begin{tabular}{l|r}
	\begin{minipage}[t]{7cm}
    Le {\tt POIDS} en $Kg$ d\'epend de la {\tt TAILLE} en $cm$\\
     et d'un coefficient~:\\
{\tt POIDS = ECART  $\times$ COEF  \\
	  {\rm o\`u} ECART {\rm repr\'esente} TAILLE - 100, }
\end{minipage} & \begin{minipage}[t]{7cm}
                 \raggedleft
     {\tt coef} est fonction du sexe\\
{\tt \begin{tabular}{r|r|l}
      COEF = & 1.1   & SI SEXE = "masculin"  \\
                & 1 & SI SEXE = "f\'eminin"
      \end{tabular} } \end{minipage} \\
\end{tabular}}
\vspace{.5cm}
\centerline{%\tt %\footnotesize
\begin{tabular}{|l|l|c|} 
\hline
\multicolumn{3}{|l|}
{\small
\bf Poids-id\'eal} \\
\hline
{\small
\bf  Lexique } &{\small
\bf D\'efinitions }&{\small
\bf  S\'equence } \\
\hline
    & \Rond{1} \underline{R\'esultat} = \underline{\'ecrire}  POIDS  &   6   \\
                                &                                     &       \\
\Rond{2}  POIDS (\underline{r\'eel})~: poids en Kg   & \Rond{3} POIDS~$=$ ECART $\times$ COEF       &   5   \\
                                &                                     &       \\
\Rond{4} ECART  (\underline{entier})     & \Rond{5} ECART = TAILLE - 100                &   4   \\
\Rond{6}  TAILLE (\underline{entier})~: taille     & \Rond{7} TAILLE~$=$  \underline{donn\'ee}                    &       \\
  en cm (4)                     &       ('taille en cm $>$ 150')  &   2   \\
                                &                                     &       \\
\Rond{8} COEF  (\underline{r\'eel})~: coefficient & \Rond{9}   {\bf if} (SEXE~ == "masculin")         &   3   \\
de pond\'eration fonction du&                             &       \\
sexe de la personne (10) &   ~~~~~  COEF~$=$ 1.1           &       \\
                                &           {\bf else}                      &       \\
                                &  ~~~~~      COEF~$=$ 1          &       \\
                                &                                     &       \\
\Rond{10}  SEXE (\underline{cha\^\i ne})~: sexe de     & \Rond{11} SEXE~$=$  \underline{donn\'ee}                 &   1   \\
   la personne             &  ("sexe ? r\'epondre par masculin ou   &       \\
                                &   f\'eminin")                         &       \\
\hline
\multicolumn{3}{c}{} \\
\end{tabular}
}
\addcontentsline{lof}{section}{Grille d'analyse du calcul du poids id\'eal}

 Les num\'eros entre parenth\`eses sont indiqu\'es optionnellement pour montrer
   l'ordre de remplissage.
\newpage

%----------------------------------------------------------------------
\addtocounter{section}{1}
\addcontentsline{toc}{section}{Probl\`eme~: Facture de pi\`eces
identiques}
\addcontentsline{lof}{section}{Calcul d'une facture de pi\`eces identiques}
\addcontentsline{lof}{section}{Contr\^ole de l'algorithme \`a l'aide du lexique}
%\markboth{Algorithme}{Algorithme}
\centerline{\Large\bf Probl\`eme~: Facture de pi\`eces identiques}
 
\index{facture (calcul d'une)}
\index{contr\^ole (de l'algorithme)}

%\paragraph{Grille d'analyse}

\centerline{\tt
\begin{tabular}{|l|l|l|} 
\hline
\multicolumn{3}{|l|}{pi\`eces\_id} \\
\hline
       Lexique                  &          D\'efinitions             & S\'equence \\
\hline
PHT (\underline{r\'eel}) prix hors taxe  &  \underline{r\'esultat} =               &     7    \\
   des pi\`eces   & ~~~~ \underline{\'ecrire} "prix hors taxe=" PHT  \underline{\`a la ligne} &   \\
                              &~~~~"prix toute taxe =" PTT       &    \\
                              &                                    &          \\
 PTT (\underline{r\'eel}) prix toute taxe &    PHT~$=$ PU $\times$ N       &     4    \\
   des pi\`eces                 &                                    &          \\
                              &    PTT~$=$ PHT + TAXE               &     6    \\
 PU (\underline{r\'eel}) prix unitaire ht &                               &          \\
                       &    PU~$=$  \underline{donn\'ee} ("prix unitaire")  &     3    \\
 N (\underline{entier}) nombre de pi\`eces&                               &          \\
                              & N~$=$ \underline{donn\'ee} ("nombre de pi\`eces") &     2    \\
TAXE (\underline{r\'eel}) TVA       &                                    &          \\
                              &    TAXE~$=$  TAUX $\times$ PHT         &     5    \\
 TAUX (\underline{r\'eel}) taux de TVA  &                                    &          \\
                              &    TAUX~$=$ 0.186                      &     1    \\
\hline
\end{tabular}
}
 \paragraph{Contr\^ole de l'algorithme \`a l'aide du lexique}~\\
  \`A chaque d\'efinition algorithmique de l'identificateur d'une donn\'ee ou
  d'un r\'esultat interm\'ediaire, on {\em coche} celui-ci dans le
  lexique (fig. ci-dessous).  Ce lexique montre donc \`a {\em chaque
  instant} ce qui reste \`a d\'efinir.

    \begin{figurette}
    \setlength{\unitlength}{0.0125in}%
\begin{picture}(485,310)(10,521)
\thicklines
\put(290,613){\circle{30}}
\put( 80,663){\circle{30}}
\put( 25,701){\circle{30}}
\put(280,733){\circle{30}}
\put( 90,777){\circle{30}}
\put( 35,575){\vector( 0,-1){ 13}}
\put(400,609){\line( 0,-1){  9}}
\put(400,600){\line(-1, 0){365}}
\put( 35,600){\line( 0,-1){  6}}
\put(435,609){\line( 1, 0){ 40}}
\put(385,609){\line( 1, 0){ 30}}
\put(305,613){\vector( 1, 0){ 70}}
\put(275,613){\line(-1, 0){260}}
\put( 30,616){\line( 1, 0){ 35}}
\put( 35,632){\line( 1, 0){ 25}}
\put( 80,654){\vector( 0,-1){ 10}}
\put(420,730){\line( 0,-1){ 38}}
\put(420,692){\line(-1, 0){340}}
\put( 80,692){\line( 0,-1){ 19}}
\put(375,730){\line( 0,-1){ 38}}
\put( 25,692){\vector( 0,-1){ 13}}
\put(345,730){\line( 0,-1){ 10}}
\put(345,720){\line(-1, 0){320}}
\put( 25,720){\line( 0,-1){  9}}
\put(410,730){\line( 1, 0){ 20}}
\put(370,730){\line( 1, 0){ 15}}
\put(335,730){\line( 1, 0){ 20}}
\put(295,733){\vector( 1, 0){ 35}}
\put( 20,733){\line( 1, 0){245}}
\put( 45,739){\line( 1, 0){ 25}}
\put( 90,768){\vector( 0,-1){ 13}}
\put(335,799){\line( 0,-1){  6}}
\put(335,793){\line(-1, 0){245}}
\put( 90,793){\line( 0,-1){  6}}
\put( 35,584){\circle{30}}
\put(310,799){\line( 1, 0){ 45}}
\put( 45,710){\makebox(0,0)[lb]{\raisebox{0pt}[0pt][0pt]{ lorsqu'une variable est d\'efinie algorithmiquement, il faut la cocher dans le lexique}}}
\put(240,831){\line( 0,-1){ 16}}
\put( 15,815){\line( 1, 0){480}}
\put( 15,831){\line( 1, 0){480}}
\multiput(240,815)(0.00000,-8.09524){32}{\line( 0,-1){  4.048}}
\put( 85,521){\makebox(0,0)[lb]{\raisebox{0pt}[0pt][0pt]{ L'algorithme est termin\'e}}}
\put( 85,533){\makebox(0,0)[lb]{\raisebox{0pt}[0pt][0pt]{ Tous les identificateurs sont coch\'es (donc d\'efinis)}}}
\put( 20,546){\makebox(0,0)[lb]{\raisebox{0pt}[0pt][0pt]{ N  ...}}}
\put( 20,556){\makebox(0,0)[lb]{\raisebox{0pt}[0pt][0pt]{ PU  ...}}}
\put( 32,579){\makebox(0,0)[lb]{\raisebox{0pt}[0pt][0pt]{ 6}}}
\put(385,613){\makebox(0,0)[lb]{\raisebox{0pt}[0pt][0pt]{ PU,  N  =  donn\'ees}}}
\put(286,608){\makebox(0,0)[lb]{\raisebox{0pt}[0pt][0pt]{ 5}}}
\put( 20,619){\makebox(0,0)[lb]{\raisebox{0pt}[0pt][0pt]{ N (entier) nombre de pi\`eces}}}
\put( 20,635){\makebox(0,0)[lb]{\raisebox{0pt}[0pt][0pt]{ PU (reel) prix unitaire}}}
\put( 77,659){\makebox(0,0)[lb]{\raisebox{0pt}[0pt][0pt]{ 4}}}
\put( 15,673){\makebox(0,0)[lb]{\raisebox{0pt}[0pt][0pt]{ PHT  ...}}}
\put( 22,697){\makebox(0,0)[lb]{\raisebox{0pt}[0pt][0pt]{ 3}}}
\put(340,733){\makebox(0,0)[lb]{\raisebox{0pt}[0pt][0pt]{ PHT = PU x N}}}
\put(276,728){\makebox(0,0)[lb]{\raisebox{0pt}[0pt][0pt]{ 2}}}
\put( 20,742){\makebox(0,0)[lb]{\raisebox{0pt}[0pt][0pt]{ PHT (r\'eel) prix hors taxe }}}
\put( 88,772){\makebox(0,0)[lb]{\raisebox{0pt}[0pt][0pt]{ 1}}}
\put(255,803){\makebox(0,0)[lb]{\raisebox{0pt}[0pt][0pt]{ r\'esultat = \'ecrire PHT}}}
\put(335,818){\makebox(0,0)[lb]{\raisebox{0pt}[0pt][0pt]{ D\'efinitions}}}
\put( 25,818){\makebox(0,0)[lb]{\raisebox{0pt}[0pt][0pt]{ Lexique}}}
\put(130,775){\makebox(0,0)[lb]{\raisebox{0pt}[0pt][0pt]{ la d\'efinition algorithmique implique une d\'efinition lexicale}}}
\put(175,750){\makebox(0,0)[lb]{\raisebox{0pt}[0pt][0pt]{ tant qu'une variable n'est pas coch\'ee, il lui faut une d\'efinition algorithmique}}}
\end{picture}

    \end{figurette}
