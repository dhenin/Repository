\addcontentsline{toc}{chapter}{\thechapter ~ Affectation}
\setcounter{section}{0}
\markboth{Affectation}{Affectation}
\centerline{\Large \bf L'affectation}
\label{Affectation}
\index{affectation}
\index{\'etat (du programme)}
\index{relation (entre les variables)}
\index{assertion}
\index{s\'emantique}
\index{incr\'ementation}

\noindent\hrulefill

\begin{multicols}{2}

\section*{Propri\'et\'e de l'affectation}

Lorsque l'on fait une affectation\footnote{Exemple pr\'esent\'e page
\pageref{Variable} : {\sc somme = somme + 5}.
}, on poursuit un objectif : se rapprocher
d'un r\'esultat escompt\'e.
la nouvelle valeur de la variable modifie 
l'\'etat du programme, c'est \`a dire les relations entre les variables.

\begin{floatingtable}[l]{
\begin{tabular}{rl}
(P) & /* $x > n$ */ \\
    & $x = x + 1$ ; \\
(Q) & /* $x > n + 1$ */\\
\end{tabular}}\end{floatingtable}


Dans l'exemple ci-contre, 
partant de la situation (P) $x>n$, 
on ex\'ecute l'instruction $x = x + 1 ;$ pour \'etablir la relation (Q) $x>n+1$,  
Ce type d'affectation est si fr\'equemment utilis\'e qu'il porte un nom : {\bf incr\'ementation}, et une \'ecriture abr\'eg\'ee en langage C : {\tt x++}. 

Autre exemple :

\begin{floatingtable}[l]{
\begin{tabular}{rl}
\multicolumn{2}{c}{\hbox to 3cm {}}\\
(P) & /* $0 < x < 1$ */\\
    & $x = \frac{1}{x} +y ; $\\
(Q) & /* $ x > y + 1 $*/  \\
\end{tabular}
}\end{floatingtable}

Pour avoir la relation (Q) $ x > y + 1$, lorsque l'on a la relation (P) 
$0 < x < 1 $, il faut ex\'ecuter : $x = \frac{1}{x} +y $.

En supposant que, dans la relation (P), $x$ vaut $x_{0}$, \\
dans la relation (Q), $x$ vaut $\frac{1}{x_{0}}+y$
$\Rightarrow \frac{1}{x_{0}}+y > y+1$\\
{\tt $0 < x_{0} < 1 $ $\Rightarrow$ $\frac{1}{x_{0}} > 1$
$\Rightarrow$ $\frac{1}{x_{0}}+y>y+1$}

\addcontentsline{lof}{section}{Propri�t� de l'affectation}

%\end{multicols}