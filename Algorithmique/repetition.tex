\addcontentsline{toc}{chapter}{\thechapter ~ R\'ep\'etition}
\label{Repetition}
\addcontentsline{lof}{section}{Propri\'et\'e de la r\'ep\'etition}
\addcontentsline{lof}{section}{Fonction de
                               localisation d'un \'el\'ement dans une liste}
\markboth{Sch\'ema de la  r\'ep\'etition}{Sch\'ema de la  r\'ep\'etition}
\centerline{\Large\bf R\'ep\'etition}
\index{r\'ep\'etition}
\index{propri\'et\'e (de la r\'ep\'etition)}
\index{tant que (while)}
\index{while (tant que)}
\index{invariant (de boucle)}
\index{condition (d'arr\^et)}
\index{localiser (dans une liste)}

 
 \noindent\hrulefill  

\section*{Propri\'et\'es de la r\'ep\'etition}



\`A la ``sortie'' d'une boucle {\tt tant que}, la condition ($C$) de
boucle est toujours {\em fausse} ($\overline{C}$).
Ceci correspond \`a l'utilisation naturelle
de la boucle {\bf while } :
on veut obtenir, en un certain point du
programme, la validit\'e d'une affirmation $\overline{C}$.
On conna\^\i t une action {\tt $A$} (qui
peut \'evidemment avoir une certaine complexit\'e)
dont on esp\`ere qu'elle {\em rapproche}
l'\'etat initial
d'un \'etat o\`u $\overline{C}$ est vraie.
 On {\tt r\'ep\'ete $A$ tant que $C$ est vraie}.

\begin{floatingtable}{ 
\begin{tabular}{|l|}
\hline
\\
\begin{minipage}[t]{3cm}
{\tt
$/* P$ et $C */$\\
while (C)\\
\hbox to .5cm{}\{\\
\hbox to .5cm{} A ;\\
\hbox to .5cm{}\}\\
}
$/* P$ et $\overline{C} */$
\end{minipage} \\
\hline
\end{tabular}}\end{floatingtable}
Si l'action $A$ ne modifie pas la relation $P$, $P$ est 
{\bf invariant} pour la  boucle {\tt while (C) \{ A ; \}}.

Cette propri\'et\'e est extr\^emement importante.
La notion d'invariant de boucle joue un r\^ole d\'ecisif dans la construction de
programmes par des m\'ethodes syst\'ematiques.
En fait, on peut consid\'erer qu'une
boucle {\tt tant que} est enti\`e\-re\-ment d\'efinie par sa condition d'arr\^et
{\bf et} un invariant.
Aussi souvent que possible, nous indiquerons en m\^eme temps qu'une
boucle un invariant significatif associ\'e.

Un invariant est souvent de la forme : ``telle variable  a telle  valeur''.
\`A titre d'exemple, soit le programme ci-dessous, qui localise
dans une liste l'\'el\'ement $x$ :

   {\tt /*} la variable ``{\tt trouve}'' vaut {\tt vrai} si et seulement si $L[p]=x$\\
   et pour tout $i$ compris entre {\tt Premier(L)} et {\tt Acceder(p-1, L)
   $L[p]\neq x$} {\tt */ }

\begin{center}{\tt
\begin{tabular}{|ll|}
\hline
& \\
\multicolumn{2}{|l|}{Localiser$(x, L)$}\\
{\bf \{}  & \\
    & vrai = 1 ; NonTrouve = -1 ; trouve = faux ; \\
    & p = Premier(L) ;     \\
    & /* TANT QUE p n'a pas parcouru toute la liste */   \\
    & {\bf while (}p !=  Fin(L){\bf)} \\
	& {\bf ~~\{} \\
    & \hspace{3mm}{\bf if (}Identique ((Acceder(p, L), x)){\bf)} \\
    & \hspace{6mm}trouve = vrai ; {\bf break} ;  \\
    & \hspace{3mm}{\bf else} p =  Suivant(p, L) ;  \\
	& {\bf ~~\}} \\
	& /* Si trouve == vrai alors renvoie p sinon renvoie NonTrouve */\\
    & {\bf return } (trouve == vrai) {\bf ?} p {\bf : } NonTrouve ; \\
{\bf \}}  & \\
& \\
\hline
\multicolumn{2}{c}{}\\
\end{tabular}} \end{center}


Cet invariant \'etant vrai initialement (puisque {\tt trou\-ve = faux}), il
reste vrai au sortir de la boucle. Joint \`a la n\'egation de la condition de
bouclage, c'est-\`a-dire {\tt /* p = Fin(L) ou trouve */}, il donne com\-me
assertion finale : 

   {\tt /* trouve == faux et aucun \'el\'ement de L ne vaut x, \\
   ou bien trouve == vrai et x = L(p) */}

ce qui est bien le but recherch\'e.

\newpage
%----------------------------------------------------------------------
\addtocounter{section}{1}
\addcontentsline{toc}{section}{Corriger un programme faux}
\addcontentsline{lof}{section}{Un programme faux}
\addcontentsline{lof}{section}{Correction du programme faux}
\centerline{\Large\bf  Corriger un programme faux}
 \index{faux (programme)}
 \index{individuel (l'ordinateur)}
 \index{puissance (calcul de la)}
 \index{corriger (un programme)}
 \index{test d'arr\^et}
 \index{arr\^et (test d')}
 \index{situation ($\neq$ action)}
 \index{boucler}
 \index{d\'emarrage (d'une boucle)}

 
 \noindent\hrulefill  

Dans le num\'ero de janvier-f\'evrier 1979 de l'{\em ordinateur individuel},
est paru ce programme de calcul de {\tt x} puissance {\tt n}, que nous \'etudierons r\'e\'ecrit en langage C~:
{\setlength {\parindent} {0 cm}
\begin{multicols}{2}%{7cm}

{\small%
\begin{verbatim}01 void main()
02    {
03     int x, n, a, i = 0, t ;
04     scanf("%d", &x) ; scanf ("%d, &n) ;
05     a = x ;
06     
07      do{
08        a = a * x ;
09        i = i + 1 ;
10        t = n - 1 ;
11        } while (i < t) ;
12    printf("%d\n", a) ;
13    }
\end{verbatim} }
La ligne (8) op\`ere par
multiplications r\'ep\'et\'ees par $x$. Il est facile de voir
qu'il est faux pour des valeurs simples~:

$ x = 2 $   et  $ n = 1 $

(5) $a$ prend la valeur de $x$, donc $ a = 2$. \\
(3)  $ i = 0$, 
(8)  $a$ est multipli\'e par $x$, $ a = 2 \times 2 = 4 $\\
(9)  $ i$ est augment\'e de 1  $ i = 1 $ \\
(11)  $ i = 1 $ n'est pas inf\'erieur \`a $t$ $(t = 0) $, on
passe donc en (12) et on affiche le r\'esultat {\bf 4},
\'evidemment faux.
\end{multicols}


Examinons la boucle (6-11) :  on arrive la 1$^{\grave ere}$ fois en (6)
en venant de  (5) avec
/* $a = x   \mbox { et } i = 0$ */

Essayons l'assertion $a = x^{i+1}$ 
$\Biggl\{$\begin{tabular}{rrlr}
(9)  &/* $a=x^{x+1}$ */& $a=a\times x$&/* $a=x^{x+2}$ */\\
(10) &/* $a=x^{x+2}$ */& $i=i+1$      &/* $a=x^{x+1}$ */\\
\end{tabular}  

Remarquons que l'assertion $ a = x^{x+1} $ est r\'etablie, avec une
valeur de $i$ plus grande.
Mais la donn\'ee introduite est $i$ .  Il faut donc situer $i$ par
rapport \`a $ n$. 

{\tt
(11) /* $a = x^{i+1}$ */ {\bf si} $i < t$ {\bf alors} $a = x^{i+1}
\mbox { et } i < n - 1$ {\bf boucler} \\
(12) /* $a = x^{i+1} \mbox { et } i>=n - 1 $ */ 
}

Si, en (11), $ i$ devient \'egal \`a $ n - 1 $ alors $ a = x^{n} $
et le programme est correct.
Or par l'initialisation $ a = x  $ et $ i = 0 $ 

Il faut donc avoir $ 0 < n - 1 $  ou $ n > 1 $. Nous constatons que le programme
calcule $ a = x^{n} $ {\bf si}  \mbox { et} {\bf si seulement} $ n > 1 $.

Il n'est pas tr\`es facile de corriger le programme. Il para\^\i t probable que
l'auteur a mal choisi son test d'arr\^et, et qu'il a cherch\'e \`a le corriger en
introduisant la variable $t$ calcul\'ee ligne (10) dont la valeur est constante.
Son calcul dans la boucle n'est pas normal.

{\bf Il est plus important de consid\'erer les situations que les actions.} Pour
{\bf r\'ediger} le programme,
il faut d'abord proposer une situation g\'en\'erale.

\setbox0=\hbox{\begin{minipage}{7cm}
{\small
\begin{verbatim}01 void main()
02     {
03     int x, n, a, i = 0, t ;
04     scanf("%d", &x) ; scanf ("%d, &n) ;
05     a = 1 ;
06     while (i < n)
07     {
07bis     /* a == x^i et i < n */
08        a = a * x ;
09        i = i + 1 ;
10
11      } 
11bis   /* a == x^i et i == n */
12      printf("%d\n", a) ;
13      }
\end{verbatim} }
\end{minipage} }

\begin{multicols}{2}
Supposons que l'on ait fait une partie du travail, 
et calcul\'e $ a = x^{i} $ pour $ i \leq n$. 

{\tt
(7bis)   /* on a calcul\'e $a = x^{i}$  et $i \leq n$ */
}

Si $ i = n $, c'est fini ; 
si non, il faut se rapprocher de la solution en faisant cro\^\i tre $ i$ 
{\tt
(8-9)  $a =  a \times x$ ; $i = i + 1$ ; {\bf boucler en}(6)
}

Il reste \`a voir le d\'emarrage, c'est \`a dire trouver des valeurs de $ a$ et $ i$
telles que l'assertion soit v\'erifi\'ee pour tout $ n$ positif ou nul. Il faut
prendre $ i = 0 $ et donc $ a = x^{0} = 1 $. D'o\`u le nouveau programme correct, cette fois.

\vspace{2mm}

\begin{flushright} \fbox{\box0} \end{flushright}
\end{multicols}}
