\setcounter{footnote}{0}
%\addtocounter{section}{1}
\addcontentsline{toc}{chapter}{\thechapter ~ D�pendances et pr�suppositions}
% \addcontentsline{toc}{section}{D�pendances et pr�suppositions}
\addcontentsline{lof}{section}{Pr�suppositions}
\addcontentsline{lof}{section}{Exemple de pens�e par les pr�suppos�s}
\markboth{D�pendances et pr�suppositions}{D�pendances et pr�suppositions}
\label{makefile}
%\label{pr\'esupposition}
\index{makefile}
\index{pr\'esupposition}
\index{d\'ependance}
 
{\centering {\Large\bf D�pendances et pr�suppositions}\par}


\noindent\hrulefill

\begin{multicols}{2}
\section*{Makefile}
\noindent La r�alisation d'un projet pr�suppose la r�alisation d'�tapes pr�alables.
Par exemple la confection d'un journal n�cessite la r�daction des acticles qui
le constituent. Il est possible de repr�senter cette d�pendance par un graphe
:

\vspace{0.3cm}
{\centering \includegraphics{Figures/journal.eps} \par}
\vspace{0.3cm}

\noindent Chaque nouveau \textit{num�ro} suivra la m\^eme organisation. 

On peut �crire un fichier de description (Makefile) de la confection du
journal~:

\setbox0=\hbox{\footnotesize\begin{minipage}[t]{7.5cm}{\begin{verbatim}
$ARTICLES = article1 article2 article3

maquette : mise_en_page editorial
    assembler pages editorial

editorial : 
    rediger editorial

mise_en_page : sommaire illustrations $ARTICLES
    ajuster espaces sommaire illustrations $ARTICLES

sommaire : $ARTICLES
    lister titres ($ARTICLES)
\end{verbatim}}
\end{minipage}}
%\setbox0=\vtop{\entoure{\box0}}
\box0
\vspace{0.3cm}
\centerline{\box0}
\vspace{0.3cm}

Au cours de la r�alisation d'une maquette, il peut arriver qu'un article soit
r��crit. Il n'est pas pour autant n�cessaire de refaire les autres articles ;
par contre il est probable que le sommaire sera revu (si le titre de l'article
est modifi� par exemple) ainsi que la mise page ; par cons�quent il faudra
refaire la maquette.


\section*{Exemple de pens�e par les pr�suppos�s}

Au moyen de deux r\'ecipients dont les capacit\'es respectives sont neuf litres et
quatre litres, nous souhaitons disposer d'une quantit\'e d'eau de six litres.

Repr\'esentons-nous clairement nos instruments de travail,
c'est-\`a-dire, les deux r\'ecipients\footnote{Proverbe num\'ero 1 {\em un petit
dessin vaut mieux qu'un grand discours}.}.  Imaginons qu'ils soient
cylindriques, de bases \'egales et de hauteurs neuf et quatre (cf.  fig.~1).

S'il y avait, sur la surface lat\'erale de chacun d'eux, une graduation
aux lignes horizontales \'egalement espac\'ees, ce qui donnerait \`a tout
moment la hauteur du niveau de l'eau, notre probl\`eme serait facile.
Mais cette graduation n'existant pas, nous sommes encore loin de la
solution.

\begin{figurette}
    \begin{center}
    \setlength{\unitlength}{3947sp}%
%
\begingroup\makeatletter\ifx\SetFigFont\undefined%
\gdef\SetFigFont#1#2#3#4#5{%
  \reset@font\fontsize{#1}{#2pt}%
  \fontfamily{#3}\fontseries{#4}\fontshape{#5}%
  \selectfont}%
\fi\endgroup%
\begin{picture}(1681,1947)(889,-2446)
\thinlines
\put(1937,-1375){\line( 0,-1){691}}
\put(1937,-2066){\line( 1, 0){518}}
\put(2455,-2066){\line( 0, 1){691}}
\put(901,-511){\line( 0,-1){1555}}
\put(901,-2066){\line( 1, 0){518}}
\put(1419,-2066){\line( 0, 1){1555}}
\put(1247,-2412){\makebox(0,0)[lb]{\smash{\SetFigFont{9}{10.8}{\rmdefault}{\mddefault}{\itdefault}Figure 1}}}
\put(2570,-1432){\makebox(0,0)[lb]{\smash{\SetFigFont{9}{10.8}{\rmdefault}{\mddefault}{\itdefault}4}}}
\put(1535,-626){\makebox(0,0)[lb]{\smash{\SetFigFont{9}{10.8}{\rmdefault}{\mddefault}{\itdefault}9}}}
\end{picture}

    \end{center}
\end{figurette}

Nous ne savons pas encore comment mesurer exactement ; mais
pourrions-nous mesurer une autre quantit\'e ? Faisons des essais, t\^atonnons un
peu. Nous pouvons remplir compl\`etement le plus grand~; si, avec son contenu,
nous remplissons alors le petit, il nous reste  cinq litres dans le grand.
Pouvons-nous \'egalement en obtenir six ? Vidons \`a nouveau les deux r\'ecipients.
Nous pourrions aussi\ldots

Nous agissons ainsi comme la plupart des gens \`a qui l'on pose ce probl\`eme.
Partant de deux r\'ecipients vides, nous faisons un essai, puis un autre, les
vidant et les remplissant \`a tour de r\^ole, et, apr\`es chaque \'echec, nous
recommen\c cons et cherchons autre chose. En somme, nous progressons, en partant
de la situation donn\'ee au d\'ebut, vers la situation finale d\'esir\'ee, c'est \`a
dire en allant du connu vers l'inconnu. Il se peut qu'apr\`es maintes
tentatives nous finissions par r\'eussir mais ce sera par hasard.

Que nous demande-t-on ? Repr\'esentons-nous le plus distinctement possible la
situation finale que nous cherchons \`a atteindre. Imaginons que nous avons l\`a,
devant nous, le grand r\'ecipient contenant exactement six litres et le petit
vide, comme \`a la figure 2. {\em (Partons de ce qui est demand\'e et admettons que
ce que l'on cherche est d\'ej\`a trouv\'e)}.
\begin{figurette}
    \begin{center}
    \setlength{\unitlength}{3947sp}%
%
\begingroup\makeatletter\ifx\SetFigFont\undefined%
\gdef\SetFigFont#1#2#3#4#5{%
  \reset@font\fontsize{#1}{#2pt}%
  \fontfamily{#3}\fontseries{#4}\fontshape{#5}%
  \selectfont}%
\fi\endgroup%
\begin{picture}(1615,1992)(889,-2491)
\put(1255,-2456){\makebox(0,0)[lb]{\smash{\SetFigFont{9}{10.8}{\rmdefault}{\mddefault}{\itdefault}Figure 2}}}
\thinlines
\put(1078,-2102){\line( 1, 1){353.500}}
\put(901,-1218){\line( 1, 1){177}}
\put(901,-1395){\line( 1, 1){354}}
\put(901,-1572){\line( 1, 1){530.500}}
\put(901,-1748){\line( 1, 1){530}}
\put(901,-1925){\line( 1, 1){530}}
\put(901,-2102){\line( 1, 1){530}}
\put(901,-1041){\line( 1, 0){530}}
\put(1962,-1395){\line( 0,-1){707}}
\put(1962,-2102){\line( 1, 0){530}}
\put(2492,-2102){\line( 0, 1){707}}
\put(901,-511){\line( 0,-1){1591}}
\put(901,-2102){\line( 1, 0){530}}
\put(1431,-2102){\line( 0, 1){1591}}
\put(1549,-1041){\makebox(0,0)[lb]{\smash{\SetFigFont{9}{10.8}{\rmdefault}{\mddefault}{\itdefault}6}}}
\put(1255,-2102){\line( 1, 1){176.500}}
\end{picture}

    \end{center}
\end{figurette}
A partir de quelle situation pr\'ec\'edant imm\'ediatement celle-ci pourrions-nous
obtenir la situation finale d\'esir\'ee, comme \`a la figure 2~? {\em (Cherchons \`a
partir de quel ant\'ec\'edent le r\'esultat final pourrait \^etre obtenu)}. Nous
pourrions remplir compl\`etement le grand r\'ecipient, donc, y verser neuf litres~; mais il faudra en retirer trois litres exactement. Pour cela \ldots, il
faudrait avoir d\'ej\`a un litre dans le petit. Voil\`a l'id\'ee~! (Cf. fig. 3);
\begin{figurette}
    \begin{center}
    \setlength{\unitlength}{3947sp}%
%
\begingroup\makeatletter\ifx\SetFigFont\undefined%
\gdef\SetFigFont#1#2#3#4#5{%
  \reset@font\fontsize{#1}{#2pt}%
  \fontfamily{#3}\fontseries{#4}\fontshape{#5}%
  \selectfont}%
\fi\endgroup%
\begin{picture}(1837,2126)(889,-2625)
\put(1593,-637){\makebox(0,0)[lb]{\smash{\SetFigFont{9}{10.8}{\rmdefault}{\mddefault}{\itdefault}9}}}
\thinlines
\put(2223,-2211){\line( 1, 1){188.500}}
\put(2034,-2211){\line( 1, 1){189}}
\put(2034,-2022){\line( 1, 0){566}}
\put(1279,-2211){\line( 1, 1){188.500}}
\put(1090,-2211){\line( 1, 1){377.500}}
\put(901,-2211){\line( 1, 1){566.500}}
\put(901,-2022){\line( 1, 1){566.500}}
\put(901,-1833){\line( 1, 1){566.500}}
\put(901,-1644){\line( 1, 1){566}}
\put(901,-1455){\line( 1, 1){566}}
\put(901,-1266){\line( 1, 1){566}}
\put(901,-1078){\line( 1, 1){566.500}}
\put(901,-889){\line( 1, 1){378}}
\put(901,-700){\line( 1, 1){189}}
\put(901,-511){\line( 1, 0){566}}
\put(901,-511){\line( 0,-1){1700}}
\put(901,-2211){\line( 1, 0){566}}
\put(1467,-2211){\line( 0, 1){1700}}
\put(2034,-1455){\line( 0,-1){756}}
\put(2034,-2211){\line( 1, 0){566}}
\put(2600,-2211){\line( 0, 1){756}}
\put(2726,-2085){\makebox(0,0)[lb]{\smash{\SetFigFont{9}{10.8}{\rmdefault}{\mddefault}{\itdefault}1}}}
\put(1279,-2588){\makebox(0,0)[lb]{\smash{\SetFigFont{9}{10.8}{\rmdefault}{\mddefault}{\itdefault}Figure  3}}}
\put(2411,-2211){\line( 1, 1){189}}
\end{picture}

    \end{center}
\end{figurette}
Mais comment atteindre la situation ainsi trouv\'ee qu'illustre la figure~3~?
{\em (Cherchons \`a nouveau quel pourrait \^etre l'ant\'ec\'edent de cet ant\'ec\'edent)}.
Etant donn\'e qu'il est toujours possible de re-transvaser un r\'ecipient dans le
r\'ecipient d'origine, la situation de la figure 2 est \'equivalente aux
situations des figures 3 et 4.
\begin{figurette}
    \begin{center}
    \setlength{\unitlength}{3947sp}%
%
\begingroup\makeatletter\ifx\SetFigFont\undefined%
\gdef\SetFigFont#1#2#3#4#5{%
  \reset@font\fontsize{#1}{#2pt}%
  \fontfamily{#3}\fontseries{#4}\fontshape{#5}%
  \selectfont}%
\fi\endgroup%
\begin{picture}(1837,2126)(889,-2625)
\put(1279,-2588){\makebox(0,0)[lb]{\smash{\SetFigFont{9}{10.8}{\rmdefault}{\mddefault}{\itdefault}Figure 4}}}
\thinlines
\put(2223,-2211){\line( 1, 1){188.500}}
\put(2034,-2211){\line( 1, 1){189}}
\put(2034,-2022){\line( 1, 0){566}}
\put(2034,-1455){\line( 0,-1){756}}
\put(2034,-2211){\line( 1, 0){566}}
\put(2600,-2211){\line( 0, 1){756}}
\put(901,-511){\line( 0,-1){1700}}
\put(901,-2211){\line( 1, 0){566}}
\put(1467,-2211){\line( 0, 1){1700}}
\put(2726,-2022){\makebox(0,0)[lb]{\smash{\SetFigFont{9}{10.8}{\rmdefault}{\mddefault}{\updefault}1}}}
\put(2411,-2211){\line( 1, 1){189}}
\end{picture}

    \end{center}
\end{figurette}
Il est facile de reconna\^\i tre que, si l'on obtient l'une quelconque des
situations des figures 2, 3 et 4, on obtiendra aussi bien les deux autres ;
mais il n'est pas si facile de tomber juste sur la situation de la figure 4, \`a
moins de l'{\em avoir d\'ej\`a rencontr\'ee}, de l'avoir vue accidentellement, au
cours d'une de nos pr\'ec\'edentes tentatives. En multipliant les exp\'eriences avec
les deux r\'ecipients, nous pouvons avoir r\'ealis\'e quelque chose d'analogue et
nous rappeler, au bon moment, que la situation de la figure 4 peut se
pr\'esenter comme elle est sugg\'er\'ee \`a la figure suivante : en remplissant le grand
\begin{figurette}
    \begin{center}
    \setlength{\unitlength}{3947sp}%
%
\begingroup\makeatletter\ifx\SetFigFont\undefined%
\gdef\SetFigFont#1#2#3#4#5{%
  \reset@font\fontsize{#1}{#2pt}%
  \fontfamily{#3}\fontseries{#4}\fontshape{#5}%
  \selectfont}%
\fi\endgroup%
\begin{picture}(1578,1946)(889,-2445)
\put(1246,-2411){\makebox(0,0)[lb]{\smash{\SetFigFont{8}{9.6}{\rmdefault}{\mddefault}{\itdefault}Figure  5}}}
\thinlines
\put(1074,-2066){\line( 1, 1){172.500}}
\put(901,-2066){\line( 1, 1){173}}
\put(901,-1893){\line( 1, 0){518}}
\put(1937,-1375){\line( 0,-1){691}}
\put(1937,-2066){\line( 1, 0){518}}
\put(2455,-2066){\line( 0, 1){691}}
\put(901,-511){\line( 0,-1){1555}}
\put(901,-2066){\line( 1, 0){518}}
\put(1419,-2066){\line( 0, 1){1555}}
\put(1477,-1893){\makebox(0,0)[lb]{\smash{\SetFigFont{8}{9.6}{\rmdefault}{\mddefault}{\updefault}1}}}
\put(1246,-2066){\line( 1, 1){173}}
\end{picture}

    \end{center}
\end{figurette}
r\'ecipient, puis en vidant deux fois de suite quatre litres dans le petit et de
l\`a dans le r\'ecipient d'origine, nous rencontrons finalement quelque chose de
d\'ej\`a connu ; ainsi, par la m\'ethode de l'analyse, par {\em raisonnement
r\'egressif} nous avons d\'ecouvert la succession d'op\'erations appropri\'ee.
\begin{figurette}
    \begin{center}
    \setlength{\unitlength}{3947sp}%
%
\begingroup\makeatletter\ifx\SetFigFont\undefined%
\gdef\SetFigFont#1#2#3#4#5{%
  \reset@font\fontsize{#1}{#2pt}%
  \fontfamily{#3}\fontseries{#4}\fontshape{#5}%
  \selectfont}%
\fi\endgroup%
\begin{picture}(1798,2081)(889,-2580)
\put(1579,-634){\makebox(0,0)[lb]{\smash{\SetFigFont{9}{10.8}{\rmdefault}{\mddefault}{\itdefault}9}}}
\thinlines
\put(2195,-2174){\line( 1, 1){369}}
\put(2010,-2174){\line( 1, 1){554}}
\put(2010,-1990){\line( 1, 1){554.500}}
\put(2010,-1805){\line( 1, 1){369.500}}
\put(1271,-2174){\line( 1, 1){184}}
\put(1086,-2174){\line( 1, 1){184.500}}
\put(1086,-1990){\line(-1,-1){184.500}}
\put(1455,-1065){\line(-1,-1){184.500}}
\put(1455,-881){\line(-1,-1){369}}
\put(1455,-696){\line(-1,-1){554}}
\put(1086,-511){\line(-1,-1){185}}
\put(1271,-511){\line(-1,-1){370}}
\put(1455,-511){\line(-1,-1){554}}
\put(901,-1805){\line( 1,-1){185}}
\put(901,-1620){\line( 1,-1){370}}
\put(901,-1435){\line( 1,-1){554.500}}
\put(901,-1250){\line( 1,-1){554.500}}
\put(1086,-1250){\line( 1,-1){369.500}}
\put(1271,-1250){\line( 1,-1){184.500}}
\put(901,-1250){\line( 1, 0){554}}
\put(901,-1990){\line( 1, 0){554}}
\put(901,-511){\line( 1, 0){554}}
\put(2010,-1435){\line( 1, 0){554}}
\put(2010,-1435){\line( 0,-1){739}}
\put(2010,-2174){\line( 1, 0){554}}
\put(2564,-2174){\line( 0, 1){739}}
\put(901,-511){\line( 0,-1){1663}}
\put(901,-2174){\line( 1, 0){554}}
\put(1455,-2174){\line( 0, 1){1663}}
\put(1517,-1990){\makebox(0,0)[lb]{\smash{\SetFigFont{9}{10.8}{\rmdefault}{\mddefault}{\updefault}1}}}
\put(1517,-1250){\makebox(0,0)[lb]{\smash{\SetFigFont{9}{10.8}{\rmdefault}{\mddefault}{\updefault}5}}}
\put(1271,-2544){\makebox(0,0)[lb]{\smash{\SetFigFont{9}{10.8}{\rmdefault}{\mddefault}{\itdefault}Figure 6}}}
\put(2687,-1497){\makebox(0,0)[lb]{\smash{\SetFigFont{9}{10.8}{\rmdefault}{\mddefault}{\itdefault}4}}}
\put(2379,-2174){\line( 1, 1){184.500}}
\end{picture}

    \end{center}
\end{figurette}
Il est vrai que cela s'est fait \`a rebours, mais nous n'avons plus qu'\`a {\em
renverser le processus} en {\em partant du dernier point atteint dans notre
analyse}. Nous faisons les op\'erations sugg\'er\'ee par la figure 5 et obtenons la
figure 4, puis nous passons \`a la figure 3, de l\`a \`a la figure 2 et finalement \`a
la figure 1. {\em En revenant sur nos pas, nous arrivons finalement \`a trouver
ce qui nous \'etait demand\'e}.\footnote{C'est � Platon que la tradition grecque
attribuait la d\'ecouverte de la m\'ethode d'analyse.}

3. Il y a certainement dans cette m\'ethode quelque chose d'assez profond.
L'obligation de sinuer, de s'\'eloigner du but, en revenant en arri\`ere, de ne
pas prendre la route qui m\`ene directement au r\'esultat d\'esir\'e entra\^\i ne
certaines difficult\'es, dans le domaine de l'esprit. Pour d\'ecouvrir la
succession d'op\'erations appropri\'ees, notre intellect doit suivre un ordre
exactement \`a l'inverse de l'ordre r\'eel. Il n'est nul besoin de g\'enie pour
r\'esoudre un probl\`eme en revenant en arri\`ere. Il suffit de se concentrer sur le
but d\'esir\'e, de se repr\'esenter la situation finale que l'on veut obtenir. A
partir de quelle situation pr\'ec\'edente pourrions-nous y parvenir ? Il est
essentiel de se poser cette question et, ce faisant, l'on revient en arri\`ere.
\end{multicols}
