\setcounter{section}{6}
%\addtocounter{section}{1}
\addcontentsline{toc}{section}{Exemple : Carr\'es magiques}
%\markboth{Algorithme}{Algorithme}
\centerline{\Large \bf Carr\'es magiques\footnote{{\it in PPC }
Dh\'enin Jean-Jacques}}
\label{Kmag}
\index{carr\'es}
\index{carr\'es magique}
\index{magique (carr\'es)}
\index{espace (occup\'e)}
\index{c\^ot\'e (du carr\'e)}
\index{Dh\'enin@Dh\'enin, Jean-Jacques}

\noindent\hrulefill

\vspace{.5cm}

%\renewcommand{\theenumi}{\Roman{enumi}}
%\renewcommand{\theenumi}{\arabic{enumi}}

%\lettrine
{Remplir} un carr\'e magique : c'est 
%toujours cette expression que l'on retrouve dans l'\'enonc\'e du probl\`eme ;
%il s'agit de 
disposer les nombres 0 \`a $p$
dans un carr\'e de $n \times n$ cases, de telle sorte que la somme des nombres
dans chaque ligne, chaque colonne et sur les deux diagonales soit la m\^eme. D\`es
que le carr\'e est sup\'erieur \`a $5 \times 5$ cases, il devient n\'ecessaire de
disposer d'une m\'ethode. Nous allons consid\'erer un exemple.

\addcontentsline{lof}{section}{Remplir un carr\'e magique}

Remplissons une grille de $7 \times 7$.

\begin{floatingfigure}[l]{45mm}
\setlength{\unitlength}{0.00087500in}%
%
\begingroup\makeatletter\ifx\SetFigFont\undefined%
\gdef\SetFigFont#1#2#3#4#5{%
  \reset@font\fontsize{#1}{#2pt}%
  \fontfamily{#3}\fontseries{#4}\fontshape{#5}%
  \selectfont}%
\fi\endgroup%
\begin{picture}(1734,1734)(394,-1288)
\put(766,-691){\makebox(0,0)[lb]{\smash{\SetFigFont{12}{14.4}{\sfdefault}{\mddefault}{\updefault}6}}}
\thicklines
\multiput(406,-241)(5.62500,-5.62500){25}{\makebox(6.3500,9.5250){\SetFigFont{7}{8.4}{\rmdefault}{\mddefault}{\updefault}.}}
\put(406,-241){\vector(-1, 1){0}}
\multiput(1981,-241)(5.62500,-5.62500){25}{\makebox(6.3500,9.5250){\SetFigFont{7}{8.4}{\rmdefault}{\mddefault}{\updefault}.}}
\put(1981,-241){\vector(-1, 1){0}}
\multiput(1756,-16)(5.62500,-5.62500){25}{\makebox(6.3500,9.5250){\SetFigFont{7}{8.4}{\rmdefault}{\mddefault}{\updefault}.}}
\put(1756,-16){\vector(-1, 1){0}}
\multiput(1531,209)(5.62500,-5.62500){25}{\makebox(6.3500,9.5250){\SetFigFont{7}{8.4}{\rmdefault}{\mddefault}{\updefault}.}}
\put(1531,209){\vector(-1, 1){0}}
\multiput(1306,434)(5.62500,-5.62500){25}{\makebox(6.3500,9.5250){\SetFigFont{7}{8.4}{\rmdefault}{\mddefault}{\updefault}.}}
\put(1306,434){\vector(-1, 1){0}}
\multiput(1306,-1141)(5.62500,-5.62500){25}{\makebox(6.3500,9.5250){\SetFigFont{7}{8.4}{\rmdefault}{\mddefault}{\updefault}.}}
\put(1306,-1141){\vector(-1, 1){0}}
\multiput(1081,-916)(5.62500,-5.62500){25}{\makebox(6.3500,9.5250){\SetFigFont{7}{8.4}{\rmdefault}{\mddefault}{\updefault}.}}
\put(1081,-916){\vector(-1, 1){0}}
\multiput(856,-691)(5.62500,-5.62500){25}{\makebox(6.3500,9.5250){\SetFigFont{7}{8.4}{\rmdefault}{\mddefault}{\updefault}.}}
\put(856,-691){\vector(-1, 1){0}}
\multiput(856,-466)(5.62500,-5.62500){25}{\makebox(6.3500,9.5250){\SetFigFont{7}{8.4}{\rmdefault}{\mddefault}{\updefault}.}}
\put(856,-466){\vector(-1, 1){0}}
\multiput(631,-241)(5.62500,-5.62500){25}{\makebox(6.3500,9.5250){\SetFigFont{7}{8.4}{\rmdefault}{\mddefault}{\updefault}.}}
\put(631,-241){\vector(-1, 1){0}}
\put(856,-601){\vector( 1, 0){135}}
\multiput(451,164)(12.02290,0.00000){132}{\makebox(6.3500,9.5250){\SetFigFont{10}{12}{\rmdefault}{\mddefault}{\updefault}.}}
\multiput(451,-61)(12.02290,0.00000){132}{\makebox(6.3500,9.5250){\SetFigFont{10}{12}{\rmdefault}{\mddefault}{\updefault}.}}
\multiput(451,-286)(12.02290,0.00000){132}{\makebox(6.3500,9.5250){\SetFigFont{10}{12}{\rmdefault}{\mddefault}{\updefault}.}}
\multiput(451,-511)(12.02290,0.00000){132}{\makebox(6.3500,9.5250){\SetFigFont{10}{12}{\rmdefault}{\mddefault}{\updefault}.}}
\multiput(451,-736)(12.02290,0.00000){132}{\makebox(6.3500,9.5250){\SetFigFont{10}{12}{\rmdefault}{\mddefault}{\updefault}.}}
\multiput(451,-961)(12.02290,0.00000){132}{\makebox(6.3500,9.5250){\SetFigFont{10}{12}{\rmdefault}{\mddefault}{\updefault}.}}
\multiput(676,389)(0.00000,-12.02290){132}{\makebox(6.3500,9.5250){\SetFigFont{10}{12}{\rmdefault}{\mddefault}{\updefault}.}}
\put(901,389){\makebox(6.3500,9.5250){\SetFigFont{10}{12}{\rmdefault}{\mddefault}{\updefault}.}}
\multiput(901,389)(0.00000,-12.02290){132}{\makebox(6.3500,9.5250){\SetFigFont{10}{12}{\rmdefault}{\mddefault}{\updefault}.}}
\multiput(1351,389)(0.00000,-12.02290){132}{\makebox(6.3500,9.5250){\SetFigFont{10}{12}{\rmdefault}{\mddefault}{\updefault}.}}
\multiput(1576,389)(0.00000,-12.02290){132}{\makebox(6.3500,9.5250){\SetFigFont{10}{12}{\rmdefault}{\mddefault}{\updefault}.}}
\multiput(1801,389)(0.00000,-12.02290){132}{\makebox(6.3500,9.5250){\SetFigFont{10}{12}{\rmdefault}{\mddefault}{\updefault}.}}
\multiput(1126,389)(0.00000,-12.02290){132}{\makebox(6.3500,9.5250){\SetFigFont{10}{12}{\rmdefault}{\mddefault}{\updefault}.}}
\put(541,-466){\makebox(0,0)[lb]{\smash{\SetFigFont{12}{14.4}{\sfdefault}{\mddefault}{\updefault}0}}}
\put(1891,-241){\makebox(0,0)[lb]{\smash{\SetFigFont{12}{14.4}{\sfdefault}{\mddefault}{\updefault}1}}}
\put(1666,-16){\makebox(0,0)[lb]{\smash{\SetFigFont{12}{14.4}{\sfdefault}{\mddefault}{\updefault}2}}}
\put(1441,209){\makebox(0,0)[lb]{\smash{\SetFigFont{12}{14.4}{\sfdefault}{\mddefault}{\updefault}3}}}
\put(1216,-1141){\makebox(0,0)[lb]{\smash{\SetFigFont{12}{14.4}{\sfdefault}{\mddefault}{\updefault}4}}}
\put(991,-691){\makebox(0,0)[lb]{\smash{\SetFigFont{12}{14.4}{\sfdefault}{\mddefault}{\updefault}7}}}
\put(991,-871){\makebox(0,0)[lb]{\smash{\SetFigFont{12}{14.4}{\sfdefault}{\mddefault}{\updefault}5}}}
\put(766,-466){\makebox(0,0)[lb]{\smash{\SetFigFont{12}{14.4}{\sfdefault}{\mddefault}{\updefault}8}}}
\put(451,-1186){\framebox(1575,1575){}}
\end{picture}

\end{floatingfigure}

Nous pla\c cons tout d'abord 0 dans la case du milieu du bord gauche. En suivant
la petite fl\`eche nous sortirions du carr\'e. Il faut donc placer 1 dans la case
correspondante sur le bord oppos\'e, et ainsi de suite jusqu'\`a 6. Il n'est plus
possible de continuer puisque la case suivante est occup\'ee. Il suffit de
placer 7 \`a droite de 6 puis de reprendre la progression.

Essayez de terminer seul ; vous pourrez v\'erifier que vous avez r\'ealis\'e
un {\it carr\'e magique} puisque la somme des nombres plac\'es dans chaque ligne,
chaque colonne et chaque diagonale est constante (168).


L'inconv\'enient majeur de cette solution c'est l'espace m\'emoire occup\'e
puisqu'il faut se doter d'un tableau de $n \times n$ cases.

Un autre
algorithme consiste \`a produire la valeur de chaque case dans l'ordre de
lecture, de gauche \`a droite et de bas en haut.

Supposons, pour simplifier, un {\it carr\'e magique} de $5 \times 5$ d\'ej\`a
r\'ealis\'e et \'etudions sa composition : 

\begin{quote}  
\centerline{\begin{tabular}{|c|c|c|c|c|c|c|c|c|c|c|}
\cline{1-5} \cline{7-11}
14 & 15 & 21 &  2 &  8 & & $2n+4$ & $3n+0$ & $4n+1$ & $0n+2$ & $n+3$ \\
\cline{1-5} \cline{7-11}
 7 & 13 & 19 & 20 &  1 & & $n+2$ & $2n+3$ & $3n+4$ & $4n+0$ & $0n+1$ \\
\cline{1-5} \cline{7-11}
 0 & 6 & 12 & 18 & 24 & & $0n+0$ & $n+1$ & $2n + 2 $ & $ 3n+3$ & $4n+4$ \\
\cline{1-5} \cline{7-11}
23 & 4 & 5 & 11 & 17 & & $4n + 3 $ & $ 0n + 4 $ & $n + 0 $ & $2n+1$ & $3n+2$ \\
\cline{1-5} \cline{7-11}
16 & 22 & 3 & 9 & 10 & & $3n+1 $ & $ 4n + 2$ & $ 0n + 3 $ & $n +4 $ & $2n+0$ \\
\cline{1-5} \cline{7-11}
\end{tabular}}
\end{quote}  
\addcontentsline{lof}{section}{Un carr\'e magique de $5 \times 5$}

En exprimant le contenu de chaque case par rapport au c\^ot\'e du carr\'e $n$ la
progression d'une case \`a la suivante est simple : les coefficients progressent
r\'eguli\`erement de 0 \`a $n-1$. 

$ X_{(ij)+1} = ( a_{ij}+1 \times n ) + b_{ij} +1 $ ou $a$ et $b \in \{0, 1, 2\ldots n-1\} $

Lorsqu'une ligne est remplie, on passe \`a la premi\`ere case de la ligne suivante
en retranchant 1 au contenu de la derni\`ere case de la ligne qui vient de
s'achever. Enfin le premier terme du {\it carr\'e magique} est obtenu par :
$ x_{0} = \frac{n-1}{2} \times n + n -1$ 

D'o\`u l'algorithme : 

\begin{center}\begin{tabular}{|l|l|}
\hline
\begin{minipage}[t]{7.5cm}{\begin{verbatim}  

cote = SAISIR () ;
b = cote - 1 ;
a = (cote - 1) / 2 ;

for (ligne = 0 ; ligne < cote ; ligne++)
  {
  for (col = 0 ; col < cote ; col++)
    {
    IMPRIMER (a * cote) + b ;
\end{verbatim}
\hfill $_{...}/^{...}$

~}\end{minipage} & \begin{minipage}[t]{6cm}{\begin{verbatim}  

    if (col != (cote - 1))
      {
      a++ ; a = a modulo cote ;
      b++ ; b = b modulo cote ;
      }
    }
    b-- ;
  }
\end{verbatim} }\end{minipage}\\
\hline
\end{tabular}
\end{center} 
\addcontentsline{lof}{section}{Algorithme du carr\'e magique}
\newpage
~
