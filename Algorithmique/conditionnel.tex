\addcontentsline{toc}{chapter}{\thechapter ~ Choix conditionnel}
\setcounter{section}{0}
\markboth{Choix conditionnel}{Choix conditionnel}
\centerline{\Large\bf Choix conditionnel}
\label{Conditionnel}
\index{condition}
\index{d\'eterminisme}
\index{raisonner}
\index{compteur programme}
\index{s\'election}
 
 \noindent\hrulefill  
\section*{Les s\'elections : choisir c'est exclure}
L'algorithme lin\'eaire\footnote{Page \pageref{Lineaire}}
correspond \`a un sch\'ema tr\`es simple : les actions
s'encha\^\i nent dans un ordre fig\'e. La r\'ealit\'e est plus souvent construite
suivant un sch\'ema conditionnel. La mise en \oe uvre d'algorithmes
conditionnels permet de supprimer le d\'eterminisme li\'e aux algorithmes
lin\'eaires. En programmant la prise de d\'ecision nous donnerons \`a
l'ordinateur la capacit\'e de {\em raisonner}, c'est-\`a-dire de suivre une
d\'emarche logique (exemple : jouer aux \'echecs) donnant au profane
l'impression que l'ordinateur est capable de {\em penser}.


La puissance de calcul de votre ordinateur est mise en \oe uvre lorsqu'il
\'evalue les expressions contenues dans les lignes de programmes.
Le pouvoir de d\'ecision est utilis\'e pour d\'eterminer l'ordre
d'ex\'ecution des lignes. 

Pour bien saisir le concept de prise de d\'ecision d'un ordinateur, il
faut savoir ce qu'est le compteur programme. Le compteur programme est
la partie du syst\`eme interne de l'ordinateur capable d'indiquer \`a
l'ordinateur la prochaine ligne \`a ex\'ecuter. \`A moins d'une indication
contraire, le compteur programme s'incr\'emente \`a la fin de chaque ligne
afin d'indiquer la prochaine ligne du programme.


%\paragraph{S\'election}~\\
La s\'election ou ex\'ecution conditionnelle constitue le c\oe ur du pouvoir
de d\'ecision d'un ordinateur. Comme ce nom l'indique, en fonction des
r\'esultats d'un test ou d'une condition, une partie de programme est
ex\'ecut\'ee ou non. Ceci est la fonction fondamentale qui nous fait
croire que la machine est capable de raisonner. En effet, dans le cas de
l'ex\'ecution conditionnelle, le programme prend en compte et refl\`ete
totalement le raisonnement du programmeur.

Prenons comme exemple un laboratoire de chimie. Un ordinateur n'y sera
pas d'une grande utilit\'e si sa fonction se limite \`a ouvrir une valve
lorsqu'un technicien appuie sur le bouton \fbox{\sc start} . Dans ce cas, le
technicien fera aussi bien de l'ouvrir lui-m\^eme. Toutefois, l'ordinateur
ex\'ecutera une t\^ache beaucoup plus utile s'il ouvre la valve lorsqu'on
appuie sur \fbox{\sc start} et la ferme lorsqu'on atteint la valeur donn\'ee du
pH. Il peut surpasser le technicien par exemple dans l'utilisation de
valves t\'el\'ecommand\'ees et des dispositifs de mesure \'electroniques du pH.
Dans cet exemple, le c\^ot\'e utile de l'ordinateur demeure sa capacit\'e de
d\'ecider de la fermeture de la valve. En fait, c'est le programmeur qui
sp\'ecifie les crit\`eres de d\'ecision. Ces crit\`eres sont ensuite communiqu\'es
\`a l'ordinateur gr\^ace aux structures d'ex\'ecution conditionnelle du
programme. En cons\'equence, L'ordinateur est capable d'interpr\'eter la
d\'ecision du programmeur \`a une vitesse et \`a une pr\'ecision plus grandes
que celles d'un \^etre humain.


Il existe diverses applications d'instructions d'ex\'ecution conditionnelle 
\begin{enumerate}
 \item Choix conditionnel d'un segment parmi deux\footnote{Page \pageref{If}},
 \item Ex\'ecution conditionnelle d'un segment (Sch\'ema conditionnel simplifi\'e).
 \item Choix conditionnel d'un segment parmi plusieurs\footnote{Page
 \pageref{Ventilation}}.
\end{enumerate}
