\stepcounter{section}
\addcontentsline{toc}{section}{Encapsulation}
\markboth{Encapsulation}{Encapsulation}
\centerline{\Large\bf L'encapsulation}
 \index{Encapsulation}
 \label{Encapsulation}
 
 \noindent\hrulefill  
\begin{multicols}{2}

La modification imperceptible des variables globales est une des principales
difficult\'es rencontr\'ees dans la maintenance des programmes. C'est pourquoi
la premi\`ere recommandation est de {\bf ne pas utiliser de variable globale}.

Dans la construction des types de donn\'ees abstraits, 
la syst\'ematisation de cette id\'ee conduit \`a n'admettre l'acc\`es aux donn\'ees
qu'au moyen de fonctions associ\'ees au types de donn\'ees.





% \section{Domaine de visibilit\'e des variables et fonctions}

R\`egles des espaces de visibilit\'e~: 
\begin{itemize}

\item Un programme C est compos\'e de nombreux \'el\'ements qui sont des variables
ou des fonctions.  Afin d'\'eviter les conflits de noms, le langage permet
de restreindre la visibilit\'e de certains objets qui n'ont aucune raison
d'\^etre globaux (comme les indices de boucles).

\item L'espace de visibilit\'e d'une variable ou d'une fonction correspond \`a la
partie de source pour laquelle elle est d\'efinie (connue).

\item De mani\`ere g\'en\'erale une variable est d\'efinie depuis sa d\'eclaration
jusqu'\`a l'accolade fermente qui la suit ou jusqu'\`a la fin du module si
elle n'est pas \`a l'int\'erieur d'une paire d'accolades.

\item Une variable d\'efinie dans une fonction n'est donc visible que dans cette
fonction.  Une variable d\'eclar\'ee hors de toute fonction est visible
jusqu'\`a la fin de son module.  Elle peut, dans ce cas, \^etre rendue
visible depuis les autres modules de l'application (c'est \`a dire
export\'ee) ou rester priv\'ee au module qui la d\'eclare.

\item Une fonction est visible dans l'ensemble du fichier qui la contient.

\item Elle peut \^etre rendue visible des autres modules ou conserv\'ee priv\'ee.
Exemple~: 

\begin{small}
\begin{verbatim}
/* Hors du module,
 * importee
 */
extern int V1 ; 

/* Globale au module,
 * exportee vers les autres
 */
int V2 ;

/* Globale au module,
 * non visible depuis les autres
 */
static int V3 ;

/* V1, V2, V3, F1 et F2
 * sont visibles
 */

/* Fonction non exportee */
static int F1 ()
{
    int V4 ; /* Locale a F1 */
    int V2 ;
    /* La variable V2 globale
     * est cachee par celle-ci 
     * V1, V2, V3, V4, F1 et F2
     * sont visibles 
     * La version globale de V2
     * est inaccessible
     */
}

/* V1, V2, V3, F1 et F2
 * sont visibles
 */
int V5 ;

/* V1, V2, V3, V5, F1 et F2
 * sont visibles
 */

int F2 () /* Fonction exportee */
{
    /* Variable locale permanente */
     static V6 ;
    /* V1, V2, V3, V5, V6, F1 et F2
     * sont visibles
     */
}

/* V1, V2, V3, V5, F1 et F2
 * sont visibles
 */

/* V2, V5 et F2
 * sont visibles depuis
 * d'autres modules
 */

\end{verbatim}
\end{small}

\addcontentsline{lof}{section}{Visibilit� des variables}

\item Si l'on veut qu'une variable d\'eclar\'ee dans un autre fichier soit visible
dans le module courant, elle doit \^etre d\'eclar\'ee externe dans celui-ci.
{\it A contrario}, une fonction appel\'ee dans un module est suppos\'ee externe.

Si le compilateur la rencontre dans ce m\^eme module, il supprime son
importation.  En d'autres termes, on peut appeler toutes les fonctions
de tous les modules sans avoir \`a annoncer explicitement qu'elles sont
externes.
\end{itemize}

La derni\`ere \'etape de construction de l'application consiste \`a collecter
l'ensemble de ses constituants pour cr\'eer l'ex\'ecutable.  C'est l'\'edition
des liens.  L'application ne peut \^etre correctement construite que si
chacun des objets import\'es par chacun des modules est publi\'e par
exactement un autre module.  C'est \`a dire que tous les objets que les
modules importent, d'autres les exportent.
\end{multicols}
