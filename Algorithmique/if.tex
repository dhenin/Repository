\markboth{Sch\'ema alternatif}{Sch\'ema alternatif}
\label{If}
\index{if}
\index{then}
\index{else}
\index{sauter (un segment)}
\index{segment (de programme)}
\index{expression}
\index{alternatif (sch\'ema)}
\index{vraie (expression)}
\index{fausse (expression)}
\index{pr\'edicat}
\index{fonction propositionnelle}
\index{bool\'een}
\index{valeur (bool\'eenne)}
\index{propri\'et\'e (de l'alternative)}


 
%\noindent\hrulefill  

\section*{L'alternative}

%\section{If\ldots then\ldots else}
Gr\^ace \`a l'instruction  {\bf if } \ldots  {\bf else}, vous pouvez ex\'ecuter ou
sauter un segment de programme. Cette instruction contient une
expression  qui peut \^etre vraie ou fausse. Si elle est vraie
(diff\'erente de z\'ero), le segment conditionnel est ex\'ecut\'e. Si elle est
fausse (\'egale \`a z\'ero), le segment conditionnel est ignor\'e.

Le segment conditionnel peut \^etre soit une seule instruction, soit
une partie de programme comprenant un nombre quelconque d'instructions.


\vspace{3mm}

\begin{tabular}{ll}
\begin{tabular}{ll}
                          {\bf if}& {\bf (}condition{\bf )}  \\
                           & {\bf \{} action(s) 1 {\bf \}}\\
                             {\bf else} & \\
                           & {\bf \{} action(s) 2 {\bf \}} \\
\end{tabular} & \begin{minipage}{10 cm}
La condition ou {\em pr\'edicat} est {\em une fonction
   propositionnelle} dont le r\'esultat est bool\'een c'est-\`a-dire a 2
   valeurs : vrai, faux.

Il existe un sch\'ema conditionnel simplifi\'e qui omet le deuxi\`eme terme de
l'alternative\footnote{il existe en langage C une abr\'eviation pour ce sch\'ema ; 
par exemple le calcul du maximum de 2 nombres peut s'\'ecrire : max  = a $>$ b {\bf ?} a {\bf :}  b ;}.
\end{minipage} \\
\end{tabular} 

Les actions internes 1 et 2 du sch\'ema conditionnel peuvent \^etre elles-m\^emes
conditionnelles. On obtient alors des structures conditionnelles
embo\^\i t\'ees\footnote{Page \pageref{Emboite}}. 


\section*{Propri\'et\'e de l'alternative}


La succession d'instructions poursuit un objectif : se rapprocher
d'un r\'esultat pr\'evu. Avec la structure de contr\^ole {\bf if } \ldots  {\bf
else}, il y a 2 chemins possibles pour se rapprocher du m\^eme objectif, 
il y a 2 possibilit\'es de modifier les relations entre les variables.

Supposons 

\begin{itemize} 
\item les relations $P$ et $C$ entre les variables, et qu'une action $A$ 
conduise \`a une nouvelle relation $Q$,
\item les relations $P$ et $\overline{C}$ entre les variables,
et qu'une action $B$ conduise \`a la m\^eme relation $Q$.
\end{itemize} 

La propri\'et\'e de la structure de contr\^ole {\bf if } \ldots  {\bf else}, s'\'ecrit~:
\begin{verbatim}  
/* P */ if (C) A ; else B ; /* Q */
\end{verbatim} 

Exemple :

\begin{tabular}{ll}
{\tt \begin{tabular}{ll|}
\multicolumn{2}{c}{\rm pour montrer que}\\
\multicolumn{2}{c}{/* $x < y$ */}    \\
\multicolumn{2}{l}{    if  ($x<-2$)}   \\
              \{ &     $y = x^{2}$ ; \\
	             &      $x = -x $ ;   \\
             \} & \\
	           else &   $y = x+ y + 3 $ ; \\
\multicolumn{2}{c}{/* $y > x + 1 $ */} \\
\end{tabular} } & \begin{minipage}{11 cm}
il suffit, d'apr\`es les propri\'et\'es de l'alternative, de montrer s\'epar\'ement deux
choses :
{\tt
\begin{enumerate}
\item /* $x<y$ et $x<-2$ */ \hfil $y= x^{2}$ ; $ x= -x$ ; \hfil /* $y>x+1$ */
\item /* $x<y$ et $x\geq -2$ */ \hfil $y = x+y+3$ ; \hfil /* $y>x+1$ */
\end{enumerate}
}
\end{minipage} \\
& \\
\end{tabular} 

Pour montrer la premi\`ere partie, il faut appliquer les r\`egles de l'affectation
et de l'encha\^\i nement. Raisonnant de {\em droite \`a gauche}, nous sommes ramen\'es
\`a montrer (application des r\`egles de l'affectation \`a $x = -x$ et $P =$
{\tt  /* $y > x+1$ */)} que :
 
 {\tt /* $x<y$ et $x<-2$ */~~~~$y = x^{2}$ ; ~~~~/* $y>-x+1$ */ }

 ce qui est vrai\footnote{ Pour \'eviter le calcul du discriminant
 voici une d\'emonstration simple :
$  x < -2  \mbox{    }\rightarrow \mbox{    }  0 > x + 2  $\\ 
 $  x^{2} > -2x $ 
 par addition :
 $  x^{2}  > -2x +x +2 $ 
 et donc $ x^{2} > -x + 2 > -x + 1 $} puisque $x < -2 \Rightarrow x^{2}>-x+1$

 Pour montrer la deuxi\`eme partie, il suffit de montrer que (substitution de
 $x+y+3$ pour $y$ dans {\tt /*~$y>x+1$~*/)}~:
 
 {\tt /* $x<y$ et $x \geq -2$ */  $\Rightarrow $ /* $ x+y+3 > x+1$ */ }
 
 ce qui est vrai puisque :
 
 {\tt /* $x<y$ et $ x \geq -2$ */ $\Rightarrow $ /* $y>-2$ */ }
