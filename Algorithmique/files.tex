\addcontentsline{toc}{chapter}{\thechapter ~ Les files}
\markboth{Files}{Files}
\centerline{\Large\bf Les files}
\index{file}
 
\noindent\hrulefill  
\begin{multicols}{2}
\addcontentsline{lof}{section}{Implantation d'une file dans un tableau.}
\addcontentsline{lof}{section}{Fonctionnement d'une file circulaire dans un tableau.}
\label{File}
  
  Peut-\^etre plus encore que les piles, les files d'attente (ou simplement
  files) font partie de notre vie courante~; du moins en sommes-nous plus
  conscients, puisqu'il nous arrive quotidiennement de faire la queue
  devant un guichet.
  
  Une file est un TDA form\'e d'un nombre variable, \'eventuellement nul, de
  donn\'ees, sur lequel on peut effectuer les op\'erations suivantes~:
  \begin{itemize}
      
      \item ajout d'une nouvelle donn\'ee ; 
      
      \item test d\'eterminant si la file est vide ou non ; 
      
      \item consultation de la premi\`ere donn\'ee ajout\'ee et non supprim\'ee
            depuis (donc la plus ancienne) s'il y en a une ; 
      
      \item suppression de la donn\'ee la plus ancienne.
      
  \end{itemize}
  
  Cette conception s'accorde bien avec la conception intuitive que l'on a
  d'une file d'attente.
  
  Les files d'attente ont une grande importance en informatique~; elles
  s'appliquent \`a deux types de probl\`emes~:
  
  \begin{itemize}
      
      \item la simulation de files r\'eelles ; les techniques modernes de
            communication (terminaux reli\'es \`a un ou plusieurs centres de
            communication).  Il est devenu courant d'\'ecrire des programmes qui
            \'etudient les comportements de r\'eseaux.
      
      \item la r\'esolution de probl\`emes purement informatiques, en
	    particulier dans le domaine des syst\`emes d'exploitation.

  \end{itemize}

\setcounter{section}{0}
\stepcounter{section}
\section*{Analyse fonctionnelle~:}

  La file est constitu\'ee
  \begin{itemize}
      \item d'une suite d'\'el\'ements ordonn\'es $a_{1}, a_{2}, \ldots,
             a_{n}$, d\'esign\'ee ici par $F$, \'eventuellement vide.

      \item des primitives suivantes~:
      \begin{description}

	  \item{\tt \bf creer(F)} Cette fonction cr\'ee une file de nom F
	     et retourne la valeur {\tt vrai} si l'op\'eration a pu
	     s'effectuer sans probl\`eme, sinon elle retourne {\tt faux}.

	  \item{\tt \bf filevide(F)} Cette fonction teste la vacuit\'e de
	     la file F et retourne {\tt vrai} si la file est vide {\tt
	     faux} sinon.

	  \item{\tt \bf enfiler(x, F)} Cette fonction ajoute un \'el\'ement
	     en queue de file, renvoie {\tt faux} s'il l'\'el\'ement n'a pas
	     pu \^etre introduit, {\tt vrai} sinon.

	  \item{\tt \bf defiler(F)} Cette primitive retire l'\'el\'ement de
	     t\^ete de la file.

	  \item{\tt \bf premier(F)} Cette fonction renvoie la valeur de
	     l'\'el\'ement en t\^ete de file, si la pile n'est pas vide, sinon
	     retourne un code d'erreur.
	\end{description}
    \end{itemize}

  \subsection{Description logique et fonctionnement}

  Impl�ment�e dans un tableau, la file ne peut avoir qu'une taille
  maximale \'egale \`a la dimension du
  tableau, et peut \^etre sch\'ematis\'ee selon la figure \ref{file_tab}


  Notez que le fait de choisir les indices {\tt TETE} et {\tt QUEUE}
  comme nous l'avons fait, avec {\tt TETE} d\'esignant la position pass\'ee
  de la t\^ete de la file, n'influe pas sur le probl\`eme.  Cette convention
  facilite uniquement l'\'ecriture des fonctions {\tt vide} et {\tt
  enfiler}.

      \begin{figurette}
      \setlength{\unitlength}{0.00083333in}
%
\begingroup\makeatletter\ifx\SetFigFont\undefined%
\gdef\SetFigFont#1#2#3#4#5{%
  \reset@font\fontsize{#1}{#2pt}%
  \fontfamily{#3}\fontseries{#4}\fontshape{#5}%
  \selectfont}%
\fi\endgroup%
{\renewcommand{\dashlinestretch}{30}
\begin{picture}(3161,1757)(0,-10)
\put(2347,1616){\makebox(0,0)[lb]{\smash{{{\SetFigFont{8}{9.6}{\rmdefault}{\mddefault}{\itdefault}n}}}}}
\drawline(2634,1329)(3149,1329)(3149,1158)
	(2634,1158)(2634,1329)
\drawline(1260,1730)(1260,12)(2290,12)
	(2290,1730)(1260,1730)
\drawline(1260,1558)(2290,1558)
\drawline(1260,1387)(2290,1387)
\drawline(1260,1043)(2290,1043)
\drawline(1260,871)(2290,871)
\drawline(1260,700)(2290,700)
\drawline(1260,528)(2290,528)
\drawline(1260,184)(2290,184)
\drawline(2634,814)(3149,814)(3149,642)
	(2634,642)(2634,814)
\drawline(2634,1215)(2519,1215)(2519,1100)(2290,1100)
\drawline(2393.060,1128.630)(2290.000,1100.000)(2393.060,1071.370)
\drawline(2634,700)(2519,700)(2519,413)(2290,413)
\drawline(2393.060,441.630)(2290.000,413.000)(2393.060,384.370)
\drawline(2405,528)(2405,757)
\drawline(2433.630,653.940)(2405.000,757.000)(2376.370,653.940)
\drawline(2405,1215)(2405,1444)
\drawline(2433.630,1340.940)(2405.000,1444.000)(2376.370,1340.940)
\drawline(1145,1215)(1031,1215)(1031,356)(1145,356)
\put(0,757){\makebox(0,0)[lb]{\smash{{{\SetFigFont{8}{9.6}{\rmdefault}{\mddefault}{\itdefault}partie utile de la file }}}}}
\put(2691,1215){\makebox(0,0)[lb]{\smash{{{\SetFigFont{8}{9.6}{\rmdefault}{\mddefault}{\itdefault}QUEUE}}}}}
\put(1374,1100){\makebox(0,0)[lb]{\smash{{{\SetFigFont{8}{9.6}{\rmdefault}{\mddefault}{\itdefault}FILE(QUEUE)}}}}}
\put(1374,413){\makebox(0,0)[lb]{\smash{{{\SetFigFont{8}{9.6}{\rmdefault}{\mddefault}{\itdefault}FILE(TETE+1)}}}}}
\put(2691,700){\makebox(0,0)[lb]{\smash{{{\SetFigFont{8}{9.6}{\rmdefault}{\mddefault}{\itdefault}TETE}}}}}
\put(2347,70){\makebox(0,0)[lb]{\smash{{{\SetFigFont{8}{9.6}{\rmdefault}{\mddefault}{\itdefault}1}}}}}
\thicklines
\drawline(1260,1215)(2290,1215)(2290,356)
	(1260,356)(1260,1215)
\end{picture}
}

\medskip
\centerline{{\sc Fig.} {\thesection}.1 -- {\it Une file dans un tableau}}
\addcontentsline{lof}{section}{Concr�tisation d'une file dans un tableau}
      \label{file_tab}
      \end{figurette}

  Un probl\`eme se pose~:  m\^eme si la taille de la file reste constamment
  en dessous du maximum permis $n$, la file ``monte'' inexorablement,
  puisque les d\'efilages s'effectuent par le bas et les enfilages par le
  haut.  Si l'on n'y prend garde, la file d\'ebordera du tableau au bout
  de $n$ op\'erations {\tt enfiler}.

      \begin{figurette}
      \setlength{\unitlength}{0.00083333in}
%
\begingroup\makeatletter\ifx\SetFigFont\undefined%
\gdef\SetFigFont#1#2#3#4#5{%
  \reset@font\fontsize{#1}{#2pt}%
  \fontfamily{#3}\fontseries{#4}\fontshape{#5}%
  \selectfont}%
\fi\endgroup%
{\renewcommand{\dashlinestretch}{30}
\begin{picture}(3184,1758)(0,-10)
\put(2549,1211){\makebox(0,0)[lb]{\smash{{{\SetFigFont{8}{9.6}{\rmdefault}{\mddefault}{\itdefault}QUEUE}}}}}
\thicklines
\drawline(1133,1721)(2152,1721)(2152,1381)
	(1133,1381)(1133,1721)
\thinlines
\drawline(509,985)(679,985)
\drawline(1019,532)(906,532)(906,22)(1019,22)
\drawline(1019,1721)(906,1721)(906,1381)(1019,1381)
\drawline(906,1551)(679,1551)(679,249)(906,249)
\drawline(2266,475)(2266,702)
\drawline(2288.655,611.370)(2266.000,702.000)(2243.345,611.370)
\drawline(2492,758)(2379,758)(2379,1325)(2152,1325)
\drawline(2242.630,1347.655)(2152.000,1325.000)(2242.630,1302.345)
\drawline(3002,1211)(3172,1211)(3172,418)(2152,418)
\drawline(2242.630,440.655)(2152.000,418.000)(2242.630,395.345)
\drawline(1133,192)(2152,192)
\drawline(1133,362)(2152,362)
\drawline(1133,702)(2152,702)
\drawline(1133,871)(2152,871)
\drawline(1133,1041)(2152,1041)
\drawline(1133,1211)(2152,1211)
\drawline(1133,1551)(2152,1551)
\drawline(1133,1721)(2152,1721)(2152,22)
	(1133,22)(1133,1721)
\drawline(2379,1438)(2379,1664)
\drawline(2401.655,1573.370)(2379.000,1664.000)(2356.345,1573.370)
\drawline(2492,815)(3002,815)(3002,645)
	(2492,645)(2492,815)
\drawline(2492,1325)(3002,1325)(3002,1155)
	(2492,1155)(2492,1325)
\put(0,871){\makebox(0,0)[lb]{\smash{{{\SetFigFont{8}{9.6}{\rmdefault}{\mddefault}{\updefault}de  la file}}}}}
\put(0,1041){\makebox(0,0)[lb]{\smash{{{\SetFigFont{8}{9.6}{\rmdefault}{\mddefault}{\updefault}partie utile}}}}}
\put(2209,1608){\makebox(0,0)[lb]{\smash{{{\SetFigFont{8}{9.6}{\rmdefault}{\mddefault}{\itdefault}n}}}}}
\put(2209,79){\makebox(0,0)[lb]{\smash{{{\SetFigFont{8}{9.6}{\rmdefault}{\mddefault}{\itdefault}1}}}}}
\put(2606,702){\makebox(0,0)[lb]{\smash{{{\SetFigFont{8}{9.6}{\rmdefault}{\mddefault}{\itdefault}TETE}}}}}
\put(1189,1438){\makebox(0,0)[lb]{\smash{{{\SetFigFont{8}{9.6}{\rmdefault}{\mddefault}{\itdefault}FILE(TETE+1)}}}}}
\put(1246,418){\makebox(0,0)[lb]{\smash{{{\SetFigFont{8}{9.6}{\rmdefault}{\mddefault}{\itdefault}FILE(QUEUE)}}}}}
\thicklines
\drawline(1133,532)(2152,532)(2152,22)
	(1133,22)(1133,532)
\end{picture}
}
 
\medskip
\centerline{{\sc Fig.} {\thesection}.2 -- {\it Une file circulaire dans un tableau}}
\addcontentsline{lof}{section}{Fonctionnement d'une file circulaire dans un tableau}
      \label{file_tab2}
      \end{figurette}

  Plusieurs solutions sont possibles~: 
  \begin{enumerate}
    
    \item \`a chaque d\'efilage, r\'ecup\'erer l'espace lib\'er\'e en bas du tableau
      en ``redescendant'' toute la file d'un cran.  C'est la solution simple
      \`a mettre en oeuvre, mais co\^uteuse sur les grandes files.
    
    \item laisser la file monter tant qu'il reste de la place pour
      effectuer les enfilages~; quand il n'y a plus de place et que l'on
      veut op\'erer un enfilage, on r\'ecup\`ere d'un seul coup la place lib\'er\'ee
      par les d\'efilages en ``redescendant'' la file de toute la hauteur
      possible.  Cette solution est plus \'economique que la pr\'ec\'edente, mais
      exige encore des transferts d'informations inutiles.
    
    \item quand la queue atteint le haut du tableau, effectuer les
      enfilages suivants \`a partir du bas du tableau, qui prend l'aspect de
      la figure \ref{file_tab2}
    \end{enumerate}

  L'int\'er\^et de cette m\'ethode est qu'elle ne n\'ecessite aucun d\'ecalage.
  On parle d'une repr\'esentation par file circulaire, on peut repr\'esenter
  chacune des deux figures pr\'ec\'edentes par un anneau.

  La programmation de cette solution pr\'esente un pi\`ege~:  le test
  d\'eterminant si la file est vide s'\'ecrit maintenant {\tt TETE = QUEUE}
  si la file \`a $n$ \'el\'ements.  Pour pouvoir distinguer entre ces deux cas
  (file vide et file pleine), on imposera \`a la file la capacit\'e maximale
  $n-1$ (et non $n$).  Dans ces conditions {\tt |TETE-QUEUE|$\geq$1} si
  la file n'est pas vide. 

\vspace{.5cm}

\stepcounter{section}
\section*{Une file g\'en\'erique}

Il est ais\'e de former le type de donn\'ees abstrait d'une file au moyen du code
du type de la liste. Seul le fichier {\tt File.tda} doit \^etre \'ecrit sur le
mod\`ele de {\tt Liste.tda} ou {\tt Pile.tda}.

\vspace{3cm}
{\footnotesize \listing{Prog/Objet/File.tda}}
\end{multicols}
