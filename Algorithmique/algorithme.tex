\centerline{\Large\bf Algorithme} 
\addcontentsline{toc}{chapter}{\thechapter ~ Algorithme}
\addtocounter{section}{1}
%\addcontentsline{toc}{section}{\thesection ~~ D\'efinitions}
\addcontentsline{toc}{section}{D\'efinitions}
\markboth {Algorithme} {Algorithme}
\label{Algo}
\index{algorithme}
\index{Baruk@Baruk, Stella}

\noindent\hrulefill

%\vspace{.5cm}

{\Large \bf Algorithme\footnote{{\it Dictionnaire de math\'ematiques
\'el\'ementaires} Stella Baruk, {\sc Seuil}, p. 77}} \label{Algorithme}

{\bf n. m., XIII$^{e}$s., voir $\S$ III.}


\renewcommand{\theenumi}{\Roman{enumi}}
\begin{enumerate} 
\item \begin{enumerate} \item {\bf Un algorithme est une succession de 
       man{\oe}uvres \`a accomplir toujours dans le m\^eme ordre et de la m\^eme 
	   fa\c con, man{\oe}uvres qui sont en nombre fini et s'appliquent \`a un
	   nombre fini de donn\'ees.}

	   Par exemple, on conna\^\i t depuis l'enfance des algorithmes de calcul, 
	   ceux qui permettent de trouver ce que vaut la somme, le produit, la
	   diff\'erence ou le quotient de deux nombres quand ils s'\'ecrivent avec
	   plus d'un chiffre. Il ne s'agit pas d'autre chose quand on entend dire
	   << je ne sais plus faire une division >>, par exemple celle de 793 par
	   32 ; en fait, c'est l'algorithme de calcul du quotient qu'on n'a plus en
	   m\'emoire, c'est-\`a-dire le << j'ai deux chiffres au diviseur, j'en prends
	   deux au dividende, je dis en 79 combien de fois 32 ou en 7 combien de
	   fois 3, il y va deux fois, etc >>

	   \item Les livres donnent volontiers comme exemples d'algorithmes ceux
	   qui consistent \`a trouver le nouveau  prix d'un objet s'il y a au moment
	   de l'achat une baisse ou une hausse de, mettons, 10~\% ; si le prix
	   initial est $x$, la hausse ou la baisse est les $10/100$ de $x$, soit
	   $0,1x$ ; le nouveau prix est donc :

	   \begin{itemize} 
  
	   \item en cas de baisse : $x - 0,1x = 0,9x$,

	   \item en cas de hausse : $x + 0,1x = 1,1x$.

	   \end{itemize} 

	   \item Un algorithme est g\'en\'eralement r\'ep\'etitif ; c'est le cas si on
	   place une somme d'argent \`a la banque \`a un taux de 4~\% ; au bout d'une
	   ann\'ee, elle devient les 104/100 de ce qu'elle \'etait, au bout de deux
	   ann\'ees les 104/100 des 104/100 de ce qu'elle \'etait, etc.

	   En d\'esignant cette somme par S, on a donc, au bout de $n$ ann\'ees, une
	   somme :

	   %\centerline{$S_{n} = \underbrace{(104/100)(104/100) \ldots
	   %(104/100)}_{{\it n} {\rm fois}}S = 1,04^{n}S$}
	   \[S_{n} = \underbrace{(104/100)(104/100) \ldots
	   (104/100)}_{{\it n} {\rm fois}}S = 1,04^{n}S\]

      \end{enumerate} 

\item On d\'esigne par algorithme un proc\'ed\'e automatique que l'on peut confier \`a
      un ordinateur et qu'il r\'ep\'etera autant de fois qu'il le faudra pour
	  arriver au r\'esultat. Imaginons qu'on l'ait programm\'e pour la suite
	  d'op\'erations suivantes : \`a partir d'un entier naturel $n$ quelconque, si
	  $n$ est pair le diviser par 2 ; s'il est impair, prendre son triple et
	  ajouter 1, c'est-\`a-dire fabriquer $3n+1$ ; dans chaque cas, recommencer
	  avec le nouveau nombre obtenu. Essayons avec quelques nombres :

	  \begin{tabular}{rrrrrrrrrrrrr}
	  \multicolumn{13}{l}{\footnotesize \bf Nombre}\\
	  \multicolumn{13}{l}{\footnotesize \bf de d\'epart}\\
	  16 & 8 & 4 & 2 & 1 \\
	  17 & 52 & 26 & 13 & 40 & 20 & 10 & 5 & 16 & 8 & 4 & 2 & 1 \\
	  18 & 9 & 28 & 14 & 7 & 22 & 11 & 34 & 17 & \multicolumn{4}{l}{\ldots} \\
	  19 & 58 & 29 & 88 & 44 & 22 & 11 & 34 & 17 & \multicolumn{4}{l}{\ldots} \\
	  \end{tabular} 

	  Il est clair qu'en arrivant \`a 1, le processus va 'se boucler' sur
	  lui-m\^eme, puisqu'on aura la suite 1, 4, 2, 1, 4, 2, 1, etc. On voit que
	  c'est ce qui se produit pour 16 et 17 ; or, pour 18 et 19, on retombe
	  sur 17, donc on arrivera aussi \`a 1.

	  La {\bf conjecture} qui s'\'etablit \`a partir de plusieurs tentatives
	  qui, toutes, am\`enent \`a 1 est que, quel que soit le nombre de d\'epart et
	  le nombre d'\'etapes, cet algorithme produira toujours 1 ; mais elle
	  n'est, malgr\'e son apparente simplicit\'e, toujours pas d\'emontr\'ee
	  aujourd'hui. Ce 'beau' probl\`eme s'appelle "le probl\`eme de Collatz", du
	  nom du professeur de Hambourg qui l'a lanc\'e [G$_{17}$].
\index{Collatz}

\item {\it Algorithme} s'est d'abord dit {\it algorisme}, du bas latin {\it
	algorismus}, d\'eformation d'apr\`es le mot grec {\it arithmos}, "nombre", du
	nom propre Al-Khwarizmi.

\end{enumerate} 
\renewcommand{\theenumi}{\arabic{enumi}}
\newpage

%----------------------------------------------------------------------
\setcounter{footnote}{0}
\addtocounter{section}{1}
%\addcontentsline{toc}{section}{\thesection ~~ Sept \'epatant}
\addcontentsline{toc}{section}{Sept \'epatant}

{
{\Large \bf Sept \'epatant}\footnote{A. Thuizat, in {\sc Le petit Archim\`ede} N$^{o}$ 3, Association
pour le d\'eveloppement de la culture scientifique.}

\noindent\hrulefill

Le v\'eritable probl\`eme fut pos\'e quand le p\`ere Mathieu revint de la foire,
poussant devant lui les vingt-huit moutons acquis le matin m\^eme. Jusqu'alors,
les op\'erations s'\'etaient d\'eroul\'ees sans aucune difficult\'e.  Mais il fallait
maintenant r\'epartir ces vingt-huit b\^etes dans les sept bergeries que
comportait la ferme, et \c ca, croyez-en le p\`ere Mathieu, ce n'\'etait pas une
mince affaire.

-- Toine, dit-il \`a son fils a\^\i n\'e,
tu vas me prendre ces vingt-huit b\^etes et me les
installer dans nos sept bergeries. T'en mettras le m\^eme nombre dans chacune.

-- Et \c ca en fait combien donc dans chaque ? Questionna le Toine.

-- D\'ecid\'ement, Toine, t'es pas bien fut\'e. Apprends que, pour faire un partage,
on pose une division. Tiens prends une feuille de papier, je vas te montrer.

Et le p\`ere Mathieu expliqua au Toine les subtilit\'es de l'op\'eration :

\begin{floatingtable}[r]{{\begin{tabular}{r|l}
{\large  28} & {\large 7} \\
\cline{2-2}
{\large 21} & {\large 13} \\
{\large 0}  & \\
\end{tabular}}}\end{floatingtable}

      -- Vingt-huit divis\'e par sept : en 8 combien de fois 7 ? Il
	  y va une fois. Une fois sept fait 7 ; \^ot\'e de 8 il reste 1.
	  J'abaisse le 2. En 21 combien de fois 7 ? Il y va 3 fois. 3
	  fois 7 font 21 ; \^ot\'e de 21, il reste 0. Tu mettras donc 13
	  moutons dans chaque bergerie.

-- Bien, p\`ere, fit le Toine, convaincu par la science.

Il partit incontinent, pour proc\'eder \`a la r\'epartition.
Une heure plus tard,
Mathieu le vit revenir tout piteux :

-- J'y arrive pas, p\`ere. Il doit y avoir une erreur.

-- \'Ecoute-moi bien, lui dit son p\`ere. Y a pas d'erreur possible. D'ailleurs
pour te le prouver, on va proc\'eder autrement. Je t'ai dit 13 moutons dans
chaque bergerie. Si on multiplie 13 par 7, on doit retrouver les 28 t\^etes.
Allons-y :

\begin{floatingtable}[r]{
\begin{tabular}{rr}
&{\large 13} \\
$\times$& {\large  7}\\
\hline
 & {\large 21}\\
 & {\large 7}\\
\hline
 & {\large 28}\\
 & \\
\multicolumn{2}{c}{\setlength{\unitlength}{0.00087500in}%
%
\begingroup\makeatletter\ifx\SetFigFont\undefined%
\gdef\SetFigFont#1#2#3#4#5{%
  \reset@font\fontsize{#1}{#2pt}%
  \fontfamily{#3}\fontseries{#4}\fontshape{#5}%
  \selectfont}%
\fi\endgroup%
\begin{picture}(545,600)(2419,-1072)
\thicklines
\put(2431,-511){\line( 6,-5){513.443}}
\put(2926,-961){\line( 0, 1){ 45}}
\put(2476,-961){\line( 1, 1){450}}
\put(2476,-826){\makebox(0,0)[b]{\smash{\SetFigFont{12}{14.4}{\familydefault}{\mddefault}{\updefault}1}}}
\put(2701,-1051){\makebox(0,0)[b]{\smash{\SetFigFont{12}{14.4}{\familydefault}{\mddefault}{\updefault}7}}}
\put(2701,-556){\makebox(0,0)[b]{\smash{\SetFigFont{12}{14.4}{\familydefault}{\mddefault}{\updefault}4}}}
\put(2926,-826){\makebox(0,0)[b]{\smash{\SetFigFont{12}{14.4}{\familydefault}{\mddefault}{\updefault}1}}}
\end{picture}
} \\
\end{tabular} }\end{floatingtable}
     Treize multipli\'e par sept : 7 fois 3 font 21 ; et 7 fois 1
     fait 7. Tu vois que 21 et 7, \c ca fait bien 28.

 D'ailleurs, pour \^etre plus s\^ur, on va faire la preuve par neuf :

 3 et 1 font 4. Je pose 4 en haut et j'\'ecris 7 en dessous.
 7 fois 4 font 28. 8 et 2 font 10. J'\'ecris 1 \`a gauche.
 Maintenant le r\'esultat : 8 et 2 font 10. J'\'ecris 1 \`a droite.
 Tu vois bien que c'est juste. Allez, va-t-en me mettre treize
 b\^etes dans chaque bergerie.

 (Ici, normalement, Mathieu aurait d\^u s'inqui\'eter, puisque 7
 fois 13, comme 7 fois 4 font \'egalement 28. Mais s'il fallait
 encore s'attacher \`a tant de menus d\'etails, on n'avancerait
 jamais. On continua donc).

C'est un Toine effondr\'e qui revint une heure plus tard. 

-- J'y arrive toujours pas. Y a s\^urement quelque chose qui ne va pas dans les
comptes.

-- Y a surtout qu't'es pas bien malin, fils, dit le p\`ere Mathieu. La division,
la multiplication, c'est trop fort pour toi. L'addition, \c ca doit aller mieux.

\begin{floatingtable}[r]{
\begin{tabular}{c}
13\\
13\\
13\\
13\\
13\\
13\\
13\\
\cline{1-1}
28\\
\end{tabular} }\end{floatingtable}
                J'\'ecris 13, sept fois de suite, et j'additionne : 3 et 3, 6 ;
et 3, 9 ; et 3, 12 ; et 3, 15 ; et 3, 18 ; et 3, 21 ; et 1, 22 ;
et 1,  23 ; 24 ; 25 ; 26 ; 27 ; 28.\\
			    Es-tu convaincu, cette fois ? Allez, va.\\
	    Et le Toine repartit encore une fois, loger les maudites b\^etes.\\
		    Et en fin de soir\'ee, il revint triomphant.\\
		--  \c Ca y est, p\`ere, tous les moutons sont rentr\'es !\\
			    -- Comment que t'as fait ?

-- Je les ai fait rentrer un par un en faisant le tour des bergeries.
Et pour \^etre tout \`a fait s\^ur, quand ils ont \'et\'e plac\'es, moi aussi,
j'ai fait mes comptes : j'ai compt\'e les pattes ; j'ai trouv\'e 16 pattes dans
chaque bergerie.

\begin{floatingtable}[r]{
\begin{tabular}{r|l}
{\large 16} & {\large 4} \\
\cline {2-2}
{\large 12} & {\large 13}\\
{\large 0}  & \\
\end{tabular} }\end{floatingtable}

-- Attends voir, dit le P\`ere Mathieu. Faut pas s'emballer. \'Etant donn\'e qu'un
mouton a 4 pattes, si je divise 16 par 4, je saurai combien tu as mis de b\^etes
dans chacune.

	Et la nouvelle division fut pos\'ee : seize divis\'e par quatre :
	en 6 combien de fois 4 ? Il y va une fois. Une fois 4 fait 4 ;
	\^ot\'e de 6, il reste 2. J'abaisse mon 1. En 12 combien de fois 4 ?
	Il y va 3 fois. 3 fois 4 font 12 ; \^ot\'e de 12, il reste 0.
} 
