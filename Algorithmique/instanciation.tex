\stepcounter{section}
\addcontentsline{toc}{section}{Instanciation}
\markboth{Instanciation}{Instanciation}
\centerline{\Large\bf L'instanciation}
 \index{instanciation}
 
 \noindent\hrulefill  
\begin{multicols}{2}

La programmation simpliste d'un type de donn\'ees abstrait, consiste \`a 
r\'e\'ecrire chaque fois que n\'ecessaire l'ensemble des fonctions applicables
aux donn\'ees manipul\'ees.  Ainsi on \'ecrit une pile
de nombres pour une calculatrice et une pile de blocs libres pour la gestion
d'un disque et une pile pour la saisie des caract\`eres. De m\^eme on \'ecrit un
programme de gestion d'un arbre pour les r\'epertoires et un autre pour l'analyse
syntaxique d'une expression. 


Une autre approche r\'ealise une fois pour toutes un module qui ne repr\'esente
aucun objet concret, mais seulement un {\it mod\`ele} de la structure de
donn\'ees, c'est \`a dire son fonctionnement. L'ensemble des fonctions est pr\'evu
pour s'appliquer aussi bien \`a des entiers, \`a des flottants, des tableaux,
des structures ou autre.


Le programmeur-utilisateur de ce module doit alors explicitement cr\'eer un {\it
objet} \`a l'aide d'une op\'eration appel\'ee {\it constructeur} d\'efinie dans le
module. On dit que l'on cr\'ee une {\it instance} du type abstrait. 
La fonction {\it constructeur} associe les donn\'ees du programme effectivement
trait\'ees aux fonctions pr\'ealablement d\'efinies. 

Dans l'exemple ci-dessous\footnote{Le programme complet est en annexe},
on commence par indiquer que
la liste s'applique \`a des entiers ({\tt liste(entier)}), puis dans le {\tt
main}, on cr\'e\'e une variable {\tt l} du type liste d'entier ({\tt entierliste l})
et enfin on instancie cette liste d'entiers au moyen du constructeur ({\tt l =
entierliste\_creer (entier\_copier, 0, entier\_editer) ;}).

Apr\`es cette phase d'initialisation, on peut utiliser toutes les fonctions 
du type de donn\'ees choisi en r\'epondant \`a la question {\it que faire ?}
et non plus {\it comment faire ? }.

\vspace{1cm}

{\small
\listing {Prog/Objet/usager1.c}}

\addcontentsline{lof}{section}{Instanciation d'une liste d'entiers}
\newpage
~

\end{multicols}
