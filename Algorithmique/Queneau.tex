\addtocounter{section}{1}
\addcontentsline{toc}{section}{Un conte \`a votre fa\c con}
\addcontentsline{lof}{section}{Texte de R. Queneau, {\em Un conte \`a votre fa\c con}}
\index{Queneau@Queneau, Raymond}
\index{Oulipo}
\index{conte}
Raymond Queneau a propos\'e, sous le titre {\it Un conte \`a votre fa\c con}, le
texte suivant %\footnote{Cit\'e par B. Meyer et C. Baudoin {\it M\'ethodes de
%programmation} Coll. de la direction des \'Etudes et Recherches
%d'\'Electricit\'e de France, Eyrolles 1986}
(extrait de : {\it L'Oulipo}, Coll. Id\'ees, Gallimard 1972).

{\footnotesize
\begin{multicols}{2}%[Queneau]{7cm}
{\sc Un conte \`a votre fa\c con}

{\em Ce texte soumis \`a la 83$^{e}$ r\'eunion de travail de l'Ouvroir de
Litt\'erature Potentielle, s'inspire de la pr\'esentation des instructions
destin\'ees aux ordinateurs ou bien encore de l'enseignement programm\'e. C'est
une structure analogue \`a la litt\'erature ``en arbre'' propos\'ee par F. Lionnais
\`a la 79$^{e}$ r\'eunion.}

\begin{enumerate} 
\item D\'esirez-vous conna\^\i tre l'histoire des trois alertes petits pois ?

si oui, passez \`a 4,\\
si non, passez \`a 2.

\item Pr\'ef\'erez-vous celle des trois minces grands \'echalas ?

si oui, passez \`a 16,\\
si non, passez \`a 3.

\item Pr\'ef\'erez-vous celles des trois moyens m\'ediocres arbustes ?

si oui, passez \`a 17,\\
si non, passez \`a 21.

\item Il y avait une fois trois petits pois v\^etus de vert qui dormaient
gentiment dans leur cosse. Leur visage bien rond respirait par les trous de
leurs narines et l'on entendait leur ronflement doux et harmonieux.

si vous pr\'ef\'erez une autre description, passez en 9\\
si celle-ci vous convient, passez \`a 5.

\item Ils ne r\^evaient pas. Ces petits \^etres en effet ne r\^event jamais.

si vous pr\'ef\'erez qu'ils r\^event, passez \`a 6,\\
sinon, passez \`a 7.

\item Ils r\^evaient. Ces \^etres en effet r\^event toujours et leurs nuits
s\'ecr\`etent des songes charmants.

si vous d\'esirez conna\^\i tre ces songes, passez \`a 11.\\

\item Leurs pieds mignons trempaient dans de chaudes chaussettes et ils
portaient au lit des gants de velours noirs.

si vous pr\'ef\'erez des gants d'une autre couleur, passez en 8,\\
si cette couleur vous convient, passez en 10

\item Ils portaient au lit des gants de velours bleu.

si vous pr\'ef\'erez des gants d'une autre couleur, passez en 7,\\
si cette couleur vous convient, passez en 10.

\item Il y avait une fois trois petits pois qui roulaient leur bosse sur les
grands chemins. Le soir venu, fatigu\'es et las, ils s'endormirent tr\`es
rapidement.

si vous d\'esirez conna\^\i tre la suite, passez \`a 5\\
sinon, passez \`a 21.

\item Tous les trois faisaient le m\^eme r\^eve, ils s'aimaient en effet
tendrement et, en bons fiers trumeaux, songeaient toujours semblablement.

si vous d\'esirez conna\^\i tre leur r\^eve, passez \`a 11,\\
si non, passez \`a 12.

\item Ils r\^evaient qu'ils allaient chercher leur soupe \`a la cantine populaire
et qu'en ouvrant leur gamelle ils d\'ecouvraient que c'\'etait de la soupe d'ers.
D'horreur, il s'\'eveillent.

si vous voulez savoir pourquoi il s'\'eveillent d'horreur, consultez le Larousse
au mot ``ers'' et n'en parlons plus.\\
si vous jugez inutile d'approfondir la question, passez \`a 12.

\item Opopo\"\i ! s'\'ecrient-ils en ouvrant les yeux. Opopo\"\i ! Quel songe
avons-nous enfant\'e l\`a ! Mauvais pr\'esage, dit le premier. Oui-da, dit le
second, c'est bien vrai, me voil\`a triste. Ne vous troublez pas ainsi, dit le
troisi\`eme qui \'etait le plus fut\'e, il ne s'agit pas de s'\'emouvoir, mais de
comprendre, bref, je m'en vais vous analyser \c ca.

si vous d\'esirez conna\^\i tre tout de suite l'interpr\'etation de ce songe, passez \`a
15,\\
si vous souhaitez au contraire conna\^\i tre les r\'eactions des deux autres, passez
\`a 13.

\item Tu nous la bailles belle, dit le premier. Depuis quand sais-tu analyser
les songes ? Oui, depuis quand ? Ajouta le second.

si vous d\'esirez aussi savoir depuis quand, passez \`a 14,\\
si non, passez \`a 14 tout de m\^eme, car vous ne le saurez pas plus.

\item Depuis quand ? s'\'ecria le troisi\`eme. Est-ce que je sais moi ! Le fait
est que je pratique la chose. Vous allez voir !

si vous voulez aussi voir, passez \`a 15,\\
si non, passez \'egalement \`a 15, car vous ne verrez rien.

\item Eh bien ! Voyons, dirent ses fr\`eres. Votre ironie ne me pla\^\i t pas,
r\'epliqua l'autre, et vous ne saurez rien. D'ailleurs, au cours de cette
conversation d'un ton assez vif, votre sentiment d'horreur ne s'est-il pas
estomp\'e ? effac\'e m\^eme ? Alors \`a quoi bon remuer le bourbier de votre
inconscient de papilionac\'ees ? Allons plut\^ot nous laver \`a la fontaine et
saluer ce gai matin dans l'hygi\`ene et la sainte euphorie ! Aussit\^ot dit,
aussit\^ot fait : les voil\`a qui se glissent hors de leur cosse, se laissent
doucement rouler sur le sol et puis au petit trot gagnent joyeusement le
th\'e\^atre de leurs ablutions.

si vous d\'esirez savoir ce qui se passe sur le th\'e\^atre de leurs ablutions,
passez \`a 16,\\
si vous ne le d\'esirez pas, vous passez \`a 21.

\item Trois grands \'echalas les regardaient faire. 

si les trois grands \'echalas vous d\'eplaisent passez \`a 21,\\
s'ils vous conviennent passez \`a 18.

\item Trois moyens m\'ediocres arbustes les regardaient faire.

si les trois moyens m\'ediocres arbustes vous d\'eplaisent passez \`a 21,\\
s'ils vous conviennent passez \`a 18.

\item Se voyant ainsi zyeut\'es, les trois alertes petits pois qui \'etaient
fort pudiques s'ensauv\`erent.

si vous d\'esirez savoir ce qu'ils firent ensuite, passez \`a 19,\\
si vous ne le d\'esirez pas passez \`a 21.

\item Ils coururent bien fort pour regagner leur cosse et, refermant celle-ci
derri\`ere eux, s'y endormirent de nouveau.

si vous d\'esirez conna\^\i tre la suite, passez \`a 20,\\
si vous ne le d\'esirez pas passez \`a 21.

\item Il n'y a pas de suite, le conte est termin\'e.

\item Dans ce cas, le conte est \'egalement termin\'e.

\end{enumerate} 

\begin{flushright}
Raymond Queneau.
\end{flushright}

\end{multicols}
}%fin de footnotesize 
