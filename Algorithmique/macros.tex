\addcontentsline{toc}{section}{Les macros}
\markboth{Les macros}{Les macros}
\centerline{\Large\bf Les macros (cha\^\i nes de remplacement)}
 \index{Cha\^\i ne de remplacement}
 \index{macro}
 
 \noindent\hrulefill  
\begin{multicols}{2}

La premi\`ere r\`egle est : {\it ne les utilisez pas} si vous n'avez pas \`a le
faire\footnote{On lira avec int\'er\^et le fichier {\tt /usr/include/ctype.h}}.
Il a pu \^etre observ\'e qu'\`a peu pr\`es chaque macro d\'emontre un d\'efaut
dans la programmation. Puisqu'elles r\'eorganisent le texte d'un programme avant
que le compilateur ne les voie, les macros sont un probl\`eme majeur pour la
plupart des outils de d\'eveloppement (d\'ebogueurs, profileurs\ldots).

%\stepcounter{subsection}
% \subsection*{\thesubsection \hbox to .45cm {}  Rappel}
\subsection*{Rappel}

Une directive (macro-instruction\footnote{Mammeri M.
{\it programmation} {\sc \'ecole centrale de paris} {\tiny 1997-1998} p. 112})
de pr\'etraitement de la forme :

{\small
\tt \#define } {\it identificateur chaine-symbole}

provoque le remplacement par le pr�compilateur de toutes les instances
suivantes de l'identificateur avec la s\'equence de symboles donn\'ee. Les espaces
entourant la s\'equence de symboles de remplacement sont annul\'es. Par exemple :

{\small
\tt \#define  COTE 8

char jeu[COTE][COTE] ;}

devient apr\`es le passage du pr\'ecompilateur :

{\small
\tt char jeu [8] [8] ;}

%\stepcounter{subsection}
%\subsection*{\thesubsection \hbox to .45cm {}  Op\'erateur \#}
\subsection*{Op\'erateur \#}

Si une occurrence d'un param\`etre dans une s\'equence de symboles de remplacement
est imm\'ediatement pr\'ec\'ed\'ee par un symbole {\tt \#}, le param\`etre et
l'op\'erateur {\tt \#} seront remplac\'es dans le d\'eveloppement par un litt\'eral
cha\^\i ne contenant l'orthographe de l'argument correspondant. Un caract\`ere
$\backslash$ est ins\'er\'e dans un litt\'eral cha\^\i ne avant chaque occurrence d'un 
$\backslash$ ou d'un {\tt "} \`a l'int\'erieur d'une (ou d\'elimitant) une
constante caract\`ere ou un litt\'eral cha\^\i ne dans l'argument.

{\small
\tt \#define path(logid,cmd) "/users/" \#logid "/bin/" \#cmd
\#define jjd dhenin}

\`a l'appel 

{\small
\tt char * outil = path(jjd, listlp) ;}

sera interpr\'et\'e :

{\small
\tt char * outil = "/users/" "jjd" "/bin/" "listlp" ;}

qui sera ensuite concat\'en\'e pour devenir :

{\small
\tt char * outil = "/users/jjd/bin/listlp" ;}

%\stepcounter{subsection}
%\subsection*{\thesubsection \hbox to .45cm {}  Op\'erateur \#\#}
\subsection*{Op\'erateur \#\#}

Si un op\'erateur {\tt \#\#} appara\^\i t entre deux symboles dans une s\'equence de
symboles de remplacement, et si l'un ou l'autre des symboles est un param\`etre,
il est tout d'abord remplac\'e, puis l'op\'erateur {\tt \#\#} et tous les espaces
l'entourant sont ensuite supprim\'es. L'effet de l'op\'erateur {\tt \#\#} est donc
une concat\'enation.

Par exemple dans la cr\'eation du TDA liste ({\tt Liste.h} et {\tt Liste.tda})

{\small
\begin{verbatim}#define nom(a,b) a##b
#define liste(type_objet)            \
void nom (type_objet, liste_premier) \
    (nom(type_objet, liste) l)       \
{                                    \
    (*Liste.premier) (l->rep) ;      \
}

liste(Entier) ; \end{verbatim} }

produira :

{\small
\begin{verbatim}   void Entierliste_premier(Entierliste l)
{
    (*Liste.premier) (l->rep) ; 
}\end{verbatim} }

Mais toute macro utilis\'ee comme un des symboles adjacents  \`a {\tt \#\#} n'est
pas expans\'ee, \`a l'inverse du r\'esultat de la concat\'enation.

{\small
\begin{verbatim}  
#define concat(a)       a##valise
#define mot             B
#define motvalise       rafiscotche

contate(mot) \end{verbatim} }

donne 

{\small
\tt rafiscotche}

et non pas

{\small
\tt Bvalise}

{\small
\begin{verbatim}  
main ()
{
     int a = 1 ;
     int B = 2 ;
     int Bvalise = 3 ;
     int rafiscotche = 4 ;


     printf ("%d\n", concat(mot)) ;
}
\end{verbatim} }

\newpage
~

\end{multicols}
