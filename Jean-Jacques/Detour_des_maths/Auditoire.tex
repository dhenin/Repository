\ifdefined\COMPLETE % test 
\else
    \input{./preambule_detour_de_maths.ltx}
    \begin{document}
\fi

Tomber d'accords.

accéder, accepter, accorder, acquiescer, adhérer, admettre, adopter, applaudir, approuver, assentir, autoriser, avoir pour agréable, avouer, céder, condescendre, consentir, convenir, dire amen, être d'accord, octroyer, opiner, permettre, reconnaître, se laisser faire, se prêter, se soumettre, souscrire, toper, vouloir bien

amphi, amphithéâtre, assemblée, assistance, audience, auditeur, chambrée, foule, galerie, public, réunion, salle, spectateur, spectateurs


Commençons par les définitions, commençons par nous accorder un  langage  commun, accordons nos instruments. 

À qui le dites-vous ? 

À première vue, cette réaction a l'air d'une question, vu l'inversion du sujet et du verbe. Pourtant, malgré sa forme interrogative, elle ne sert nullement à interroger. Elle ne s'utilise d'ailleurs pas avec l'intonation montante si typique pour la question réelle. Son sens est plus proche de la déclaration soutenue, ce qui justifie dans le texte imprimé le point d'exclamation. D'ailleurs, si c'était une simple question, la réponse serait tellement évidente, qu'elle couvrirait de ridicule l'innocent qui la poserait. Question : « À qui le dites-vous ? » Réponse : « Mais à vous, bien sûr, puisque c'est à vous que je m'adresse ! »




\ifdefined\COMPLETE
\else
    \end{document}
\fi