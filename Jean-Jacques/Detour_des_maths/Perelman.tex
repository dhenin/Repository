
\ifdefined\COMPLETE
\else
    \input{preambule-sacha-utf8.ltx}   
    \usepackage{variations}
    \begin{document}
\fi

P 2 : 
La conception nettement exprimée par Descartes dans la première partie du  Discours de la Méthode 
était de tenir « presque pour faux tout ce qui n'était que vraisemblable ». C'est lui qui, faisant de 
l'évidence la marque de la raison, n'a voulu considérer comme rationnelles que les démonstrations 
qui, à partir d'idées claires et distinctes, propageaient, à l'aide des preuves apodictiques, l'évidence 
des axiomes à tous les théorèmes. » 

\bigskip

P  2 :  « Le  raisonnement  more  geometrico  était  le  modèle  que  l'on  proposait  aux  philosophes 
désireux de construire un système de pensée qui puisse atteindre à la dignité d'une science.  Une 
science  rationnelle  ne  peut,  en  effet, se  contenter  d'opinions  plus  ou  moins  vraisemblables,  mais 
élabore  un  système  de  propositions  nécessaires  qui  s'impose  à  tous  les  êtres  raisonnables,  et  sur 
lesquelles l'accord est inévitable. Il en résulte que le désaccord est signe d'erreur. « Toutes les fois 
que deux hommes portent sur la même chose un jugement contraire, il est certain, dit Descartes, 
que l'un des deux se trompe. Il y a plus, aucun d'eux ne possède la vérité; car s'il en avait une vue 
claire  et  nette,  il  pourrait  l'exposer  à  son  adversaire  de  telle  sorte  qu'elle  finirait  par  forcer  sa 
conviction (1). » 

\bigskip

Pour  les  partisans  des  sciences  expérimentales  et  inductives,  ce  qui  compte,  c'est  moins  la 
nécessité  des  propositions,  que  leur  vérité,  leur  conformité  avec  les  faits.  L'empiriste  considère 
comme preuve non pas « la force à laquelle l'esprit cède et se trouve contraint de céder, mais celle à 
laquelle il devrait céder, celle qui, en s'imposant à lui, rendrait sa croyance conforme au fait » (2). 
Si l'évidence qu'il reconnaît n'est pas  celle de l'intuition rationnelle, mais bien celle  de l'intuition 
sensible,  si  la  méthode  qu'il  préconise  n'est  pas  celle  des  sciences  déductives,  mais  des  sciences 
expérimentales,  il  n'en  est  pas  moins  convaincu  que  les  seules  preuves  valables  sont  les  preuves 
reconnues par les sciences naturelles. » 
\bigskip
\bigskip
\bigskip
\bigskip
1 
\bigskip
(1) Descartes, Œuvres, t. XI : Règles pour la direction de l'esprit, pp. 205-206.  
(2) John Stuart Mill, Système de logique déductive et inductive, liv. III, chap. XXI, § 1, vol. II, p. 
94. 
\bigskip
P 3 : « Est rationnel, dans le sens élargi de ce mot, ce qui est conforme aux méthodes scientifiques, 
et  les  ouvrages  de  logique  consacrés  à  l'étude  des  moyens  de  preuve,  limités  essentiellement  à 
l'étude de la  déduction et  d'habitude complétés par des indications sur le raisonnement inductif, 
réduites d'ailleurs aux moyens non pas de construire mais de vérifier les hypothèses, s'aventurent 
bien rarement dans l'examen des moyens de preuve utilisés dans les sciences humaines. En effet, le 
logicien, s'inspirant de l'idéal cartésien, ne se sent à l'aise que dans l'étude des preuves qu'Aristote 
qualifiait d'analytiques, tous les autres moyens ne présentant pas le même caractère de nécessité. 
Et  cette  tendance  s'est  encore  fortement  accentuée  depuis  un  siècle  où,  sous  l'influence  de 
logiciens-mathématiciens,  la  logique  a  été  limitée  à  la  logique  formelle,  c'est-à-dire  à  l'étude  des 
moyens  de  preuve  utilisés  dans  les  sciences  mathématiques.  Il  en  résulte  que  les  raisonnements 
étrangers  au  domaine  purement  formel  échappent  à  la  logique,  et  par  là  aussi  à  la  raison.  Cette 
raison,  dont  Descartes  espérait  qu'elle  permettrait,  du  moins  en  principe,  de  résoudre  tous  les 
problèmes qui se posent aux hommes et dont l'esprit divin possède déjà la solution, a été de plus 
en  plus  limitée  dans  sa  compétence,  de  sorte  que  ce  qui  échappe  à  une  réduction  au  formel  lui 
présente des difficultés insurmontables. 
\bigskip
Faut-il  tirer  de  cette  évolution  de  la  logique,  et  des  progrès  incontestables  qu'elle  a  réalisés,  la 
conclusion que la raison est tout à fait incompétente dans les domaines qui échappent au calcul et 
que là où ni l'expérience ni la déduction logique ne peuvent nous fournir la solution d'un problème, 
nous n'avons plus qu'à nous abandonner aux forces irrationnelles, à nos instincts, à la suggestion 
ou à la violence ? » 
\bigskip
P 3-4 : « En opposant la volonté à l'entendement, l'esprit de finesse à l'esprit de géométrie, le cœur 
à  la  raison,  et  l'art  de  persuader  à  celui  de  convaincre,  Pascal  avait  déjà  cherché  à  obvier  aux 
insuffisances  de  la  méthode  géométrique  résultant  de  ce  que  l'homme,  déchu,  n'est  plus 
uniquement un être de raison. » 
\bigskip
P 4 : « C'est à des fins analogues que correspondent l'opposition kantienne de la foi et de la science 
et  l'antithèse  bergsonienne  de  l'intuition  et  de  la  raison.  Mais  qu'il  s'agisse  de  philosophes 
rationalistes  ou  de  ceux  que  l'on  qualifie  d'antirationalistes,  tous  continuent  la  tradition 
cartésienne par la limitation imposée à l'idée de raison.  
\bigskip
Il  nous  semble,  au  contraire,  que  c'est  là  une  limitation  indue  et  Parfaitement  injustifiée  du 
domaine ou intervient notre faculté de raisonner et de prouver.  En effet, alors que déjà Aristote 
avait  analysé  les  preuves  dialectiques  à  côté  des  preuves  analytiques,  celles  qui  concernent  le 
vraisemblable  à  côté  de  celles  qui  sont  nécessaires,  celles  qui  servent  dans  la  délibération  et 
l'argumentation  à  côté  de  celles  qui  sont  utilisées  dans  la  démonstration,  la  conception  post-
cartésienne de la raison nous oblige de faire intervenir des éléments irrationnels, chaque fois que 
l'objet de la connaissance n'est pas évident. Que ces éléments consistent en obstacles qu'il s'agit de 
surmonter  -  tels  que  l'imagination,  la  passion,  ou  la  suggestion  -  ou  en  des  sources 
suprarationnelles de certitude comme le cœur, la grâce, l'Einfühlung ou l'intuition bergsonienne, 
cette  conception  introduit  une  dichotomie,  une  distinction  des  facultés  humaines  entièrement 
artificielle et contraire aux démarches réelles de notre pensée. 
\bigskip
C'est à  l'idée d'évidence,  comme caractérisant la raison, qu'il faut s'attaquer si l'on vent faire une 
place à une théorie de l'argumentation, qui admette l'usage de la raison pour diriger notre action et 
pour  influer  sur  celle  des  autres.  L'évidence  est  conçue,  à  la  fois,  comme  la  force  à  laquelle  tout 
esprit normal ne peut que céder et comme signe de vérité de ce qui s'impose parce qu'évident (1). 
L'évidence  relierait  le  psychologique  au  logique  et  permettrait  de  passer  de  l'un  de  ces  plans,  à 
\bigskip
\bigskip
\bigskip
2 
\bigskip
l'autre.  Toute  preuve  serait  réduction  à  l'évidence  et  ce  qui  est  évident  n'aurait  nul  besoin  (la 
preuve : c'est l'application immédiate, par Pascal, de la théorie cartésienne de l'évidence (1). » 
\bigskip
(1) Cf. Ch. Perelman, De la preuve en philosophie, dans Rhétorique et philosophie, pp. 123 et suiv. 
(1) Pascal, Bibl. de la Pléiade, De l'art de persuader, Règles pour les démonstrations, p. 380. 
\bigskip
P 5 : « Déjà Leibniz s'insurgeait contre cette limitation que l'on voulait imposer, par là, à la logique. 
Il voulait, en effet, « qu'on démontrât ou donnât le moyen de demonstrer tous les  Axiomes qui lie 
sont  point  primitifs  ;  sans  distinguer  l'opinion  que  les  hommes  en  ont,  et  sans  se  soucier  s'ils  y 
donnent leur consentement ou non » (2). 
\bigskip
Or la théorie logique de la démonstration s'est développée en suivant Leibniz et non pas Pascal, et 
n'a  pas  admis  que  ce  qui  était  évident  n'avait  nul  besoin  de  preuve;  de  même,  la  théorie  de 
l'argumentation ne peut se développer si toute preuve est conçue comme réduction à l'évidence. En 
effet,  l'objet  de  cette  théorie  est  l'étude  des  techniques  discursives  permettant  de  provoquer  ou 
d'accroître l'adhésion des esprits aux thèses qu'on présente à leur assentiment. Ce qui caractérise 
l'adhésion des esprits c'est que son intensité est variable : rien ne nous oblige à limiter notre étude 
à un degré particulier d'adhésion, caractérisé par l'évidence, rien ne nous permet de considérer  a 
Priori comme proportionnels les degrés d'adhésion à une thèse avec sa probabilité, et d'identifier 
évidence  et  vérité.  Il  est  de  bonne  méthode  de  ne  pas  confondre,  au  départ,  les  aspects  du 
raisonnement  relatifs  à  la  vérité  et  ceux  qui  sont  relatifs  à  l'adhésion,  mais  de  les  étudier 
séparément, quitte à se préoccuper ultérieurement de leur interférence ou de leur correspondance 
éventuelles.  C'est  seulement  à  cette  condition  qu'est  possible  le  développement  d'une  théorie  de 
l'argumentation ayant une portée philosophique. » 
\bigskip
(2) Leibniz, éd. Gerhardt, 5e vol., Nouveaux essais sur l'entendement, p. 67. 
\bigskip
\bigskip
II 
\bigskip
P  6 :  « Si  pendant  ces  trois  derniers  siècles  ont  paru  des  ouvrages  d'ecclésiastiques  que 
préoccupaient des problèmes posés par la foi et la prédication (1), si le XXe siècle a même pu être 
qualifié de siècle de la publicité et de la propagande, et si de nombreux travaux ont été consacrés à 
cette  matière  (2),  les  logiciens  et  les  philosophes  modernes  se  sont,  eux,  complètement 
désintéressés  de  notre  sujet.  C'est  la  raison  pour  laquelle  notre  traité  se  rattache  surtout  aux 
préoccupations  de  la  Renaissance  et,  par  delà,  à  celles  des  auteurs grecs  et  latins,  qui  ont  étudié 
l'art  de  persuader  et  de  convaincre,  la  technique  de  la  délibération  et  de  la  discussion.  C'est  la 
raison aussi pour laquelle nous le présentons comme une nouvelle rhétorique. 
\bigskip
Notre  analyse  concerne  les  preuves  qu'Aristote  appelle  dialectiques,  qu'il  examine  dans  ses 
Topiques et dont il montre l'utilisation dans sa Rhétorique. Ce rappel de la terminologie d'Aristote 
aurait  justifié  le  rapprochement  de  la  théorie  de  l'argumentation  avec  la  dialectique,  conçue  par 
Aristote lui-même comme l'art de raisonner à partir d'opinions généralement acceptées (mot grec) 
(3). Mais plusieurs raisons nous ont incités à préférer le rapprochement avec la rhétorique. » 
\bigskip
(1)  Cf.  notamment  Richard  D.  D.  WHATELY,  Elements  of  Rhetoric,  1828  ;  cardinal  NEWMAN, 
Grammar of Assent, 1870. 
(2) Pour la bibliographie, voir H. D. LASSWELL, R. D. CASEY and B. L. SMITH Propaganda  and 
Promotional  Activities,  1935  ;  B.  L.  SMITH,  H.  D.  LASSWELL  and  R.  D  .  CAsEv,  Propaganda, 
Communication and Publie Opinion, 1946. 
(3) ARISTOTE, Topiques, liv. L chap. 1, 100 a. 
\bigskip
P 6-7 : « La première d'entre elles est la confusion que risquerait d'apporter ce retour à Aristote. 
Car si le mot dialectique a servi, pendant des siècles, à désigner la logique elle-même, depuis Hegel 
et  sous  l'influence  de  doctrines  qui  s'en  inspirent,  il  a  acquis  un  sens  fort  éloigné  de  son  sens 
\bigskip
\bigskip
\bigskip
3 
\bigskip
primitif et qui est assez généralement accepté dans la terminologie philosophique contemporaine. 
Il  n'en  est  pas  de  même  du  mot  rhétorique  dont  l'usage  philosophique  est  tellement  tombé  en 
désuétude  que  l'on  n'en  trouve  même  pas  mention  dans  le  vocabulaire  de  la  philosophie  de  A. 
Lalande : nous espérons que notre tentative fera revivre une tradition glorieuse et séculaire. » 
\bigskip
P 7 : « Mais une autre raison bien plus importante à nos yeux a motivé notre choix;  c'est l'esprit 
même  dans  lequel  l'Antiquité  s'est  occupée  de  dialectique  et  de  rhétorique.  Le  raisonnement 
dialectique  est  considéré  comme  parallèle  au  raisonnement  analytique,  mais  traite  du 
vraisemblable  au  lieu  de  traiter  de  propositions  nécessaires.  L'idée  même  que  la  dialectique 
concerne  des  opinions,  c'est-à-dire  des  thèses  auxquelles  on  adhère  avec  une  intensité  variable, 
n'est pas mise à profit. On dirait que le statut de l'opinable est impersonnel et que les opinions ne 
sont pas relatives aux esprits qui y adhèrent. Par contre, cette idée d'adhésion et d'esprits auxquels 
on  adresse  un  discours  est  essentielle  dans  toutes  les  théories  anciennes  de  la  rhétorique.  Notre 
rapprochement avec cette dernière vise à souligner le fait que  c'est en fonction d'un auditoire que 
se  développe  toute  argumentation  ;  l'étude  de  l'opinable  des  Topiques  pourra,  dans  ce  cadre, 
s'insérer à sa place. 
\bigskip
Il va de soi, pourtant, que notre traité d'argumentation débordera par certains côtés, et largement, 
les  bornes  de  la  rhétorique  des  Anciens,  tout  en  négligeant  certains  aspects  qui  avaient  attiré 
l'attention des maîtres de rhétorique. 
\bigskip
L'objet de la rhétorique des Anciens était, avant tout, l'art de parler en publie de façon persuasive : 
elle  concernait  donc  l'usage  du  langage  parlé,  du  discours,  devant  une  foule  réunie  sur  la  place 
publique, dans le but d'obtenir l'adhésion de celle-ci à une thèse qu'on lui présentait. On voit, par 
là, que le but de l'art oratoire, l'adhésion des esprits, est le même que celui de toute argumentation. 
Mais nous n'avons pas de raisons de limiter notre étude à la présentation d'une argumentation par 
la parole et de limiter à une foule réunie sur une place le genre d'auditoire auquel on s'adresse. » 
\bigskip
P  8 :  « Le  rejet  de  la  première  limitation  résulte  du  fait  que  nos  préoccupations  sont  bien  plus 
celles  d'un  logicien  désireux  de  comprendre  le  mécanisme  de  la  pensée  que  d’un  maître 
d'éloquence  ce  soucieux  de  former  des  praticiens  ;  il  nous suffit  de  citer  la  Rhétorique  d'Aristote 
pour  montrer  que  notre  façon  d'envisager  la  rhétorique  peut  se  prévaloir  d'illustres  exemples. 
Notre étude, se préoccupant surtout de la structure de l'argumentation, n'insistera donc pas sur la 
manière dont s’effectue la communication avec l'auditoire. 
\bigskip
S'il  est  vrai  que  la  technique  du  discours  public  diffère  de  celle  de  l'argumentation  écrite,  notre 
souci  étant  d'analyser  l'argumentation,  nous  ne  pouvons  nous  limiter  à  l'examen  de  la  technique 
du  discours  parlé.  Bien  plus,  vu  l'importance  et  le  rôle  modernes  de  l'imprimerie,  nos  analyses 
concerneront surtout les textes imprimés. 
\bigskip
Par  contre,  nous  négligerons  la  mnémotechnique  et  l'étude  du  débit  ou  l'action  oratoire;  ces 
problèmes  sont  du  ressort  des  conservatoires  et  des  écoles  d'art  dramatique  ;  nous  nous 
dispenserons de leur examen. 
\bigskip
L'accent mis sur les textes écrits aura pour conséquence, ceux-ci se présentant sous les formes les 
plus variées, que notre étude sera conçue dans toute sa généralité et ne s'arrêtera pas spécialement 
à  des  discours  envisagés  comme  une  unité  d'une  structure  et  d'une  longueur  plus  on  moins 
conventionnellement  admises.  Comme,  d'autre  part,  la  discussion  avec  un  seul  interlocuteur  ou 
même la délibération intime relèvent, selon nous, d'une théorie générale de l'argumentation, l'on 
comprend  que  l'idée  que  nous  avons  de  l'objet  de  notre  étude  déborde  largement  celui  de  la 
rhétorique classique. » 
\bigskip
\bigskip
\bigskip
\bigskip
4 
\bigskip
P 8-9 : « Ce que nous conservons de la rhétorique traditionnelle, c'est l'idée même d'auditoire, qui 
est  immédiatement  évoquée,  dès  que  l'on  pense  à  un  discours.  Tout  discours  s'adresse  à  un 
auditoire et on oublie trop souvent qu'il en est de même de tout écrit. Tandis que le discours est 
conçu  en  fonction  même  de  l'auditoire,  l'absence  matérielle  des  lecteurs  peut  faire  croire  à 
l'écrivain  qu'il  est  seul  au  monde,  bien  qu'en  fait  son  texte  soit  toujours  conditionné, 
consciemment ou inconsciemment, par ceux auxquels il prétend s’adresser. » 
\bigskip
P  9 :  « Aussi,  pour  des  raisons  de  commodité  technique,  et  pour  ne jamais  perdre  de  vue  ce rôle 
essentiel de l'auditoire, quand nous utiliserons les termes « discours », « orateur » et « auditoire», 
nous  comprendrons  par  là  l'argumentation,  celui  qui  la  présente  et  ceux  auxquels  elle  s'adresse 
sans nous arrêter au fait qu'il s'agit d'une présentation par la parole ou par l'écrit, sans distinguer 
discours en forme et expression fragmentaire de la pensée. 
\bigskip
Si,  chez  les  Anciens,  la  rhétorique  se  présentait  comme  l'étude  d'une  technique  à  l'usage  du 
vulgaire  impatient  d'arriver  rapidement  à  des  conclusions,  de  se  former  une  opinion,  sans  s'être 
donné  au  préalable  la  peine  d'une  investigation  sérieuse  (1),  quant  à  nous,  nous  ne  voulons  pas 
limiter l'étude de l'argumentation à celle qui est adaptée à un publie d'ignorants. C'est cet aspect-là 
de  la  rhétorique  qui  explique  qu'elle ait  été  férocement  combattue  par  Platon,  dans  son  Gorgias 
(2), et qui a favorisé son déclin dans l'opinion philosophique. » 
\bigskip
(1) Cf. Aristote Rhétorique, liv. I, chap. 2, 1357 (t.  
(2) Platon, Gorgias, cf. notamment 455, 457 a, 463, 471 d.  
\bigskip
P 9-10 : « En effet, l'orateur est obligé, s'il veut agir, de s'adapter à son auditoire et l'on comprend 
sans peine que le discours le plus efficace sur un auditoire incompétent ne soit pas nécessairement 
celui  qui  emporte  la  conviction  du  philosophe.  Mais  pourquoi  ne  pas  admettre  que  des 
argumentations  puissent  être  adressées  à  toute  espèce  d'auditoires  ?  Quand  Platon  rêve,  dans  le 
Phèdre,  d'une  rhétorique  qui,  elle,  serait  digne  du  philosophe,  ce  qu'il  préconise,  c'est  une 
technique  qui  pourrait  convaincre 
les  dieux  eux-mêmes  (3).  En  changeant  d'auditoire 
l'argumentation  change  d'aspect,  et  si  le  but  qu'elle  vise  est  toujours  d'agir  efficacement  sur  les 
esprits, pour juger de sa valeur on ne peut pas ne pas tenir compte de la qualité des esprits qu'elle 
parvient à convaincre. » 
\bigskip
(3) PLATON, Phèdre, 273 e. 
\bigskip
P 10 : « Ceci justifie l'importance particulière que nous accorderons à l'analyse des argumentations 
philosophiques,  traditionnellement  considérées  comme  les  plus  «  rationnelles  »  qui  soient, 
justement  parce  qu'elles  sont  censées  s'adresser  à  des  lecteurs  sur  lesquels  la  suggestion,  la 
pression  ou  l'intérêt  ont  peu  de  prise.  Nous  montrerons  d'ailleurs  que  les  mêmes  techniques 
d'argumentation  se  retrouvent  à  tous  les  niveaux,  à  celui  de  la  discussion  autour  d'une  table 
familiale,  comme  à  celui  du  débat  dans  un  milieu  très  spécialisé.  Si  la  qualité  des  esprits  qui 
adhèrent  à  certains  arguments,  dans  des  domaines  hautement  spéculatifs,  présente  une  garantie 
pour  leur  valeur,  la  communauté  de  leur  structure  avec  celle  des  arguments  utilisés  dans  les 
discussions quotidiennes expliquera pourquoi et comment on arrive à les comprendre. 
\bigskip
Notre  traité  ne  s'occupera  que  de  moyens  discursifs  d'obtenir  l'adhésion  des  esprits  :  seule  la 
technique utilisant le langage pour persuader et pour convaincre sera examinée par la suite. 
\bigskip
Cette  limitation  n'implique  nullement  que,  à  nos  yeux,  ce  soi  vraiment  le  mode  le  plus  efficace 
d'agir sur les esprits, bien au contraire. Nous sommes fermement convaincus que les croyances les 
plus  solides  sont  celles  qui  non  seulement  sont  admises  sans  preuve,  mais  qui,  bien  souvent  ne 
sont même pas explicitées. Et quand il s'agit d'obtenir l'adhésion, rien de plus sûr que l'expérience 
externe  ou  interne  et  le  calcul  conforme  à  des  règles  préalablement  admises.  Mais  le  recours  à 
\bigskip
\bigskip
\bigskip
5 
\bigskip
l'argumentation  ne  peut  être  évité  quand  ces  preuves  sont  discutées  par  l'une  des  parties,  quand 
l'on n'est pas d'accord sur leur portée ou leur interprétation, sur leur valeur ou leur rapport avec 
les problèmes controversés. » 
\bigskip
P  10-11 :  « D'autre  part,  toute  action  visant  à  obtenir  l'adhésion  tombe  hors  du  champ  de 
l'argumentation,  dans  la  mesure  où  aucun  usage  du  langage  ne  vient  l'appuyer  ou  l'interpréter  : 
celui qui prêche d'exemple sans rien dire, celui qui use de la caresse ou de la gifle peuvent obtenir 
un résultat appréciable. Nous ne nous intéresserons à ces procédés que lorsque, grâce au langage, 
on les met en évidence, que l'on recourt à des promesses ou à des menaces. Encore y a-t-il des cas - 
tels  la  bénédiction  et  la  malédiction  où  le  langage  est  utilisé  comme  moyen  d'action  directe 
magique  et  non  comme  moyen  de  communication.  Nous  n'en  traiterons  que  si  cette  action  est 
intégrée dans une argumentation. » 
\bigskip
P 11 : « Un des facteurs essentiels de la propagande, telle qu'elle s'est surtout développée au XXe 
siècle,  mais  dont  l'usage  était  bien  connu  dès  l'Antiquité  et  que  l'Église  catholique  a  mis  à  profit 
avec un art incomparable, est le conditionnement de l'auditoire grâce à des techniques nombreuses 
et  variées qui utilisent  tout  ce  qui  peut  influer  sur  le  comportement.  Ces  techniques  exercent  un 
effet indéniable pour préparer l'auditoire, pour le rendre plus accessible aux arguments qu'on lui 
présentera. Voilà  encore un point  de vue que notre analyse négligera : nous ne traiterons que du 
conditionnement de l'auditoire par le discours, dont résultent des considérations sur l'ordre dans 
lequel les arguments doivent être présentés pour exercer le plus grand effet. » 
\bigskip
P  11-12 :  « Enfin,  les  preuves  extra-techniques,  comme  les  appelle  Aristote  (1)  -  entendant  par  là 
celles qui ne relèvent pas de la technique rhétorique - n'entreront dans notre étude que lorsqu'il y a 
désaccord  au  sujet  des  conclusions  que  l'on  en  peut  tirer.  Car  nous  nous  intéressons  moins  au 
déroulement  complet  d'un  débat  qu'aux  schèmes  argumentatifs  mis  en  jeu.  La  dénomination 
ancienne de « preuves extra-techniques » est bien faite pour nous rappeler que, tandis que notre 
civilisation,  caractérisée  par  son  extrême  ingéniosité  dans  les  techniques  destinées  à  agir  sur  les 
choses, a complètement oublié la théorie de l'argumentation, de l'action sur les esprits au moyen 
du discours, celle-ci était considérée par les Grecs, sous le nom de rhétorique, comme la -mot grec 
par excellence. » 
\bigskip
(1) Aristote, Rhétorique, liv. 1, chap. 2, 1355 b 
\bigskip
\bigskip
III 
\bigskip
P 12 : « La théorie de l'argumentation visant, grâce ail discours, à obtenir une action efficace sur les 
esprits, aurait pu être traitée comme une branche de la psychologie. En effet, si les arguments ne 
sont  pas  contraignants,  s'ils  ne  doivent  pas  nécessairement  convaincre  mais  possèdent  une 
certaine force, qui peut d'ailleurs varier selon les auditoires, n'est-ce pas à l'effet produit que l'on 
peut juger de celle-ci ? L'étude de l'argumentation deviendrait ainsi un des objets de la psychologie 
expérimentale,  où  des  argumentations  variées  seraient  mises  à  l'épreuve  devant  des  auditoires 
variés,  suffisamment  bien  connus  pour  que  l'on  puisse,  à  partir  de  ces  expériences,  tirer  des 
conclusions  d'une  certaine  généralité.  Des  psychologues  américains  n'ont  pas  manqué  de 
s'adonner à de pareilles études, dont l'intérêt n'est d'ailleurs pas contestable (1). 
\bigskip
Mais  notre  façon  de  procéder  sera  différente.  Nous  cherchons,  tout  d'abord,  à  caractériser  les 
diverses  structures  argumentatives,  dont  l'analyse  doit  précéder  toute  épreuve  expérimentale  à 
laquelle on voudrait soumettre leur efficacité. Et d'autre part, nous ne pensons pas que la méthode 
de  laboratoire  puisse  déterminer  la  valeur  des  argumentations  utilisées  dans  les  sciences 
humaines, en droit et en philosophie, car la méthodologie même du psychologue constitue déjà un 
objet de controverse, et relève de notre étude. » 
\bigskip
\bigskip
\bigskip
\bigskip
6 
\bigskip
(1)  Consulter  notamment  H.  L.  Hollingworth,  The  psychology  of  the  audience,  1935  Carl  I. 
Hovland,  Effects  Of  the  Mass  Media  of  Communication,  chap.  28  du  Handbook  of  social 
psychology, édité par Gardner LINDZEY, 1954. 
\bigskip
P 12-13 : « Notre démarche différera radicalement de la démarche adoptée par les philosophes qui 
s'efforcent  de  réduire  les  raisonnements  en  matière  sociale,  politique  on  philosophique,  en 
s'inspirant  des  modèles  fournis  par  les  sciences  déductives  ou  expérimentales,  et  qui  rejettent 
comme sans valeur tout ce qui ne se  conforme pas aux schèmes préalablement imposés. Bien au 
contraire : nous nous inspirerons des logiciens, mais c'est pour imiter les méthodes qui leur ont si 
bien réussi depuis un siècle environ. » 
\bigskip
P  13 :  « N'oublions  pas,  en  effet,  que  dans  la  première  moitié  du  XIXe  siècle  la  logique  n'avait 
aucun prestige ni auprès des milieux scientifiques ni dans le grand publie. Whately pouvait écrire, 
vers 1828, que si la rhétorique ne jouit plus de l'estime du publie, la logique jouit encore moins de 
ses faveurs (1). 
\bigskip
La  logique  a  pu  prendre  un  brillant  essor  pendant  les  cent  dernières  années  quand,  cessant  de 
ressasser  de  vieilles  formules,  elle  s'est  proposé  d'analyser  les  moyens  de  preuve  effectivement 
utilisés  par  les  mathématiciens.  La  logique  formelle  moderne  s'est  constituée  comme  l'étude  des 
moyens  de  démonstration,  utilisés  dans  les  sciences  mathématiques.  Mais  il  en  résulte  que  son 
domaine  est  limité,  car  tout  ce  qui  est  ignoré  par  les  mathématiciens  est  étranger  à  la  logique 
formelle.  Les  logiciens  se  doivent  de  compléter  la  théorie  de  la  démonstration  ainsi  obtenue  par 
une  théorie  de  l'argumentation.  Nous  chercherons  à  la  construire  en  analysant  les  moyens  de 
preuve  dont  se  servent  les  sciences  humaines,  le  droit  et  la  philosophie;  nous  examinerons  des 
argumentations présentées par des publicistes dans leurs journaux, par des politiciens dans leurs 
discours,  par  des  avocats  dans  leurs  plaidoiries,  par  des  juges  dans  leurs  attendus,  par  des 
philosophes dans leurs traités. 
\bigskip
Notre champ d'études, qui est immense, est resté en friche pendant des siècles. Nous espérons que 
nos premiers résultats inciteront d'autres chercheurs à les compléter et à les perfectionner. » 
\bigskip
(1) Richard D. D. Whately, Elements of Rhetoric, 1828, Préface. 
\bigskip
\bigskip
PREMIERE PARTIE : LES CADRES DE L’ARGUMENTATION 
\bigskip
§ I. DEMONSTRATION ET ARGUMENTATION 
\bigskip
P 17 : « Pour bien exposer les caractères particuliers de l'argumentation et les problèmes inhérents 
à l'étude de celle-ci, rien de tel que de l'opposer à la  conception classique de la démonstration et 
plus  spécialement  à  la  logique  formelle  qui  se  limite  à  l'examen  des  moyens  de  preuve 
démonstratifs. » 
\bigskip
P 17-18 : « Dans la logique moderne, issue d'une réflexion sur le raisonnement mathématique, on 
ne rattache plus les systèmes formels à une quelconque évidence rationnelle. Le logicien est libre 
d'élaborer  comme  il  lui  plaît  le  langage  artificiel  du  système  qu'il  construit,  de  déterminer  les 
signes et combinaisons de signes qui pourront être utilisés. A lui de décider quels sont les axiomes, 
c'est-à-dire les expressions considérées sans preuve comme valables dans son système, et de dire 
quelles  sont  les  règles  de  transformation  qu'il  introduit  et  qui  permettent  de  déduire,  des 
expressions valables, d'autres expressions également valables dans le système. La seule obligation 
qui s'impose au constructeur de systèmes axiomatiques formalisés et qui rend les démonstrations 
contraignantes, est de choisir signes et règles de façon à éviter doute et, ambiguïté. Il faut que, sans 
hésiter, et même mécaniquement, il soit possible d'établir si une suite de signes est admise dans le 
système,  si  elle  est  de  même  forme  qu'une  autre  suite  de  signes,  si  elle  est  considérée  comme 
\bigskip
\bigskip
\bigskip
7 
\bigskip
valable, parce qu'axiome ou expression déductible, à partir des axiomes, d'une façon conforme aux 
règles  de  déduction.  Toute  considération  relative  à  l'origine  des  axiomes  ou  des  règles  de 
déduction, au rôle que le système axiomatique est censé jouer dans l'élaboration de la pensée, est 
étrangère à la logique ainsi conçue, en ce sens qu'elle sort des cadres du formalisme en question. 
La recherche de l'univocité indiscutable a même conduit les logiciens formalistes à construire des 
systèmes  où  l'on  ne  se  préoccupe  pas  du  sens  des  expressions  :  ils  sont  contents  si  les  signes 
introduits  et  les  transformations  qui  les  concernent  sont  hors  discussion.  Ils  abandonnent 
l'interprétation des éléments du système axiomatique à ceux qui appliqueront celui-ci et qui auront 
à se préoccuper de son adéquation au but poursuivi. » 
\bigskip
P 18 : « Quand il s'agit de démontrer une proposition, il suffit d'indiquer à l'aide de quels procédés 
elle  peut  être  obtenue  comme  dernière  expression  d'une  suite  déductive  dont  les  premiers 
éléments  sont  fournis  par  celui  qui  a  construit  le  système  axiomatique  à  l'intérieur  duquel  on 
effectue  la  démonstration.  D'où  viennent  ces  éléments,  sont-ce  des  vérités  impersonnelles,  des 
pensées divines, des résultats d'expérience on des postulats propres à l'auteur, voilà des questions 
que  le  logicien  formaliste  considère  comme  étrangères  à  sa  discipline.  Mais  quand  il  s'agit 
d'argumenter, d'influer au moyen du discours sur l'intensité d'adhésion d'un  auditoire à certaines 
thèses, il n'est plus possible de négliger complètement, en les considérant comme irrelevantes, les 
conditions  psychiques  et  sociales  à  défaut  desquelles  l'argumentation  serait  sans  objet  ou  sans 
effet.  Car  toute  argumentation  vise  à  l'adhésion  des  esprits  et,  par  le  fait  même,  suppose 
L'existence d'un contact intellectuel. 
\bigskip
Pour  qu'il  y  ait  argumentation,  il  faut  que,  à  un  moment  donné,  une  communauté  des  esprits 
effective  se réalise.  Il  faut  que  l'on  soit  d'accord,  tout  d'abord  et  en  principe,  sur  la formation  de 
cette  communauté  intellectuelle  et,  ensuite,  sur  le  fait  de  débattre  ensemble  une  question 
déterminée : or, cela ne va nullement de soi. » 
\bigskip
P  18-19 :  « Même  sur  le  plan  de  la  délibération  intime  il  existe  des  conditions  préalables  à 
l'argumentation : il faut notamment se concevoir comme divisé en deux interlocuteurs, au moins, 
qui  participent  à  la  délibération.  Et,  cette  division,  rien  ne  nous  autorise  à  la  considérer  comme 
nécessaire.  Elle  paraît  constituée  sur  le  modèle  de  la  délibération  avec  autrui.  Dès  lors,  il  est  à 
prévoir  que  nous  retrouverons,  transposés  à  la  délibération  avec  soi-même,  la  plupart  des 
problèmes  relatifs  aux  conditions  préalables  à  la  discussion  avec  autrui.  Bien  des  expressions  en 
témoignent; ne mentionnons que des formules telles : « N'écoute point ton mauvais génie », «Ne 
remets  plus  cela  en  discussion  »,  qui  sont  relatives  l'une  à  des  conditions  préalables  tenant  aux 
personnes, l'autre à des conditions préalables tenant à l'objet de l'argumentation. » 
\bigskip
§ 2. LE CONTACT DES ESPRITS 
\bigskip
P 19 : « La formation d'une communauté effective des esprits exige un ensemble de conditions. 
\bigskip % JJD 2016 10 17
Le  minimum  indispensable  à  l'argumentation  semble  l'existence  d'un  langage  commun,  d'une 
technique permettant la communication. 

\bigskip
Cela ne suffit point. Nul ne le montre mieux que l'auteur d'Alice au  Pays des Merveilles. En effet, 
les  êtres  de  ce  pays  comprennent  à  peu  près  le  langage  d'Alice.  Mais  le  problème  pour  elle  est 
d'entrer en contact, d'entamer une discussion, car dans le monde des Merveilles, il n'y a  aucune 
raison pour que les discussions commencent. On ne sait pas pourquoi l'un s'y adresserait à l'autre. 
Parfois Alice prend l'initiative et utilise tout bonnement le vocatif « ô souris (1). » Elle considère 
comme  un  succès  d'avoir  pu  échanger  quelques  paroles  indifférentes  avec  la  duchesse  (2).  Par 
contre, dans l'entrée en matière avec la chenille, on arrive aussitôt à un point mort: « je crois que 
vous  devriez  me  dire,  d'abord,  qui  vous  êtes  ?  »  -  «  Pourquoi,  dit  la  chenille  (1)  ?  »  Dans  notre 
monde hiérarchisé, ordonné, il existe généralement des règles établissant comment la conversation 
peut  s'engager, un  accord  préalable  résultant  des  normes  mêmes  de  la  vie  sociale.  Entre  Alice  et 
ceux du Pays des Merveilles, il n'y a ni  hiérarchie, ni préséance, ni fonctions qui fassent que l'un 
doive répondre plutôt que l'autre. Même les conversations entamées, tournent souvent court, telle 
la conversation avec le loir. Celui-ci se prévaut de son âge : 
\bigskip
« Et ceci, Alice ne voulait pas le permettre sans savoir quel âge il avait et comme le loir refusait 
carrément de donner son âge, il n'y avait plus rien à dire (2) ». 
\bigskip
(1) Lewis Carroll, Alice's Adventures in Wonderland, p. 41.  
(2) Ibid., p. 82. 
(1) L. Carroll, ibid., p. 65. 
(2) Ibid., p. 44. 
\bigskip
P  20 :  « La  seule  des  conditions  préalables  ici  réalisée,  c'est  le  désir  d'Alice  d'entrer  en 
conversation avec les êtres de ce nouvel univers. 
\bigskip
% ----------
L'ensemble de ceux auxquels on désire s'adresser est fort variable. Il est loin de comprendre, pour 
chacun, tous les êtres humains. Par contre, l'univers auquel l'enfant veut s'adresser, dans la mesure 
précisément  où  le  monde  des  adultes  lui  est  fermé,  s'accroît  par  l'adjonction  des  animaux,  et  de 
tous les objets inanimés qu'il considère comme ses interlocuteurs naturels (3). 
\bigskip
Il y a des êtres avec lesquels tout  contact peut sembler superflu ou peu désirable. Il y a des êtres 
auxquels on ne se soucie pas d'adresser la parole ; il y en a aussi avec qui on ne veut pas discuter, 
mais auxquels on se contente d'ordonner. » 
\bigskip
(3) E. Cassirer, Le langage et la construction du monde des objets, J. de Psychologie, 1933, XXX, p. 
39. 
\bigskip
P 20-21 : « Il faut, en effet, pour argumenter, attacher du prix à l'adhésion de son interlocuteur, à 
son consentement, à son concours mental. C'est donc parfois une distinction appréciée que d'être 
une  personne  avec  qui  l'on  discute.  Le  rationalisme  et  l'humanisme  des  derniers  siècles  font 
paraître  étrange  l'idée  que  ce  soit  une  qualité  que  d'être  quelqu'un  de  l'avis  duquel  on  se 
préoccupe, mais dans bien des sociétés, on n'adresse pas la parole à n'importe qui, comme on ne se 
battait  pas  en  duel  avec  n'importe  qui.  Remarquons  d'ailleurs  que  vouloir  convaincre  quelqu'un 
implique toujours une certaine modestie de la part de celui qui argumente, ce qu'il dit ne constitue 
pas « parole d'Êvangile »,  il ne dispose pas de cette autorité qui fait que ce qu'il dit est indiscuté, et 
emporte  immédiatement  la  conviction.  Il  admet  qu'il  doit  persuader,  penser  aux  arguments  qui 
peuvent agir sur son interlocuteur, se soucier de lui, s'intéresser à son état d'esprit. 
\bigskip
Les  êtres  qui  veulent  compter  pour  autrui,  adultes  ou  enfants,  souhaitent qu'on  ne  leur  ordonne 
plus,  mais  qu'on  les  raisonne,  qu'on  se  préoccupe  de  leurs  réactions,  qu'on  les  considère  comme 
des membres d'une société plus ou moins égalitaire. Celui qui n'a cure d'un pareil contact avec les 
autres sera jugé hautain, peu sympathique, au contraire de ceux qui, quelle que soit l'importance 
de leurs fonctions, n'hésitent pas à marquer par leurs discours au public, le prix qu'ils attachent à 
son appréciation. 
\bigskip
Mais,  on  l'a  dit  maintes  fois,  il  n'est  pas  toujours  louable  de  vouloir  persuader  quelqu'un  :  les 
conditions  dans  lesquelles  le  contact  des  esprits  s'effectue  peuvent,  en  effet,  paraître  peu 
honorables.  On  connaît  la  célèbre  anecdote  concernant  Aristippe  à  qui  l'on  reprochait  de  s'être 
abaissé devant le tyran Dionysios au point de se mettre à ses pieds pour être entendu. Aristippe se 
défendit en disant que ce n'était pas sa faute, mais celle de Dionysios qui avait les oreilles dans les 
pieds. Serait-il donc indifférent où les oreilles se trouvent (1) ? » 
\bigskip
(1) Bacon, 0f the advancement of learning, p. 25. 
\bigskip
\bigskip
\bigskip
9 
\bigskip
 
P 22 : « Pour Aristote, le danger de discuter avec certaines personnes est que l'on y perd soi-même 
la qualité de son argumentation : 
\bigskip
Il ne faut pas discuter avec tout le monde, ni pratiquer la Dialectique avec le premier venu car, à 
l'égard  de  certaines  gens,  les  raisonnements  s'enveniment  toujours.  Contre  un  adversaire,  en 
effet, qui essaye par tous les moyens de paraître se dérober, il est légitime de tenter par tous les 
moyens d'arriver à la conclusion ; mais ce procédé manque d'élégance (1). 
\bigskip
Il ne suffit pas de parler ou d'écrire, il faut encore être écouté, être lu. Ce n'est pas rien que d'avoir 
l'oreille de quelqu'un, d'avoir une large audience, d'être admis à prendre la parole dans certaines 
circonstances,  dans  certaines  assemblées,  dans  certains  milieux.  N'oublions  pas  qu'écouter 
quelqu'un, c'est se montrer disposé à admettre éventuellement son point de vue. Quand Churchill 
interdit  aux  diplomates  anglais  même  d'écouter  les  propositions  de  paix  que  les  émissaires 
allemands  pourraient  leur  transmettre  ou  quand  un  parti  politique  fait  savoir  qu'il  est  disposé  à 
entendre  les  propositions  que  pourrait  lui  présenter  un  formateur  de  cabinet,  ces  deux  attitudes 
sont  significatives,  parce  qu'elles  empêchent  l'établissement  ou  reconnaissent  l'existence  des 
conditions préalables à une argumentation éventuelle. 
\bigskip
Faire partie d'un même milieu, se fréquenter, entretenir des relations sociales, tout cela facilite la 
réalisation des conditions préalables au contact des esprits. Les discussions frivoles et sans intérêt 
apparent  ne  sont  pas  toujours  dénuées  d'importance  en  ce  qu'elles  contribuent  au  bon 
fonctionnement d'un mécanisme social indispensable. »  
\bigskip
(1) Aristote, Topiques, liv. VIII, chap. 14,164 b. 
\bigskip
§ 3. L'ORATEUR ET SON AUDITOIRE 
\bigskip
P 22-23 : « Les auteurs de communications ou de mémoires scientifiques pensent souvent qu'il 
leur suffit de rapporter certaines expériences, de mentionner certains faits d'énoncer un certain 
nombre de vérités pour susciter immanquablement l'intérêt de leurs auditeurs ou lecteurs 
éventuels. Cette attitude résulte de l'illusion, fort répandue dans certains milieux rationalistes et 
scientistes, que les faits parlent par eux-mêmes et impriment une empreinte indélébile sur tout 
esprit humain, dont ils forcent l'adhésion, quelles que soient ses dispositions. K. F. Bruner, 
secrétaire de rédaction d'une revue psychologique, compare ces auteurs, peu préoccupés de leur 
auditoire, à un visiteur discourtois: 
\bigskip
Ils  s'affalent  sur  une  chaise,  fixant  maussadement  leurs  souliers  et  annoncent  brusquement,  à 
eux-mêmes ou à d'autres, on ne le sait jamais : « Un tel et un tel ont montré... que la femelle du 
rat  blanc  répond  négativement  au  choc  électrique...  »  Très  bien,  monsieur,  leur  dis-je,  alors 
quoi ? Dites-moi d'abord pourquoi je dois m'en soucier, alors j'écouterai (1). » 
\bigskip
(1)  K.  F.  Bruner,  Of  psychological  writing,  Journal  of  abnormal  and  social  Psychology,  1942,  vol. 
37, p. 62. 
\bigskip
P 23 : « Il est vrai que ces auteurs, pour autant qu'ils prennent la parole dans une société savante, 
ou publient un article dans une revue spécialisée, peuvent négliger les moyens d'entrer en contact 
avec  leur  public,  parce  qu'une  institution  scientifique,  société  ou  revue,  fournit  ce  lien 
indispensable entre l'orateur et son auditoire. Le rôle de l'auteur n'est que de maintenir, entre lui 
et le publie, ce contact que l'institution scientifique a permis d'établir. » 
\bigskip
P 23-24 : « Mais tout le monde ne se trouve pas dans une situation aussi privilégiée. Pour qu'une 
argumentation se développe, il faut, en effet, que ceux auxquels elle est destinée y prêtent quelque 
attention.  La  plupart  des  formes  de  publicité  et  de  propagande  se  préoccupent,  avant  tout, 
d'accrocher  l'intérêt  d'un  public  indifférent,  condition  indispensable  pour  la  mise  en  œuvre  de 
n'importe quelle argumentation. Ce n'est pas parce que, dans un grand nombre de domaines - qu'il 
s'agisse  d'éducation,  de  politique,  de  science,  ou  d'administration  de  la  justice  -  toute  société 
possède  des  institutions  facilitant  et  organisant  ce  contact  des  esprits,  qu'il  faut  méconnaître 
l'importance de ce problème préalable. » 
\bigskip
P  24 :  « Normalement,  il  faut  quelque  qualité  pour  prendre  la  parole  et  être  écouté.  Dans  notre 
civilisation,  où  l'imprimé,  devenu  marchandise,  profite  de  l'organisation  économique  pour 
s'imposer  à  l'attention,  cette  condition  n'apparaît  nettement  que  dans  les  cas  oit  le  contact  entre 
l'orateur et son auditoire ne peut s'établir grâce aux techniques de distribution. On la perçoit donc 
mieux  quand  l'argumentation  est  développée  par  un  orateur  s'adressant  verbalement  à  un 
auditoire  déterminé  que  lorsqu'elle  est  contenue  dans  un  livre  mis  en  vente  en  librairie.  Cette 
qualité de l'orateur, sans laquelle il ne sera pas écouté et même, bien souvent, ne sera pas autorisé 
à prendre la parole, peut varier selon les circonstances. Parfois il suffira de se présenter comme un 
être  humain,  décemment  habillé,  parfois  il  faudra  être  adulte,  parfois,  membre  quelconque  d'un 
groupe constitué, parfois, porte-parole de ce groupe. Il y a des fonctions qui, seules, autorisent à 
prendre la parole dans certains cas, ou devant certains auditoires, des domaines où ces problèmes 
d'habilitation sont minutieusement réglementés. 
\bigskip
Ce contact entre l'orateur et son auditoire ne concerne pas uniquement les conditions préalables à 
l'argumentation : il est essentiel également pour tout le développement de celle-ci. En effet, comme 
l'argumentation  vise  à  obtenir  l'adhésion  de  ceux  auxquels  elle  s'adresse,  elle  est,  tout  entière, 
relative à l'auditoire qu'elle cherche à influencer. » 
\bigskip
P  24-25 :  « Comment  définir  pareil  auditoire  ?  Est-ce  la  personne  que  l'orateur  interpelle 
nommément ? Pas toujours : le député qui, au Parlement anglais, doit s'adresser au président, peut 
chercher à convaincre, non seulement ceux qui l'écoutent, mais encore l'opinion publique de son 
pays. Est-ce l'ensemble des personnes que l'orateur voit devant lui quand il prend la parole? Pas 
nécessairement.  Il  peut  parfaitement en  négliger  une  partie :  un  chef  de  gouvernement,  dans  un 
discours  au  Parlement,  peut  renoncer  d'avance  à  convaincre  les  membres  de  l'opposition  et  se 
contenter  de  l'adhésion  de  sa  majorité.  Par  ailleurs,  celui  qui  accorde  une  interview  à  un 
journaliste considère que son auditoire est constitué par les lecteurs du journal plutôt que par la 
personne qui se trouve devant lui. Le secret des délibérations, en modifiant l'idée que l'orateur se 
fait de son auditoire, peut transformer les termes de son discours. On voit immédiatement par ces 
quelques exemples, combien il est difficile de déterminer à l'aide de critères purement matériels, 
l'auditoire  de  celui  qui  parle  ;  cette  difficulté  est  bien  plus  grande  encore  quand  il  s'agit  de 
l'auditoire  de  l'écrivain,  car,  dans  la  plupart  des  cas,  les  lecteurs  ne  peuvent  être  repérés  avec 
certitude. » 
\bigskip
P  25 :  « C'est  la  raison  pour  laquelle,  il  nous  semble  préférable  de  définir  l'auditoire,  en  matière 
rhétorique, comme l'ensemble de ceux sur lesquels l'orateur veut influer par son argumentation. 
Chaque orateur pense, d'une façon plus ou moins consciente, à ceux qu'il cherche à persuader et 
qui constituent l'auditoire auquel s'adressent ses discours. 
\bigskip
§ 4. L'AUDITOIRE COMME CONSTRUCTION DE L'ORATEUR 
\bigskip
L'auditoire  présumé  est  toujours,  pour  celui  qui  argumente,  une  construction  plus  ou  moins 
systématisée. On peut tenter d'en déterminer les origines psychologiques (1) ou sociologiques (2); 
ce qui importe, à celui qui se propose de persuader effectivement des individus concrets, c'est que 
la construction de l'auditoire ne soit point inadéquate à l'expérience. » 
\bigskip
(1) Cf. Harry Stack Sullivan, The Interpersonal Theory of Psychiatry, New York, 1953. 
(2) M. Millioud, La propagation des idées, Revue phil., 1910, vol. 69, pp. 580600 ; vol. 70, pp. 168-
191. 
\bigskip
\bigskip
\bigskip
11 
\bigskip
 
P  25-26 :  « Il  n'en  va  pas  de  même  pour  celui  qui  se  livre  à  des  essais  sans  portée  réelle.  La 
rhétorique,  devenue  exercice  scolaire,  s'adresse  à  des  auditoires  conventionnels  et  peut,  sans 
inconvénient, s'en tenir à des visions stéréotypées de ceux-ci, ce qui a contribué, tout autant que la 
facticité des thèmes, à la faire dégénérer (1). » 
\bigskip
(1) H. I. Marrou, Histoire de l'éducation dans l'Antiquité, p. 278. 
\bigskip
P  26 :  « L'argumentation  effective  se  doit  de  concevoir  l'auditoire  présumé  aussi  proche  de  la 
réalité  que  possible.  Une  image  inadéquate  de  l'auditoire,  qu'elle  résulte  de  l'ignorance  ou  d'un 
concours  imprévu  de  circonstances,  peut  avoir  les  conséquences  les  plus  fâcheuses.  Une 
argumentation  que  l'on  considère  comme  persuasive  risque  d'avoir  un  effet  révulsif  sur  un 
auditoire pour lequel les raisons pour sont, en fait, des raisons contre. Ce que l'on dira en faveur 
d'une  mesure  en  alléguant  qu'elle  est  susceptible  de  diminuer  la  tension  sociale  dressera  contre 
cette mesure tous ceux qui souhaitent que des troubles se produisent. 
\bigskip
La connaissance de ceux que l'on se propose de gagner est donc une condition préalable de toute 
argumentation efficace. 
\bigskip
Le  souci  de  l'auditoire  transforme  certains  chapitres  des  anciens  traités  de  rhétorique  en  de 
véritables études de psychologie. C'est dans sa Rhétorique qu'Aristote, parlant d'auditoires classés 
d'après l'âge et la fortune, a inséré maintes descriptions fines et toujours valables de psychologie 
différentielle (2). Cicéron démontre qu'il faut parler autrement à l'espèce d'hommes « ignorante et 
grossière,  qui  préfère  toujours  l'utile  à  l'honnête  »  et  à  «  l'autre,  éclairée  et  cultivée  qui  met  la 
dignité morale au-dessus de tout » (3). Quintilien, après lui, s'attache aux différences de caractère, 
importantes pour l'orateur (4). 
\bigskip
(2)  Aristote,  Rhétorique,  liv.  11,  chap.  12  à  17,  1388  b  à  1391  b.  Voir  étude  de  S.  Decoster, 
L'idéalisme des jeunes, dans -Moraleet enseignement, 1951-52, N°2 et 3. 
(3) Cicéron, Partitiones oratoriae, § 90. 
(4) Quintillien, De Institutione Oratoria, vol. 1, liv. 111, chap. VIII, §§ 38 et suiv. 
\bigskip
P 26-27 : « L'étude des auditoires pourrait également constituer un chapitre de sociologie, car, plus 
que  de  son  caractère  propre,  les  opinions  d'un  homme  dépendent  de  son  milieu  social,  de  son 
entourage, des gens qu'il fréquente et parmi lesquels il vit : « Voulez-vous, disait M. Millioud, que 
l'homme inculte change d'opinions ? Transplantez-le (1). » Chaque milieu pourrait être caractérisé 
par  ses  opinions  dominantes,  par  ses  convictions  indiscutées,  par  les  prémisses  qu'il  admet  sans 
hésiter : ces conceptions font partie de sa culture et tout orateur qui veut persuader un auditoire 
particulier ne peut que S'y adapter. Aussi la culture propre de chaque auditoire transparaît-elle à 
travers les discours qui lui sont destinés, de manière telle que c'est dans une large mesure de ces 
discours  eux-mêmes  que  nous  nous  croyons  autorisés  à  tirer  quelque  information  au  sujet  des 
civilisations révolues. » 
\bigskip
(1) H. Millioud, Op. cit., vol. 70, p. 173.  
\bigskip
P  27 :  « Les  considérations  sociologiques  utiles  à  l'orateur  peuvent  porter  sur  un  objet 
particulièrement précis : à savoir les fonctions sociales remplies par les auditeurs. En effet, ceux-ci 
adoptent souvent des attitudes qui sont liées au rôle qui leur est confié dans certaines institutions 
sociales. Ce fait a été souligné par le créateur de la psychologie de la forme : 
\bigskip
On peut observer, écrit-il (2), de merveilleux changements dans les individus, comme lorsqu'une 
personne  passionnément  partisane  devient  membre  d'un  jury,  ou  arbitre,  on  juge,  et  que  ses 
\bigskip
\bigskip
\bigskip
12 
\bigskip
actions montrent alors le délicat passage de l'attitude partisane à un honnête effort pour traiter le 
problème en cause d'une manière juste et objective. 
\bigskip
Il  en  est  de  même  de  la  mentalité  d'un  homme  politique  dont  la  vision  change  quand,  après  des 
années passées dans l'opposition, il devient membre responsable du gouvernement. » 
\bigskip
(2) M. Wertheimer, Productive Thinking, pp. 135-136. 
\bigskip
P 27-28 : « L'auditeur, dans ses nouvelles fonctions, a pris une personnalité nouvelle, que l'orateur 
ne peut ignorer. Et ce qui vaut pour chaque auditeur particulier, n'est pas moins valable pour les 
auditoires,  dans  leur  ensemble,  à  tel  point  même  que  les  théoriciens  de  la  rhétorique  ont  cru 
pouvoir distinguer des genres oratoires d'après le rôle que remplit l'auditoire auquel on s'adresse. 
Les genres oratoires, tels que les définissaient les Anciens, genre délibératif, judiciaire, épidictique, 
correspondaient  respectivement,  selon  eux,  à  des  auditoires  en  train  de  délibérer,  de  juger  ou, 
simplement,  de  jouir  en  spectateur  du  développement  oratoire,  sans  devoir  se  prononcer  sur  le 
fond de l'affaire (1). » 
\bigskip
(1) Aristote, Rhétorique, liv. I, chap. 3, 1358 b, 2-7 ; Cicéron, Orator, § 37 Partitiones oratoriae, § 
10 ; Quintillien, Vol. 1, liv. III, chap. IV. 
\bigskip
P 28 : «  Il s'agit, ici, d'une distinction purement pratique dont les défauts et les insuffisances sont 
manifestes,  surtout  dans  la  conception  qu'elle  présente  du  genre  épidictique  ;  nous  aurons 
d'ailleurs à y revenir (2). Mais si cette classification des discours ne peut être acceptée telle quelle 
par  celui  qui  étudie  la  technique  de  l'argumentation,  elle  a  pourtant  le  mérite  de  souligner 
l'importance que doit attacher l'orateur aux fonctions de son auditoire. 
\bigskip
Il  arrive  bien  souvent  que  l'orateur  ait  à  persuader  un  auditoire  composite,  réunissant  des 
personnes  différenciées  par  leur  caractère,  leurs  attaches  ou  leurs  fonctions.  Il  devra  utiliser  des 
arguments multiples pour gagner les divers éléments de son auditoire. C'est l'art de tenir compte, 
dans son argumentation, de cet auditoire composite qui caractérise le grand orateur. On pourrait 
trouver  des  échantillons  de  cet  art  en  analysant  les  discours  tenus  devant  les  Parlements,  où  les 
éléments de l'auditoire composite sont facilement discernables. » 
\bigskip
(2) Cf. § Il : Le genre épidictique. 
\bigskip
P 28-29 : « Il n'est pas nécessaire de se trouver devant plusieurs factions organisées pour penser 
au  caractère  composite  de  son  auditoire.  En  effet,  on  peut  considérer  chacun  de  ses  auditeurs 
comme  faisant  partie,  à  divers  points  de  vue,  mais  simultanément,  de  groupes  multiples.  Même 
lorsque l'orateur se trouve en face d'un nombre limité d'auditeurs, voire d'un auditeur unique, il se 
peut qu'il hésite à reconnaître les arguments qui paraîtront les plus convaincants à son auditoire ; 
il l'insère alors en quelque sorte fictivement en une série d'auditoires différents. Dans son Tristram 
Shandy -auquel nous nous référerons encore maintes fois, parce que l'argumentation en constitue 
l'un des thèmes principaux - Sterne décrit une discussion entre les parents du héros. Mon père, dit 
celui-ci,  qui voulait  convaincre  ma  mère  de  prendre  un  accoucheur  fit  valoir  ses  arguments  sous 
tous les angles ; discuta en chrétien,en païen en mari, en père, en patriote, en homme: ma mère ne 
répondit jamais qu'en femme. Ce fut un jeu dur pour elle : incapable d'adopter pour le combat tant 
de masques divers, elle soutenait une partie 
inégale, se battait à un contre sept (1). » 
\bigskip
(1) Sterne, Vie et opinions de Tristram Shandy, liv. 1, chap. XVIII, p. 45. 
\bigskip
P  29 : « Or,  prenons-y  garde,  ce  n'est  point  seulement  l'orateur qui  change  ainsi  de  visage  :  c'est 
bien plus encore l'auditoire auquel il s'adresse -sa pauvre épouse en l'occurrence - qu'il transforme 
\bigskip
\bigskip
\bigskip
13 
\bigskip
ainsi au gré de sa fantaisie pour en saisir les aspects les plus vulnérables. Mais l'initiative de cette 
décomposition de l'auditoire revenant à l'orateur, c'est à lui-même que l'on applique les termes « 
en chrétien », « en païen », « en mari », « en père »... 
\bigskip
Devant  une  assemblée,  l'orateur  peut  tenter  de  situer  l'auditoire  dans  ses  cadres  sociaux.  Il  se 
demandera si son auditoire est inclus tout entier dans un seul groupe social ou s'il doit répartir ses 
auditeurs  en  groupes  multiples,  voire  opposés.  Dans  ce  cas,  plusieurs  points  de  départ  sont 
toujours possibles : on peut, en effet, diviser idéalement l'auditoire en fonction de groupes sociaux 
-  par  exemple  politiques,  professionnels,  religieux  -  auxquels  les  individus  appartiennent,  ou  en 
fonction  de  valeurs  auxquelles  certains  auditeurs  adhèrent.  Ces  divisions  idéales  ne  sont  point 
indépendantes  l'une  de  l'autre  ;  néanmoins,  elles  peuvent  mener  à  la  constitution  d'auditoires 
partiels très différents. » 
\bigskip
P  29-30 :  « La  subdivision  d'une  assemblée  en  sous-groupes  dépendra  d'ailleurs  de  la  position 
propre  de  l'orateur  :  si  celui-ci  entretient,  sur  une  question,  des  vues  extrémistes,  rien  ne 
s'opposera à ce qu'il envisage tous ses interlocuteurs comme formant un seul auditoire. Par contre, 
s'il est d'opinion modérée, il sera amené à les envisager comme formant au moins deux auditoires 
distincts (1). » 
\bigskip
(1) Cf. les remarques de L. Festinger sur la moindre tendance à la communication chez les tenants 
d'opinions médianes, Psychol. Review, vol. 57, n° 5, sept. 1950, p. 275. 
\bigskip
P  30 :  « La  connaissance  de  l'auditoire  ne  se  conçoit  pas  indépendamment  de  celle  des  moyens 
susceptibles  d'agir  sur  lui-  En  effet,  le  problème  de  la  nature  de  l'auditoire  est  lié  à  celui  de  son 
conditionnement.  Ce  terme  implique,  au  premier  abord,  qu'il  s'agit  de  facteurs  extrinsèques  à 
l'auditoire.  Et  toute  étude  de  ce  conditionnement  suppose  que  celui-ci  est  considéré  comme 
s’appliquant  à  une  entité  qui,  elle,  serait  l'auditoire  pris  en  lui-même.  Mais,  à  y  regarder  de  plus 
près,  connaître  l'auditoire,  c'est  aussi  savoir,  d'une  part,  comment  on  peut  assurer  son 
conditionnement, d'autre part, quel est, à chaque instant du discours, le conditionnement qui a été 
réalisé. 
\bigskip
Pour pouvoir mieux agir sur un auditoire on peut le conditionner par des moyens divers : musique, 
éclairage, jeux de niasses humaines,  paysage, régie théâtrale. Ces moyens ont été connus de tout 
temps,  ils  ont  été  mis  en  œuvre  aussi  bien  par  les  primitifs  que  par  les  Grecs,  les  Romains,  les 
hommes  du  moyen  âge;  les  perfectionnements  techniques  ont  permis,  de  nos  jours,  de  les 
développer  puissamment;  si  bien  que  l'on  a  vu  dans  ces  moyens  l'essentiel  de  l'action  sur  les 
esprits. » 
\bigskip
P  30 :  « A  côté  de  ce  conditionnement,  dont  nous  ne  pouvons  aborder  l'étude,  existe  un 
conditionnement par le discours lui-même, de sorte que l'auditoire n'est plus, en fin de discours, 
exactement  le  même  qu'au  début.  Ce  dernier  conditionnement  ne  peut  être  réalisé  que  par 
l'adaptation continue de l'orateur à l'auditoire. » 
\bigskip
§ 5. ADAPTATION DE L'ORATEUR A L'AUDITOIRE 
\bigskip
P  31 :  « Tout  l'objet  de  l'éloquence,  écrit  Vico,  est  relatif  à  nos  auditeurs,  et  c'est  suivant  leurs 
opinions que nous devons régler nos  discours (1). » L'important, dans l'argumentation, n'est pas 
de savoir ce que l'orateur considère lui-même comme vrai ou comme probant, mais quel est l'avis 
de  ceux  auxquels  elle  s'adresse.  Il  en  est  d'un  discours,  pour  reprendre  une  comparaison  de 
Gracian " comme d'un festin, où les viandes ne s'apprêtent pas au goût des cuisiniers, mais à celui 
des conviez » (2). 
\bigskip
Le grand orateur, celui qui a prise sur autrui, paraît animé par l'esprit même de son auditoire. Ce 
n'est pas le cas de l'homme passionné qui ne se préoccupe que de ce qu'il ressent lui-même. Si ce 
\bigskip
\bigskip
\bigskip
14 
\bigskip
dernier  peut  exercer  quelque  action  sur  les  personnes  suggestibles,  son  discours  paraîtra  le  plus 
souvent  déraisonnable  aux  auditeurs.  Le  discours  du  passionné,  affirme  M.  Pradines,  s'il  peut 
toucher, ne rend pas un son « vrai », toujours la vraie figure « crève le masque logique », car dit-il, 
« la passion est incommensurable aux raisons » (3). Ce qui paraît expliquer ce point de vue, c'est 
que  l'homme  passionné,  alors  qu'il  argumente,  le  fait  sans  tenir  compte  suffisamment  de 
l'auditoire auquel il s'adresse : emporté par son enthousiasme, il imagine l'auditoire sensible aux 
mêmes arguments que ceux qui l'ont persuadé lui-même. Ce que la passion provoque, c'est donc, 
par cet oubli de l'auditoire, moins une absence de raisons qu'un mauvais choix des raisons. » 
\bigskip
(1) Vico, éd. Ferrari, vol. Il, De nostri temporis studiorum ratione, p. 10. 
(2) Gracian, L'homme de cour, trad. Amelot De La Houssaie, P. 85. 
(3) M. Pradines, Traité de psychologie générale, vol. 11, pp. 324-325. 
\bigskip
P  31-32 :  « Parce  que  les  chefs  de  la  démocratie  athénienne  adoptaient  la  technique  de  l'habile 
orateur,  un  philosophe  comme  Platon  leur  reprochait  de  «  flatter  »  la  foule  qu'ils  auraient  dû 
diriger.  Mais  aucun  orateur,  pas  même  l'orateur  sacré,  ne  peut  négliger  cet  effort  d'adaptation  à 
l'auditoire.  C'est  aux  auditeurs,  dit  Bossuet  (1),  de  faire  les  prédicateurs. Dans  sa  lutte  contre  les 
démagogues, Démosthène demande au peuple athénien de s'améliorer pour améliorer la manière 
des orateurs : 
\bigskip
« Jamais vos orateurs, dit-il, ne vous rendent bons ou mauvais ; c'est vous qui faites d'eux ce que 
vous voulez. Vous, en effet, vous ne vous proposez pas de vous conformer à leur volonté, tandis 
n'eux se règlent sur les désirs qu'ils vous attribuent. Ayez donc des volontés saines et tout ira bien. 
Car, de deux choses, l'une : ou personne ne dira rien de mal, ou celui qui en dira n'en profitera 
pas, faute d'auditeurs disposés à se laisser persuader (2). » 
\bigskip
(1) Bossuet, Sermons, Vol. II : Sur la parole de Dieu, p. 153. 
(2) Démosthène, Harangues et plaidoyers politiques, t. I : Sur l'organisation financière, § 36. 
\bigskip
P  32 :  « C'est  en  effet  à  l'auditoire  que  revient  le  rôle  majeur  pour  déterminer  la  qualité  de 
l'argumentation et le comportement des orateurs (3). » 
\bigskip
(3) Cf. § 2 : Le contact des esprits. 
\bigskip
P  32-33 :  « Si  les  orateurs  ont  pu  être  comparés,  dans  leurs  relations  avec  les  auditeurs,  non 
seulement  à  des  cuisiniers,  mais  même  à  des  parasites  qui  «  pour  avoir  place  dans  les  bonnes 
tables  tiennent  presque  toujours  un  langage  contraire  à  leurs  sentiments  »  (4),  n'oublions  pas 
toutefois  que,  presque  toujours,  l'orateur  est  libre,  lorsqu'il  ne  pourrait  le  faire  efficacement  que 
d'une façon qui lui répugne, de renoncer à persuader un auditoire déterminé: il ne faut pas croire 
qu'il  soit  toujours  honorable,  en  cette  matière,  ni  de  réussir  ni  même  de  se  le  proposer.  Le 
problème de concilier les scrupules de l'honnête homme avec la soumission à l'auditoire est un de 
ceux qui ont le plus préoccupé Quintilien (5). Pour ce dernier, la rhétorique  scientia bene dicendi 
(6)  implique  que  l'orateur  accompli  persuade  bien,  mais  aussi  qu'il  dit  le  bien.  Or,  si  l'on  admet 
qu'il y a des auditoires de gens dépravés que l'on ne veut pas renoncer à convaincre, et si l'on se 
place au point de vue de la qualité morale de l'orateur, on est incité, pour résoudre la difficulté, à 
établir des dissociations et des distinctions qui ne vont pas de soi. » 
\bigskip
(4) Saint-Evremond, t. IX, p. 19, d'après Pétrone, Satiricon, 111, p. 3. 
(5) Quintillien, Vol. I, liv. III, chap. VIII ; vol. IV, liv. XII, chap. I. 
(6) Quintillien, Vol. I, liv. Il, chap. XV, § 34. 
\bigskip
P 33 : « L'obligation pour l'orateur de s'adapter à son auditoire, et la limitation de ce dernier à la 
foule  incompétente,  incapable  de  comprendre  un  raisonnement  suivi  et  dont  l'attention  est  à  la 
\bigskip
\bigskip
\bigskip
15 
\bigskip
merci de la moindre distraction, ont lion seulement mené au discrédit de la rhétorique, mais ont 
introduit dans la théorie du discours des règles générales dont la validité semble pourtant limitée à 
des  cas  d'espèce.  Nous  ne  voyons  pas,  par  exemple,  pourquoi,  en  principe,  l'utilisation  d'une 
argumentation technique éloignerait de la rhétorique et de la dialectique (1). 
\bigskip
Il n'y a qu'une règle en cette matière, qui est l'adaptation du discours à l'auditoire, quel qu'il soit : 
le fond et  la forme de  certains arguments, appropriés à certaines circonstances, peuvent paraître 
ridicules dans d'autres (2). 
\bigskip
La réalité des mêmes événements décrits dans un ouvrage qui se prétend scientifique ou dans un 
roman historique ne doit pas être prouvée de la même façon, et tel qui aurait trouvé saugrenues les 
preuves  fournies  par  J.  Romains  de  suspension  volontaire  des  mouvements  cardiaques,  si  elles 
avaient paru dans une revue médicale, peut y voir une hypothèse à laquelle il prend intérêt, quand 
il la trouve développée dans un roman. (3) » 
\bigskip
(1) Aristote, Rhétorique. liv. 1, chap. 2, 1357 a et 1358 a. 
(2) Richard D. D. Whately, Elements of Rhetoric, Part 111, chap. 1, § 2, p. 174. 
(3) A. Reyes, EI Deslinde, p. 40. (Jules Romains, Les hommes de bonne volonté, vol. XII : Les 
créateurs, chap. 1 à VII) ; cf. Y. Belaval, Les philosophes et leur langage, p. 138. 
\bigskip
P  33-34 :  « L'étendue  de  l'auditoire  conditionne  dans  une  certaine  mesure  les  procédés 
argumentatifs, et cela indépendamment des considérations relatives aux accords sur lesquels on se 
base,  et  qui  diffèrent  suivant  les  auditoires.  Étudiant  le  style  en  fonction  des  circonstances  où 
s'exerce la parole, J. Marouzeau signale : l'espèce de déférence et de respect humain qu'impose le 
nombre  à  mesure  que  diminue  l'intimité,  le  scrupule  augmente,  scrupule  d'être  bien  jugé,  de 
recueillir l'applaudissement ou du moins l'assentiment des regards et des attitudes... (1). » 
\bigskip
(1) Marouzeau, Précis de Stylistique française, P. 208. 
\bigskip
P  34 :  « Bien  d'autres  réflexions  relatives  à  des  particularités  des  auditoires  qui  influencent  le 
comportement et l'argumentation de l'orateur pourraient être développées avec pertinence. C'est, 
pensons-nous,  en  se  basant  sur  la  considération  des  auditoires  sous  leur  aspect  concret, 
particulier,  multiforme,  que  notre  étude  sera  féconde.  Toutefois,  nous  voudrions  surtout  nous 
étendre  dans  les  quatre  paragraphes  qui  suivent  sur  les  caractères  de  quelques  auditoires  dont 
l'importance est indéniable pour tous, mais spécialement pour le philosophe. » 
\bigskip
§ 6. PERSUADER ET CONVAINCRE 
\bigskip
P  34 :  « Les  pages  qui  précèdent  montrent  à  suffisance  que  la  variété  des  auditoires  est  quasi 
infinie  et  que,  à  vouloir  s'adapter  à  toutes  leurs  particularités,  l'orateur  se  trouve  confronté  avec 
des problèmes innombrables. Peut-être est-ce une  des raisons  pour lesquelles ce qui suscite par-
dessus  tout  l'intérêt,  c'est  une  technique  argumentative  qui  s'imposerait  à  tous  les  auditoires 
indifféremment  ou,  du  moins,  à  tous  les  auditoires  composes  d'hommes  compétents  ou 
raisonnables. La recherche d'une objectivité, quelle que soit sa nature, correspond à cet idéal, à ce 
désir  de  transcender  les  particularités  historiques  ou  locales  de  façon  que  les  thèses  défendues 
puissent  être  admises  par  tous.  A  ce  titre,  comme  le  dit  Husserl,  dans  l'émouvant  discours  où  il 
défend  l'effort  de  rationalité  occidental  :  «Nous  sommes,  dans  notre  travail  philosophique,  des 
fonctionnaires  de  l'humanité  (1).  »  C'est  dans  le  même  esprit  que  J.  Benda  accuse  les  clercs  de 
trahir  quand  ils  abandonnent  le  souci  de  l'éternel  et  de  l'universel,  pour  défendre  des  valeurs 
temporelles  et  locales  (2).  En  fait,  nous  assistons  ici  à  la  reprise  du  débat  séculaire  entre  les 
partisans  de  la  vérité  et  ceux  de  l'opinion,  entre  philosophes,  chercheurs  d'absolu  et  rhéteurs, 
engagés  dans  l'action.  C'est  à  l'occasion  de  ce  débat  que  semble  s'élaborer  la  distinction  entre 
persuader  et  convaincre,  que  nous  voudrions  reprendre  en  fonction  d'une  théorie  de 
l'argumentation et du rôle joué par certains auditoires (3). » 
\bigskip
\bigskip
\bigskip
16 
\bigskip
 
(1) E. Husserl, La crise des sciences 
(2) J. Benda, La trahison des clercs, 1928. 
(3) Cf. Ch. Perelman et L. Olbrechts-Tyteca, Rhétorique et philosophie, pp. 3 et suiv. (Logique et 
rhétorique). 
\bigskip
P 35 : « Pour qui se préoccupe du résultat, persuader est plus que convaincre, la conviction n'étant 
que le premier stade qui mène à l'action (4). Pour Rousseau, ce n'est rien de convaincre un enfant , 
si l'on ne sait le persuader » (5). 
\bigskip
Par  contre,  pour  qui  est  préoccupé  du  caractère  rationnel  de  l'adhésion,  convaincre  est  plus  que 
persuader.  Tantôt,  d'ailleurs,  ce  caractère  rationnel  de  la  conviction  tiendra  aux  moyens  utilisés, 
tantôt aux facultés auxquelles on s'adresse. Pour Pascal (6), c'est l'automate qu'on persuade, et il 
entend  par  là,  le  corps,  l'imagination,  le  sentiment,  bref  tout  ce  qui  n'est  point  la  raison.  Très 
souvent la persuasion sera considérée comme une transposition injustifiée de la démonstration : 
selon  Dumas  (7),  dans  la  persuasion  on  «  se  paie  de  raisons  affectives  et  personnelles  »,  la 
persuasion  étant  souvent  «  sophistique  ».  Mais,  il  ne  précise  pas  en  quoi  cette  preuve  affective 
différerait techniquement d'une preuve objective. » 
\bigskip
(4) Richard D.D. Whately, Elements of Rhetoric, Part II : Of Persuasion, chap. II : of persuasion 
chap. I, § 1, p. 115. Cf. Charles L. Stevenson, Ethics and Language. pp. 139-140. 
(5) Rousseau, Emile, liv. III, p. 203. 
(6) Pascal, Bibl. de la Pléiade, Pensées, 470 (195), p. 961 (252 éd. Brunschvieg). 
(7) G. Traité de psychologie, t. II, p. 740. 
\bigskip
P  36 :  « Les  critères  par  lesquels  on  croit  pouvoir  séparer  conviction  et  persuasion  sont  toujours 
fondés  sur  une  décision  qui  prétend  isoler  d'un  ensemble,  ensemble  de  procédés,  ensemble  de 
facultés,  certains  éléments  que  l'on  considère  comme  rationnels.  Il  est  à  souligner  que  cet 
isolement  porte  parfois  sur  les  raisonnements  eux-mêmes;  par  exemple,  on  montrera  que  tel 
syllogisme, tout en entraînant la conviction, n'entraînera point la persuasion : mais parler ainsi de 
ce  syllogisme,  c'est  l'isoler  de  tout  un  contexte,  c'est  supposer  que  ses  prémisses  existent  dans 
l'esprit  indépendamment  du  reste,  c'est  les  transformer  en  vérités  inébranlables,  intangibles.  On 
nous dira, par exemple, que telle personne, convaincue du danger de mastiquer trop rapidement, 
ne  cessera  point  pour  autant  de  le  faire  (1)  ;  c'est  que  l'on  isole  le  raisonnement  sur  lequel  cette 
conviction  est  basée  de  tout  un  ensemble.  On  oublie,  par  exemple,  que  cette  conviction  peut  se 
heurter à une autre conviction, celle qui nous affirme qu'il y a gain de temps à manger plus vite. On 
voit  donc  que  la  conception  de  ce  qui  constitue  la  conviction,  qui  peut  sembler  basée  sur  une 
différenciation  des  moyens  de  preuve  ou  des  facultés  mises  en  jeu,  l'est  souvent  aussi  sur 
l'isolement de certaines données au sein d'un ensemble beaucoup plus complexe. 
\bigskip
Si l'on se refuse, comme nous le faisons, à adopter ces distinctions au sein d'une pensée vivante, il 
faut néanmoins reconnaître que notre langage utilise deux notions, convaincre et persuader, entre 
lesquelles on estime généralement qu'il existe une nuance saisissable. » 
\bigskip
(1) W. Dill Scott, Influencing men in business, p. 32. 
\bigskip
P  36-37  :  «  Nous  nous  proposons  d'appeler  persuasive  une  argumentation  qui  ne  prétend  valoir 
que pour un auditoire particulier et d'appeler convaincante celle qui est censée obtenir l'adhésion 
de  tout  être  de  raison.  La  nuance  est  assez  délicate  et  dépend,  essentiellement,  de  l'idée  que 
l'orateur  se  fait  dé  l'incarnation  de  la  raison.  Chaque  homme  croit  en  un  ensemble  de  faits,  de 
vérités, que tout homme « normal » doit, selon lui, admettre, parce qu'ils sont valables pour tout 
être raisonnable. Mais en est-il vraiment ainsi ? Cette prétention à une validité absolue pour tout 
auditoire  composé  d'êtres  raisonnables  n'est-elle  pas  exorbitante  ?  Même  l'auteur  le  plus 
consciencieux  ne  peut,  sur  ce  point,  que  se  soumettre  à  l'épreuve  des  faits,  au  jugement  de  ses 
lecteurs  (1).  Il  aura,  en  tout  cas,  fait  ce  qui  dépend  de  lui  pour  convaincre,  s'il  croit  s'adresser 
valablement à pareil auditoire. » 
\bigskip
(1) Cf. Kant, Critique de la raison pure, p. 9. 
\bigskip
P 37 : « Nous préférons notre critère à celui, assez proche dans ses conséquences quoique différent 
dans son principe, qui a été proposé par Kant dans sa Critique de la raison Pure. La conviction et 
la persuasion sont, pour Kant, deux espèces de croyance : 
\bigskip
« Quand  elle  est  valable  pour  chacun,  en  tant  du  moins  qu'il  a  de  la  raison,  son  principe  est 
objectivement suffisant et la croyance se nomme conviction. Si elle n'a son fondement que dans la 
nature particulière du sujet, elle se nomme Persuasion. La persuasion est une simple apparence, 
parce que le principe du jugement qui est uniquement dans le sujet est tenu pour objectif. Aussi 
un  jugement  de  ce  genre  n'a-t-il  qu'une  valeur  individuelle  et  la  croyance  ne  peut-elle  pas  se 
communiquer... Donc la persuasion ne peut pas, à la vérité, être distinguée subjectivement de la 
conviction,  si  le  sujet  ne  se  représente  la  croyance  que  comme  un  simple  phénomène  de  son 
propre  esprit;  mais  l'essai  que  l'on  fait  sur  l'entendement  des  autres,  des  principes  qui  sont 
valables  pour  nous,  afin  de  voir  s'ils  produisent  exactement  sur  une  raison  étrangère  le  même 
effet que sur la nôtre est un moyen qui, tout en étant seulement subjectif, sert non pas à produire 
la conviction, mais cependant à découvrir la valeur particulière du jugement, c'est-à-dire ce qui 
n'est en lui que simple persuasion... je peux garder pour moi la persuasion, si je m'en trouve bien, 
mais je ne puis, ni ne dois la faire valoir hors de moi (2). » 
\bigskip
(2) Kant, Critique de la raison pure, pp. 634-635. 
\bigskip
P  37-38 :  « La  conception  kantienne,  quoique  par  ses  conséquences  assez  pro  e  de  la  nôtre,  en 
diffère  parce  qu'elle  fait  de  l'opposition  subjectif-objectif  le  critère  de  la  distinction  entre 
persuasion et conviction. Si la conviction est fondée sur la vérité de son objet, et par là valable pour 
tout  être  raisonnable,  elle  seule  peut  être  prouvée,  la  persuasion  ayant  une  portée  uniquement 
individuelle.  On  voit,  par  là,  que  Kant  n'admet  que  la  preuve  purement  logique,  l'argumentation 
non contraignante étant, par lui, exclue de la philosophie. Sa conception n'est défendable que dans 
la mesure où l'on admet que ce qui n'est pas nécessaire, n'est pas communicable, ce qui exclurait 
toute  argumentation  à  l'égard  d'auditoires  particuliers  :  or,  celle-ci  est  le  champ  d'élection  de  la 
rhétorique. A partir du moment où l'on admet qu'il existe d'autres moyens de preuve que la preuve 
nécessaire, l'argumentation à l'égard d'auditoires particuliers a une portée qui dépasse la croyance 
purement subjective. » 
\bigskip
P  38 :  « La  distinction  que  nous  proposons  entre  persuasion  et  conviction  rend  compte 
indirectement  du  lien  que  l'on  établit  souvent,  encore  que  confusément,  d'une  part  entre 
persuasion et action, d'autre part entre conviction et intelligence. Lu effet, le caractère intemporel 
de  certains  auditoires  explique  que  les  arguments  qui  leur  sont  destinés  ne  constituent  point  un 
appel à l'action immédiate. 
\bigskip
Cette  distinction,  fondée  sur  les  caractères  de  l'auditoire  auquel  on  s'adresse,  ne  semble  pas,  au 
premier abord, expliquer la distinction entre conviction et persuasion telle qu'elle est ressentie par 
l'auditeur lui-même. Mais il est aisé de voir que le même critère peut néanmoins s'appliquer, si l'on 
tient compte de ce que cet auditeur imagine le transfert à d'autres auditoires des arguments qu'on 
lui présente et se préoccupe de l'accueil qui leur serait réservé. » 
\bigskip
P 38-39 : « Notre point de vue permet de comprendre que la nuance entre les termes convaincre et 
persuader  soit  toujours  imprécise,  et  que,  en  pratique,  elle  doive  le  rester.  Car  tandis  que  les 
frontières  entre  l'intelligence  et  la  volonté,  entre  la  raison  et  l'irrationnel  peuvent  constituer  une 
\bigskip
\bigskip
\bigskip
18 
\bigskip
limite précise, la distinction entre divers auditoires est beaucoup plus incertaine, et cela d'autant 
plus  que  la  représentation  que  l'orateur  se  fait  des  auditoires  est  le  résultat  d'un  effort  toujours 
susceptible d'être repris. » 
\bigskip
P  39 :  « Notre  distinction  entre  persuader  et  convaincre  rejoint  donc  par  bien  des  traits,  des 
distinctions anciennes (1), même si elle n'en adopté pas les critères; elle explique aussi l'usage que 
d'aucuns  font,  par  modestie,  du  terme  persuasion  dans  son  opposition  avec  conviction.  Ainsi 
Claparède,  dans  l'avant-propos  à  l'un  de  ses  livres,  nous  dit  que  s'il  s'est  décidé  à  exhumer  son 
manuscrit,  "  c'est  à  la  demande  de  Mme  Antipoff  qui  m'a  persuadé  (mais non  convaincu)  qu'il  v 
aurait intérêt à publier ces recherches » (2). L'auteur, ici, ne songe point à établir une distinction 
théorique entre les deux termes, mais il se sert de leur différence pour exprimer à la fois le peu de 
valeur  objective  assurée  et  la  force  des  raisons  données  par  sa  collaboratrice  :  la  nuance  dont  se 
sert Claparède peut correspondre à la conception kantienne ; elle répond beaucoup mieux encore, 
semble-t-il,  au  fait  qu'il  s'agit  de  raisons  probantes  pour  lui,  mais  dont  il  conçoit  qu'elles  ne  le 
soient point pour tout le monde. » 
\bigskip
(1) Cf. notamment Fénelon, Dialogues sur l'éloquence, éd. Lebel. t. XXI, p. 43.   
(2) Ed. Clarapède, La genèse de l'hypothèse, Avant-propos. 
\bigskip
P  39-40 :  « C'est  donc  la  nature  de  l'auditoire  auquel  des  arguments  peuvent  être  soumis  avec 
succès  qui  détermine  dans  une  large  mesure  et  l'aspect  que  prendront  les  argumentations  et  le 
caractère,  la  portée  qu'on  leur  attribuera.  Comment  se  représentera-t-on  les  auditoires  auxquels 
est dévolu le rôle normatif permettant de décider du caractère convaincant d'une argumentation ? 
Nous  trouvons  trois  espèces  d'auditoires,  considérés  comme  privilégiés  à  cet  égard,  tant  dans  la 
pratique  courante  que  dans  la  pensée  philosophique.  Le  premier,  constitué  par  l'humanité  tout 
entière, ou du moins par tous les hommes adultes et normaux et que nous appellerons l'auditoire 
universel  -,  le  second  formé,  dans  le  dialogue,  par  le  seul  interlocuteur  auquel  on  s'adresse;  le 
troisième, enfin, constitué par le sujet lui-même, quand il délibère ou se représente les raison,-, de 
ses actes. Disons immédiatement que c'est seulement lorsque l'homme aux prises avec lui-même et 
l'interlocuteur  du  dialogue  sont  considérés  comme  incarnation  de  l'auditoire  universel,  qu'ils 
acquièrent  le  privilège  philosophique  confié  à  la  raison,  en  vertu  duquel  l'argumentation  qui 
s'adresse  à  eux  a  souvent  été  assimilée  à  un  discours  logique.  En  effet,  si  l'auditoire universel  de 
chaque orateur peut être considéré, d'un point de vue extérieur, comme un auditoire particulier, il 
n'en  reste  pas  moins  que,  à  chaque  instant  et  pour  chacun,  il  existe  un  auditoire  qui  transcende 
tous les autres, et qu'il est malaisé de cerner comme auditoire particulier. Par contre, l'individu qui 
délibère  ou  l'interlocuteur  du  dialogue  peuvent  être  perçus  comme  un  auditoire  particulier,  dont 
nous  connaissons  les  réactions,  dont  nous  sommes  à  même  tout  au  moins  d'étudier  les 
caractéristiques.  D'où  l'importance  primordiale  de  l'auditoire  universel  en  tant  que  norme  de 
l'argumentation  objective,  le  partenaire  du  dialogue  et  l'individu  délibérant  avec  lui-même  n'en 
étant que des incarnations toujours précaires. » 
\bigskip
§ 7. L'AUDITOIRE UNIVERSEL 
\bigskip
P 40-41 : « Toute argumentation qui vise seulement un auditoire particulier offre un inconvénient, 
c'est que l'orateur, dans la mesure précisément où il s'adapte aux vues de ses auditeurs, risque de 
prendre  appui  sur  des  thèses  qui  sont  étrangères  on  même  franchement  opposées  à  ce 
qu'admettent d'autres personnes que celles auxquelles, pour le moment, il s'adresse. Ce danger est 
apparent quand il s'agit d'un auditoire composite, que l'orateur doit décomposer pour les besoins 
de  son  argumentation.  En  effet,  cet  auditoire,  telle  une  assemblée  parlementaire,  devra  se 
regrouper  en  un  tout  pour  prendre  une  décision,  et  rien  de  plus  facile,  pour  l'adversaire,  que  de 
retourner contre son prédécesseur imprudent tous les arguments dont il a fait usage à l'égard des 
diverses  parties  de  l'auditoire,  soit  en  les  opposant  les  uns  aux  autres  pour  montrer  leur 
incompatibilité, soit en les présentant à ceux auxquels ils n'étaient pas destinés. De là la faiblesse 
relative des arguments qui ne sont admis que par des auditoires particuliers et la valeur accordée 
\bigskip
\bigskip
\bigskip
19 
\bigskip
aux opinions qui jouissent d'une approbation unanime, et spécialement de la part de personnes ou 
de groupes qui s'entendent sur bien peu de choses. » 
\bigskip
P 41 : « Il va de soi que la valeur de cette unanimité dépend du nombre et de la qualité de ceux qui 
la  manifestent,  la  limite  étant  atteinte,  dans  ce  domaine,  par  l'accord  de  l'auditoire  universel.  Il 
s'agit  évidemment,  dans  ce  cas,  non  pas  d'un  fait  expérimentalement  éprouvé,  mais  d'une 
universalité et d'une unanimité que se représente l'orateur, de l'accord d'un auditoire qui devrait 
être universel, ceux qui n'y participent pas pouvant, pour des raisons légitimes, ne pas être pris en 
considération. 
\bigskip
Les philosophes prétendent toujours s'adresser à un pareil auditoire, non pas parce qu'ils espèrent 
obtenir  le  consentement  effectif  de  tous  les  hommes  -  ils  savent  très  bien  que,  seule,  une  petite 
minorité aura jamais l'occasion de connaître leurs écrits mais parce qu'ils croient que tous ceux qui 
comprendront  leurs  raisons  ne  pourront  qu'adhérer  à  leurs  conclusions.  L'accord  d'un  auditoire 
universel  n'est  donc  pas  une  question  de  fait,  mais  de  droit.  C'est  parce qu'on  affirme  ce  qui  est 
conforme à un fait objectif , ce qui constitue une assertion vraie et même nécessaire, que l'on table 
sur  l'adhésion  de  ceux  qui  se  soumettent  aux  données  de  l'expérience  ou  aux  lumières  de  la 
raison. » 
\bigskip
P 41-42 : « Une argumentation qui s'adresse à un auditoire universel doit convaincre le lecteur du 
caractère  contraignant  des  raisons  fournies,  de  leur  évidence,  de  leur  validité  intemporelle  et 
absolue, indépendante des contingences locales ou historiques. « La vérité, nous dit Kant, repose 
sur  l'accord  avec  l'objet  et,  par  conséquent,  par  rapport  à  cet  objet,  les  jugements  de  tout 
entendement  doivent être d'accord. » Toute croyance  objective peut se communiquer car elle est 
«valable pour la raison de tout homme ». Seule une telle assertion peut être  affirmée, c'est-à-dire 
exprimée " comme un jugement nécessairement valable pour chacun » (1). » 
\bigskip
(1) Kant, Critique de la raison pure, p. 635. 
\bigskip
P  42 :  « En  fait,  un  pareil  jugement  est  censé  s'imposer  à  tout  le  monde,  parce  que  l'orateur  lui-
même est convaincu  de ce qu'il ne saurait être mis en doute. Dumas a  décrit, en un langage très 
expressif, cette certitude cartésienne : 
\bigskip
« La  certitude  est  la  pleine  croyance,  qui  exclut  entièrement  le  doute;  elle  est  affirmation 
nécessaire  et  universelle;  c'est-à-dire  que  l'homme  certain  ne  se  représente  pas  la  possibilité  de 
préférer l'affirmation contraire et qu'il se représente son affirmation comme devant s'imposer à 
tous dans les mêmes circonstances. En somme elle est l'état où nous avons conscience de penser la 
vérité,  qui  est  justement  cette  contrainte  universelle,  cette  obligation  mentale  ;  la  subjectivité 
disparaît, l'homme pense comme intelligence, comme homme et non plus comme individu. L'état 
de  certitude  a  été  souvent  décrit  à  l'aide  de  métaphores  comme  la  lumière  et  la  clarté  ;  mai,-, 
l'illumination de la certitude rationnelle apporte son explication. Il est repos et détente, même si 
la  certitude  est  pénible,  car  elle  met  fin  à  la  tension  et  à  l'inquiétude  de  la  recherche  et  de 
l'indécision. Il s'accompagne d'un sentiment de puissance et en  -même temps d'anéantissement; 
on  sent  que  la  prévention,  la  passion,  le  caprice  individuel  ont  disparu.  ...  Dans  la  croyance 
rationnelle, la vérité devient nôtre et nous devenons la vérité (2). » 
\bigskip
On  remarque que,  là  où  s'insère  l'évidence  rationnelle,  l'adhésion  de  l'esprit  semble  suspendue  à 
une vérité contraignante et les procédés d'argumentation ne jouent aucun rôle. L'individu, avec sa 
liberté  de  délibération  et  de  choix,  s'efface  devant  la  raison  qui  le  contraint  et  lui  enlève  toute 
possibilité  de  doute.  A  la  limite,  la  rhétorique  efficace  pour  un  auditoire universel  serait  celle  ne 
maniant que la preuve logique. » 
\bigskip
(2) G. Dumas, Traité de psychologie, t. 11, pp. 197-198, 200. 
\bigskip
\bigskip
\bigskip
20 
\bigskip
 
P  42-43 :  « Le  rationalisme,  avec  ses  prétentions  d'éliminer  toute  rhétorique  de  la  philosophie, 
avait  énoncé  un  programme  très  ambitieux  qui  devait  amener  l'accord  des  esprits  grâce  à 
l'évidence  rationnelle  s'imposant  à  tout  le  monde.  Mais  à  peine  les  exigences  de  la  méthode 
cartésienne  étaient-elles  énoncées  que  déjà  Descartes  avançait,  en  leur  nom,  des  assertions  fort 
contestables.  En  effet,  comment  distinguer  les  vraies  des  fausses  évidences  ?  Est-ce  qu'on 
s'imagine  que  ce  qui  convainc  un  auditoire  universel,  dont  on  se  considère  soi-même  comme  le 
représentant idéal, possède vraiment cette validité objective ? Pareto a excellemment remarqué en 
des  pages  pénétrantes  (1),  que  le  consentement  universel  invoqué  n'est  bien  souvent  que  la 
généralisation  illégitime  d'une  intuition  particulière.  C'est  la  raison  pour  laquelle  il  est  toujours 
aventureux d'identifier avec la logique l'argumentation à l'usage de l'auditoire universel, tel qu'on 
l'a conçu soi-même. Les conceptions que les hommes se sont données au cours de l'histoire, des « 
faits  objectifs  »  ou  des  «  vérités  évidentes  »  ont  suffisamment  varié  pour  que  l'on  se  montre 
méfiant à cet égard. Au lieu de croire à l'existence d'un auditoire universel, analogue à l'esprit divin 
qui ne peut donner son consentement qu'à.« la vérité », on pourrait, à plus juste titre, caractériser 
chaque orateur par l'image qu'il se forme lui-même de l'auditoire universel qu'il cherche à gagner à 
ses propres vues. » 
\bigskip
(1) V. Pareto, Traité de sociologie générale, t. I, chap. IV, §§ 589 et 599. 
\bigskip
P 43 : « L'auditoire universel est constitué par chacun à partir de ce qu'il sait de ses semblables, de 
manière à transcender les quelques oppositions dont il a conscience. Ainsi chaque culture, chaque 
individu  a  sa  propre  conception  de  l'auditoire  universel,  et  l'étude  de  ces  variations  serait  fort 
instructive, car elle nous ferait connaître ce que les hommes ont considéré, au cours de l'histoire, 
comme réel, vrai et objectivement valable. » 
\bigskip
P  43-44 :  « Si  l'argumentation  adressée  à  l'auditoire  universel  et  qui  devrait  convaincre,  ne 
convainc  pourtant pas tout le monde, il reste toujours la ressource de  disqualifier le récalcitrant 
en le considérant comme stupide ou anormal. Cette façon de procéder, fréquente chez les penseurs 
du moyen âge, se retrouve également chez les modernes (1). Une telle exclusion de la communauté 
humaine  ne  peut  obtenir  l'adhésion  que  si  le  nombre  et  la  valeur  intellectuelle  des  proscrits  ne 
menacent  pas  de  rendre  ridicule  pareille  procédure.  Si  ce  danger  existe,  on  doit  recourir  à  une 
autre  argumentation,  et  opposer  à  l'auditoire  universel  un  auditoire  d'élite,  doué  de  moyens  de 
connaissance  exceptionnels  et  infaillibles.  Ceux  qui  se  targuent  d'une  révélation  surnaturelle  ou 
d'un  savoir  mystique,  ceux  qui  font  appel  aux  bons,  aux  croyants,  aux  hommes  qui  ont  la  grâce, 
manifestent leur préférence pour un auditoire d'élite ; cet auditoire d'élite peut même se confondre 
avec l'Etre parfait. » 
\bigskip
(1)  Notamment  H.  Lefebvre,  A  la  lumière  du  matérialisme  dialectique,  I,  Logique  formelle, 
logique dialectique, p. 29. 
\bigskip
P  44 :  « L'auditoire  d'élite  n'est  pas  toujours,  tant  s'en  faut,  considéré  comme  assimilable  à 
l'auditoire  universel.  En  effet,  il  arrive  souvent  que  l'auditoire  d'élite  veuille  rester  distinct  du 
commun  des  hommes:  l'élite  dans  ce  cas,  est  caractérisée  par  sa  situation  hiérarchique.  Mais 
souvent  aussi,  l'auditoire  d'élite  est  considéré  comme  le  modèle  auquel  doivent  se  conformer  les 
hommes pour être dignes de ce nom : l'auditoire d'élite crée la norme pour tout le monde. Dans ce 
cas l'élite est l'avant-garde que tous suivront, et à laquelle ils se conformeront. Son opinion seule 
importe, parce que c'est, en fin de compte, celle qui sera déterminante. 
\bigskip
L'auditoire  d'élite  n'incarne  l'auditoire  universel  que  pour  ceux  qui  lui  reconnaissent  ce  rôle 
d'avant-garde  et  de  modèle.  Pour  les  autres,  au  contraire,  il  ne  constituera  qu'un  auditoire 
particulier. Le statut d'un auditoire varie selon les conceptions que l'on entretient. » 
\bigskip
\bigskip
\bigskip
\bigskip
21 
\bigskip
P  44-45 :  « Certains  auditoires  spécialisés  sont  volontiers  assimilés  à  l'auditoire  universel,  tel 
l'auditoire  du  savant  s'adressant  à  ses  pairs.  Le  savant  s'adresse  à  certains  hommes 
particulièrement compétents et qui admettent les données d'un système bien défini, constitué par 
la  science  dans  laquelle  ils  sont  spécialisés.  Pourtant,  cet  auditoire  si  limité  est  généralement 
considéré  par  le  savant  non  comme  un  auditoire  particulier,  mais  comme  étant  vraiment 
l'auditoire  universel  :  il  suppose  que  tous  les  hommes,  avec  le  même  entraînement,  la  même 
compétence et la même information, adopteraient les mêmes conclusions. » 
\bigskip
P 45 : « Il en est de même quand il s'agit de morale. Nous nous attendons à ce que nos jugements 
soient confirmés par les réactions des autres. Les « autres » auxquels nous faisons ainsi appel ne 
sont  pas  cependant  n'importe  quels  «  autres  ».  Nous  faisons  appel  seulement  à  ceux  qui  ont 
dûment « réfléchi » sur la conduite que nous approuvons ou nous désapprouvons. Comme le dit 
Findlay : 
\bigskip
Nous  en  appelons  par-dessus  les  têtes  irréfléchies  de  la  «  compagnie  présente  »  à  la  «  grande 
compagnie des personnes réfléchies » où qu'elle puisse être située dans l'espace ou le temps (1). 
\bigskip
Un pareil appel est critiqué par J.-P. Sartre dans ses remarquables conférences sur l'auditoire de 
l'écrivain : 
\bigskip
Nous avons dit que l'écrivain s'adressait en principe à tous les hommes. Mais, tout de suite après, 
nous avons remarqué qu'il était lu seulement de quelques-uns. De l'écart entre le publie idéal et le 
public  réel  est  née  l'idée  d'universalité  abstraite.  C'est-à-dire  que  auteur  postule  la  perpétuelle 
répétition  dans  un  futur  indéfini  de  la  poignée  de  lecteurs,  dont  il  dispose  dans  le  présent.  ...  le 
recours  à  l'infinité  du  temps  cherche  à  compenser  l'échec  dans  l'espace  (retour  à  l'infini  de 
l'honnête homme de l'auteur du XVIIe siècle, extension à l'infini du club des écrivains et du public 
de  spécialistes  pour  celui  du  XIXe  siècle).  ...  Par  l'universalité  concrète,  il  faut  entendre  au 
contraire la totalité des hommes vivant dans une société donnée (2). » 
\bigskip
(1) J. N. Findlay, Morality by Convention,  Mind, vol. LIII, p. 160. Cf.  Arthur N. Prior,  Logic and 
the basis of ethics, p. 84. 
(2) Jean-Paul Sartre, Situations, II, pp. 192-193. 
\bigskip
P  45-46 :  « Sartre  reproche  aux  écrivains  de  négliger  l'universalité  concrète  à  laquelle  ils 
pourraient,  et  devraient,  s'adresser  pour  se  contenter  de  l'illusoire  universalité  abstraite.  Mais 
n'est-ce  point  l'auditoire  universel  de  Sartre  qui  devra  juger  du  bien-fondé  de  cette  critique,  qui 
devra décider si, oui ou non, il y a eu jusqu'ici chez l'écrivain illusion volontaire ou involontaire, si 
l'écrivain  a  manqué  Jusqu'ici  à  ce  qu'il  s'était  assigné  «  comme  mission  »  ?  Et  cet  auditoire 
universel  de  Sartre,  c'est  celui  auquel  il  s'adresse  pour  exposer  ses vues mêmes  sur  l'universalité 
abstraite et concrète. » 
\bigskip
P  46 :  « Nous  croyons  donc  que  les  auditoires  lie  sont  point  indépendants  ;  que  ce  sont  des 
auditoires concrets particuliers qui peuvent faire valoir une conception de l'auditoire universel qui 
leur est propre; mais, par contre, c'est l'auditoire universel non défini qui est invoqué pour juger de 
la  conception  de  l'auditoire  universel  propre  à  tel  auditoire  concret,  pour  examiner,  à  la  fois,  la 
manière dont il a été composé, quels sont les individus qui, suivant le critère adopté, en font partie 
et quelle est la légitimité de ce critère. On peut dire que les auditoires se jugent les uns les autres. » 
\bigskip
§ 8. L'ARGUMENTATION DEVANT UN SEUL AUDITEUR 
\bigskip
P  46-47 :  « La  portée  philosophique  de  l'argumentation  présentée  à  un  seul  auditeur  et  sa 
supériorité sur celle adressée à un vaste auditoire a été admise par tous ceux qui, dans l'Antiquité, 
proclamaient  le  primat  de  la  dialectique  par  rapport  à  la  rhétorique.  Celle-ci  se  limitait  à  la 
technique  du  long  discours  continu.  Mais  pareil  discours,  avec  toute  l'action  oratoire  qu'il 
comporte, serait ridicule et inefficace devant un seul auditeur (1). Il est normal de tenir compte de 
ses  réactions,  de  ses  dénégations  et  de  ses  hésitations  et,  quand  on  les  constate,  il  n'est  pas 
question de s'esquiver : il faut prouver le point contesté, s'informer des raisons de la résistance de 
l'interlocuteur,  se  pénétrer  de  ses  objections  :  le  discours  dégénère  invariablement  en  dialogue. 
C'est pourquoi, selon Quintilien, la dialectique, comme technique du dialogue, était comparée par 
Zénon,  à  cause  du  caractère  plus  serré  de  l'argumentation,  à  un  poing  fermé,  alors  que  la 
rhétorique  lui  paraissait  semblable  à  une  main  ouverte (1).  Il  n'y  a  pas  de  doute,  en  effet,  que  la 
possibilité qui lui est offerte de poser des questions, de présenter des objections, donne à l'auditeur 
l'impression  que  les  thèses  auxquelles  il  adhère,  pour  finir,  sont  plus  solidement  étayées  que  les 
conclusions  de  l'orateur  qui  développe  un  discours  continu.  Le  dialecticien,  qui  se  préoccupe,  à 
chaque pas de son raisonnement, de l'accord de son interlocuteur, serait plus sûr, d'après Platon, 
de suivre le chemin de la vérité. Cette opinion est clairement exprimée dans ce petit discours que 
Socrate adresse à Calliclès : 
\bigskip
Voilà donc une question réglée; chaque fois que nous serons d'accord sur un point, ce point sera 
considéré  comme  suffisamment  éprouvé  de  part  et  d'autre,  salis  qu'il  y  ait  lieu  de  l'examiner  à 
nouveau. Tu ne pouvais en effet me l'accorder faute de science ni par excès de timidité, et tu ne 
saurais,  en  le  faisant,  vouloir  nie  tromper:  car  tu  es  mon  ami,  dis-tu.  Notre  accord,  par 
conséquent, prouvera réellement que nous aurons atteint la vérité (2). » 
\bigskip
(1) Quintillien, vol. I, liv. I, chap. II, 29; cf. aussi Dale Carnegie, L'art de parier en publie, p. 154, et 
la distinction de K. Riezler entre « one-way communication » et « two-way communication » dans 
Political decisions in modern Society, Ethics, janv. 1954, 2, II, pp. 45-46. 
(1) Quintillien, Vol. I, liv. II, chap. XX, 7 . 
(2) Platon, Gorgias, 487 d-e. 
\bigskip
P 47 : « Cette façon de transformer l'adhésion d'un seul en indice de vérité, serait ridicule, et c'est 
d'ailleurs  l'avis  de  Pareto  (3),  si  l'interlocuteur  de  Socrate  exprimait  un  point  de  vue  purement 
personnel.  Il  serait  peut-être  exagéré  de  dire,  avec  Goblot,  que  "  Platon  pense  être  sûr  qu'aucun 
interlocuteur ne pourrait répondre autrement que celui qu'il fait parler » (4), mais il est certain, en 
tout cas, que chaque interlocuteur de Socrate est le porte-parole, supposé le meilleur, des tenants 
d'un point de vue déterminé, et dont les objections doivent être écartées au préalable pour faciliter 
l'adhésion du publie aux thèses développées. » 
\bigskip
(3) V. Pareto, Traité de sociologie générale, t. I, § 612, p. 324. 
(4 E. Goblot, La logique des jugements de valeur, p. 17. 
\bigskip
P 47-48 : « Ce qui confère au dialogue, comme genre philosophique, et à la dialectique, telle que l'a 
conçue Platon, une portée éminente, ce n'est pas l'adhésion effective d'un interlocuteur déterminé 
- car celui-ci ne constitue qu'un auditoire particulier parmi une infinité d'autres  - mais l'adhésion 
d'un personnage qui, quel qu'il soit, ne peut que s'incliner devant l'évidence de la vérité, parce que 
sa  conviction  résulte  d'une  confrontation  serrée  de  sa  pensée  avec  celle  de  l'orateur.  Le  rapport 
entre  dialogue  et  vérité  est  tel  que  E.  Dupréel  incline  à  croire  que  Gorgias  n'a  pas  dû  pratiquer 
spontanément le dialogue : la prédilection pour le procédé du dialogue aurait été, croit-il, le propre 
d'un adversaire de la rhétorique, partisan du primat de la seule vérité, à savoir, Hippias d'Élis (1). » 
\bigskip
(1) Eugène Dupréel, Les Sophistes, pp. 76, 77, 260, 263. 
\bigskip
P  48 :  « Le  dialogue  écrit  suppose,  plus  encore  que  le  dialogue  effectif,  que  cet  auditeur  unique 
incarne l'auditoire universel. Et cette conception parait justifiée surtout lorsqu'on admet, comme 
Platon,  qu'il  existe  dans  l'homme  des  principes  internes  contraignants  qui  le  guident  dans  le 
développement de sa pensée (2). » 
\bigskip
\bigskip
\bigskip
\bigskip
23 
\bigskip
(2) Cf. Ch. Perelman, La méthode dialectique et le rôle de l'interlocuteur dans le dialogue, Revue 
de métaphysique et de morale, 1955, pp. 26 à 31. 
\bigskip
P  48-49 :  « L'argumentation  d'un  pareil  dialogue  n'a  de  signification  philosophique  que  si  elle 
prétend  valoir  aux  yeux  de  tous.  On  comprend  aisément  que  la  dialectique,  tout  comme 
l'argumentation visant l'auditoire universel, ait été identifiée avec la logique. Cette conception est 
celle des stoïciens et du moyen âge (3). Nous ne voyons en elle qu'une illusion, ou un procédé, dont 
l'importance  pourtant  a  été  indéniable  dans  le  développement  de  la  philosophie  absolutiste, 
cherchant  par  tous  les  moyens  de  passer  de  l'adhésion  à  la  vérité.  L'adhésion  de  l'interlocuteur 
dans  le  dialogue  tire  sa  signification  philosophique  de  ce  que  celui-ci  est  considéré  comme  une 
incarnation  de  l'auditoire  universel.  On  admet  que  l'auditeur  dispose  des  mêmes  ressources  de 
raisonnement que les autres membres de l'auditoire universel, les éléments d'appréciation relatifs 
à  la  seule  compétence  technique  étant  fournis  par  l'orateur  ou  présumés  être  largement  à  la 
disposition de l'auditeur de par sa situation sociale. » 
\bigskip
(3) Cf. Karl Dürr, Die Entwicklung der Dialektik von Platon bis Hegel, Dialectica, vol. 1, n° 1, 1947 ; 
Richard Mc Keon, Dialectic and political thought and action, Ethics, oct. 1954. 
\bigskip
P  49-50 :  « Il  ne  faudrait  pas,  cependant,  que  l'adhésion  de  l'interlocuteur  ait  été  obtenue 
uniquement  grâce  à  la  supériorité  dialectique  de  l'orateur.  Celui  qui  cède  ne  doit  pas  avoir  été 
vaincu  dans  une  joute  éristique,  mais  est  censé  s'être  incliné  devant  l'évidence  de  la  vérité.  C'est 
que  le  dialogue,  tel  qu'il  est  envisagé  ici,  ne  doit  pas  constituer  un  débat,  où  des  convictions 
établies  et  opposées  sont  défendues  par  leurs  partisans  respectifs,  mais  une  discussion,  où  les 
interlocuteurs  recherchent  honnêtement  et  sans  parti  pris  la  meilleure  solution  d'un  problème 
controversé.  En  opposant  au  point  de  vue  éristique,  le  point  de  vue  heuristique,  certains  auteurs 
contemporains  présentent la discussion comme l'instrument idéal pour arriver à des conclusions 
objectivement  valables  (1).  On  suppose  que,  dans  la  discussion,  les  interlocuteurs  ne  se 
préoccupent que de faire connaître et d'éprouver tous les arguments, pour ou contre, concernant 
les  diverses  thèses  en  présence.  La  discussion,  menée  à  bonne  fin,  devrait  conduire  à  une 
conclusion  inévitable  et  unanimement  admise,  si  les  arguments,  présumés  de  même  poids  pour 
tous,  sont  disposés  comme  sur  les  plateaux  d'une  balance.  Dans  le  débat,  par  contre,  chaque 
interlocuteur  n'avancerait  que  des  arguments  favorables  à  sa  thèse  et  ne  se  préoccuperait  des 
arguments qui lui sont défavorables que pour les réfuter ou limiter leur portée. L'homme de parti 
pris est donc partial et parce qu'il a pris parti et parce qu'il ne peut plus faire valoir que la partie 
des  arguments  Pertinents  qui  lui  est  favorable,  les  autres  restant,  pour  ainsi  dire,  gelés  et 
n'apparaissant dans le débat que si l'adversaire les avance. Comme ce dernier est supposé adopter 
la même attitude, on comprend que la discussion soit présentée comme une recherche sincère de 
la vérité, alors que, dans le débat, l'on se préoccupe surtout du triomphe de sa propre thèse. » 
\bigskip
(1) Cf. A. C. Baird, Argumentation, Discussion and Débate, p. 307. 
\bigskip
P  50 :  « Si,  idéalement,  la  distinction  est  utile,  elle  ne  permet  pourtant  que  moyennant  une 
généralisation  bien  audacieuse  de  considérer  les  participants  à  une  discussion  désintéressée 
comme  des  porte-parole  de  l'auditoire  universel,  et  ce  n'est  qu'en  vertu  d'une  vision  assez 
schématique  de  la  réalité  que  l'on  pourrait  assimiler  la  détermination  du  poids  des  arguments  à 
une pesée de lingots. D'autre part, celui qui défend un point de vue déterminé est, bien souvent, 
convaincu qu'il s'agit d'une thèse qui est objectivement la meilleure et que son triomphe est celui 
de la bonne cause. 
\bigskip
Par  ailleurs,  pratiquement,  cette  distinction  entre  la  discussion  et  le  débat  semble  en  maintes 
rencontres difficile à préciser. En effet, dans la plupart des cas, elle repose sur l'intention que nous 
prêtons, à tort ou à raison, aux participants du dialogue, intention qui, elle-même, peut varier au 
cours  de  celui-ci.  C'est  seulement  dans  les  cas  privilégiés  où  l'attitude  des  participants  est  réglée 
par  les  institutions,  que  nous  pouvons  connaître  d'avance  leurs  intentions  :  dans  la  procédure 
judiciaire,  nous  savons  que  l'avocat  de  chaque  partie  tend  moins  à  s'éclairer  qu'à  développer  des 
arguments  en  faveur  d'une  thèse.  En  fixant  les  points  à  débattre,  le,  droit  favorise  cette  attitude 
unilatérale, ces prises de position, que le plaideur n'a plus qu'à soutenir avec constance contre son 
adversaire.  Dans  maints  autres  cas,  les  institutions  interviennent  d'une  manière  plus  discrète, 
quoique  effective  :  lorsqu'un  récipiendaire  défend  une  thèse  contre  les  membres  du  jury  qui  la' 
critiquent,  lorsqu'un  membre  du  Parlement  défend  le  programme  de,  son  parti.  Enfin,  cette 
attitude peut résulter d'engagements pris par l'orateur: si celui-ci a promis à quelqu'un de défendre 
sa candidature devant une  commission compétente, le  dialogue qu'il po suivra avec les membres 
de cette commission sera, en fait, plus plaidoyer qu'une recherche de la vérité  - en l'occurrence la 
détermination du meilleur candidat. » 
\bigskip
P  51 :  « On  voit  que,  sauf  lorsque  nous  savons  pour  quelle  raison  institutionnelle  ou  autre  - 
l'attitude des participants est celle du plaidoyer et, par voie de conséquence, implique le désir de 
mettre l'adversaire dans l'embarras, la distinction nette entre un dialogue qui tend à la vérité et un 
dialogue  qui  serait  une  succession  de  plaidoyers,  est  difficile  à  maintenir.  Elle  ne  pourrait  se 
soutenir que moyennant une distinction, préalable et certaine, entre la vérité et l'erreur, distinction 
que, sauf preuve de mauvaise foi, l'existence même de la discussion rend malaisée à établir. 
\bigskip
Le dialogue heuristique où l'interlocuteur est une incarnation de l'auditoire universel, le dialogue 
éristique  qui  aurait  pour  but  de  dominer  l'adversaire,  ne  sont,  l'un  et  l'autre,  que  des  cas 
exceptionnels  ;  dans  le  dialogue  habituel,  les  participants  tendent  tout  simplement  à  persuader 
leur auditoire en vue de déterminer une action immédiate ou future: c'est sur ce plan pratique que 
se  développent  la  plupart  de  nos  dialogues  journaliers.  Il  est  d'ailleurs  curieux  de  souligner  que 
cette  activité  journalière  de  discussion  persuasive  est  celle  qui  a  le  moins  attiré  l'attention  des 
théoriciens  :  la  plupart  des  auteurs  de  traités  de  rhétorique  la  considéraient  comme  étrangère  à 
leur  discipline.  Les  philosophes  qui  s'occupaient  du  dialogue  l'envisageaient  généralement  sous 
son aspect privilégié où l'interlocuteur incarne l'auditoire universel ; ou bien encore sous l'aspect 
plus psychologique, mais aussi plus scolaire, du dialogue éristique, dominé par le souci de ce que 
Schopenhauer (1) appelle « Rechthaberei ». A. Reyes a noté avec raison (2) que le discours privé 
constitue un terrain contigu à celui de l'ancienne rhétorique; en fait, c'est au cours des entretiens 
quotidiens que l'argumentation a le plus d'occasion de s'exercer. » 
\bigskip
(1) Schopenhauer, éd. Piper, vol. 6 : Eristische Dialektik, p. 394.  
(2) A. Reyes, EI Deslinde, p. 203. 
\bigskip
P 51-52 : « Ajoutons que, même lorsque l'auditeur unique, que ce soit l'auditeur actif du dialogue 
ou  un  auditeur  silencieux  auquel  l'orateur  s'adresse,  est  considéré  comme  l'incarnation  d'un 
auditoire,  ce  n'est  pas  toujours  de  l'auditoire  universel.  Il  peut  aussi  -  et  très  souvent  -être 
l'incarnation d'un auditoire particulier. » 
\bigskip
P  52 :  « Cela  est  vrai  évidemment  lorsque  l'auditeur  unique  représente  un  groupe  dont  il  est  le 
délégué,  le  porte-parole,  au  nom  duquel  il  pourra  prendre  des  décisions.  Mais  c'est  aussi  le  cas 
lorsque l'auditeur est considéré comme un échantillon de tout un genre d'auditeurs. Le professeur 
pourra  choisir,  pour  s'adresser  à  lui, l'étudiant  qui  lui  paraît  le  moins  doué,  ou  l'étudiant  le  plus 
intelligent, ou l'étudiant le moins bien placé pour l'entendre. 
\bigskip
Le  choix  de  l'auditeur  unique  qui  incarnera  l'auditoire  est  déterminé  par  les  buts  que  s'assigne 
l'orateur, mais aussi par l'idée qu'il se f ait de la manière dont un groupe doit être caractérisé. Le 
choix  de  l'individu  qui  incarne  un  auditoire  particulier  influence  souvent  les  procédés  de 
l'argumentation. Si Bentham (1) approuve l'usage suivi aux Communes de s'adresser au président, 
c'est  pour  rendre  les  débats  aussi  courtois  que  possible.  L'auditeur unique  est  dans  ce  cas  choisi 
\bigskip
\bigskip
\bigskip
25 
\bigskip
non pour ses qualités, mais pour ses fonctions ; c'est le choix qui engage le moins l'orateur et qui 
révèle le moins l'opinion qu'il a de son auditoire. » 
\bigskip
(1) Bentham, t. I : Tactique des assemblées politiques délibérantes, p. 391. 
\bigskip
P 52-53 : « Il n'en va pas de même dans les autres choix : l'individu choisi pour incarner l'auditoire 
particulier  auquel  on  s'adresse  révèle,  d'une  part,  l'idée  que  l'on  se  fait  de  cet  auditoire,  d'autre 
part, les buts que l'on espère atteindre. Ronsard s'adressant à Hélène voit en elle l'incarnation de 
tous les jeunes pour qui s'entend le conseil « Cueillez dés aujourdhuy les roses de la vie » (2). Mais, 
adressé à Hélène, ce conseil perd toute prétention didactique pour n'être plus que le reflet d'une 
émotion, d'une sympathie, voire d'un espoir. Cette technique, nous la retrouvons, tout au long de 
l'histoire littéraire et politique. Bien rare est le discours publié dont le destinataire individualisé ne 
doive être considéré comme l'incarnation d'un auditoire particulier déterminé. » 
\bigskip
(2) Ronsard, Bibl. de la Pléiade, vol. I : Sonnets pour Hélène, liv. II, XLIII, p. 260. 
\bigskip
§ 9. LA DELIBERATION AVEC SOI-MEME 
\bigskip
P  53 :  « Le  sujet  qui  délibère  est  considéré  souvent  comme  une  incarnation  de  l'auditoire 
universel. » 
\bigskip
P 53-54 : « En effet, il semble que l'homme doué de raison , qui cherche à se faire une conviction, 
ne peut que faire fi de tous les procédés visant à gagner autrui : il ne peut croit-on, qu'être sincère 
avec lui-même et est, plus que quiconque, à même d'éprouver la valeur de ses propres arguments. 
« Le consentement de vous-même à vous-même et la voix constante de votre raison » (1), est pour 
Pascal le meilleur critère de vérité ; c'est aussi celui dont se sert Descartes, dans ses  Méditations 
(2), pour passer des raisons qui l'ont convaincu lui-même à l'affirmation qu'il est «parvenu à une 
certaine  et  évidente  connaissance  de  la  vérité  ».  Par  opposition  à  la  dialectique,  qui  serait  la 
technique de la controverse avec autrui et à la rhétorique, technique du discours adressé au grand 
nombre,  la  logique  s'identifie,  pour  Schopenhauer  (3)  comme  pour  J.  S.  Mill  (4)  avec  les  règles 
appliquées  pour  conduire  sa  propre  pensée.  C'est  que,  dans  ce  dernier  cas,  l'esprit  ne  se 
préoccuperait  pas  de  plaider,  de  chercher  uniquement  des  arguments  qui  favorisent  un  point  de 
vue déterminé, mais de réunir tous ceux qui présentent quelque valeur à ses yeux, sans devoir en 
taire aucun et, après avoir pesé le pour et le contre, de se  décider, en âme et conscience, pour la 
solution  qui  lui  semble  la  meilleure.  De  même  que  l'on  n'accorde  pas  une  égale  importance  aux 
arguments développés en séance publique et à ceux qui sont présentés en comité secret, de même 
le secret de la ,délibération intime semble garant de la sincérité et de la valeur de cette dernière. 
Ainsi Chaignet, dans le dernier ouvrage en langue française qui ait considéré la rhétorique comme 
une technique de la persuasion, oppose celle-ci à la conviction dans les termes suivants : 
\bigskip
Quand nous sommes convaincus, nous ne sommes vaincus que par nous-même, par nos propres 
idées. Quand nous sommes persuadés, nous le sommes toujours par autrui (1). 
\bigskip
(1) Pascal, Bibl. de la Pléiade, Pensées, 249 (561), p. 891 (260 éd. Brunschvicg).  
(2) Cf. Préface de l'auteur au lecteur. 
(3) Schopenhauer éd. Brockhaus, vol. 3 : Die Weil als Wille und Vorstellung, 2.  Band, chap. IX, p. 
112. 
(4) J. S. Mill, Système de logique, vol. I. Introduction. p. 5. 
(1) A. Ed. Chaignet, La rhétorique et son histoire, p. 93.  
\bigskip
P 54 : « L'individualisme des auteurs qui accordent une nette prééminence à la façon de conduire 
nos propres pensées et la considèrent comme seule digne de l'intérêt du philosophe  - le  discours 
adressé  à  autrui  n'étant  qu'apparence  et  tromperie  -  a  été  pour  beaucoup  dans  le  discrédit  non 
seulement de la rhétorique, mais, en général, de toute théorie de l'argumentation. Il nous semble, 
par contre, qu'il y a tout intérêt à considérer la délibération intime comme une espèce particulière 
d'argumentation.  Tout  en  n'oubliant  pas  les  caractères  propres  à  la  délibération  intime,  nous 
pensons qu'il y a tout à gagner en ne négligeant pas cet avis d'Isocrate : 
\bigskip
Les  arguments  par  lesquels  nous  convainquons  les  autres  en  parlant,  sont  les  mêmes  que  ceux 
que nous utilisons lorsque nous réfléchissons; nous appelons orateurs ceux qui sont capables de 
parler devant la foule et nous considérons comme de bon conseil ceux lui peuvent sur les affaires 
s'entretenir avec eux-mêmes de la façon a plus judicieuse (2). 
\bigskip
Très  souvent  d'ailleurs,  une  discussion  avec  autrui  n'est  qu'un  moyen  que  nous  utilisons  pour 
mieux nous éclairer. L'accord avec soi-même n'est qu'un cas particulier de l'accord avec les autres. 
Aussi,  de  notre  point  de  vue,  c'est  l'analyse  de  l'argumentation  adressée  à  autrui  qui  nous  fera 
comprendre le mieux la délibération avec soi-même, et non l'inverse. » 
\bigskip
 (2) Isocrate, Discours, t. II. : Nicoclés, § 8. 
\bigskip
P  55-56 :  « En  effet,  ne  peut-on  pas  distinguer,  dans  la  délibération  intime,  une  réflexion  qui 
correspondrait à une discussion et une autre qui ne serait qu'une recherche d'arguments en faveur 
d'une position adoptée à l'avance ? Peut-on se fier entièrement à la sincérité du sujet qui délibère 
pour  nous    dire  s'il  est  en  quête  de  la  meilleure  ligne  de  conduite,  ou  s'il  élabore  un  plaidoyer 
intime  ?  La  psychologie  des  profondeurs  nous  a  appris  à  nous  méfier  même  de  ce  qui  paraît 
indubitable  à  notre  propre  conscience.  Mais  les  distinctions  qu'elle  établit  entre  raisons  et 
rationalisations  ne  peuvent  se  comprendre  si  l'on  ne  traite  pas  la  délibération  comme  un  cas 
particulier  d'argumentation.  Le  psychologue  dira  que  les  motifs  allégués  par  le  sujet,  pour 
expliquer  sa  conduite,  constituent  des  rationalisations,  s'ils  diffèrent  des  mobiles  réels  qui  l'ont 
déterminé à agir, et que le sujet ignore. Quant à nous, nous prendrons le terme de rationalisation 
dans un sens plus large sans nous attacher au fait que le sujet ignore, ou non, les véritables motifs 
de sa conduite. S'il semble ridicule, à première vue, qu'un être pondéré qui, après avoir agi pour 
des  motifs  très  « raisonnables »  s'efforce  de  donner,  dans  son  for  intérieur,  des  raisons  bien 
différentes à ses actes, moins vraisemblables, mais qui le placent dans une  plus belle lumière (1), 
une  pareille  rationalisation  s'explique  parfaitement  quand  on  la  considère  comme  un  plaidoyer 
anticipé  à  l'usage  des  autres,  qui  peut  d'ailleurs  être  spécialement  adapté  à  tel  ou  tel  auditeur 
présume. Cette rationalisation ne signifie nullement, comme le croit Schopenhauer (2) que notre « 
intellect » ne fait que camoufler les véritables motifs de nos actes qui seraient, eux, complètement 
irrationnels.  Il  se  peut  que  des  actes  aient  été  parfaitement  réfléchis,  et  qu'ils  aient  eu  d'autres 
raisons  que  celles  que  l'on  cherche,  après  coup,  à  faire  admettre  par  sa  conscience.  Ceux  qui  ne 
voient  pas,  ou  n'admettent  pas,  l'importance  de  l'argumentation  ne  peuvent  s'expliquer  la 
rationalisation qui ne serait pour eux que l'ombre d'une ombre. » 
\bigskip
(1) R. Crawshay-Williams, The comforts of unreason, pp. 74 et suiv. 
(2) Schopenhauer, éd. Brockhaus, vol. 6 : Parerga und Paralipomena, II, chap. VIII : « Zur Ethik 
»,  § 118, p. 249. 
\bigskip
P  56 :  «   Il  semble  qu'une  comparaison  avec  la  situation  ci-après,  décrite  par  J.  S.  Mill,  nous 
permettra d'en mieux apprécier la portée : 
\bigskip
Tout le monde, nous dit-il, connaît le conseil donné par lord Mansfield à un homme d'un très bon 
sens pratique qui, ayant été nommé gouverneur d'une colonie, avait, salis expérience des affaires 
judiciaires  et  sans  connaissances  de  droit,  à  présider  une  cour  de  justice.  Le  conseil  était  de 
donner sa décision résolument car elle serait probablement juste, mais de ne s'aventurer jamais 
à en exposer les raisons, car elles seraient presque infailliblement mauvaises (1). 
\bigskip
\bigskip
\bigskip
\bigskip
27 
\bigskip
En fait, si le conseil de lord Mansfield était bon, c'est parce que, après que le président aurait jugé 
en  équité,  ses  assesseurs  seuls  auraient  pu  «  rationaliser  »  son  verdict,  en  le  faisant  précéder 
d'attendus  ignorés  par  le  gouverneur,  mais  plus  conformes  à  la  législation  en  vigueur  que  les 
raisons  qui  auraient  motivé  sa  décision.  Il  arrive  d'ailleurs  très  souvent,  et  ce  n'est  pas 
nécessairement déplorable, que même un magistrat connaissant le droit, formule son jugement en 
deux temps, les  conclusions étant  d'abord inspirées par ce qui lui semble le plus conforme à son 
sens de l'équité, la motivation technique ne venant que par surcroît. Faut-il conclure dans ce cas, 
que la décision a été prise sans aucune délibération préalable ? Nullement, car le pour et le contre 
pouvaient avoir été pesés avec le plus grand soin, mais en dehors de considérations de technique 
juridique. Celle-ci n'intervient que pour justifier la décision devant un autre auditoire, et non pas 
du tout, comme l'explique Mill, pour formuler d'une façon experte les maximes générales dont le 
gouverneur avait une impression assez vague. Le scientisme de Mill, qui lui fait tout concevoir en 
fonction  d'un  seul  auditoire,  l'auditoire  universel,  ne  lui  permet  pas  de  fournir  une  explication 
adéquate du phénomène. » 
\bigskip
(1) J. S. Mill, Système de logique, vol. 1, liv. 11, chap. 111, § 3, p. 213. 
\bigskip
P  56-57 :  « Les  argumentations  nouvelles,  postérieures  à  la  décision,  peuvent  consister  dans 
l'insertion de la conclusion dans un cadre technique, comme dans le cas que nous venons de citer; 
elles peuvent être non techniques, comme dans ce récit d'Antoine de La Salle (1), où un seigneur et 
sa femme devisent la nuit. Le seigneur doit choisir entre le sacrifice de sa ville et celui de son fils. 
La décision n'est pas douteuse, mais néanmoins Antoine de La Salle attache de l'importance aux 
paroles de l'épouse, qu'il rapporte avec force détails. Ces paroles transforment la manière dont la 
décision est envisagée : la femme donne au mari l'orgueil de lui-même, l'équilibre, la confiance, la 
consolation,  c'est  elle  qui  met  de  l'ordre  dans  ses  idées,  insère  la  décision  dans  un  cadre,  et  la 
renforce par le fait même. Elle agit comme le théologien qui fournit les preuves rationnelles d'un 
dogme auquel tous les membres de l'Église croyaient déjà auparavant. » 
\bigskip
(1) Analysé par E. Auerbach,  mimemis, pp. 234-235. (Le réconfort de Madame du Fresne, publié 
par J. Nève, Antoine de La Salle, pp. 109-140.) 
\bigskip
P  57 :  « La  vie  politique  offre  également  des  situations  où  la  justification  d'une  décision  est 
attendue  avec  impatience,  car  c'est  de  cette  justification  que  dépendra  l'adhésion  de  l'opinion 
publique.  Lors  du  bannissement  du  roi  nègre  Seretse,  la  presse  annonçait  que  le  gouvernement 
britannique, sans rien changer à sa décision, ferait une concession à l'opinion publique en donnant 
de cette décision une justification meilleure et plus détaillée, c'est-à-dire qui puisse être admise par 
l'auditoire auquel elle s'adressait. 
\bigskip
Cette  préférence  pour  certains  arguments  peut  résulter  de  ce  que  l'auditeur  souhaite  avoir  à  sa 
disposition  des  arguments  qui  seraient  valables  pour  un  autre  auditoire,  voire  pour  l'auditoire 
universel, qui seraient donc transposables dans une situation modifiée. » 
\bigskip
P 57-58 : « Il ressort de tout ce que nous venons de dire au sujet des auditoires, que, de notre point 
de  vue,  la  valeur  rhétorique  d'un  énoncé  ne  saurait  être  annihilée  par  le  fait  qu'il  s'agirait  d'une 
argumentation que l'on estime bâtie après coup, alors que la décision intime, était prise, ou par le 
fait qu'il s'agit d'une argumentation basée sur des prémisses auxquelles l'orateur n'adhère pas lui-
même.  Dans  ces  deux  cas,  qui  sont  distincts  bien  que  liés  par  un  certain  biais,  le  reproche 
d'insincérité,  d'hypocrisie,  pourra  être  fait  par  un  observateur,  on  par  un  adversaire.  Mais  ce  ne 
sera  là  qu'un  moyen  de  disqualification  dont  la  portée  ne  subsiste  que  si  l'on  se  place  dans  une 
perspective toute différente de la nôtre ; le plus souvent d'ailleurs cette perspective est basée sur 
une conception bien définie du réel, ou de la personne. » 

P 58 : « Notre thèse est que d'une, part une croyance une fois établie peut toujours être intensifiée 
et que, d'autre part, l'argumentation est fonction de l'auditoire auquel on s'adresse. Dès lors, il est 
légitime que celui qui a acquis une certaine conviction s'attache à l'affermir vis-à-vis de lui-même, 
et surtout vis-à-vis des attaques pouvant venir de l'extérieur; il est normal qu'il envisage tous les 
arguments susceptibles de la renforcer. Ces nouvelles raisons peuvent intensifier la conviction, la 
protéger contre certaines attaques auxquelles on n'avait pas pensé dès le début, préciser sa portée. 
\bigskip
Par  ailleurs,  c'est  uniquement  quand  l'orateur  s'adresse  à  un  auditoire  auquel  il  est  censé 
appartenir  -  et  c'est  évidemment  le  cas  de  l'auditoire  universel  -  que  toute  discordance  entre  les 
arguments  qui  l'ont  convaincu  lui-même  et  ceux  qu'il  profère  pourrait  lui  être  reprochée.  Mais, 
même dans ce cas privilégié, il n'est pas exclu que la conviction intime de l'orateur soit fondée sur 
des  éléments  qui  lui  sont  propres  -  telle  une  intuition  incommunicable  -  et  qu'il  soit  obligé  de 
recourir à une argumentation pour faire partager la croyance qu'ils ont engendrée. » 
\bigskip
P  58-59 :  « En  conclusion,  si  l'étude  de  l'argumentation  nous  permet  de  comprendre  les  raisons 
qui  ont  incité  tant  d'auteurs  à  accorder  à  la  délibération  intime  un  statut  privilégié,  cette  même 
étude nous fournit les movens de distinguer les diverses espèces de délibération et de comprendre, 
à la fois, ce qu'il y a de fondé dans l'opposition entre raisons et rationalisations, et l'intérêt réel qui 
s'attache, au point de vue argumentatif, à ces rationalisations trop méprisées. » 
\bigskip
§ 10. LES EFFETS DE L'ARGUMENTATION 
\bigskip
P 59 : « Le but de toute argumentation, avons-nous dit, est de provoquer ou d'accroître l'adhésion 
des esprits aux thèses qu'on présente à leur assentiment : une argumentation efficace est celle qui 
réussit  à  accroître  cette  intensité  d'adhésion  de  façon  à  déclencher  chez  les  auditeurs  l'action 
envisagée (action positive ou abstention), ou du moins à créer, chez eux, une disposition à l'action, 
qui se manifestera au moment opportun. 
\bigskip
L'éloquence  pratique,  qui  comportait  les  genres  judiciaire  et  délibératif,  était  le  champ  de 
prédilection  où  s'affrontaient  plaideurs  et  hommes  politiques  qui  défendaient,  en  argumentant, 
des  thèses  opposées  et  parfois  même  contradictoires.  Dans  ces  joutes  oratoires  les  adversaires 
cherchaient  à  gagner  l'adhésion  de  leur  auditoire  sur  des  sujets  controversés,  où  le  pour  et  le 
contre  trouvaient  souvent  des  défenseurs  également  habiles  et,  apparemment,  également 
honorables. » 
\bigskip
P  59-60 :  « Les  détracteurs  de  la  rhétorique  -  pour  lesquels  il  n'y  avait  qu'une  vérité,  en  toute 
matière  -  déploraient  pareil  état  de  choses  :  selon  eux,  les  protagonistes  conduisaient  leurs 
argumentations  divergentes  à  l'aide  de  raisonnements  dont  la  valeur  probante  ne  pouvait  être 
qu'illusoire. La rhétorique digne du philosophe, nous dit Platon dans le Phèdre, celle qui gagnerait, 
par ses raisons, les dieux eux-mêmes, devrait au contraire se placer sous le signe de la vérité. Et, 
vingt  siècles  plus  tard,  Leibniz  qui  se  rend  compte  de  ce  que  le  savoir  humain  est  borné  et 
incapable  souvent  de  fournir  des  preuves  suffisantes  de  la  vérité  de  toute  assertion,  voudrait  au 
moins  que  le  degré  de  l'assentiment  accordé  à  n'importe  quelle  thèse  soit  proportionnel  à  ce 
qu'enseigne le calcul des probabilités nu des présomptions (1). » 
\bigskip
(1) Leibniz, éd. Gerhardt, vol. 5 : Nouveaux essais sur l'entendement, pp. 445-448. 
\bigskip
P  60 :  « Les  attaques  dont  fit  l'objet  de  la  part  des  philosophes  la  théorie  de  la  persuasion 
raisonnée,  développée  dans  les  ouvrages  de  rhétorique,  paraissaient  d'ailleurs  d'autant  plus 
fondées  que  l'objet  de  l'argumentation  se  limitait,  pour  les  théoriciens,  à  des  questions  que  nous 
pourrions ramener à des problèmes de conjecture et de qualification. Les problèmes de conjecture 
concernent les faits, faits passés dans les débats judiciaires, faits futurs dans les débats politiques : 
«X a-t-il accompli ce qu'on lui reproche ? », « Tel acte entraînera-t-il ou non telle conséquence ? », 
voilà  le  type  de  questions  que  nous  qualifions  de  conjecturales.  Dans  les  problèmes  de 
\bigskip
\bigskip
\bigskip
29 
\bigskip
qualification, on se demande si tel fait peut être qualifié de telle ou telle façon. Dans les deux cas, il 
semblait scandaleux que l'on pût défendre honnêtement plus d'un point de vue. Ce point de vue, 
c'était  au  philosophe,  qui  étudiait  d'une  façon  désintéressée  les  problèmes  d'ordre  général,  à  le 
fournir  et  à  le  justifier.  Les  conclusions  pratiques  qu'il  faudrait  tirer  de  l'étude  des  faits 
s'imposeraient d'elles mêmes à tout esprit raisonnable. 
\bigskip
Dans  une  pareille  perspective  l'argumentation,  telle  que  nous  la  concevons,  n'a  plus  de  raison 
d'être.  Les  faits,  les  vérités  ou  du  moins  les  vraisemblances,  soumises  au  calcul  des  probabilités, 
triomphent d'eux-mêmes. Celui qui les présente ne joue aucun rôle essentiel, ses démonstrations 
sont intemporelles, et il n'y a pas lieu de distinguer les auditoires auxquels on s'adresse, puisque 
tous sont censés s'incliner devant ce qui est objectivement valable. » 
\bigskip
P  60-61 :  « Et,  sans  aucun  doute,  dans  le  domaine  des  sciences  purement  formelles,  telles  la 
logique  symbolique  on  les  mathématiques,  ainsi  que  dans  le  domaine  purement  expérimental, 
cette fiction qui isole du sujet connaissant le fait, la vérité ou la probabilité, présente des avantages 
indéniables. Aussi, parce que cette technique « objective » réussit en science, a-t-on la conviction 
que dans d'autres domaines, son usage est également légitime. Mais là où un accord n'existe pas, 
même  chez  des  personnes  compétentes  en  la  matière,  qu'est-elle,  sinon  un  procédé  à  exorciser, 
cette  affirmation  que  les  thèses  préconisées  sont  la  manifestation  d'une  réalité  ou  d'une  vérité 
devant laquelle un esprit non prévenu ne peut que s'incliner ? » 
\bigskip
P 61 : « Il semble, bien au contraire, que l'on risque moins de simplifier et de déformer la situation 
dans  laquelle  s'effectue  le  processus  argumentatif  en  considérant  comme  un  cas  particulier, 
quoique  très  important,  celui  où  la  preuve  de  la  vérité  ou  de  la  probabilité  d'une  thèse  peut  être 
administrée  à  l'intérieur  d'un  domaine  formellement,  scientifiquement  ou  techniquement 
circonscrit,  de  commun  accord,  par  tous  les  interlocuteurs.  C'est  alors  uniquement  que  la 
possibilité de prouver le pour et le contre est l'indice d'une contradiction qu'il faut éliminer. Dans 
les autres cas, la possibilité d'argumenter de façon à aboutir à des conclusions opposées implique 
justement que l'on ne se trouve pas dans cette situation particulière que l'usage des sciences nous a 
rendu  familière.  Ce  sera  toujours  le  cas  quand  l'argumentation  tend  à  provoquer  une  action  qui 
résulte d'un choix délibéré entre plusieurs possibles, sans qu'il y ait accord préalable sur un critère 
permettant de hiérarchiser les solutions. » 
\bigskip
P 61-62 : « Les philosophes qui s'indignaient de ce que l'on pût ne pas se conduire conformément à 
la  conclusion  qui  paraissait  la  seule  raisonnable,  ont  été  obligés  de  compléter  leur  vision  de 
l'homme  en  le  douant  de  passions  et  d'intérêts  capables  de  s'opposer  aux  enseignements  de  la 
raison.  Pour  reprendre  la  distinction  pascalienne,  à  l'action  sur  l'entendement,  on  ajoutera  les 
moyens  d'agir  sur  la  volonté.  Dans  cette  perspective,  tandis  que  la  tâche  du  philosophe,  dans  la 
mesure où il s'adresse à un auditoire particulier, sera de faire taire des passions qui sont propres à 
celui-ci,  de  façon  à  faciliter  la  considération  «  objective  »  des  problèmes  en  discussion,  celui  qui 
vise  à  une  action  précise,  se  déclenchant  au  moment  opportun,  devra,  au  contraire,  exciter  les 
passions,  émouvoir  ses  auditeurs,  dé  façon  à  déterminer  une  adhésion  suffisamment  intense, 
capable de vaincre à la fois l'inévitable inertie et les forces qui agissent dans un sens différent de 
celui souhaité par l'orateur. » 
\bigskip
P  62 :  « On  peut  se  demander  si  l'existence  chez  Aristote  de  deux  traités  consacrés  à 
l'argumentation,  Topiques  et  Rhétorique,  l'un  se  référant  à  la  discussion  théorique  de  thèses, 
l'autre  tenant  compte  des  particularités  des  auditoires,  n'a  pas  favorisé  cette  distinction 
traditionnelle  entre  l'action  sur  l'entendement  et  l'action  sur  la  volonté.  Quant  à  nous,  nous 
croyons  que  cette  distinction,  qui  présente  la  première  comme  entièrement  impersonnelle  et 
intemporelle, et la seconde comme tout à fait irrationnelle, est fondée sur une erreur et conduit à 
une  impasse.  L'erreur  est  de  concevoir  l'homme  comme  constitué  de  facultés  complètement 
séparées. L'impasse est d'enlever à l'action fondée sur le choix toute justification rationnelle, et de  rendre par là absurde l'exercice de la liberté humaine. Seule l'argumentation, dont la délibération 
constitue  un  cas  particulier,  permet  de  comprendre  nos  décisions.  C'est  la  raison  pour  laquelle 
nous envisagerons surtout l'argumentation dans ses effets pratiques : tournée vers l'avenir, elle se 
propose de provoquer une action ou d'y préparer, en agissant par des moyens discursifs sur l'esprit 
des auditeurs. Cette façon de l'envisager permettra de comprendre plusieurs de ses particularités, 
et  notamment  l'intérêt  que  présente  pour  elle  le  genre  oratoire  que  les  Anciens  ont  qualifié 
d'épidictique. 


§ 11. LE GENRE EPIDICTIQUE 
\bigskip
Aristote et tous les théoriciens qui s'en inspirent font une place, dans leurs traités de Rhétorique, à 
côté des genres oratoires délibératif et judiciaire, au genre épidictique. » 
\bigskip
P 62-64 : « Ce dernier s'était, indéniablement, affirmé avec vigueur. La plupart des chefs-d'œuvre  
de  l'éloquence  scolaire,  les  éloges  et  panégyriques  d'un  Gorgias  ou  d'un  Isocrate,  morceaux 
d'apparat  célèbres  dans  toute  la  Grèce,  constituaient  des  discours  du  genre  épidictique. 
Contrairement aux débats politiques et judiciaires, vrais combats où deux adversaires cherchaient, 
sur  des  matières  controversées,  à  gagner  l'adhésion  d'un  auditoire  qui  décidait  de  l'issue  d'un 
procès  ou  d'une  action  à  entreprendre,  les  discours  épidictiques  n'étaient  rien  de  tout  cela.  Un 
orateur solitaire q i, souvent, n'apparaissait même pas devant le publie, mais se contentait de faire 
circuler  sa  composition  écrite,  présentait  un  discours  auquel  personne  ne  s'opposait,  sur  des 
matières qui ne semblaient pas douteuses et dont on ne voyait aucune conséquence pratique. Qu'il 
s'agisse  d'un  éloge  funèbre  ou  de  celui  d'une  ville  devant  ses  habitants,  d'un  sujet  dépourvu 
d'actualité,  tel  que  l'exaltation  d'une  vertu  ou  d'une  divinité,  les  auditeurs  n'y  jouaient,  selon  les 
théoriciens, que le rôle de spectateurs. Après avoir écouté l'orateur, ils n'avaient qu'à applaudir et à 
s'en  aller.  Ces  discours  formaient  d'ailleurs  une  attraction  de  choix  aux  fêtes  qui  réunissaient 
périodiquement les habitants d'une ville ou ceux de plusieurs cités, et son effet le plus visible était 
d'illustrer  le  nom  de  son  auteur.  Un  pareil  morceau  d'apparat  était  apprécié  comme  une  œuvre  
d'artiste,  de  virtuose,  mais  on  voyait  dans  cette  appréciation  flatteuse  une  fin,  et  non  la 
conséquence de ce que l'orateur avait atteint un certain but. On traitait le discours à la manière de 
spectacles  de  théâtre  on  de  joutes  athlétiques,  dont  le  but  semblait  être  la  mise  en  vedette  des 
participants.  Son  caractère  particulier  en  avait  fait  abandonner  l'étude  aux  grammairiens  par  les 
rhéteurs  romains  qui  exerçaient  leurs  élèves  dans  les  deux  autres  genres,  considérés  comme 
relevant  de  l'éloquence  pratique  (1).  Il  présentait,  pour  les  théoriciens,  une  forme  dégénérée 
d'éloquence qui ne cherchait qu'à plaire, à rehausser, en les ornant, des faits certains ou, du moins, 
incontestés  (1).  Ce  n'est  pas  que  les  Anciens  n'aient  vu  d'autre  fin  au  discours  épidictique.  Pour 
Aristote,  l'orateur  se  Propose  d'atteindre  selon  le  genre  de  discours  des  fins  différentes,  dans  le 
délibératif,  conseillant  Futile  c'est-à-dire  le  meilleur,  dans  le  judiciaire  plaidant  le  juste,  et,  dans 
l'épidictique, qui traite de l'éloge et du blâme, n'ayant à s'occuper que de ce qui est beau ou laid. Il 
s'agit  donc  bien  de  reconnaître  des  valeurs.  Mais  la  notion  de  jugement  de  valeur,  et  celle 
d'intensité  d'adhésion,  faisant  défaut,  les  théoriciens  du  discours,  à  la  suite  d'Aristote,  mêlent 
incontinent  l'idée  de  beau,  objet  du  discours,  équivalente  d'ailleurs  à  celle  de  bon,  et  l'idée  de  la 
valeur esthétique du discours lui-même (2). » 
\bigskip
(1) Quintilien, Vol. I, liv. 11, chap. 1, §§ 1, 2, 8, 9. Cf. A. Ed. Chaignet. La rhétorique et son histoire, 
p. 235. 
(1) Cf. Aubrey Gwynn, Roman education from Cicero to Quintilian, pp. 98-99.  
(2) Cf. Aristote, Rhétorique, liv. 1, chap. 3, 1358 b, et 1358 b, 28-29e V. plus haut l'auditoire comme 
spectateur dans 4 : L'auditoire comme construction de l'orateur. 
\bigskip
P 64 : « Par là, le genre épidictique semblait relever plus de la littérature que de l'argumentation. 
C'est ainsi que la distinction des genres a contribué à la désagrégation ultérieure de la rhétorique, 
car les deux premiers genres ont été annexés par la philosophie et la dialectique, le troisième ayant 
\bigskip
\bigskip
\bigskip
31 
\bigskip
été englobé dans la  prose littéraire. Et Whately au XIXe siècle, reprochera à Aristote de lui avoir 
encore attribué trop d'importance (3). 
\bigskip
Or nous croyons que les discours épidictiques constituent une partie centrale de l'art de persuader 
et  l'incompréhension  manifestée  à  leur  égard  résulte  d'une  fausse  conception  des  effets  de 
l'argumentation. » 
\bigskip
(3) Richard D. D. Whately, Elements of Rhetoric, Part III, chap. I, § 6, p. 190. 
\bigskip
P  64-65 :  « L'efficacité  d'un  exposé,  tendant  à  obtenir  des  auditeurs  une  adhésion  suffisante  aux 
thèses  qu'on  leur  présente,  ne  peut  être  jugée  que  d'après  le  but  que  se  propose  l'orateur. 
L'intensité de l'adhésion qu'il s'agit d'obtenir ne se limite pas à la production de résultats purement 
intellectuels, au fait de déclarer qu'une thèse paraît plus probable qu'une autre, mais bien souvent 
sera  renforcée  jusqu'à  ce  que  l'action,  qu'elle  devait  déclencher,  se  soit  produite.  Démosthène, 
considéré comme un des modèles de  l'éloquence classique, a consacré le plus clair de ses efforts, 
non seulement à obtenir des Athéniens qu'ils Prennent des décisions conformes à ses désirs, mais 
à  les  presser,  par  tous  les  moyens,  pour  que  ces  décisions,  une  fois  prises,  soient  exécutées.  Il 
Voulait,  en  effet,  que  les  Athéniens  fassent  à  Philippe  non  «  une  guerre  de  décrets  et  de  lettres 
seulement, mais une guerre en action » (1). Il devait rappeler constamment à ses concitoyens : 
\bigskip
... un décret  n'est rien par lui-même, si vous n'y ajoutez la volonté d'exécuter énergiquement ce 
que  vous  avez  décrété  »  car,  «  si  les  décrets  pouvaient  ou  vous  obliger  à  faire  ce  qu'il  faut  ou 
exécuter eux-mêmes ce qu'ils ordonnent, vous n'aboutiriez pas, après tant de votes, à de si minces 
résultats, ou pour mieux dire, à rien ... (2). » 
\bigskip
(1) Démosthène, Harangues et plaidoyers politiques, t. I : Première Philippique, ~ 30. 
(2) ID., Troisième Olynthienne, § 14. 
\bigskip
P 65 : « La décision prise se trouve, pour ainsi dire, à mi-chemin entre la disposition à l'action et 
l'action elle-même, entre la pure spéculation et l'action efficace. » 
\bigskip
P 65-66 : « L'intensité d'adhésion, visant à l'action efficace, ne peut être mesurée par le degré de 
probabilité accordé à la thèse admise, mais bien plutôt par les obstacles que l'action surmonte, les 
sacrifices  et  les  choix  qu'elle  entraîne  et  que  l'adhésion  permet  de  justifier.  L'existence  d'un 
intervalle de temps, plus ou moins grand, entre le moment de l'adhésion et celui de l'action qu'elle 
devait susciter (3), explique à suffisance l'intervention dans le débat, jugé clos antérieurement, de 
certaines  valeurs  oubliées  ou  minimisées,  d'éléments  nouveaux  qui  ont  surgi  peut-être  depuis  la 
prise de décision. Cette interférence, qui a d'autant plus de chances de se produire que la situation 
a évolué dans  l'intervalle, entraîne une double conséquence :  d'une part, la mesure de l'efficacité 
d'un  discours  est'  aléatoire,  d'autre  part  l'adhésion  qu'il  provoque  peut  toujours  utilement  être 
renforcée.  C'est  dans  cette  perspective,  parce  qu'il  renforce  une  disposition  à  l'action,  en 
augmentant  l'adhésion  aux  valeurs  qu'il  exalte,  que  le  discours  épidictique  est  significatif  et 
important pour l'argumentation. C'est parce que la réputation de l'orateur n'est pas la fin exclusive 
des discours épidictiques, qu'elle en est tout au plus une conséquence, qu'un éloge funèbre peut, 
sans  indécence,  être  prononcé  devant  une  tombe  fraîchement  ouverte,  qu'un  discours  de  carême 
peut viser autre chose que la gloire du prédicateur. » 
\bigskip
(3) Si le temps écoulé diminue généralement l'effet d'un discours, il n'en est pas toujours ainsi. Les 
psychologues américains eurent la surprise de déceler, dans certains cas, un effet différé (« sleeper 
effect  »).  Cf.  C.  I.  Hovland,  A.  A.  Lumsdaine  et  F.  D.  Sheffield,  Experiments  on  Mass 
Communication, pp. 71, 182, 188-200. Pour l'interprétation du phénomène, cf. C. I. Hovland et W. 
Weiss,  The  influence  of  source  credibility  on  communication  effectiveness,  Publ.  Op.  Quarterly, 
1952, 15, DP. 635-650; H. C. Kelman et C. I. Hovland,  « Reinstatement » of the Communicator in 
\bigskip
\bigskip
\bigskip
32 
\bigskip
delayed measurement of opinion change, J. a/ abn. and soc. Psych., 48, 3, pp. 327-335; W. Weiss, 
A « sleeper effect » in opinion change, J. of abn. and soc. Psych., 48, 2, pp. 173-180. 
\bigskip
P  66 :  « On  a  essayé  de  montrer  que  l'oraison  funèbre  des  Grecs  s'était  transformée  avec  le 
christianisme en moyen d'édification (1). En fait, il s'agit bien du même discours mais qui porte sur 
des valeurs nouvelles. Celles-ci sont incompatibles avec la recherche de la gloire terrestre. Aussi la 
crainte  de  voir  le  discours  sacré  considéré  comme  un  spectacle  est  telle  que  Bossuet,  dans  le 
Sermon sur la parole de Dieu, développe une longue analogie entre la chaire et l'autel pour aboutir 
à cette conclusion : 
\bigskip
... vous devez maintenant être convaincus que les prédicateurs de l'Évangile ne montent pas dans 
les chaires pour y faire de vains discours qu'il faille entendre pour se divertir (2). » 
\bigskip
(1)  Verdun  L.  Saulnier,  L'oraison  funèbre  au  XVIe  siècle,  Bibliothèque  d'Humanisme  et 
Renaissance, t. X, 1948, pp. 126-127. 
(2) Bossuet, Sermons, Vol. II : Sur la parole de Dieu, pp. 148-149. 
\bigskip
P 66-67 : « Et ce n'est point là seulement précautions d'un orateur, précautions qui, elles-mêmes, 
pourraient  n'être  qu'une  feinte,  prévision  d'un  danger  imaginaire.  Il  est  certain  que  le  discours  - 
particulièrement  le  discours  épidictique  -  est  souvent  jugé  comme  un  spectacle.  La  Bruyère  s'en 
moque abondamment : 
\bigskip
... ils en sont émus et touchés au point de résoudre dans leur cœur, sur ce sermon de Théodore, 
qu'il est encore plus beau que le dernier qu'il a prêché (1). » 
\bigskip
(1) La Bruyère, Bibl. de la Pléiade, Les caractères, De la chaire, 11, p. 460.   
\bigskip
P 67 : « Contrairement à la  démonstration d'un théorème de géométrie, qui établit une fois pour 
toutes  un  lien  logique  entre  des  vérités  spéculatives,  l'argumentation  du  discours  épidictique  se 
propose  d'accroître  l'intensité  de  l'adhésion  à  certaines  valeurs,  dont  on  ne  doute  peut-être  pas 
quand on les considère isolément, mais qui pourraient néanmoins ne pas prévaloir contre d'autres 
valeurs  qui  viendraient  à  entrer  en  conflit  avec  elles.  L'orateur  cherche  à  créer  une  communion 
autour de certaines valeurs reconnues par l'auditoire, en se servant de l'ensemble des moyens dont 
dispose la rhétorique pour amplifier et valoriser. 
\bigskip
C'est  dans  l'épidictique  que  tous  les  procédés  de  l'art  littéraire  sont  de  mise,  car  il  s'agit  de  faire 
concourir  tout  ce  qui  peut  favoriser  cette  communion  de  l'auditoire.  C'est  le  seul  genre  qui, 
immédiatement, fait penser à de la littérature, le seul que l'on aurait  pu comparer au livret d'une 
cantate  (2),  celui  qui  risque  le  plus  facilement  de  tourner  à  la  déclamation,  de  devenir  de  la 
rhétorique, dans le sens péjoratif et habituel du mot. » 
\bigskip
 (2) A. Boulanger, Elius Aristide, p. 94. 
\bigskip
P 67-68 : « La conception même de ce genre oratoire, qui rappelle plus, pour parler comme Tarde 
(3),  une  procession  qu'une  lutte,  le  fera  pratiquer  de  préférence  par  ceux  qui,  dans  une  société, 
défendent les valeurs traditionnelles, les valeurs admises, celles qui sont l'objet de l'éducation, et 
non  les  valeurs  révolutionnaires,  les  valeurs  nouvelles  qui  suscitent  des  polémiques  et  des 
controverses. Il y a un côté  optimiste, un côté bénisseur dans l'épidictique qui n'a pas échappé  à 
certains  observateurs  perspicaces  (1).  Ne  craignant  pas  la  contradiction,  l'orateur  y  transforme 
facilement en valeurs universelles, sinon en vérités éternelles, ce qui, grâce à l'unanimité sociale, a 
acquis  de  la  consistance.  Les  discours  épidictiques  feront  le  plus  facilement  appel  à  un  ordre 
universel, à une nature ou à une divinité qui seraient garants des valeurs incontestées, et que l'on 
juge incontestables. Dans l'épidictique, l'orateur se fait éducateur. » 
\bigskip
\bigskip
\bigskip
33 
\bigskip
 
(3) G. Tarde, La logique sociale, p. 439. 
(1) Timon (Cormenin), Livre des orateurs, pp. 152 à 172. 
\bigskip
§ 12. ÉDUCATION ET PROPAGANDE 
\bigskip
P  68 :  « L'analyse  du  genre  épidictique,  de  son  objet  et  du  rôle  qu'y  joue  l'orateur,  permettra 
d'élucider  une  question  controversée  et  qui  préoccupe  tant  de  théoriciens  à  l'heure  actuelle,  la 
distinction entre éducation et propagande. J. Driencourt, dans un livre récent et bien documenté 
(2), analyse et rejette de nombreuses tentatives pour distinguer l'éducation de la propagande , et 
n'aboutit à aucune conclusion satisfaisante, faute de situer son étude dans  le cadre  d'une théorie 
générale  de  l'argumentation.  Harold  D.  Lasswell,  le  spécialiste  américain  de  ces  questions,  croit 
que  l'éducateur  diff  ère  du  propagandiste  essentiellement  parce  que  son  propos  porte  sur  des 
matières  qui  ne  sont  pas,  pour  son  auditoire,  objet  de  controverse  (3).  Le  prêtre  catholique 
enseignant  les  préceptes  de  sa  religion  à  des  enfants  catholiques  de  sa  paroisse  remplit  un  rôle 
d'éducateur,  alors  qu'il  est  propagandiste  s'il  s'adresse,  dans  le  même  but,  aux  adultes  membres 
d'un  autre  groupe  religieux.  Mais,  à  notre  avis,  il  y  a  plus.  Alors  que  le  propagandiste  doit  se 
concilier, au préalable, l'audience de son publie, l'éducateur a été chargé par une communauté de 
se faire le porte-parole des valeurs reconnues par elle et, comme tel, il jouit d'un prestige dû à ses 
fonctions. » 
\bigskip
(2) J. Driencourt, La propagande. Nouvelle force politique, 1950. 
(3)  Harold  D.  Lasswell,  The  study  and  practice  of  propaganda,  dans  H.  D.  Lasswell,  Ralph  D. 
Casey  et  Bruce  Lannes  Smith,  Propaganda  and  Promotional  Activities,  an  annotated 
bibliography, 1935, p. 3. 
\bigskip
P 69 : « Or un instant de réflexion suffit pour constater que, à ce point de vue, l'orateur du discours 
épidictique  est  très  proche  de  l'éducateur.  Comme  ce  qu'il  va  dire  ne  suscite  pas  de  controverse, 
qu'un  intérêt  pratique  immédiat  n'y  est  jamais  engagé,-qu'il  ne  s'agit  pas  de  défendre  ou 
d'attaquer, mais de promouvoir des valeurs qui sont l'objet d'une communion sociale, l'orateur, s'il 
est  par  avance  assuré  de  la  bonne  volonté  de  son  auditoire,  doit  pourtant  posséder  un  prestige 
reconnu. Dans l'épidictique, plus que dans n'importe quel autre genre oratoire, il faut, pour ne pas 
être ridicule, avoir des titres à prendre la parole et ne pas être malhabile dans son usage. Ce n'est 
plus,  en  effet,  sa  propre  cause  ni  son  propre  point  de  vue,  que  l'on  défend,  mais  celui  de  tout 
l'auditoire  :  on  est,  pour  ainsi  dire,  l'éducateur  de  celui-ci,  et  s'il  est  nécessaire  de  jouir  d'un 
prestige  préalable,  c'est  pour  pouvoir  servir,  à  l'aide  de  sa  propre  autorité,  les  valeurs  que  l'on 
soutient. 
\bigskip
Il faut d'ailleurs que les valeurs dont on fait l'éloge soient jugées dignes de guider notre action, car 
sinon, comme le dit spirituellement Isocrate, 
\bigskip
à  quoi  bon  écrire  des  discours  dont  le  plus  grand  avantage  ne  saurait  être  que  de  ne  pouvoir 
persuader aucun des auditeurs (1) ? 
\bigskip
Les discours épidictiques ont pour but d'accroître l'intensité d'adhésion aux valeurs communes de 
l'auditoire  et  de  l'orateur;  leur  rôle  est  important,  car  sans  ces  valeurs  communes,  sur  quoi 
pourraient  s'appuyer  les  discours  délibératifs  et  judiciaires  ?  Alors  que  ces  derniers  genres  se 
servent  des  dispositions  existant  déjà  dans  l'auditoire,  que  les  valeurs  y  sont  des  moyens 
permettant de déterminer une action, dans l'épidictique la communion autour des valeurs est  une 
fin  que  l'on  poursuit,  indépendamment  des  circonstances  précises  dans  lesquelles  cette 
communion sera mise à l'épreuve. » 
\bigskip
(1) Isocrate, Discours, t, 1 : Busiris, § 47, 
\bigskip
\bigskip
\bigskip
\bigskip
34 
\bigskip
P 70 : « S. Weil, analysant les moyens dont les Français de Londres eussent pu se servir, pendant la 
guerre, pour galvaniser les Français de l'intérieur énumère parmi ceux-ci : 
\bigskip
l'expression soit officielle, soit approuvée par une autorité officielle, d p des pensées qui, dès avant 
d'avoir  été  exprimées,  se  trouvaient  réellement  au  coeur  des  foules,  ou  au  coeur  de  certains 
éléments actifs de la nation... Si l'on entend formuler cette pensée hors de soi-même, par autrui et 
par quelqu'un aux paroles de qui on attache de l'attention, elle en reçoit une force centuplée et peut 
parfois produire une transformation intérieure (1). 
\bigskip
Ce  qu'elle  met  ainsi  fort  bien  en  évidence,  c'est  précisément  le  rôle  des  discours  épidictiques  : 
appels à des valeurs communes, non contestées bien qu'informulées, et par quelqu'un qui a qualité 
pour  le  faire  ;  renforcement,  par  là,  de  l'adhésion  à  ces  valeurs  en  vue  d'actions  ultérieures 
possibles.  Ce  que  l'on  appelait  la  propagande  de  Londres,  devient,  dans  cette  perspective, 
beaucoup plus proche de l'éducation que de la propagande. 
\bigskip
Le fait que l'épidictique est destiné à promouvoir des valeurs sur lesquelles on s'accorde, explique 
que l'on éprouve l'impression d'un abus lorsque, à l'occasion d'un pareil discours, quelqu'un prend 
position dans une matière controversée, détourne son argumentation vers des valeurs contestées, 
introduit  des  dissonances  dans  une  circonstance  créée  pour  favoriser  la  communion,  lors  d'une 
cérémonie funèbre par exemple. Le même abus existe quand un éducateur se fait propagandiste. » 
\bigskip
(1) S. Weil, L'enracinement, p. 164. 
\bigskip
P  70-71 :  « Dans  l'éducation,  quel  qu'en  soit  l'objet,  on  suppose  que  le  discours  de  l'orateur,  s'il 
n'exprime  pas  toujours  des  vérités,  c'est-à-dire  des  thèses  admises  par  tout  le  monde,  défend  au 
moins des valeurs qui ne sont pas, dans le milieu qui l'a délégué, sujet à controverse. Il est censé 
jouir  d'une  si  grande  confiance  que,  contrairement  à  tout  autre,  il  ne  doit  pas  s'adapter  à  ses 
auditeurs  et  partir  de  thèses  que  ceux-ci  admettent,  mais  peut  procéder  à  l'aide  d'arguments 
qu'Aristote  appelle  didactiques  (1),  et  que-les  auditeurs  adoptent  parce  que  «le  maître  l'a  dit  ». 
Alors  que,  dans  une  tentative  (le  vulgarisation,  l'orateur  se  fait  comme  le  propagandiste  de  la 
spécialité  et  doit  insérer  celle-ci  dans  les  -cadres  d'un  savoir  commun,  lorsqu'il  s'agit  d'une 
initiation à une discipline particulière, le maître commencera par élioncer les principes propres à 
cette discipline (2). De même, quand il est chargé d'inculquer les valeurs d'une société déterminée 
à  de  tout  jeunes  enfants,  l'éducateur  doit  procéder  par  affirmation,  sans  s'engager  dans  une 
controverse  où  l'on  défendrait  librement  le  pour  et  le  contre.  Ceci  serait  d'ailleurs  contraire  à 
l'esprit  même  de  la  première  éducation,  car  toute  discussion  suppose  l'adhésion  préalable  à 
certaines thèses, sans quoi aucune argumentation n'est possible (3). » 
\bigskip
(1) Aristote, Réfutations sophistiques, chap. 2, 165 h. 
(2) Cf. § 26 : Accords de certains auditoires particuliers. 
(3)  Cf.  Ch.  Perelman,  Education  et  rhétorique,  Revue  belge  de  psychologie  et  de  pédagogie, 
décembre 1952. 
\bigskip
P  71 :  « Le  discours  éducatif,  tout  comme  l'épidictique,  vise  non  à  la  mise  en  valeur  de  l'orateur, 
mais  à  la  création  d'une  certaine  disposition  chez  les  auditeurs.  Contrairement  aux  genres 
délibératif  et judiciaire,  qui  se  proposent  d'obtenir  nue  décision  d'action,  l'épidictique,  comme  le 
discours  éducatif,  créent  une  simple disposition  à  l'action,  par  quoi  on  peut  les  rapprocher  de  la 
pensée  philosophique.  Si  cette  distinction  entre  genres  oratoires  n'est  pas  touj  ours  facile  à 
appliquer,  néanmoins  elle  présente,  à  notre  point  de  vue,  l'avantage  d'offrir,  à  l'étude  de 
l'argumentation,  un  cadre  unitaire:  toute  argumentation  ne  se  conçoit,  dans  cette  perspective, 
qu'en  fonction  de  l'action  qu'elle  prépare  ou  qu'elle  détermine.  C'est  une  raison  supplémentaire 
pour  notre  rapprochement  de  la  théorie  de  l'argumentation  avec  la  rhétorique  plutôt  qu'avec  la 
\bigskip
\bigskip
\bigskip
35 
\bigskip
dialectique des Anciens, celle-ci se limitant à la seule spéculation, celle-là mettant au premier plan 
l'action exercée par le discours sur la personnalité tout entière des auditeurs. » 
\bigskip
P  72 :  « Le  discours  épidictique  -  et  toute  éducation  -  visent  moins  à  un  changement  dans  les 
croyances  qu'à  une  augmentation  de  l'adhésion  à  ce  qui  est  déjà  admis,  alors  que  la  propagande 
bénéficie  de  tout  le  côté  spectaculaire  des  changements  perceptibles  qu'elle  cherche  à réaliser,  et 
qu'elle  réalise  parfois.  Néanmoins,  dans  la  mesure  oh  l'éducation  augmente  la  résistance  contre 
une  propagande  adverse,  il  est  utile  de  considérer  éducation  et  propagande  comme  des  forces 
agissant en sens contraire. Par ailleurs, nous verrons que toute argumentation peut être envisagée 
comme un substitut de la force matérielle qui, par la contrainte se proposerait d'obtenir des effets 
de même nature. 
\bigskip
§ 13. ARGUMENTATION ET VIOLENCE 
\bigskip
impossible,  aux  adversaires, 
\bigskip
L'argumentation est une action qui tend toujours à modifier un état de choses préexistant. Cela est 
vrai, même du discours épidictique : c'est par là qu'il est argumentatif. Mais tandis que celui qui 
prend  l'initiative  d'un  débat  est  comparable  à  un  agresseur,  celui  qui,  par  son  discours,  désire 
renforcer  des  valeurs  établies, ressemblera  à  ce  gardien  protecteur  des  digues  qui  subissent  sans 
cesse l'assaut de l'Océan. » 
\bigskip
P 72-73 : « Toute société qui tient à ses valeurs propres ne  peut donc que favoriser les occasions 
qui  permettent  aux  discours  épidictiques  de  se  reproduire  à  un  rythme  régulier  :  cérémonies 
commémorant  des  faits  intéressant  le  pays,  offices  religieux,  éloges  des  disparus  et  autres 
manifestations  servant  la  communion  des  esprits.  Dans  la  mesure  oil  les  dirigeants  du  groupe 
cherchent à augmenter leur emprise sur la pensée de ses membres, ils multiplieront les réunions 
de  caractère  éducatif,  et  certains  iront  même,  à  la  limite,  jusqu'à  employer  la  menace  ou  la 
contrainte  pour  amener  les  récalcitrants  à  se  soumettre  aux  discours  qui  les  imprégneront  de 
valeurs  communautaires.  Par  contre,  considérant  tout  assaut  contre  des  valeurs  officiellement 
reconnues  comme  un  acte  révolutionnaire,  ces  mêmes  dirigeants,  par  l'établissement  d'une 
censure, d'un index, par le contrôle des moyens de communiquer les idées, s'efforceront de rendre 
difficile,  si  pas 
la  réalisation  des  conditions  préalables  à 
l'argumentation. Ces derniers seront acculés, s'ils veulent continuer la lutte, à l'usage de la force. » 
\bigskip
P 73 : « On peut, en eff et, essayer d'obtenir un même résultat soit par le recours à la violence soit 
par le discours visant à l'adhésion des esprits. C'est en fonction de cette alternative que se conçoit 
le  plus  nettement  l'opposition  entre  liberté  spirituelle  et  contrainte.  L'usage  de  l'argumentation 
implique que l'on a renoncé à recourir uniquement à la force, que l'on attache du prix à l'adhésion 
de  l'interlocuteur,  obtenue  à  l'aide  d'une  persuasion  raisonnée,  qu'on  ne  le  traite  pas  comme  un 
objet,  mais  que  l'on  fait  appel  à  sa  liberté  de  jugement.  Le  recours  à  l'argumentation  suppose 
l'établissement  d'une  communauté  des  esprits  qui,  pendant  qu'elle  dure,  exclut  l'usage  de  la 
violence (1). Consentir à la discussion, c'est accepter de se placer au point de vue de l'interlocuteur, 
c'est  ne  s'attacher  qu'à  ce  qu'il  admet  et  ne  se  prévaloir  de  ses  propres  croyances  que  dans  la 
mesure  où  celui  que  nous  cherchons  à  persuader  est  disposé  à  leur  accorder  son  assentiment.  « 
Toute  justification,  dit  E.  Dupréel,  est  déjà  par  essence,  un  acte  modérateur,  un  pas  vers  plus  de 
communion des consciences (2). » 
\bigskip
(1) Cf. E. Weil, Logique de la philosophie, p. 24. 
(2) Fragments pour la théorie de la connaissance de M. E. Dupréel, Dialectique, 5, p. 76. Sur la 
rhétorique comme triomphe de la persuasion sur la force brutale, cf. G. Toffanin, Storia dell' 
umanesimo, pp. 173-175. 
\bigskip
P  73-74 :  « D'aucuns  prétendront  que  parfois,  voire  toujours,  le  recours  à  l'argumentation  n'est 
qu'une  feinte.  Il  n'y  aurait  qu'un  semblant  de  débat  argumentatif,  soit  que  l'orateur  impose  à 
l'auditoire  l'obligation  de  l'écouter,  soit  que  ce  dernier  se  contente  d'en  faire  le  simulacre  :  dans 
\bigskip
\bigskip
\bigskip
36 
\bigskip
l'un  comme  dans  l'autre  cas,  l'argumentation  ne  serait  qu'un  leurre,  l'accord  acquis  ne  serait 
qu'une forme déguisée de coercition ou un symbole de bon vouloir. Pareille opinion sur la nature 
du  débat  argurnentatif  ne  peut  être  exclue  a  priori  :  encore  la  mise  en  branle  de  l'appareil 
argumentatif s'explique-t-elle mal, si, dans certains cas au moins, il n'y a pas persuasion véritable. 
En  fait,  toute  communauté,  qu'elle  soit  nationale  ou  internationale,  prévoit  des  institutions 
juridiques,  politiques  ou  diplomatiques,  permettant  de  régler  certains  conflits  sans  que  l'on  soit 
obligé  d'avoir  recours  à  la  violence.  Mais  c'est  une  illusion  de  croire  que  les  conditions  de  cette 
communion  des  consciences,  soient  inscrites  dans  la  nature  des  choses.  A  défaut  de  pouvoir  se 
référer  à  celle-ci,  les  défenseurs  de  la  philosophie  critique,  tels  Guido  Calogero,  voient  dans  la 
volonté  de  comprendre  autrui,  dans  le  principe  du  dialogue,  le  fondement  absolu  d'une  éthique 
libérale (1). Calogero conçoit le devoir du dialogue comme 
\bigskip
liberté d'exprimer sa foi et de tâcher d'y convertir les autres, devoir de laisser les autres faire la 
même chose avec nous et de les écouter avec la même bonne volonté de comprendre leurs vérités 
et les faire nôtres que nous réclamons d'eux par rapport aux nôtres (2). » 
\bigskip
(1)  G.  Calogero,  Why  do  we  ask  why,  Actes  du  XP  Congrès  international  de  Philosophie,  XIV,  P. 
260. 
(2) G. Calogero, Vérité et liberté, Actes du Xe Congrès international de Philosophie, p. 97. En 
italien, en appendice à Logo e Dialogo, p. 195. 
\bigskip
P  74 :  « Ce  «  devoir  du  dialogue  »  que  Calogero  présente  comme  un  com  promis  entre 
l'absolutisme  de  Platon  et  le  scepticisme  de  Protagoras,  ne  constitue  nullement  une  vérité 
nécessaire ni même une assertion qui va de soi. Il s'agit là d'un idéal que poursuivent un très petit 
nombre  de  personnes,  celles  qui  accordent  plus  d'importance  à  la  pensée  qu'à  l'action  et  encore, 
parmi celles-là, ce principe ne vaudrait que pour les philosophes non absolutistes. » 
\bigskip
P 74-75 : « En fait, bien peu de gens admettraient que toutes les questions puissent être mises en 
discussion. Aristote considère que: 
\bigskip
Il  ne  faut  pas,  du  reste,  examiner  toute  thèse,  ni  tout  problème  c'est  seulement  au  cas  où  la 
difficulté  est  proposée  par  des  gens  en  quête  d'arguments  et  non  pas  quand  c'est  un  châtiment 
qu'elle requiert, ou quand il suffit d'ouvrir les yeux. Ceux qui, par exemple, se posent la question 
de  savoir  s'il  faut  ou  non  honorer  les  dieux  et  aimer  ses  parents,  n'ont  besoin  que  d'une  bonne 
correction, et ceux qui se demandent si la neige est blanche ou non, n'ont qu'à regarder (1). » 
\bigskip
(1) Aristote, Topiques, liv. 1, chap. 11, 105 a. 
\bigskip
P 75 : « Il va même plus loin et conseille à ses lecteurs de ne soutenir aucune proposition qui soit 
improbable,  ou  contraire  à  la  conscience:  comme  «  tout  se  ment  »  ou  «  rien  ne  se  meut  »,  «  le 
plaisir est le bien » ou « commettre l'injustice vaut mieux que la subir » (2). Sans doute ne sont-ce 
là  que  conseils  adressés  au  dialecticien.  Mais  ils  reflètent  l'attitude  du  sens  commun.  Celui-ci 
admet l'existence de vérités indiscutées et indiscutables, il admet que certaines règles soient « hors 
discussion » et que certaines suggestions « ne méritent pas discussion ». Un fait établi, une vérité 
évidente, une règle absolue, portent avec eux l'affirmation de leur caractère indiscutable, excluant 
la  possibilité  de  défendre  le  pour  et  le  contre.  L'accord  unanime  sur  certaines  propositions  peut 
rendre  fort  difficile  leur  mise  en  doute.  On  connaît  le  conte  d'Orient  dans  lequel,  à  l'encontre  de 
tous, seul un enfant, naïf et innocent, a osé affirmer que le roi était nu, rompant ainsi l'unanimité 
née de la crainte de dire la vérité (3). 
\bigskip
(2) Aristote, ibid., liv. VIII, chap. 9, 160 b. 
(3) Cf. § 71 : Les techniques de rupture et de, freinage opposées à l'interaction acte-personne. 
\bigskip
\bigskip
\bigskip
\bigskip
37 
\bigskip
P  75-76 :  « Être  d'un  avis  qui  s'écarte  de  celui  de  tous  les  autres,  c'est  rompre  une  communion 
sociale  fondée,  croit-on  -  et  le  plus  souvent  à  juste  titre  -  sur  des  données  d'ordre  objectif.  Le 
XVIIIe siècle, français et allemand, nous fournit l'exemple d'une tentative, certes utopique, mais à 
coup sûr émouvante, d'établir une catholicité des esprits sur la base d'un rationalisme dogmatique, 
permettant  d'assurer  des  fondements  sociaux  stables  à  une  humanité  pénétrée  des  principes 
rationnels.  Cet  essai  de  résoudre,  grâce  à  la  raison,  tous  les  problèmes  que  pose  l'action,  s'il  a 
contribué à la généralisation de l'instruction, a malheureusement échoué parce que, bien vite, on 
s'est rendu compte de ce que l'unanimité était précaire, illusoire, ou même impensable. » 
\bigskip
P 76 : « Cette unanimité, pourtant, toutes les sociétés tiennent à l'assurer, car elles en connaissent 
la valeur et la force (1). Aussi l'opposition à une norme admise peut-elle mener l'homme en prison 
ou dans un asile d'aliénés. 
\bigskip
Parfois la simple mise en question de certaines décisions sera sévèrement punie. Démosthène fait 
allusion,  dans  sa  Première  Olynthienne,  au  décret  athénien  interdisant,  sous  peine  de  mort, 
d'introduire un projet de loi modifiant l'affectation du fonds de réserve de la cité (2). 
\bigskip
Même quand la discussion est admise en principe, il y a des moments où la prolongation n'en est 
plus  tolérée  à  cause  des  nécessités  de  l'action.  La  réglementation  d'un  débat  peut  porter  non 
seulement  sur  les  questions  préalables,  telles  la  compétence  des  orateurs  et  auditeurs,  la 
délimitation de l'objet, mais encore sur la durée des discours, leur ordre, la manière de conclure et 
sur  les,  conditions  dans  lesquelles  la  discussion  peut  être  reprise.  Ce  dernier  point  est  très 
important.  En  effet,  la  vie  sociale  exige  que  l'on  reconnaisse  l'autorité  de  la  chose  jugée.  Mais  la 
discussion  peut  être  reprise.  Et  cette  reprise  est  même  souvent  organisée,  de  sorte  qu'il  ne  faille 
pas  attendre  une  décision  particulière  dont  l'initiative  incomberait  à  quelqu'un  :  le  système 
bicaméral en fournit un exemple. » 
\bigskip
(1) Sur la tendance à l'unanimité,  cf. L. Festinger, Inforinal social communication,  Psychol. Rev., 
57, 1950, pp. 271-282, et expériences de L. Festinger et J. Thibaut, Interpersonal communication 
in  small  groups,  J.  of  abnormal  and  social  Psychol.,  46,1951,  pp.  92-99,de  K.W.Back,  Influence 
through social communication, J. of abn. and social Psychol., 46, 1951, pp. 9-23. 
(2) Démosthène, Harangues et plaidoyers politiques, t. 1 ; Première Olynthienne, 19. Cf. note de 
M. Croiset, P. 93, 
\bigskip
P 76-77 : « L'institutionalisation n'est pas toujours complète : toutes les nuances peuvent exister. 
Mais le plus souvent, il ne faut plus que, à chaque coup, une décision intervienne : la reprise est 
souvent prévue, on s'attend à la voir se produire, son organisation répond à des nécessités sociales 
profondes. Même si, pour que la reprise ait lieu, il faut une initiative, celle-ci est souvent réglée ; on 
est  invité  à  la  prendre  par  les  institutions  mêmes  :  l'ordre  judiciaire  avec  ses  cours  d'appel  et  de 
cassation est l'une des plus caractéristiques à cet égard. » 
\bigskip
P 77 : « Notons que les cas d'interdiction de la reprise ne sont pas limités au système juridique. On 
peut se référer au principe de la chose Jugée même en dehors des tribunaux : bien avant que son 
impossibilité  fût  démontrée,  la  recherche  de  la  quadrature  du  cercle  était  considérée  par 
l'Académie des Sciences de Paris comme définitivement hors de discussion. 
\bigskip
Ajoutons  d'ailleurs  que,  dans  la  vie  sociale,  il  est  rare  que  la  reprise  d'une  discussion  soit,  sans 
doute  aucun,  permise  ou  interdite.  Il  existe  toute  une  zone  intermédiaire  entre  l'interdiction 
absolue de reprise et la permission de reprise inconditionnée : cette zone est régie en grande partie 
par des traditions, des coutumes extrêmement complexes. C'est là un des aspects non négligeables 
de la vie d'une communauté. 
\bigskip
\bigskip
\bigskip
\bigskip
38 
\bigskip
L'interdiction  de  reprendre  certaines  discussions  peut  être  une  manifestation  d'intolérance  au 
même titre que l'interdiction de mettre en question certains problèmes. Toutefois, il subsiste une 
différence  majeure  :  c'est  qu'un  verdict  définitif,  quel  qu'il  soit,  aussi  longtemps  qu'il  est  conçu 
comme verdict, ne sera pas entièrement détaché de tout ce qui le précède. Ce que la vie sociale de 
la communauté traîne, dès lors, avec elle, c'est une décision, mais, en plus, les argumentations qui 
l'ont précédée. 
\bigskip
Ceci  se  rattache  d'ailleurs  à  un  problème  théorique  assez  grave  le  but  de  l'argumentation  étant 
d'obtenir  un  assentiment,  on  pourrait  dire  que  l'argumentation  vise  à  supprimer  les  conditions 
préalables à une argumentation future. Mais la preuve rhétorique n'étant jamais contraignante, le 
silence  imposé  ne  doit  pas  être  considéré  comme  définitif,  si,  par  ailleurs,  les  conditions  qui 
permettent une argumentation sont réalisées. 
\bigskip
Les institutions réglant les discussions ont  de l'importance parce que la pensée argumentative et 
l'action  qu'elle  prépare  ou  détermine  sont  intimement  liées.  C'est  à  cause  des  rapports  qu'elle 
possède  avec  l'action,  parce  que  l'argumentation  ne  se  déroule  pas  dans  le  vide,  mais  dans  une 
situation  socialement  et  psychologiquement  déterminée,  qu'elle  engage  pratiquement  ceux  qui  y 
participent. C'est aux problèmes que pose cet engagement que sera consacré le dernier paragraphe 
de cette Première Partie. » 
\bigskip
§ 14. ARGUMENTATION ET ENGAGEMENT 
\bigskip
78 :  « L'impossibilité  de  considérer  l'argumentation  comme  un  exercice  intellectuel  entièrement 
détaché de toute préoccupation d'ordre pratique oblige de transposer certaines notions concernant 
la  connaissance,  qui  ont  été  élaborées  dans  une  tout  autre  perspective  philosophique,  telle 
l'opposition  de  l'objectif  et  du  subjectif.  L'objectivité,  quand  elle  concerne  l'argumentation,  doit 
être repensée, réinterprétée, pour qu'elle puisse avoir un sens dans une conception qui se refuse à 
séparer une affirmation, de la personne de celui qui la pose. 
\bigskip
Bien des fois, lors d'un débat qui oppose les uns aux autres des partisans que l'on juge intéressés, 
de thèses opposées, on entend des gens demander que l'on fasse appel à des tiers qui trancheront 
le  débat  en  recourant  à  des  critères  objectifs.  Mais  suffit-il  d'être  complètement  étranger  aux 
intérêts en présence pour disposer d'un critère objectif qui s'imposerait à tous ? Si c'était le cas, ne 
serait-il  pas  plus  simple  de  réunir  en  un  volume  toutes  ces  règles  objectivement  valables  qui 
permettraient  de  résoudre  des  conflits  aussi  simplement  que  des  problèmes  d'arithmétique  ?  En 
fait, de pareils ouvrages existent, ce sont les divers traités de morale, de droit, les réglementations 
reconnues  dans  les  domaines  les  plus  divers.  Mais,  comme  on  sait,  ces  traités  et  ces 
réglementations ne sont pas d'une validité universelle et d'une univocité parfaite. » 
\bigskip
P 78-80 : « Si, malgré ces réglementations, des divergences peuvent se produire, de bonne foi, c'est 
ou  bien  parce  que  l'une  au  moins  des  parties  ne  reconnaît  pas  la  validité  d'une  certaine 
réglementation, ou bien parce que les réglementations admises donnent lieu à des interprétations 
différentes.  Les  difficultés  sont  encore  plus  grandes  quand  aucune  réglementation  ne  régit  la 
question, quand il s'agit de choisir le meilleur candidat pour un poste responsable, et que l'on n'est 
pas  d'accord  sur  des  critères  permettant  de  classer  les  candidats  disponibles,  quand  il  s'agit  de 
prendre la meilleure décision d'ordre politique et que celle-ci échappe à toute règle préexistante. 
Suffit-il de dire que l'on se place au point de vue de Sirius, que l'on est parfaitement désintéressé, 
pour  pouvoir  fournir  un  avis  objectivement  valable  ?  La  réaction  qu'une  pareille  intrusion  ne 
manquerait  pas  de  provoquer  de  la  part  des  partis  en  présence  sera  l'étonnement,  si  pas 
l'indignation,  de  ce  qu'un  étranger  au  débat  ose  se  mêler  de  ce  qui  ne  le  regarde  pas.  En  effet, 
comme  ces  débats  doivent  aboutir  à  une  décision,  qu'ils  doivent  déterminer  une  action,  être  un 
spectateur désintéressé ne confère pas pour autant le droit de participer à la discussion et d'influer 
sur le sens de son issue. Contrairement à ce qui se passe en science, où il suffit, pour résoudre un 
problème,  de  connaître  les  techniques  permettant  d'y  arriver,  il  faut,  pour  intervenir  dans  une 
\bigskip
\bigskip
\bigskip
39 
\bigskip
controverse  dont  l'issue  affectera  un  groupe  déterminé,  faire  partie  de  ce  groupe  ou  en  être 
solidaire. Là où une opinion exerce une influence sur l'action, l'objectivité ne suffit plus, à moins 
d'entendre par là le point de vue d'un groupe plus large qui englobe à la fois les adversaires et le « 
neutre ». Ce dernier est apte à juger non comme neutre, auquel chacun peut reprocher d'ailleurs sa 
neutralité au nom de principes communs de justice ou de droit, mais parce qu'il est impartial : être 
impartial ce n'est pas être objectif, c'est faire partie d'un même groupe que ceux que l'on Juge, sans 
avoir pris préalablement parti pour aucun d'eux. Dans bien des débats, le problème de savoir qui a 
qualité pour intervenir, voire pour juger, est pénible et délicat, parce que les uns ont pris parti, et 
que les autres ne sont pas membres du groupe. Lorsqu'il s'est agi de juger de l'attitude des officiers 
français qui avaient préféré leur loyalisme militaire à la poursuite de la guerre contre  l'Allemagne 
en 1940, les Français étaient mal venus d'en juger parce qu'ils avaient pris parti, les étrangers et 
particulièrement les neutres, parce qu'ils ne faisaient pas partie du groupe en cause. » 
\bigskip
P  80 :  « L'impartialité,  si  elle  est  conçue  comme  celle  d'un  spectateur,  peut  sembler  l'absence  de 
toute attraction, une recherche dénuée de participation aux débats, une attitude qui transcende les 
querelles.  Par  contre,  si  elle  doit  caractériser  un  agent,  c'est  plutôt  un  équilibre  des  forces,  une 
attention maximum aux intérêts en cause, mais répartie également entre les points de vue (1). 
\bigskip
L'impartialité se trouve ainsi, dans les domaines oil la pensée et l'action sont intimement mêlées, 
entre  l'objectivité qui  ne  donne  au  tiers  aucune  qualité  pour  intervenir,  et  l'esprit  partisan  qui  le 
disqualifie. 
\bigskip
On a trop souvent méconnu, sous l'empire d'un objectivisme abstrait, que la pensée qui détermine 
l'action a un statut différent des énoncés intégrés dans un système scientifique. Mais, d'autre part, 
il  est  essentiel  de  prévoir  une  possibilité  de  dissocier  nos  convictions  de  nos  intérêts  et  de  nos 
passions. 
\bigskip
C'est  presque  un  lieu  commun  que  l'insistance  sur  la  façon  dont  nos  espoirs  et  nos  désirs 
déterminent nos convictions.  
\bigskip
Tout ce qu'il y a d'hommes, nous dit Pascal, sont presque toujours emportés à croire non pas par 
la preuve, mais par l'agrément (2). » 
\bigskip
(1) Cf. Edwin N. Garlan, Legal realism and justice, p. 78. 
(2) Pascal, Bibl. de la Pléiade, De l'art de persuader, p. 376. 
\bigskip
P 80-81 : « et il cherche à expliquer ce phénomène en insistant sur le fait que 
\bigskip
les  choses  sont  vraies  ou  fausses,  selon  la  face  par  où  on  les  regarde.  La  volonté  qui  se  plaît  à 
l'une  plus  qu'à  l'autre,  détourne  l'esprit  de  considérer  les  qualités  de  celles  qu'elle  n'aime  pas  à 
voir;  et  ainsi  l'esprit,  marchant  d'une  pièce  avec  la  volonté,  s'arrête  à  regarder  la  face  qu'elle 
aime; et ainsi il en juge par ce qu'il y voit (1). » 
\bigskip
(1) Pascal, ibid., Pensées, 472 (141), p. 962 (99 éd. Brunschvicg). 
\bigskip
P  81 :  « William  James  justifiait  les  opinions  qui  favorisent  nos  désirs,  car  en  renforçant  ces 
derniers,  elles  rendent  plus  probables  leurs  chances  de  succès  (2).  D'autres  écrivains,  plus 
rationalistes,  décomptent  les  effets  de  ce  facteur  de  désirabilité,  qu'ils  considèrent  comme 
responsable du caractère irrationnel de nos opinions (3). 
\bigskip
Mais  il  ne  s'agit,  dans  les  deux  cas,  que  d'hypothèses  d'ordre  général,  dont  la  vérification  est 
difficile, lorsque manquent les critères d'une opinion « objectivement fondée ». Aussi une étude, 
comme  celle  de  Lund  (4)  qui  montre  une  corrélation  de  0,88  entre  la  désirabilité  de  certaines 
\bigskip
\bigskip
\bigskip
40 
\bigskip
thèses  et  le  degré  de  conviction  qu'elles  inspirent,  tandis  que  la  corrélation  serait  faible  entre 
conviction et connaissance, ou entre conviction et éléments de preuve, a-t-elle été critiquée par le 
sociologue américain Bird en des termes non dépourvus d'ironie : 
\bigskip
je  crains,  écrit-il,  que  l'analyse  des  coefficients  de  corrélation  ne  laisse  grande  place  à 
l'imagination, de telle sorte que le désir peut avoir déterminé la croyance que le désir détermine 
la croyance (5). » 
\bigskip
(2) W. JAMES, Essays in Pragmatism p. 31. 
(3) R. CRAWSHAY-WILLiAms, The comforts of unreason, pp. 8 et suiv. 
(4) LUND, The Psychology of belief, J. of abn. and social Psychology, XX, avr. et juill. 1925. 
(5) Ch. BIRD, Social Psychology, p. 211. 
\bigskip
P  81-82 :  « Chaque  fois  qu'il  importe  de  réfuter  l'accusation  que  ce  sont  nos  désirs  qui  ont 
déterminé nos croyances, il est indispensable de fournir des preuves, non de notre objectivité, ce 
qui  est  irréalisable,  mais  de  notre  impartialité,  en  indiquant  les  circonstances  où,  dans  une 
situation  analogue,  nous  avons  agi  contrairement  à  ce  qui  pouvait  paraître  notre  intérêt,  et  en 
précisant  si  possible  la  règle  ou  les  critères  que  nous  suivons,  lesquels  seraient  valables  pour  un 
groupe  plus  large  qui  engloberait  tous  les  interlocuteurs  et,  à  la  limite,  s'identifierait  avec 
l'auditoire universel. » 
\bigskip
P  82 :  « Il  ne  faut  pourtant  jamais  oublier  que,  même  dans  ce  cas,  c'est  sa  propre  conception  de 
l'auditoire universel que l'on présente et que les thèses que l'on prétend devoir être valables pour 
tout le monde pourraient trouver des détracteurs, qui ne sont pas nécessairement insensés ou de 
mauvaise  foi.  Ne  pas  en  convenir  serait  s'exposer  au  reproche  de  fanatisme.  Quand  il  s'agit  de 
vérités,  dont  l'établissement  fait  appel  à  des  critères  reconnus  indiscutables,  comme  on  ne  se 
trouve pas dans une situation où le recours à l'argumentation est possible, il ne peut être question 
de  fanatisme.  Le  fanatique  est  celui  qui,  adhérant  à  une  thèse  contestée,  et  dont  la  preuve 
indiscutable ne peut être fournie, refuse néanmoins d'envisager la possibilité de la soumettre à une 
libre discussion, et par conséquent refuse les conditions préalables qui permettraient, sur ce point, 
l'exercice de l'argumentation. 
\bigskip
En assimilant l'adhésion à une thèse à la reconnaissance de la vérité absolue de celle-ci, on aboutit 
parfois  non  au  fanatisme,  mais  au  scepticisme.  Celui  qui  exige  d'une  argumentation  qu'elle 
fournisse  des  preuves  contraignantes,  des  preuves  démonstratives,  et  qui  ne  se  contente  pas  de 
moins  pour  adhérer  à  une  thèse,  méconnaît  autant  que  le  fanatique  le  caractère  propre  de  la 
démarche  argumentative.  Celle-ci,  parce  qu'elle  tend  précisément  à  justifier  des  choix,  ne  peut 
fournir  des  justifications  qui  tendraient  à  montrer  qu'il  n'y  a  pas  de  choix,  mais  qu'une  seule 
solution s'offre à ceux qui examinent le problème. » 
\bigskip
P  82-83 :  « La  preuve  rhétorique  n'étant  jamais  tout  à  fait  nécessaire,  l'esprit  qui  donne  son 
adhésion  aux  conclusions  d'une  argumentation,  le  fait  par  un  acte  qui  l'engage  et  dont  il  est 
responsable.  Le  fanatique  accepte  cet  engagement,  mais  à  la  manière  de  quelqu'un  qui  s'incline 
devant une vérité absolue et irréfragable; le sceptique refuse cet engagement sous prétexte qu'il lie 
lui paraît pas pouvoir être définitif. Il refuse d'adhérer parce qu'il se fait de l'adhésion une idée qui 
ressemble à celle du fanatique : l'un et l'autre méconnaissent que l'argumentation vise à un choix 
entre  des  possibles  ;  en  proposant  et  justifiant  leur  hiérarchie,  elle  vise  à  rendre  rationnelle  une 
décision. Fanatisme et scepticisme nient ce rôle de l'argumentation dans nos décisions. Ils tendent 
tous  deux  à  laisser,  à  défaut  de  raison  contraignante,  libre  champ  à  la  violence,  en  récusant 
l'engagement de la personne. » 
\bigskip
\bigskip
\bigskip
41 
\bigskip
DEUXIEME PARTIE :  LE POINT DE DEPART DE L’ARGUMENTATION 
\bigskip
CHAPITRE PREMIER : L'ACCORD 
\bigskip
§ 15. LES PREMISSES DE L'ARGUMENTATION 
\bigskip
P  87 :  « Notre  analyse  de  l'argumentation  concernera  d'abord  ce  qui  est  admis  comme  point  de 
départ de raisonnements et ensuite la manière dont ceux-ci se développent, grâce à un ensemble 
de  procédés  de  liaison  et  de  dissociation.  Cette  division,  indispensable  pour  l'exposé,  ne  doit  pas 
être mal comprise. En effet, le déroulement aussi bien que le point de départ de l'argumentation 
supposent  accord  de  l'auditoire.  Cet  accord  porte  tantôt  sur  le  contenu  de  prémisses  explicites, 
tantôt sur les liaisons particulières utilisées, tantôt sur la façon de se servir de ces liaisons : d'un 
bout  à  l'autre,  l'analyse  de  l'argumentation  concerne  ce  qui  est  censé  admis  par  les  auditeurs. 
D'autre  part,  le  choix  même  des  prémisses  et  leur  formulation,  avec  les  aménagements  qu'ils 
comportent, ne sont que rarement exempts de valeur argumentative : il s'agit d'une préparation au 
raisonnement qui plus qu'une mise en place des éléments, constitue déjà un premier pas dans leur 
utilisation persuasive. » 
\bigskip
P 87-88 : « L'orateur, en utilisant les prémisses qui serviront de fondement à sa construction, table 
sur l'adhésion de ses auditeurs aux propositions de départ, mais ceux-ci peuvent la lui refuser, soit 
parce  qu'ils  n'adhèrent  pas  à  ce  que  l'orateur  leur  présente  comme  acquis,  soit  parce  qu'ils 
perçoivent  le  caractère  unilatéral  du  choix  des  prémisses,  soit  parce  qu'ils  sont  heurtés  par  le 
caractère tendancieux de leur présentation. C'est en raison de ce que la critique d'un même énoncé 
peut  se  porter  sur  trois  plans  différents  que  notre  analyse  des  prémisses  comportera  trois 
chapitres,  consacrés  successivement  à  l'accord  concernant  les  prémisses  à  leur  choix  et  à  leur 
présentation. » 
\bigskip
P 88 : « Nous traiterons pour commencer de la matière des accords pouvant servir de prémisses. 
Notre  examen  ne  tendra  évidemment  pas  à  établir  l'inventaire  de  tout  ce  qui  est  susceptible  de 
constituer objet de croyance ou d'adhésion : nous nous demanderons quels sont les types d'objets 
d'accord qui jouent un rôle différent dans le processus argumentatif. Nous croyons qu'il sera utile, 
à ce point de vue, de grouper ces objets en deux catégories, l'une relative au réel, qui comporterait 
les faits, les vérités et les présomptions, l'autre relative au préférable, qui contiendrait les valeurs, 
les hiérarchies et les lieux du préférable. 
\bigskip
La  conception  que  l'on  se  fait  du  réel peut,  dans  de  larges  limites,  varier  selon  les  vues 
philosophiques que l'on professe. Cependant tout ce qui, dans l'argumentation, est censé porter sur 
le  réel,  se  caractérise  par  une  prétention  de  validité  pour  l'auditoire  universel.  Par  contre  ce  qui 
porte  sur  le  préférable,  ce  qui  détermine  nos  choix  et  qui  n'est  pas  conforme  à  une  réalité 
préexistante,  sera  lié  à  un  point  de  vue  déterminé  que  l'on  ne  peut  identifier  qu'avec  celui  d'un 
auditoire particulier, aussi vaste soit-il. 
\bigskip
On  pourrait  aisément  contester  le  bien-fondé  d'un  classement  en  types  d'objets  d'accord,  tel  que 
nous le proposerons, mais nous croyons difficile de ne pas y recourir si l'on veut faire une analyse 
technique  et  portant  sur  les  argumentations  telles  qu'elles  se  présentent.  Chaque  auditoire 
n'admettra  évidemment  qu'un  nombre  déterminé  d'objets  relevant  de  chacun  de  ces  types.  Mais 
des objets de chaque type se retrouvent dans les argumentations les plus diverses. Ils se retrouvent 
d'ailleurs  également  comme  types  d'objets  de  désaccord,  c'est-à-dire  comme  points  sur  lesquels 
peut porter un litige. » 
\bigskip
P  89 :  « Outre  la  matière  des  accords,  deux  ordres  de  considérations  nous  retiendront  dans  ce 
premier  chapitre  :  il  s'agit  des  conditions  dans  lesquelles  se  trouvent  les  prémisses,  soit  à  raison 
d'accords  spéciaux  qui  régissent  certains  auditoires,  soit  à  raison  de  l'état  de  la  discussion.  Le 
premier ordre de considérations est plutôt statique, en ce sens qu'il étudie le caractère des accords 
de certains auditoires constitués ; l'autre est plus dynamique, en ce sens qu'il s'attache aux accords 
en tant que liés au progrès de la discussion. Mais ce qui nous intéressera dans ce dynamisme, étant 
donné que nous étudions les prémisses, ce sera de montrer l'effort de l'orateur pour rechercher les 
manifestations explicites ou implicites d'une adhésion sur laquelle il puisse tabler. » 
\bigskip
A) LES TYPES D'OBJET D'ACCORD 
\bigskip
§ 16. LES FAITS ET LES VERITES 
\bigskip
P 89-90 : « Parmi les objets d'accord appartenant au réel nous distinguerons, d'une part les faits et 
vérités, d'autre part les présomptions. Il ne serait ni possible ni conforme à notre propos de donner 
du fait une définition qui permette, en tous temps et en tous lieux, de classer telle ou telle donnée 
concrète comme étant un fait. Il nous faut, au contraire, insister sur ce que, dans l'argumentation, 
la notion de « fait » est caractérisée uniquement par l'idée que l'on a d'un certain genre d'accords 
au sujet de certaines données, celles qui se réfèrent à une réalité objective et qui désigneraient, en 
dernière  analyse,  pour  citer  H.  Poincaré  (1)  «  ce  qui  est  commun  à  plusieurs  êtres  pensants  et 
pourrait  être  commun  à  tous  ».  Ces  derniers  mots  suggèrent  immédiatement  ce  que  nous  avons 
appelé l'accord de l'auditoire universel. La manière de concevoir cet auditoire, les incarnations de 
cet auditoire que l'on reconnaît, seront donc déterminantes pour décider de ce qui, dans tel ou tel 
cas, sera considéré comme un fait et qui se caractérise par une adhésion de l'auditoire universel, 
adhésion  telle  qu'il  soit  inutile  de  la  renforcer.  Les  faits  sont  soustraits,  tout  au  moins 
provisoirement, à l'argumentation, c'est-à-dire que l'intensité d'adhésion n'a pas à être augmentée, 
ni  à  être  généralisée,  et  que  cette  adhésion  n'a  nul  besoin  de  justification.  L'adhésion  au  fait  ne 
sera, pour l'individu, qu'une réaction subjective à quelque chose qui s'impose à tous. » 
\bigskip
(1) H. Poincarré, La valeur de la science, Introduction, p. 65. 
\bigskip
P 90 : « Nous ne sommes en présence d'un fait, au point de vue argumentatif, que si nous pouvons 
postuler à son sujet un accord universel, non controversé. Mais dès lors aucun énoncé n'est assuré 
de jouir définitivement de ce statut, car l'accord est toujours susceptible d'être remis en question 
(1) et l'une des parties au débat peut refuser la qualité de fait à ce qu'affirme son adversaire. Il y 
aura  donc  deux  façons  normales  pour  un  événement  de  perdre  le  statut  de  fait  :  quand  sont 
soulevés des doutes au sein de l'auditoire auquel il était présenté, et quand on étend cet auditoire 
en lui adjoignant d'autres membres, dont on reconnaît la qualité pour en juger et qui n'admettent 
pas  qu'il  s'agit  d'un  fait.  Ce  deuxième  processus  entre  en  jeu  à  partir  du  moment  où  l'on  peut 
montrer  efficacement  que  l'auditoire  qui  admettait  le  fait  n'est  qu'un  auditoire  particulier,  aux 
conceptions duquel s'opposent celles des membres d'un auditoire élargi. » 
\bigskip
(1)  Cf.  Ch.  Perelman  et  L.  Olbrechts-Tyteca,  Rhétorique  et  philosophie,  p.  2  (Logique  et 
rhétorique), p. 51 (Acte et personne dans l'argumentation). 
\bigskip
P  90-91 :  « Nous  ne  tablons  sur  aucun  critère  qui  nous  permette,  en  toutes  circonstances  et 
indépendamment de l'attitude des auditeurs, d'affirmer que quelque chose est un fait. Néanmoins 
nous  pouvons  reconnaître  qu'il  existe  certaines  conditions  qui  favorisent  cet  accord,  qui 
permettent aisément de défendre le fait contre la méfiance ou la mauvaise volonté d'un adversaire 
:  ce  sera  le  cas,  notamment,  lorsque  l'on  dispose  d'un  accord  au  sujet  des  conditions  de 
vérification; cependant dès que nous devons faire effectivement intervenir cet accord nous sommes 
en pleine argumentation. Le fait comme prémisse est un fait non-controversé. » 
\bigskip
P 91 : « La simple mise en question suffit donc pour faire perdre à un énoncé son statut privilégié. 
Mais, le plus souvent, pour combattre le prestige de ce qui a été admis comme fait,  l'interlocuteur 
ne se contentera pas d'une simple dénégation que l'on pourrait trouver tout bonnement ridicule. Il 
s'efforcera  de  justifier  son  attitude  soit  en  montrant  l'incompatibilité  de  l'énoncé  avec  d'autres 
faits,  et  en  condamnant  le  premier  au  nom  de  la  cohérence  du  réel,  soit  en  montrant  que  le  soi-disant  fait  ne  constitue  que  la  conclusion  d'une  argumentation,  qui,  comme  telle,  n'est  pas 
contraignante. 
\bigskip
Dès  qu'il  n'est  plus  utilisé  comme  point  de  départ  possible,  mais  comme  conclusion  d'une 
argumentation, le fait perd son statut : il pourra le recouvrer, mais à condition d'être détaché du 
contexte  argumentatif,  c'est-à-dire  si,  à  nouveau,  on  se  trouve  en  présence  d'un  accord  qui  ne 
dépend pas des conditions argumentatives permettant l'établissement de sa preuve. Notons que la 
perte du statut de fait, par insertion dans un contexte argumentatif dont il n'est plus la base, mais 
l'une  des  conclusions,  se  présente  fréquemment  en  philosophie,  où  la  construction  d'un  système 
argumentatif amène fort souvent à lier les faits, auparavant les plus banalement admis comme tels, 
à une argumentation qui prétende les fonder. » 
\bigskip
P 91-92 : « Les faits que l'on admet peuvent être soit des faits d'observation - et ce sera, peut-être, 
la fraction la plus importante des prémisses - soit des faits supposés, convenus, des faits possibles 
on  probables.  Il  y  a  là  une  masse  considérable  d'éléments  qui  s'impose  ou  que  l'on  s'efforce 
d'imposer à l'auditeur. Les uns, comme les autres, peuvent être récusés et perdre leur statut de fait. 
Mais aussi longtemps qu'ils en jouissent, ils devront se conformer aux structures du réel admises 
par l'auditoire, et devront se défendre contre d'autres faits qui viendraient en concurrence avec eux 
dans un même contexte argumentatif. » 
\bigskip
P  92 :  « Nous  appliquons,  à  ce  que  l'on  nomme  des  vérités,  tout  ce  que  nous  venons  de  dire  des 
faits. Ou parle généralement de faits pour désigner des objets d'accord précis, limités; par contre, 
on  désignera  de  préférence,  sous  le  nom  de  vérités  des  systèmes  plus  complexes,  relatifs  à  des 
liaisons entre des faits, qu'il s'agisse de théories scientifiques ou de conceptions philosophiques ou 
religieuses transcendant l'expérience. 
\bigskip
Bien  que,  ainsi  que  le  souligne  Piaget,  les  données  psychologiques  actuellement  connues  ne 
permettent même pas d'imaginer que l'on puisse atteindre des faits isolés (1), la distinction entre 
faits  et  vérités  nous  paraît  opportune  et  légitime  pour  notre  objet,  parce  qu'elle  correspond  à 
l'usage  habituel  de  l'argumentation  qui  s'appuie,  tantôt  sur  des  faits,  tantôt  sur  des  systèmes  de 
portée  plus  générale.  Mais  nous  n'aimerions  pas  trancher,  une  fois  pour  toutes,  le  problème 
philosophique  des  rapports  entre  faits  et  vérités  :  ces  rapports  caractérisent  des  conceptions 
d'auditoires différents. Pour les uns le fait s'oppose à la vérité théorique comme le contingent au 
nécessaire, pour d'autres comme le réel au schématique ; on peut aussi concevoir leur rapport de 
telle façon que l'énoncé d'un fait soit une vérité et que toute vérité énonce un fait. » 
\bigskip
(1) J. Piaget, Traité de logique, p. 30. 
\bigskip
P  92-93 :  « Lorsqu'un  primat  des  faits  ou  des  vérités  résulte  de  la  manière  de  concevoir  leurs 
rapports réciproques, faits et vérités ne peuvent être utilisés tout à fait au même titre comme point 
de départ de  l'argumentation. L'un  des deux seulement est censé jouir pleinement de l'accord de 
l'auditoire universel. Mais n'oublions pas que ce primat n'est généralement invoqué que lorsque les 
deux  types  d'objets  sont  confrontés.  Par  contre,  dans  la  pratique  journalière,  faits  et  systèmes 
peuvent être pris en considération indifféremment comme point de départ de l'argumentation. » 
\bigskip
P  93 :  « Le  plus  souvent,  on  utilise  faits  et  vérités  (théories  scientifiques,  vérités  religieuses,  par 
exemple)  comme  des  objets  d'accord  distincts,  mais  entre  lesquels  existent  des  liens  qui 
permettent le transfert de l'accord : la certitude du fait A, combiné avec la croyance au système S, 
entraîne  la  certitude  du  fait  B,  c'est-à-dire  qu'admettre  le  fait  A,  plus  la  théorie  S,  revient  à 
admettre B. 
\bigskip
Au lieu d'être admis comme un lien certain, le rapport entre A et B peut n'être que probable : on 
admettra  que  l'apparition  du  fait  A  entraîne,  avec  une  certaine  probabilité,  l'apparition  de  B. 
\bigskip
\bigskip
\bigskip
44 
\bigskip
Quand  le  degré  de  probabilité  de  B  peut  être  calculé  en  fonction  de  faits  et  d'une  théorie  sur 
lesquels  l'accord  est  incontesté,  la  probabilité  envisagée  n'est  pas  l'objet  d'un  accord  d'une  autre 
nature que l'accord concernant le fait certain. C'est la raison pour laquelle 
\bigskip
nous  assimilons  à  des  accords  sur  les  faits,  ceux  concernant  la  probabilité des  événements  d'une 
certaine espèce, dans la mesure où il s'agit de probabilités calculables. 
\bigskip
Kneebone (1) souligne très justement à cet égard que la vraisemblance «(Iikelihood») s'applique à 
des  propositions,  aux  conclusions  inductives  notamment,  et  par  là,  n'est  pas  une  quantité 
mesurable tandis que la probabilité est un rapport numérique entre deux propositions s'appliquant 
à des données empiriques spécifiques, bien définies, simples. Le domaine de la probabilité est donc 
lié à celui des faits et vérités, et se caractérise pour chaque auditoire en fonction de ceux-ci. » 
\bigskip
(1) G. T. Kneebone, Induction and Probability, Proceedings of the Aristotelian Society, New Series, 
vol. L, p. 36. 
\bigskip
§ 17. LES PRESOMPTIONS 
\bigskip
P 93-94 : « Outre les faits et les vérités, tous les auditoires admettent des présomptions. Celles-ci 
jouissent  également  de  l'accord  universel,  toutefois  l'adhésion  aux  présomptions  n'est  pas 
maximum,  on  s'attend  à  ce  que  cette  adhésion  soit  renforcée,  à  un  moment  donné,  par  d'autres 
éléments. Ceux qui admettent la présomption escomptent même, d'habitude, ce renforcement. » 
\bigskip
P 94 : « Une argumentation préalable peut tendre à établir qu'il existe certaines présomptions de 
même qu'une argumentation peut tendre à montrer que l'on est en présence d'un fait. Mais comme 
les  présomptions,  de  par  leur  nature,  sont  sujettes  à  être renforcées,  il  semble  que,  sur  ce  point, 
une nuance importante doive être soulignée : tandis que la justification d'un fait risque toujours de 
diminuer son statut, il n'en est pas de même en ce qui concerne les présomptions ; pour conserver 
leur  statut  point  n'est  besoin  donc  de  les  détacher  d'une  argumentation  préalable  éventuelle. 
Toutefois,  le  plus  souvent  les  présomptions  sont  admises  d'emblée,  comme  point  de  départ  des 
argumentations. Nous verrons même que certaines peuvent être imposées à des auditoires liés par 
des conventions. 
\bigskip
L'usage des présomptions aboutit à des énoncés dont la vraisemblance ne dérive point d'un calcul 
appliqué  à  des  données  de  fait  et  ne  pourrait  dériver  de  pareil  calcul  même  perfectionné.  Bien 
entendu,  les  frontières  entre  probabilité  calculable  -  au  moins  en  principe  -  et  vraisemblance 
peuvent varier selon les conceptions philosophiques. Mais pour ramener les énoncés résultant de 
présomptions  à  des  énoncés  de  probabilité  calculable,  il  faudrait  en  modifier,  en  tout  cas,  la 
formulation  et  la  portée  argumentative.  Citons  quelques  présomptions  d'usage  courant  :  la 
présomption que la qualité d'un acte manifeste celle de la personne qui l'a posé ; la présomption de 
crédulité  naturelle qui  fait que  notre premier  mouvement  est  d'accueillir  comme  vrai  ce que  l'on 
nous  dit,  et  qui  est  admise  aussi  longtemps  et  dans  la  mesure  où  nous  n'avons  pas  de  raison  de 
nous  méfier  ;  la  présomption  d'intérêt  d'après  laquelle  nous  concluons  que  tout  énoncé  porté  à 
notre  connaissance  est  censé  nous  intéresser;  la  présomption  concernant  le  caractère  sensé  de 
toute action humaine. » 
\bigskip
P 94-95 : « Les présomptions sont liées dans chaque cas particulier au normal et au vraisemblable. 
Une  présomption  plus  générale  que  toutes  celles  que  nous  avons  mentionnées,  c'est  qu'il  existe 
pour chaque catégorie de faits et notamment pour chaque catégorie de comportements, un aspect 
considéré comme normal qui peut servir de base aux raisonnements. L'existence même de ce lien 
entre  les  présomptions  et  le  normal  constitue  une  présomption  générale  admise  par  tous  les 
auditoires. On présume, jusqu'à preuve du contraire, que le normal est ce qui se produira, ou s'est 
produit,  ou  plutôt  que  le  normal  est  une  base  sur  laquelle  nous  pouvons  tabler  dans  nos 
raisonnements  (1).  Cette  base  correspond-elle  à  une  représentation  définissable  en  termes  de 
\bigskip
\bigskip
\bigskip
45 
\bigskip
distribution statistique des fréquences ? Non, sans doute. Et c'est l'une des raisons qui nous oblige 
à  parler  de  présomptions  et  non  de  probabilité  calculée.  Tout  au  plus  peut-on  dire  que,  grosso 
modo, l'idée que nous nous faisons du normal, dans nos raisonnements  -en dehors des cas où le 
calcul des fréquences est effectivement pratiqué et où l'idée courante du normal est éliminée pour 
faire  place  à  celle  de  caractéristiques  d'une  distribution  -  oseille  entre  différents  aspects.  Nous 
servant  du  langage  statistique  pour  décrire  ces  aspects,  nous  dirons  que  la  notion  de  normal 
recouvre le plus souvent, en même temps et d'une façon diversement accentuée, suivant les cas, les 
idées de moyenne, de mode et aussi de partie plus ou moins étendue d'une distribution. » 
\bigskip
(1) Cf. F. Gonseth, La notion du normal, Dialectica, 3, pp. 243-252. 
\bigskip
P 95 : « Ainsi le normal, lorsqu'il s'agit de la capacité que l'on exige d'un chauffeur, est tout ce qui 
dépasse un minimum ; lorsqu'il s'agit de la vitesse d'une voiture qui a renversé un piéton, c'est tout 
ce qui est inférieur à un maximum. Dans d'autres cas, l'attention porte sur toute la partie centrale 
de  la  courbe  de  distribution  et  le  normal  s'oppose  à  l'exceptionnel  :  si  nous  imaginons  une 
distribution  binomiale  le  normal  porterait  le  plus  souvent  sur  le  mode  entouré  d'une  certaine 
marge dans les deux sens. » 
\bigskip
P 96 : « Comme caractéristique d'une population (au sens large de ce terme et quels qu'en soient 
les  éléments,  animés  ou  inanimés,  objets  ou  comportements),  c'est  le  mode  qui  certainement 
domine dans toutes les présomptions basées sur l'habituel plutôt que la  moyenne ; c'est le mode 
que  nous  retrouvons  comme  point  de  comparaison  dans  les  appréciations  de  grand  et  de  petit  ; 
c'est lui que nous trouvons à la base de tous les raisonnements sur le comportement, à la base des 
présomptions qui peuvent justifier l'Einfühlung et que les orateurs utilisent si largement lorsqu'ils 
supplient l'auditoire de se mettre à la place de leurs protégés. 
\bigskip
Si  la  présomption  basée  sur  le  normal  ne  peut  que  rarement  être  ramenée  à  une  évaluation  de  f 
fréquences, et à l'utilisation de caractéristiques déterminées de distribution statistique, il n'en est 
pas moins utile d'éclairer la notion usuelle du normal en montrant qu'il dépend toujours du groupe 
de référence, c'est-à-dire de la catégorie totale en considération de laquelle il s'établit. Il faut noter 
que ce groupe - qui est souvent un groupe social -n'est presque jamais explicitement désigné. Peut-
être les interlocuteurs y songent-ils rarement; il est clair néanmoins que toutes les présomptions 
basées sur le normal impliquent un accord au sujet de ce groupe de référence. 
\bigskip
Ce  groupe  est,  dans  la  plupart  des  cas,  éminemment  instable.  En  effet,  si  certains  individus 
s'écartent  dans  leur  comportement  de  ce  qui  est  considéré  comme  normal,  leur  conduite  peut 
modifier  ce  normal  (statistiquement,  nous  dirons  qu'il  peut  modifier  la  moyenne)  ;  mais  si 
l'individu s'en écarte au delà de certaines limites, il sera exclu du groupe et, dès lors, c'est le groupe 
de référence qui sera modifié. L'individu sera considéré comme fou et exclu de la communauté, ou 
comme  trop  mal  élevé  pour  faire  partie  de  ceux  que  fréquentent  les  gens  de  bien.  Citons  un 
procédé qui nous paraît basé sur pareille exclusion : 
\bigskip
Vous, monsieur, dit Bloch, en se tournant vers M. d'Argencourt a qui on l'avait nommé en même 
temps que les autres personnes, vous êtes certainement dreyfusard : à l'étranger tout le monde 
l'est. » 
\bigskip
P  97 :  « C'est  une  affaire  qui  ne  regarde  que  les  Français  entre  eux,  n'estce  pas  ?  répondit  M. 
d'Argencourt  avec  cette  insolence  particulière  qui  consiste  à  prêter  à  l'interlocuteur  une  opinion 
qu'on sait manifestement qu'il ne partage pas, puisqu'il vient d'en émettre une opposée. » (1) 
\bigskip
C'est exclure l'interlocuteur des gens comme il faut chez qui cette opinion est normale et chez qui 
on a donc droit de la présumer. 
\bigskip
\bigskip
\bigskip
\bigskip
46 
\bigskip
Non seulement le groupe de référence est instable, mais la manière de l'envisager peut varier : on 
songe parfois au groupe réel ou fictif qui agit d'une certaine façon,  parfois à l'opinion commune, 
concernant ceux qui agissent de telle façon, ou à l'opinion de ceux que l'on considère comme porte-
parole  de  cette  opinion  commune,  ou  à  ce  que  l'on  considère  communément  comme  étant 
l'opinion  de  ces  porte-parole.  Ces  diverses  conceptions  du  groupe  de  référence  joueront  souvent, 
dans l'argumentation, l'une contre l'autre. 
\bigskip
Dans  toute  l'argumentation  judiciaire  interviennent  les  variations  du  groupe  de  référence. 
L'ancienne opposition entre l'argumentation par les motifs du crime et par la conduite de l'accusé 
correspond  à  deux  groupes  de  référence  différents  :  le  premier  plus  large,  le  deuxième  plus 
spécifique, c'est-à-dire que, dans le second cas, on tire les présomptions de ce qui est normal pour 
des hommes qui se sont, toute leur vie, conduits comme l'accusé. » 
\bigskip
(1) M. Proust, A la recherche du temps perdu, vol. VII : Le côté de Guermantes, p. 85. 
\bigskip
P  97-98 :  « D'une  manière  générale  tout  complément  d'information  peut  provoquer  un 
changement du groupe de référence, et par là modifier notre conception de ce qui est remarquable, 
monstrueux.  Le  rôle  de  l'orateur  sera  souvent  de  favoriser  cette  modification  en  faisant  part 
d'informations  nouvelles.  Lorsque  l'avocat  du  prévenu  fait  état  de  circonstances  atténuantes,  il 
suggère  le  changement  de  groupe  de  référence  :  le  comportement  présumé,  celui  qui  servira  de 
critère  pour juger  le  prévenu,  sera  désormais  le  comportement  normal  de  ce  nouveau  groupe  de 
référence. Par ailleurs, si le cercle de nos relations s'étend, des dons naturels qui nous paraissent 
remarquables  perdront  ce  caractère,  parce  que  nous  aurons  l'occasion  de  les  rencontrer  plus 
souvent. Inversement, qu'un décès se produise parmi les habitants d'une grande ville, rien de plus 
régulier  ;  que  le  même  événement  affecte  le  petit  cercle  de  nos  relations,  et  on  le  trouve 
extraordinaire. C'est l'opposition entre les deux groupes de référence qui permet, à la fois, aux uns 
de s'étonner qu'un mortel soit mort et à d'autres de s'étonner de cet étonnement. » 
\bigskip
P 98 : « Si donc les présomptions liées au normal sont un objet d'accord, il faut, en outre, qu'il y ait 
un accord sous-jacent quant au groupe de référence  de ce  normal. La  plupart des arguments qui 
tendent  à  montrer  qu'il  est  extraordinaire,  contraire  à  toute  présomption,  que  l'homme  ait  pu 
trouver un globe à sa mesure, supposent, sans le  dire le plus souvent, que le groupe de référence, 
celui  des  globes  habitables,  est  extrêmement  réduit. Par  contre  un  astronome  comme  Hoyle,  qui 
estime  que  les  mondes  habitables  sont  extrêmement  nombreux,  dira  avec  humour,  que  si  notre 
globe n'était pas habitable, nous serions ailleurs (1). 
\bigskip
Souvent, les notions mêmes utilisées dans l'argumentation supposent un ou plusieurs groupes de 
référence  déterminant  le  normal,  sans  que  cela  soit  explicite;  c'est  le  cas,  par  exemple,  pour  la 
notion  juridique  de  négligence  :  les  discussions  relatives  à  cette  notion  feront  seules  apparaître 
l'existence de ces groupes. » 
\bigskip
(1) F. Hoyle, The nature of the universe, P. 90. 
\bigskip
P 98-99 : « L'accord basé sur la présomption du normal est censé valable pour l'auditoire universel 
au  même  titre  que  l'accord  sur  les  faits  avérés  et  les  vérités.  Aussi  cet  accord  est  souvent 
malaisément dis cernable de l'accord  sur des faits. Les faits présumés sont, à un moment donné, 
traités  comme  équivalents  à  des  faits  observés,  et,  peuvent  servir,  au  même  titre  qu'eux,  comme 
prémisse à des argu. mentations. Ce, bien entendu, jusqu'à mise en discussion de la présomption. 
Un saut s'est donc opéré, par lequel le normal arrive à coïncider avec quelque chose d'unique, qui 
n'est  arrivé  qu'une  fois  et  n'arrivera  plus  jamais.  Notons  qu'en  précisant  de  plus  en  plus  les 
conditions auxquelles doivent satisfaire les membres du groupe de référence, on pourrait aboutir, 
effectivement,  à  réduire  ce  dernier  à  un  seul  individu.  Néanmoins,  même  alors,  la  présomption 
concernant la conduite de cet individu et la conduite réelle de ce dernier ne se confondent pas et 
\bigskip
\bigskip
\bigskip
47 
\bigskip
l'étrange saut en question, qui permet de raisonner sur les faits présumés de la même manière que 
sur des faits observés, subsisterait encore. » 
\bigskip
§ 18. LES VALEURS 
\bigskip
P  99 :  « A  côté  (les faits,  des  vérités, et  des  présomptions,  caractérisés  par  l'accord  de  l'auditoire 
universel, il faut faire place, dans notre inventaire, à des objets d'accord à propos desquels on ne 
prétend qu'à l'adhésion de groupes particuliers : ce sont les valeurs, les hiérarchies, et les lieux du 
préférable. » 
\bigskip
P 99-100 : « Être d'accord à propos d'une valeur, c'est admettre qu'un objet, un être ou un idéal, 
doit exercer sur l'action et les dispositions à l'action une influence déterminée, dont on peut faire 
état  dans  une  argumentation,  sans  que  l'on  considère  cependant  que  ce  point  de  vue  s'impose  à 
tout le monde. L'existence des valeurs, comme objets d'accord permettant une communion sur des 
façons  particulières  d'agir,  est  liée  à  l'idée  de  multiplicité  des  groupes.  Pour  les  Anciens,  les 
énoncés  concernant ce que nous appelons des valeurs, dans la mesure oh ils n'étaient pas traités 
comme des vérités indiscutables, étaient englobés avec toute espèce d'affirmations vraisemblables 
dans le groupe indifférencié des opinions. C'est encore la manière dont les envisage Descartes dans 
les maximes de sa morale provisoire :  
\bigskip
Et ainsi, les actions de la vie ne souffrant souvent aucun délai, c' est une vérité très certaine que, 
lorsqu'il n'est pas en notre pouvoir de discerner les plus vraies opinions, nous devons suivre les 
plus  probables  ;  ...  et  les  considérer  après,  non  plus  comme  douteuses,  en  tant  qu'elles  se 
rapportent à la pratique, mais comme très vraies et très certaines, à cause que la raison qui nous 
y a fait déterminer se trouve telle (1). » 
\bigskip
(1) Descartes, Discours (le la méthode 1W Partie, 1). 75. 
\bigskip
P 100 : « Descartes marque bien, dans cette maxime, le caractère à la fois précaire et indispensable 
des valeurs. Il parle d'opinions probables, mais en fait il s'agit d'option se rapportant à ce que nous 
appellerions aujourd'hui des valeurs. En effet, ce qu'il qualifie de raison très vraie et très certaine, 
c'est, en attendant une certitude philosophique, la valeur apparemment incontestable qui s'attache 
à une conduite humaine efficace. 
\bigskip
Les  valeurs  interviennent,  à  un  moment  donné,  dans  toutes  les  argumentations.  Dans  les 
raisonnements  d'ordre  scientifique,  elles  sont  généralement  refoulées  à  l'origine  de  la  formation 
des  concepts  et  des  règles  qui  constituent  le  système  en  cause,  et  au  terme  du  raisonnement,  en 
tant  que  celui-ci  vise  la  valeur  de  vérité.  Le  déroulement  du  raisonnement  en  est,  autant  que 
possible, exempt ; cette purification atteint son maximum dans les sciences formelles. Mais dans 
les  domaines  juridique,  politique,  philosophique, 
les  valeurs  interviennent  comme  base 
d'argumentation tout au long des développements. On y fait appel pour engager l'auditeur à faire 
certains  choix  plutôt  que  d'autres,  et  surtout  pour  justifier  ceux-ci,  de  manière  à  les  rendre 
acceptables et approuvés par autrui. » 
\bigskip
P  100-101 :  « Dans  une  discussion,  on  ne  peut  se  soustraire  à  la  valeur  en  la  niant  purement  et 
simplement  :  de  même  que,  si  l'on  conteste  que  quelque  chose  soit  un  fait,  il  faut  donner  les 
raisons de cette allégation (« je ne perçois pas cela », ce qui équivaut à dire «je perçois autre chose 
»),  de  même  lorsqu'il  s'agit  d'une  valeur  on  peut  la  disqualifier,  la  subordonner  à  d'autres  ou 
l'interpréter,  mais  on  ne  peut,  en  bloc,  rejeter  toutes  les  valeurs  :  on  serait,  dès  lors,  dans  le 
domaine de la force et non plus dans celui de la discussion. Le gangster qui donne la primauté à sa 
sécurité  personnelle  peut  le  faire  sans  explication,  s'il  se  borne  au  domaine  de  l'action.  Mais  dès 
qu'il  veut  justifier  ce  primat  devant  autrui  ou  devant  lui-même,  il  doit  reconnaître  les  autres 
valeurs qu'on lui oppose pour pouvoir les combattre. En ce sens, les valeurs sont comparables aux 
\bigskip
\bigskip
\bigskip
48 
\bigskip
faits : dès que l'un des interlocuteurs les pose, il faut argumenter pour s'en délivrer, sous peine de 
refuser l'entretien ; et généralement, l'argument impliquera que l'on admet d'autres valeurs. » 
\bigskip
P  101 :  « Notre  conception,  qui  considère  les  valeurs  comme  objets  d'accord  ne  prétendant  pas  à 
l'adhésion de l'auditoire universel, se heurte à diverses objections. 
\bigskip
Ne néglige-t-on pas, au profit de cette distinction, d'autres différences plus essentielles ? Ne peut-
on  pas  se  contenter  de  dire  que  les  faits  et  vérités  expriment  le  réel  tandis  que  les  valeurs 
concernent une attitude envers le réel ? Mais si l'attitude envers le réel était universelle, on ne la 
distinguerait  pas  des  vérités.  Seul  son  aspect  non  universel  permet  de  lui  accorder  un  statut 
particulier. Il est en effet malaisé de  croire que des critères purement formels puissent entrer en 
ligne  de  compte.  Car  un  même  énoncé,  suivant  la  place  qu'il  occupe  dans  le  discours,  suivant  ce 
qu'il  annonce,  ce  qu'il  réfute,  ce  qu'il  corrige,  pourra  être  compris  comme  relatif  à  ce  que  l'on 
considère  communément  comme  fait  ou  à  ce  que  l'on  considère  comme  valeur.  D'autre  part,  le 
statue des enonces évolue : insérées dans un système de croyances que l'on prétend valoir aux yeux 
de  tous,  des  valeurs  peuvent  être  traitées  comme  des  faits  ou  des  vérités.  Au  cours  de 
l'argumentation,  et  parfois  par  un  processus  assez  lent,  on  reconnaîtra  peut-être  qu'il  s'agit 
d'objets d'accord qui ne peuvent prétendre à l'adhésion de l'auditoire universel. » 
\bigskip
P  101-102 :  « Mais,  si  c'est  là,  selon  nous,  la  caractéristique  des  valeurs,  que  dire  de  ce  que  l'on 
considère  d'emblée  comme  des  valeurs  universelles  ou  absolues,  telles,  le  Vrai,  le  Bien,  le  Beau, 
l'Absolu ? » 
\bigskip
P  102 :  « La  prétention  à  l'accord  universel,  en  ce  qui  les  concerne,  nous  semble  résulter 
uniquement  de  leur  généralité  ;  on  ne  peut  les  considérer  comme  valant  pour  un  auditoire 
universel qu'à la condition de ne pas spécifier leur contenu. A partir du moment où nous tentons 
de les préciser, nous ne rencontrons plus que l'adhésion d'auditoires particuliers. 
\bigskip
Les valeurs universelles méritent, selon E. Dupréel, d'être appelées des « valeurs de persuasion » 
parce que ce sont des moyens de persuasion qui, au point de vue du sociologue, -ne sont que cela, 
purs,  sorte  d'outils  spirituels  totalement  séparables  de  la  matière  qu'ils  permettent  de  façonner, 
antérieurs  au  moment  de  s'en  servir,  et  demeurant  intacts  après  qu'ils  ont  servi,  disponibles, 
comme avant, pour d'autres occasions (1). » 
\bigskip
(1) E. Dupréel, Sociologie générale, pp. 181-182. 
\bigskip
P 102-103 : « Cette conception met admirablement en évidence le rôle argumentatif de ces valeurs. 
Ces  outils,  comme  les  appelle  Dupréel,  sont  utilisables  devant  tous  les  auditoires  :  les  valeurs 
particulières  peuvent  toujours  être  rattachées  aux  valeurs  universelles,  et  servir  à  les  préciser. 
L'auditoire réel pourra se considérer comme d'autant plus proche d'un auditoire universel, que la 
valeur particulière paraîtra s'effacer devant la valeur universelle qu'elle détermine. C'est donc dans 
la  mesure  où  elles  sont  vagues  que  ces  valeurs  se  présentent  comme  universelles  et  qu'elles 
prétendent à un statut semblable à celui des faits. Dans la mesure où elles  sont précises, elles se 
présentent  simplement  comme  conformes  aux  aspirations  de  certains  groupes  particuliers.  Leur 
rôle est donc de justifier des choix sur lesquels il n'y a pas d'accord unanime en insérant ces choix 
dans une sorte de cadre vide, mais sur lequel règne un accord plus large. Bien que réalisé au sujet 
d'une forme vide, celui-ci n'en a pas moins une signification considérable : il témoigne de ce que 
l'on est décidé à dépasser les accords particuliers, tout au moins en intention, et que l'on reconnaît 
l'importance qu'il faut attribuer à l'accord universel que ces valeurs permettent de réaliser. » 
\bigskip
§ 19. VALEURS ABSTRAITES ET VALEURS CONCRETES 
\bigskip
P  103 :  « L'argumentation  sur 
les  valeurs  nécessite  une  distinction,  (lue  nous  jugeons 
fondamentale,  et  qui  a  été  trop  négligée,  entre  (les  valeurs  abstraites  telles  que  la  justice  ou  la 
\bigskip
\bigskip
\bigskip
49 
\bigskip
véracité,  et  des  valeurs  concrètes  telles  que  la  France  ou  l'Église.  La  valeur  concrète  est  celle  qui 
s'attache à un être vivant, un groupe détermine, un objet particulier, quand on les envisage dans 
leur  unicité.  La  valorisation  du  concret  et  la  valeur  accordée  à  l'unique  sont  étroitement  liées  : 
dévoiler  le  caractère  unique  de  quelque  chose,  c'est  la  valoriser  par  le  fait  même.  Les  écrivains 
romantiques en nous révélant le caractère unique de certains êtres, de certains groupes, de certains 
moments  historiques,  ont  provoqué,  même  dans  la  pensée  philosophique,  une  réaction  contre  le 
rationalisme  abstrait,  réaction  qui  se  marque  par  la  place  éminente  accordée  à  la  personne 
humaine, valeur concrète par excellence. » 
\bigskip
P  103-104 :  « Alors  que  la  morale  occidentale,  dans  la  mesure  où  elle  s'inspire  des  conceptions 
gréco-romaines,  attache  surtout  du  prix  à  l'observation  de  règles  valables  pour  tous  et  en  toutes 
circonstances,  il  existe  des  comportements  et  des  vertus  qui  ne  peuvent  se  concevoir  que  par 
rapport à des valeurs concrètes. Les notions d'engagement, de fidélité, de loyauté, de solidarité, de 
discipline sont de cette espèce. De même les cinq devoirs d'obligation universelle de Confucius (1), 
entre  gouvernants  et  gouvernés,  entre  père  et  fils,  entre  mari  et  femme,  entre  frère  aîné  et  frère 
cadet,  entre  amis,  sont  l'expression  de  l'importance  accordée  à  des  relations  personnelles  entre 
êtres qui constituent les uns pour les autres des valeurs concrètes. » 
\bigskip
(1)  Kou  Hong  Ming  et  Francis  Borrey,  Le  catéchisme  de  Confucius,  p.  69,  d'après  le  Tchoung-
young, chap. XX, § 7 (G. Pauthier, Confucius et Meneius, P* 83) ; cf. aussi le Hsiâo King (classique 
de la piété filiale). Sacred books of the East, vol. 111, traduit par J. Legge, notamment p. 482. 
\bigskip
P  104 :  « En  fait,  quelles  que  soient  les  valeurs  dominantes  dans  un  milieu  de  culture,  la  vie  de 
l'esprit  ne  peut  éviter  de  prendre  appui  aussi  bien  sur  des  valeurs  abstraites  que  sur  des  valeurs 
concrètes.  Il  semble  qu'il  y  a  toujours  eu  des  personnes  accordant  plus  d'importance  aux  unes 
qu'aux autres ; elles constituent peut-être des familles caractérielles. En tout cas, celles-ci auraient 
pour  trait  distinctif  non  de  négliger  complètement  des  valeurs  d'une  espèce  mais  de  les 
subordonner  à  celles  de  l'autre.  On  opposera  à  Érasme,  préférant  une  paix  injuste  à  une  guerre 
juste, celui qui préfère à l'amitié de Platon la valeur abstraite de la vérité. 
\bigskip
L'argumentation  se  base,  selon  les  circonstances,  tantôt  sur  les  valeurs  abstraites,  tantôt  sur  les 
valeurs  concrètes  ;  parfois,  il  est  malaisé  d'apercevoir  le  rôle  que  jouent  les  unes  ou  les  autres. 
Lorsqu'on  dit  que  les  hommes  sont  égaux  parce  qu'enfants  d'un  même  Dieu,  on  semble  prendre 
appui  sur  une  valeur  concrète  pour  retrouver  une  valeur  abstraite,  celle  de  l'égalité  ;  mais  on 
pourrait  dire  aussi  qu'il  ne  s'agit,  en  l'occurrence,  que  de  la  valeur  abstraite  qui  s'exprime  en 
faisant  appel,  par  analogie,  à  une  relation  concrète  ;  malgré  l'emploi  du  parce  que,  le  point  de 
départ se trouverait dans la valeur abstraite. » 
\bigskip
P  104-105 :  « Nulle  part  on  n'observe  mieux  ce  va-et-vient  de  la  valeur  concrète  aux  valeurs 
abstraites et inversement que dans les raisonnements concernant Dieu, considéré, à la fois, comme 
valeur  abstraite  absolue  et  comme  Être  parfait.  Dieu  est-il  parfait  parce  qu'il  est  l'incarnation  de 
toutes  les  valeurs  abstraites  ?  Une  qualité  est-elle  perfection  parce  que  certaines  conceptions  de 
Dieu  permettent  de  la  lui  accorder  ?  Il  est  difficile  de  déterminer  en  cette  matière  une  priorité 
quelconque. Les prises de position contradictoires d'un Leibniz, à ce sujet, sont très instructives. Il 
sait  que  Dieu  est  parfait,  mais  il  voudrait  que  cette  perfection  soit  justifiable  et  que  tout  ce  que 
Dieu décide ne soit pas bon uniquement pour cette raison même que Dieu l'a fait (1). L'universalité 
du  principe  de  la  raison  suffisante  exige  qu'il  existe  une  raison  suffisante,  une  conformité  à  une 
règle, qui justifie le choix divin. Mais, par contre, la croyance en la perfection divine précède toute 
preuve  (lue  Leibniz  pourrait  en  fournir  et  constitue  le  point  de  départ  de  sa  théologie.  Chez  un 
grand nombre de penseurs, Dieu est le modèle qu'il faut suivre, en toute matière. Aussi Kenneth 
Burke  a-t-il  pu  fournir  une  liste  fort  longue  de  toutes  les  valeurs  abstraites  qui  ont  trouvé  leur 
fondement dans l'Être parfait (2). » 
\bigskip
\bigskip
\bigskip
\bigskip
50 
\bigskip
(1) Leibniz, éd. Gerhardt, vol. 4, Discours de métaphysique, II, p. 427.  
(2) Kenneth Burke, A Rhetoric of motives, pp. 299-300. 
\bigskip
P  105 :  « Des  idéologies  qui  ne  voulaient  pas  reconnaître  en  Dieu  le  fondement  de  toutes  les 
valeurs ont été obligées de recourir à des notions, d'un autre ordre, telles l'État ou l'humanité : ces 
notions,  elles  aussi,  peuvent  être  conçues,  soit  comme  des  valeurs  concrètes  du  type  de  la 
personne, soit comme l'aboutissement de raisonnements basés sur des valeurs abstraites. 
\bigskip
Une  même  réalité,  un  groupe  social,  par  exemple,  sera  traitée  tantôt  comme  valeur  concrète  et 
comme  unique,  tantôt  comme  une  multiplicité  d'individus  que  l'on  opposera  à  un  seul  ou  à 
quelquesuns, au moyen d'argumentations par le nombre 'auxquelles toute idée de valeur concrète 
est  complètement  étrangère.  Ce  qui  est  valeur  concrète,  dans  certains  cas,  ne  l'est  pas  toujours  : 
pour qu'une valeur soit concrète, il faut l'envisager sous son aspèct de réalité unique; déclarer que 
telle  valeur  est,  une  fois  pour  toutes,  une  valeur  concrète,  constitue  une  prise  de  position 
arbitraire. » 
\bigskip
P 105-106 : « Des valeurs concrètes sont utilisées le plus souvent pour fonder les valeurs abstraites, 
et inversement. Pour savoir quelle conduite est vertueuse, nous nous tournons fréquemment vers 
un modèle que nous nous efforçons d'imiter. La relation d'amitié et les actes qu'elle incite à poser 
fourniront à Aristote un critère d'évaluation : 
\bigskip
Et les choses que nous préférons faire pour un ami sont plus désirables que celles que nous ferions 
pour le premier venu : par exemple, pratiquer la justice et faire du bien plutôt que de le paraître 
seulement,  car  nous  préférons  faire  réellement  du  bien  à  nos  amis  plutôt  que  d'en  avoir 
l'apparence, tandis que, pour des indifférents, c'est l'inverse (1). » 
\bigskip
(1) Aristote, Topiques liv III, chap. 2, 118 a. 
\bigskip
P  106 :  « Fénelon,  par  contre,  s'indigne  de  ce  que  l'on  prône  certaines  vertus  plutôt que  d'autres 
parce qu'un homme qu'on veut louer les a pratiquées alors que « il ne faut louer un héros que pour 
apprendre ses vertus au peuple, que pour l'exciter à les imiter » (2). 
\bigskip
Le besoin de s'appuyer sur des valeurs abstraites est peut-être lié essentiellement au changement. 
Elles  manifesteraient  un  esprit  révolutionnaire.  Nous  avons  vu  l'importance  que  les  Chinois 
accordaient aux valeurs concrètes. Celle-ci serait fonction de l'immobilisme de la Chine. » 
\bigskip
(2) Fénelon, éd. Lebel, t. XXI : Dialogue \& sur l'éloquence, pp. 24-25. 
\bigskip
P 106-107 : « Les valeurs abstraites peuvent servir aisément à la critique parce qu'elles ne font pas 
acception  de  personnes  et  semblent  fournir  des  critères  à  celui  qui  veut  modifier  l'ordre  établi. 
D'autre  part,  tant  qu'un  changement  n'est  pas  désiré,  il  n'y  a  aucune  raison  de  poser  des 
incompatibilités.  Or  les  valeurs  concrètes  peuvent  toujours'  s'harmoniser  ;  puisque  le  concret 
existe,  c'est  qu'il  est  possible,  c'est  qu'il  réalise  une  certaine  harmonie.  Par  contre,  les  valeurs 
abstraites,  poussées  à  leur  extrême,  sont  inconciliables  :  il  est  impossible  de  concilier  dans 
l'abstrait  des  vertus  telles  que  la  justice  et  la  charité,  Peut-être  le  besoin  de  changement,  en 
Occident,  a-t-il  incité  à  l'argumentation  sur  les  valeurs  abstraites,  se  prêtant  mieux  à  poser  des 
incompatibilités.  Par  ailleurs,  la  confusion  de  ces  notions  abstraites,  permettrait,  après  que  ces 
incompatibilités ont été posées, de former de nouvelle- conceptions de ces valeurs. Une vie intense 
des valeurs serait ainsi rendue possible, une refonte incessante, un remodèlement constant. » 
\bigskip
P  107 :  « L'appui  sur  les  valeurs  concrètes  serait  donc  beaucoup  plus  aisé  lorsqu'il  s'agit  de 
conserver que lorsqu'il s'agit de rénover. Et la raison pour laquelle les conservateurs se croient des 
réalistes est, peut-être, qu'ils mettent au premier plan pareilles valeurs. Les notions de fidélité, de 
\bigskip
\bigskip
\bigskip
51 
\bigskip
loyauté  et  de  solidarité, 
l'argumentation conservatrice. 
\bigskip
liées  à  des  valeurs  concrètes,  caractérisent  d'ailleurs  souvent 
\bigskip
§ 20. LES HIERARCHIES 
\bigskip
L'argumentation  prend  appui,  non  seulement  sur  des  valeurs,  abstraites  et  concrètes,  mais 
également sur des hiérarchies, telles (lue la supériorité des hommes sur les animaux, des dieux sur 
les hommes. Sans doute ces hiérarchies seraient-elles justifiables à l'aide de valeurs, mais le plus 
généralement  il  ne  sera  question  de  leur  chercher  un  fondement  que  lorsqu'il  s'agira  de  les 
défendre  ;  souvent,  d'ailleurs,  elles  resteront  implicites,  telle  la  hiérarchie  entre  personnes  et 
choses  dans  ce  passage  où  Scheler,  ayant  montré  que  les  valeurs  peuvent  se  hiérarchiser  selon 
leurs  supports  conclut  (lue,  de  par  leur  nature  même,  les  valeurs  relatives  aux  personnes  sont 
supérieures aux valeurs relatives aux choses (1). 
\bigskip
Les hiérarchies admises se présentent pratiquement sous deux aspects caractéristiques : à côté de 
hiérarchies concrètes, comme celle qui exprime la supériorité des hommes sur les animaux, il y a 
(les  hiérarchies  abstraites,  comme  celle  qui  exprime  la  supériorité  du  juste  sur  l'utile.  Les 
hiérarchies  concrètes  peuvent  évidemment  se  référer,  comme  dans  l'exemple  ci-dessus,  à  des 
classes d'objets mais chacun d'eux est envisagé dans son unicité concrète. » 
\bigskip
(1) Max Scheler, Der Formalismus in der Ethik und die materiale Wertethik, pp. 98-99. 
\bigskip
P  108  :  « On  peut  concevoir  que,  dans  une  hiérarchie  à  plusieurs  ternies, A  soit  supérieur  à  B  et 
que B soit supérieur à C, sans que les fondements que l'on pourrait alléguer en faveur de chacune 
de  ces  supériorités  soient  les  mêmes,  voire  sans  que  l'on-  fournisse  de  raison  à  ces  supériorités. 
Mais  si  l'on  a  recours  à  des  principes  abstraits,  ceux-ci  introduisent  généralement  dans  les 
rapports  entre  choses  un  ordre  qui  transforme  la  simple  supériorité,  le  préférable,  en  hiérarchie 
systématique,  en  hiérarchie  au  sens  strict.  Dans  ces  cas-là,  un  même  principe  abstrait, 
d'application répétable, peut établir l'ensemble de la hiérarchie : par exemple l’antériorité, le fait 
d'engendrer, de contenir, peuvent constituer le critère de hiérarchisation. 
\bigskip
Pareille hiérarchie se distingue nettement du simple préférable en ce qu'elle assure une ordination 
de tout ce qui est soumis au principe qui la régit. C'est ainsi que, selon Plotin, tous les éléments du 
réel  forment  une  hiérarchie  systématisée,  ce  qui  est  cause  ou  principe  devant  occuper  un  rang 
supérieur à ce qui est effet ou conséquence (1). Dans certains cas, un second principe peut établir 
une  hiérarchie  entre  des  termes que  le  premier  principe  ne  permet  pas  de hiérarchiser :  on  peut 
compléter  une  hiérarchisation  des  genres  animaux,  selon  un  certain  principe,  par  une 
hiérarchisation  des  espèces  de  chaque  genre  selon  un  autre  principe.  On  trouverait  chez  saint 
Thomas de curieuses applications de cette dualité de principes, notamment à la hiérarchisation des 
anges (2). 
\bigskip
L'un des principes hiérarchisants les plus usuels est la quantité plus ou moins grande de quelque 
chose.  C'est  ainsi  que  nous  aurons  à  côté  des  hiérarchies  de  valeurs  basées  sur  la  préférence 
accordée  à  l'une  de  ces  valeurs,  des  hiérarchies  proprement  dites  basées  sur  la  quantité  d'une 
même valeur : l'échelon supérieur est caractérisé par une plus grande quantité de tel caractère. » 
\bigskip
(1) Plotin, Ennéades, V, 5, § 12. 
(2) Cf. Gilson, Le thomisme, pp. 240-242. 
\bigskip
P  109 :  « A  ces  hiérarchies  quantitatives  s'opposeraient  les  hiérarchies  hétérogènes.  La 
hiérarchisation  des  valeurs  abstraites  non  ordonnées  quantitativement  n'implique  pas  que  ces 
valeurs soient indépendantes. Bien au contraire. Nous verrons que les valeurs sont généralement 
considérées  comme  liées  entre  elles  ;  cette  liaison  même  est  souvent  le  fondement  de  leur 
subordination : par exemple, lorsque la valeur qui est fin est jugée supérieure à celle qui est moyen, 
\bigskip
\bigskip
\bigskip
52 
\bigskip
la  valeur  qui  est  cause  supérieure  à  celle  qui  est  effet.  Toutefois  dans  beaucoup  de  cas,  la 
hiérarchisation admise serait bien susceptible d'être fondée en recourant à des schèmes de liaison 
mais  ceux-ci  ne  sont  pas  explicites  et  nous  n'avons  pas  l'assurance  qu'ils  sont  présents  aux 
auditeurs ; par exemple, le vrai sera, par certains, admis comme supérieur au bien, sans que l'on 
songe à expliciter les fondements possibles de cette supériorité, sans que l'on cherche à établir quel 
serait le lien de subordination de l'un à l'autre, ni même de quelle nature il pourrait être. 
\bigskip
Les hiérarchies de valeurs sont, sans doute, plus importantes au point de vue de la structure d'une 
argumentation que les valeurs elles-mêmes. En effet, la plupart de  celles-ci sont communes à un 
grand nombre d'auditoires. Ce qui caractérise chaque auditoire, c'est moins les valeurs qu'il admet, 
que la manière dont il les hiérarchise. » 
\bigskip
P  109-110 :  « Les  valeurs,  même  si  elles  sont  admises  par  maints  auditoires  particuliers,  le  sont 
avec  plus  ou  moins  de  force.  L'intensité  d'adhésion  à  une  valeur  par  rapport  à  l'intensité  avec 
laquelle  on  adhère  à  une  autre,  détermine  entre  ces  valeurs  une  hiérarchie  dont  il  faut  tenir 
compte.  Lorsque  cette  intensité  n’est  pas  connue  avec  une  précision  suffisante,  l'orateur  peut 
utiliser  en  quelque  sorte  librement  chacune  des  valeurs,  sans  avoir  à  justifier  nécessairement  la 
préférence qu'il accorde à l'une d'elles, puisqu'il ne s'agit pas de renverser une hiérarchie admise. 
Mais  ce  cas  est  relativement  rare.  Presque  toujours,  non  seulement  les  valeurs  jouissent  d'une 
adhésion  d'intensité  différente,  mais  en  outre,  des  principes  sont  admis  qui  permettent  de  les 
hiérarchiser.  C'est  un  des  points  sur  lesquels  beaucoup  de  philosophes  traitant  des  valeurs  ont 
négligé  d'attirer  l'attention.  Parce  qu'ils  ont  étudié  les  valeurs,  en quelque  sorte,  en  elles-mêmes, 
indépendamment  de  leur  utilisation  argumentative  pratique,  ils  ont  insisté,  à  juste  titre,  sur  la 
convergence des valeurs, négligeant trop souvent leur hiérarchisation, solution de conflits qui les 
opposent. » 
\bigskip
P  110 :  « Notons  pourtant  que  ces  hiérarchies  n'empêchent  pas  la  relative  indépendance  des 
valeurs.  Celle-ci  serait  compromise  si  les  principes  permettant  cette  hiérarchisation  étaient  fixés 
une fois pour toutes: on aboutirait alors à un monisme des valeurs. Mais ce n'est point ainsi que se 
présentent les hiérarchies dans la pratique : leurs fondements sont aussi multiples que les valeurs 
mêmes qu'elles coordonnent. 
\bigskip
Prenons, pour illustrer notre thèse, différentes manières d'envisager les rapports qui existent entre 
la certitude d'une connaissance et l'importance ou l'intérêt qu'elle peut présenter. Isocrate et saint 
Thomas accordent le primat à l'importance plutôt qu'à la certitude. Pour Isocrate : 
\bigskip
... il vaut mieux apporter sur des sujets utiles une opinion raisonnable, que sur des inutilités des 
connaissances exactes (1). » 
\bigskip
(1) Isocrate, Discours, t.I : Eloge I'Hélène, § 5. 
\bigskip
P 110-111 : « On retrouve comme un écho amplifié et dramatisé de ce passage, mais dans une tout 
autre perspective, dans la pensée de saint Thomas : 
\bigskip
Aux  esprits  que  tourmente  la  soif  du  divin,  c'est  vainement  qu'on  offrira  les  connaissances  les 
plus certaines touchant les lois des nombres ou la disposition de cet univers. Tendus vers un objet 
qui se dérobe à leurs prises, ils s'efforcent de soulever un coin du voile, trop heureux d'apercevoir, 
parfois même sous d'épaisses ténèbres, quelque reflet de la lumière éternelle qui doit les illuminer 
un jour. A ceux-là les moindres connaissances touchant les réalités les plus hautes semblent plus 
désirables que les certitudes les plus complètes touchant de inoindres objets (1). » 
\bigskip
(1) E. Gilson, Le thomisme, p. 40. (Cf. Sum. theol., I, 1, 5, ad I m ; ibid., I-II, 66, 5, ad 3m ; Sup. lib. 
de Causis, lect. I). 
\bigskip
\bigskip
\bigskip
53 
\bigskip
 
P  111 :  « Dans  un  sens  inverse,  J.  Benda  rappelle  un  passage  d'une  lettre  inédite  de  Lachelier  à 
Ravaisson : « Le sujet que je prendrai pour thèse n'est pas celui que je vous ai annoncé ; c'est un 
sujet plus étroit, c'est-à-dire plus sérieux (2). » 
\bigskip
Le  fait  que  l'on  se  sent  obligé  de  hiérarchiser  des  valeurs,  quel  que  soit  le  résultat  de  cette 
hiérarchisation,  vient  de  ce  que  la  poursuite  simultanée  de  ces  valeurs  crée  des  incompatibilités, 
oblige à des choix. C'est même un des problèmes fondamentaux que doivent résoudre presque tous 
les hommes de science. Prenons l'exemple de ceux qui s'adonnent à la «Content analvsis », qui a 
pour but de décrire objectivement, systématiquement et quantitativement le contenu manifeste de 
toute espèce de communication (3). 
\bigskip
Un problème de l'analyse des contenus qui revient toujours, écrit Lasswell, est de trouver la juste 
balance entre la sûreté et la valeur significative. Nous pouvons être tout à fait sûrs de la fréquence 
avec laquelle n'importe quel mot choisi se présente, mais cela peut être d'importance minime (4). 
\bigskip
Dans  ces  divers  cas,  les  problèmes  sont  différents,  ainsi  que  les  contextes,  dans  lesquels  ils  se 
présentent,  la  justification  de  la  hiérarchisation,  quand  elle  est  donnée,  peut  varier,  mais  le 
processus  argumentatif  présente  des  analogies  frappantes  :  il  suppose  l'existence  de  valeurs 
admises, mais incompatibles dans une certaine situation ; la hiérarchisation, qu'elle résulte d'une 
argumentation ou qu'elle soit posée dès l'abord, désignera celle que l'on décide de sacrifier (5). 
\bigskip
(2) J. Benda, Du style d'idées, p. 82, note. 
(3)  Cf.  Bernard  Berelson,  Content  Analysis,  Handbook  of  social  psychology,  edited  by  Gardner 
Lindzey. 
(4) H. D. Lasswell, N. Leites and Associates, Language of politics, P. 66, note. 
(5) Cf. § 46 : Contradiction et incompatibilité. 
\bigskip
§ 21. LES LIEUX 
\bigskip
P  112 :  « Quand  il  s'agit  de  fonder  des  valeurs  ou  des  hiérarchies,  ou  de  renforcer  l'intensité  de 
l'adhésion qu'elles suscitent, on peut les rattacher à d'autres valeurs ou à d'autres hiérarchies, pour 
les  consolider,  mais  on  peut  aussi  avoir  recours  à  des  prémisses  d'ordre  très  général,  que  nous 
qualifierons du nom de lieux, les ???? , d'où dérivent les Topiques, ou traités consacrés 
au raisonnement dialectique. 
\bigskip
Pour  les  Anciens,  et  ceci  semble  lié  au  souci  d'aider  l'effort  d'invention  de  l'orateur,  les  lieux 
désignent des rubriques sous lesquelles on peut classer les arguments : il s'agissait de grouper, afin 
de le retrouver plus aisément, en  cas  de besoin,  le matériel nécessaire (1) ; d'où la définition des 
lieux  comme  des  magasins  d'arguments  (2).  Aristote  distinguait  les  lieux  communs,  qui  peuvent 
servir indifféremment en n'importe quelle science et ne relèvent d'aucune ; et les lieux spécifiques, 
qui sont propres soit à une science particulière soit à un genre oratoire bien défini (3). » 
\bigskip
(1) Aristote, Topiques, liv. VIII, chap. 14, 163 b. 
(2) Cicéron, Topiques, II,  7 ; Partitiones oratoriae, § 5 ; Quintilien, vol. II, liv. V, chap. X, 20. 
(3) Aristote, Rhétorique, I, chap. 2, 1358 a; cf. Th. Viehweg, Topik und Jurisprudenz ; Johannes 
Stroux, Römische Rechiswissenschalt und Rhetorik. 
\bigskip
P  112-113 :  « Les  lieux  communs  se  caractérisaient  donc,  primitivement,  par  leur  très  grande 
généralité,  qui  les  rendait  utilisables  en  toutes  circonstances.  C'est  la  dégénérescence  de  la 
rhétorique,  et  le manque  d'intérêt  pour  l'étude  des  lieux  de  la  part  des  logiciens,  qui  a  conduit  à 
cette conséquence imprévue que des développements oratoires contre le luxe, la luxure, la paresse, 
etc.,  que  les  exercices  scolaires  ont  fait  répéter  jusqu'à  la  nausée,  ont  été  qualifiés  de  lieux 
communs, malgré leur caractère tout à fait particulier. Déjà Quintilien a cherché à réagir contre cet 
\bigskip
\bigskip
\bigskip
54 
\bigskip
abus (4), mais il n'y réussit guère. De plus en plus, on  entend par lieux communs ceux que Vico, 
par exemple, appelle les lieux oratoires pour les opposer à ceux dont traitent les Topiques (1). Les 
lieux communs de nos jours se caractérisent par une banalité qui n'exclut nullement la spécificité. 
Ces  lieux  communs  ne  sont,  à  vrai  dire,  qu'une  application  à  des  sujets  particuliers  des  lieux 
communs,  au  sens  aristotélicien.  Mais,  parce  que  cette  application  est  faite  à  un  sujet  sonvent 
traité, se déroule dans un certain ordre, avec des connexions prévues entre lieux, on lie songe plus 
qu'à  sa  banalité,  méconnaissait  sa  valeur  argumentative.  C'est  au  point  que  l'on  a  tendance  à 
oublier que les lieux forment un arsenal indispensable, dans lequel, quoi qu'il en ait, celui qui veut 
persuader autrui devra puiser. » 
\bigskip
(4) L. c. 
(1) Vico, Instituzioni oratorie, P. 20. 
\bigskip
P 113 : « Aristote étudie, dans ses Topiques, toute espèce de lieux pouvant servir de prémisse à des 
syllogismes  dialectiques  ou  rhétoriques,  et  il  les  classe,  selon  les  perspectives  établies  par  sa 
philosophie,  en  lieux  de  l'accident,  du  genre,  du  propre,  de  la  définition  et  de  l'identité.  Notre 
propos sera différent. D'une part, nous ne voulons pas lier notre point de vue à une métaphysique 
particulière et, d'autre part, comme nous distinguons les types d'objets d'accord concernant le réel 
de ceux qui concernent le préférable, nous n'appellerons  lieux que des prémisses d'ordre général 
permettant  de  fonder  des  valeurs  et  des  hiérarchies  et  qu'Aristote  étudie  parmi  les  lieux  de 
l'accident  (2).  Ces  lieux  constituent  les  prémisses  les  plus  générales,  souvent  d'ailleurs  sous-
entendues, qui interviennent pour justifier la plupart de nos choix. » 
\bigskip
(2) Cf. Aristote, Topiques, liv. III, 116 a-119 a et Rhétorique, liv. 1, chap. 6, 7, 1362 a-1365 b. 
\bigskip
P  113-114 :  « Une  énumération  des  lieux  qui  seraient  des  accords  premiers  dans  le  domaine  du 
préférable, dont tous les autres pourraient être déduits et qu'ils permettraient donc de justifier, est 
une  entreprise  dont  la  possibilité  est  sujette  à  discussion.  En  tout  cas  une  pareille  entreprise 
relèverait de la métaphysique ou de l'axiologie, ce qui n'est point notre propos. Notre but se limite 
à  l'examen  et  à  l'analyse  d'argumentations  concrètes.  Celles-ci  s'arrêtent  à des  niveaux  variables. 
Quand un accord est constaté, nous pouvons présumer qu'il est fondé sur des lieux plus généraux 
admis par les interlocuteurs; toutefois, pour les indiquer, il faudrait recourir à des hypothèses qui 
ne sont rien moins que certaines. Ainsi on se croirait autorisé peut-être à ramener l'affirmation que 
ce qui est plus durable et plus stable est préférable à ce qui l'est moins, à uti lieu, qui semble plus 
général, celui de la supériorité du tout sur la partie :  mais il importe de noter que ce dernier lieu 
n'est  pas  exprimé  dans  le  lieu  sur  le  durable,  qu'il  s'agit  d'une  interprétation  à  laquelle  les 
interlocuteurs  pourraient  ne  pas  donner  leur  assentiment.  Cependant,  un  lieu  quelconque  étant 
utilisé, on peut toujours exiger de l'interlocuteur qu'il le justifie. » 
\bigskip
P 114 : « Si les lieux les plus généraux attirent le plus volontiers notre attention, il y a néanmoins 
un intérêt indéniable à l'examen des lieux plus particuliers qui prévalent dans diverses sociétés, et 
qui permettent de les caractériser. D'autre part, même quand il s'agit des lieux très généraux, il est 
remarquable qu'à chaque lien on pourrait opposer un lien adverse : à la supériorité du durable, qui 
est un lieu classique, on pourrait opposer celle du précaire, de ce qui ne dure qu'un instant, et qui 
est  un  lieu  romantique.  D'où  la  possibilité  de  caractériser  les  sociétés,  non  seulement  par  les 
valeurs  particulières  qui  ont  leur  préférence,  mais  aussi  par  l'intensité  de  l'adhésion  qu'elles 
accordent à tel on tel membre d'un couple de lieux antithétiques. » 
\bigskip
P 114-115 : « Nous ne croyons pas utile, pour l'intelligence générale de l'argumentation, de fournir 
une  liste  exhaustive  des  lieux  utilisés.  Cette  tâche  nous  semble  d'ailleurs  difficilement  réalisable. 
Ce qui nous intéresse, c'est l'aspect par lequel tous les auditoires, quels qu'ils soient, sont amenés à 
tenir  compte  des  lieux,  que  nous  grouperons  sous  quelques  chefs  très  généraux  :  lieux  de  la 
quantité,  de  la  qualité,  de  l'ordre,  de  l'existant,  de  l'essence,  de  la  personne.  Le  classement  que 
\bigskip
\bigskip
\bigskip
55 
\bigskip
nous  présentons  se  justifie,  à  nos  yeux,  par  l'importance,  dans  la  pratique  argumentative,  des 
considérations relatives à ces catégories. Nous sommes obligés d'en parler un peu longuement afin 
que la notion du lieu soit, pour tous nos lecteurs, autre chose qu'un cadre vide. » 
\bigskip
22. LIEUX DE LA QUANTITE 
\bigskip
P  115 :  « Nous  entendons  par  lieux  de  la  quantité  les  lieux  communs  qui  affirment  que  quelque 
chose vaut mieux qu'autre chose pour des raisons quantitatives. Le plus souvent d'ailleurs, le lieu 
de la quantité constitue une majeure sous-entendue, mais sans laquelle la conclusion ne serait pas 
fondée. Aristote signale quelques-uns (le ces lieux : un plus grand nombre de biens est préférable à 
un moins grand nombre (1), le bien qui sert à un plus grand nombre de fins est préférable à ce qui 
n'est pas utile au même degré (2), ce qui est plus durable et plus stable est préférable à ce qui l'est 
moins (3). Notons, à ce  propos, que la supériorité dont il s'agit s'applique aussi bien aux valeurs 
positives que négatives, en ce sens qu'un mal durable est un plus grand mal qu'un mal passager. 
\bigskip
Pour Isocrate, le mérite est proportionnel à la quantité de personnes auxquelles on rend service (4) 
: les athlètes sont inférieurs aux éducateurs, parce qu'ils bénéficient seuls de leur force, tandis que 
les hommes qui pensent bien sont profitables à tous (5). C'est le même argument qu'utilise Timon 
pour valoriser le pamphlet: 
\bigskip
L'orateur  parle  aux  députés,  le  publiciste  aux  hommes  d'État,  le  journal  à  ses  abonnés,  le 
Pamphlet à tout le monde... Où le livre ne pénètre pas, le journal arrive.  Où le journal n'arrive 
pas, le Pamphlet circule (6). 
\bigskip
(1) Aristote, Topiques, liv. III, chap. 2, 117 a.  
(2) Ibid., liv. III, chap. 3, 118 b.  
(3) Ibid. liv. III, chap. 1, 116 a.  
(4) Isocrate, Discours, t. II : A  Nicoclès, § 8 
(5)  Ibid., t. II : Panégyrique d'Athènes, 2.  
(6) Timon, Livre des orateurs, pp. 90-91. 
\bigskip
P 116 : « Le tout vaut mieux que la partie » semble transposer, en termes de préférence, l'axiome « 
le tout est plus grand que la partie », et même Bergson, quand il se propose d'établir la supériorité 
du devenir, de l'évolution, sur le figé et le statique, n'hésite pas à utiliser le lieu de la quantité : 
\bigskip
Nous  disions  qu'il  y  a  plus  dans  un  mouvement  que  dans  les  positions  successives  attribuées  au 
mobile, plus dans un devenir que dans les formes traversées tour à tour, plus dans l'évolution de la 
forme  que  les  formes  réalisées  l'une  après  l'autre.  La  philosophie  pourra  donc,  des  termes  du 
premier  genre, tirer  ceux  du  second,  mais  non  pas  du  second  le  premier...  Comment,  ayant  posé 
l'immutabilité toute seule, en fera-t-on sortir le changement ?... Au fond de la philosophie antique 
gît nécessairement ce postulat: il y a plus dans l'immobile que dans le mouvant, et l'on passe, par 
voie de diminution ou d'atténuation, de l'immutabilité au devenir (1). 
\bigskip
C'est le lieu de la quantité, la supériorité de ce qui est admis par le plus grand nombre, qui fonde 
certaines conceptions de la démocratie, et aussi les conceptions de la raison qui assimilent celle-ci 
au «sens commun ». Même quand certains philosophes, tel Platon, opposent la vérité à l'opinion 
du  grand  nombre,  c'est  néanmoins  à  l'aide  d'un  lieu  de  la  quantité,  qu'ils valorisent  la  vérité,  en 
faisant d'elle un élément d'accord de tous les dieux, et qui devrait susciter celui de tous les hommes 
(2) ; le lieu quantitatif du durable permet aussi de valoriser la vérité, comme ce qui est éternel par 
rapport aux opinions instables et passagères. 
\bigskip
Un autre lieu d'Aristote affirme que : 
\bigskip
\bigskip
\bigskip
\bigskip
56 
\bigskip
Est  aussi  plus  désirable  ce  qui  est  plus  utile  en  toute  occasion  ou  la  plupart  du  temps  :  par 
exemple,  la  justice  et  la  tempérance  sont  préférables  au  courage,  car  les  deux  premières  sont 
toujours utiles, tandis que le courage ne l'est qu'à certains moments (3). » 
\bigskip
(1) Bergson, Evolution créatrice, pp. 341-342 (italiques de Bergson).  
(2) Platon, Phèdre, 273 d-e.  
(3) Aristote, Topiques, liv. III, chap. 2, 117 a, 
\bigskip
P  117 :  « Rousseau  affectionne  les  raisonnements  de  cette  espèce.  C'est  sur  de  pareilles 
considérations d'universalité qu'est fondée la supériorité de l'éducation qu'il préconise : 
\bigskip
Dans l'ordre social, où toutes les places sont marquées, chacun doit être élevé pour la sienne. Si un 
particulier  formé  pour  sa  place  en  sort,  il  n'est  plus  propre  à  rien.  ...  Dans  l'ordre  naturel,  les 
hommes étant tous égaux, leur vocation commune est l'état d'homme ; et quiconque est bien élevé 
pour celui-là ne peut mal remplir ceux qui s'y rapportent. ... Il faut donc généraliser nos vues, et 
considérer  dans  notre  élève  l'homme  abstrait,  l'homme  exposé  à  tous  les  accidents  de  la  vie 
humaine (1). 
\bigskip
La validité générale d'un bien sera définie aussi comme étant ce dont aucun autre bien ne rendra 
l'usage superflu; par ce biais peut se justifier une nouvelle fois la préférence accordée à la justice 
plutôt qu'au courage. 
\bigskip
Est préférable, dit Aristote, ... de deux choses, celle qui, tout le inonde la possédant, nous ôterait le 
besoin de l'autre, ... : si tout le monde était juste, le courage ne servirait à rien, tandis que si tout 
le monde était courageux, la justice serait encore utile (2). 
\bigskip
On  peut  considérer  comme  des  lieux  de  la  quantité  la  préférence  accordée  au  probable  sur 
l'improbable, au facile sur le difficile, à ce qui risque moins de nous échapper. La plupart des lieux 
tendant à montrer l'efficacité d'un moyen seront des lieux de la quantité. Ainsi dans ses Topiques, 
Cicéron groupe sous la rubrique de l'efficacité (vis) les lieux suivants : 
\bigskip
La cause efficiente l'emporte sur celle qui ne l'est pas; les choses complètes par elles-mêmes sont 
préférables à celles qui ont besoin du concours d'autres choses, celles qui sont en notre pouvoir à 
celles qui sont au pouvoir des autres, celles qui sont stables à celles qui sont mal assurées, celles 
qui ne peuvent nous être enlevées à celles qui peuvent l'être (3). » 
\bigskip
(1) Rousseau, Emile, pp. 11-12.  
(2) Aristote, Topiques, liv. III, chap. 2, 117 a-b.  
(3) Cicéron, Topiques, § 70. 
\bigskip
P 118 : « Ce qui se présente le plus souvent, l'habituel, le normal, est l'objet d'un des lieux le plus 
fréquemment utilisés, à tel point que le passage de ce qui se fait à ce qu'il faut faire, du normal à la 
norme, paraît, pour beaucoup, aller de soi. Seul le lieu de la quantité autorise cette assimilation, ce 
passage  du  normal,  qui  exprime  une  fréquence,  un  aspect  quantitatif  des  choses,  à  la  norme  qui 
affirme que cette fréquence est favorable et qu'il faut s'y conformer. Alors que tout le monde peut 
s'entendre  sur  le  caractère  normal  d'un  événement  à  condition  de  s'accorder  sur  le  critère  du 
normal qui sera utilisé, la présentation du normal comme norme exige, en outre, l'usage du lieu de 
la quantité. 
\bigskip
L'assimilation dit normal au normatif conduit Quetelet à considérer son homme moyen imaginaire 
comme le modèle même du beau (1), et Pascal en tire des pensées paradoxales, telle celle-ci : « Les 
hommes sont si nécessairement fous, que ce serait être fou par un autre tour de folie, de n'être pas 
fou (2). » 
\bigskip
\bigskip
\bigskip
57 
\bigskip
 
Le  passage  du  normal  an  normatif,  qui  se  retrouve  chez  tous  ceux  qui  fondent  l'éthique  sur 
l'expérience,  a  été  considéré,  à  juste  titre,  comme  une  faute  de  logique  (3).  Mais  nous  devons  y 
reconnaître  un  des  fondements  valables  de  l'argumentation,  en  ce  sens  que  ce  passage  est 
implicitement admis, quel que soit le domaine envisagé. On en retrouve la trace dans l'expression 
allemande Pflicht, proche de man pflegt; on la retrouve dans toutes les expressions qui recouvrent, 
à la fois, l'appartenance à un groupe et une manière d'être des individus appartenant à ce groupe : 
«américain  »,  «  socialiste»,  font  allusion  à  la  fois,  et  selon  les  circonstances,  à  une  norme  de 
conduite on à une conduite normale. 
\bigskip
(1) A. Quetelet, Physique sociale, t. II, p. 386. 
(2) Pascal, Bibl. de la Pléiade, Pensées, 184 (484), p. 871 (414 éd. Brunschvicg). 
(3) M. Ossowska, Podslawy nauki o moralnosci (Les fondements d'une science de la morale), p. 
83. 
\bigskip
P 118-119 : « Le passage du normal à la norme est un phénomène tout à fait courant, et qui semble 
aller de soi. C'est la dissociation des deux, et leur opposition par l'affirmation de la primauté de la 
norme sur le normal, qui nécessite une argumentation la justifiant : cette argumentation tendra à 
la dévalorisation du normal, le plus souvent par l'usage d'autres lieux que ceux de la quantité. » 
\bigskip
P  119 :  « L'exceptionnel  est  tenu  en  méfiance,  sauf  démonstration  de  sa  valeur.  Descartes  va  ju 
squ'à faire de cette méfiance une règle de sa morale provisoire : 
\bigskip
et entre plusieurs opinions également reçues, je ne choisissais que les plus modérées; tant à cause 
que  ce  sont  toujours  les  plus  commodes  pour  la  pratique,  et  vraisemblablement  les  meilleures, 
tous excès ayant coutume d'être mauvais... (1). 
\bigskip
Toute  situation  exceptionnelle  est  jugée  précaire  :  «  La  roche  tarpéienne  est  près  du  Capitole.  » 
Aussi  le  caractère  anormal  d'une  situation,  même  favorable,  peut-il  devenir  un  argument  contre 
celle-ci. 
\bigskip
§ 23. LIEUX DE LA QUALITE 
\bigskip
Les lieux de la qualité apparaissent, dans l'argumentation, et sont le mieux saisissables quand l'on 
conteste  la  vertu  du  nombre.  Ce  sera  le  cas  des  réformateurs,  de  ceux  qui  se  révoltent  contre 
l'opinion  commune,  tel  Calvin  qui  met  en  garde  François  Ier  à  l'égard  de  ceux  qui  arguent,  à 
l'encontre  de  sa  doctrine  «  qu'elle  est  desja  condamnee  par  un  commun  consentement  de  tous 
estats » (2). Il rejette la coutume, car « la vie des hommes n'a jamais esté si bien reiglee, que les 
meilleures choses pleussent à la plus grand' part » (3). Il oppose au nombre la qualité de la vérité 
garantie par Dieu : 
\bigskip
A l'encontre de toute ceste multitude est envoyé Jeremie, pour denoncer de la part de Dieu, que la 
Loy perira entre les Prestres, le conseil sera osté aux sages, et la doctrine aux Prophetes (4). » 
\bigskip
(1) Descartes, Discours de la méthode, IIIa Partie, pp. 73-74.  
(2) CALVIN', Institution de la religion chrétienne, Au Roy de France, p. 5.  
(3) Ibid., p. 11.  
(4) Ibid., p. 13. 
\bigskip
P 120 : « Les chefs  eux-mêmes peuvent donc se tromper.  Il  ne s'agit pas, au point extrême où se 
place  Calvin,  d'une  science  supérieure  accordée  à  l'élite.  Il  ne  s'agit  plus,  non  plus,  d'une 
connaissance  de  la  vérité  correspondant  à  ce  qu'admettrait,  comme  chez  Platon,  un  auditoire 
universel de dieux et d'hommes. Il s'agit de la lutte de celui qui détient la vérité, garantie par Dieu, 
contre la multitude qui erre. Le vrai ne peut succomber, quel que soit le nombre de ses adversaires 
: nous sommes en présence d'une valeur d'un ordre supérieur, incomparable. C'est sur cet aspect 
que les protagonistes du lieu de la qualité ne peuvent manquer de mettre l'accent : à la limite, le 
lieu de la qualité aboutit à la valorisation de l'unique, qui, tout comme le normal, est un des pivots 
de l'argumentation. 
\bigskip
L'unique est lié à une valeur concrète : ce que nous considérons comme une valeur concrète nous 
paraît unique, mais c'est ce qui nous paraît unique qui nous devient précieux : 
\bigskip
Sa  ressemblance  avec  moi,  nous  dit  Jouhandeau,  ce  qui  nous  rassemble,  nous  confond,  ne 
m'intéresse pas; c'est le signe particulier qui isole X, sa " singularité » qui m'importe, m'impose (1). 
\bigskip
Considérer  des  êtres  comme  interchangeables,  ne  pas  voir  ce  qui  fait  la  spécificité  de  leur 
personnalité,  c'est  les  dévaluer.  Il  suffit  parfois  d'un  renversement  des  termes  pour  que  se 
manifeste  le  caractère  falot  de  ceux  qu'ils  désignent  :  «  Thanks,  Rosencrantz  and  gentle 
Guildenstern » dit le Roi. « Thanks, Guildenstern and gentle Rosencrantz » reprend la Reine (2).” 
\bigskip
(1) M. JOUHANDEAU, Essai sur moi-même, p. 153. 
(2) SHAKESPEARE, Hamlet, acte 11, scène II. 
\bigskip
P 120-121 : “ Ces exemples tendent à montrer que l'unicité d'un être ou d'un objet quelconque tient 
à  la  manière  dont  nous  concevons  nos  relations  avec  lui  :  pour  l'un,  tel  animal  n'est  qu'un 
échantillon  d'une  espèce ;  pour  l'autre,  il  s'agit  d'un  être  unique  avec  lequel  il  entretient  des 
rapports  singuliers.  Des  philosophes  comme  Martin  Buber,  comme  Gabriel  Marcel,  s'insurgent 
contre le fongible, le mécanique, l'universalisable : 
\bigskip
Mieux vaut encore, dira Buber, violenter un être que l'on a réellement possédé, que de pratiquer 
une bienfaisance falote envers des numéros sans visage (1) ! » 
\bigskip
(1) M. Buber, Je et Tu, p. 46. 
\bigskip
P 121 : « Pour G. Marcel, la valeur d'une rencontre avec un être naît de ce qu'elle est « unique en 
son genre » (2). Ce qui est unique n'a pas de prix, et sa valeur augmente par le fait même qu'elle est 
inappréciable. Aussi Quintilien conseille à l'orateur de ne pas faire payer sa collaboration, pour la 
raison que « la plupart des choses peuvent sembler sans importance, par cela seul qu'on y met un 
prix » (3). 
\bigskip
La valeur de l'unique peut s'exprimer par son opposition au commun, au banal, au vulgaire. Ceux-
ci  seraient  la  forme  dépréciative  du  multiple  opposé  à  l'unique.  L'unique  est  original,  il  se 
distingue, et par là est remarquable et plaît même à la multitude. C'est la valorisation de l'unique, 
ou du moins de ce qui paraît tel, qui est le fond des maximes de Gracian et des conseils qu'il donne 
à  l'homme  de  cour.  Il  faut  éviter  de  se  répéter,  il  faut  paraître  inépuisable,  mystérieux,  non 
classable  aisément  (4)  :  la  qualité  unique  devient  un  moyen  en  vue  d'obtenir  le  suffrage  du  plus 
grand  nombre.  Même  le  grand  nombre  apprécie  ce  qui  se  distingue,  qui  est  rare  et  difficile  à 
réaliser. 
\bigskip
Le  plus  difficile,  dira  Aristote,  est  préférable  à  ce  qui  l'est  moins  car  nous  apprécions  mieux  la 
possession des choses qui ne sont pas faciles à acquérir (5). » 
\bigskip
(2) G. Marcel, Le monde cassé, suivi de Position et approches concrètes du mystère ontologique, 
pp. 270-271. 
(3) Quintillien, Vol. IV, liv. XII, chap. VII, S. 
(4) Gracian, L'homme de cour, pp. 2, 8, 102, 113, etc. 
(5) Aristote, Topiques, liv. III, chap. 2, 117 b. 
\bigskip
\bigskip
\bigskip
59 
\bigskip
 
P 122 : “On remarque qu'Aristote ne se contente pas d'énoncer le lieu. Il esquisse une explication. 
Il  le  rattache  à  la  personne,  à  l'effort.  Le  rare  concerne  surtout  l'objet,  le  difficile  sujet,  entant 
qu'agent. Présenter quelque chose comme difficile ou rare est un moyen sûr de le valoriser. 
\bigskip
La précarité peut être considérée comme la valeur qualitative opposée à la valeur quantitative de la 
durée ; elle est corrélative de l'unique, de l'original. On sait que tout ce qui est menacé prend une 
valeur éminente: Carpe diem. La poésie de Ronsard joue habilement sur ce thème qui nous touche 
immédiatement. La précarité n'est pas toujours menace de mort, elle peut concerner une situation 
: celle des amants aux yeux l'un de l'autre, comparée à celle des époux, est opposition de la valeur 
du précaire à celle du stable. 
\bigskip
Ce lien est lié à un lieu très important cité par Aristote, et qui serait celui de l'opportunité : 
\bigskip
Chaque chose est préférable dans le moment où elle a le plus d'importance : par exemple, l'absence 
de  chagrin  est  plus  désirable  dans  la  vieillesse que  dans  la  jeunesse,  car  elle  a  plus  d'importance 
dans la vieillesse (1). 
\bigskip
Si  l'on  renverse  l'exemple  d'Aristote,  si  l'on  insiste  sur  les  choses  importantes  pour  l'enfant  ou 
l'adolescent, on verra que, en faisant dépendre la valeur des circonstances transitoires, on insiste 
sur  la  précarité  de  cette  valeur  et  en  même  temps,  pendant  qu'elle  est  valable,  on  augmente  son 
prix. » 
\bigskip
(1) Aristote, Topiques, liv. III, chap. 2,117 a. 
\bigskip
P 122-123 : « Le lieu de l'irréparable se présente comme une limite, qui vient accentuer le lieu du 
précaire : la force argumentative, liée à son évocation peut être d'un effet foudroyant. Exemple la 
célèbre  péroraison  de  saint  Vincent  de  Paul,  s'adressant  aux  dames  pieuses  et  leur  montrant  les 
orphelins qu'il protégeait : 
\bigskip
Vous  avez  été  leurs  mères  selon  la  grâce,  depuis  que  leurs  mères  selon  la  nature  les  ont 
abandonnés.  Voyez  maintenant  si  vous  voulez  aussi  les  abandonner  pour  toujours...  ;  leur  vie  et 
leur mort sont entre vos mains... Ils vivront, si vous continuez d'en avoir un soin charitable; mais, 
je vous le déclare devant Dieu, ils seront tous morts demain, si vous les délaissez (1). » 
\bigskip
(1) D'après Baron, De la Rhétorique, p. 212. 
\bigskip
P  123 :  « Si  cette  péroraison  eut  tant  de  succès  (l'appel  aboutit  à  la  fondation  de  l'Hôpital  des 
Enfants-Trouvés), c'est au lieu de l'irréparable qu'elle le doit. 
\bigskip
La valeur de l'irréparable peut, si l'on veut en rechercher les fondements se rattacher à la quantité : 
durée infinie du temps (lui s'écoulera après que l'irréparable aura été fait ou constaté, certitude de 
ce  que  les  effets,  voulus  ou  non,  se  prolongeront  indéfiniment.  Mais  elle  peut  aussi  se  lier  à  la 
qualité  :  l'unicité  est  conférée  à  l'événement  que  l'on  qualifie  d'irréparable.  Qu'il  soit  bon  ou 
mauvais  dans  ses  conséquences,  il  est  source  d'effroi  pour  l'homme;  pour  qu'une  action  soit 
irréparable, il faut qu'elle ne puisse pas être répétée : elle acquiert une valeur par cela même qu'elle 
est considérée sous cet aspect. 
\bigskip
L'irréparable s'applique tantôt au sujet tantôt à l'objet quelque chose peut être irréparable en soi 
ou par rapport à tel sujet : on pourra replanter devant ma porte un nouveau chêne, mais ce n'est 
plus moi qui m'assiérai à son ombre. 
\bigskip
\bigskip
\bigskip
\bigskip
60 
\bigskip
On  voit  que  l'irréparable  dans  l'argumentation  est  bien  un  lieu  du  préférable,  en  ce  sens  que, 
lorsqu'il porte sur l'objet, ce ne peut être que dans la mesure où celui-ci est porteur d'une valeur; 
\bigskip
on ne mentionnera pas l'irréparable, l'irrémédiable, lorsqu'il s'agit d'une irréparabilité n'entraînant 
aucune  conséquence  dans  la  conduite.  On  parlera  peut-être  dans  un  discours  scientifique  de  la 
deuxième  loi  de  la  thermodynamique,  mais  celle-ci  ne  sera  considérée  comme  argument  de 
l'irréparable que si l'on attribue une valeur à un certain état de l'univers. » 
\bigskip
P  123-124 :  « Une  décision  dont  les  conséquences  seraient  irrémédiables,  est  valorisée  par  le  fait 
même. Dans l'action, on s'attache généralement à ce qui est urgent : les valeurs d'intensité, liées à 
l'unique,  au  précaire,  à  l'irrémédiable  y  sont  au  premier  plan.  C'est  ainsi  que  Pascal  se  sert  des 
lieux  de  la  quantité  pour  nous  montrer  qu'il  faut  préférer  la  vie  éternelle  à  la  vie  terrestre,  mais 
lorsqu'il nous presse de prendre une décision, il nous affirme que nous sommes embarqués et qu'il 
faut choisir, que l'hésitation ne peut durer, qu'il y a urgence et crainte de naufrage. » 
\bigskip
P 124 : « Outre les usages du lieu de l'unique comme original et rare, dont l'existence est précaire 
et la perte irrémédiable, par quoi on l'oppose à ce qui est fongible et commun, que l'on ne risque 
pas de perdre et qui est facilement remplaçable, il y a, dans un tout autre ordre d'idées, un usage 
du lieu de l'unique comme opposé au divers. L'unique est, dans ce cas, ce qui peut servir de norme 
:  celle-ci  prend  une  valeur  qualitative  par  rapport  à  la  multiplicité  quantitative  du  divers.  On 
opposera l'unicité de la vérité à la diversité des opinions. La supériorité des humanités classiques 
par rapport aux humanités modernes, dira un auteur (1), tient à ce que les Anciens présentent des 
modèles fixes, reconnus, éternels et universels. Les auteurs modernes, même s'ils sont aussi bons 
que les anciens, offrent l'inconvénient de ne pas pouvoir servir de norme, de modèle indiscutable : 
c'est la multiplicité des valeurs représentées par les modernes qui fait leur infériorité pédagogique. 
Ce même lieu sert à Pascal pour justifier la valeur de la coutume : 
\bigskip
Pourquoi  suit-on  les  anciennes  lois  et  anciennes  opinions  ?  Est-ce  qu'elles  sont  les  plus  saines  ? 
non, mais elles sont uniques, et nous otent la racine de la diversité (2). » 
\bigskip
(1) Baron, De la Rhétorique, n. 5, p. 451. 
(2) Pascal, Bibl. de la Pléiade, Pensées, 240 (429), p. 889 (301 éd. Brunschvicg). 
\bigskip
P  124-125 :  « Ce  qui  est  unique  bénéficie  d'un  prestige  certain  :  à  l'instar  de  Pascal,  on  peut 
expliquer par là un phénomène d'adhésion, en le fondant sur cette valeur positive que l'on prend 
comme base d'une argumentation sans devoir la fonder à son tour. L'infériorité du multiple, que ce 
soit le fongible ou le divers, semble admise très généralement, quelles que soient les justifications 
fort variées d'ailleurs que l'on serait à même de lui trouver. » 
\bigskip
§ 24. AUTRES LIEUX 
\bigskip
P 125 : « On pourrait songer à réduire tous les lieux à ceux de la quantité ou de la qualité, ou même 
à  réduire  tous  les  lieux  à  ceux  d'une  seule  espèce  -nous  aurons  l'occasion  de  traiter  de  ces 
tentatives - mais il nous semble plus utile, étant donné le rôle qu'ils ont joué et continuent à jouer 
comme  point  de  départ  des  argumentations  de  consacrer  quelques  développements  aux  lieux  de 
l'ordre, de l'existant, de l'essence, et de la personne. 
\bigskip
Les lieux de l'ordre affirment la supériorité de l'antérieur sur le postérieur, tantôt de la cause, des 
principes, tantôt de la fin ou du but. 
\bigskip
La supériorité des principes, des lois, sur les faits, sur le concret, qui semblent en être l'application, 
est admise dans la pensée non-empiriste. Ce qui est cause est raison d'être des effets et par là leur 
est supérieur : 
\bigskip
\bigskip
\bigskip
\bigskip
61 
\bigskip
Si ces formes produites, dira Plotin, ... existaient seules, elles ne seraient pas au dernier rang ; [si 
elles y sont, c'est que] là-bas sont les choses primitives, les causes productrices qui, parce qu'elles 
sont causes, sont au premier rang (1). 
\bigskip
Beaucoup  de  grandes  querelles  philosophiques  sont  axées  sur  la  question  de  savoir  ce  qui  est 
antérieur  et  ce  qui  est  postérieur,  pour  en  tirer  des  conclusions  quant  à  la  prédominance  d'un 
aspect  du  réel  sur  l'autre.  Les  théories  finalistes,  pour  valoriser  le  but,  le  transforment  en  vraie 
cause  et  origine  d'un  processus.  La  pensée  existentielle  qui  insiste  sur  l'importance  de  l'action 
tournée  vers  l'avenir,  rattache  le  projet  à  la  structure  de  l'homme  et  par  là  «cherche  toujours  à 
remonter vers l'originaire, vers la source » (2). » 
\bigskip
(1) Plotin, Ennéades, V, 3, § 10. 
(2) J. Wahl, Sur les philosophies de l'existence, Glanes 15-16, p.16.  
\bigskip
P 126 : « Les lieux de l'existant affirment la supériorité de ce qui existe, de ce qui est actuel, de ce 
qui est réel, sur le possible, l'éventuel, ou l'impossible. Le Molloy de Samuel Beckett exprime ainsi 
l'avantage de ce qui existe sur ce qui doit encore être réalisé, sur le projet : 
\bigskip
Car étant dans la forêt, endroit ni pire ni meilleur que les autres, et étant libre d'y rester, n'étais-je 
pas en droit d'y voir des avantages, non pas à cause de ce qu'elle était, mais parce que j'y étais. Car 
j'y étais. Et y étant je n'avais plus besoin d'y aller... (1). 
\bigskip
-1,'utilisation des lieux de l'existant suppose un accord sur la forme du réel auquel on les applique : 
dans  un  grand  nombre  de  controverses  philosophiques,  tout  en  admettant  que  l'accord  sur  ces 
lieux est acquis, on s'efforce d'en tirer un parti inattendu, grâce à un changement de niveau dans 
leur application ou grâce à une nouvelle conception de l'existant. 
\bigskip
Nous entendons par lieu de l'essence, non pas l'attitude métaphysique qui affirmerait la supériorité 
de l'essence sur chacune de ses incarnations  - et qui est fondée sur un lieu de l'ordre mais le fait 
d'accorder une valeur supérieure aux individus en tant que représentants bien caractérisés de cette 
essence.  Il  s'agit  d'une  comparaison  entre  individus  concrets  :  c'est  ainsi  que  nous  attribuons 
d'emblée une valeur à un lapin qui présente toutes les qualités d'un lapin; ce sera, pour nous un « 
beau  lapin  ».  Ce  qui  incarne  le  mieux  un  type,  une  essence,  une  fonction  est  valorisé  par  le  fait 
même. On connaît ces vers de Marot à François Ier 
\bigskip
Roi plus que Mars d'honneur environné  
Roi le plus Roi, qui fût onc couronné (2). » 
\bigskip
(1) S. Becket, Molloy, P. 132. 
(2) Cités par LA houssaie dans son épître à Louis XIV, en tête de sa traduction de B. Gracian, 
L'homme de cour, a 4, note. 
\bigskip
P 127 : « Proust se sert du même lieu pour valoriser la duchesse de Guermantes : 
\bigskip
... la duchesse de Guermantes, laquelle à vrai dire, à force d'être Guermantes, devenait dans une 
certaine mesure quelque chose d'autre et de plus agréable... (1). 
\bigskip
Une éthique ou une esthétique pourraient être fondées sur la supériorité de ce qui incarne le mieux 
l'essence et sur l'obligation qu'il y a à y parvenir, sur la beauté de ce qui y parvient. C'est parce que 
l'homme  est  fait  pour  penser  que,  pour  Pascal,  bien  penser  est  le  premier principe  de  la  morale. 
C'est parce que, pour Marangoni, les déformations sont inhérentes à l'essence de l'art, que l'on ne 
peut trouver d’œuvre sans déformation parmi celles que l'on considère comme réussies (2). 
\bigskip
\bigskip
\bigskip
\bigskip
62 
\bigskip
Dans  la  vie  héroïque,  selon  Saint-Exupéry,  le  chef  voit  une  justification  de  ses  plus  grandes 
duretés, des sacrifices qu'il impose à ses hommes, non dans le rendement qu'il en obtient ni dans 
la  domination  qu'il  exerce,  mais  dans  ce  que  ses  subordonnés  réalisent  ainsi  leurs  possibilités 
extrêmes, qu'ils accomplissent ce dont ils sont capables (3). La morale du surhomme tire du lieu de 
l'essence tout son attrait et tout son prestige. 
\bigskip
Examinons,  pour  terminer  ce  rapide  tour  d'horizon,  quelques  lieux  dérivés  de  la  valeur  de  la 
personne, liés à sa dignité, à son mérite, à son autonomie. 
\bigskip
Ce  qu'on  ne  peut  pas  se  procurer  par  autrui,  dit  Aristote,  est  préférable  à  ce  qu'on  peut  se 
procurer par lui : c'est le cas, par exemple, de la justice par rapport au courage (4). 
\bigskip
(1) Proust, A la recherche du temps perdu, t. 8 : Le côté de Guermantes, III. p  74. 
(2) M. Marangoni, Apprendre à voir, p. 103.  
(3) Saint-Exupéry, Vol de nuit, p. 131.  
(4) Aristote, Topiques, liv. III, chap. 2, 118 a. 
\bigskip
P 128 : « Ce lieu permet à Pascal de critiquer le divertissement : 
\bigskip
N'est-ce  pas  être  heureux  que  de  pouvoir  être  réjoui  par  le  divertissement  ?  Non  ;  car  il  vient 
d'ailleurs et de dehors... (1). 
\bigskip
Ce lieu confère aussi de la valeur à ce qui est fait avec soin, à ce qui demande un effort. 
\bigskip
Les lieux que nous avons mentionnés, et qui sont parmi les plus généralement utilisés, pourraient 
être complétés par bien d'autres, mais dont la signification est plus limitée. D'ailleurs, en spécifiant 
les  lieux,  on  passerait  par  des  degrés  insensibles  aux  accords  que  nous  qualifierions  plutôt 
d'accords sur des valeurs ou des hiérarchies. 
\bigskip
§  25.  UTILISATION  ET  REDUCTION  DES  LIEUX  ESPRIT  CLASSIQUE  ET  ESPRIT 
\bigskip
ROMANTIQUE 
Il serait intéressant de relever, aux différentes époques et dans différents milieux, les lieux qui sont 
le plus généralement  admis ou du moins qui semblent admis par l'auditoire, tel que se l'imagine 
l'orateur. Cette tâche serait d'ailleurs délicate, car les lieux que l'on considère comme indiscutables 
sont utilisés, sans être exprimés. On insiste, par contre, sur ceux que l'on désire réfuter ou nuancer 
dans leur application. 
\bigskip
Un  même  but  peut  être  réalisé  en  se  servant  de  lieux  fort  divers.  Pour  accentuer  l'horreur  d'une 
hérésie  ou  d'une  révolution,  on  se  servira  tantôt  des  lieux  de  la  quantité,  en  montrant  que  cette 
hérésie  cumule  toutes  les  hérésies  du  passé,  que  cette  révolution  entasse  bouleversements  sur 
bouleversements plus qu'aucune autre, tantôt des lieux de la qualité, en montrant qu'elle préconise 
une déviation toute nouvelle ou un système qui n'a jamais existé auparavant (2). » 
\bigskip
(1) Pascal, Bibl. de la Pléiade, Pensées, 216 (c. 53), p. 884 (170 éd. Brunschvicg). 
(2) Voir exemples chez Rivadeneira, Vida del bienaventurado Padre Ignacio, de Loyola, p. 194 et 
chez PITT, Orations on the French war, p. 42 (30 mai 1794.) 
\bigskip
P  128-129 :  « Il  faut  remarquer  pourtant  que  l'usage  de  certains  lieux  ou  de  certaines 
argumentations  ne  caractérise  pas  nécessairement  un  milieu  de  culture  déterminé,  mais  peut 
résulter,  et  c'est  d'ailleurs  le  cas  fort  souvent,  de  la  situation  argumentative  particulière  dans 
laquelle on se trouve. Des argumentations que Ruth Benedict, dans son intéressant ouvrage sur le 
Japon, considérait comme caractéristiques de la mentalité japonaise s'expliquent, pour nous, par 
le  fait  que  le  Japon  était  l'agresseur  :  or  celui  qui  veut  changer  ce  qui  est,  aura  tendance  à 
introduire,  comme  justification,  un  élément  normatif,  tels  la  substitution  de  l'ordre  à  l'anarchie, 
l'établissement d'une hiérarchie (1). » 
\bigskip
(1) R. Benedict, The Chrysanthemum and the Sword, pp. 20 et suiv. 
\bigskip
P 129 : « La situation argumentative, essentielle pour la détermination des lieux auxquels on aura 
recours, est elle-même un complexe qui comprend, à la fois, le but que l'on vise et les arguments 
auxquels on risque de se heurter. Ces deux éléments sont d'ailleurs intimement liés; en effet, le but 
que  l'on  vise,  même  s'il  s'agit  de  déclencher  une  action  bien  définie,  est  en  même  temps  la 
transformation de certaines convictions, la riposte à certains arguments, transformation et riposte 
qui sont indispensables au déclenchement de cette action. C'est ainsi que le choix entre différents 
lieux,  lieux  de  la  quantité,  ou  de  la  qualité,  par  exemple,  peut  résulter  de  l'une  ou  l'autre  des 
composantes  de  la  situation  argumentative  :  tantôt  on  verra  nettement  que  c'est  l'attitude  de 
l'adversaire qui influe sur ce choix, tantôt on verra au contraire le lien entre ce choix et l'action à 
déclencher.  Nous  savons  que  Calvin  utilise  souvent  des  lieux  de  la  qualité.  C'est,  disions-nous, 
caractère  fréquent  de  l'argumentation  de  ceux  qui  veulent  changer  l'ordre  établi.  Dans  quelle 
mesure  est-ce  aussi  parce  que  les  adversaires  de  Calvin  avaient  eu  recours  aux  lieux  de  la 
quantité ? 
\bigskip
Ils mettent grand'peine à recueillir force témoignages de l'Escriure, afin que s'ils ne peuvent 
vaincre par en avoir de meilleurs et lus propres que nous, que pour le moins ils nous puissent 
accabler e la multitude (2). 
\bigskip
(2) Calvin, Institution de la religion chrétienne, liv. 11, chap. V, § 6. 
\bigskip
P 130 : « On pourrait trouver un exemple beaucoup plus général de pareille opposition dans l'effort 
fait  par  les  romantiques  pour  renverser  certaines  positions  du  classicisme  :  là  où  ils  percevaient 
que  celui-ci  pouvait  se  défendre  à  l'aide  des  lieux  de  la  quantité,  le  romantisme  avait  tout 
naturellement recours aux lieux de la qualité. Si les classiques visaient l'auditoire universel, ce qui 
par un certain biais est un appel à la quantité, il était normal que les romantiques, dont l'ambition 
se borne le plus souvent à persuader un auditoire particulier, aient recours à des lieux de la qualité 
l'unique, l'irrationnel, l'élite, le génie. 
\bigskip
En fait, quand il s'agit de lieux, moins encore que quand il s’agit de valeurs, celui qui argumente 
cherche-t-il à éliminer complètement au profit d'autres, certains éléments ; il cherche plutôt à les 
subordonner, à les réduire à ceux qu'il considère comme fondamentaux. 
\bigskip
Quand les lieux de l'ordre sont ramenés à ceux de la quantité, l'antérieur est considéré comme plus 
durable,  plus  stable,  plus  général;  s'ils  sont  ramenés  aux  lieux  de  la  qualité,  on  considérera  le 
principe  comme  originaire,  d'une  réalité  supérieure,  comme  modèle,  comme  déterminant  les 
possibilités  extrêmes  d'un  développement.  Si  l'ancien  est  valorisé  comme ayant  subsisté  pendant 
plus longtemps et incarnant une tradition, le nouveau sera valorisé comme original et rare. 
\bigskip
Les lieux de l'existant peuvent être rattachés aux lieux de la quantité, liés au durable, au stable, à 
l'habituel, au normal. Mais ils peuvent aussi être rattachés aux lieux de la qualité, liés à l'unique et 
au précaire: l'existant tire sa valeur de ce qu'il s'impose en tant que vécu, en tant qu'irréductible à 
tout  autre  objet,  en  tant  qu'actuel.  On  pourrait  d'ailleurs  soutenir  que  l'existant,  comme  concret, 
fonde  les  lieux  de  la  qualité,  donne  sa  valeur  à  l'unique,  et  que  l'existant,  comme  réel,  fonde  les 
lieux de la quantité et donne leur sens ait durable et à ce qui s'impose universellement. » 
\bigskip
P 130-131 : « Le lieu de l'essence petit être rattaché, au normal qui permet seul, pour les penseurs 
empiristes,  la  constitution  de  types,  de  structures,  dont  on  apprécie  la  parfaite  réalisation  chez 
certains de leurs représentants. Mais pour les rationalistes, pour un Kant, par exemple, c'est l'idéal, 
l'archétype abstrait, qui est le seul fondement valable de toute normalité (1) : que cet archétype soit 
valorisé comme source et origine, ou comme réalité d'une espèce supérieure, comme universel ou 
comme  rationnel,  c'est  encore  un  autre  problème.  La  supériorité  de  ce  qui  incarne  le  mieux 
l'essence  pourrait  d'ailleurs  tantôt  être  fondée  sur  l'aspect  classique  et  universellement  valable, 
tantôt sur l'aspect exceptionnel de cette réussite considérée comme rare et difficile. » 
\bigskip
(1) KANT, Critique de la raison pure, pp. 305 et suiv. 
\bigskip
P 131 : « Les lieux de la personne peuvent être fondés sur ceux de l'essence, de l'autonomie, de la 
stabilité, mais aussi sur l'unicité et l'originalité de ce qui se rattache à la personnalité humaine. 
\bigskip
Parfois  ces  liaisons  et justifications  de  lieux  ne  sont  qu'occasionnelles,  mais  il  arrive que  pareille 
tentative  résulte  d'une  prise  de  position  métaphysique  et  caractérise  une  vision  du  monde.  C'est 
ainsi que le primat que l'on accorde aux lieux de la quantité et l'essai de ramener à ce point de vue 
tous  les  autres  lieux  caractérise  l'esprit  classique ;  l'esprit romantique  argumente,  par  contre,  en 
ramenant les lieux à ceux de la qualité. 
\bigskip
Ce  qui  est  universel  et  éternel,  ce  qui  est  rationnel  et  généralement  valable,  ce  qui  est  stable, 
durable,  essentiel,  ce  qui  intéresse  le  plus  grand  nombre,  sera  considéré  comme  supérieur  et 
fondement de valeur chez les classiques. 
\bigskip
L'unique,  l'original  et  le  nouveau,  le  distingué  et  le  marquant  dans  l'histoire,  le  précaire  et 
l'irrémédiable sont des lieux romantiques. » 
\bigskip
P 131-132 : « Aux vertus classiques de véracité et de justice, le romantique opposera celles d'amour, 
de  charité  et  de  fidélité  ;  si  les  classiques  s'attachent  aux  valeurs  abstraites,  ou  du  moins 
universelles, les romantiques préconisent les valeurs concrètes et particulières ; à la supériorité de 
la pensée et de la contemplation, préconisée par les classiques, les romantiques opposeront celle de 
l'action efficace. » 
\bigskip
P  132 :  « Les  classiques  s'efforceront  même  de  justifier  le  prix  qu'ils  accordent  aux  lieux  de  la 
qualité  en  les  présentant  comme  un  aspect  de  la  quantité.  La  supériorité  d'une  personnalité 
originale  sera  justifiée  par  le  caractère  inépuisable  de  son  génie,  l'influence  qu'elle  exerce  sur  le 
grand nombre, la grandeur des changements qu'elle occasionne. Le concret sera ramené à l'infini 
de ses éléments constituants, l'irrémédiable à la durée du temps pendant lequel il ne pourra être 
remplacé. 
\bigskip
Pour les romantiques, les aspects quantitatifs dont on tiendra compte pourraient se ramener à une 
hiérarchie  purement  qualitative  :  il  s'agira  alors  d'une  vérité  plus  importante,  qui  formera  une 
réalité  d'un  niveau  supérieur.  Lorsque  le  romantique  oppose  à  la  volonté  individuelle  celle  du 
grand  nombre,  cette  dernière  peut  être  conçue  comme  manifestation  d'une  volonté  supérieure, 
celle du groupe, que l'on décrira comme un être unique, ayant son histoire, son originalité et son 
génie propres. 
\bigskip
C'est ainsi que la systématisation des lieux, leur conception en fonction de lieux que l'on considère 
comme fondamentaux, leur donne des aspects variables et que le même lieu, la même hiérarchie, 
peuvent, grâce à une autre justification, aboutir à une vision différente du réel. » 
\bigskip
B) LES ACCORDS PROPRES A CERTAINES ARGUMENTATIONS 
\bigskip
§ 26. ACCORDS DE CERTAINS AUDITOIRES PARTICULIERS 
\bigskip
P  132-133 :  « Ce  que  l'on  appelle  habituellement  le  sens  commun  consiste  en  une  série  de 
croyances admises au sein d'une société déterminée et que ses membres présument être partagées 
par tout être raisonnable. Mais à côté de ces croyances, il  existe des accords, propres aux tenants 
d'une  discipline  particulière,  qu'elle  soit  de  nature  scientifique  ou  technique,  juridique  ou 
théologique. Ces accords constituent le corps d'une science ou d'une technique, ils peuvent résulter 
de certaines conventions ou de l'adhésion à certains textes, et caractérisent certains auditoires. » 
\bigskip
P 133 : « Ces auditoires se distinguent généralement par l'usage d'un langage technique qui leur est 
propre. C'est dans les disciplines formalisées, que ce langage se différencie ail maximum de celui 
(lue, par ailleurs, les membres de pareil auditoire utilisent dans leur relations journalières et qu'ils 
comprennent en tant que membres d'un auditoire plus général; mais même des disciplines, telles 
que  le  droit,  qui  empruntent  beaucoup  de  leurs  termes  techniques  au  langage  courant,  ont  pu 
paraître hermétiques aux non-initiés. Car ces termes, que l'on désire rendre aussi univoques que 
possible dans le contexte de la discipline, finissent par résumer un ensemble de connaissances, de 
règles  et  de  conventions,  l'ignorance  desquelles  fait que  leur  compréhension,  en  tant  que  termes 
devenus techniques, échappe entièrement aux profanes. » 
\bigskip
P  133-134 :  « Pour  entrer  dans  un  groupe  spécialisé,  une  initiation  est  nécessaire.  Alors  que 
l'orateur  doit  normalement  s'adapter  à  son  auditoire,  il  n'en  va  pas  ainsi  du  maître  chargé 
d'enseigner  à  ses  élèves  ce  qui  est  admis  dans  le  groupe  particulier  auquel  ceux-ci  désirent 
s'agréger ou, du moins, auquel désirent les agréger les personnes responsables de leur éducation. 
La persuasion est, dans ce cas, préalable à l'initiation. Elle doit obtenir la soumission aux exigences 
du  groupe  spécialisé  dont  le  maître  apparaît  comme  le  porte-parole.  L'initiation  à  une  discipline 
particulière consiste à faire part des règles et des techniques, des notions spécifiques, de tout ce qui 
y  est  admis,  et  de  la  manière  de  critiquer  ses résultats  en  fonction  des  exigences  de  la  discipline 
elle-même.  Par  ces  particularités,  l'initiation  se  distingue  de  la  vulgarisation  qui  s'adresse  au 
publie,  en  général,  pour  lui  faire  part  de  certains  résultats  intéressants,  dans  un  langage  non 
technique,  et  sans  le  mettre  à  même  ni  de  se  servir  des  méthodes  qui  ont  permis  d'établir  ces 
résultats  ni,  a  fortiori,  d'entreprendre  la  critique  de  ces  derniers.  Ces  résultats  sont,  en  quelque 
sorte, présentés comme indépendants de la science qui les a élaborés : ils  ont acquis le statut de 
vérités, de faits. La différence entre la science qui s'édifie, celle des savants, et la science admise, 
qui  devient  celle  de  l'auditoire  universel,  est  caractéristique  de  la  différence  entre  initiation  et 
vulgarisation (1). » 
\bigskip
(1)  Cf.  Ch.  Perelman,  La  vulgarisation  scientifique,  problème  philosophique,  Revue  des  Alumni, 
mars 1953, XXI, 4. 
\bigskip
P 134 : « A la question de savoir si une argumentation se poursuit à l'usage d'un auditoire lié par 
des  accords  particuliers  ou  à  l'usage  d'un  auditoire  non  spécialisé,  il  n'est  pas  toujours  facile  de 
répondre.  Certaines  controverses,  concernant  les  fraudes  en  archéologie,  par  exemple,  feront 
appel,  à  la  fois,  aux  spécialistes  et  à  l'opinion  publique  (2) ;  il  en  sera  de même  souvent  lors  des 
procès criminels où le débat se situe, à la fois, sur le plan juridique et sur le plan moral. 
\bigskip
Par ailleurs, il existe des domaines dont, selon la conception que l'on s'en fait, on dira, soit qu'ils 
sont spécialisés, soit qu'ils échappent à toute convention ou à tout accord particulier : c'est, d'une 
façon éminente, le cas de la philosophie. » 
\bigskip
(2) Cf. Vayson de Pradenne, Les fraudes en archéologie préhistorique, notamment p. 637. 
\bigskip
P 134-135 : « Alors qu'une philosophie d'école, se déroulant dans les cadres du système élaboré par 
le maître, peut être considérée comme spécialisée et rapprochée d'une théologie, peut-on admettre 
qu'un effort philosophique indépendant présuppose l'initiation préalable à une technique savante, 
qui serait celle des philosophes professionnels ? C'est l'avis exprimé, dans un ouvrage posthume, 
très suggestif, par un jeune auteur allemand, E. Rogge, qui oppose à une « philosophie populaire », 
comme  celle  de  Nietzsche,  par  exemple,  les  philosophies  contemporaines  qui  supposent,  toutes, 
une  connaissance  approfondie  de  l'histoire  de  la  philosophie,  par  rapport  à  laquelle  elles  sont 
amenées, d'une façon ou d'une autre, à se penser elles-mêmes (1). » 
\bigskip
(1) E. Rogge, Axiomatik alles möglichen Philosophierens, pp. 100 et suiv. 
\bigskip
P 135 : « Mais le philosophe qui prend position vis-à-vis de cette histoire de la philosophie, qui lui 
attribue  une  signification  déterminée,  et  qui  dès  lors  doit  admettre  que  sa  propre  conception 
répond à cette signification, renonce-t-il totalement à s'adresser à l'auditoire universel ? Ne petit-
on  pas  dire  que  l'auditoire  universel,  tel  que  le  philosophe  se  le  représente,  est  un  auditoire  qui 
admet  certains  faits,  et  notamment  l'acquis  des  sciences,  celui  plus  particulièrement  de  l'histoire 
scientifique de la philosophie, mais qui néanmoins reste souverain pour insérer ces faits dans des 
argumentations nouvelles, voire les renverser ? Dans ce cas, tout philosophe continue à s'adresser 
à  l'auditoire  universel,  au  même  titre  que  le  philosophe  populaire,  et  il  ne  semble  pas  que  l'on 
puisse  en  philosophie  faire  état  d'un  ensemble  de  connaissances,  de  règles  et  de  techniques 
comparable au corps d'une discipline scientifique et commun à tous ceux qui la pratiquent. 
\bigskip
L'exemple  de  la  philosophie  montre  bien  que  la  question  de  savoir  quels  sont  les  auditoires 
spécialisés est une question de fait qui doit être tranchée dans chaque cas. Mais il y a des auditoires 
tels  ceux  de  juristes  onde  théologiens,  pour  lesquels  cette  question  est  réglée  grâce  à  des 
considérations d'ordre formel : contrairement au droit naturel et à la théologie rationnelle, le droit 
et  la  théologie  positifs,  liés  par  des  textes  bien  déterminés,  constituent  des  domaines 
d'argumentation spécifiques. » 
\bigskip
P 135-136 : « Quelle que soit l'origine, quel que soit le fondement des textes de droit positif ou de 
théologie  positive  -  problème  qui  ne  nous  concerne  pas  actuellement  -  ce  qui  est  essentiel,  c'est 
qu'ils  constituent  le  point  de  départ  de  nouveaux  raisonnements.  L'argumentation  juridique  ou 
théologique  doit  se  développer  à  l'intérieur  d'un  système  défini,  ce  qui  mettra  au  premier  plan 
certains problèmes, notamment ceux relatifs à l'interprétation des textes. » 
\bigskip
P 136 : « Certaines notions, telles que celle d'évidence ou de fait, prennent un sens particulier dans 
des disciplines liées par des textes. 
\bigskip
Quand,  en  argumentant  devant  un  auditoire  qui  n'est  pas  lié  par  des  textes,  on  dit  d'une 
proposition qu'elle est évidente, c'est qu'on n'éprouve pas le désir on ne dispose pas d'un moyen de 
rejeter cette proposition. Par contre quand on dit d'une règle de droit qu'elle paraît évidente, c'est 
que  l'on  croit  qu'aucune  contestation  ne  pourrait  surgir  au  sujet  de  son  applicabilité  à  un  cas 
particulier. Car la non-évidence attribuée à certaines règles, la soi-disant nécessité de les justifier, 
résulte  de  ce  que  l'on  transpose  immédiatement  la  possibilité  de  contestations  en  une  quête  des 
fondements ; c'est que toute difficulté d'application, même si les valeurs que la loi protège ne sont 
pas  discutées,  risque  de  mettre  en  branle  toute  une  argumentation  dans  laquelle  interviendront 
vraisemblablement  les fondements  possibles  de  la  règle.  De même,  dire  d'un  texte  sacré  qu'il  est 
évident, c'est, puisqu'il n'est pas question de le rejeter, prétendre qu'il n'y a qu'une seule façon de 
l'interpréter. 
\bigskip
Les  accords  d'auditoires  spécialisés  peuvent  comporter  des  définitions  particulières  de  certains 
types  d'objets  d'accord,  par  exemple  de  ce  qui  est  un  fait.  Ils  portent  aussi  sur  la  manière  dont 
ceux-ci peuvent être invoqués ou critiqués. » 
\bigskip
P  136-137 :  « Pour  le  théologien  ou  le  juriste,  est  considéré  comme  un  fait  non  pas  ce  qui  peut 
prétendre à l'accord de l'auditoire universel, mais ce que les textes exigent ou permettent de traiter 
comme tel. Un théologien ne peut mettre en doute des faits ou des vérités attestés par des dogmes 
sans  s'exclure  de  l'auditoire  particulier  qui  les  considère  comme  avérés.  En  droit  existent  des 
fictions qui obligent à traiter une chose, même si elle n'existe pas, comme si elle existait ou à ne pas 
reconnaître  comme  existant  quelque  chose  qui  existe.  Ce  qui  est  admis  comme  un  fait  de  sens 
commun  petit  être  privé  de  toute  conséquence  juridique.  C'est  ainsi  que  le  juge  «  n'est  point 
autorisé à déclarer un fait constant, par cela seul qu'il en aurait personnellement acquis, en dehors 
du  procès,  la  connaissance  positive  »  (1).  L'intervention  du  juge  risquerait  de  modifier  les 
prétentions  des  parties,  or  ce  sont  les  parties  qui  déterminent  le  procès,  dans  le  cadre  de  la  loi. 
Nous  voyons  donc  que,  pour  certains  auditoires,  le  fait  est  lié  à  la  preuve  que  l'on  veut  ou  peut 
administrer. » 
\bigskip
(1) Aubry et Rau, Cours de droit civil français, t. XII, pp. 73-74. 
\bigskip
P 137 : « Dans les sciences naturelles contemporaines, le fait est subordonné, de plus en plus, à la 
possibilité  d'une  mesure,  dans  le  sens  large  de  ce  terme.  Elles  opposent  une  résistance  à  toute 
observation  qui  ne  peut  s'insérer  dans  un  système  de  mesures.  De  plus, un  savant  qui  vérifie  les 
conclusions  qu'un  autre  savant  à  proposées,  suite  à  une  expérimentation  déterminée,  tiendra 
compte de tous les faits qui se présentent et sont relevants au bien-fondé de cette théorie, mais il 
ne  se  croit  pas  autorisé,  dans  cette  controverse,  à  faire  état  d'autres  faits  qui,  dans  les  limites 
proposées, ne sont pas relevants ; à la différence, toutefois, de ce qui se passe en droit, il n'existe 
pas  en  science  de  règles  de  procédure  donnant  à  la  prétention  des  parties  une  fixité  relative  :  le 
savant, juge, y est toujours en même temps, partie, et introduira bientôt de nouvelles prétentions. 
Ce n'est donc que par analogie avec ce qui se passe en droit que nous pouvons percevoir des phases 
de  débat,  phases  dans 
lesquelles  certains  faits  sont  considérés  provisoirement  comme 
irrelevants. » 
\bigskip
P  137-138 :  « Même  dans  la  vie  journalière,  certains  faits  sont  considérés  comme  non  avenus,  et 
cela parce qu'il serait de mauvais goût d'en faire état. L'orateur qui attaque un adversaire ne peut 
avancer certaines informations relatives au comportement de ce dernier sans se dévaluer lui-même 
: une multitude de règles morales, de règles d'étiquette ou de déontologie empêchent l'introduction 
de certains faits dans un débat. L'auditoire juridique ne constitue, à cet égard, un cas privilégié que 
parce  que  les  restrictions  y  sont  codifiées  et  obligatoires  pour  toutes  les  parties  :  c'est  ce  qui 
distingue essentiellement la preuve judiciaire de la preuve historique (1). » 
\bigskip
(1) Aubry et Rau, t. XII, p. 63, note 2 bis de Bartin. 
\bigskip
P 138 : « Une distinction tout aussi importante concerne les présomptions: le lien qui nuit certains 
faits à d'autres peut être considéré par la loi « comme tellement fort que la probabilité que celui-ci 
est accompagné de celui-là équivaut à la certitude judiciaire de celui-là » (2). 
\bigskip
Les présomptions légales sont souvent de même nature que celles qui seraient admises dans la vie 
extra-juridique ; la loi, notamment, règle souvent ce qu'elle considère comme normal. Cependant 
l'origine de ces présomptions juridiques importe peu : il est vraisemblable que la présomption de 
l'innocence de l'accusé, en matière pénale, provient de ce que l'on craint les conséquences sociales 
et  morales  d'une  autre  convention  et  non  de  ce  que  le  droit  ait  adopté  une  présomption  de  sens 
commun liée au normal. 
\bigskip
Ce qui caractérise généralement les présomptions légales c'est la difficulté qu'il v a à les renverser: 
elles  sont  souvent  irréfragables  ou  ne  peuvent  être  récusées  que  suivant  des  règles  très  précises. 
Parfois  elles  ne  concernent  que  le  fardeau  de  la  preuve.  Celui-ci  est  presque  toujours,  et  devant 
n'importe  quel  auditoire,  fonction  de  présomptions  admises.  Mais  le  choix  de  celles-ci  n'est  pas 
imposé comme il l'est en certaines matières juridiques. 
\bigskip
Ces  remarques  concernant  les  accords  spécifiques,  propres  à  certains  auditoires,  indiquent  à 
suffisance  combien  des  arguments  valables  pour  certaines  personnes  ne  le  sont  nullement  pour 
d'autres, auxquelles ils peuvent paraître extrêmement étranges. 

Un  profane,  dit  Jouhandeau,  qui  assiste  à  une  discussion  de  théologiens  n'est  pas  éloigné  de 
penser  qu'il  découvre  un  monde  où  l'on  s'applique  à  déraisonner  de  compagnie  avec  la  même 
logique imperturbable que les pensionnaires d'une maison de fous (3). » 
\bigskip
(2) Ibid., p. 100, note 1 bis de Bartin. 
(3) M. Jouhandeau, De la grandeur, p. 98. 
\bigskip
P  139 :  « Il  en  résulte  que  l'orateur  peut  avoir  avantage  à  faire  choix  d'un  auditoire  déterminé. 
Quand  l'auditoire  n'est  pas  imposé  par  les  circonstances,  une  argumentation  peut  être  présentée 
d'abord  à  certaines  personnes,  puis  à  d'autres,  et  tirer  bénéfice  soit  de  l'adhésion  des  premières, 
soit, cas plus curieux, du rejet par celles-ci : le choix des auditoires et des interlocuteurs, ainsi que 
l'ordre  dans  lequel  se  présentent  les  argumentations,  exercent  une  grande  influence  dans  la  vie 
politique. » 
\bigskip
P 139-140 : « Le voisinage des auditoires, spécialisés et non-spécialisés, réagit sur l'argumentation. 
Un  artifice  signalé  par  Schopenhauer,  comme  utilisable  lors  d'une  discussion  entre  savants  en 
présence  d'un  publie  incompétent,  consiste  à  lancer  une  objection  non  pertinente,  mais  que 
l'adversaire  ne  saurait  réfuter  sans  de  longs  développements  techniques  (1).  Ce  procédé  place 
l'adversaire  dans  une  situation  difficile  parce  qu'il  l'oblige  à  se  servir  de  raisonnements  que  les 
auditeurs  sont  incapables  de  suivre.  L'adversaire  pourra  cependant,  en  dénonçant  la  manœuvre, 
discréditer  celui  qui  y  a  recours,  et  cette  disqualification,  qui  ne  requiert  pas  de  prémisses 
techniques, pourra être opérante auprès de tous les membres de l'auditoire, tant laïcs que savants. 
De  même,  dans  un  procès,  la  tendance  à  juger  en  droit  se  combine  avec  la  tendance  à  juger  en 
équité.  Si  cette  dernière  a  moins  d'importance  pour  un  juge  technicien,  néanmoins  celui-ci  ne 
saurait être entièrement fermé aux arguments qu'on lui présente en tant que membre d'un groupe 
social particulier mais non spécialisé ou en tant que membre de l'auditoire universel : cet appel à 
son sens moral peut l'inciter à inventer des arguments valables dans son cadre conventionnel, ou à 
apprécier différemment ceux dont il dispose. D'autre part, le souci de l'opinion actuelle ou future 
des auditoires spécialisés réagit sur les discours adressés à des auditoires non spécialisés: certains 
actes  de  la  vie  courante,  tels  achats,  ventes,  s'accomplissent  et  se  discutent  en  tenant  compte  de 
leur portée présente, mais en tenant compte aussi de ce qu'ils peuvent être un jour evoqués dans 
un contexte juridique. De même, l'homme du commun qui observe  certains phénomènes naturels 
peut  le  faire  en  tenant  compte  de  ce  qui  importera,  croit-il,  à  un  auditoire  de  savants.  Ainsi  les 
argumentations  entre  non-spécialistes  sont  formulées  de  façon,  soit  à  échapper  à  l'avis  d'un 
spécialiste  soit  à  tomber  sous  le  coup  de  sa  décision  :  en  tout  cas,  l'intervention  possible  du 
spécialiste influera sur un grand nombre de controverses entre laïcs. » 
\bigskip
(1) Schopenhauer, éd. Piper, vol. 6 : Eristische Dialektik, p. 418 (Kunstgriff 28). 
\bigskip
§ 27. ACCORDS PROPRES A CHAQUE DISCUSSION 
\bigskip
P 140 : « Les prémisses de l'argumentation consistent en propositions admises par les auditeurs. 
Quand  ceux-ci  ne  sont  pas  liés  par  des  règles  précises  les  obligeant  à  reconnaître  certaines 
propositions,  tout  l'édifice  de  celui  qui  argumente  ne  se  fonde  que  sur  un  fait  d'ordre 
psychologique, l'adhésion des auditeurs. Celle-ci n'est d'ailleurs, le plus souvent, que présumée par 
l'orateur. Quand  les  conclusions  de  ce  dernier  déplaisent  à  ses interlocuteurs, ils  peuvent,  s'ils  le 
jugent utile, opposer à cette présomption d'accord sur les prémisses, une dénégation qui aura pour 
effet  de  miner  toute  l'argumentation  par  la  base.  Ce  rejet  des  prémisses  ne  va  toutefois  pas 
toujours  sans  inconvénient  pour  les  auditeurs  -  nous  en  parlerons  plus  longuement  quand,  en 
analysant les techniques argumentatives, nous traiterons du ridicule (1). » 
\bigskip
(1) Cf. § 49 : Le ridicule, et son rôle dans l'argumentation. 
\bigskip
\bigskip
\bigskip
\bigskip
69 
\bigskip
P  140-141 :   « Il  arrive  que  l'orateur  ait  pour  caution  l'adhésion  expresse  des  interlocuteurs  à  ses 
thèses  de  départ.  Cette  adhésion  n'est  pas  une  garantie  absolue  de  stabilité  mais  elle  augmente 
celle-ci,  sans  quoi  nous  n'aurions  pas  le  minimum  de  confiance  nécessaire  à  la  vie  en  société. 
Quand  Alice,  conversant  avec  les  êtres  du  Pays  des  Merveilles,  veut  reprendre  une  de  ses 
affirmations, elle s'entend objecter : «Lorsque vous avez une fois dit quelque chose, cela le fixe, et 
vous  devez  en  accepter  les  conséquences  (1).  »  Réplique  bizarre  si  l'on  se  place  sur  le  plan  de  la 
vérité,  où  le  changement  est  toujours  permis,  car  on  peut  exciper  d'une  erreur.  Mais  remarque 
profonde  si  l'on  se  place  dans  le  domaine  de  l'action  où  les  propos  constituent  une  espèce 
d'engagement  qui  ne  pourrait  être  violé,  sans  raison  suffisante,  sous  peine  de  détruire  toute 
possibilité de vie commune. » 
\bigskip
P  141 :  « Aussi  les  manifestations  de  l'adhésion  explicite  ou  implicite,  sont-elles  recherchées  par 
l'orateur  :  une  série  de  techniques  sont  utilisées  pour  souligner  l'adhésion  ou  pour  surprendre 
celle-ci. Ces techniques sont particulièrement élaborées par certains auditoires, notamment par les 
auditoires juridiques. Mais elles ne leur sont nullement réservées. 
\bigskip
D'une  manière  générale,  tout  l'apparat  dont  on  entoure  la  promulgation  de  certains  textes,  le 
prononcé  de  certaines  paroles,  tend  à  rendre  leur  répudiation  plus  difficile  et  à  augmenter  la 
confiance sociale. Le serment, en particulier, ajoute à l'adhésion exprimée une sanction religieuse 
ou  quasi  religieuse.  Il  peut  concerner  la  vérité  des  faits,  l'adhésion  à  des  normes,  s'étendre  à  un 
ensemble de dogmes : le relaps était passible des plus grandes peines, parce qu'il contrevenait à un 
serment. 
\bigskip
La  technique  de  la  chose  jugée  tend  à  stabiliser  certains  jugements,  à  interdire  la  remise  en 
question de certaines décisions. En science, en distinguant certaines propositions que l'on qualifie 
d'axiomes, on leur accorde explicitement une situation privilégiée au sein du système : la révision 
d'un axiome ne pourra plus se produire que moyennant une répudiation tout aussi explicite; elle 
ne  pourra  se  faire  par  une  argumentation  qui  se  déroulerait  à  l'intérieur  du  système  dont  cet 
axiome fait partie. » 
\bigskip
(1) L. Carroll, Through the looking-glass, p. 293. 
\bigskip
P  142 :  « Le  plus  souvent,  pourtant,  l'orateur  ne  peut  tabler,  pour  ses  présomptions,  que  sur 
l'inertie psychique et sociale, qui, dans les consciences et dans les sociétés, fait pendant à l'inertie 
en  physique.  On  peut  présumer, 
l'attitude  adoptée 
antérieurement - opinion exprimée, conduite préférée se continuera dans l'avenir, que ce soit par 
désir de cohérence ou grâce à la force de l'habitude. L'étrangeté de notre condition, selon Paulhan, 
est  qu'il  soit  facile  de  trouver  des  raisons  aux  actes  singuliers,  difficile  aux  actes  communs.  Un 
homme qui mange du bœuf ne sait pas pourquoi il mange du boeuf ; mais s'il abandonne à jamais 
le boeuf pour les salsifis ou les grenouilles, ce n'est pas sans inventer mille preuves, les unes plus 
sages que les autres (1). 
\bigskip
En fait, l'inertie permet de compter sur le normal, l'habituel, le réel, l'actuel, et de le valoriser, qu'il 
s'agisse d'une situation existante, d'une opinion admise ou d'un état de développement continu et 
régulier.  Le  changement,  par  contre,  doit  être  justifié  ;  une  décision,  une  fois  prise,  ne  peut  être 
renversée  que  pour  des  raisons  suffisantes.  Un  grand  nombre  d'argumentations  insistent  sur  ce 
que rien en l'occurrence ne justifie un changement. Partisan de la continuation de la guerre avec la 
France, Pitt s'oppose, en ces termes, à toute idée de négociation : 
\bigskip
Les  circonstances  et  la  situation  du  pays  ont-elles  substantiellement  changé  depuis  la  dernière 
motion  sur  ce  sujet  ou  depuis  que  mon  honorable  ami,  pour  la  première  fois,  s'est  trouvé  être 
l'avocat de la négociation ? La situation des affaires a-t-elle varie de is ce temps, de telle sorte que 
\bigskip
jusqu'à  preuve  du  contraire,  que 
\bigskip
\bigskip
\bigskip
70 
\bigskip
la  négociation  serait  à  l'heure  actuelle  plus  désirable  qu'elle  ne  l'était  à  n'importe  quel  moment 
antérieur ? (2). » 
\bigskip
(1) J. Paulhan, Les fleurs de Tarbes, p. 212.  
(2) W. Pitt, Orations on the French war, p. 93 (27 mai 1795). 
\bigskip
P  142-143  :  “A  la  justification  du  changement  se  substituera  bien  souvent  une  tentative  pour 
prouver qu'il n'y a pas eu de changement réel. Cette tentative est parfois nécessitée par le fait que 
le changement est interdit : le juge qui ne peut changer la loi soutiendra que son interprétation ne 
modifie  pas  celle-ci,  qu'elle  correspond  mieux  à  l'intention  du  législateur  ;  la  réforme  de  l'Église 
sera  présentée  comme  un  retour  à  la  religion  primitive  et  aux  Écritures.  La  justification  du 
changement et l'argumentation tendant à montrer qu'il n'y a pas eu de changement lie s'adressent 
pas,  en  principe,  au  même  auditoire.  Mais  l'une  comme  l'autre  tendent  au  même  but,  qui  est  de 
répondre aux exigences de l'inertie dans la vie sociale. » 
\bigskip
P 143 : « La justification du changement se fera soit par l'indication d'une modification objective, à 
laquelle  le  sujet  a  dû  s'adapter,  soit  par  l'indication  d'un  changement  dans  le  sujet,  considéré 
comme un progrès : ainsi le changement qui, parce qu'il ébranle la confiance sociale, est toujours 
quelque  peu  dévaluant,  pourra  néanmoins  être  apprécié  comme  preuve  de  sincérité.  Un 
changement  qui  a  réussi  à  son  auteur  est  susceptible  de  devenir  exemplaire  pour  ceux  qui 
hésiteraient à s'engager dans la même voie : c'est ainsi que W. Lippman présente en modèle  aux 
Républicains  des  États-Unis 
l'évolution  du  sénateur  Vandeilberg  qui,  traditionnellement 
isolationniste,  est  devenu,  après  la  dernière  guerre,  un  partisan  convaincu  et  prestigieux  d'une 
politique de collaboration internationale (1). 
\bigskip
L'inertie peut être opposée, en principe, à tous les nouveaux projets et a fortiori à des projets qui, 
connus  depuis  longtemps,  n'ont  pas  été  acceptés  jusqu'à  ce  jour.  Ce  que  Bentham  appelle  le 
sophisme  de  la  peur  de  l'innovation  ou  aussi  le  sophisme  du  veto  universel,  qui  consiste  à 
s'opposer  à  toute  mesure  nouvelle,  simplement  parce  qu'elle  est  nouvelle,  n'est  nullement  un 
sophisme, mais l'effet de l'inertie qui joue en faveur de l'état de choses existant. Celui ci ne doit être 
modifié que s'il y a des raisons en faveur de la réforme. » 
\bigskip
(1) New York Herald  Tribune (édition (le Paris), du 12 mars 1948. 
\bigskip
P 144 : « Bentham le comprend en réalité fort bien, car à celui qui argue de ce que si la mesure eÛt 
été bonne elle serait déjà prise depuis longtemps, il riposte que des intérêts particuliers pouvaient 
s'y opposer ou qu'elle pouvait nécessiter un progrès des connaissances, se chargeant ainsi, en fait, 
du fardeau de la preuve (1). Notons à cet égard que si, en droit, le demandeur a très généralement 
la charge de la preuve, c'est que le droit se conforme à l'inertie; il est conçu de manière à ratifier, 
jusqu'à plus ample informé, les faits tels qu'ils sont (2). 
\bigskip
C'est  grâce  à  l'inertie  que  la  technique  de  la  chose  jugée  se  prolonge,  pour  ainsi  dire,  par  la 
technique du précédent. La répétition du précédent ne diffère de la continuation d'un état existant 
que  parce  que  les  faits  sont  envisagés  comme  du  discontinu.  Sous  cette  optique  légèrement 
différente, nous voyons toujours l'inertie à l'œuvre : de même qu'il faut faire la preuve de l'utilité 
de  changer  un  état  de  choses,  il  faudrait  faire  la  preuve  de  l'opportunité  de  changer  de  conduite 
devant une situation qui se répète. 
\bigskip
Dans des pays traditionalistes le précédent devient ainsi partie intégrante du système juridique, un 
modèle  dont  on  peut  se  prévaloir  à  condition  de  montrer  que  le  cas  nouveau  ressemble 
suffisamment à l'ancien. De là, la crainte de créer un précédent qui intervient dans bon nombre de 
décisions : « Vous allez décider, non point au sujet d'Isocrate, mais au sujet d'une règle de vie, s'il 
faut philosopher (3). » En effet, le fait de poser certains actes, qu'il s'agisse d'appréciations ou de 
\bigskip
\bigskip
\bigskip
71 
\bigskip
décisions, est considéré comme un consentement implicite à leur utilisation à titre de précédents, 
comme une sorte d'engagement à se conduire semblablement dans des situations analogues. » 
\bigskip
(1) Bentham, Œuvres, t. 1 : Traité des sophismes politiques, pp. 449-450. 
(2) Cf. R. Demogue, Les notions fondamentales du droit privé, p. 543. 
(3) Cité par Aristote (d'après Isocrate, sur l'Echange 173), pour illustrer le lieu de l'antécédent et 
du conséquent, Rhétorique, liv. II, chap. 23, 1399 b. 
\bigskip
P 144-145 : « De même lorsque quelqu'un observe une règle, et particulièrement lorsqu'il proclame 
qu'il l'observe, il manifeste qu'elle est bonne à suivre. Cette prise de position est assimilable à un 
aveu, qu'on pourrait, le cas échéant, rappeler. C'est ainsi que, en attaquant Eschine, Démosthène 
invoque le témoignage de son adversaire sur la façon dont un procès doit être conduit : 
\bigskip
...  a  tenu  des  discours  qui  subsistent  pour  le  perdre.  Car  ce  que  tu  as  défini  comme  la  justice 
quand  tu  faisais  le  procès  de  Timarque,  cela  même  doit  aussi  valoir  pour  les  autres  à  ton 
désavantage (1). » 
\bigskip
1) Démosthène, Harangues et plaidoyers politiques, t. III : Sur les forfaitures de l'ambassade, § 
241. 
\bigskip
P 145 : « Il est efficace de reprendre,  pour l'utiliser contre lui, tout ce que l'on peut considérer, à 
cause de l'adhésion qui s'y manifeste, comme un aveu de  l'adversaire. Pareille reprise immédiate 
des  paroles  de  l'interlocuteur  constitue  l'essentiel  de  ce  qu'on  appelle  communément  l'esprit  de 
répartie. 
\bigskip
En droit, quand seul l'intérêt des parties est en jeu, l'aveu de l'une des parties, tout comme l'accord 
des deux, fournit un élément stable sur lequel le juge peut s'appuyer; quand il s'agit des matières 
auxquelles l'ordre publie est intéressé, l'aveu n'a pas la même force probante, car c'est le juge, et 
non point les parties, qui alors détermine ce qui peut être considéré comme acquis. 
\bigskip
Au lieu de se baser sur les jugements de l'interlocuteur, on utilisera parfois de simples indices de 
son aveu, on se prévaudra notamment de son silence. 
\bigskip
Le  silence  peut  être  interprété,  soit  comme  l'indice  qu'aucune  objection  ou  réfutation  n'a  été 
trouvée,  soit  comme  l'indice  que  l'affaire  est  indiscutable.  La  première  interprétation  affirme  un 
accord  de  fait  de  l'interlocuteur  ;  la  seconde  en  tire  le  droit.  «  Ce  que  l'adversaire  lie  nie  pas  », 
constitue pour Quintilien un élément sur lequel le juge peut s'appuyer (2). » 
\bigskip
(2) Quintillien, Vol. 11. liv. V, chap. X, § 13. 
\bigskip
P  145-146 :  « C'est  le  danger  de  l'accord  tiré  du  silence  qui  explique  que,  dans  beaucoup  de 
circonstances,  on  choisit  de  répondre  quelque  chose,  même  si  l'objection  dont  on  dispose 
momentanément est faible. » 
\bigskip
P 146 : « L'association que l'on établit entre silence et aveu peut cependant jouer au détriment de 
certaines affirmations : le silence absolu devant certaines mesures prises par les pouvoirs publics 
paraît  suspect,  car  il  est  difficile  de  l'interpréter  comme  une  approbation  unanime  ;  on  préfère, 
pour l'interpréter, recourir à l'hypothèse de l'intimidation. 
\bigskip
Les indices d'où l'on tire l'aveu sont multiples : il pourra être dépisté dans une abstention, mieux 
encore  dans  un  revirement  dont  on  croit  avoir  repéré  la  trace.  C'est  ainsi  que  le  rejet  par  le 
législateur d'un article dans une loi, qui lui est soumise, sera traité ultérieurement comme un aveu, 
c'est-à-dire comme l'affirmation implicite qu'il a pensé à cette chose et n'en voulait pas. » 
\bigskip
\bigskip
\bigskip
72 
\bigskip
 
P 146-147 : « L'usage dialectique des questions et des réponses tend essentiellement à assurer des 
accords  explicites  dont  on  pourra  ensuite  se  prévaloir  ;  c'est  une  des  caractéristiques  de  la 
technique  socratique.  Une  des  applications  de  cette  méthode  consiste  dans  la  recherche  d'un 
accord explicite sur le point à juger, celui dont l'adversaire fera dépendre l'issue du débat, ou sur 
les preuves qu'il admettra et considérera comme concluantes. Nous avons cité ailleurs (1) le cas de 
ce chef d'entreprise américain qui, pendant tout un jour, sollicite les objections des représentants 
ouvriers,  et  les  fait  noter  soigneusement  au  tableau  noir  :  il  parvient  ainsi  à  obtenir  leur  accord 
explicite sur les points auxquels il s'agit de répondre; le fait d'en ajouter de nouveaux, après coup, 
serait interprété comme un indice de mauvaise volonté. En fixant l'objet de la  controverse, on la 
rend  plus  serrée  :  l'interlocuteur  ne  pourrait  trouver  une  échappatoire,  pour  refuser  son  accord, 
alors que les conditions admises ont été remplies, qu'au risque de se déjuger. D'ailleurs, comme ce 
sont des tiers qui, le plus souvent, sont juges de l'issue du débat, il y a peu de chances que pareil 
désaveu ait d'autres conséquences que de déconsidérer son auteur. » 
\bigskip
(1) Rhétorique et philosophie, p. 20. D'après Dale CARNEGIE, P. 344. 
\bigskip
P 147 : « Quintilien donne ce conseil aux avocats : 
\bigskip
Aussi  pourra-t-il  être  avantageux  de  dissimuler  certaines  de  nos  armes.  Car  l'adversaire  les 
réclame avec insistance, et souvent en fait dépendre toute l'issue de la cause, croyant que nous ne 
les avons pas; en réclamant nos preuves, fi leur donne de l'autorité (1). 
\bigskip
L'accord préliminaire à la discussion peut porter, non sur l'objet du débat ou sur les preuves, mais 
sur  la  façon  de  conduire  la  discussion.  Cet  accord  peut  être  quasi  rituel,  comme  dans  les 
discussions  judiciaires,  parlementaires  ou  académiques;  mais  il  peut  résulter,  au  moins 
partiellement, de la discussion particulière en cours et d'une initiative prise par l'une des parties. 
C'est ainsi que Démosthène présente à l'Eschine les modalités de sa défense : 
\bigskip
La défense juste et sincère, c'est de prouver, ou bien que les faits incriminés n'ont pas eu lieu, ou 
bien qu'ayant eu lieu, ils sont utiles à l'Etat (2). 
\bigskip
Craignant  que  l'accusé  ne  détourne  l'attention  de  l'assemblée  sur  des  points  secondaires, 
Démosthène  lui  prescrit,  pour  ainsi  dire,  la  technique  de  sa  défense,  dont,  par  le  fait  même,  il 
s'engage à reconnaître la valeur. C'est ainsi que l'interlocuteur qui, dans une controverse, reprend, 
point par point, les allégations de son prédécesseur, en acceptant l'ordre de son discours, prouve sa 
loyauté dans le débat. » 
\bigskip
(1) Quintillien, Vol. 11, liv. VI, chap. IV, § 17. 
(2) Démosthène, Harangues et plaidoyers politiques, t. 111 : Sur les forfaitures de l'ambassade, § 
203. 
\bigskip
P  147-148 :  « S'assurer  certains  accords  ou  certains  rejets  est  donc  un  des  objectifs  déterminant 
l'ordre  dans  l'argumentation.  En  effet  la  construction  d'un  discours  n'est  pas  uniquement  le 
développement  de  prémisses  données  au  départ  ;  elle  est  aussi  établissement  de  prémisses, 
explicitation et stabilisation des accords (1). » 
\bigskip
(1) Cf. § 103 : Ordre et persuasion. 
\bigskip
P  148 :  « C'est  ainsi  que  chaque  discussion  présente  des  étapes,  jalonnées  par  les  accords  qu'il 
s'agit d'établir, qui résultent parfois de l'attitude des parties et qui, parfois, sont institutionnalisées 
grâce à des habitudes prises ou à des règles explicites de procédure. 
\bigskip
\bigskip
\bigskip
73 
\bigskip
§ 28. L'ARGUMENTATION « AD HOMINEM » ET LA PETITION DE PRINCIPE 
\bigskip
Les possibilités d'argumentation dépendent de ce que chacun est disposé à concéder, des valeurs 
qu'il  reconnaît,  des  faits  sur  lesquels  il  marque  son  accord  :  par  là,  toute  argumentation  est  une 
argumentation  ad  hominem  ou  ex  concessis.  Si  pourtant 
l'on  oppose  fréquemment  à 
l'argumentation  ad  hominem  l'argumentation  ad  rem  (2),  la  première  portant  sur  l'opinion  et  la 
seconde concernant la vérité ou la chose elle-même, c'est que l'on oublie que la vérité dont il s'agit 
doit  être  admise.  En  termes  de  notre  théorie,  l'argumentation  ad  rem  correspond  à  une 
argumentation  que  l'on  prétend  valable  pour  toute  l'humanité  -  raisonnable,  c'est-à-dire  ad 
humanitatem. Celle-ci serait un cas particulier, mais éminent, de l'argumentation ad hominem. 
\bigskip
L'argumentation  qui  vise  l'auditoire  universel,  l'argumentation  ad  humanitatem,  évitera,  autant 
que  possible,  l'usage  d'arguments  qui  ne  seraient  valables  que  pour  des  groupes  particuliers.  Ce 
sera le souci, notamment, de l'argumentation philosophique. 
\bigskip
Nous  pourrions  distinguer  des  types  d'arguments  ad  hominem  aussi  variés  que  les  auditoires 
auxquels  ils  s'adressent  ;  nous  proposons  de  les  qualifier  d'arguments  ad  hominem,  au  sens 
restreint,  lorsque  l'orateur  sait  qu'ils seraient  sans  poids  pour  l'auditoire  universel,  tel  qu'il  se  le 
représente. » 
\bigskip
(2) Cf. Schopenhauer,  éd. Brockhaus, vol. 6: Parerga und Paralipomena, II, p. 29. 
\bigskip
P 148-149 : « En voici un exemple très simple. On sera onze à déjeuner. La bonne s'exclame : « Oh, 
cela porte malheur! » Pressée, la maîtresse répond : « Non, Marie, vous vous trompez : c'est treize 
qui  porte  malheur.  »  L'argument  est  sans  réplique  et  met  immédiatement  fin  au  dialogue.  Cette 
réponse peut être considérée comme un type d'argumentation  ad hominem. Elle ne met en cause 
aucun  intérêt  personnel  de  la  servante,  mais  se  base  sur  ce  que  celle-ci  admet.  Plus  rapidement 
efficace que ne le serait une dissertation sur le ridicule des superstitions, elle permet d'argumenter 
dans le cadre du préjugé, au lieu de le combattre. » 
\bigskip
P  149 :  « Les  arguments  ad  hominem  sont  souvent  qualifiés  de  pseudo-arguments:  c'est  que  ce 
sont des arguments qui persuadent manifestement certaines personnes, alors qu'ils ne le devraient 
pas,  pour  la  bonne  raison,  pense  celui  qui  les  dévalue  ainsi,  qu'ils  n'auraient  aucun  effet  sur  lui-
même.  En  fait,  celui  qui  les  traite  avec  un  tel  mépris,  croit,  d'une  part,  que  la  seule  vraie 
argumentation est celle qui s'adresse à l'auditoire universel, et, d'autre part, s'érige en représentant 
authentique  de  cet  auditoire.  C'est  parce  que,  à  leurs  yeux,  toute  argumentation  doit  valoir  pour 
l'auditoire  universel,  que  certains  verront  dans  l'efficacité  des  arguments  ad  hominem  stricto 
sensu un signe de la faiblesse humaine. Schopenhauer qualifiera d'artifice (Kunstgriff) l'usage de 
l'argument  ad  hominem  consistant  à  mettre  l'interlocuteur  en  contradiction  avec  ses  propres 
affirmations, avec les enseignements d'un parti qu'il approuve ou avec ses propres actes (i). Mais il 
n'y  a  rien  d'illégitime  dans  cette  façon  de  procéder.  Nous  pourrions  même  qualifier  pareille 
argumentation de rationnelle tout en  admettant que les prémisses discutées ne sont pas admises 
par  tous.  Ce  sont  ces  prémisses  qui  déterminent  le  cadre  dans  lequel  se  meut  l'argumentation  : 
c'est  pour  cela  d'ailleurs  que  nous  rattachons  l'examen  de  cette  question  aux  accords  propres  à 
certaines argumentations. » 
\bigskip
(1) Schopenhauer, éd. Piper, vol. 6: Eristische Dialektik, p. 415 (Kunstgriff 16). 
\bigskip
P 150 : « Il ne faut pas confondre l'argument  ad hominem avec l'argument ad personam, c'est-à-
dire avec une attaque contre la personne de l'adversaire et qui vise, essentiellement, à disqualifier 
ce  dernier.  La  confusion  peut  s'établir  parce  que  souvent  les  deux  espèces  d'argumentation 
interagissent.  Celui  dont  la  thèse  a  été  réfutée grâce  à  une  argumentation  ad  hominem,  voit  son 
prestige diminué, mais n'oublions pas que c'est là une conséquence de toute réfutation, quelle que 
\bigskip
\bigskip
\bigskip
74 
\bigskip
soit la technique utilisée : « Une erreur de fait, a déjà remarqué La Bruyère, jette un homme sage 
dans le ridicule (1). » 
\bigskip
D'autre  part,  en  utilisant  la  technique  de  l'aveu  que  nous  venons  d'examiner,  on  peut  passer,  à 
partir  des  actes  accomplis  par  quelqu'un,  aux  règles  de  conduite  qu'il  semble  implicitement 
approuver  et  qui  serviront  d'appui  à  une  argumentation  ad  hominem.  Les  argumentations  ad 
personam et ad hominem sont alors intimement mêlées, comme dans ce petit dialogue que nous 
trouvons chez Stevenson : 
\bigskip
A) Vous êtes beaucoup trop dur envers vos employés. 
\bigskip
B)  Mais  ce  n'est  certainement  pas  à  vous  de  parler  ainsi.  Votre  propre  usine  supporterait  une 
enquête bien moins facilement que la mienne (2). 
\bigskip
C'est en fonction de l'argumentation en général, et de l'argumentation ad hominem en particulier, 
que l'on peut comprendre en quoi consiste la Pétition de principe. 
\bigskip
Celle-ci est souvent considérée comme une faute dans la technique de la démonstration et Aristote 
en  traite  non  seulement  dans  les  Topiques  mais  aussi  dans  les  Analytiques  :  (3)  elle  consisterait 
dans le fait de postuler ce que l'on veut prouver. 
\bigskip
Constatons  immédiatement  que,  sur  le  plan  de  la  logique  formelle,  l'accusation  de  pétition  de 
principe est dénuée de sens. » 
\bigskip
(1) LA Bruyère, Bibl. la Pléiade, Les caractères, Des jugements, 47, p. 379. 
(2) Ch. L. Stevenson, Ethnics and language, p. 127. 
\bigskip
P  151  :  « On  pourrait,  en  effet,  prétendre  que  toute  déduction  formellement  correcte  consiste  en 
une pétition de principe, et le principe d'identité, qui affirme que toute proposition s'implique elle-
même, serait même la pétition de principe mise en forme. 
\bigskip
En fait, la pétition de principe, qui ne concerne pas la vérité, mais l'adhésion des interlocuteurs aux 
prémisses  que  l'on  présuppose,  n'est  pas  une  faute  de  logique,  mais  de  rhétorique  ;  elle  se 
comprend,  non  à  l'intérieur  d'une  théorie  de  la  démonstration,  mais  par  rapport  à  la  technique 
argumentative.  La  pétition  de  principe  consiste  en  un  usage  de  l'argument  ad  hominem  lorsqu'il 
n'est  pas  utilisable,  parce  qu'elle  suppose  que  l'interlocuteur  a  déjà  adhéré  à  une  thèse  que  l'on 
s'efforce justement de lui faire admettre. Encore faut-il que les deux propositions, le principe et la 
conclusion, qui ne sont jamais exactement les mêmes, soient suffisamment proches l'une de l'autre 
pour que l'accusation de pétition de principe soit justifiée. Aussi une discussion surgit-elle presque 
toujours au sujet du bien-fondé de l'accusation. 
\bigskip
L'auditeur  ne  pourra  prétendre  qu'il  y  a  vraiment  pétition  de  principe  que  si  la  prémisse  qu'il 
conteste n'a, en l'occurrence, aucun autre fondement que la conclusion même que l'on a voulu en 
tirer, et pour laquelle cette prémisse constituerait un chaînon indispensable dans le raisonnement. 
Il est extrêmement rare que cette dépendance soit suffisamment assurée pour que l'accusation soit 
admise  sans  réplique.  En  effet,  pareille  accusation  implique  que 
l'on  peut,  dans  une 
argumentation, discerner avec précision non seulement si, oui on non, l'énoncé d'une prémisse se 
distingue  de  l'énoncé  de  la  conclusion,  mais  encore  quelle  part  revient  à  un  certain  type 
d'arguments, et exclusivement à lui, dans le rapport « conclusion-prémisse-conclusion ». Or c'est à 
cause de la complexité de ce rapport que la discussion pour savoir s'il y a pétition de principe peut, 
en pratique, se développer. » 
\bigskip
\bigskip
\bigskip
\bigskip
75 
\bigskip
P  151-152 :  « L'importance  de  la  manière  dont  on  envisage  les  relations  entre  les  prémisses  et  la 
conclusion  se  montre  nettement  dans  cet  exemple  où  il  s'agit  des  rapports  entre  les  actes  et  la 
nature  d'une  personne.  Si  l'on  veut  faire  admettre  que  X  est  une  nature  courageuse,  et  si,  à  cet 
effet,  on  présente  un  de  ses  actes  comme  étant  une  manifestation  de  cette  nature  courageuse, 
l'interlocuteur pourra prétendre qu'il s'agit d'une pétition de principe ; par contre, cette accusation 
sera plus malaisée à soutenir si l'on considère ce même acte comme un exemple devant permettre 
une  généralisation.  C'est  ainsi  que,  pour  montrer  qu'il  ne  s'agit  pas  de  pétition  de  principe, 
l'orateur soulignera que la prémisse attaquée a un autre fondement que la conclusion, et que son 
rapport  argumentatif  avec  la  conclusion  est  d'une  autre  espèce  que  celui  qui  avait  été  supposé. 
Celui qui accuse son interlocuteur de commettre une pétition de principe aura donc tout intérêt à 
mettre le raisonnement en forme. » 
\bigskip
P 152 : « Voici une pétition de principe, signalée par Navarre à la suite de Blass, dans un passage 
du discours d'Antiphon sur le meurtre d'Hérodès 1731 : 
\bigskip
Sachez  bien  que  je  inérite  votre  pitié  beaucoup  plus  qu'un  châtiment.  Le  châtiment  revient,  en 
effet, aux coupables, la pitié à ceux qui sont l'objet d'une accusation injuste (1). » 
\bigskip
(1) O. NAVARRE, Essai sur la rhétorique grecque avant Aristote, p. 141, n. 1. Cf. F. BLASS, Die 
attische Beredsamkeit, 1, P. 122. 
\bigskip
P 152-153 : « L'ordre de la majeure et de la conclusion est inversé. La mineure sous-entendue « je 
suis l'objet d'une accusation injuste » ne peut être admise par les auditeurs parce que, si elle avait 
été accordée, le  procès serait jugé. C'est la raison  pour laquelle Antiphon,  au lieu de présenter le 
droit qu'il prétend avoir à la pitié comme la conclusion d'un syllogisme, présente son affirmation 
avant  la  majeure,  pour  lui  donner  une  sorte  de  validité  indépendante.  Remarquons  à  ce  propos, 
que  les  auteurs  anciens  affectionnent,  dans  leurs  discours,  de  présenter  les  questions  comme 
jugées en leur faveur et s'ingénient, par des artifices de forme, à dérouter ceux qui voudraient leur 
imputer une pétition de principe. Et ils y arrivent bien souvent. C'est ainsi que ni Blass, ni Navarre 
ne considèrent comme une pétition de principe une argumentation que l'on trouve dans l'exorde 
du  même  discours  d'Antiphon  [1  à  8],  et  qui  présente  une  structure  analogue  à  celle  que  nous 
venons d'analyser. »  
\bigskip
P  153 :  « Bentham  a  qualifié  de  «  pétition  de  principe  cachée  dans  un  seul  mot  »  l'utilisation 
d'appréciations  valorisantes  ou  dévalorisantes  dans  la  description  de  certains  phénomènes  (1). 
C'est le procédé que dénonce Schopenhauer lorsqu'il remarque que ce qui serait considéré comme 
« phénomène de culte », par un observateur neutre, le partisan l'appellera « expression de piété » 
et  l'adversaire  le  qualifiera  de  «  superstition  »  (2).  Mais  nous  ne  croyons  pas  que  l'on  puisse  en 
pareils  cas  parler  de  pétition  de  principe,  sauf  si  les qualifications  sont  censées  être  admises  par 
l'interlocuteur  qui  précisément  les  conteste;  faute  de  quoi  on  en  arriverait  à  considérer  comme 
pétition de principe toute affirmation de valeur. 
\bigskip
Pour  conclure, 
faute  d'argumentation.  Elle  concerne 
l'argumentation  ad hominem  et la présuppose, car son domaine n'est pas  celui de la vérité,  mais 
celui  de  l'adhésion.  Si  l'on  reconnaît  qu'il  est  illégitime  de  commettre  des  pétitions  de  principe, 
c'est-à-dire  de  fonder  son  argumentation  sur  des  prémisses  que  l'auditoire  rejette,  cela  implique 
que l'on peut se servir de celles qu'il admet. Quand il s'agit de vérité, et non d'adhésion, l'argument 
ad hominem est à proscrire, mais dans ce cas la pétition de principe est impossible. Les deux sont 
corrélatifs : on ne peut tenir compte de l'accusation de « pétition de principe » et peser la légitimité 
de la critique qu'elle implique que dans le cadre d'une théorie de l'argumentation. 
\bigskip
(1) BENTHAM, Œuvres, t. 1 : Traité des sophismes politiques, p. 481.  
(2) SCHOPENHAUER, éd. Piper, vol. 6 : Eristische Dialektik, p. 414 (Kunstgriff 12). 
\bigskip
la  pétition  de  principe  est  une 
\bigskip
\bigskip
\bigskip
76 
\bigskip
CHAPITRE  II  LE  CHOIX  DES  DONNÉES  ET  LEUR  ADAPTATION  EN  VUE  DE 
\bigskip
L'ARGUMENTATION 
\bigskip
§ 29. LA SELECTION DES DONNEES ET LA PRESENCE 
\bigskip
P 154 : « Les accords  dont l'orateur dispose, sur lesquels il peut prendre appui pour argumenter, 
constituent un donné, mais qui est si vaste et susceptible d'être utilisé de façons si diverses, que la 
manière de s'en prévaloir présente une importance capitale. Avant d'examiner l'usage argumentatif 
de ce donné, il est donc indispensable d'attirer l'attention sur le rôle de la sélection préalable des 
éléments,  qui  serviront  de  point  de  départ  à  l'argumentation,  et  de  leur  adaptation  aux  buts  de 
cette dernière. » 
\bigskip
P  154-155 :  « Précisons  pourtant  que  pouvoir  choisir  parmi  les  données  n'implique  pas  que  l'on 
puisse  faire fi  des  éléments  qui  seront  inutilisés.  Pour  chaque  auditoire  il  existe  un  ensemble  de 
choses admises qui toutes risquent d'influer sur ses réactions. Cet ensemble est relativement aisé à 
discerner  lorsqu'il  s'agit  d'un  auditoire  spécialisé  :  ce  sera  le  corpus  du  savoir  reconnu  par  les 
tenants  d'une  discipline  scientifique  (1)  ;  ce  sera  le  système  juridique  entier  dans  lequel  une 
décision judiciaire s'insère (1). Sauf s'il s'agit d'un domaine formalisé, complètement isolable, cet 
ensemble est fluide, toujours ouvert. Les contours en sont particulièrement vagues lorsqu'il s'agit 
d'un  auditoire  non-spécialisé,  encore  que  l'élaboration  philosophique  puisse  contribuer,  à 
certaines  époques,  à  quelque  peu  le  préciser.  En  tout  cas  il  constitue  pour  chaque  auditoire  un 
système de référence servant à éprouver les argumentations. » 
\bigskip
(1) Cf. G. T. Kneebone, Induction and Probabilitv, Proceedings of the Aristotelian Society, vol. L, 
1949-1950, p. 35. Pour les mathématiques, Cf. R. L. Wilder, The origin and growth of mathematical 
concepts, Bull. Amer. Math. Society, sept. 53, pp. 424-425. 
(1)  Cf.  C.  Cossio,  Phenomenology  of  the  decision  dans  Latin-American  legal  philosophy,  p.  399. 
Cité par V. Goldschmidt, Le système stoïcien, p. 97, n. 7. 
\bigskip
P 155 : « Ceci dit, le rôle de la sélection est si évident que, lorsque quelqu'un mentionne des faits, 
on doit toujours se demander ce que ceux-ci peuvent servir à confirmer ou à infirmer. La presse, 
gouvernementale  ou  d'opposition,  nous  a  habitués  à  cette  sélection  des  faits  en  vue  soit  d'une 
argumentation  explicite,  soit  d'une  argumentation  que  l'on  espère  voir  le  lecteur  effectuer  par 
luimême.  Dans  les  rhétoriques  traditionnelles,  au  chapitre  de  la  narration,  les  conseils  ne 
manquent pas sur la manière de choisir les faits de la cause  (2). Mais ce choix domine aussi dans 
les débats scientifiques : choix des faits estimés relevants, choix des hypothèses, choix des théories 
que l'on croira devoir confronter avec les faits, choix des éléments eux-mêmes qui constituent des 
faits.  La  méthode  de  chaque  science  implique  un  tel  choix  qui,  relativement  stable  dans  les 
sciences naturelles, est beaucoup plus variable dans les sciences humaines. 
\bigskip
Le  fait  de  sélectionner  certains  éléments  et  de  les  présenter  à  l'auditoire,  implique  déjà  leur 
importance  et  leur  pertinence  dans  le  débat.  En  effet,  pareil  choix  accorde  à  ces  éléments  une 
présence, qui est un facteur essentiel de l'argumentation, beaucoup trop négligé d'ailleurs dans les 
conceptions rationalistes du raisonnement. » 
\bigskip
(2) Rhétorique à Herennius, liv. 1, § 12. Cicéron, De Inventione, liv. I, § 30. Quintilien, Vol. II, liv. 
IV, chap. II, cf. notamment § 57. 
\bigskip
P 156 : « Un joli récit chinois illustrera notre pensée : 
\bigskip
Un  roi  voit  passer  un  boeuf  qui  doit  être  sacrifié.  Il  en  a  pitié  et  ordonne  qu'on  y  substitue  un 
mouton. Il avoue que cela est arrivé parce qu'il voyait le boeuf et qu'il ne voyait pas le mouton (1). 
\bigskip
\bigskip
\bigskip
\bigskip
77 
\bigskip
La  présence  agit  d'une  manière  directe  sur  notre  sensibilité.  C'est  un  donné  psychologique  qui, 
comme le montre Piaget, exerce une action dès le niveau de la perception : lors de la confrontation 
de deux éléments, par exemple un étalon fixe et des grandeurs variables auxquelles on le compare, 
ce sur quoi le regard est centré, ce qui est vu mieux ou plus souvent est, de ce seul fait, surévalué 
(2). Ainsi, ce qui est présent à la conscience acquiert une importance dont la pratique et la théorie 
de l'argumentation doivent tenir compte. En effet, il ne suffit pas qu'une chose existe pour que l'on 
ait le sentiment de sa présence. Ceci est vrai même dans des controverses savantes, témoin le rôle 
que  joua,  dans  la  querelle  gassendiste,  un  livre  où  jean  de  Lannoy  montrait  les  variations  dans 
l'attitude de l'Église envers Aristote : 
\bigskip
Certes,  nous  dit  à  ce  propos  l'abbé  Lenoble,  nul  n'ignore  que  l'Eglise  est  bien  antérieure  à 
l'Aristotélisme  du  XIIIe  siècle.  Cela,  tous  les  protagonistes  le  savent,  seulement,  personne  n'y 
pense (3). 
\bigskip
Aussi l'une des préoccupations de l'orateur sera-t-elle de rendre présent, par la seule magie de son 
verbe,  ce  qui  est  effectivement  absent,  et  qu'il  considère  comme  important  pour  son 
argumentation, ou de valoriser, en les rendant plus présents, certains des éléments effectivement 
offerts à la conscience. » 
\bigskip
(1) Meng-Tseu, Premier livre, § 7 (Pauthier, Confucius et Mencius, pp. 230 et suiv.). Résumé par 
Pareto, Traité de sociologie, I, p. 600 (§ 1135) à propos de son analyse de la pitié comme résidu. 
(2) Piaget, introduction à l'épistémologie génétique, vol. I, pp. 174-175. 
(3) B. Lenoble, Histoire et physique, Rev. d'Histoire des Sciences el de leurs applications, 1953, p. 
125. 
\bigskip
P  156-157 :  « Chez  Bacon,  le  rôle  de  la  rhétorique,  comme  technique  permettant  «  d'appliquer  la 
raison à l'imagination pour mieux mouvoir la volonté » (1) est lié essentiellement aux effets de la 
présence: 
\bigskip
Le sentiment considère seulement le présent; la raison considère l'avenir et la suite des temps. Et 
à  cause  de  cela,  le  présent  remplissant  plus  l'imagination,  la  raison  est  généralement  vaincue; 
mais après que la force de l'éloquence et la persuasion ont fait apparaître les choses éloignées et 
futures comme présentes, alors la raison prévaut sur la révolte de l'imagination (2). » 
\bigskip
(1) Bacon, 0f the advancement of learning, 2e livre, p. 156,  
(2) Ibid., p. 157. 
\bigskip
P 157 : « Bacon exprime, dans le langage philosophique de son temps, une idée proche de la nôtre : 
la  présence,  phénomène  psychologique  pour  commencer,  devient  un  élément  essentiel  dans 
l'argumentation. 
\bigskip
Certains  maîtres  de  rhétorique,  partisans  d'effets  faciles,  préconisent  le  recours,  pour  émouvoir 
l'auditoire, à des objets concrets, telle la tunique ensanglantée de César que brandit Antoine devant 
les  Romains,  tels  les  enfants  du  prévenu  que  l'on  amène  devant  les  juges  pour  exciter  leur  pitié. 
L'objet réel doit entraîner une adhésion que sa seule description semble incapable de produire ; il 
est un adjuvant précieux, mais à condition que l'argumentation mette en valeur ses aspects utiles. 
En  effet,  le  réel  peut  présenter  des  caractères  défavorables  qu'il  sera  malaisé  de  soustraire  au 
spectateur  ;  par  ailleurs  l'objet  concret  pourrait  distraire  l'attention  de  l'auditeur  dans  une 
direction qui s'éloigne de ce qui importe à l'orateur. Il ne faut donc pas confondre la présence, et 
les efforts en vue d'accroître le sentiment de présence, avec la fidélité au réel. 
\bigskip
\bigskip
\bigskip
\bigskip
78 
\bigskip
D'autre part, il ne faut pas non plus, comme on serait tenté de le faire en rationalisant par trop la 
pensée, vouloir réduire la présence à la certitude et traiter les événements plus éloignés du présent 
comme moins importants, parce que moins probables. » 
\bigskip
P 158 : « C'est la seule solution qui, selon Lewis, permettrait de rendre compatibles avec le calcul 
utilitaire, la proximité et l'éloignement, proposés par Bentham comme une dimension des plaisirs 
(1). Quelque anormale qu'elle soit dans son système, cette dimension supplémentaire que Bentham 
y  introduit  est,  pour  nous  qui  l'interprétons  en  fonction  de  la  présence,  parfaitement  justifiée, 
parce que conforme à des tendances psychiques indéniables. 
\bigskip
Whately reprend, dans un appendice  de sa  Rhétorique (2), une longue note de Campbell relative 
aux conditions de temps, de lieu, de connexion et d'intérêt personnel par lesquelles un événement 
nous affecte : ces conditions sont aussi celles qui déterminent la présence. La présence n'est donc 
pas  liée  exclusivement  à  la  proximité  dans  le  temps,  quoique  celle-ci  en  constitue  un  élément 
essentiel. Remarquons d'ailleurs que l'effort pour rendre présent à la conscience peut porter non 
seulement sur un objet réel, mais aussi sur un jugement ou tout un développement argumentatif. 
Cet effort vise, dans la mesure du possible, à f aire occuper, par cette présence, tout le champ de la 
conscience  et  à  l'isoler,  pour  ainsi  dire,  de  l'ensemble  mental  de  l'auditeur.  Et  cela  est  capital.  Si 
l'on  a  pu  constater  qu'un  syllogisme  bien  conduit,  et  accepté  par  l'auditeur,  ne  détermine  pas 
nécessairement  ce  dernier  à  agir  conformément  aux  conclusions,  c'est  que  les  prémisses,  isolées 
pendant  la  démonstration,  ont  pu  se  heurter  à  des  obstacles,  une  fois  rentrées  dans  le  circuit 
mental de celui qu'elles devaient persuader (3). 
\bigskip
(1) C. I. Lewis, An Analysis of Knowledge and Valuation, p. 493. 
(2) Richard D. D. Whately, Elements of Rhetoric, Appendix C, pp. 366 et suiv. 
(3) Cf. § 6 : Persuader et convaincre. 
\bigskip
P 158-159 : « L'importance de la présence dans l'argumentation ne se marque pas seulement d'une 
façon  positive  :  la  suppression  délibérée  de  la  présence  constitue  un  phénomène  tout  aussi 
remarquable  et  qui  mériterait  une  étude  détaillée.  Donnons  seulement  une  indication,  qui  nous 
paraît  essentielle,  sur  le  caractère  irréel  de  tout  ce  qui  ne  fait  pas  partie  de  notre  action,  ne  se 
rattache pas à nos convictions. Stephen Spender note, à ce sujet très justement :  
\bigskip
... presque tous les êtres humains ont une saisie très intermittente de la réalité. Un petit nombre 
de choses seulement qui illustrent leurs propres intérêts sont réelles pour eux : les autres choses 
qui, en fait, sont tout aussi réelles, leur apparaissent comme des abstractions... Vos amis, parce 
que  vos  alliés,  sont  de  vrais  êtres  humains...  Vos  adversaires  ne  sont  que  d'ennuyeuses,  peu 
raisonnables,  inutiles  thèses  dont  les  vies  ne  sont  que  de  faux  jugements  que  vous  souhaiteriez 
effacer avec une balle de plomb... (1). 
\bigskip
P 159 : « Et, en appliquant cette conception aux réactions qu'il éprouvait, lors de la guerre civile en 
Espagne, devant les atrocités des franquistes et celles des défenseurs de la République, il ajoute : 
\bigskip
Dans le premier cas, je voyais des cadavres, dans le second, seulement des mots. 
\bigskip
Dans le même livre, et à propos d'exécutions que nécessitait la bonne cause, Koestler remarque, à 
un certain moment : 
\bigskip
Maintenant ces deux individus devenaient pour moi plus réels que la cause au nom de laquelle ils 
allaient être sacrifiés (2). 
\bigskip
L'individu que l'on est prêt à sacrifier au système est irréel non seulement  en droit, parce  qu'il a 
perdu  son  statut  ontologique,  mais  aussi  en  fait,  parce  qu'il  est  privé  de  la  présence.  Le  choc  se 
\bigskip
\bigskip
\bigskip
79 
\bigskip
produit  soit  à  cause  du  doute  théorique  soit  quand,  dans  la  situation  concrète,  la  présence  de 
l'homme que l'on va sacrifier ne peut plus être refoulée de la conscience. 
\bigskip
La notion de présence, dont nous nous servons ici, et que nous croyons d'une importance capitale 
pour  la  technique  de  l'argumentation,  n'est  pas  une  notion  philosophiquement  élaborée.  Une 
philosophie qui ferait de la présence une pierre angulaire de sa constitution, comme celle de Buber 
ou de Sartre, la rattacherait à une ontologie ou une anthropologie. Tel n'est pas notre propos. » 
\bigskip
(1) Dans le livre collectif édité par R. Crosmann, The God that failed, pp. 253-254.  
(2) Ibid., p. 80. 
\bigskip
P 160 : « Nous tenons à l'aspect technique de cette notion qui mène à la conclusion inévitable que 
toute argumentation est sélective. Elle choisit les éléments et la façon de les rendre présents. Par là 
elle s'expose inévitablement au reproche d'être partielle, et donc partiale et tendancieuse. Et c'est 
un  reproche  dont  il  faut  tenir  compte  quand  il  s'agit  d'une  argumentation  que  l'on  veut 
convaincante,  c'est-à-dire  valable  pour  l'auditoire  universel.  Une  argumentation  tendancieuse, 
adoptée de propos délibéré, en vue d'un parti que l'on favorise par intérêt ou par fonction, devrait 
être complétée par l'argumentation adverse, afin de permettre un équilibre dans l'appréciation des 
éléments connus. Le juge ne décidera qu'après avoir entendu les deux parties. Mais passer de cette 
exigence à l'affirmation qu'il faut présenter la totalité des éléments d'information, en accordant à 
chacun la place qui lui revient, c'est supposer qu'il existe un critère permettant de déterminer quels 
sont  ces  éléments  relevants,  et  c'est  supposer  que  la  totalité  ainsi  définie  pourrait  être  épuisée. 
Nous pensons que c'est là une illusion et que le passage du subjectif à l'objectif ne peut se faire que 
par des élargissements successifs, dont aucun ne peut être considéré comme étant le dernier. Celui 
qui  effectue  un  nouvel  élargissement  mettra  nécessairement  l'accent  sur  ce  que  les  exposés 
précédents avaient procédé à un choix des données et parviendra sans  doute assez aisément à le 
montrer. Ajoutons que dans les sciences humaines, comme dans les sciences de la nature, ce choix 
n'est d'ailleurs pas seulement sélection, mais aussi construction et interprétation (1). 
\bigskip
Toute  argumentation  suppose  donc  un  choix  qui  consiste  non  seulement  dans  la  sélection  des 
éléments dont on se sert mais aussi dans la technique de leur présentation. Les questions de forme 
se mêlent à des questions de fond pour réaliser la présence. Pour les besoins de l'exposé, nous les 
traiterons successivement (2). 
\bigskip
(1) Cf. E. Aron, lntroduction à lu philosophie, (le l'Histoire. 1). 115. 
(2) Cf. § 37 : Problèmes techniques de présentation des données ;  § 42 : Les figures du choix, de la 
présence et de la communion. 
\bigskip
§ 30. L'INTERPRETATION DES DONNEES 
\bigskip
P  161 :  « L'utilisation  des  données  en  vue  de  l'argumentation  ne  peut  aire  sans  une  élaboration 
conceptuelle  qui  leur  donne  un  sens  et  les  rend  relevantes  pour  la  suite  du  discours.  Ce  sont  les 
aspects de cette élaboration - de cette mise en forme - qui fournissent un des biais par lesquels on 
peut le mieux saisir ce qui distingue une argumentation d'une démonstration. 
\bigskip
Toute  démonstration  exige  l'univocité  des  éléments  sur  lesquels  elle  se  fonde.  Ces  derniers  sont 
censés  être  compris  par  tous  de  la  même  façon,  grâce  à  des  moyens  de  connaissance  que  l'on 
suppose intersubjectifs, et, si ce n'est pas le cas, on réduit artificiellement l'objet du raisonnement 
aux  seuls  éléments  dont  toute  ambiguïté  semble,  en  fait,  écartée.  Ou  bien  le  donné  est  présenté 
immédiatement  comme  clair  et  significatif,  dans  une  conception  rationaliste  de  la  déduction,  ou 
bien l'on ne s'intéresse qu'aux seules formes des signes qui sont censées être perçues par tous de la 
même  façon,  sans  que  le  maniement  de  ces  derniers  prête  à  équivoque;  c'est  la  conception  des 
formalistes  modernes.  Dans  tous  ces  cas,  l'interprétation  ne  pose  aucun  problème  ou,  du  moins, 
\bigskip
\bigskip
\bigskip
80 
\bigskip
les  problèmes  qu'elle  pose  sont  éliminés  de  la  théorie.  Il  n'en  est  pas  de  même  quand  il  s'agit 
d'argumentation. » 
\bigskip
P 161-162 : « L'étude de l'argumentation nous oblige en effet à tenir compte non seulement de la 
sélection des données, mais également de la façon dont on les interprète, de la signification qu'on 
choisit de leur attribuer. C'est dans la mesure où elle constitue un choix, conscient ou inconscient, 
entre  plusieurs  modes  de  signification,  que  l'interprétation  peut  être  distinguée  des  données  que 
l'on interprète, et opposée à celles-ci. Ceci ne veut évidemment pas dire que nous adhérons à une 
métaphysique qui séparerait les données immédiates et irréductibles, des constructions théoriques 
élaborées à partir d'elles. Si nous devions adopter une position métaphysique, nous serions plutôt 
enclins  à  admettre  l'existence  d'un  lien  indissoluble  entre  la  théorie  et  l'expérience,  tel  que 
l'exprime le principe de  dualité de F.  Gonseth (1). Mais, pour l'instant, nos prétentions sont plus 
limitées.  Nous  voulons  seulement  insister  sur  le  fait  que,  dans  la  pratique  argumentative,  les 
données  constituent  des  éléments  sur  lesquels  semble  exister  un  accord  que  l'on  considère,  du 
moins  provisoirement  ou  conventionnellement,  comme  univoque  et  hors  discussion.  A  ces 
données,  on  opposera,  d'une  façon  consciente,  leur  interprétation,  quand  celle-ci  apparaîtra 
comme un choix entre des significations qui ne semblent pas faire corps, pour ainsi dire, avec ce 
qu'elles  interprètent.  C'est  justement  quand  des  interprétations  incompatibles  nous  font  hésiter 
sur  la  manière  de  concevoir  le  donné  que  le  problème  de  l'interprétation  se  pose  avec  force  ;  il 
passe à l'arrière-plan dès que, l'une des interprétations ayant paru la plus adéquate, elle est seule 
présente à la conscience. » 
\bigskip
(1)  Cf.  les  comptes  rendus  des  Troisièmes  Entretiens  de  Zurich  sur  le  principe  de  dualité, 
Dialectica, 22 à 25. 
\bigskip
P 162-163 : « Le problème qui nous préoccupe apparaîtra seulement dans toute son ampleur à qui 
se rendra compte  de ce que l'interprétation ne consiste pas seulement dans le choix, sur un plan 
bien défini, entre interprétations qui semblent incompatibles - quand on se demande, par exemple, 
si c'est le train où l'on se trouve ou le train voisin qui vient de se mettre en mouvement - mais aussi 
dans  le  choix  du  plan  sur  lequel  portera  l'effort  d'interprétation.  Un  même  processus  peut  être 
décrit,  en  effet,  comme  le fait  de  serrer  un boulon,  d'assembler  un  véhicule,  de  gagner  sa  vie,  de 
favoriser  le  courant  d'exportations  (2).  D'autre  part,  un  acte  peut  être  considéré  en  soi,  cerné 
autant  que  possible,  envisagé  sous  son  aspect  le  plus  contingent,  et  isolé  de  la  situation.  Mais  il 
peut  aussi  être  interprété  comme  symbole,  comme  moyen,  comme  précédent,  comme  jalon  dans 
une  direction.  Qu'elles  présentent  le  phénomène  à  tel  ou  tel  niveau  d'abstraction  ou  qu'elles  le 
rattachent à une situation d'ensemble - et notons, à ce propos, que l'interprétation peut être, non 
pas simple sélection, mais aussi création, invention  de signification  - ces diverses interprétations 
ne sont pas toujours incompatibles, mais la mise en évidence de l'une d'entre elles, la place qu'elle 
occupe  à  l'avant-plan  de  la  conscience,  rejette  souvent  les  autres  dans  l'ombre.  L'essentiel  d'un 
grand  nombre  d'argumentations  résulte  de  ce  jeu  d'interprétations  innombrables  et  de  la  lutte 
pour en imposer certaines, en écarter d'autres. » 
\bigskip
(2) Cf. E. Gellner, Maxims, Mind, July 1951, p. 393. 
\bigskip
P  163 :  « L'infinie  complexité  des  interprétations,  leur  mobilité  et  leur  interaction,  expliquent  à 
suffisance  l'impossibilité  de  réduire  tous  les  énoncés  à  des  propositions  dont  la  probabilité 
numérique  puisse  être  déterminée.  Même  si  une  augmentation  de  nos  connaissances  permet  de 
préciser  ces  probabilités,  c'est  uniquement  si  l'on  reste  dans  les  cadres  d'une  interprétation 
déterminée. Conventionnellement rien ne s'y oppose, mais rien non plus ne pourra empêcher une 
interprétation  nouvelle  d'être  mise  en  avant  ou  suggérée  implicitement  :  les  possibilités 
d'interprétation semblent inépuisables. » 
\bigskip
\bigskip
\bigskip
\bigskip
81 
\bigskip
P  163-164 :  « Parfois,  l'effort  de  ceux  qui  argumentent  ne  vise  pas  tellement  à  imposer  une 
interprétation  déterminée  qu'à  montrer  l'ambiguïté  de  la  situation  et  les  manières  diverses  de  la 
comprendre. Le fait d'accorder sa préférence à une certaine interprétation, ou même celui de croire 
à  l'existence  d'une  seule  interprétation  valable,  peuvent  être  révélateurs  d'un  système  particulier 
de  croyances  ou  même  d'une  conception  du  monde.  On  peut,  en  effet,  postuler  l'unicité 
d'interprétation  non  seulement  dans  un  cas  déterminé,  in  ais  aussi  comme  règle  générale.  Pour 
Pascal,  ce  qui  nous  empêche  de  reconnaître  les  vérités,  c'est  notre  volonté  corrompue  (1)  :  dans 
cette  conception  on  ne  conçoit  pas  de  justification  rationnelle  possible  pour  une  multiplicité 
d'interprétations. Les Anciens avaient qualifié de couleur les interprétations favorables à un parti : 
ce  terme  a  chez  eux  un  sens  péjoratif,  qui  tient  à  ce que  l'on  admet  qu'il  y  a  une  vérité  des  faits, 
connue de l'avocat et dont la couleur serait une altération (1). 
\bigskip
(1) Bibli. de la Pléiade, De l'esprit géométrique et de l'art de persuader, section II, p. 378. 
(1) Quintillien, Vol. II, liv. IV, chap. 11, 88. 
\bigskip
P  164-165 :  « Chez  les  Anciens,  qu'ils  soient  philosophes,  juristes  ou  théologiens,  l'interprétation 
concerne normalement des textes : ce sont surtout les psychologues modernes qui ont insisté sur 
l'ubiquité de l'interprétation, laquelle n'est pas absente même au niveau de la perception (2). Pour 
dissiper  quelque  peu  les  confusions  que  ces  usages  multiples  de  la  notion  d'interprétation  ne 
manqueraient pas de provoquer, nous suggérons une distinction - qui semble essentielle, dans une 
théorie de l'argumentation - entre l'interprétation de signes et celle d'indices. Nous entendons par 
signes  tous  phénomènes  susceptibles d'évoquer un  autre  phénomène,  dans  la  mesure  où ils  sont 
utilisés  dans  un  acte  de  communication,  en  vue  de  cette  évocation.  Qu'ils soient  linguistiques  ou 
non,  l'important,  pour  nous,  est  l'intention  de  communiquer  par  laquelle  ils  sont  caractérisés. 
L'indice, par contre, permet d'évoquer un autre phénomène, d'une façon, pour ainsi dire, objective, 
indépendamment  de  toute  intentionnalité.  Le  même  acte,  celui  de  fermer une  fenêtre,  peut  être, 
selon  les  cas,  signe  convenu  ou  indice  de  ce  que quelqu'un  a  froid.  L'ordre  «  sortez  !  »  peut  être 
simultanément interprété non seulement comme une invitation à sortir adressée à quelqu'un, mais 
aussi comme indice de la colère de celui qui le profère. Notre distinction, que l'on peut rapprocher 
de  celle  de  Jaspers,  entre  expression  et  symptôme  (3),  en  diffère  pourtant  en  ce  qu'elle  est 
strictement technique. En effet, l'interprétation comme signe ou comme indice pose des problèmes 
distincts,  quoique  ces  deux  espèces  d'interprétation 
inextricablement 
enchevêtrées. » 
\bigskip
(2) Cf. Clarapède, La genèse de l'hypothèse ; Merleau-Ponty, Phénoménologie de la perception. 
(3) K. Jaspers, Psychopathologie générale, chap. III. 
\bigskip
soient  parfois 
\bigskip
§ 31. L'INTERPRETATION DU DISCOURS ET SES PROBLEMES 
\bigskip
P  165 :  « Nos  considérations  ont  mis  en  évidence  l'ambiguïté  du  donné  argumentatif  qu'il  s'agit 
d'interpréter, comme la multiplicité des aspects, en constante interaction, par lesquels il se prête à 
l'interprétation.  Les  études  actuelles  sur  le  langage  comme  moyen  de  communication  sont 
dominées  par  les  problèmes  que  pose  l'interprétation.  On  ne  s'est  jamais  autant  émerveillé  qu'à 
l'époque contemporaine de ce que quelqu'un pouvait communiquer à autrui quelque chose qui eût, 
pour  l'auditeur,  une  signification  prévisible.  On  n'a  plus  considéré  l'incompréhension,  l'erreur 
d'interprétation, comme un accident évitable, mais comme la condition même du langage. On n'a 
plus distingué seulement la lettre et l'esprit pour les opposer, pour soutenir le droit à interpréter 
autrement que la lettre n'y autorise : on a vu dans la lettre elle-même un mirage qui se dissolvait, 
en  quelque  sorte,  entre  les  interprétations  possibles.  Dès  lors  on  assiste  à  un  effort  pour  trouver 
des  règles  permettant  de  limiter  les  trop  larges  possibilités  d'interprétation  théoriquement 
admissibles. 
\bigskip
Nul  n'y  a  travaillé  avec  plus  de  passion  que  I.  A.  Richards.  Pour  lui,  la  rhétorique  n'est  pas 
essentiellement,  comme  pour  nous,  liée  à  l'argumentation;  elle  est,  comme  pour  jean  Paulhan, 
\bigskip
\bigskip
\bigskip
82 
\bigskip
étude  de  l'expression,  mais  plus  spécialement  encore,  de  l'interprétation  linguistique  :  la 
rhétorique devrait être, d'après lui, l'étude du malentendu et des façons d'y remédier (1). » 
\bigskip
(1) A. Richards, The Philosophy of Rhetoric, p. 3. 
\bigskip
P  165-166 :  « Après  avoir  libéré  la  pensée  du  soi-disant  sens  unique  des  mots,  Richards  propose 
une  technique  d'interprétation.  Elle  consiste  à  chercher  un  sens  proche  de  celui  que  l'orateur 
attribuerait à ses propres paroles s'il pouvait observer lui-même son discours (1) L'auditeur trouve 
ce  sens  en  cherchant  «  ce  qui  lui  donne  satisfaction  (2)  »,  critère  applicable  parce  que  auteur  et 
auditeur ont en commun à la fois des expériences et des modes de réagir. La bonne interprétation 
d'une expression serait donc celle que l'auteur pourrait approuver, étant donné le contexte. » 
\bigskip
(1) A. Richards, Principles of literary criticism, p. 226. 
(2) ID., Interpretation in teaching, p. 68. 
\bigskip
P  166 :  « C'est  toujours  le  contexte,  nous  dit  Richards,  qui  assigne  à  un  mot  sa  fonction,  et  c'est 
seulement par le contexte que nous pouvons découvrir ce qu'il accomplit (3). Mais ce contexte - qui 
ne saurait être purement verbal - quels éléments de la situation englobe-t-il ? Lorsque l'enfant crie 
au  loup  pour  la  dixième  fois,  et  n'attire  plus  l'attention  malgré  le  danger  réel  qui,  cette  fois,  le 
menace, c'est que l'interprétation de ses cris a été déterminée par l'ensemble de la situation, dont 
les appels antérieurs font également partie. L'enfant ne souhaite pas cette extension du contexte. 
Dans  d'autres  cas,  par  contre,  l'auteur  lui-même  fait  effort  pour  que  certains  éléments  y  soient 
englobés. Tel auteur de théâtre donnera pour cadre à son dialogue la loge d'un concierge, tel autre, 
le monde naturel et surnaturel tout entier (4). » 
\bigskip
(3) Ibid., pp. VIII, 48, 62. Richards and Gibson, Learning basic English, 88. 
(4) Cf. Kenneth Burke, A Grammar of motives, p. 77. 
\bigskip
P  166-167 :  « Tout  auteur  doit  pouvoir  compter  sur  la  bonne  volonté  de  l'interprète  (5)  ;  celui-ci 
sera d'autant plus disposé à l'effort que le texte est prestigieux. Mais ne risque-t-on pas, par le fait 
même, d'imposer à l'auteur une interprétation qui sera fonction des convictions propres du lecteur. 
Quand  le  croyant  interprète  un  passage  de  la  Bible,  il  suppose  que  le  texte  est  non  seulement 
cohérent, mais encore véridique. Comme le dit Pascal: « Quand la parole de Dieu, qui est véritable, 
est  fausse  littéralement,  elle  est  vraie  spirituellement  ...  (1)  »  Mais  celui  qui  est  décidé  à  ne  rien 
rejeter  de  l'Écriture  lie  pourra  l'interpréter  qu'en  fonction  des  vérités  auxquelles  il  adhère 
préalablement.  Bien  que  dans  une  mesure  moindre,  dès  que  l'auteur  jouit  d'un  certain  crédit,  la 
bonne  volonté  dans  l'interprétation  de  son  texte  n'est  pas  indépendante  de  ce  que  l'interprète 
admet  puisqu'il  doit  incorporer  ce  que  l'auteur  apporte,  à  ses  propres  convictions.  Or  les  thèses 
admises  peuvent  varier  selon 
interne 
d'interprétation,  telle  la  cohérence,  se  double  inéluctablement  de  critères  venus  de  l'interprète. 
Rejeter les interprétations incohérentes, est, a priori, recommandable, mais cette préoccupation ne 
nous  fournit  pas  une  règle  de  conduite  suffisante  pour  nous  guider  dans  chaque  cas,  vers 
l'interprétation qui serait objectivement la meilleure. » 
\bigskip
(5) I. A. Richards, A symposium on emotive meaning, The philosophical Review, 1948, p. 145. 
(1) Pascal, Bibl. de la Pléiade, Pensées, 555 (31), p. 1003 (687 éd. Brunschvieg). 
\bigskip
P 167-168  : « Si l'interprétation d'un texte doit traduire l'ensemble des intentions de son auteur, il 
f  aut  tenir  compte  de  ce  que  ce  texte  comporte  souvent  une  argumentation  implicite,  qui  en 
constitue l'essentiel. Par exemple, lorsque Isocrate fait dire au fils d'Alcibiade : 
\bigskip
Tous  savent  en  effet  que  les  mêmes  hommes  ont  causé  la  destruction  de  la  démocratie  et  lé 
bannissement de mon père (2). 
\bigskip
lors,  toute  règle  soi-disant 
\bigskip
les 
\bigskip
interprètes.  Dès 
\bigskip
\bigskip
\bigskip
83 
\bigskip
 
il s'agit de faits vérifiables, mais ces paroles signifient : le bannissement de mon père a constitué 
un acte politique aussi condamnable que la destruction de la démocratie. Tout le sens de la phrase 
est dans l'argumentation implicite qui doit conduire à cette dernière conclusion. Alors que l'énoncé 
ne semble concerner que des faits, ce qu'il suggère est une appréciation. Or, la distinction entre ce 
qui est dit et ce qui n'est que construction surajoutée et sujette à controverse dépend de l'accord ou 
du  désaccord  concernant  l'interprétation  :  de  même  que  le  choix  effectué  par  l'orateur  d'une 
interprétation des faits ne s'en distingue que quand une autre interprétation apparaît possible, de 
même l'interprétation du texte vient se surajouter à ce dernier comme un élément distinct, quand 
il y a des raisons de la distinguer. » 
\bigskip
(2) Isocrate, Discours, t. 1 : Sur l'attelage, § 4. 
\bigskip
P 168 : « Outre les cas, que l'on ne peut exclure a priori, où  l'ambiguïté d'un texte est voulue et où 
tout effort pour le rendre univoque résulte d'une incompréhension, il est rare que, dans un langage 
non  formalisé,  le  texte  apparaisse,  aux  yeux  de  tous,  comme  absolument  clair.  Le  plus  souvent 
l'impression de clarté, liée à l'univocité, résulte de l'ignorance ou d'un manque d'imagination. C'est 
ce que Locke a fort bien noté quand il écrit : 
\bigskip
Plus  d'un  homme  qui,  à  première  lecture,  avait  cru  comprendre  un  passage  de  l'Ecriture  on  une 
clause du Code en a tout à fait perdu l'intelligence, après avoir consulté des commentateurs dont 
les  élucidations  ont  accru  ses  doutes  ou  leur  ont  donné  naissance,  et  plongé  le  texte  dans 
l'obscurité (1). 
\bigskip
La clarté d'un texte est conditionnée par les possibilités d'interprétation qu'il présente. Mais pour 
que  l'attention  soit  attirée  par  l'existence  d'interprétations  non-équivalentes,  il  faut  que  les 
conséquences  découlant  de  l'une  d'entre  elles  diffèrent,  en  quelque  manière,  de  celles  découlant 
d'une  autre  ;  or  il  se  peut  que  ce  ne  soit  que  dans  un  contexte  particulier  que  la  divergence 
parvienne à être perçue. La clarté d'un texte, ou d'une notion, ne peut donc jamais être absolument 
assurée, si ce n'est conventionnellement, en limitant volontairement le contexte dans lequel il y a 
lieu  de  l'interpréter.  La  nécessité  d'interpréter  se  présente  donc  comme  la  règle,  l'élimination  de 
toute interprétation constitue une situation exceptionnelle et artificielle. » 
\bigskip
(1) Locke, An Essay concerning human understanding, liv. III, chap. IX, § 9, p. 389, 
\bigskip
§ 32. LE CHOIX DES QUALIFICATIONS 
\bigskip
P 169 : « L'aménagement des données en vue de l'argumentation consiste non seulement dans leur 
interprétation, dans la signification qu'on leur accorde, mais aussi dans la présentation de certains 
aspects de ces données, grâce aux accords sous-jacents au langage dont on se sert. 
\bigskip
Ce choix se manifeste de la façon la plus apparente par l'usage de  l'épithète. Celui-ci résulte de la 
sélection  visible  d'une  qualité que  l'on  met  en  avant  et qui  doit  compléter notre  connaissance  de 
l'objet.  Cette  épithète  est  utilisée  sans  justification,  parce  qu'elle  est  censée  énoncer  des  faits 
incontestables ; seul le choix de ces faits apparaîtra comme tendancieux. Il est permis d'appeler la 
révolution française « cette sanglante révolution », mais ce n'est pas l'unique façon de la qualifier 
et d'autres épithètes pourraient être choisies tout aussi bien. Le rôle argumentatif des épithètes est 
le  plus  clairement  perçu  lorsque  deux  qualifications  symétriques  et  de  valeur  opposée  paraissent 
également possibles : qualifier Oreste « meurtrier de sa mère » ou « vengeur de son père », dire 
d'une mule « fille de baudet » ou « fille de coursier aux pieds rapides » (1), c'est choisir nettement 
un  point  de  vue  dont  on  perçoit  le  caractère  tendancieux  parce  que  l'on  voit  immédiatement 
comment on  pourrait le corriger. Mais toutes les épithètes ne se présentent pas comme un choix 
entre  deux  Points  de  vue  qui  exigent,  pour  ainsi  dire,  d'être  complétés  l'un  par  l'autre  :  le  plus 
souvent les aspects d'une réalité se situent sur des plans différents, et une vision plus complète du 
\bigskip
\bigskip
\bigskip
84 
\bigskip
réel  ne  peut  consister  que  dans  une  multiplication  progressive  d'aspects  sur  lesquels  on  attire 
l'attention. » 
\bigskip
(1) Aristote, Rhétorique liv. III, chap. 2, § 14, 1405 b. 
\bigskip
P 169-170 : « Si, lors du choix d'une épithète, l'aspect tendancieux de la présentation est aisément 
décelable, il n'en est pas de même quand il s'agit de la simple insertion d'un être dans une classe et 
de sa désignation pal: cette qualification même (1). Quand on désigne quelqu'un par les mots « le 
meurtrier »   .  le  choix  n'apparaît  point  aussi  nettement  que  dans  l'expression,  «  Oreste,  le 
meurtrier », parce que le choix paraît se confondre avec l'usage même des notions (2). Mais en fait, 
les  classifications  existantes  que  l'on  utilise  dans  la  qualification  sont  nombreuses,  et  il  n'est  pas 
possible  de  qualifier  sans  choisir,  en  même  temps,  la  classification  à  laquelle  on  accordera  la 
prééminence. Ce choix est rarement dépourvu d'intention argumentative. En effet, les classes sont 
caractérisées lion seulement par des caractères communs à leurs membres, mais encore, et parfois 
surtout,  par  l'attitude  adoptée  à  leur  égard,  la  manière  de  les  juger  et  de  les  traiter.  Les  diverses 
législations  réglementent  ce  rapport  :  déclarer  de  quelqu'un  qu'il  a  commis  un  vol,  c'est  aussi 
déterminer les peines dont il est passible, Dire de quelqu'un qu'il souffre de telle maladie, c'est déjà 
préjuger, du moins partiellement, du traitement qu'on lui fera subir. » 
\bigskip
(1)  Cf.  Cil.  Perelman  et  L.  Olbrechts-Tyteca,  Les  notions  et  l'argumentation,  vol.  Semantica, 
Archivio di Filosofia, 1955. 
(2) Ces considérations relatives à l'épithète et à l'insertion d'un être dans une classe valent, mutatis 
mutandis,  pour  les  adverbes  et  les  verbes,  qui,  les  uns  comme  les  autres,  permettent  de  choisir 
certains aspects des données pour les mettre en évidence. Le choix exprimé par J'adverbe sera plus 
visible  que  celui  exprimé  par  le  verbe.  Au  lieu  de  «  avancer  péniblement  "  on  emploiera  avec 
souvent  beaucoup  plus  d'efficacité  les  verbes  «  ramper  -  se  faufiler»  (cf.  Richard  Weaver,  The 
Ethics of Rhetoric, p. 135). Toutefois l'efficacité est plutôt celle de la métaphore endormie (cf. § 88 
; Les expressions à sens métaphorique ou métaphores endormies). 
\bigskip
P 170 : « Ainsi toute pensée conceptuelle s'insère dans des cadres tout formés (3), dont il faut se 
servir, et qu'il faut aménager au mieux des besoins de l'action sur autrui. » 
\bigskip
(3)  Cf.  Benjamin  Lee  Whorf,  The  Relation  of  Habitual  Thought  and  Behavior  to  Language  dans 
Language, Meaning and Waturity, edited by S. 1. Hayakawa, P. 22.). 
\bigskip
P 170-171 : « Non seulement l'argumentation concrète implique l'existence de classifications, mais 
parfois même on s'autorise de ces dernières pour disqualifier ce qui ne s'y insère pas et paraît, pour 
cette  raison,  défectueux.  Les  marxistes  classent  toutes  les  philosophies  en  matérialistes  ou 
idéalistes ; dès lors, les métaphysiciens qui ne se rangent pas dans l'une ou l'autre catégorie seront 
accusés de manquer de courage (1). » 
\bigskip
(1) H. Lefebvre, A la lumière du matérialisme dialectique, I : Logique formelle, logique dialectique, 
p. 25. 
\bigskip
P 171 : « Ces classifications peuvent être combattues, modifiées et adaptées, mais, le plus souvent, 
on  se  contentera  de  leur  opposer  d'autres  classifications,  jugées  plus  importantes,  plus 
intéressantes  ou  plus  fécondes.  Au  lieu  de  séparer  les  individus  en  pauvres  et  riches,  il  suffit  de 
mettre  à  l'avant-plan  l'opposition  des  noirs  et  des  blancs,  pour  que  le  pauvre  blanc  se  sente 
valorisé.  «  De  même,  nous  dit  S.  de  Beauvoir,  le  plus  médiocre  des  mâles  se  croit  en  face  des 
femmes un demi-dieu (2) » : une classification dominante, sur laquelle on porte l'attention, laisse 
dans l'ombre les autres classifications et les conséquences qu'elles comporteraient. 
\bigskip
C'est ainsi que, nous dit encore S. de Beauvoir, 
\bigskip
\bigskip
\bigskip
85 
\bigskip
 
une foi sincère, aide beaucoup la fillette à éviter tout complexe d'infériorité: elle n'est ni mâle ni 
femelle, mais une créature de Dieu (3). 
\bigskip
Saint Thomas se servira d'un procédé analogue, pour suggérer la supériorité de la connaissance 
relative  au  salut  sur  la  connaissance  des  phénomènes  sensibles  :  il  invite  l'homme,  nous  dit 
Gilson,  à  tourner  de  préférence  ses  regards  vers  un  autre  domaine  qui  n 1  est  plus simplement 
celui de l'homme, mais celui des enfants de Dieu (4). » 
\bigskip
(2) S. De Beauvoir, Le deuxième sexe, vol. 1, p. 25.  
(3) Ibid., vol. 11, p. 449.  
(4) Gilson, Le thomisme, p. 523. 
\bigskip
P  171-172 :  « Non  seulement  le  nom  commun  et  l'adjectif,  mais  aussi  le  nom  propre  petit  être 
utilisé  pour  opérer  ce  changement  de  point  de  vue.  Lorsque,  par  exemple,  après  le  désastre  des 
armées  anglaises  en  Hollande,  Pitt  demande  au  -Parlement  «  si  ce  n'était  pas  un  immense 
avantage pour l'Europe en général, que la Hollande n'ait pas été ajoutée sans lutte à la France (5) » 
il  modifie  l'appréciation  des  événements,  en  rapportant  le  désastre,  non  plus  au  groupe  restreint 
qu'est la Hollande, ni à l'Angleterre, dont il n'oserait pas mettre les seuls intérêts en cause, mais à 
une  notion  qui  les  englobe  toutes  les  deux  et  fournit  une  certaine  consolation  à  la  victime,  en 
rendant son sort solidaire d'un continent dont la défaite est loin d'être consommée. » 
\bigskip
(5) W. Pitt, Orations on the French war, p. 90. 
\bigskip
P  172 :  « Les  qualifications  présentent  parfois  un  caractère  tellement  inattendu  qu'on  y  verrait 
plutôt qu'un choix une figure. L'important est de voir ce qui en fait une figure argumentative (1). 
C'est la forme classificatoire, qui produit un effet saisissant. Voici un exemple tiré de Bossuet : 
\bigskip
Dans ces états déplorables [de misère publique] peut-on songer à orner son corps; et ne tremble-t-
on. pas de porter sur soi'la subsistance, la vie, le patrimoine des pauvres ? (2). 
\bigskip
Les  ornements  sont  qualifiés,  sans  plus,  de  subsistance  du  pauvre  :  la  forme  classificatoire 
considère comme acquis ce à quoi tend justement le sermon de Bossuet. 
\bigskip
La  qualification,  l'insertion  dans  une  classe,  peut  s'exprimer  non  par  l'emploi  d'une  notion  déjà 
élaborée,  mais  par  celui  d'une  conjonction  de  coordination,  telle  «  et  »  «  ou  »  «  ni  ».  Nous 
prendrons  deux exemples,  dans un même volume de Gide, oh il commence par s'insurger contre 
un procédé, qu'il n'hésite pas à utiliser quelques pages plus loin : 
\bigskip
Et je ne vous en parlerais même pas [du livre de Stirner], chère Angèle, si, par un procédé digne 
des lois scélérates, certains ne voulaient à présent lier le sort de Nietzsche à celui de Stirner, juger 
l'un avec l'autre pour les englober mieux tous deux dans une admiration ou une réprobation plus 
facile.  ...  indignez-vous  tout  simplement  en  entendant  dire  :  «  Stirner  et  Nietzsche  »  comme 
Nietzsche lui-même s'indignait en entendant dire : « Gcethe et Schiller » (3). » 
\bigskip
(1) Cf. § 41 : Figures (le rhétorique et argumentation. 
(2) Bossuet, Sermons, vol. Il : Sur l'intégrité de la pénitence, p. 616. 
(3) Gide, Prétextes, P. 135. 
\bigskip
P 173 : « Gide applique bientôt lui-même la technique honnie : 
\bigskip
On peut aimer ou ne comprendre point la Bible, aimer ou ne prendre point les Mlle Nuits et une 
Nuit,  mais,  s'il  vous  plaît,  je  partagerai  la  foule  des  pensants  en  deux  classes,  à  cause  de  deux 
\bigskip
\bigskip
\bigskip
86 
\bigskip
formes inconciliables d'esprit : ceux qui devant ces deux livres s'émeuvent ; ceux qui devant ces 
livres restent et resteront fermés (1). 
\bigskip
Ici point de conjonction « et » exprimée, mais c'est tout comme les deux livres sont insérés dans 
une même classe envers laquelle la réaction sera identique. Ici aussi il y a homogénéisation et, par 
là, égalisation des valeurs. Ni dans l'un ni dans l'autre cas, il n'y a argumentation en faveur de cette 
égalisation.  Mais  il  y  a  présentation  des  deux  termes  comme  si  leur  insertion  dans  une  même 
classe  allait  de  soi,  et  formation  d'une  classe  ad  hoc  par  la  réunion  des  deux  termes  sur  un  plan 
d'égalité.  Ce  procédé  de  qualification par  coordination  peut  s'appliquer  à  n'importe quel  objet.  Il 
suffit,  pour  y  arriver,  de  traiter  ces  objets  de  la  même  manière.  Les  auteurs  humoristiques,  les 
créateurs d'utopies, arrivent souvent à produire un effet comique en traitant de la même façon des 
comportements régis par des conventions sociales et d'autres qui ne le sont nullement. » 
\bigskip
P  173-174 :  « Pareil  traitement  n'aboutit  pas  nécessairement  à  la  formation  de  classes 
techniquement élaborées. Le plus souvent aucune notion ne permettra de les désigner : il suffit que 
les  individus  ainsi  juxtaposés  et  formant  classe  réagissent  les  uns  sur  les  autres  dans  l'esprit  de 
l'auditeur et c'est par là que cette technique prend une valeur argumentative. Il n'est cependant pas 
indifférent que l'insertion dans une classe se fasse ou non par emploi d'une qualification. La notion 
dont  on  se  sert  joue  souvent  un  rôle  essentiel,  ne  fût-ce  qu'à  cause  de  la  nuance  d'éloge  ou  de 
blâme qui s'y attache. Nous avons déjà vu que l'usage tendancieux de qualifications telles « tyran » 
ou « pirate » a été condamné par Bentham sous le nom de « pétition de principe en un seul mot » 
(I) Ce rôle des notions nous amène à considérer le choix sous son aspect peut-être le plus profond, 
c'est-à-dire le plus insidieux et aussi le plus inéluctable. » 
\bigskip
(1) Gide, Ibid., p. 175. 
\bigskip
§ 33. DE L'USAGE DES NOTIONS 
\bigskip
P 174 : « La qualification des données, leur insertion dans des classes constituent les deux aspects 
d'une  même  activité,  envisagée  tantôt  en  compréhension  tantôt  en  extension,  et  qui  est 
l'application  des  notions  à  l'objet  du  discours.  Ces  notions,  aussi  longtemps  que  leur  emploi  ne 
suscite  pas  de  difficultés,  se  présentent  également  comme  des  données  sur  lesquelles  on  croit 
pouvoir  tabler,  et  sur  lesquelles  on  table  en  effet  efficacement.  Mais  la  nature  de  cet  accord,  la 
conscience  de  sa  précarité,  de  ses  limites,  et  aussi  des  possibilités  argumentatives  qu'il  recèle, 
peuvent s'interpréter diversement. » 
\bigskip
(1) Cf. § 28 : L'argumentation ad hominem et lit pétition de principe. 
\bigskip
P  174-175 :  « Le  passage  univoque  du  mot  à  l'idée  qu'il  représenterait  est,  aux  yeux  des  anciens 
théoriciens, un phénomène découlant du bon usage du langage. On suppose en outre que cette idée 
peut être déterminée avec précision par le recours à d'autres idées, elles-mêmes exprimées par des 
termes univoques, ou qu'elle peut faire l'objet d'une intuition rationnelle (2). Le langage artificiel 
des  mathématiciens  fournit,  depuis  des  siècles,  à  beaucoup  de  bons  esprits,  un  idéal  de  clarté  et 
d'univocité  que  les  langues  naturelles,  moins  élaborées,  devraient  s'efforcer  d'imiter.  Toute 
ambiguïté,  toute  obscurité,  toute  confusion  sont,  dans  cette  perspective,  considérées  comme  des 
imperfections,  éliminables  non  seulement  en  principe,  mais  encore  en  fait.  L'univocité  et  la 
précision de ses termes feraient du langage scientifique l'instrument le meilleur pour les fonctions 
de  démonstration  et  de  vérification,  et  ce  sont  ces  caractères  que  l'on  voudrait  imposer  à  tout 
langage. » 
\bigskip
(2) Cf. Pascal, Bibl. de la Pléiade, De l'esprit géométrique, pp. 363-364. 
\bigskip
P 175 : « Mais toutes les fonctions du langage sont-elles pareillement liées à ces qualités et, peut-on 
même dire que le langage scientifique soit réellement exempt de toute ambiguïté ? Une discussion 
\bigskip
\bigskip
\bigskip
87 
\bigskip
qui s'est déroulée, à la suite d'un article de M. Black (1), dans une revue consacrée à la philosophie 
des sciences (2), permet à A. Benjamin d'aboutir à la conclusion que les idées vagues font partie 
intégrante de la science, et que toute théorie de la signification qui les nie n'est pas une théorie de 
la science (3). » 
\bigskip
(1) M. Black, Vagueness, dans Philosophy of Science, 4, 1937. 
(2) V. articles de Hempel, Copilowish et Benjamin, dans Philosophy of Science, 6, 1939. 
(3) Ibid., p. 430. 
\bigskip
P 175-176 : « Comment expliquer ce revirement ? Il résulte, semble-t-il, de ce que l'on a reconnu 
qu'une  notion  ne  peut  être  considérée  comme  univoque  que  si  son  champ  d'application  est 
entièrement  déterminé,  ce  qui  n'est  possible  que  dans  un  système  formel  dont  on  a  pu  éliminer 
tout imprévu : la notion de « fou » au jeu d'échecs satisfait à cette condition. Mais il n'en est pas de 
même  quand  il  s'agit  de  notions  élaborées  au  sein  d'un  système  scientifique  ou  juridique,  et  qui 
doivent s'appliquer à des événements futurs dont la nature ne peut pas toujours être complètement 
précisée. C'est pour tenir compte de cette situation que F. Waismann, dans un article remarquable, 
nous  demande  d'abandonner  l'idée  que  les  notions  scientifiquement  utilisables  puissent  être 
réduites  à  des  sense-data,  car  leur  usage  suppose  une  texture  adaptable  aux  exigences  d'une 
expérience future : 
\bigskip
Par exemple, écrit-il, nous définissons l'or par opposition à d'autres métaux, tels les alliages. Cela 
suffit pour nos besoins actuels et nous ne cherchons pas plus loin. Nous avons tendance à négliger 
le fait qu'il y a toujours d'autres directions dans lesquelles le concept n'a pas été défini. Pt si nous le 
faisions, nous pourrions aisément imaginer des conditions qui pourraient nécessiter de nouvelles 
limitations. Bref, il n'est pas possible de définir un concept comme l'or avec une absolue précision, 
c'est-à-dire de manière telle que tout recoin et toute fissure soient bloqués contre l'entrée du doute 
(1). » 
\bigskip
(1) F. Waismann, Verifiability, dans A. Flew, Essays on Logic and Language, p. 120. 
\bigskip
P  176  :  “Dans  la  mesure  où  les  expériences  futures  et  la  manière  de  les  examiner  ne  sont  pas 
entièrement  prévisibles,  il  est  indispensable  de  concevoir  les  termes  les  mieux  précisés  comme 
entourés  d'une  frange  d'indétermination  suffisante  pour  qu'ils  puissent  s'appliquer  au  réel.  Une 
notion parfaitement claire est celle dont tous les cas d'application sont connus, et qui n'admet donc 
pas  de  nouvel  usage  qui  serait  un  usage  imprévu  (2)  :  seule  une  connaissance  divine  ou 
conventionnellement limitée est adéquate à une telle exigence. 
\bigskip
Pour ces raisons, il n'est pas possible, comme le suggère Bobbio, de rapprocher la rigueur du droit 
de celle des mathématiques (3) ni, comme le propose Kelsen, de ne voir dans le droit qu'un ordre 
fermé (4)- En effet, le juge ne peut, à l'instar du logicien formaliste, limiter, une fois pour toutes, le 
champ d'application de son système. Il risque de se rendre coupable de déni de justice s'il refuse de 
juger  «  sous  prétexte  du  silence,  de  l'obscurité  ou  de  l'insuffisance  de  la  loi  »  (art.  4  du  Code 
Napoléon).  Il  doit,  chaque  fois,  pouvoir  juger  si  la  disposition  légale  invoquée  est  on  n'est  pas 
applicable à la situation, même si cette dernière n'a pas été prévue par le législateur : ceci l'oblige à 
prendre  une  décision  motivée  quant  à  la  manière  dont  il  précisera  l'une  ou  l'autre  catégorie 
juridique (5). » 
\bigskip
(2)  Cf.  Ch.  Perelman,  Problèmes  de  logique  juridique,  dans  Essais  de  logique  juridique,  Journal 
des Tribunaux, 22 avril 1956, p. 272. 
(3) Cf. N. Bobbio, Scienza del diritto e analisi del linguaggio, dans Saggi di critica delle scienze, p. 
55. 
(4) H. Kelsen, Reine Rechtslehre, 1934. 
\bigskip
\bigskip
\bigskip
88 
\bigskip
(5) Cf. Ch. Perelman, Le rôle de la décision dans la théorie de la connaissance, Actes du IIe Congrès 
international de Philosophie des Sciences, I, p. 150. 
\bigskip
P  176-177 :  « Lorsque  l'usage  des  notions  n'est  pas  formalisé,  l'application  de  celles-ci  pose  donc 
des problèmes relatifs à l'aménagement et à la précision des concepts. Ces problèmes sont d'autant 
plus inéluctables que les notions dont on se sert sont plus floues et plus confuses. C'est le cas, plus 
particulièrement,  des  notions  qui,  explicitement  ou  implicitement  se  réfèrent  à  des  ensembles 
indéterminés,  telles  les  tournures  négatives,  comme  «  ce  qui  n'est  pas  vivant  »,  «  ceux  qui  ne 
payent pas d'impôts ». C'est le cas surtout des notions confuses, telle la notion de justice (1), qui ne 
peuvent être précisées et appliquées que si l'on choisit et met en évidence certains de leurs aspects, 
incompatibles  avec  d'autres,  ou  encore  des  notions  comme  celle  de  mérite  dont  l'usage  ne  se 
conçoit qu'en fonction de leur confusion même : il s'agit d'évaluer en se référant, à la fois, au sujet 
agissant et au résultat obtenu (2). » 
\bigskip
(1) Cf. Ch. Perelman, De la Justice. 
(2) Cf  E. Dupréel, Sur les rapports de la logique et de la sociologie, ou théorie des Idées confuses, 
Rev. de métaphysique et de morale, juil. 1911 ; Le rapport social, pp. 227 et suiv. ; La logique et les 
sociologues, Rev. de l'Institut de Sociol. Solvay, 1924, n. 1, 2 ; La pensée confuse, Annales de l'Ecole 
des Hautes Etudes de Gand, III, 1939, repris dans Essais pluralistes. 
\bigskip
P 177-178 : « L'utilisation des notions d'une langue vivante se présente ainsi, très souvent, non plus 
comme  simple  choix  de  données  applicables  à  d'autres  données,  mais  comme  construction  de 
théories et interprétation du réel grâce aux notions qu'elles permettent d'élaborer. Il y a plus. Le 
langage  n'est  pas  seulement  moyen  de  communication  :  il  est  aussi  instrument  d'action  sur  les 
esprits, moyen de persuasion. Or on n'a pas encore mis suffisamment en évidence l'influence des 
besoins de l'argumentation sur la malléabilité des notions (3). En ce qui concerne notamment les 
notions fondamentales de la morale et de la philosophie, seules l'argumentation et la controverse 
permettent  d'expliquer  pourquoi  on  les  nuance,  pourquoi  l'on  introduit  des  distinctions  qui 
montrent  l'ambiguïté  de  ce  qui  avait  été  considéré  comme  clair  auparavant.  Et  c'est  justement 
parce que les notions utilisées dans l'argumentation ne sont  pas univoques et que leur sens n'est 
pas fixé ne varietur, que les conclusions d'une argumentation ne sont pas contraignantes. » 
\bigskip
(3) Cf. Ch. Perelman et L. Olbrechts-Tyteca, Les notions et l'argumentation, Archivio di Filosofia, 
1955. 
\bigskip
P  178 :  « Les  valeurs  admises  par  l'auditoire,  le  prestige  de  l'orateur,  le  langage  même  dont  il  se 
sert,  tous  ces  éléments  sont  en  constante  interaction  quand  il  s'agit  de  gagner  l'adhésion  des 
esprits. La logique formelle a éliminé tous ces problèmes de sa technique démonstrative, grâce à 
un  ensemble  de  conventions  parfaitement  fondées  dans  un  domaine  du  savoir  purement 
théorique.  Mais,  ce  serait  s'aveugler  et  méconnaître  certains  aspects  fondamentaux  de  la  pensée 
humaine, que d'ignorer l'influence que les besoins de décision et d'action exercent sur le langage et 
la pensée. » 
\bigskip
§ 34- CLARIFICATION ET OBSCURCISSEMENT DES NOTIONS 
\bigskip
P 178 : « La nécessité d'un langage univoque, qui domine la pensée scientifique, a fait de la clarté 
des  notions  un  idéal  que  l'on  croit  devoir  s'efforcer  toujours  de  réaliser,  en  oubliant  que  cette 
même clarté peut faire obstacle à d'autres fonctions du langage (1). C'est en raison d'ailleurs de cet 
idéal  que  l'on  s'est  occupé,  techniquement,  de  réaliser  cette  clarification  des  notions,  et, 
théoriquement, de la décrire, en ne s'occupant pas des occasions et des usages qui provoquent leur 
obscurcissement,  tout  comme,  dans  un  jardin  bien  tenu,  on  ne  se  préoccupe  pas  de  la  manière 
dont poussent les mauvaises herbes : on se contente de les arracher. Nous croyons, au contraire, 
que  l'usage  des  notions  et  la  réglementation  de  celui-ci  en  fonction  des  besoins,  doit  nous  faire 
\bigskip
\bigskip
\bigskip
89 
\bigskip
comprendre,  à  la  fois,  comment  les  notions  se  clarifient,  s'obscurcissent,  et  comment  parfois  la 
clarification des unes peut entraîner l'obscurcissement des autres. » 
\bigskip
(1) Cf. B. Parain, Recherches sur la nature et les fonctions du langage, p. 96. 
\bigskip
P 178-179 : « Nous venons de voir qu'une notion ne peut être parfaitement claire qu'au sein d'un 
Système  formel.  Dès  lors  que  certaines  expériences  sont  rapprochées  d'un  système  formel  qui 
devrait permettre (le les décrire et de les prévoir, une certaine indétermination s'introduit déjà du 
fait  qu'il  n'est  pas  dit,  a  priori,  comment  cette  intégration  de  l'expérience  sera  réalisée.  Une  fois 
l'intégration  réalisée,  le  système  en  question  comportera,  outre  les  règles  formelles,  des  règles 
sémantiques concernant l'interprétation des signes, leur application à un aspect déterminé du réel, 
considéré comme modèle du système envisagé. Il en résulte que, en dehors d'un pur formalisme, 
les  notions  ne  peuvent  rester  claires  et  univoques  que  par  rapport  à  un  domaine  d'application 
connu  et  déterminé.  Une  même  notion,  comme  celle  de  nombre,  dont  l'usage  est  parfaitement 
univoque dans un système formel, cessera d'avoir cette limpidité quand on s'en sert en ontologie. 
Inversement,  une  notion  éminemment  confuse,  comme  celle  de  liberté,  peut  voir  certains  de  ses 
usages clarifiés dans un système juridique où le statut des hommes libres est défini par opposition 
à celui des esclaves. Mais notons tout de suite que l'accord sur certains usages clairs d'une notion 
confuse,  s'il  rend  des  services  indéniables  dans  un  domaine  déterminé,  sera  inutilisable  dans  la 
plupart  des  cas  où  la  notion  confuse  était  employée  auparavant.  C'est  ce  qui  résulte  nettement 
d'une analyse comme celle entreprise par Dupréel de la notion de mérite (1). » 
\bigskip
P 179 : « Salvador de Madariaga nous rappelle, à ce propos, ce qui avait eté souvent dit des Anglais 
: 
\bigskip
Le sens de la complexité de la vie, qui rend la pensée anglaise concrète, la rend aussi vague. 
\bigskip
et, plus loin 
\bigskip
Le caractère complexe et vital de la  pensée anglaise demande donc  comme norme quelque  chose 
de plus compliqué et en même temps de plus élastique que la raison. Cette norme, c'est la sagesse 
(2). » 
\bigskip
(1) E. Dupréel, Essais pluralistes, pp. 328-329 (La pensée confuse).  
(2) Salvador De Madariaga, Anglais, Français, Espagnols, pp. 70, 77-78. 
\bigskip
P 180 : « Il faut noter, pourtant, que cet usage vague des notions se complète par la spécification de 
situations  traditionnellement  réglementées  ou  l'utilisation  de  ces  mêmes  notions  est  précisée  au 
maximum.  Mais  une  notion  confuse  ne  peut  être  épuisée  par  l'énumération  de  ses  cas 
d'application.  C'est  dire  aussi  qu'elle  ne  peut  être  rejetée  de  nos  préoccupations  par  la  critique 
successive d'une série de ses aspects : il ne suffit point de montrer que toutes les formes de justice, 
de liberté, de sagesse que l'on envisage sont un leurre pour dévaluer définitivement ces notions. 
\bigskip
Quand leur système de référence n'est pas indiqué et ne peut être suppléé d'une manière univoque 
ou,  même,  lorsqu'elles  sont  intégrées  dans  des  systèmes  idéologiques  fort  différents  les  uns  des 
autres, les notions confuses permettent la cristallisation d'un effort de bonne volonté global; mais 
leur application particulière aux fins d'une action concertée nécessitera, chaque fois, des mises au 
point  appropriées.  C'est  ainsi  que  l'adoption  de  la  déclaration  universelle  des  droits  de  l'homme 
par des partisans d'idéologies fort différentes a permis, comme le dit J. Maritain, d'aboutir à des 
normes  pratiques  qui  «  diversement  justifiées  pour  chacun,  sont  pour  les  uns  et  les  autres  des 
principes d'action analogiquement communs » (1). Seul l'usage de notions confuses, comprises et 
interprétées par chacun selon ses valeurs propres, a permis cet accord, dont le principal mérite est 
de  favoriser  un  dialogue  ultérieur.  Le  jour  où  des  tiers,  juges  ou  arbitres,  seront  désignés  pour 
\bigskip
\bigskip
\bigskip
90 
\bigskip
trancher  des  conflits,  sur  la  base  de  la  charte  adoptée,  l'interprétation  variable  de  chacun  des 
signataires comptera moins que le fait même d'avoir accepté le texte dont l'interprétation n'est pas 
univoque, ce qui augmentera d'autant le pouvoir d'appréciation des juges. » 
\bigskip
(1) Autour de la nouvelle déclaration universelle des droits de l'homme, lntroduction, p. 12. 
\bigskip
P  180-181 :  « Comme  le  sens  des  notions  dépend  des  systèmes  dans  lesquels  elles  sont  utilisées, 
pour changer le sens d'une notion, il suffit de l'insérer dans un nouveau contexte et notamment de 
l'intégrer  dans  de  nouveaux  raisonnements.  C'est  ce  que  remarque  finement  Kenneth  Burke  à 
propos des preuves cartésiennes de l'existence de Dieu : 
\bigskip
Un des éditeurs de Descartes, John Veitch a dit que lorsque Descartes mettait en doute un vieux 
dogme, plutôt que de l'attaquer de front, il visait à «saper ses fondements ». Et il se débarrassait 
des principes traditionnels « non tant par attaque directe qu'en substituant de nouvelles preuves 
et prémisses». Veitch cite aussi un défenseur de Descartes qui dit ironiquement que ses ennemis 
l'appelaient  athée  «  vraisemblablement  parce  qu'il  avait  donné  de  nouvelles  preuves  de 
l'existence de Dieu! » Mais ces nouvelles preuves étaient, en effet, de nouvelles déterminations de 
Dieu. Et par là, elles changeaient subtilement la nature de « Dieu » comme terme de motivation... 
(1). » 
\bigskip
(1) Kenneth BURKE, A Grammar of motives, p. 55. 
\bigskip
P  181  :  “Chaque  fois  que  l'on  présente  comme  élément  d'un  système  bien  structuré  une  notion 
traditionnellement  confuse,  le  lecteur  peut  avoir  l'impression  que  l'on  vient  d'exprimer  ce  qu'il a 
toujours pensé, s'il ne possédait pas lui-même de contexte suffisamment précis qui aurait fourni à 
cette notion certaines de ses déterminations. Mais si ce contexte existait, le lecteur croira plutôt à 
la trahison, comme c'est le cas des scolastiques indignés par les hardiesses d'un Descartes. » 
\bigskip
P 181-182 : « Les notions confuses mettent celui qui s'en sert en présence de difficultés qui, pour 
être  résolues,  demandent  un  aménagement  des  concepts,  une  décision  concernant  la  manière  de 
les comprendre dans un cas donné. Cette décision, une fois admise, aura pour effet de clarifier la 
notion  dans  certains  de  ses  usages  où  elle  pourra  jouer  le  rôle  de  notion  technique.  Une  notion 
paraît suffisamment claire aussi longtemps que l'on ne voit pas de situations où elle se prêterait à 
des  interprétations  divergentes.  Quand  une  pareille  situation  surgit,  la  notion  s'obscurcit,  mais 
après  une  décision  réglant  son  application  univoque,  elle  semblera  plus  claire  qu'elle  ne  l'était 
auparavant,  à  condition  que  cette  décision  soit  unanimement  admise,  si  pas  par  tous,  du  moins 
par tous les membres d'un groupe spécialisé, scientifique ou juridique. » 
\bigskip
P  182 :  « Les  notions  ont  d'autant  plus  de  chance  d'être  obscurcies  que  les  propositions  dans 
lesquelles elles sont insérées paraissent difficiles à rejeter, soit parce qu'elles confirment certaines 
valeurs universelles soit parce qu'elles sont obligatoirement valables, comme des textes sacrés ou 
des  prescriptions  légales.  Tout  l'effort,  en  effet,  ne  peut  porter  que  sur  l'interprétation  de  ces 
propositions. 
\bigskip
Rappelons à ce propos une pensée de La Bruyère : 
\bigskip
Les  mourants  qui  parlent  dans  leurs  testaments,  peuvent  s'attendre  à  être  écoutés  comme  des 
oracles : chacun les tire de son côté, et les interprète à sa manière, je veux dire selon ses désirs ou 
ses intérêts (1). 
\bigskip
Les mobiles qui poussent à des interprétations variées peuvent être plus nobles que ceux que cite 
La Bruyère : il peut s'agir, dans le cas du théologien, d'un souci de cohérence, dans le cas du juge, 
\bigskip
\bigskip
\bigskip
91 
\bigskip
d'un souci d'équité; ce qui nous importe, c'est de signaler les circonstances où des interprétations 
variées risquent de se produire et de contribuer à l'obscurcissement des notions. » 
\bigskip
(1) La Bruyère, Bibl. de la Pléiade, Caractères, De quelques usages, 56, p. 442. 
\bigskip
P  182-183 :  « Les  notions  s'obscurcissent  également  par  suite  des  troubles  que  des  situations 
nouvelles  peuvent  introduire  dans  les  rapports  admis  entre  leurs  différents  aspects.  Si  certains 
êtres se conduisent d'une façon déterminée, une liaison entre leur nature et leur comportement se 
fera  normalement  :  ce  dernier  sera  considéré  comme  l'expression  de  leur  essence.  Le  même 
adjectif  en  viendra  à  exprimer  d'une  façon  indiscernable  et  ambiguë,  une  détermination  dans 
l'espace  ou  dans  le  temps,  l'appartenance  à  un  parti,  et  une  façon  de  se  manifester  :  européen, 
moyenâgeux, libéral, qualifient une culture, un art, une politique, par leurs déterminations et par 
la  nature  de  leurs  manifestations.  Si  ces  dernières  en  arrivent  à  ne  plus  coïncider,  si  la  culture 
européenne  se  répand  dans  d'autres  continents,  si  les  églises  gothiques  sont  construites  au  xxe 
siècle,  si  des  membres  d'autres  partis  adhèrent  à  une  politique  libérale,  on  si,  inversement  des 
habitants de l'Europe se laissent influencer par la culture de l'Inde, si l'on trouve, au moyen âge, 
des  manifestations  d'art  classique,  et  si  des  membres  du  parti  libéral  préconisent  des  mesures 
socialistes,  les  notions  s'obscurcissent  et  l'on  se  demande  s'il  n'y  a  pas  lieu,  à  nouveau,  de 
rechercher un critère permettant leur application univoque. » 
\bigskip
P 183 : « Par ailleurs, l'usage des notions étant lié à leurs conséquences pratiques, la modification 
de ces  conséquences entraîne, par contrecoup, des réactions quant à  leur usage. En Belgique, un 
grand nombre de mesures légales ont été édictées après 1939 avec la clause qu'elles prendraient fin 
ail  jour  qui  serait  fixé,  par  arrêté royal,  pour  la  «  remise  de  l'armée  sur  pied  de  paix  ».  En  1947, 
deux ans après la fin des hostilités, alors que depuis longtemps l'armée belge avait été démobilisée, 
cet  arrêté  royal  n'avait  pas  encore  été  promulgué.  Comme  l'expliquait  M.  Lilar,  ministre  de  la 
justice à l'époque, 
\bigskip
Si  la  remise  de  l'année  sur  pied  de  paix  n'est  pas  encore  réalisée  à  l'heure  actuelle,  cela  tient 
exclusivement à des difficultés d'ordre juridique. En effet, cette remise de l'armée sur pied de paix 
est un acte d'une portée juridique considérable, qui nécessite la révision, texte par texte, de toute 
la législation de guerre et notamment de tous les arrêtés-lois issus des pouvoirs extraordinaires 
du 2o mars 1945, et frappés de caducité par le fait de la remise de l'armée sur pied de paix (1). » 
\bigskip
(1) Annales parlementaires de Belgique. Chambre des Représentants, séance du 5 février 1947, p. 
6. 
\bigskip
P 183-184 : « La limite de validité des pouvoirs spéciaux en vertu desquels des mesures avaient été 
prises, avait été fixée d'une manière  plus précise en se référant à un fait déterminé, la remise de 
l'armée  sur  pied  de  paix,  qu'elle  ne  l'eût  été  par  la  simple  mention  du  «  retour  à  des  conditions 
normales de vie ». Mais cet usage de la notion de « remise de l'armée sur pied de paix » n'était pas 
sans  réagir  sur  elle  :  suffisamment  claire  auparavant,  elle  s'obscurcissait  par  la  solidarité  établie 
entre elle et l'ensemble de ses conséquences juridiques. » 
\bigskip
P 184 : « Tout usage analogique ou métaphorique d'une notion l'obscurcit. En effet, pour qu'il y ait 
usage analogique, il faut que la notion soit appliquée à un domaine autre que son champ normal 
d'application  et  cet  usage  ne  peut  donc  pas  être  réglementé  et  précisé  (1).  Les  usages  futurs 
garderont,  qu'on  le  veuille  ou  non,  de  cet  usage  analogique,  une  trace  qui,  n'étant  pas 
nécessairement la même chez tous les usagers, ne peut que rendre la notion plus indéterminée. 
\bigskip
L'ensemble de ces circonstances, auxquelles il faut ajouter celles, très nombreuses, où la notion est 
modifiée par les besoins de l'argumentation elle-même, et dont nous traiterons dans le paragraphe 
\bigskip
\bigskip
\bigskip
92 
\bigskip
suivant,  contribue  à  ce  qu'on  appelle  la  vie  du  langage  et  de  la  pensée,  et  qui  conduit  à  une 
évolution du sens des mots. » 
\bigskip
(1) Cf. § 82 : Qu'est-ce que l'analogie. 
\bigskip
P  184-185 :  « Cette  évolution  peut  être,  à  son  tour,  utilisée  pour  obtenir  des  effets  poétiques 
capables eux aussi, de réagir sur l'usage linguistique. Charles Chassé a montré que Mallarmé s'est 
servi  de  beaucoup  de  mots  dans  leur  sens  ancien  et  périmé,  allant  jusqu'à  écrire  «La  clé  de 
Mallarmé  est  chez  Littré  »  (2)  ;  il  suffirait,  d'après  lui,  de  se  référer  à  ce  sens  pour  comprendre 
certains poèmes jugés obscurs. Mais insistons avec G. Jamati, et avec R. Caillois (3), sur ce que l'on 
ne peut, pour comprendre de tels textes, se contenter du sens ancien. En effet, on n'attend pas du 
lecteur qu'il fasse abstraction du sens actuel des mots ; ce dernier interfère avec l'ancien pour créer 
un ensemble conceptuel évocateur qui ne correspond à aucun moment de l'évolution sémantique 
et qui est plus flou que les sens déjà connus. » 
\bigskip
(2)  Ch.  Chassé,  La  clé  de  Mallarmé  est  chez  Littré,  Quo  radis,  mars-mai  1950  Les  clés  de 
Mallarmé, 19.54. 
(3)  G.  Jamati,  Le  langage  poétique,  dans  Formes  de  l'art,  formes  de  l'esprit,  pp.  271-272.  R. 
Caillois, Poétique de St-John Perse, pp. 22 et suiv. 
\bigskip
P 185 : « Notons, pour finir, que l'évolution des notions, suite à leur usage, aura un effet d'autant 
plus  déplorable  pour  leur  compréhension  univoque  que,  pour  la  plupart  des  esprits,  toute  cette 
évolution ne présente que des aspects fragmentaires, des mises au point, des approximations d'un 
même  concept,  qui  interagissent  les  uns  sur  les  autres.  L'orateur  devra,  à  tout  coup,  mettre  en 
évidence, rendre présents, certains de ces aspects au détriment d'autres. Il le fera le plus souvent, 
en se servant de leur plasticité et en adaptant les notions aux besoins de l'argumentation. C'est à 
l'examen de ces techniques d'adaptation que nous voudrions consacrer le prochain paragraphe. » 
\bigskip
§ 35. USAGES ARGUMENTATIFS ET PLASTICITE DES NOTIONS 
\bigskip
« La manière dont on présente les notions fondamentales dans une discussion dépend souvent du 
fait qu'elles sont liées aux thèses que l'on défend ou à celles de l'adversaire. En général, quand une 
notion  caractérise  sa  propre  position,  l'orateur  la  présente  comme  étant  non  pas  confuse,  mais 
souple, riche, c'est-à-dire comme recélant de grandes possibilités de valorisation et surtout comme 
pouvant  résister  aux  assauts  d'expériences  nouvelles.  Par  contre  les  notions  liées  aux  thèses  de 
l'adversaire seront figées, présentées comme immuables. En procédant ainsi l'orateur fait jouer, à 
son profit, l'inertie. La souplesse de la notion, que l'on postule dès l'abord et que l'on revendique 
comme  lui  étant  inhérente  permet  de  minimiser,  tout  en  les  soulignant,  les  changements  que 
l'expérience nouvelle imposerait, que les objections réclameraient : l'adaptabilité de principe à des 
circonstances  nouvelles  permettra  de  soutenir  que  l'on  maintient  vive  la  même  notion.  Voici 
quelques exemples. » 
\bigskip
P  186 :  « H.  Lefebvre  défend  un  matérialisme  souple  et  -riche,  alors  qu'il  fige  le  concept 
d'idéalisme : 
\bigskip
Pour le matérialisme moderne, l'idéalisme se définit et se critique, par son unilatéralité. Mais les 
matérialistes  ne  doivent  pas  laisser  simplifier  les  vérités  premières  du  matérialisme,  les  laisser 
retomber  au  niveau  du  matérialisme  vulgaire,  par  oubli  des  résultats  précieux  obtenus  par  les 
idéalistes dans l'histoire de la connaissance, et spécialement en logique (1). 
\bigskip
Le  matérialisme  petit  et  doit  englober  tout  ce  qui  est  valable,  il  bénéficie  d'une  plasticité  qui  est 
déniée explicitement à l'idéalisme, lequel se définit, comme dit l'auteur, par son « unilatéralité ». 
\bigskip
\bigskip
\bigskip
\bigskip
93 
\bigskip
La  même  rigidité  est  imposée  à  la  notion  de  «  métaphysique  »  considérée  comme  exprimant  un 
état  de  la  connaissance  dépassé  ;  l'auteur  se  demande  même  comment  la  métaphysique  fut 
possible  (2)  :  son  attitude  suppose  que  la  métaphysique  est  incapable  d'adaptation  et  de 
renouvellement, qu'elle est délimitée, une fois pour toutes, et que ses fonctions sont définitivement 
figées. On pourrait opposer à ce  point de vue les réflexions sur la métaphysique développées par 
l'un de nous qui a présenté les élargissements successifs de la métaphysique, et cherché à montrer 
la  permanence  de  celle-ci  :  métaphysique  comme  ontologie,  puis  comme  épistémologie,  puis 
comme élucidation des raisons de l'option axiologique, métaphysique future enfin, aux frontières 
imprévisibles (3). Sans le vouloir il a donné ainsi un exemple d'assouplissement d'une notion. 
\bigskip
Il  semble  que  la  technique  se  développe  souvent  sur  un  double  plan.  D'une  part,  nous 
assouplissons  en  fait  les  notions,  ce  qui  permet  leur  utilisation  dans  des  circonstances  s'écartant 
fort de leur usage primitif , d'autre part, nous qualifions de souples les notions en question. » 
\bigskip
(1)  H.  Lefebvre,  A  la  lumière  du  matérialisme  dialectique,  1  :  Logique  formelle,  logique 
dialectique, pp. 38-39. 
(2) Ibid., p. 20. 
(3)  Ch.  Perelman,  Philosophies  premières  et  philosophie  régressive,  dans  Rhétorique  et 
philosophie, pp. 85 et suiv. 
\bigskip
P  187 :  « Le  caractère  figé  des  concepts  de  l'adversaire  facilite  leur  réfutation  et  permet  de  les 
considérer  comme  périmés,  inadaptables  et,  par  là,  dépassés.  Les  conceptions  que  l'on  défend 
seront  celles  d'une  pensée  vivante,  souple,  adaptable,  et,  par  là,  toujours  actuelles.  Ces  divers 
procédés,  aussi  spontanés  qu'ils  soient,  sont  néanmoins  interprétés'  souvent,  par  l'adversaire, 
comme  indice  d'incompréhension  on  de  mauvaise  foi,  contre  laquelle  il  ne  manque  pas  de 
protester. 
\bigskip
L’assouplissement et le durcissement des notions est une technique adoptée lorsque l'appréciation 
qui les concerne doit résulter, au moins en partie, de l'argumentation. Par contre, quand la valeur 
désignée  par  la  notion  est  nettement  établie,  et  préalable  à  l'argumentation,  on  se  servira  d'une 
autre technique, portant plutôt sur l'extension des notions. Elle consiste, tout simplement, à élargir 
ou à restreindre le champ d'une notion de manière qu'elle englobe ou non certains êtres, certaines 
choses, certaines idées, certaines situations. Par exemple, on étendra le champ du terme péjoratif « 
fasciste » pour y englober certains adversaires ; tandis que l'on restreindra l'extension du terme « 
démocratique » qui est valorisant, pour les en exclure. Inversement, on limitera le sens du mot « 
fasciste » pour en exclure les amis que l'on soutient, et l'on étendra le sens du mot « démocratique 
»  pour  les  y  inclure.  Cette  technique  n'est  pas  uniquement  utilisée  en  politique.  On  la  rencontre 
même  dans  des  controverses  scientifiques.  C'est  ainsi  que  Claparède  constate  que  lorsque 
l'associationnisme a été abandonné par les psychologues, ils se critiquaient les uns les autres en se 
traitant  d'associationnistes  et  en  élargissant  progressivement  cette  notion  pour  permettre  d'y 
englober l'adversaire. Et Claparède termine son amusante analyse en concluant : « On est toujours 
l'associationniste de quelqu'un (1). » 
\bigskip
(1) Clarapède, La genèse de l'hypothèse, p. 45. 
\bigskip
P  188 :  « Il  résulte  de  ces  quelques  observations  que  l'usage  des  notions  en  fonction  du  désir  de 
valoriser  ou  de  dévaloriser  ce  qu'elles  qualifient  n'est  pas  sans  influer  profondément  sur  leur 
signification.  Celle-ci  n'est  point,  comme  certaines  analyses  tendraient  à  le  faire  croire,  une 
juxtaposition de deux éléments, l'un descriptif, l'autre émotif. Ce que l'on a appelé le « sens émotif 
» des notions (1) est une composante que le théoricien soucieux de rendre compte de la complexité 
des  eff  ets  du  langage  est  tenu  d'introduire  lorsqu'il  veut  corriger,  après  coup,  l'idée  que  la 
signification des notions est essentiellement descriptive, c'est-à-dire lorsqu'on a envisagé celles-ci 
d'une  manière  statique.  Mais  si  l'on  envisage  cette  signification  d'une  manière  dynamique,  en 
\bigskip
\bigskip
\bigskip
94 
\bigskip
fonction  des  usages  argumentatifs  de  la  notion,  on  voit  que  le  champ  d'application  de  la  notion 
varie avec ces usages, et que la plasticité des notions est liée à ceux-ci. La « signification émotive » 
fait partie intégrante de la signification de la notion; ce n'est point une adjonction supplémentaire, 
adventice,  étrangère  au  caractère  symbolique  du  langage  (2).  L'usage  argumentatif  des  notions 
influe  donc  sur  leur  confusion.  C'est dans  la  mesure  où  elles  servent  d'instrument  de  persuasion 
que  l'accord  sur  leur  utilisation  se  fera  plus  difficilement.  Les  valeurs  universelles,  que  l'on 
considère  comme  des  instruments  de  persuasion  par  excellence,  sont  désignées,  nul  ne  s'en 
étonnera, par les notions les plus confuses de notre pensée. 
\bigskip
Ces  remarques  suffisent,  pour  le  moment,  à  mettre  en  évidence  le  fait  que  la  présentation  des 
données  ne  consiste  pas  dans  un  simple  choix  entre  éléments  préalables,  mais  dans  un 
aménagement qui explique, au moins partiellement, le dynamisme du langage et de la pensée. » 
\bigskip
(1)  Ogden  and  Richards,  The  meaning  of  meaninq,  Ch.  L.  Stevenson,  Ethics  and  Language  ;  cf. 
aussi A Symposium on emotive meaning, Phil. Rev., 1948, pp. 111157. 
(2) Cf. Ch. Perelman, et L. Olbrechts-Tyteca, Les notions et l'argumentation. 
\bigskip
P  189 :  « Le  choix  des  prémisses  offrait  ceci  de  particulier  pour  notre  étude  qu'il  était  loisible  et 
utile de le reconnaître sous des avatars très divers : la sélection des données qui a pour corollaire 
de  leur  accorder  la  présence,  le  rôle  de  l'interprétation,  le  choix  de  certains  aspects  des  données 
réalisé  par  l'usage  de  l'épithète,  par  l'insertion  des  phénomènes  dans  l'une  ou  l'autre  classe 
préalablement connue des auditeurs et enfin le choix qui s'opère par l'usage et la transformation 
des notions elles-mêmes. Nous avons cru bon de sérier notre examen de telle manière que celui-ci 
apparaisse comme un approfondissement continu. Il ne faut pas se dissimuler que, en traitant de 
la sélection des données, de l'interprétation, de l'usage de l'épithète, de l'insertion dans une classe, 
du  recours  à  la  plasticité  des  notions,  nous  avons  repris  souvent,  sous  des  aspects  nouveaux, 
l'examen  d'un  même  processus  fondamental.  Il  nous  semble  pourtant  que  l'examen  d'aucun  des 
aspects  que  nous  avons  envisagés  ne  peut  être  négligé,  si  l'on  se  refuse  à  une  systématisation 
philosophique, ou même simplement technique, à tout le moins prématurée. L'ordre adopté dans 
notre  étude  nous  a  amenés  à  considérer  en  dernier  lieu  l'usage  et  la  transformation  des  notions, 
c'est-à-dire l'aspect sous lequel le problème du choix nous oblige à repenser , dans une perspective 
rhétorique, la plupart des problèmes sémantiques. » 
\bigskip
P 189-190 : « C'est dire que la forme sous laquelle sont énoncées les données est nécessairement en 
cause dans tout ce qui précède. Et l'on pourrait se demander si, du point de vue du raisonnement, 
d'autres  problèmes  qui  concerneraient  plus  spécialement  la  forme  sont  à  envisager.  C'est  ce  que 
nous  examinerons  au  cours  d'un  troisième  chapitre  relatif  à  la  présentation  des  données  et  à  la 
forme du discours. En quoi ce chapitre se distinguera-t-il surtout des précédents ? Uniquement en 
ce  que,  au  lieu  de  partir  de  points  de  vue  qui  traditionnellement  concernent  le  raisonnement,  la 
croyance, l'adhésion, bref ce qui est l'objet, ou la fin, de la persuasion, nous partirons de points  de 
vue  qui  traditionnellement  concernent  la  forme,  l'expression  de  la  pensée,  et  nous  nous 
attacherons à voir le rôle éventuel que diverses caractéristiques d'expression peuvent avoir dans la 
présentation  des  données.  C'est  dire  que  le  terme  «  forme  »  sera  utilisé  dans  un  sens  beaucoup 
plus proche de celui de l'écrivain que de celui du logicien. » 
\bigskip
CHAPITRE III PRÉSENTATION DES DONNÉES ET FORME DU DISCOURS 
\bigskip
§ 36. MATIÈRE ET FORME DU DISCOURS 
\bigskip
P 191 : « Nous avons déjà eu l'occasion de signaler, au chapitre précédent, quel rôle éminent il faut 
attribuer, dans l'argumentation, à la présence, à la mise en évidence, pour leur permettre d'occuper 
l'avant-plan de la conscience, de certains éléments sur lesquels l'orateur désire centrer l'attention. 
Avant même d'argumenter à partir de certaines prémisses, il est essentiel que le contenu de celles-
ci se détache sur le fond indifférencié des éléments d'accord disponibles : ce choix des prémisses se 
\bigskip
\bigskip
\bigskip
95 
\bigskip
confond  avec  leur  présentation.  Une  présentation  efficace,  qui  impressionne  la  conscience  des 
auditeurs,  est  essentielle  non  seulement  dans  toute  argumentation  visant  à  l'action  immédiate, 
mais  aussi  dans  celle  qui  vise  à  orienter  l'esprit  d'une  certaine  façon,  à  faire  prévaloir  certains 
schèmes interprétatifs, à insérer les éléments d'accord dans un cadre qui les rende significatifs et 
leur confère la place qui leur revient dans un ensemble. » 
\bigskip
P  191-192 :  « Cette  technique  de  la  présentation  a  même  pris  un  tel  développement  que  l'on  a 
réduit à son étude toute la matière de la  rhétorique, conçue comme art de bien parler et de bien 
écrire, comme un art d'expression  de la  pensée, de pure forme. C'est contre cette conception qui 
est à l'origine de la dégénérescence de la rhétorique, de sa stérilité, de son verbalisme, et du mépris 
qu'elle  a  finalement  inspiré,  que  nous  devons  nous  insurger.  Nous  refusons  de  separer  ,  dans  le 
discours, la forme du fond, d'étudier les structures et les figures de style indépendamment du but 
qu'elles  doivent  remplir  dans  l'argumentation.  Nous  irons  même  plus  loin.  Nous  savons  que 
certaines façons de s'exprimer peuvent produire un effet esthétique, lié à l'harmonie, au rythme, à 
d'autres  qualités  purement  formelles,  et  qu'elles  peuvent  avoir  une  influence  argumentative  par 
l'admiration,  la  joie,  la  détente,  l'excitation,  les  reprises  et  les  chutes  d'attention  qu'elles 
provoquent,  sans  que  ces  divers  éléments  soient  analysables  en 
fonction  directe  de 
l'argumentation.  Nous  exclurons  néanmoins  l'étude  de  ces  mécanismes,  malgré  leur  importance 
incontestable dans l'action oratoire, de notre présente analyse de l'argumentation. » 
\bigskip
P  192 :  « Ce  qui  nous  retiendra  dans  l'examen  de  la  forme  du  discours,  pour  autant  que  nous  la 
croyions  discernable  de  sa  matière,  ce  sont  les  moyens  grâce  auxquels  une  certaine  présentation 
des  données  situe  l'accord  à  un  certain  niveau,  l'imprime  avec  une  certaine  intensité  dans  les 
consciences, met en relief certains de ses aspects. C'est en pensant à des variations de forme, à des 
présentations diverses d'un certain contenu, qui n'est d'ailleurs pas tout à fait le même quand il a 
été autrement présenté, qu'il sera possible de dépister le choix d'une forme déterminée. De même 
que l'existence de plus d'une interprétation possible nous permet de ne pas confondre le texte avec 
les sens qu'on lui attribue, de même, c'est en pensant aux divers moyens dont l'orateur aurait pu se 
servir pour faire connaître à son auditoire la matière de son discours que nous arriverons, pour les 
besoins  de  l'exposé,  à  distinguer  les  problèmes  que  pose  la  présentation  des  données  de  ceux 
relatifs à leur choix. » 
\bigskip
§ 37. PROBLEMES TECHNIQUES DE PRESENTATION DES DONNEES 
\bigskip
P 193 : « Tout discours est limité dans le temps et il en est pratiquement de même de tout écrit qui 
s'adresse  à  des  tiers.  Que  cette  limitation  soit  conventionnellement  imposée  ou  dépende  de 
l'opportunité, de l'attention des auditeurs , de leur intérêt, de la place disponible dans un journal 
ou une revue, des frais qu'entraîne l'impression d'un texte, la forme du discours ne peut pas ne pas 
en  tenir  compte.  Le  problème  général  de  l'ampleur  du  discours  retentit  immédiatement  sur  la 
place  que  l'on  accordera  à  l'exposé  des  éléments  de  départ,  sur  le  choix  de  ceux-ci  et  la  manière 
dont  on  les  présentera  aux  auditeurs.  Celui  qui  prononce  un  discours,  visant  à  la  persuasion  - 
contrairement  aux  exigences  d'une  démonstration  formelle  où,  en  principe,  rien  ne  devrait  être 
sousentendu  -  se  doit  de  ménager  son  temps  et  l'attention  des  auditeurs  :  il  est  normal  qu'il 
accorde à chaque partie de son exposé une place proportionnelle à l'importance qu'il voudrait lui 
voir attribuer dans la conscience de ceux qui l'écoutent. 
\bigskip
Quand une certaine prémisse est connue de tout le monde, et qu'elle n'est pas en discussion, le fait 
de l'énoncer pourrait paraître ridicule : 
\bigskip
Si l'une des prémisses est connue, écrit Aristote, il n'est même pas besoin de l'énoncer; l'auditeur la 
supplée; par exemple, pour conclure que Dorieus a reçu une couronne comme prix de sa victoire, il 
suffit de dire : il a été vainqueur à Olympie; inutile d'ajouter : à Olympie, le vainqueur reçoit' une 
couronne; c'est un fait connu de tout le inonde (1). » 
\bigskip
\bigskip
\bigskip
\bigskip
96 
\bigskip
(1) Aristote, Rhétorique, 1, chap. 2, 1357 a. 
\bigskip
P 193-194 : « Cette remarque, qui est indiscutablement juste, appelle pourtant deux observations. 
Il n'est pas toujours aussi facile d'indiquer la prémisse sous-entendue, et cette prémisse n'est pas 
toujours  aussi  assurée  que  dans  l'exemple  cité.  Des  orateurs  ne  se  font  pas  faute  d'utiliser  cette 
latitude  pour  passer  sous  silence  des  prémisses fort  discutables  au  contraire,  et  sur  lesquelles  ils 
préfèrent  ne  pas  attirer  l'attention  de  leur  auditoire.  D'autre  part,  certains éléments  indubitables 
méritent pourtant qu'on s'étende longuement sur leur signification et leur importance, au lieu de 
les sous-entendre ou simplement de les mentionner. En prolongeant l'attention qu'on leur accorde, 
on augmente leur présence dans la conscience des auditeurs. Certains conseils précis des rhéteurs 
anciens sont destinés à nous rappeler cette technique d'accentuation d'un point, par le temps qu'on 
lui consacre : 
\bigskip
J'avais coutume aussi, écrit Quintilien, de détacher les points dont mon adversaire et moi étions 
d'accord...  et  non  seulement  de  tirer  des  aveux  de  mon  adversaire  toutes  les  conséquences 
possibles, mais de les multiplier au moyen de la division (1). » 
\bigskip
P 194 : « Le conseil d'Aristote, juste quand il s'agit d'un fait qui sert uniquement de chaînon dans 
une argumentation, doit être remplacé par celui de Quintilien quand il s'agit de faits indubitables, 
mais qu'il y a lieu de mettre en valeur en les rendant familiers. Si le style rapide est favorable au 
raisonnement, le style lent est créateur d'émotion : « car l'amour se forme par l'habitude... D'oil il 
advient que les orateurs concis et brefs pénètrent peu le coeur, et émeuvent moins (2). » 
\bigskip
P  194-195 :  « La  répétition  constitue  la  technique  la  plus  simple  pour  créer  cette  présence  ; 
l'accentuation de certains passages, par le son de la voix ou par le silence dont on les fait précéder, 
vise  au  même  effet.  L'accumulation  de  récits,  même  contradictoires,  sur  un  sujet  donné  ,  peut 
susciter l'idée de l'importance de celui-ci. Une avalanche de livres relatifs à un même pays agit non 
seulement  par  leur  contenu,  mais  aussi  par  le  seul  effet  d'une  présence  accrue.  La  littérature 
romantique, drame et nouvelle, remit en honneur le moyen âge et, en lui rendant la présence, elle 
servit comme le dit justement Reyes, d'éperon à la pensée historique (1). » 
\bigskip
(1) Quintilien, Vol. III, liv. VII, chap. 1, § 29. 
(2) Vico Mlle instituzioni oratorie, p. 87. 
(1) A. Reyes, EI Deslinde, p. 101. 
\bigskip
P 195 : « L'insistance peut être réalisée d'ailleurs par des moyens plus indirects : il est permis de se 
demander si l'un des effets bienfaisants de certains textes obscurs n'est pas de vivifier l'attention ; 
la « présence d'esprit » rendrait présent ce que l'on veut communiquer (2). Parfois l'auteur spécule 
sur ce que l'auditeur, devant un signe qui dément son attente, accordera à celui-ci une importance 
accrue. Aragon lui-même analyse ce mécanisme à propos de deux vers du cantique à Elsa : 
\bigskip
\bigskip
Ce -ne sont plus les jours du vivre séparés 
\bigskip
.................... 
\bigskip
Et jamais tu ne lus si lointaine à mon gré... 
\bigskip
\bigskip
On  me  concédera  que  le  pluriel  de  séparés,  impliquant  deux  personnes,  ajoute  à  l'expression.  Si 
j'avais  alors  choisi  une  rime  plurielle,  I's  finale  de  séparés  passerait  pour  une  cheville  ou  une 
erreur, et l'intention en échapperait (3). 
\bigskip
L'accumulation de récits contradictoires sur un sujet donné n'agit pas seulement, sans doute, par 
l'effet  de  niasse  auquel  nous  faisions  plus  haut  allusion,  mais  encore  par  le  problème  qu'évoque 
cette multiplicité. » 
\bigskip
\bigskip
\bigskip
\bigskip
97 
\bigskip
(2) Cf. J. Cocteau, La difficulté d'être, p. 177. 
(3) Aragon, Les yeux d'Elsa, p. 23. 
\bigskip
P  195-196 :  « La  technique  de  l'accumulation,  de  l'insistance,  est  fréquemment  liée  à  une  autre 
technique, celle de l'évocation des détails, au point que les deux sont souvent indiscernables. On 
traitera un sujet en faisant se succéder description synthétique, globale, et analyse ou énumération 
des détails. Dans son oraison funèbre de Turenne, Fléchier décrit les réactions provoquées par la 
mort du maréchal : 
\bigskip
Que  de  soûpirs  alors,  que  de  plaintes,  que  de  louanges  retentissent  dans  les  villes,  dans  la 
campagne !  L'un voyant croître ses moissons, bénit la mémoire de celuy...  L'autre... souhaite une 
éternelle paix à celuy qui... Icy l'on offre le Sacrifice adorable de Jésus-Christ pour l'aine de celuy 
qui...  Là  on  luy  dresse  une  pompe  funèbre...  Ainsi  tout  le  Royaume  pleure  la  mort  de  son 
défenseur; ... (1). » 
\bigskip
(1) Fléchier,  oraison funèbre de Henri de La Tour d'Auvergne, vicomte de Turenne, Paris, 1676, 
pp. 100-101. 
\bigskip
P 196 : « Dans d'autres cas, on explicitera les étapes successives d'un phénomène, la manière dont 
on en a pris conscience. Les étapes évoquées peuvent être celles de l'action à accomplir. Les agents 
de publicité savent qu'en indiquant le détail des opérations à faire pour passer la commande, ils la 
rendent présente à la conscience et facilitent la prise de décision. L'impression de réalité est créée 
de  même  par  l'amoncellement  de  toutes  les  conditions  qui  précèdent  un  acte  ou  l'indication  de 
toutes ses conséquences. Voici deux exemples de ces procédés empruntés à Proust : 
\bigskip
[Tante Léonie affirme qu'elle va sortir.] A Françoise incrédule elle faisait non seulement préparer 
d'avance  ses  affaires,  faire  prendre  l'air  à  celles  qui  étaient  depuis  longtemps  enfermées,  mais 
même commander la voiture, régler, à un quart d'heure près, tous les détails de la journée (2). 
\bigskip
De  même,  pour  qu'Albertine  ne  pût  pas  croire  que  j'exagérais  et  pour  la  faire  aller  le  plus  loin 
possible dans l'idée que nous nous quittions, tirant moi-même les déductions de ce que je venais 
d'avancer,  je  m'étais  mis  à  anticiper  le  temps  qui  allait  commencer  le  lendemain  et  qui  durerait 
toujours, le temps où nous serions séparés, adressant à Albertine les mêmes recommandations que 
si nous n'allions pas nous réconcilier tout à l'heure (3). 
\bigskip
Il y a un parallélisme frappant entre  ces procédés qui donnent la présence et la méthodologie de 
l'hypothèse. Formuler une hypothèse, ce n'est pas poser une affirmation isolée, car l'explicitation 
de  celle-ci  n'est  possible  que  par  l'énumération  des  conditions  qu'on  lui  impose  et  des 
conséquences  qu'on  en  déduit.  C'est  la  raison  pour  laquelle  à  côté  d'hypothèses  scientifiques qui 
servent à l'invention, nous rencontrons des hypothèses argumentatives. 
\bigskip
(2) Proust, A la recherche du temps perdu, vol. 12 : La prisonnière, II, p. 190. 
(3) Ibid., p. 191. 
\bigskip
P 197 : « Dans un de ses discours, Démosthène évoque l'hypothèse où Eschine serait l'accusateur, 
Philippe le juge et lui-même l'accusé (1). Il imagine, dans cette situation fictive, le comportement, 
les réactions de chacun, pour en déduire ce que doivent être le comportement et les réactions, dans 
la situation réelle. Parfois, au contraire, l'hypothèse est décrite dans tous ses détails, pour la rendre 
violemment  indésirable  ou  choquante.  Ces  deux  possibilités  nous  indiquent  les  deux  usages 
argumentatifs  habituels  de  toutes  les  formes  de  l'utopie.  Comme  le  prétend  très  justement  R. 
Ruyer, l'utopie cherche moins la vérité qu'une augmentation de conscience, elle confronte le réel 
avec  une  présence  imaginaire,  qu'elle  impose  pour  en  tirer  des  réactions  plus  durables  (2).  C'est 
pourquoi, l'utopie proprement dite tend à se développer dans ses plus infimes détails : on n'hésite 
\bigskip
\bigskip
\bigskip
98 
\bigskip
pas  à  maintenir  l'auditoire  dans  ce  milieu  nouveau  pendant  de  longues  heures.  La  réussite  n'est 
possible que si la structure logique du milieu imaginaire est la même que celle du milieu habituel 
du lecteur, et si les événements y produisent normalement les mêmes conséquences. Les mythes 
collectifs, les récits légendaires qui font partie d'un fonds commun de culture, ont cet avantage sur 
les hypothèses et les utopies qu'ils bénéficient bien plus facilement de la présence. Pour combattre 
la croyance en la supériorité de la main droite sur la main gauche, Platon nous dit : « Si l'on avait 
cent  mains,  comme  Géryon  et  Briarée,  il  faudrait,  de  ces  cent  mains,  être capable  de  lancer  cent 
traits (3). » Il passe  ainsi de l'ancienne structure où il y avait une différence qualitative entre les 
deux  mains  à  une  structure  où  les  mains  sont  homogènes.  Parce  qu'elle  peut  se  référer  à  la 
mythologie,  l'hypothèse  de  Platon  s'impose  plus  aisément  à  l'attention  ;  elle  semble  moins 
arbitraire, moins abstraite. » 
\bigskip
(1) Démosthène, Harangues ei plaidoyers politiques, t. III : Sur l'ambassade, 1, § 214. 
(2) B. Ruyer, L'utopie et les utopies, chap. Il. 
(3) Platon, Lois, VII, 794 d et suiv., cf. P.-M. Schuhl, Le merveilleux, la pensée et l'action, p. 186. 
\bigskip
P  198 :  « Pour  créer  l'émotion  (1),  la  spécification  est  indispensable.  Les  notions  générales,  les 
schèmes  abstraits  n'agissent  guère  sur  l'imagination.  Whately  signale,  dans  une  note,  qu'un 
auditoire  qui  était  resté  insensible  devant  des  propositions  générales  sur  le  carnage  dont  fut 
marquée la bataille de Fontenoy, fut ému jusqu'aux larmes par un petit détail relatif à la mort de 
deux  jeunes  gens  (2).  Pour  donner  l'impression  de  présence  il  est  utile  de  préciser  le  lieu  et  le 
moment  d'une  action  ;  Whately  conseille  même  d'utiliser,  chaque  fois  qu'il  y  a  moyen,  le  terme 
concret  au  lieu  du  terme  abstrait.  Plus  les  termes  sont  spéciaux,  plus  l'image  qu'ils  évoquent  est 
vive, plus ils sont généraux, plus elle est faible. C'est ainsi que dans le discours d'Antoine, dans le 
jules César de Shakespeare, les conjurés ne sont pas désignés comme ceux qui ont « tué » César, 
mais ceux dont « les poignards ont percé » César (3). Le terme concret accroît la présence. 
\bigskip
Le  conseil  semble  bon,  en  règle  générale,  mais  si  l'on  veut  préciser  l'opposition  entre  termes 
abstraits  et  concrets,  on  constatera  l'existence  de  plusieurs  espèces  d'abstractions  qui  agissent 
certainement d'une façon variable sur le sentiment de la présence. On s'efforce souvent de définir 
ces  espèces  d'abstractions,  soit  par  ce  qu'elles  abandonnent  du  concret,  soit  par  leur  caractère 
constructif  :  «  homme  »  serait  du  premier  type,  «  vérité  »  du  second  (4).  Mais  on  voit 
immédiatement que la ligne de clivage entre concret et abstrait dépend dans tous les cas du point 
de départ que l'on se donne, lequel sera fourni par notre conception du réel. » 
\bigskip
1.  Emotion  et  présence  sont  intimement  liées  si  on  suppose,  comme  le  fait  D.  0.  Hebb  dans  The 
organization  of  behavior,  que  l'émotion  ralentit  le  processus  de  pensée,  rendant  ainsi  l'objet  " 
intéressant ». Cf. Hebb et Thompson, Handbook of social psychology, édité par Lindzey, Vol. I. P. 
553. 
(2) Whately, Elements of Rhetoric, p. 130, note. 
(3) Ibid., pp. 194 à 197. 
(4) Cf. notamment Schopenhauer, éd. Brockhaus, vol. 2 : Die Welt als Wille und Vorstellung, Band 
1, § 9, p. 49. 
\bigskip
P  198-199 :  « Abandonnant  l'opposition  entre  concret  et  abstrait,  on  peut  tenter  de  hiérarchiser 
certains  niveaux  d'abstraction.  Commentant  l'échelle  de  Korzybski  qui  monte  vers  les  niveaux 
d'abstraction  les  plus  élevés,  Hayakawa  signale  qu'aux  niveaux  les  plus  bas,  les  faits  eux-mêmes 
agissent directement sur notre affectivité (1) ; mais cela n'est pas toujours vrai, si l'on pense que, 
pour  Korzybski,  la  vache  que  nous  percevons  serait  plus  abstraite  que  les  atomes,  les  électrons, 
dont elle est constituée et qu'appréhende la science (2). » 
\bigskip
P 199 : «  Polir se rendre compte de la complexité du problème, il suffit de réfléchir à l'action que 
produisent  sur  notre  imagination  les  mêmes  faits,  dont  l'aspect  quantitatif  nous  est  présenté, 
\bigskip
\bigskip
\bigskip
99 
\bigskip
tantôt en chiffres absolus, tantôt en chiffres relatifs. Très généralement, les chiffres absolus parlent 
plus vivement à l'imagination ; les objets, quoique envisagés sous leur aspect purement quantitatif, 
sont  en  effet  des  individualités  indépendantes  présentes  au  plus  haut  point.  Mais  l'inverse  peut 
aussi se produire, notamment lorsque le chiffre relatif, qui n'est certainement pas plus concret que 
le chiffre absolu, se réfère à un événement qui nous touche : par exemple, la probabilité de mourir 
dans  l'année  de  telle  ou  telle  maladie.  Le  même  rapport  numérique  peut  sembler  plus  ou  moins 
concret  suivant  l'intérêt  que  nous  y  portons.  Le  degré  d'abstraction  ne  déterminerait  pas  tant  la 
présence qu'il ne serait déterminé, apparemment, par elle. Et au lieu de résoudre le problème qui 
nous intéresse, à l'aide de considérations ontologiques, ne serait-il pas plus juste de faire dépendre 
notre  idée  du  concret  de  l'impression  de  présence  que  provoquent  en  nous  certains  niveaux  de 
présentation des phénomènes ? » 
\bigskip
(1) Hayakawa, Language in Thought and Action, p. 127.  
(2) Ibid., p. 169. 
\bigskip
P 199-200 : « La façon intuitive de s'exprimer, l'usage du terme qui frappe nlest pas toujours sans 
inconvénient. Schopenhauer a constaté que certains écrivains évitent, dans la mesure du possible, 
l'expression la plus définie, et préfèrent l'usage de notions plus abstraites qui leur permettent plus 
facilement  d'échapper  aux  objections  (1).  La  remarque  est  juste,  et  pleine  d'enseignements.  Si  le 
terme concret et précis permet l'établissement d'un accord, à la fois grâce à la présence qu'il crée et 
l'univocité  qu'il  favorise,  il  ne  faut  jamais  oublier  que,  dans  certains  cas,  seul  l'usage  d'un  terme 
abstrait permet de ne pas dépasser les possibilités d'un accord. A la limite, le terme le plus concret, 
le plus présent, peut correspondre à l'inexprimable, ne plus être que le démonstratif fugace d'une 
présence infiniment mouvante. Le désir d'exprimer le concret dans son unicité, poussé trop loin, 
peut  être  non  pas  la  base  d'une  bonne  entente,  mais  le  renoncement  à  toute  entente.  La 
présentation  des  données  doit  s'adapter,  dans  chaque  cas,  aux  conditions  d'une  argumentation 
efficace. » 
\bigskip
§ 38. FORMES VERBALES ET ARGUMENTATION 
\bigskip
P 200-201 : « La présentation des  données  n'est pas indépendante des  problèmes de langage.  Le 
choix des termes, pour exprimer sa pensée, est rarement sans portée argumentative. Ce n'est que 
suite  à  la  suppression  délibérée  ou  inconsciente  de  l'intention  argumentative  que  l'on  peut 
admettre 
l'existence  de  synonymes,  de  termes  qui  seraient  susceptibles  d'être  utilisés 
indifféremment  l'un  pour  l'autre  ;  c'est  alors  uniquement  que  le  choix  de  l'un  de  ces  termes  est 
pure  question  de  forme,  et  dépend  de  raisons  de  variété,  d'euphonie,  de  rythme  oratoire.  Cette 
intention  négative  semble  aller  de  soi  chaque  fois  que  l'intention  argumentative  ne  peut  être 
connue, comme dans les dictionnaires où les termes paraissent interchangeables, parce qu'ils sont 
mentionnés  à  toutes  fins  utiles.  Mais  quand  il  s'agit  de  leur  utilisation  par  un  orateur  dans  un 
discours  particulier,  l'équivalence  des  synonymes  ne  peut  être  assurée  qu'en  égard  à  la  situation 
d'ensemble dans laquelle s'insère son discours, et notamment à certaines conventions sociales qui 
pourraient le régir. Parfois le choix d'un terme sera destiné à servir d'indice, indice de distinction, 
de  familiarité  ou  de  simplicité.  Parfois  il  servira  plus  directement  à  l'argumentation,  en  situant 
l'objet du discours dans une catégorie mieux que ne le ferait l'usage du synonyme : c'est à pareille 
intention que pourrait correspondre le choix du mot « hexaèdre » au lieu de «cube ». 
\bigskip
(1) Schopenhauer, éd. Brockhaus, vol. 6 : Parerga und Paralipomena, II, § 283, p. 552. 
\bigskip
P 201 : « On repère généralement l'intention argumentative par l'indice que présente l'usage d'un 
terme s'écartant du langage habituel. Il va sans dire que le choix du terme habituel peut également 
avoir valeur d'argument ; d'autre part, il y aurait lieu de préciser où et quand l'usage d'un terme 
déterminé  peut  être  considéré  comme  habituel  ;  grosso  modo,  nous  pourrons  considérer  comme 
habituel le terme qui passe inaperçu. Il n'existe pas de choix neutre -mais il y a un choix qui paraît 
neutre et c'est à partir de celui-là que peuvent s'étudier les modifications argumentatives. Le terme 
\bigskip
\bigskip
\bigskip
100 
\bigskip
neutre  dépend  évidemment  du  milieu.  Par  exemple,  sous  l'occupation  allemande,  en  Belgique, 
dans certains milieux il était sans doute normal de désigner l'Allemand sous le terme de « boche)?. 
Dès  lors  l'emploi  du  terme  «  allemand  »  pouvait  indiquer  soit  un  fléchissement  général  de 
l'attitude  hostile  envers  l'ennemi,  soit  une  estime  particulière  pour  un  Allemand  déterminé  qui 
méritait cette considération. De même l'usage de la périphrase « personne ayant une disposition à 
induire  en  erreur  »  pour  désigner  «  le  menteur  »  peut  avoir  pour  but  de  dépouiller  autant  que 
possible  ce  terme  de  l'élément  dévalorisant  pour  l'assimiler  à  un  terme  descriptif  et  donner  au 
jugement  dans  lequel  il  intervient  l'allure  d'un  jugement  de  fait  (1),  d'où  signification 
argumentative de cette périphrase, que ne possède pas le terme « menteur ». 
\bigskip
( 1) Nous nous servirons a plusieurs reprises des notions de jugement de valeur et jugement de fait 
au sens généralement admis aujourd'hui. Notre traité tend par ailleurs a montrer qu'il n'existe pas 
de distinction nette et fondamentale entre ces deux espèces de jugements. 
\bigskip
P  202 :  « Ces  deux  exemples  montrent  bien  que  le  terme  (tue  nous  appelons  neutre,  c'est-à-dire 
celui qui passe inaperçu, est loin d'être toujours celui que l'on appelle généralement descriptif ou 
factuel.  Rien  à  cet  égard  n'est  plus  arbitraire  que  les  distinctions  scolaires entre  discours  factuel, 
neutre,  descriptif  et  discours  sentimental,  émotif  :  ces  distinctions  n'ont  d'intérêt  que  dans  la 
mesure  oh  elles  attirent  l'attention  de  l'étudiant  sur  l'introduction  manifeste  de  jugements  de 
valeur  dans  l'argumentation,  mais  elles  sont  néfastes  dans  la  mesure  où  elles  font  sous-entendre 
qu'il  existe  des  manières  de  s'exprimer  qui  seraient  descriptives  en  soi,  des  discours  oh  seuls 
interviennent les faits et leur objectivité indiscutable. » 
\bigskip
P 202-203 : « Pour discerner l'usage argumentatif d'un terme, il est donc important de connaître 
les mots ou les expressions dont l'orateur aurait pu se servir et auxquels il a préféré le mot utilisé : 
l'ensemble  de  locutions  disponibles  pourrait  être  appelé,  pour  suivre  la  terminologie  des 
significistes  hollandais,  une  famille  de  mots,  qui  ne  sont  pas  des  mots  liés  par  un  système  de 
dérivations,  mais  des  expressions  apparentées  par  leur  sens  (1).  Bien  entendu,  la  constitution 
d'une  pareille  famille  de  mots  n'est  pas  dépourvue  de  quelque  arbitraire,  car  cette  famille  n'est 
déterminée  par  aucun  autre  critère  que  l'idée  préalable  que  nous  avons  du  concept  que  cette 
famille  permettra  d'élucider.  L'évolution  du  concept  dépendrait  des  variations  dans  l'usage  de 
chacun des quasi-synonymes (2) faisant partie de cette famille : ces termes formeraient un système 
en  interaction  (3).  Rien  ne  s'opposerait  d'ailleurs  à  ce  que  les  termes  de  plusieurs  langues 
différentes  soient  considérés  comme  faisant  partie  d'une  même  famille  de  mois,  à  condition  que 
les contacts entre ces cercles linguistiques différents soient suffisants. Peut-être est-ce même dans 
ce domaine peu étudié que l'intervention de la notion de famille de mots pourrait rendre le plus de 
services.  On  verrait  sans  doute  que  l'introduction  d'un  terme  étranger,  avec  ses  nuances 
particulières de signification, peut avoir pour effet de modifier le concept déjà existant et aussi de 
donner  à  chacun  des  quasi-synonymes  un  arrière-plan  nouveau.  Aujourd'hui  le  sens  du  terme  « 
honneur  »  est  certainement  influencé  à  la  fois  par  le  terme  français  «  honneur  »  et  le  terme 
espagnol « honor », tout au moins dans la conscience des lettrés, pour lesquels ils appartiennent à 
une même famille de mots. » 
\bigskip
(1) Cf. G. Mannoury,  Handboek der analytische signifika, 1, pp. 43, 126; B. Stokvis,  Psychologie 
der suggestie en autosuggestie, p. 19. 
(2) Le terme de quasi-synonyme est à prendre ici au sens le plus large car il peut englober des mots 
de forme grammaticale très diverse, tel " nécessaire " et a cause de ". 
(3)  Les  notions  de  Sprachfeld (champ  linguistique),  de  Bedeutungsfeld (champ  de  signification), 
introduites  par  la  linguistique  structurelle,  pourraient  également  servir  à  l'étude  du  choix 
argumentatif.  Cf.  J.  Trier,  Der  deutsche  Wortschalz  im  Sinnbezirk  des  Verstandes,  pp.    1  à  26  ; 
Sprachliche Felder, Zeitsch. für deutsche Bildung, janv. 1932, pp. 417 à 427 ; Das Sprachliche Feld, 
Neue  Jahrbücher  für  Wissenschaft und  Jugendbildung,  1934,  5,  pp.  428-480.  Sur ',ces  "  champs 
sémantiques, cf. S. Ullmann, Précis de sémantique française, pp. 303-309. 
\bigskip
\bigskip
\bigskip
101 
\bigskip
 
P  203 :  « Les  termes  d'une  même  famille  forment  un  ensemble  par  rapport  auquel  un  terme  se 
spécifie  :  ils  sont  en  quelque  sorte  le  fonds  sur  lequel  le  terme  utilisé  se  détache.  Par  contre,  les 
termes  apparentés  par  dérivation  s'influencent  directement  l'un  l'autre.  Les  Anciens  parlaient 
volontiers de l'argument par les flexions (1), qui consiste à appliquer un même prédicat aux termes 
dérivés  l'un  de  l'autre,  tels  «  justement  »  et  «  juste  ».  Ce  genre  d'argument  est  sujet  à  bien  des 
objections  en  ce  qu'il  néglige  notamment  la  divergence  des  évolutions  sémantiques. Mais il reste 
que,  avant  toute  argumentation,  il  est  important  souvent  de  présenter  un  énoncé  en  termes 
susceptibles d'en évoquer d'autres par dérivation, vraie ou imaginaire. » 
\bigskip
(1) Cf. Aristote, Rhétorique, liv. 11, chap. 23, 1397 a ; cf. aussi les arguments apparentés ", Cicéron, 
Topiques, § 12 ; Quintilien, Vol. 11, liv. V, chap. X, § 85. 
\bigskip
Plus  efficace  encore  sera  le  rapprochement  des  termes  dans  un  même  contexte.  Quand  on 
disqualifie une théorie, en la traitant de simpliste, dire immédiatement après d'une autre théorie 
qu'elle  n'est  guère  moins  simple  (2)  c'est  lui  attacher  une  nuance  péjorative,  alors  que, 
habituellement,  la  simplicité  d'une  théorie  scientifique  constitue  une  qualité  indéniable.  Nous 
trouvons  chez  Jouhandeau  un  bel  exemple  d'interaction  de  termes,  réalisés  grâce  à  des  effets 
purement  formels.  A  un  général  allemand  qui  avait  pris  possession,  en  1940,  de  son  manoir  et 
venait de lui faire l'éloge de la France, une aristocrate nonagénaire répond : 
\bigskip
En effet, monsieur, mon pays est un grand pays, mais qui a connu depuis si longtemps de si petits 
régimes que, pour employer le langage de Mme du Deffand, après les trompeurs et les trompés, il 
fallait s'attendre (et nul n'en est moins surpris que moi) à voir vos trompettes (1). » 
\bigskip
(2) B. Nogaro, La valeur logique des théories économiques, p. 155. 
(1) M. Jouhandeau, Un monde, p. 17. 
\bigskip
P 204 : « Là où le rapport entre formes usuelles ne suffit pas, on pourra recourir aux métagrammes 
et autres mutations pour effectuer les rapprochements souhaités. 
\bigskip
L'analyse du rôle argumentatif de certaines variations d'expression ne peut se faire que grâce aux 
divergences  par  rapport  à  l'expression  qui  passe  inaperçue.  Prise  à  la  lettre,  cette  méthode 
laisserait  croire  que  se  servir  d'expressions  qui  passent  inaperçues  n'est  pas  un  procédé 
d'argumentation.  Il  n'en  est  évidemment  rien.  Mais  toute  étude  portant  non  sur  des  divergences 
mais sur ce qui passe inaperçu aura un caractère global. Elle ne pourra s'attacher à l'effet de telle 
modalité  d'expression  particulière  :  tout  au  plus  pourra-t-on  chercher  pourquoi  il  y  a  intérêt  à 
s'exprimer  de  manière  neutre,  non  comment  on  y  parvient.  Car  dès  que  la  manière  peut  être 
appréhendée,  c'est  qu'elle  présente  des  caractères  spéciaux,  définissables  autrement  que  par  la 
neutralité. » 
\bigskip
P  204-205 :  « Lorsque  nous  nous  demandons  pourquoi un  orateur  s'exprime  de  manière  neutre, 
nous sous-entendons qu'il pourrait ne pas le faire, et que, en le faisant, il a un but. C'est, sons l'un 
de ses nombreux aspects, le problème du procédé qui se pose. Nous le rencontrerons à chaque pas 
: l'absence de technique peut être une méthode, il n'est pas de naturel qui ne puisse être voulu. » 
\bigskip
P  205 :  « Bornons-nous  ici  à  quelques  remarques  sur  les  effets  de  la  sobriété  au  point  de  vue  de 
l'argumentation. Yves Gandon dans une étude du style de Gide, signale: 
\bigskip
D'aucuns vont jusqu'à dire qu'un style éclatant ne l'aurait pas également servi. Ce vocabulaire sans 
aspérité, cet énonce qui ne vise qu'à l'essentiel et éteint, pourrait-on croire, le propos de l'auteur 
sous  des  phrases  sans  relief,  des  locutions  vidées  de  tout  sens  agressif,  formeraient  l'instrument 
idéal pour l'écrivain en quête de climats fiévreux ou maudits (1). 
\bigskip
\bigskip
\bigskip
102 
\bigskip
 
Gandon réfute ce raisonnement « trop visiblement institué pour la seule commodité de M. Gide et 
que l'exemple d'un Mauriac... suffit à détruire ». Mais Yves Gandon se trompe lorsqu'il compare le 
climat fiévreux de Mauriac à celui de Gide. Mauriac est dans la tradition chrétienne, alors que Gide 
tente de promouvoir des normes nouvelles : il est fiévreux par ce qu'il approuve et non par ce qu'il 
décrit. Or, il semble bien que, à l'intérieur d'une orthodoxie, tous les procédés soient utilisables : 
par  contre,  lorsqu'on  essaye  de  promouvoir  des  jugements  de  valeur inusités,  choquants,  ceux-ci 
seraient plus facilement admis lorsque le style, lui, ne choque pas. Il n'est donc pas impossible que 
le  style  neutre  de  Gide  ait  pu  le  servir  réellement  dans  son  effort  de  persuasion.  Si  nous  avons 
mentionné ces remarques de Gandon, c'est parce qu'elles attirent immédiatement  l’attention   sur 
un  des  avantages  du  style  neutre  :  celui  de  suggérer  une  transposition  de  l'assentiment  général 
donné au langage, à l'assentiment aux normes exprimées. Il ne faut pas oublier, en effet, que parmi 
les éléments d'accord, le langage est l’un des premiers.  Quintilien déjà avait souligné, à la suite de 
Cicéron,  que  pour  l'orateur  «  le  défaut  peut-être  le  plus  grave  est  de  reculer  devant  le  langage 
ordinaire et devant les idées généralement reçues » (1). Le rapprochement entre langage ordinaire 
et  idées  reçues  n'est  pas  fortuit:  le  langage  ordinaire  est,  en  lui-même,  la  manifestation  d’un 
accord,  d’une communauté, au même titre que les idées reçues. Il peut servir à favoriser l'accord 
sur les idées. » 
\bigskip
(1) Yves Gandon, Le démon du style, p. 16. 
(1) Quintillien, Vol. III, liv. VIII, avant-propos, § 25. 
\bigskip
P  206 :  « Le  recours  au  style  neutre  peut  aussi  être  considéré  -comme  un  cas  particulier  de 
renoncement tendant à renforcer le crédit consenti aux prémisses (2). Le style neutre augmente la 
crédibilité  par  contraste  avec  ce  qu'eût  pu  être  un  style  argumentatif  plus  appuyé;  il  agit  par  la 
connaissance  que  nous  avons  par  ailleurs  de  la  force  argumentative  de  certaines  variations  de 
style. 
\bigskip
Nous  rencontrons  ici  un  phénomène  qu'on  ne  saurait  trop  souligner  :  c'est  que  la  connaissance 
généralisée - au moins intuitive - des techniques argumentatives, de leurs conditions d'application, 
de  leurs  effets,  est  à  la  base  de  beaucoup  de  mécanismes  argumentatifs  :  l'auditeur  n'est  pas 
considéré comme un ignorant mais au contraire comme quelqu'un d'averti. 
\bigskip
Prenons, pour illustrer ce rapport entre l'art et l'argumentation, l'ébauche et la version définitive 
de  certains  passages  de  Bossuet.  Nous  choisissons  à  dessein  un  passage  cité  dans  un  traité  de 
rhétorique (3). L'auteur du traité insiste sur les progrès du style : 
\bigskip
Première version : « Quand on assiste à des funérailles, ou bien que l'on entend parler de quelque 
mort  imprévue,  on  se  parle...  »  Deuxième  version  :  «  On  n'entend  dans  les  funérailles  que  des 
paroles d'étonnement, de ce que ce mortel est mort... (4) » 
\bigskip
(2) Cf. § 96 : La rhétorique comme procédé. 
(3) Saint-Aubin, Guide pour la classe de rhétorique, 1). 136. 
(4) Bossuet, Sermons, Vol. II : Sermon sur la mort, p. 449. 
\bigskip
P 206-207 : « Progrès dans le  nombre, l'harmonie,  la force, la densité. Et  par là même, le plaisir 
d'art  est  accru.  Mais  surtout,  Bossuet  a  retrouvé,  pour  l'incorporer  à  son  discours,  une 
argumentation d'Épictète : pourquoi s'étonner de ce qu'un vase fragile se brise, de ce qu'un mortel 
meure.  Par  l'usage  de  la  classification,  en  situant  le  mort  parmi  les  mortels,  s'introduit 
explicitement dans la deuxième version une argumentation qui n'était qu'implicite dans l'ébauche. 
Ajoutons  une  exagération,  un  renforcement  de  l'observation  :  non  seulement  on  se  parle  d'une 
mort imprévue mais « on n'entend que des paroles d'étonnement »; l'absurdité apparaîtra comme 
d'autant  plus  grave  que  sa  manifestation  est  plus  fréquente.  Nous  pouvons  donc  dire  que, 
\bigskip
\bigskip
\bigskip
103 
\bigskip
paraissant choisir une forme nouvelle pour sa pensée, Bossuet a en réalité transformé la portée des 
prémisses  de  son  argumentation  :  bien  plus,  il  a  déjà  incorporé  l'argumentation  à  l'expression 
même  de  ces  prémisses.  Ne  l'oublions  pas,  c'est  artificiellement  et  pour  les  commodités  de 
l'analyse que l'on sépare prémisses et argumentation : en réalité il y a déjà argumentation dans la 
position même des prémisses. Ceci apparaît surtout si l'on juxtapose certaines variations de forme 
dans l'expression de ces dernières : sans ces variations, l'argumentation passerait inaperçue. Mais 
le fait même que nous pouvons la mettre en évidence par l'étude de ces variations prouve bien que, 
même là où l'expression  paraît neutre et passe inaperçue, il y a  déjà souvent choix des termes et 
ébauche d'argumentation. » 
\bigskip
§ 39. LES MODALITES DANS L'EXPRESSION DE LA PENSEE 
\bigskip
p 207 : « La manière dont nous formulons notre pensée fait connaître certaines de ses modalités, 
qui modifient la réalité, la certitude ou l'importance des données du discours. » 
\bigskip
P  207-208 :  « On  est  à  Peu  près  d'accord  aujourd'hui  pour  reconnaître  que  les  modalités  de  la 
signification  sont  rendues,  de  préférence,  par  certaines  formes  grammaticales,  mais  que  celles-ci 
peuvent servir aussi à exprimer d'autres modalités. Cette indépendance relative est soulignée aussi 
bien  par  les  nouvelles  écoles  sémioticistes  (1)  que  par  les  tenants  des  vieilles  disciplines 
philologiques (1). La conscience de cette souplesse donne lieu à la recherche de catégories de sens, 
de « catégories affectives », qui ne correspondent pas aux catégories grammaticales et qui peuvent 
s'exprimer par différents moyens grammaticaux. Notons, toutefois, que parallèlement à cet effort 
pour  retrouver  les  modalités  de  pensée  sous  les  formes  grammaticales  variables,  on  assiste  à 
certains  efforts  inverses,  c'est-à-dire  qui  tendent  à  rattacher  au  choix  de  telle  ou  telle  catégorie 
d'expression  verbale  une  conduite  bien  définie.  Témoin  Fr.  Rostand,  qui  tente  une  exégèse 
psychanalytique des formes grammaticales et de leur acquisition par l'enfant (2). » 
\bigskip
(1) Cf. Morris, Signs, Language and Behavior, pp. 62 et suiv., pp. 82, 93, 103, note A, p. 257. 
(1) Cf. Brunot, La pensée et le langage. 
(2) Fr. Rostand, Grammaire et affectivité. 
\bigskip
P 208 : « Quant à nous, il nous semble important d'attirer l'attention sur le rôle argumentatif joué 
par  certaines  formes  d'expression,  relevant  des  modalités,  au  sens  large  de  ce  terme.  Nos 
remarques  ne  seront  axées  exclusivement  ni  sur  la  forme  grammaticale  ni  sur  des  catégories 
psychologiques ou logiques. » 
\bigskip
P 208-209 : « La même idée peut être formulée d'une façon affirmative ou négative. Attribuer une 
qualité  à  un  objet,  c'est  déjà  la  choisir  parmi  bien  d'autres,  parce  qu'on  la  considère  comme 
importante  ou  caractéristique.  Toute  description  s'établit  sur  un  arrière-fonds  dont  on  veut 
détacher l'objet, d'une façon qui ne devient significative qu'en fonction du but poursuivi. Mais cette 
référence  à  la  situation,  et  à  la  manière  dont  on  l'oriente,  peut  n'être  pas,  décelable  par  qui  ne 
reconnaît pas le lien entre la pensée et l'action. Dans le cas de la formulation négative, la référence 
à  autre  chose  est  tout  à  fait  explicite  :  la  négation  est  une  réaction  à  une  affirmation  réelle  ou 
virtuelle  d'autrui  (3).  Pour  Bergson,  la  pensée  qui  colle  sur  la  réalité  ne  pourrait  s'exprimer  que 
d'une façon affirmative : 
\bigskip
Rendez à la connaissance soli caractère exclusivement scientifique ou philosophique, supposez, en 
d'autres ternies, que la réalité vienne s'inscrire d'elle-même sur un esprit qui ne se soucie que des 
choses et ne s’intéresse pas aux personnes : on affirmera que telle ou telle chose est, on n'affirmera, 
jamais  qu'une  chose  n'est  pas...  Ce  qui  existe  peut  venir  s'enregistrer,  mais  l'inexistence  de 
l'inexistant ne s'enregistre pas (1). 
\bigskip
(3) Cf. Guillaume, Manuel de psychologie, p. 261. 
\bigskip
\bigskip
\bigskip
\bigskip
104 
\bigskip
P 209 : « La pensée négative, d'après lui, n'intervient que si l'on s'intéresse aux personnes, c'est-à-
dire si l'on argumente. 
\bigskip
Ce  n'est  que  dans  certains  cas  bien  définis,  quand  seules  deux  possibilités  sont  présentes  à  la 
conscience,  que  le  rejet  par  négation  de  l'une  d'entre  elles,  revient  à  choisir  l'autre,  que  l'on 
présente  souvent  ainsi  comme  le  moindre  mal.  Parfois  la  négation  ne  correspond  à  aucune 
affirmation  précise,  mais révèle un  ordre  de  préoccupations.  Demandons-nous,  après  Empson  et 
après  Britton,  ce  que  signifie  la  phrase  d'Othello  :  «  Pourtant  je  ne  verserai  pas  son  sang  »  (2). 
L'auditeur,  en  pareil  cas,  doit  deviner  s'il  s'agit  du  rejet  comme  genre  ou  comme  espèce,  c'est-à-
dire si cet acte rejeté doit être interprété en fonction d'un genre d'actes dont il serait une espèce, 
une manière de tuer parmi d'autres, ou une manière de se venger parmi d'autres, que l'on a en vue, 
ou si cet énoncé représente un genre, et quel genre, c'est-à-dire si aucun meurtre ne sera accompli, 
si  aucune  vengeance  ne  sera  exercée.  Suivant  que  l'on  adopte  l'une  ou  l'autre  interprétation,  la 
négation  pourrait  être  annonce  de  vengeance  ou  annonce  de  pardon.  Mais  l'intérêt  de  la  forme 
négative provient de ce que, quelle que soit l'interprétation, la mort est évoquée irrésistiblement. » 
\bigskip
(1) Bergson, L'évolution créatrice, pp. 315-316. 
(2) " Yet l'Il not shed her blood. 1 Shakespeare, Othello, acte V, se. II ; cf. W. Empson, Seven types 
of ambiguity, pp. 185-86 ; K. Britton, Communication, p. 12. 
\bigskip
P 209-210 : « La même ambiguïté se retrouve lorsque la négation s'applique non à une assertion, 
mais à une notion, par l'apposition d'un préfixe. C'est ainsi que dans «inhumain» la négation peut 
concerner le genre, et désigner ce qui est complètement étranger à l'homme, ou désigner l'espèce 
d'hommes ou de conduites humaines qui ne réalisent pas l'idéal humain. La formulation négative 
laisse, dans l'indétermination le concept au sein duquel s'opère le découpage. » 
\bigskip
P 210 : « Lorsque nous disposons d'un certain nombre de données, de vastes possibilités s'offrent à 
nous  quant  aux  liens  que  nous  établirons  entre  elles.  Le  problème  de  la  coordination  ou  de  la 
subordination des éléments relève souvent de la hiérarchie des valeurs admises ; toutefois, dans le 
cadre  de  ces  hiérarchies  de  valeurs,  nous  pouvons  formuler  des  liaisons  entre  les  éléments  du 
discours qui modifieront considérablement les prémisses : nous opérons entre ces liens possibles 
un choix tout aussi important que celui que nous opérons par la classification ou la qualification. 
\bigskip
Les grammairiens connaissent en français des conjonctions de coordination telles « et», « mais », 
« ou », « car », « donc », «ni », et des conjonctions de subordination telles, « bien que », « malgré 
que  »,  «  puisque  ».  Mais  si  nous  examinons  la  nature  des  liaisons  ainsi  signifiées  nous  devons 
reconnaître que la subordination des propositions les unes aux autres est de règle, quelle que soit 
la conjonction utilisée. 
\bigskip
En  effet,  les  conjonctions  de  coordination  telles,  «  et  »,  «  ou  »,  «  ni  »,  «  donc  »,  peuvent  être 
envisagées comme exprimant une relation logique. Mais ce ne sera que dans certaines conditions 
bien définies que cette relation logique laissera les propositions ainsi reliées sur un pied d'égalité. 
D'une  manière  générale  on  constate  que,  dans  la  pratique  du  discours,  presque  toujours,  sous  la 
forme d'une coordination, s'insère une intention de subordination. » 
\bigskip
P 210-211 : « Prenons le cas très simple d'une succession d'événements : 
\bigskip
J’ai rencontré ton ami hier; il ne m'a  pas parlé  de toi ». La première proposition est un fait, que 
mon  interlocuteur  ne  conteste  pas,  la  deuxième  également.  Elles  sont  coordonnées  et  pourraient 
s'unir par la conjonction « et ». Mais l'interprétation normale dans certaines situations sera : « ton 
ami  ne  m'a  pas  parlé  de  toi,  bien  qu'il  en  ait  eu  l'occasion.  «  L'insertion  de  cette  première 
proposition, à cette place, 'et précédant la deuxième, à laquelle elle est effectivement subordonnée, 
modifie  donc  considérablement  l'impression  que  produirait  l'affirmation  de  ces  deux  faits 
\bigskip
\bigskip
\bigskip
105 
\bigskip
simplement  coordonnés.  Les  jugements  de  fait  se  colorent,  par  là,  d'une  interprétation  implicite, 
qui leur donne toute leur signification. » 
\bigskip
P 211 : « La subordination n'est pas exprimée uniquement par des conjonctions ; d'autres formes 
grammaticales peuvent jouer le même rôle. Fr. Rostand établit la parenté entre expressions telles 
que : « belle, car modeste », « belle, parce que modeste », « belle par la modestie », «embellie par 
la modestie», « d'une beauté créée par la modestie » (1). La dépendance entre  beauté et modestie 
est exprimée, bien que d'une façon légèrement différente, par chacune de ces formules. 
\bigskip
La  qualification  elle-même  se  prête  souvent  à  des  jeux  de  subordination.  Selon  la  subordination 
que nous établissons, nous parlerons de « pieuse douleur », ou de « piété douloureuse ». 
\bigskip
(1) Fr. Rostand, Grammaire CI affectivité, p. 66. 
\bigskip
P  211-212 :  « Les  différentes  techniques  de  présentation  permettent  de  porter  l'attention,  entre 
éléments divers, sur ceux qui ont du poids : les formules « pour l'amour de », « en considération 
de », « à cause de », soulignent à quels termes on donne le primat : « Tout tourne en bien pour les 
élus, jusqu'aux obscurités de l'Ecriture ; car ils les honorent, à cause des clartés divines... (2) » Le 
même primat s'exprime ailleurs par la proposition relative : « la clarté qui mérite qu'on révère les 
obscurités  »  (3).  La  minimisation  de  certains  éléments  est  souvent  signifiée  par  la  préposition  « 
sinon  »,  ou  par  l'expression  «à  l'exception  de  ».  Voici  comment  se  reflète  curieusement  la 
bienveillante indulgence de Julien l'Apostat envers les juifs : 
\bigskip
ils s'accordent avec les gentils, à l'exception de leur croyance en un seul Dieu. Cela leur est spécial 
et nous est étranger. Tout le reste nous est commun... (1). » 
\bigskip
(2) Pascal, Bibl. de la Pléiade, Pensées, ,80 (137), 1). 1016 (575 ed. Brunschvieg).  
(3) ID., Pensées, 400 (465), p. 933 (598 éd. Brunschvicg). 
(1) C. Gal. 306 B. Cité par J. Bidez, La vie de l'empereur Julien, p. 305. 
\bigskip
P 212 : « Enfin, des expressions comme « bien que », « malgré », « sans  doute », marquent que 
l'on fait certaines concessions, mais indiquent surtout, notamment selon leur place dans la phrase, 
quel est le degré d'importance que l'on attache à ce que l'on concède. 
\bigskip
A l'aide de  ces techniques, l'orateur est à même de guider l'auditeur d'une manière extrêmement 
efficace vers ce qu'il veut lui faire admettre ; aussi Auerbach a-t-il souligné avec raison le caractère 
stratégique (2) de la construction qui établit des relations précises entre les éléments du discours 
et qui a été qualifiée de « hypotactique ». A cette dernière, on oppose la construction paratactique, 
qui renonce à toute liaison précise entre les parties. L'exemple typique que présente Auerbach est 
la  phrase  latine  de  la  Vulgate  :  Dixitque  Deus  :  fiat  lux,  et  facta  est  lux  (3).  L'auditeur  est  libre 
d'imaginer  entre  les  événements  une  relation  qui,  de  son  imprécision  même,  prend  un  caractère 
mystérieux, magique : par là, d'ailleurs, elle peut parfois produire un effet hautement dramatique. 
C'est à la construction paratactique qu'il faut, pensons-nous, rattacher, dans certains de ses usages 
tout au moins, l'énumération. Celle-ci attire, à juste titre, l'attention d'E. Noulet dans le sonnet des 
voyelles  de  Rimbaud.  Expression  extrême  du  mouvement  (4)  ?  Peut-être.  Mais  aussi  manière 
d'exprimer  le  triomphant  mystère  de  rapports  que  le  poète  sait  exister  sans  qu'il  en  connaisse 
l'exacte teneur. » 
\bigskip
(2) Auerbach, Mimesis, p. 92. 
(3) Ibid., p. 74. 
(4) E. Noulet, Le premier visage de Rimbaud, p. 183. 
\bigskip
\bigskip
\bigskip
\bigskip
106 
\bigskip
P 213 : « La construction hypotactique est la construction argumentative par excellence : elle serait 
selon  Auerbach  -  par  opposition  à  la  construction  paratactique  propre  à  la  culture  hébraïque 
caractéristique  des  écrits  gréco-romains.  L'hypotaxe  crée  des  cadres,  constitue  une  prise  de 
position.  Elle  commande  au  lecteur,  l'oblige  à  voir  certaines  relations,  limite  les  interprétations 
qu'il  pourrait  prendre  en  considération,  s'inspire  du  raisonnement  juridique  bien  construit.  La 
parataxe  laisse  plus  de  liberté,  ne  semble  vouloir  imposer  aucun  point  de  vue  ;  aussi  est-ce  sans 
doute  parce  qu'elle  est  paratactique  que  la  phrase  composée,  balancée,  des  écrivains  anglais  du 
XVIIIe siècle donne, comme le dit R. M. Weaver une impression philosophique (1), disons plutôt 
descriptive, contemplative, impartiale. 
\bigskip
Les modalités, au sens technique du linguiste sont, admet-on généralement, au nombre de quatre : 
l'assertive, l'injonctive, l'interrogative, et l'optative. 
\bigskip
La modalité assertive convient à toute argumentation : il n'y a pas lieu d'en parler. 
\bigskip
La modalité injonctive s'exprime, dans nos langues, par l'impératif. 
\bigskip
Contrairement aux apparences, elle n'a pas de force persuasive tout son pouvoir vient de l'emprise 
de la personne qui ordonne sur celle qui exécute : c'est un rapport de forces n'impliquant aucune 
adhésion.  Quand  la  force  réelle  est  absente  ou  que  l'on  n'envisage  pas  son  utilisation,  l'impératif 
prend l'accent d'une prière. » 
\bigskip
(1) Richard 31. Weaver, The Ethics of Rhetoric, p. 125. 
\bigskip
P  213-214 :  « A  cause  de  ce  rapport  personnel  impliqué  par  la  forme  impérative,  celle-ci  est  très 
efficace  pour  augmenter  le  sentiment  de  présence.  Le  reporter  d'une  compétition  sportive,  à  la 
radio,  prescrit  parfois  aux  joueurs  de  faire  ceci  ou  cela.  Ces  impératifs  ne sont  pas  entendus  des 
joueurs,  ils  ne  concernent  pas  les  auditeurs,  mais  tout  en  leur  communiquant  indirectement  des 
jugements au sujet des joueurs - par exemple, admiration pour leur courage, désapprobation pour 
leur hésitation - ils donnent à la scène un haut degré de présence dû à ce que celui qui parle semble 
participer à l'action qu'il décrit. » 
\bigskip
P  214 :  « L'interrogatif  est  un  mode  dont  l'importance  rhétorique  est  considérable.  La  question 
suppose un objet, sur lequel elle porte, et suggère qu'il y a un accord sur l'existence de cet objet. 
Répondre  à  une  question,  c'est  confirmer  cet  accord  implicite  :  les  dialogues  socratiques  nous 
apprennent beaucoup sur l'utilité et les dangers de cette technique dialectique. 
\bigskip
Le  rôle  de  l'interrogation  dans  la  procédure  judiciaire  est  un  des  points  sur  lesquels  les  Anciens, 
notamment  Quintilien,  ont  énoncé  maintes  remarques  pratiques  qui  sont  toujours  d'actualité. 
L'usage  de  l'interrogation  vise  parfois  une  confession  sur  un  fait  réel  inconnu  de  celui  qui 
questionne, mais dont on présume l'existence ainsi que celle de ses conditions. « Qu'avez-vous fait 
ce jour-là à tel endroit ? » implique déjà que l'interpellé se trouvait à un certain moment à l'endroit 
indiqué  ;  s'il  répond,  il  marque  son  accord  à  ce  sujet.  Mais  très  souvent  l'interrogation,  tout  en 
étant  réelle,  ne  vise  pas  tant  à  éclairer  celui  qui  interroge  qu'à  acculer  l'adversaire  à  des 
incompatibilités.  Les  questions  ne  sont  souvent  qu'une  forme  habile  pour  amorcer  des 
raisonnements,  notamment  en  usant  de  l'alternative,  ou  de  la  division,  avec  la  complicité,  pour 
ainsi dire, de l'interlocuteur qui s'engage par ses réponses a adopter ce mode d'argumentation. » 
\bigskip
P 214-215 : « Les présupposés implicites dans certaines questions, font que la forme interrogative 
peut  être  considérée  comme  un  procédé  assez  hypocrite  pour  exprimer  certaines  croyances,.  En 
disant « qu'est-ce qui a bien pu conduire les Allemands à entamer dernièrement tant de guerres ? » 
on suggère souvent que les réponses  qui viendront spontanément à  l'esprit devront être rejetées. 
La question porte moins sur la recherche d'un motif que sur la recherche de la raison pour laquelle 
\bigskip
\bigskip
\bigskip
107 
\bigskip
on n'en trouvera pas; elle est surtout affirmation qu'il n'y a pas de motif suffisamment explicatif. 
C'est  pourquoi  Crawshav-Williams  croit  voir  en  pareilles  questions  le  signal  avertisseur  d'une 
tournure d'esprit irrationnelle (1). » 
\bigskip
P 215 : « L'interrogation serait, selon Wittgenstein, dans une phrase telle : 
\bigskip
Le temps n'est-il pas splendide aujourd'hui un simple jugement  (2). Bien plus, selon Sartre, dans 
ces vers de Rimbaud : 
\bigskip
0 saisons, o châteaux, 
Quelle âme est sans défauts (3) ? 
\bigskip
l'interrogation serait devenue « chose », « substance ». En fait la forme interrogative n'est pas sans 
introduire un appel à la communion avec un auditoire, fût-il le sujet lui-même (4). 
\bigskip
Ajoutons enfin qu'une question peut servir à en rejeter une autre, comme dans ce rêve où A. Gide, 
embarrassé pour répondre à la question : « Qu'est-ce que vous pensez de la Russie ? » adopte en 
réponse  la  formule  efficace  «  Pouvez-vous  le  demander  ?  »  signifiant  ainsi  que  l'accord  avec 
l'interlocuteur est hors de doute (5). 
\bigskip
La modalité optative est peut-être celle qui se prête le mieux à l'expression des normes. L'action du 
souhait, par exemple « puisse-t-il réussir », est du même ordre que celle du discours épidictique ; 
le  souhait  exprime  une  approbation,  et  indirectement  une  norme;  par  là,  il  se  rapproche  de 
l'impératif exprimant une prière, une supplication. » 
\bigskip
(1) Crawshay-Williams, The camions of unreason, p. 176. 
(2) Wittgenstein, -Philosophische Untersuchungen, p. 10. 
(3) Rimbaud, Bibl. de la Pléiade, Poésies LXXXIII, p. 139. 
(4) J.-P. Sartre, Situations, 11, pp. 68-69. 
(5) A. Gide, Journal, 1939-1942, p. 132. 
\bigskip
P  216 :  «   L'emploi  des  temps  permet,  lui  aussi,  d'agir  sur  l'auditoire.  Chaque  groupe  de  langues 
offre, à cet égard, des possibilités qui mériteraient une étude minutieuse. 
\bigskip
En ce qui concerne le français, on peut dire que le passé est l'irréfragable, le fait ; l'imparfait est le 
temps du transitoire ; le présent exprime l'universel, la loi, le normal. Le présent est le temps de la 
maxime, de la sentence, c'est-à-dire de ce qui est considéré comme toujours actuel, jamais périmé - 
c'est le présent qui semble par là avoir le rôle le plus équivoque ; c'est lui qui exprime le mieux le 
normal dans son passage vers la norme. Dans une phrase comme « la femme aime à parler », on 
insiste sur le normal au point d'en faire un caractère général : on ne peut à première vue distinguer 
pareil énoncé de l'affirmation ( l'homme est sujet à la mort ». Si nous remplacions le présent par « 
on a constaté que la femme aimait à parler », la confusion avec le sens distributif serait beaucoup 
moins forte. On échapperait à la loi pour rester dans l'observation. 
\bigskip
Le  présent  a  cette  autre  propriété  de  donner  le  plus  facilement  ce  que  nous  avons  appelé  «  le 
sentiment de présence ». Les rhétoriciens lui ont souvent reconnu ce rôle (1). Et c'est là peut-être la 
raison de son emploi par les romanciers contemporains. Nelly Cormeau cite le passage brusque au 
présent chez Mauriac : 
\bigskip
Après  un  débat  intérieur,  elle quitta  sa  couche,  glissa  dans  des  savates  ses pieds  enflés,  et,  vêtue 
d'une robe de chambre marron, une bougie au poing, sortit de la chambre. Elle descend l'escalier, 
suit un corridor, traverse la steppe du vestibule (2). » 
\bigskip
\bigskip
\bigskip
\bigskip
108 
\bigskip
(1) Cf. Longin, Trailé du sublime, chap. XXI, p. 112. 
 (2) N. Cormeau, L'art de François Mauriac, pp. 348-349 (Génitrix, p. 42). 
\bigskip
P 216-217 : « On trouvera dans l'étude d'Yves Gandon sur le style, des remarques intéressantes au 
sujet  de  l'emploi  des  temps  chez  les  écrivains.  Le  passé  simple  a  fait  place  chez  Flaubert  et  les 
naturalistes à l'imparfait. Le présent narratif est découvert ou, du moins, mis à l'honneur par les 
contemporains  :  jules  Romains,  par  exemple,  l'affectionne.  Comme  le  remarque  V.  Gandon  « 
l'illusion de lavie se trouve obtenue à moindres frais » (1). » 
\bigskip
(1) Y. GANDON, Le démon du style, p. xv, p. 86. 
\bigskip
P 217 : « Bien entendu, il n'est pas certain que dans toutes les langues ce soit la forme du présent 
qui exprime le mieux ce sentiment. On sait que dans les langues slaves 1  a forme grammaticale du 
présent  des  verbes  d'action  parfaite,  exprime  en  réalité  un  futur  et  ne  s'utilise  que  dans  cette 
acception. L'influence de la forme verbale sur la manière d'exprimer le passage du normal au nor-
matif, sur le sentiment de présence, ne peut être étudiée que pour un système linguistique donné 
et à un moment donné. Il suffit de mentionner les ressources qu'offre l'emploi de ces formes lors-
qu'elles sont utilisées, avec une intention argumentative, dans le cadre des conventions existantes. 
\bigskip
è
\bigskip
Notons  que  l'indéfini  «  on  »  est  utilisé  souvent  pour  introduire  une  norme.  «  On  fait  ceci  » 
équivaut,  à  peu  près  à  «  il  faut  faire  ceci  »  ;  parfois  le  «  on  »  peut  désigner  simplement  ce  qui 
concerne  certains  êtres  dans  des  situations  déterminées  :  «  On  ne  distingue  pas  clairement  la 
forme  de  cet  arbre  ».  Comme  le  passage  du  normal  au  normatif  est  un  lieu,  cette  tournure  peut 
avoir un net intérêt argumentatif. » 
\bigskip
P  217-218 :  « Le  remplacement  du  «  je  »  par  le  «  on  »  est,  suivant  le  chevalier  de  Méré,  tantôt 
agréable, tantôt malséant : 
\bigskip
... je voy qu'une Dame dira plutost, «On ne vous hait pas », « on vous aime », qu'elle ne dira, «je ne 
vous hai pas », on «je vous aime »; et il ajoute, « et parce que cette expression vient de modestie, 
elle ne peut avoir que fort bonne grace. Mais si c'est une fausse finesse, comme on prétend, on n'en 
demeure Pas d'accord,  elle est bien des-agréable ; et je connois des  personnes qui ne la peuvent 
souffrir (1). » 
\bigskip
(1) Chevalier de Méré, Œuvres, 11, 1). 34 (Des agrémens). 
\bigskip
P 218 : « Dans les deux cas on transforme le subjectif en normal ; on diminue en quelque sorte sa 
responsabilité dans le jugement; néanmoins le « on » est compris comme « je » et non pas comme 
une simple expression du normal. Si le premier usage est agréable, et le second pas, il s'agit moins 
de  modestie  et  de  fausse  finesse,  que  d'effets  divergents  produits  par  la  généralisation  d'une 
appréciation flatteuse ou d'un désaccord personnel auquel on donne une importance accrue. 
\bigskip
L'emploi de la troisième personne, même définie, à la place de la première, peut avoir pour effet, 
comme l'usage du « on », de diminuer la responsabilité du sujet, de créer une distance entre celui 
qui parle et ce qu'il dit. 
\bigskip
Citons, a ce propos, les jolies remarques de Jouhandeau : 
\bigskip
Au moment où l'enivrait la fierté, l'admiration qu'elle éprouvait devant elle-même, la pauvre vieille 
cessait de dire « je » ; par déférence peut-être elle parlait d'elle à la troisième personne, se traitant 
\bigskip
\bigskip
\bigskip
109 
\bigskip
de « Madame Robillard », gros comme le bras. Ou bien était-ce par modestie pour se separer de sa 
gloire on pour être plus véridique, en se faisant tout d'un coup objective (2). 
\bigskip
Même  si  la  substitution  n'est  qu'un  retour  à  une  conduite  enfantine  -  on  sait  que  le  nom  propre 
précède  chez  la  plupart  des  enfants  le  pronom  personnel  de  la  première  personne  (3)  -  l'un  des 
effets majeurs sur l'auditoire semble être l'objectivation de l'énoncé. » 
\bigskip
(2) Jouhandeau, Un monde, p. 80. 
(3) E. Pichon, Le développement psychique de l'enfant, p. 96. 
\bigskip
P 218-219 : « Un emploi argumentatif particulier du pronom ou de l'adjectif indéfini  est dû à son 
ambiguité.  Les  assertions  «certaines personnes  en  savent  trop  long  »,  «  on  ne  s'amuse  pas  ici  », 
peuvent  comprendre,  on  non,  l'orateur  on  l'interlocuteur.  Les  formules  sont  volontaire  ment 
ambiguës:  un  énoncé  trop  précis,  ne  permettant  aucun  doute  sur  son  interprétation,  est  parfois 
soigneusement évité, pour les raisons les plus variées. » 
\bigskip
P  219 :  « Notons,  par  contre,  que  la  substitution,  au  pronom  indéfini  «  on  »,  d'expressions, 
désignant  une  on  plusieurs  personnes  déterminées,  peut  avoir  un  effet  de présence  très  marqué. 
C'est  pourquoi,  «le  conteur  de  fables.'---  allègue  pour  témoins  de  ce  qu'il  avance  des  hommes 
obscurs qu'on ne peut trouver pour les convaincre de fausseté » (1). 
\bigskip
L'usage  de  l'article  défini,  du  singulier  pour  le  pluriel,  du  démonstratif,  aura  souvent  des  effets 
argumentatifs dignes de remarque. 
\bigskip
En chimie, on expérimente sur des corps particuliers et l'on en tire des affirmations concernant « 
le chlore », « le phosphore » ; l'article défini permet de traiter les échantillons comme représentant 
une  espèce.  De  même  l'usage  du  singulier  pour  le  pluriel,  «  le  juif  »,  «  le  Russe  »,  a  une 
signification  indéniable  (2).  Nous  croyons  y  reconnaître,  à  la  fois  une  action  de  présence,  par  la 
transformation  du  groupe  en  une  personne,  et  l'unification  du  point  de  vue,  l'impossibilité  de 
distinguer entre les bons et les mauvais qu'entraîne cette transformation. 
\bigskip
L’emploi inusité du démonstratif permet de créer un effet de présence très vif ; François Mauriac 
l'affectionne. Citons un exemple, entre mille : 
\bigskip
Ses  yeux  fixèrent  le  grand  lit  à  colonnes  torses  où,  huit  ans  plus  tôt,  son  frère  aîné,  Michel 
Frontenac, avait souffert cette interminable agonie (3). » 
\bigskip
(1) La. Bruyère, Les caractères de Théophraste, Du débit des nouvelles, p. 51. 
(2) Cf. V. Klemperer, L. 1'. L, Notizbuch eines Philologen, p. 186. 
(3) Le mystère Frontenac, p. 11, cf. Y. Gandon, Le démon du style, p. 65. 
\bigskip
P  219-220 :  « Cet  artifice  d'exposition,  V.  Gandon  le  qualifie  de  «  inacceptable  au  regard  de  la 
logique pure » tout en reconnaissant  qu'il est « excellent quant à la technique romanesque ». Le 
démonstratif  porte  ici  sur  quelque  chose  qui  n'est  connu  que  d'un  des  personnages,  décrit  lui-
même  du  dehors,  et  nous  sommes  à  la  première  page  du  roman.  Mais  l'effet  de  présence  est 
indéniable. » 
\bigskip
P  220 :  « Toutes  ces  formes  de  présentation  exercent  une  influence  sur  ce  que  les  logiciens 
considèrent comme les modalités : certitude, possibilité, nécessité, d'une affirillation. Bien entendu 
les  adverbes  sont-ils  normalement  aptes  à  cet  usage,  mais  on  voit  par  les  quelques  notes  qui 
précèdent  que  ce  serait  faire  bon  marché  de  la  réalité  argumentative  que  de  les  croire  seuls 
capables d'exprimer ces modalités. 
\bigskip
\bigskip
\bigskip
\bigskip
110 
\bigskip
Ce  que  l'on  vise  dans  l'argumentation  c'est  moins  la  précision  de  certaines  modalités  logiques 
attribuées aux affirmations, que les moyens d'obtenir l'adhésion de l'auditoire grâce aux variations 
dans l'expression de la pensée. » 
\bigskip
§ 40. FORME DU DISCOURS ET COMMUNION AVEC L'AUDITOIRE 
\bigskip
« La forme sous laquelle sont présentées les données n'est pas seulement destinée à produire des 
effets argumentatifs relatifs à l'objet du discours ; elle peut aussi offrir un ensemble de caractères 
relatifs à la communion avec l'auditoire. 
\bigskip
Tout  système  linguistique  implique  des  règles  formelles  de  structure  qui  lient  les  usagers  de  ce 
système,  mais  l'utilisation  de  celui-ci  s'accommode  de  divers  styles,  de  formules  particulières, 
caractéristiques d'un milieu, de la place qu'on y occupe, d'une certaine atmosphère culturelle. 
\bigskip
P 220-221 : « On connaît le rôle que jouent les vocabulaires dans la différenciation des milieux. On 
sait qu'il existe dans certaines sociétés des langues réservées aux nobles ou aux dieux (1) ; l'usage 
des termes archaïques, des patois, a une signification le plus souvent particularisante, tantôt dans 
le  sens  d'une  opposition  de  classes,  tantôt  dans  celui  d'une  opposition  d'autre  nature.  La 
signification de ces divergences tient à ce que, langue réservée ou patois coexistent avec le langage 
d'un groupe plus large, dont leurs usagers font également partie. Les langages réservés jouent donc 
un  rôle  de  ségrégation  tout  différent  de  celui  que  jouent  les  langues  de  peuples  étrangers  l'un  à 
l'autre.  Il  peut  se  faire  que  ce  langage  réservé  soit  le  langage  usuel  d'un  groupe  extérieur  plus 
étendu : c'est le cas pour les sociétés qui ont une langue de culture appartenant à un autre groupe, 
tel le latin pour les gallo-romains, et, au XIXe siècle, le français pour les habitants des Flandres. » 
\bigskip
1) Cf. W.Porzig, Das Wunder der Sprache, pp. 187-188. 
\bigskip
P  221 :  « On  sait  aussi  le  rôle  que  peut  jouer  une  expression  négligente  la  mutilation  d'un  nom 
propre  ou  la  déformation  d'un  texte  témoignent  généralement  d'un  certain  mépris  pour  l'objet 
dont  on  parle.  Ces  négligences  peuvent  créer  une  connivence  avec  l'auditeur,  souligner  une 
hiérarchie. Prenons un exemple très banal : un médecin chargé de lire un rapport financier semble 
hésiter dans sa lecture entre « milliers » ou « millions » de francs : c'est l'indication d'un mépris 
pour ces questions matérielles, communion avec les membres de l'auditoire qui partagent ce même 
mépris.  L'usage  d'un  vocabulaire  volontairement  pauvre  ou  maladroit  peut  servir  aux  mêmes 
fins. » 
\bigskip
P  221-222 :  « Il  y  a  plus.  On  commence  à  reconnaître  que,  à  chaque  structure  sociale, 
correspondraient  des  modes  particuliers  d'exprimer  la  communion  sociale.  Lasswell,  dans  ses 
travaux  sur  la  propagande,  a  insisté  sur  ce  problème.  jusqu'à,  présent,  il  semble  que  l'on  ait 
distingué deux grands styles dans la transmission de la pensée : celui des sociétés démocratiques et 
celui  des  sociétés  hiérarchiques.  Les  études  sont  encore  embryonnaires.  Il  est  intéressant  de 
relever cependant, comme l'a fait Lasswell, le caractère quasi rituel du style de certaines sociétés 
hiérarchiques.  On  a  pu  souligner  que  le  style  des  proclamations  du  roi  d'Angleterre  comme 
empereur des Indes était beaucoup plus rituel que le style des proclamations du même souverain 
comme roi d'Angleterre. Des hypothèses importantes ont déjà été émises : il semble que certaines 
structures linguistiques conviennent mieux à une société basée sur l'égalité, l'initiative individuelle 
; d'autres conviendraient mieux aux sociétés basées sur une structure hiérarchique. » 
\bigskip
P  222 :  « Dans  son  étude  intéressante  sur  l'allemand  des  nazis  (1),  Heinz  Paechter  s'efforce  de 
repérer  de  telles  structures.  La  grammaire  des  sociétés  égalitaires  mettrait  l'accent  sur  les 
prédicats,  les  évaluations  par  le  sujet.  Le  langage  des  sociétés  hiérarchiques  serait  évocateur,  sa 
grammaire et sa syntaxe seraient magiques : 
\bigskip
\bigskip
\bigskip
\bigskip
111 
\bigskip
Les symboles verbaux ne seront plus représentatifs des choses, mais tendront à devenir des choses 
par eux-mêmes, avec une place bien définie dans la hiérarchie des valeurs, et une participation au 
rituel sur leur plan propre (2). » 
\bigskip
Le langage qui, dans une société égalitaire, est à tout le inonde, et évolue quasi librement, se fige 
dans une société hiérarchique. Les expressions, les formules y deviennent rituelles, elles s'écoutent 
dans un esprit de communion et de soumission totale. 
\bigskip
Mais il suffit que les formules ne soient plus obligatoires, qu'elles ne s'écoutent plus dans le même 
esprit  de  communion,  pour  qu'elles  prennent  l'allure  d'un  cliché.  L'imitation  du  style  biblique, 
caractéristique de certains sermons, comme les tentatives plus ou moins réussies de reproduire un 
beau  vers  de  Racine  -  les  morceaux  connus  de  la  littérature  classique  ont  quelque  chose  de  la 
formule rituelle - semblent des clichés, justement à cause de leur prétention à l'originalité. » 
\bigskip
(1) Heinz Paechter, Nazi-Deutsch, cite d'après Lasswell, Language of poli ties, p. 385, note. 
(2) H. Paechter, Nazi-Deutsch, p. 6. 
\bigskip
P  222-223 :  « Si  les  clichés  sont  pourchassés,  depuis  le  romantisme,  dans  notre  culture  férue 
d'originalité  - et jean Paulhan a bien mis en évidence ce règne du  terrorisme dans la littérature  - 
c'est  que  la  formule  clichée  n'a  de  valeur  que  comme  moyen  facile,  trop  facile  parfois,  de 
communion avec les auditeurs. Cette formule résulte d'un accord sur une manière d'exprimer un 
fait, une valeur, une liaison de phénomènes ou un rapport entre personnes. Il y a des clichés de la 
poésie, des clichés de la politique. Ces formes servent au bien aller de l'interprétation : nous savons 
que l'introduction du mot « coursier » marque une intention poétique et que « votre noble patrie » 
est  une  formule  consacrée  à  l'usage  des  orateurs  de  banquets.  Les  termes  « droit»,  «liberté», 
«démocratie », permettent la communion comme le déploiement d'un drapeau. » 
\bigskip
P 223 : « Pour que ces formules, ces mots, soient perçus comme des clichés, il faut une distance, il 
faut  que  l'auditeur  ne  s'identifie  plus,  à  tous  points  de  vue,  avec  ceux  qui  les  utilisent  et  les 
acceptent.  Ce  recul  est  favorisé,  semble-t-il,  par  deux  ordres  de  considérations  qui  peuvent 
d'ailleurs se renforcer mutuellement. En effet, le cliché est, à la fois, fond et forme. C'est un objet 
d'accord  qui  s'exprime  régulièrement  d'une  certaine  manière,  une  formule  stéréotypée  qui  se 
répète.  Il  suffit  donc  pour  qu'une  expression  soit  perçue  comme  cliché  que  l'on  se  rende  compte 
qu'il y aurait moyen de dire aussi bien, voire mieux, la même chose autrement. L'auditeur qui fait 
cette  constatation  a  opéré  une  dissociation  entre  fond  et  forme  et  a  pris  du  recul  au  niveau  du 
langage. Mais il suffit aussi que l'on rejette les valeurs que le cliché exprime. Dans ce cas l'auditeur 
prendra du recul au niveau de la pensée. Dans les deux cas, l'auditeur perçoit une inadéquation ; 
elle lui rend sensible qu'il s'agit là de quelque chose de tout fait, de non parfaitement adapté à la 
situation. 
\bigskip
Si  la  formule  clichée,  admise,  favorise  le  bien  aller  de  la  discussion,  par  la  communion  qu'elle 
permet  d'établir,  refusée,  elle  peut  servir  à  disqualifier  certains  raisonnements,  à  discréditer 
certains orateurs. » 
\bigskip
P  223-224 :  « C'est  lorsqu'il  s'agit  d'exprimer  plus  ou  moins  explicitement  une  norme  que 
l'importance d'une formulation consacrée s'avère non négligeable. Les maximes ne condensent pas 
seulement la sagesse des nations - elles sont aussi un des moyens les plus efficaces de promouvoir 
cette  sagesse  et  de  la  faire  évoluer  :  l'usage  des  maximes  nous  fait  toucher  du  doigt  le  rôle  des 
valeurs  admises  et  les  procédés  de  leur transfert.  Sans  doute une  maxime peut-elle  toujours  être 
repoussée,  l'accord  qu'elle  invoque  n'est  jamais  obligatoire,  mais  sa  force  est  si  grande,  elle 
bénéficie d'une telle présomption d'accord, qu'il f aut des raisons sérieuses pour la rejeter. » 
\bigskip
\bigskip
\bigskip
\bigskip
112 
\bigskip
P  224 :  « La  maxime, ??? telle  que  la  décrit  Aristote  (1),  est  bien  ce  que  nous 
qualifierions  aujourd'hui  de  jugement  de  valeur.  Elle  confère,  dit-il,  au  discours,  un  caractère 
éthique  (2).  Sa  signification  tient  à  son  élaboration  sociale.  On  l'énonce,  pour  suggérer  son 
applicabilité  à  une  situation  particulière.  Plus  sa  forme  est  traditionnellement  reconnue,  plus 
l'énoncé, avec les conséquences qu'il entraîne, sera aisément admis. 
\bigskip
Les  proverbes,  disent  nos  dictionnaires,  sont  de  courtes  maximes,  devenues  populaires. 
Schopenhauer  les  rapproche  des  lieux  :  ce  sont,  dit-il,  des  lieux  à  tendance  pratique  (3).  Nous 
aimerions attirer J'attention sur un caractère de cette sorte de maxime qui nous paraît essentiel: le 
proverbe exprime un événement particulier et suggère une norme  ; de là, sans doute, sa diffusion 
facile,  son  aspect  populaire,  qui  l'oppose  à  l'aspect  livresque,  savant,  de  certaines  maximes. 
Ajoutons que, comme le souligne Estève (4), le génie impératif des proverbes tient certainement, 
en partie du moins, à leur rythme. » 
\bigskip
(1) Aristote. Rhétorique, liv. II, chap. 21, 1394 a et suiv. 
(2) Ibid., 1395 b, 11. 
(3) Schopenhauer, éd. l'iper, vol. 6 : Eristische Dialektik, p. 401, note. 
(4) Cl.-L. Estève, Etudes philosophiques sur l'expression littéraire, p. 217. 
\bigskip
P 224-225 : « Parce qu'il est perçu comme illustrant une norme, le proverbe pourra servir de point 
de  départ  aux  raisonnements,  à  condition  bien  entendu  que  cette  norme  soit  admise  par 
l'auditoire. Mais il ne faudrait pas en conclure que les proverbes ne servent que si leur énoncé est 
devenu pratiquement inutile. Les chapelets de proverbes que débite Sancho Pança sont autant de 
rappels  à  l'ordre  pour  qui  oublierait  certaines  des  valeurs  qu'il  convient  de  ne  pas  totalement 
négliger. » 
\bigskip
P  225 :  « Bien  que  marquant  un  accord  traditionnel,  les  proverbes  naissent  aux  aussi  :  mais  ils 
empruntent aussitôt leur statut, en tant que proverbes, aux proverbes existants, soit par imitation 
purement  formelle,  soit  parce  que  le  proverbe  nouveau  n'est  qu'une  illustration  nouvelle  de  la 
même norme qu'illustrait déjà un proverbe antérieur. jean Paulhan nous a d'ailleurs fait connaître 
ces  concours  poétiques  des  malgaches  où  la  dispute  se fait  à  coups  de  proverbes  et  de  ce  qui  est 
appelé des «images de proverbe » (1). Il s'agit de phrases stylisées qui expriment une norme, mais 
où seul l'initié peut distinguer le proverbe de ce qui n'est qu'image de proverbe. 
\bigskip
Les  slogans,  les  mots  d'ordre,  constituent  des  maximes  élaborées  pour  les  besoins  d'une  action 
particulière (2). Ils doivent s'imposer par leur rythme, leur forme concise et facile à retenir, mais 
ils sont adaptés aux circonstances, doivent toujours être renouvelés et ne participent pas encore au 
large  accord  traditionnel  dont  jouit  le  proverbe.  S'ils  peuvent  exciter  à  l'action,  ils  servent 
beaucoup moins à déterminer une croyance: leur rôle est essentiellement celui d'imposer, par leur 
forme, certaines idées à notre attention. » 
\bigskip
(1) Jean Paulhan, Les hain-tenys, p. 37. 
(2) Cf. L. Bellak, The nature of slogans, Journal of abnormal and social Psychology, vol. 37, 1942, 
pp. 496 à 510. 
\bigskip
§ 41. FIGURES DE RHETORIQUE ET ARGUMENTATION 
\bigskip
P 225-226 : « Dès l'Antiquité, et vraisemblablement dès que l'homme a médité sur le langage, on a 
reconnu  l'existence  de  certains  modes  d'expression  sortant  de  l'ordinaire,  dont  l'étude  fut 
généralement incluse dans les traités de rhétorique ; d'où leur nom de figures de rhétorique. Suite 
à la tendance de la rhétorique à se limiter aux problèmes de style et d'expression, les figures furent 
de plus en plus considérées comme de simples ornements, contribuant à rendre le style artificiel et 
fleuri.  Quand  un  orateur,  comme  Latron,  professait  l'opinion  que  les  figures  n'avaient  pas  été 
inventées  comme  ornement,  on  considérait  cet  avis  comme  digne  de  remarque  (1)  ;  l'opinion  commune,  parmi  les  théoriciens  du  discours  persuasif,  est  celle  de  Quintilien,  pour  lequel  les 
figures  sont,  sans  doute,  un  important  facteur  de  variété  et  de  bienséance,  mais  cela  «  bien  qu'il 
semble fort peu intéressant pour la preuve que les arguments soient présentés sous telle ou telle 
figure » (2). En est-il vraiment ainsi ? Prenons la définition de  l'hypotypose (demonstratio) telle 
que nous la trouvons dans la Rhétorique à Herennius, comme figure «qui expose les choses d'une 
manière telle que l'affaire semble se dérouler et la chose se passer sous nos yeux » (3). C'est donc 
une  façon  de  décrire  les  événements  qui  les  rend  présents  à  notre  conscience  :  peut-on  nier  son 
rôle éminent comme facteur de persuasion ? Si l'on néglige ce rôle argumentatif des figures, leur 
étude  paraîtra  rapidement  un  vain  passe-temps,  la  quête  de  noms  étranges  pour  des  tournures 
recherchées.  Déjà  Quintilien  (4)  considérait  avec  lassitude  la  multiplicité  des  dénominations  et 
classifications proposées, leur enchevêtrement, et les divergences même quant à savoir ce qui est 
une  figure.  jean  Paulhan  constate  que,  si  l'on  s'en  tient  à  ce  qu'on  peut  tirer  des  auteurs,  «  les 
figures ont, pour seule caractéristique, les réflexions et l'enquête que poursuivent à leur propos les 
Rhétoriqueurs  »  (5).  Ce  paradoxe  oblige  Paulhan  à  repenser  le  problème  du  rapport  entre  la 
pensée et son expression. » 
\bigskip
(1) Sénèque, Controverses el suasoires, t. I, liv. I, préface, § 24. 
(2) Quintilien, Vol. III, liv. IX, chap. 1, §§ 19 à 21. 
(3) Rhétorique à Herennius, liv. IV, § 68; el. Quintilien, vol, III, liv. IX. chap. 11, § 40. 
(4) Quintilien, vol. III, liv. chap. 1, § 10 ; liv. IX, chap. III. § 99. 
(5) J. Paulhan, Les figures ou la rhétorique décryptée, ( Cahiers du Sud, n° 295 (1949), p. 387. 
\bigskip
P 227 : « Pour nous, qui nous intéressons moins à la légitimation du mode littéraire d'expression 
qu'aux  techniques  du  discours  persuasif,  il  semble  important  non  pas  tant  d'étudier  le  problème 
des figures dans son ensemble, que de montrer  en quoi et comment l'emploi de certaines figures 
déterminées s'explique par les besoins de l'argumentation. Notons, à ce propos, que déjà Cournot 
avait reconnu que les figures n'agissaient pas seulement sur la sensibilité. Car il est facile, écrivait-
il, de s'apercevoir que « le langage des philosophes n'est pas moins figuré que celui des orateurs et 
des poètes » (1). 
\bigskip
Deux  caractéristiques  semblent  indispensables  pour  qu'il  y  ait  figure :  une  structure  discernable, 
indépendante  du  contenu,  c'est-à-dire  une  forme  (qu'elle  soit,  selon  la  distinction  des  logiciens 
modernes, syntaxique, sémantique ou pragmatique) et un emploi qui s'éloigne de la façon normale 
de  s'exprimer  et,  par  là,  attire  l'attention.  L'une  de  ces  exigences  au  moins  se  retrouve  dans  la 
plupart des définitions des figures proposées au cours des siècles ; l'autre s'y introduit par quelque 
biais.  Ainsi,  Omer  Talon  définit  la  figure  comme  une  expression  par  laquelle  l'allure  du  discours 
diffère de la droite et simple habitude ». 
\bigskip
Mais il introduit, par le truchement de l'étymologie, l'idée de forme : 
\bigskip
le  nom  de  figure  semble  pris  du  masque  et  du  vêtement  des  acteurs,  lesquels  prononçaient  les 
divers genres de discours avec des formes extérieures différentes (variis corporis figuris) (2). » 
\bigskip
(1) Cournot,  Essai sur les fondements de nos connaissances, II, p. 12. 
(2) Audomari  Talei, Rhetoricae libri duo, p. 16. 
\bigskip
P  227-228 :  « Celui  qui  étudie  les  discours  au  point  de  vue  structurel  se  trouve  en  présence  de 
formes qui, d'emblée apparaîtront comme figures (par exemple la répétition) mais aussi de formes 
qui  paraissent  normales  (l'interrogation,  par  exemple)  et  qui  néanmoins  peuvent,  dans  certains 
cas,  être  considérées  comme  figures.  Le  fait  qu'elles  puissent  être  considérées  ou  non  comme 
figures, pose immédiatement le problème sous son jour le plus délicat. En effet, en principe il n'y a 
aucune structure qui ne soit susceptible de devenir figure par son usage, mais il ne suffit pas qu'un 
usage de la langue soit inhabituel pour que nous soyons autorisés à y voir une figure. » 
\bigskip
\bigskip
\bigskip
114 
\bigskip
 
P 228 : « Il faut, pour qu'elle puisse être objet d'étude, qu'une structure soit isolable, qu'elle puisse 
être reconnue comme telle ; d'autre part, il faut que l'on sache en quoi un usage doit être considéré 
comme inhabituel. La phrase exclamative, la phrase avec reprise d'hésitation, sont des structures ; 
elles ne seraient figures qu'en dehors de leur emploi normal, c'est-à-dire en dehors de la surprise et 
de l'hésitation véritables. 
\bigskip
N'est-ce  pas  établir  un  lien  direct  entre  l'emploi  de  figures  et  la  feinte ?  D'après  Volkmann,  c'est 
bien  l'idée  que  s'en  faisaient  les  Anciens  (1).  Il  est  certain,  en  tout  cas,  qu'il  n'y  a  figure  que 
lorsqu'on  peut  opérer  une  dissociation  entre  l'usage  normal  d'une  structure  et  son  usage  dans  le 
discours, lorsque l'auditeur fait une distinction entre la forme et le fond, qui lui semble s'imposer. 
Mais  c'est  quand  cette  distinction,  perçue  au  premier  abord,  s'abolit  grâce  à  l'effet  même  du 
discours, que les figures prennent toute leur signification argumentative. 
\bigskip
Il se peut que l'usage d'une structure donnée, dans des conditions anormales, ait tout bonnement 
pour  but  de  donner  du  mouvement  à  la  pensée,  de  simuler  des  passions,  de  créer  une  situation 
dramatique  qui  n'existe  pas.  Si,  par  exemple,  l'orateur  introduit  dans  sa  période  des  objections 
pour y répondre lui-même, nous serons en présence d'une figure, la prolepse, qui ne serait qu'une 
feinte. » 
\bigskip
(1) R. Volkmann, Hermagoras oder Elemente der Rhetorik, p. 275. 
\bigskip
P  229 :  « Ces  objections  peuvent  être  manifestement  imaginaires,  mais  il peut  être  important  de 
montrer que l'on avait soi-même entrevu des objections possibles, que l'on en avait tenu compte. 
En  réalité,  il  y  a  une  série  de  degrés  entre  l'objection  réelle  et  l'objection  feinte.  Une  même 
structure peut passer d'un degré à l'autre, grâce à l'effet même produit par le discours. Des formes 
qui, au premier abord, paraîtront employées de façon inaccoutumée , pourront cependant paraître 
normales si cet emploi prend sa justification par l'ensemble du discours. Nous considérerons une 
figure  comme  argumentative  si,  entraînant  un  changement  de  perspective,  son  emploi  paraît 
normal  par  rapport  à  la  nouvelle  situation  suggérée.  Si,  par  contre,  le  discours  n'entraîne  pas 
l'adhésion  de  l'auditeur  à  cette  forme  argumentative,  la  figure  sera  perçue  comme  ornement, 
comme  figure  de  style.  Elle  pourra  susciter  l'admiration,  mais  sur  le  plan  esthétique,  ou  comme 
témoignage de l'originalité de l'orateur. 
\bigskip
On  voit,  dès  lors,  qu'on  ne  saurait  décider,  d'avance,  si  une  structure  déterminée  doit  être 
considérée ou non comme figure, ni si elle jouera le rôle de figure argumentative ou de figure de 
style ; tout au plus peut-on déceler un nombre de structures aptes à devenir figure. 
\bigskip
Certaines  figures,  comme  l'allusion,  ne  se  reconnaissent  jamais  que  dans  leur  contexte,  car  leur 
structure  n'est  ni  grammaticale,  ni  sémantique,  mais  tient  à  un  rapport  avec  quelque  chose  qui 
n'est  pas  l'objet  immédiat  du  discours.  Si  cette  manière  de  s'exprimer  est  perçue  comme 
inaccoutumée nous aurons une figure : c'est le mouvement du discours, l'adhésion de l'auditeur à 
la forme d'argumentation qu'elle favorise, qui détermineront le genre de figure auquel on a affaire. 
Notons, dès à présent, que l'allusion aura presque toujours valeur argumentative, parce qu'elle est 
essentiellement élément d'accord et de communion. » 
\bigskip
P 229-230 : « On percevra mieux encore l'importance du mouvement du discours, si l'on envisage 
certaines  métaphores. A. Smith, dans un passage célèbre, montre par quel mécanisme l'individu, 
poursuivant soit profit personnel, sert aussi l'intérêt général : 
\bigskip
...  il  a  seulement  en  vue  son  propre  gain  et,  en  cela  comme  dans  beaucoup  d'autres  cas,  il  est 
conduit par une main invisible à promouvoir une fin qui ne faisait pas partie de son intention (1) 
. » 
\bigskip
\bigskip
\bigskip
115 
\bigskip
 
(1) A. Smith, The wealth of nations, p. 423. 
\bigskip
P  230 :  « La  célèbre  expression  «  main  invisible  »,  utilisée  par  Smith  n'est  généralement  pas 
perçue  par  l'auditeur  comme  l'expression  normale  de  la  pensée,  en  ce  sens  que  peu  d'auditeurs 
admettront  qu'A.  Smith  a  réellement  en  vue  une  main  de  chair  et  d'os;  mais  l'auditeur  sent  que 
cette  main  invisible  doit  persuader  de  ce  que  l'harmonie  entre  l'intérêt  individuel  et  l'intérêt 
collectif n'est pas due au hasard, de ce qu'il serait permis d'en rendre compte par une intervention 
surnaturelle,  de  ce  que  la  prescience  refusée  à  l'homme  peut  être  celle  d'un  être  suprême.  Bref, 
nous n'analyserons pas ici le mécanisme de cette figure, mais nous voudrions montrer que du fait 
que  l'on  peut  adhérer  à  la  valeur  argumentative  qu'elle  recèle,  cette  expression  sera  bien 
considérée comme une figure, mais non comme une figure de style. Remarquons, à ce propos, que 
pour  être  perçue  comme  argumentative,  une  figure  ne  doit  pas  nécessairement  entraîner 
l'adhésion aux conclusions du discours : il suffit que l'argument soit perçu à sa pleine valeur ; peu 
importe si d'autres considérations s'opposent à l'acceptation de la thèse en question. 
\bigskip
Il  résulte  de  ce  qui  précède  qu'une  figure,  dont  l'effet  argumentatif  n'est  pas  réussi,  tombera  au 
rang  de  figure  de  style.  Ainsi,  pour  dénier  à  une  théorie  philosophique  une  valeur  autre  que 
littéraire, on prétendra n'y voir qu'une figure de rhétorique. 
\bigskip
Ce  passé  bergsonien,  écrit  Sartre,  qui  adhère  au  présent  et  le  pénètre  même,  n'est  guère  qu'une 
figure de rhétorique. Et c'est ce que montrent bien les difficultés que Bergson a rencontrées dans 
sa théorie de la mémoire (2). » 
\bigskip
(2) J.-P. Sartre, L'être et le néant, p. 179. 
\bigskip
P 231 : « Si les auteurs qui se sont occupés des figures ont eu tendance à ne percevoir que leur côté 
stylistique, cela tient donc, pensonsnon-,, à ce que, à partir (lu moment oit une figure est détachée 
du  contexte,  mise  dans  un  herbier,  elle  est  presque  nécessairement  perçue  sous  son  a,spect  le 
moins argumentatif ; pour saisir son aspect argumentatif, il faut concevoir le passage de l'habituel 
à l'inhabituel et le retour à un habituel d'un autre ordre, celui produit par l'argument au moment 
même  où  il  s'achève.  En  outre,  et  c'est  peut-être  là  le  point  le  plus  important,  il  faut  se  rendre 
compte de ce que l'expression normale est relative non seulement à un milieu, à un auditoire, mais 
à un moment déterminé du discours. Admet-on, par contre, qu'il y a une manière de s'exprimer qui 
est la bonne, l'authentique, la vraie, la normale, on ne peut concevoir la figure que comme quelque 
chose de statique : une expression est ou n'est pas figure ; on ne peut imaginer qu'elle le soit ou ne 
le soit pas suivant la réaction de l'auditeur. Une conception plus souple, qui considère le  normal 
dans toute sa mobilité, peut, seule, rendre entièrement aux figures argumentatives la place qu'elles 
occupent réellement dans le phénomène de persuasion. 
\bigskip
Nous retrouvons ainsi, par la relativisation du normal, une remarque du pseudo-Longin : 
\bigskip
Il  n'y  a  point  de  Figure  plus  excellente  que  celle  qui  est  tout-à-fait  cachée,  et  lors  qu'on  ne 
reconnoît  point  que  c'est  une  Figure.  Or  il  n'y  a  point  de  secours  ni  de  remède  plus  merveilleux 
pour  l'empêcher  de  paroître,  que  le  Sublime  et  le  Pathétique;  parce  que  l'Art  ainsi  renfermé  au 
milieu  de  quelque  chose  de  grand  et  d'éclatant,  a  tout  ce  qui  lui  manquoit,  et  n'est  plus  suspect 
d'aucune tromperie (1). 
\bigskip
Les habits de fête paraissent adéquats dans un certain contexte et ne s'y font pas remarquer. » 
\bigskip
(1) Longin, Traité du sublime, chap. XV, p. 97. 
\bigskip
\bigskip
\bigskip
116 
\bigskip
§ 42. LES FIGURES DU CHOIX, DE LA PRESENCE ET DE LA COMMUNION 
\bigskip
P  232 :  « Lorsque  nous  nous  occuperons  d'une  figure  et  examinerons  ce  qu'elle  apporte  à 
l'argumentation, nous nous servirons le plus volontiers, pour la désigner, du nom  sous lequel elle 
est traditionnellement connue ; ceci permettra une entente plus facile avec le lecteur et renverra à 
une structure qui a déjà attiré l'attention dans le passé. Les exemples eux-mêmes seront volontiers 
pris dans la tradition. Par contre, les classifications de figures, généralement utilisées, ne peuvent 
en rien nous aider. Au contraire, nous croyons qu'une des distinctions majeures, celle entre figures 
de pensées et figures de mots, inconnue à Aristote, mais qui semble obligatoire depuis le deuxième 
siècle avant notre ère, a contribué à obscurcir toute la conception des figures de rhétorique. 
\bigskip
De  notre  point  de  vue,  nous  constaterons  qu'une  même  figure,  reconnaissable  à  sa  structure,  ne 
produit  pas  nécessairement  toujours  le  même  effet  argumentatif.  Or  c'est  ce  dernier  qui  nous 
intéresse  avant  tout.  Au  lieu  de  procéder  à  un  examen  exhaustif  de  toutes  les  figures 
traditionnelles, nous nous demanderons, à propos de tel ou tel procédé ou schème argumentatif, si 
certaines  figures  sont  de  nature  à  remplir  la  fonction  que  nous  avons  reconnue  à  ce  procédé,  si 
elles peuvent être considérées comme une des manifestations de celui-ci. Par ce biais, les figures 
seront  en  quelque  sorte  démembrées.  Non  seulement  les  figures  se  repartiront  entre  divers 
chapitres de notre étude, mais  nous verrons des exemples d'une même figure trouver place dans 
des chapitres différents. C'est le démembrement même qui, pensons-nous, pourra le mieux mettre 
en relief la signification argumentative des figures. » 
\bigskip
P  232-233 :  « Pour  illustrer  notre  façon  de  procéder,  nous  passerons  rapidement  en  revue 
quelques  figures  du  choix,  de  la  présence,  de  la  communion.  Ces  termes  ne  désignent  pas  des 
genres dont certaines figures traditionnelles seraient les espèces. Ils signifient seulement que l'effet 
ou  l'un  des  effets,  de  certaines  figures  est,  dans  la  présentation  des  données,  d'imposer  ou  de 
suggérer un choix, d'accroître la présence ou de réaliser la communion avec l'auditoire. » 
\bigskip
P 233 : « L'un des modes essentiels du choix, l'interprétation, peut donner, semble-t-il, lieu à une 
figure  argumentative.  Nous  serions  enclins  à  considérer  comme  telle,  le  procédé  relevé  par 
Sénèque dans la controverse relative au fils qui, malgré l'interdiction de son père nourri un oncle. 
L'un des défenseurs du fils allègue que celui-ci cru que les ordres du père ne correspondaient pas à 
son vrai désir. Mais Cestius, plus hardi, fait dire au père par son fils :  
\bigskip
Tu l'as voulu et tu le veux encore aujourd'hui (1). 
\bigskip
L'interprétation,  fort  audacieuse,  est  donnée  comme  un  fait  et  sera  perçue  comme  une  figure 
argumentative ou comme figure de style selon l'effet produit sur l'auditoire.  
\bigskip
La définition oratoire est une figure du choix, car elle utilise la structure de la définition, non pour 
fournir  le  sens  d'un  mot,  mais  pour  mettre  en  vedette  certains  aspects  d'une  réalité  qui 
risqueraient de rester à l'arrière-plan de la conscience. Fléchier, voulant faire valoir les capacités 
d'un général, formule sa définition de l'armée, nous dit Baron, 
\bigskip
de manière que chaque proposition soit une des prémisses d'un syllogisme qui ait pour conclusion 
: donc il est difficile de commander une armée. » 
\bigskip
(1) Sénèque, controverses, liv. I, I, § 15. 
\bigskip
P 233-234 : « Voici le texte : 
\bigskip
Qu'est-ce qu'une armée ? C'est un corps animé d'une infinité de passions différentes qu'un homme 
habile  fait  mouvoir  pour  la  défense  de  la  patrie  ;  c'est  une  troupe  d'hommes  armés  qui  suivent 
aveuglément les ordres d'un chef dont ils ne savent pas les intentions., c'est une multitude d'âmes 
\bigskip
\bigskip
\bigskip
117 
\bigskip
pour  la  plupart  viles  et mercenaires, qui,  sans  songer  à  leur  propre réputation,  travaillent  à  celle 
(les rois et conquérants ; c'est un assemblage confus de libertins... (1). » 
\bigskip
(1) Baron, De la Rhétorique, p. 61.  
\bigskip
P  234 :  « Le  cas  de  la  définition  oratoire  nous  montre  nettement  que  le  caractère  anormal  d'une 
structure  peut  être  envisagé  à  un  double  point  de  vue.  D'une  part,  la  définition  oratoire,  tout  en 
présentant la structure d'une définition, ne joue pas le rôle habituel de celle-ci; d'autre part, l'eff et 
produit habituellement par l'épithète, ou la qualification, c'est-à-dire le choix, est produit cette fois 
grâce à la définition oratoire. Si l'on mettait l'accent sur le premier point de vue, on serait amené à 
traiter  de  la  définition  oratoire  à  propos  de  la  définition.  C'est  parce  que  nous  nous  référons  au 
second point de vue, à l'aspect fonctionnel, à l'action sur l'auditoire, que nous en traitons comme 
figure de choix. 
\bigskip
La périphrase peut jouer le même rôle que la définition oratoire: « les trois déesses infernales qui 
selon  la  fable,  tissent  la  trame  de  nos  jours  »  pour  désigner  les  l'arques,  sera  perçu  comme  une 
périphrase si cette expression ne sert pas à fournir une définition du terme « Parques » mais à le 
remplacer,  ce  qui  suppose  que  l'on  connaît  l'existence  du  nom  auquel  on  substitue  cette 
expression. Le rôle  argumentatif de l'énoncé est très net dans ces vers d'Athalie, dont le premier 
peut cependant être perçu comme une périphrase pour désigner Dieu : 
\bigskip
Celui qui met un frein à la fureur des flots Sait aussi des méchants arrêter les complots (2). » 
\bigskip
(2) Racine, Bibl. de la Pléiade, Athalie, acte 1, se. 1, p. 896. 
\bigskip
P  234-235 :  « Beaucoup  de  périphrases  peuvent  s'analyser  en  termes  de  figures,  telles  la 
synecdoque,  la  métonymie,  dont  la  fonction  n'est  pas  essentiellement  celle  du  choix  (1),  encore 
qu'elles puissent y servir : « les mortels » pour « les homines » est nue manière d'attirer l'attention 
sur  une  caractéristique  particulière  des  homines.  Mentionnons  surtout  ici,  en  tant  que  figure  du 
choix, l'antonomase que Littré définit comme « une sorte de synecdoque qui consiste à prendre un 
nom commun pour un nom propre ou un nom propre pour un tient commun ». Sous sa première 
forme  elle  vise  parfois  à  éviter  de  prononcer  un  nom  propre  ;  mais  parfois  aussi  à  qualifier 
quelqu'un  d'une  façon  utile  pour  l'argumentation  :  «les  petits-fils  de  l'Africain  »  pour  «  les 
Gracques », peut tendre à ce but. » 
\bigskip
P  235 :  « La  prolepse  ou  anticipation  (praesumptio)  peut  être  figure  de  choix  quand  elle  vise  à 
insinuer  qu'il  y  a  lieu  de  substituer  une  qualification  à  une  autre  qui  aurait  pu  soulever  des 
objections : 
\bigskip
Et pourtant c'était moins un châtiment qu'un moyen de prévenir le crime (2). 
\bigskip
L'hésitation  que  marque  la  reprise  (reprehensio)  n'a  sans  doute  d'autre  but  que  de  souligner  la 
légitimité d'un choix : 
\bigskip
Citoyens, dis-je, s'il est permis de les appeler de ce nom (3). 
\bigskip
De même la correction, qui remplace un mot par un autre 
\bigskip
Si l'accusé en avait prié ses hôtes, ou plutôt, s'il leur avait fait seulement un signe... (4). 
\bigskip
Les figures de la présence ont pour effet de rendre présent à la conscience l'objet du discours. » 
\bigskip
(1) Cf. § 75 : La liaison symbolique.  
\bigskip
\bigskip
\bigskip
118 
\bigskip
(2) Quintillien, Vol. IIi, liv. IX, chap. II, § 18.  
(3) Ibidem.  
(4) Rhétorique à Herennius, liv. IV, § 36. 
\bigskip
P 235-236 : « La première de ces figures est l'onomatopée. Que l'onomatopée spontanée ait été ou 
non à l'origine de certains termes du langage n'est guère relevant. L'onomatopée est perçue comme 
figure lorsqu'il y a, pour évoquer un bruit réel, soit création d'un mot, soit usage inaccoutumé de 
mots existants, peu importe d'ailleurs que le son reproduise exactement ou non le bruit de ce que 
l'on veut rendre présent : l'intention d'imitation semble seule compter. Il est amusant de constater, 
à ce propos, que Dumarsais donne comme exemple d'onomatopée « bilbit amphora » qu'il traduit 
par « la petite bouteille fait glouglou » (1). » 
\bigskip
(1) Dumarsais, Des Tropes, p. 161. 
\bigskip
P 236 : « Parmi les figures ayant pour effet d'augmenter le sentiment de présence, les plus simples 
se  rattachent  à  la  répétition,  qui  est  importante  en  argumentation,  alors  que,  dans  une 
démonstration  et  dans  le  raisonnement  scientifique  en  général,  elle  n'apporte  rien.  La  répétition 
peut  agir  directement  ;  elle  peut  aussi  accentuer  le  morcellement  d'un  événement  complexe  en 
épisodes détaillés, apte, nous le savons, à favoriser la présence. Ainsi dans cet exemple d'anaphore, 
répétition des premiers mots dans deux phrases successives : 
\bigskip
Trois fois je lui jetai les bras au cou Trois fois s'enfuit la vaine image (2). 
\bigskip
Mais  la  plupart  des  figures  que  les  rhétoriciens  classent  sous  le  nom  de  figures  de  répétition  (3) 
paraissent  avoir  un  effet  argumentatif  beaucoup  plus  complexe  que  de  donner  la  présence.  C'est 
que sous la forme de la répétition elles visent surtout à suggérer des distinctions : ainsi en est-il des 
expressions du type : 
\bigskip
« Corydon depuis lors est pour moi Corydon! » 
\bigskip
qui sont perçues comme figure en raison même de cet usage anormal de la répétition (4). » 
\bigskip
(2) Cité par Vico, Delle instituzioni oratorie, p. 142 
(3) Ibid., pp. 142 et suiv. 
(4) Cf. § 51 : Analycité, analyse et tautologie. 
\bigskip
P  236-237 :  « Plus  proches  cependant  de  figures  de  la  présence  sont  la  conduplicatio  de  la 
Rhétorique à Herennius et l'adjectio de Quintilien : 
\bigskip
Des guerres, C. Gracchus, des guerres domestiques et intestines, voilà ce que tu soulèves ... (1). 
\bigskip
J'ai tué, oui j'ai tué ... (2). » 
\bigskip
(1) Rhétorique à Herennius, liv. IV, § 38. 
(2) Quintilien, Vol. III, liv. IX, chap. III, § 28. 
\bigskip
P  237 :  « Ici  encore  le  résultat  produit  par  la  répétition  n'est  pas  seulement  de  doubler  l'effet  de 
présence ; par la répétition, le deuxième énoncé du terme parait chargé de valeur ; le premier, par 
réaction, paraît se rapporter exclusivement à un fait alors que, normalement et seull il aurait paru 
contenir  fait  et  valeur.  L'effet  de  présence  est  donc  subordonné  à  d'autres  intentions.  C'est 
pourquoi  nous  ne  pouvons  souscrire  à  l'explication  de  Chaignet,  bien  qu'elle  ait  le  mérite  de 
chercher un sens à l'emploi de la répétition : 
\bigskip
\bigskip
\bigskip
\bigskip
119 
\bigskip
Il est clair que si l'on a beaucoup à dire d'une personne ou d'une chose, on est obligé de la désigner 
plusieurs fois par son nom; réciproquement, si on  la  nomme plusieurs fois, il semble qu'on a dit 
d'elle eaucoup de choses (3). 
\bigskip
L'effet de présence s'obtient, bien plus que par une répétition littérale, par un autre procédé qui est 
l'amplification  :  nous  entendons,  par  là,  le  développement  oratoire  d'un  sujet,  indépendamment 
de l'exagération avec laquelle on l'associe généralement. 
\bigskip
Quand  et  pourquoi  l'amplification  est-elle  perçue  comme  une  figure  ?  C'est  surtout,  semble-t-il, 
lorsqu'elle utilise des formes qui, normalement, visent à un autre but que la présence : c'est le cas 
notamment dans l'amplification par énumération des parties qui rappelle une argumentation quasi 
logique (4). Voici un exemple de congérie donné par Vico : 
\bigskip
Tes  yeux  sont  formés  à  l'impudence,  le  visage  à  l'audace,  la  langue  aux  parjures,  les  mains  aux 
rapines, le ventre à la gloutonnerie... les pieds à la fuite : donc, tu es toute malignité (5). 
\bigskip
 (3) Chaignet, La rhétorique et son histoire, pp. 515-516. 
(4) Cf. § 56 : La division du tout en ses parties. 
(5) Vico, Delle instituzioni oratorie, p. 81. 
\bigskip
P 238 : « De même, la synonymie ou métabole qui se décrit comme la répétition d'une même idée 
à  l'aide  de  mots  différents,  utilise,  pour  donner  la  présence,  une  forme  qui  suggère  la  correction 
progressive. Dans : 
\bigskip
Va, cours, vole et nous venge (1). 
\bigskip
on  fait  usage  de  termes  qui  paraissent  convenir  de  mieux  en  mieux;  la  svnonymie  serait  comme 
une correction abrégée, ou même comme une prolepse abrégée : elle donnerait la présence à l'aide 
d'une forme destinée essentiellement au choix. 
\bigskip
Très  proche  de  cette  figure  est  le  procédé  (interpretatio)  qui  consiste  à  expliciter  un  membre  de 
phrase par un autre, mais cela moins aux fins de clarification que pour accroître la présence : 
\bigskip
C'est  la  république que  tu  as  renversée  de  fond  en  comble,  l'Etat  que  tu  as abattu  complètement 
(2). 
\bigskip
Dans  le  pseudo-discours  direct  on  augmente  le  sentiment  de  présence  en  attribuant  fictivement 
des paroles à une personne ou à plusieurs conversant entre elles; la tradition distingue à ce propos 
la  sermocination  du  dialogisme  (3).  Remarquons  que  les  buts  du  pseudodiscours  direct  peuvent 
être multiples; mais ils relèvent toujours de l'hypothèse.  Or nous avons déjà vu le rôle de celle-ci 
pour donner la présence (4). Le pseudo-discours direct fera connaître les intentions que l'on prête 
à quelqu'un, ou ce que l'on croit être l'opinion d'autrui sur ces intentions. Il pourra être présenté 
comme mi-prononcé, mi-pensé. Sous ce dernier mode très ambigu, Browning s'en est amplement 
servi dans son célèbre poème The Ring and the Book. » 
\bigskip
(1) Corneille, Le Cid, acte 1. sc. VI. 
(2) Rhétorique à Herennius, liv. IV, § 38. 
(3) Cf. Vico, Delle instituzioni oratorie, p. 151. 
(4) Cf. § 37 : Problèmes techniques de présentation des données. 
\bigskip
P 238-239 : « Signalons enfin les figures relatives au temps grammatical. C'est le brusque passage 
du passé, temps du récit, ait présent, temps de la description, qui fait souvent que celle-ci apparaît 
\bigskip
\bigskip
\bigskip
120 
\bigskip
comme une figure, l'hypotypose (1) dont nous avons déjà parlé; le type généralement cité en est le 
récit de la mort d'Hippolyte, dont tous les verbes sont au présent (2). » 
\bigskip
(1) Cf. Longin, Traité du sublime, chap. XXI, p. 112.  
(2) Racine, Bibl. de la Pléiade, Phèdre, acte V, cène VI, pp. 817-818. 
\bigskip
P 239 : « La substitution syntaxique d'un temps à un autre, contrairement aux liaisons normales, 
c'est-à-dire l'énallage de temps, pourra avoir un effet de présence très marqué : « Si tu parles, tu es 
mort  »  suggère  que  la  conséquence  se  produira  instantanément,  au  moment  où  l'on  transgresse 
l'injonction. 
\bigskip
Les figures de communion sont celles où, au moyen de procédés littéraires, l'on s'efforce de créer 
ou de confirmer la communion avec l'auditoire. Souvent cette communion est obtenue grâce à des 
références à une culture, une tradition, un passé communs. 
\bigskip
L'allusion,  que  beaucoup  d'auteurs  traitent  comme  une  figure,  joue  certainement  ce  rôle.  Il  y  a 
allusion,  lorsque  l'interprétation  d'un  texte,  si  l'on  négligeait  la  référence  volontaire  de  l'auteur  à 
quelque chose qu'il évoque sans le désigner, serait incomplète ; ce quelque chose pouvant consister 
en un événement du passé, en un usage ou -un fait de culture, dont la connaissance est propre aux 
membres  du  groupe  avec  lesquels  l'orateur  cherche  à  établir  cette  communion.  A  ces  faits  de 
culture s'attache généralement une affectivité particulière : attendrissement devant les souvenirs, 
orgueil de la communauté; l'allusion augmente le prestige de l'orateur qui possède et sait utiliser 
ces richesses. Ainsi fait Mirabeau dans ce passage cité par Baron: 
\bigskip
je  n'avais  pas  besoin  de  cette  leçon  pour  savoir  qu'il  n'y  a  qu'un  pas  du  Capitole  à  la  roche 
Tarpéienne (3). » 
\bigskip
(3) Baron, De la Rhétorique, p. 335. 
\bigskip
P 240 : « La citation n'est qu'une figure de communion quand elle ne sert pas à ce qui est son rôle 
normal, appuyer ce que l'on dit par le poids d'une autorité (1). 
\bigskip
Maximes et proverbes peuvent, eux aussi, être considérés comme des citations : lorsque leur usage 
ne  semble  pas  résulter  des  besoins  de  l'argumentation,  leur  contenu  passant  au  second  plan,  ils 
seront  perçus  comme  figure  ;  ils  deviennent  le  signe  de  l'enracinement  dans  une  culture  chez 
Sancho  Pança  ou  chez  Tèvié  le  Laitier  (2).  De  même  que  le  cliché,  la  citation  peut  être  perçue 
comme un formalisme. Mais le personnage dont La Bruyère nous dit : 
\bigskip
Ce n'est ni pour donner plus d'autorité à ce qu'il dit, ni peut-être pour se faire honneur de ce qu'il 
sait. Il veut citer (3). 
\bigskip
est sans doute encore, à ce moment, en quête de communion avec l'auditoire. 
\bigskip
La communion s'accroît également au moyen de toutes les figures par lesquelles l'orateur s'efforce 
de faire participer activement l'auditoire à son exposé, le prenant à partie, sollicitant son concours, 
s 1 assimilant à lui. 
\bigskip
L'apostrophe,  la  question  oratoire,  qui  ne  vise  ni  à  s'informer  ni  à  s'assurer  un  accord,  sont 
souvent  figures  de  communion  ;  dans  la  communication  oratoire  on  demande  à  l'adversaire 
même,  au  juge,  de  réfléchir  à  la  situation  dans  laquelle  on  se  trouve,  on  l'invite  à  participer  à  la 
délibération  que  l'on  semble  poursuivre  devant  lui  (4),  ou  bien  encore  l'orateur  cherche  à  se 
confondre avec son auditoire: 
\bigskip
\bigskip
\bigskip
\bigskip
121 
\bigskip
Or, je vous le demande, s'exclame Massillon, et vous le demande frappé de terreur, ne séparant pas 
en ce point mon sort du vôtre...(5) » 
\bigskip
(1) Cf. § 70 : L'argument d'autorité. 
(2) Scholem, Alei'heim, L'histoire de Tèvié. 
(3) La Bruyère, Bibl. de la Pléiade, Caractères, Des jugements, 6-1, p. 385. 
(4) Vico, Delle instituzioni oratorie, p. 147. 
(5) Cité par Saint-Aubin, Guide pour la classe de rhétorique, p. 91. MASSILLON., Carême, Sermon 
XIX : Sur le petit nombre des élus, t. 1, col. 722. 
\bigskip
P 241 : « Le même effet est également obtenu par l'énallage de la personne, le remplacement du « 
je » ou du «il » par le «tu », qui fait que « l'auditeur se croit voir lui-même au milieu du péril » (1) 
et qui est figure de présence et de communion. Et aussi par l'énallage du nombre de Personnes, le 
remplacement du « je », du « tu », par le « nous ». C'est celui qu'utilise la mère disant à l'enfant : « 
Nous allons nous coucher. » 
\bigskip
Nous  en  trouvons  un  excellent  exemple  dans  Massillon,  chez  qui  le  souci  de  l'assimilation  avec 
l'auditoire est constant : 
\bigskip
Et voilà, mon cher auditeur, de quoi vous instruire et vous confondre en même temps. Vous vous 
plaignez de l'excès de vos malheurs... Or, quoi de plus consolant dans nos peines ? Dieu me voit, il 
compte mes soupirs, il pèse rues afflictions, il regarde couler mes larmes... (2). 
\bigskip
«  Vous  »,  «  nous  »,  «  je  »,  autant  d'étapes  par  lesquelles  l'orateur  s'assimile  à  ses  auditeurs,  la 
dernière se confondant d'ailleurs avec le pseudo-discours direct qui, lui aussi, peut donc être figure 
de communion. 
\bigskip
Ces  quelques  indications  sur  le  rôle  de  certaines  figures  dans  la  présentation  des  données, 
suffisent,  pensons-nous,  à  montrer  comment  on  peut  rattacher  leurs  effets  à  des  facteurs  assez 
généraux de persuasion. Notre analyse des figures est donc subordonnée à  une analyse préalable 
de  l'argumentation.  On  pourrait  objecter  que,  par  ce  biais,  nous  ne  toucherons  jamais  à  ce  que 
d'aucuns pourraient juger essentiel dans l'étude des figures. 
\bigskip
Nous croyons cependant qu'il y a intérêt à en traiter ainsi. Nous reprendrons donc, par la suite, ce 
même point de vue chaque fois que l'occasion s'en présentera. » 
\bigskip
(1) Longin, Traité du sublime, chap. XXII, pp. 112-113. 
(2) Massillon, Sermon IV. Pour le second dimanche de l'Avent. Sur les Afflictions, t. 1, col. 241. 
\bigskip
§ 43. LE STATUT DES ELEMENTS D'ARGUMENTATION ET LEUR PRESENTATION 
\bigskip
P 242 : « L'un des effets importants de la présentation des données consiste dans la modification 
du statut des éléments du discours. 
\bigskip
Les  différents  types  d'objets  d'accord  jouissent,  nous  le  savons,  de  privilèges  différents.  Certains 
d'entre eux sont censés bénéficier de l'accord de l'auditoire universel : ce sont les faits, les vérités, 
les  présomptions.  D'autres  ne  bénéficient  que  de  l'accord  d'auditoires  particuliers  :  ce  sont,  les 
valeurs,  les  hiérarchies,  les  lieux.  La  précarité  de  ces  différents  objets  d'accord  n'est  pas  liée  aux 
mêmes conditions. D'où le grand intérêt qui s'attache à la fixation du statut des éléments utilisés, à 
la  transposition  de  certains  éléments  dans  une  autre  catégorie,  à  la  possibilité  de  mettre  l'accent 
sur un type d'objets d'accord plutôt que sur un autre. 
\bigskip
Normalement, l'orateur et son auditoire sont censés reconnaître le même statut aux éléments du 
discours, du moins jusqu'à ce qu'une divergence explicite oblige à modifier cette hypothèse. Mais il 
\bigskip
\bigskip
\bigskip
122 
\bigskip
arrive bien souvent que, dans l'intérêt de son argumentation, l'orateur fasse un effort pour situer le 
débat  sur  le  plan  qui  lui  semble  le  plus  favorable,  en  modifiant  au  besoin  le  statut  de  certaines 
données. Sur ce point la présentation joue un rôle essentiel. 
\bigskip
Le plus souvent l'effort de l'orateur tend à attribuer aux éléments sur lesquels il s'appuie, le statut 
le plus élevé possible, le statut qui jouit de l'accord le plus étendu. C'est ainsi que le statut de valeur 
sera attribué aux sentiments personnels, le statut de fait sera attribué aux valeurs. » 
\bigskip
P  242-243 :  « Les  sentiments  et  impressions  personnels  sont  souvent  exprimés  comme  des 
jugements de valeur largement partagés. Le type en serait l'affirmation du touriste qui, rentrant de 
voyage,  nous  dit:  «  Comme  il  est  agréable  de  voyager  en  France!  »  on  l'exclamation  du  jeune 
amoureux : « Comme la lune est belle ce soir ! » De telles expressions, comme le remarque Britton 
(1),  sont  plus  efficaces  dans  la  conversation,  devant  un  auditoire  d'intimes,  que  dans  des  écrits 
destinés à n'importe quel lecteur. Il s'agit moins d'un jugement de valeur, que l'on serait disposé à 
défendre, que d'une impression que l'on demande à un auditoire bienveillant de partager. » 
\bigskip
(1) K. Britton, Communication, p. 48. 
\bigskip
P 243 : « Des jugements de valeur, et même des sentiments purement subjectifs, peuvent, par des 
artifices  de  présentation,  être  transformes  en  jugement  de  fait.  La  formule  «  ces  pommes  ne  me 
disent  rien  »  pour  «  je  n’aime  pas  ces  pommes  »  permet  d'opérer  une  sorte  de  déplacement  de 
responsabilité.  On  reproche  à  l'objet  de  ne  pas  adresser  d'appel,  on  considère  que  si  l'on  réagit 
défavorablement, cela résulte d'un comportement de l'objet. Bien entendu cette assertion porte sur 
un fait invérifiable et l'auditeur pourrait refuser son accord. Mais nul n'y songe jusqu'au moment 
oil il voudrait défendre, en contradicteur, l'excellence de ces pommes. » 
\bigskip
P 243-244 : « En remplaçant la qualification « menteur » par « personne ayant une disposition à 
induire sciemment en erreur » (2), on aura l'impression d'avoir transformé le jugement de valeur 
où  cette  qualification  apparaît,  en  jugement  de  fait,  parce  que  l'énoncé,  sous  sa  nouvelle  forme, 
semble plus précis, que l'on insiste sur ses conditions de vérification. Que le terme « menteur » ne 
soit  pas  utilisé  souligne  d'ailleurs  l'intention  d'éviter  une  appréciation  défavorable.  L'usage  de 
termes  servant  habituellement  à  la  description  de  faits,  pour  inciter  à  des  jugements  de  valeur, 
sans les énoncer explicitement, est opportun devant des auditoires qui se méfient de tout ce qui ne 
semble pas vérifiable. Celui qui, au lien de dire «j'ai bien agi », déclare, « j'ai agi de telle façon », 
semble se borner à une affirmation de fait, indéniable et objective. Il obtient toutefois, de manière 
détournée, aux yeux de celui qui est tenté d'approuver cette façon d'agir, le même résultat que par 
l'affirmation de valeur. Et l'avantage de la transposition est indubitable parce que la valeur n'étant 
point énoncée, on ne l'expose pas à être inutilement mise en question. De même, au lieu de vanter 
les  mérites  d'une  personne,  il  suffit  de  signaler  certains  faits  en  s'abstenant  d'énoncer  la 
valorisation qui en dérive, laissant ce soin à l'auditeur. » 
\bigskip
(2)  § 38 : Formes verbales et argumentation. 
\bigskip
P 244 : «  Les jugements de valeur peuvent également être transformés en expression de faits en 
les  attribuant  à  quelqu'un:  ce  changement  de  statut  est  généralement  suggéré  pour  donner  du 
poids  à  l'énoncé.  Mais  il  petit  aussi  avoir  pour  effet  de  limiter  la  portée  de  celui-ci  :  une  norme, 
appuyée  de  l'autorité  d'un  personnage  célèbre,  risque  de  se  transformer  ainsi  en  simple  fait  de 
culture. 
\bigskip
Une autre technique consiste à présenter comme un fait d'expérience ce qui n'est que la conclusion 
d'une  argumentation.  Dans  l'ouvrage  qu'il  consacre  aux  fraudes  en  archéologie  préhistorique, 
Vayson de Pradenne s'attache à l'argumentation des parties et signale que Chierici, en défendant 
l'authenticité des silex de Breonio affirme : « La seule inspection de ces silex exclut tout soupçon 
\bigskip
\bigskip
\bigskip
123 
\bigskip
de travail récent (1). » Vayson de Pradenne y voit une forme de l'argument d'autorité. En réalité, 
l'intérêt de l'énoncé réside précisément en ce qu'il n'est pas présenté comme argument d'autorité, 
mais comme un témoignage concernant un fait vérifiable. 
\bigskip
Celui  qui  qualifie  la  solution  qu'il  considère  comme  la  meilleure  d'unique  solution,  opère  une 
transposition analogue du jugement de valeur en jugement de fait. » 
\bigskip
(1) Vayson De Pradenne, Les fraudes en archéologie préhistorique, p. 244. 
\bigskip
P 244-245 : « Parfois le désaccord sur les valeurs est présenté comme un désaccord sur des faits, 
parce  qu'il  est  plus  facile  de  rectifier  une  erreur  matérielle  qu'un  jugement  de  valeur  que  l'on 
désapprouve. Le type de cette technique argumentative serait le recours du pape mal informé au 
pape mieux informé: on suppose que le désaccord repose sur une information insuffisante et qu'il 
suffira  de  compléter  pour  faire  changer  d'avis  la  personne  mal  informée.  De  même,  en  présence 
d'une  loi  contestée,  on  augmentera  sa  valeur  en  proclamant  que  si  elle  a  été  transgressée,  ce  ne 
peut  être  que  par  ignorance.  On  sous-entend  que  si  on  la  connaissait  on  n'hésiterait  pas  à  la 
suivre. » 
\bigskip
P  245 :  « Un  exemple  comique  de  cette  façon  d'argumenter,  justement  parce  qu'il  s'agit  d'une 
feinte,  est  signalé  par  Quintilien.  C'est  la  réponse  d'un  chevalier  romain  à  Auguste,  qui  lui 
reprochait  de  dilapider  son  patrimoine  :  «  J'ai  cru  qu'il  était  à  moi  »  (1)  répond  le  chevalier, 
comme si le reproche n'avait d'autre fondement qu'une erreur de fait. 
\bigskip
Certaines  figures,  et  spécialement  la  métalepse,  peuvent  faciliter  la  transposition  des  valeurs  en 
faits.  «  Il  oublie  les  bienfaits  »  pour  «  il  n'est  pas  reconnaissant  »;  «  souvenez-vous  de  notre 
convention  »  pour  «  observez  notre  convention  »,  sont  manières  d'attribuer  une  conduite  à  un 
phénomène  de  mémoire,  permettant à  l'interlocuteur  de  modifier  son  attitude  tout  en  ayant  l'air 
d'avoir seulement amélioré sa connaissance des faits. De même « je ne vous connais pas » (2) pour 
« je vous méprise » transpose le jugement de valeur en un jugement d'existence. 
\bigskip
D'autres  fois  une  hypothèse  transforme  en  situation  de  fait  un  jugement  de  valeur.  Le  leader 
catholique belge Schollaert s'écrie : 
\bigskip
Messieurs,  je  voudrais  pouvoir  conduire  une  femme  chrétienne  sur  une  montagne  assez  haute 
pour qu'elle pût, de là, embrasser d'un coup d'oeil toutes les femmes et tous les peuples de la terre. 
Là...  je  lui  dirais  :  «  Regardez,  madame,  et  après  avoir  regardé,  répondez-moi.  ...  Qui  vous  a  fait 
pure, belle, royale et supérieure à toutes les malheureuses soeurs qui s'agitent à vos pieds (3) ? ». 
\bigskip
(1) Quintilien, Vol. II, liv. VI, chap. III, § 74. 
(2) Exemples cités par Dumarsais, Des Tropes, p. 70. 
(3) Discours sur l'éducation des femmes à tous les degrés, 22-23 mars 1871, d'après E. Descamps, 
Etudes d'art oratoire et de législation, p. 40. 
\bigskip
P 246 : « La situation de fait imaginée entraîne une possibilité de vision plongeante qui suggère la 
supériorité de valeur. 
\bigskip
Enfin,  certaines  tournures  grammaticales,  telle  la  plume  nominale,  peuvent  être  utilisées  pour 
suggérer  le  statut  de  fait.  R.  Caillois,  notant  leur  fréquence  chez  St-John,  Perse,  y  voit  le  ton  de 
l'homme  avare  de  paroles,  aux  affirmations  incontestées  à  cause  de  leur  évidence  ou  de  son 
autorité (1). La phrase nominale est plutôt effort pour établir ce que l'on dit hors du temps et par là 
hors de la subjectivité, de la partialité. 
\bigskip
\bigskip
\bigskip
\bigskip
124 
\bigskip
le  roman  de  Jacques  Rivière,  Aimée, 
\bigskip
Il  arrive  pourtant  que  dans  la  présentation  des  prémisses  on  ait  intérêt  à  diminuer  le  statut  de 
certains objets d'accord. 
\bigskip
Pour minimiser la gravité d'une opposition à un fait, d'une entorse à la vérité, on transformera la 
négation  d'un  fait  en  jugement  d'appréciation.  Un  bel  exemple  de  cette  transposition  peut  être 
trouvé  chez  Browning,  où  l'évêque  Blougrain,  dans  son  apologie,  tente  de  diminuer  la  portée  de 
son incroyance 
\bigskip
Tout ce que nous avons gagné par notre incroyance  
C'est une vie de doute diversifiée par la foi,  
Pour une vie de foi diversifiée par le doute :  
Nous appelions l'échiquier blanc, - nous l'appelons noir (2). 
\bigskip
(1) H. Caillois, Poétique de St-John Perse, pp. 33-34.  
(2) Browning, Poems, Bishop Blougram's Apology, p. 140. 
All we have gained then by our unbelief 
Is a life of doubt diversified by, faith, 
For one of faith diversified by doubt 
We called the chess-board white, - we call it black, 
Cf. § 56 : La division du tout en ses parties. 
\bigskip
P  246-247 :  « Parfois  on  réduit  des  normes  à  n'être  que  des  caprices,  que  l'expression  d'un 
sentiment personnel : on tend à montrer, par la formulation, qu'on ne cherche pas à les imposer à 
autrui.  Dans 
l'amoureux  est  choqué  par certains 
comportements D’Aimée. Il les lui reproche, puis il s’en veut : 
\bigskip
D'où  étais-je  autorisé  à  faire  de  mes  goûts,  de  mes  jugements,  la  règle  qu'elle  devait  suivre  ? 
Pourquoi mes valeurs devaient-elles être préférées aux siennes ? (1). » 
\bigskip
(1) J. rivière, Aimée, p. 131. 
\bigskip
P  247 :  « En  traitant  ses  normes  de  «  mes  goûts  »  l'amant  excuse  Aimée,  il  se  défend  de  la 
condamner au nom de règles qu'elle n'a pas adoptées. 
\bigskip
Le cas le plus intéressant de transposition est celui où l'argumentation est volontairement réduite à 
des  jugements  de  valeur,  où  l'on  emploie  le  schème  inverse  de  celui  du  recours  du  pape  mal 
informé  au  pape  mieux  informé,  et  cela  pour  marquer  que  ce  sont  les  divergences  de  valeur  qui 
seules  importent,  que  c'est  sur  elles  que  le  débat  est  centré.  Ainsi  N.  Bobbio,  traitant  de  l'art  en 
régime totalitaire (2), refuse d'examiner si l'artiste est plus libre en Amérique ou en U. R. S. S., si la 
qualité esthétique des productions russes est satisfaisante on non, parce que ce sont là, selon lui, 
des  questions  de  fait,  irrelevantes  pour  la  controverse,  et  Bobbio  qualifie  de  fait  tout  ce  qui  ne 
concerne pas la valeur qui est en jeu - celle de la liberté. 
\bigskip
Il est assez rare que la volonté de réduire le débat à une question de valeurs soit aussi nette : cela 
implique,  en  effet,  une  technique  et  une  réflexion  sur  les  valeurs  qui  correspondent  à  des 
préoccupations  d'aujourd'hui.  Mais  il  arrive  souvent  que,  volontairement,  les  éléments  mis  à 
l'avant-plan ne soient que des valeurs. Un exemple célèbre en est le discours de Brutus à la foule 
dans Jules César de Shakespeare, d'oit est éliminé tout ce qui est étranger à la valeur de la liberté : 
\bigskip
Préféreriez-vous que César fût vivant et mourir tous esclaves, plutôt que de voir César mort et vivre 
tous en hommes libres (3). » 
\bigskip
(2) N. Bobbio, Libertà dell'arte e politica culturale, Nuovi argomenti, 1953, n. 2. 
\bigskip
\bigskip
\bigskip
125 
\bigskip
(3) Shakespeare, Julius Caesar, acte III, scène Il. 
Had you rather Caesar were living, and die all slaves, than that Caesar were dead, to live all free 
men ? 
\bigskip
P  248 :  « Le  discours  de  Brutus  a  souvent  été  considéré  comme  celui  d'un  froid  logicien,  par 
opposition à celui d'Antoine.  Et pourtant ce qui le caractérise c'est non l'élimination des valeurs, 
bien  au  contraire,  mais  la  volonté  marquée  de  transposer  le  débat  uniquement  sur  un  choix 
particulier. 
\bigskip
Ces quelques remarques sur le statut des objets d'accord et sur les modifications que la manière de 
se servir des données peut apporter à celui-ci, rejoignent ce que nous avons dit précédemment au 
sujet  de  la  solidité  et  de  la  précarité,  tout  à  la  fois,  des  points  d'appui  de  l'argumentation.  Notre 
description des objets d'accord laissait prévoir que ce n'est que dans un contexte complet que ceux-
ci peuvent être reconnus. Nous venons de voir que la forme sous laquelle ceux-ci sont exprimés, la 
façon  dont  un  débat  est  situé,  peuvent  réagir  sur  ce  statut.  Nous  avons  utilisé  volontairement  le 
terme  de  transposition,  lui-même  ambigu,  pour  marquer  que  l'on  peut  y  voir  soit  un  simple 
déplacement  d'accord,  soit  une  modification  profonde.  Selon  les  cas  et  selon  les  points  de  vue, 
l'une  ou  l'autre  interprétation  pourra  sembler  préférable.  Il  fallait  surtout,  semble-t-il,  souligner 
l'influence  de  ces  phénomènes  infiniment  complexes  de  transposition  sur  le  déroulement  de 
l'argumentation et son efficacité possible. » 
\bigskip
TROISIEME PARTIE : LES TECHNIQUES ARGUMENTATIVES 
\bigskip
§ 44. GENERALITES 
\bigskip
P  251 :  « Le  discours  persuasif  produit  des  effets  par  son  insertion,  comme  un  tout,  dans  une 
situation, elle-même le plus souvent assez complexe. Les différents éléments du discours étant en 
interaction,  l'ampleur  de  l'argumentation,  l'ordre  des  arguments  posent  des  problèmes  que  nous 
traiterons à la fin de notre étude. Mais avant d'examiner notre sujet sous cet aspect synthétique, il 
nous faut analyser la structure des arguments isolés. 
\bigskip
Cette  façon  de  faire,  indispensable  en  première  approximation,  nous  obligera  à  séparer  des 
articulations  qui  sont,  en  réalité,  partie  intégrante  d'un  même  discours  et  constituent  une  seule 
argumentation  d'ensemble.  Or  le  sens  et  la  portée  d'un  argument isolé  ne  peuvent  que  rarement 
être compris sans ambiguïté ; l'analyse d'un chaînon de l'argumentation, en dehors du contexte et 
indépendamment de la situation dans laquelle il s'insère, présente des dangers indéniables. Ceux-
ci  ne  sont  pas  dus  uniquement  au  caractère  équivoque  du  langage,  mais  encore  au  fait  que  les 
ressorts d'une argumentation ne sont presque jamais complètement explicités. 
\bigskip
Pour dégager un schème argumentatif, nous sommes obligés d'interpréter les paroles de l'orateur, 
de  suppléer  les  chaînons  manquants,  ce  qui  ne  va  jamais  sans  risque.  En  effet,  affirmer  que  la 
pensée réelle de l'orateur et de ses auditeurs est conforme au schème que nous venons de dégager, 
n'est  qu'une  hypothèse  plus  ou  moins  vraisemblable.  Le  plus  souvent  d'ailleurs  nous  percevons 
simultanément plus d'une façon de concevoir la structure d'un argument. » 
\bigskip
P 251-252 : « A cette objection s'en ajoute une autre, chaque fois que nos analyses concernent des 
arguments  empruntés,  non  à  des  discours  effectivement  prononces,  mais  à  des  textes  littéraires. 
Quelle garantie avons-nous, en effet, que les discours imaginés ne sont pas aussi éloignés du réel 
que  les  êtres  mythologiques  ?  Et,  en  fait,  le  caractère  artificiel  de  certains  discours  d'apparat  et 
exercices d'école que nous ont laissés les rhéteurs n'est pas douteux. » 
\bigskip
P  252 :  « Ces  deux  objections  seraient  certainement  difficiles  à  écarter,  d'une  part,  s'il  s'agissait 
d'analyse d'un discours particulier, analyse que l'on voudrait conforme à une réalité historique, et 
d'autre  part,  si  l'on  prétendait  proposer  comme  modèles  de  discours  persuasif,  ceux  qui, 
\bigskip
\bigskip
\bigskip
126 
\bigskip
effectivement, se sont avérés efficaces dans le passé. Mais notre propos est différent. Ce que nous 
désirons analyser dans les chapitres qui suivent, ce sont des schèmes d'arguments pour lesquels les 
cas particuliers examinés ne servent que d'exemples, que l'on aurait pu remplacer par mille autres. 
Nous les avons empruntés à des textes que nous croyons connaître suffisamment pour réduire le 
risque  d'incompréhension.  Nous  sommes  cependant  convaincus  que  ces  mêmes  énoncés 
argumentatifs  pourraient  être  autrement  analysés,  selon  d'autres  plans  de  clivage.  C'est  que  rien 
n'empêche de considérer un même énoncé comme susceptible de traduire plusieurs schèmes qui 
agiraient simultanément sur l'esprit de diverses personnes, voire sur un seul auditeur. Il se peut, 
au  surplus,  que  ces  schèmes  agissent  sans  être  clairement  perçus  et  que  seul  un  travail 
d'explicitation,  rarement  effectué,  permette  à  l'orateur  et  surtout  à  ses  auditeurs,  de  devenir 
conscients des schèmes intellectuels qu'ils utilisent ou dont ils subissent l'action. A  cet égard, les 
textes littéraires  - roman, théâtre, discours ont souvent l'avantage de  présenter les arguments de 
manière  simplifiée,  stylisée  ou  exagérée.  Situés  hors  d'un  contexte  réel  où  tous  les  éléments  de 
l'action  oratoire  se  confondent,  ils  apparaissent  avec  plus  de  netteté.  Nous  pouvons  être  par 
ailleurs assurés que si nous les reconnaissons comme arguments, c'est qu'ils correspondent bien à 
des structures familières. » 
\bigskip
P  253 :  « Nous  aurons  recours,  pour  éclairer  notre  analyse,  à  des  exemples  comiques.  Nous  ne 
croyons  pas  qu'une  étude  du  comique  dans  l'art  oratoire  relève  directement  de  notre  propos  - 
encore  que  le  comique  soit  un  élément  très  important,  pour  se  concilier  l'auditoire  ou  plus 
généralement  affirmer  une  communauté  entre  orateur  et  auditoire,  pour  effectuer  des 
dévaluations,  notamment  pour  ridiculiser  l'adversaire,  pour  opérer  les  diversions  opportunes. 
Mais notre intérêt ne portera pas tant sur le comique dans la rhétorique que sur le comique de la 
rhétorique.  Nous  entendons  par  là  l'utilisation  comique  de  certains  types  d'argumentation.  Si, 
comme nous le croyons, il existe un comique de la rhétorique, les éléments comiques peuvent nous 
aider  à  retrouver  certains  procédés  d'argumentation  qui,  sous  leur  forme  usuelle  et  banale  se 
laisseraient plus difficilement discerner. Tout procédé peut devenir aisément source  de comique ; 
les  procédés  rhétoriques  n'y  échappent  certainement  pas.  L'effet  comique,  dans  certains  cas,  ne 
proviendrait-il  pas,  précisément,  de  ce  que  l'on  songe  aux  procédés  habituels  de  raisonnement, 
caricaturés pour la circonstance, et de ce que l'on observe l'utilisation, hors de propos, ou abusive, 
ou maladroite, de tel schème argumentatif ? 
\bigskip
Dès l'abord aussi, il nous faut insister sur ce que le discours est un acte, qui, comme tout acte, peut, 
de la part de l'auditeur, faire l'objet d'une réflexion. 
\bigskip
Pendant  que  l'orateur  argumente,  l'auditeur,  à  son  tour,  sera  enclin  à  argumenter  spontanément 
au  sujet  de  ce  discours,  afin  de  prendre  attitude  à  son  égard,  de  déterminer  le  crédit  qu'il  doit  y 
attacher. L'auditeur qui perçoit les arguments, non seulement peut percevoir ceux-ci à sa manière, 
mais  est  en  outre  l'auteur  de  nouveaux  arguments  spontanés,  le  plus  souvent  non  exprimés,  qui 
n'en interviendront pas moins pour modifier le résultat final de l'argumentation. » 
\bigskip
P 253-254 : « Il peut se faire d'ailleurs que cette réflexion soit orientée par l'orateur, que celui-ci 
fournisse  lui-même  aux  auditeurs  certains  arguments  portant  sur  les  caractères  de  son  propre 
énoncé,  ou  bien  encore  qu'il  fournisse  certains  éléments  d'information  qui  favoriseront  telle  ou 
telle  argumentation  spontanée  de  l'auditeur.  Ces  arguments  qui  prennent  le  discours  pour  objet, 
ces éléments d'information aptes à les susciter, peuvent aussi émaner de tiers : de l'adversaire de 
l'orateur, notamment dans le débat judiciaire, on peut-être aussi d'un simple spectateur. » 
\bigskip
P  254 :  « En  principe  tous  les  schèmes  argumentatifs  que  nous  rencontrerons  peuvent  donc 
s'appliquer  au  discours  lui-même.  Nous  serons  amenés  à  le  montrer  dans  certains  cas  d'une 
manière assez approfondie, notamment en ce qui concerne les arguments basés sur le rapport de la 
personne  de  l'orateur  à  son  discours,  et  en  ce  qui  concerne  la  considération  du  discours  comme 
procédé oratoire. Mais ce ne sont là que des cas éminents parmi ceux où l'argumentation ayant le 
\bigskip
\bigskip
\bigskip
127 
\bigskip
discours  pour  objet  se  superpose  à  l'argumentation  proprement  dite  de  l'orateur.  On  pourrait 
certainement, pour chaque type d'arguments, tenter une semblable étude. Il est indispensable que, 
en tout état de cause, cette réflexion sur le discours ne soit jamais perdue de vue. » 
\bigskip
P 254-255 : « Les plans sur lesquels cette réflexion se situe seront d'ailleurs très divers. Elle pourra 
envisager  le  discours  comme  acte,  comme  indice,  comme  moyen  ;  elle  pourra  se  référer 
uniquement  à  son  contenu,  ou  ne  négliger  aucun  des  facteurs  qui  le  constituent.  Elle  pourra 
notamment  se  référer  au  langage  utilisé  :  tandis  que  l'orateur  décrit  ce  qu'il  a  «  vu  »,  l'auditeur 
songera  peut-être  à  la  signification  psychologique  ou  physiologique  de  la  vision  ;  il  pourra  aussi, 
avec Ryle, noter que le verbe « voir» n'est pas un verbe qui indique un processus ou un état mais 
un  résultat  (1).  Normalement,  ces  réflexions  sur  le  langage  n'auront  pas  de  retentissement  sur 
l'effet  du  discours,  parce  que  celui-ci  visera un  plan  où  elles  sont  irrelevantes  ;  mais  ce  n'est  pas 
toujours le cas. Notons d'ailleurs que ces considérations peuvent être le fruit d'idées personnelles 
ou d'idées suggérées par un théoricien. Mais ce dernier ne prétend le plus souvent que mettre en 
évidence ce qui est la conscience verbale du commun des hommes (1). » 
\bigskip
(1) G. Ryle, Dilemmas, p. 102. 
(1) Cf. Réflexions de Wittgenstein, dans Philosophische 
\bigskip
P  255 :  « C'est  en  tenant'  compte  de  cette  superposition  d'arguments  (lue  l'on  parviendra  à 
s'expliquer  le  mieux  l'effet  pratique,  effectif,  de  l'argumentation.  Toute  analyse  qui  la  négligerait 
serait, pensons-nous, vouée à l'échec. Contrairement à ce qui se passe dans une démonstration où 
les procédés démonstratifs jouent à l'intérieur d'un système isolé, l'argumentation se caractérise en 
effet par une interaction constante entre tous ses éléments. Sans doute, la démonstration logique 
elle-même peut-elle être objet d'attention de la part de l'auditeur : celui-ci admirera son élégance, 
déplorera  sa  lourdeur,  constatera  son  adéquation  au  but  à  poursuivre.  Mais  cette  argumentation 
qui  prend  la  démonstration  pour  objet  ne  sera  point  elle-même  démonstration.  Elle  ne  se 
superposera pas à la démonstration pour en modifier la validité. Elle se développera sur un plan 
argumentatif où nous retrouverons précisément les arguments rhétoriques que nous analysons. 
\bigskip
Les schèmes que nous chercherons à dégager  - et que l'on peut aussi considérer comme des lieux 
de l'argumentation, parce que seul l'accord sur leur valeur peut justifier leur application à des cas 
particuliers - se caractérisent par des procédés de liaison et de dissociation. » 
\bigskip
P 255-256 : « Nous entendons par procédés de liaison des schèmes qui rapprochent des éléments 
distincts et permettent d'établir entre ces derniers une solidarité visant soit à les structurer, soit à 
les  valoriser  positivement  ou  négativement  l'un  par  l'autre.  Nous  entendons  par  procédés  de 
dissociation  des  techniques  de  rupture  ayant  pour  but  de  dissocier,  de  séparer,  de  désolidariser, 
des éléments considérés comme formant un tout ou du moins un ensemble solidaire au sein d'un 
même système de pensée : la dissociation aura pour effet de modifier pareil système en modifiant 
certaines  des  notions  qui  en  constituent  des  pièces  maîtresses.  C'est  par  là  que  ces  procédés  de 
dissociation sont caractéristiques de toute pensée philosophique originale. » 
\bigskip
P  256 :  « Psychologiquement  et  logiquement  toute  liaison  implique  une  dissociation  et 
inversement : la même forme qui nuit des éléments divers en un tout bien structuré les dissocie du 
fond neutre dont elle les détache. Les deux techniques sont complémentaires et toujours à l'œuvre  
en même temps ; mais l'argumentation grâce à laquelle le donné est modifié peut mettre l'accent 
sur  la  liaison  ou  la  dissociation  qu'elle  est  en  train  de  favoriser,  sans  expliciter  l'aspect 
complémentaire  qui  résultera  de  la  transformation  recherchée.  Parfois  les  deux  aspects  sont 
simultanément  présents  à  la  conscience  de  l'orateur  qui  se  demandera  sur  lequel  il  vaut  mieux 
attirer l'attention. 
\bigskip
\bigskip
\bigskip
\bigskip
128 
\bigskip
D'autre  part,  ce  qui  est  donné  avant  l'argumentation  peut  paraître  plus  solidement  établi  que  ce 
qui  résulte  uniquement  de  cette  dernière  :  faut-il  lier  des  éléments  séparés  ou  faut-il  déjà  les 
présenter  comme  formant  un  tout  ?  Un  texte  caractéristique  de  Bossuet  fera  mieux  comprendre 
notre pensée, et les problèmes que ces questions posent à l'orateur : 
\bigskip
Dans le dessein que j'ai pris de f aire tout l'entretien de cette semaine sur la triste aventure de ce 
misérable, je m'étais d'abord proposé de donner comme deux tableaux, dont l'un représenterait sa 
mauvaise vie, et l'autre sa fin malheureuse; mais j'ai cru que les pécheurs, toujours favorables à ce 
qui  éloigne  leur  conversion,  si  je  faisais  ce  partage,  se  persuaderaient  trop  facilement  qu'ils 
pourraient aussi détacher ces choses qui ne sont pour notre malheur que trop enchaînées... (1) » 
\bigskip
(1) Bossuet, Sermons, vol. II : Sur l'impénitence finale, pp. 221-222. 
\bigskip
P  257 :  « Rejetant  l'idée  qui  lui  était  venue  à  l'esprit,  de  les  rendre  solidaires  au  moyen  d'une 
liaison, Bossuet présentera la vie et la mort du pécheur comme formant une unité indissoluble : 
\bigskip
La mort, dira-t-il, n'a pas un être distinct qui la sépare de la vie; mais elle n'est autre chose sinon 
une vie qui s'achève. 
\bigskip
S'il est donc toujours loisible de traiter un même argument comme constituant, à un certain point 
de vue, une liaison, et à un autre point de vue, une dissociation, il est utile pourtant d'examiner des 
schèmes argumentatifs de l'une ou de l'autre espèce. 
\bigskip
Nous  analyserons  successivement,  en  tant  que  schèmes  de  liaison,  les  arguments  quasi  logiques, 
que l'on comprend le mieux en les rapprochant de la pensée formelle; les arguments basés sur la 
structure du réel, qui sont présentés comme conformes à la nature même des choses. Notons que 
la distinction entre ces deux groupes de raisonnements pourrait être rapprochée de la distinction 
husserlienne entre l'abstraction formalisatrice et l'abstraction généralisatrice, de la distinction de 
Piaget entre schémas nés à partir des opérations et schémas nés à partir des choses, de la double 
thématisation perceptive de Gurwitsch (1). Mais toutes ces distinctions ont un aspect génétique qui 
reste étranger à nos préoccupations. 
\bigskip
Nous examinerons ensuite les arguments qui visent à fonder la structure du réel : les arguments 
faisant  état  du  cas  particulier,  les  arguments  d'analogie  qui  s'efforcent  de  restructurer  certains 
éléments de la pensée conformément à des schèmes admis dans d'autres domaines du réel. » 
\bigskip
(1) Cf. A. Gunwitsch, Actes du XI- Congrès international de Philosophie, vol. II, pp. 43-47. 
\bigskip
P 257-258 : « Et enfin, nous consacrerons tout un chapitre aux techniques de dissociation, qui se 
caractérisent  surtout  par  les  remaniements  qu'elles  introduisent  dans  les  notions,  parce  qu'elles 
visent moins à utiliser le langage admis qu'à procéder à un modelage nouveau. Il ne faut pas croire 
que ces groupes de schèmes argumentatifs constituent des entités isolées. Nous sommes souvent 
autorisés, avons-nous dit, à interpréter un raisonnement suivant l'un ou l'autre schème. Mais, qui 
plus  est,  nous  pouvons  considérer  que  certains  arguments  appartiennent  à  l'un  aussi  bien  qu'à 
l'autre, de ces groupes de schèmes. Un énoncé tel « si le monde est régi par une providence, l'État 
demande  un  gouvernement  »  que  Quintilien  traite  comme  un  «  argument  de  voisinage  ou  de 
comparaison » (1) peut être considéré comme quasi logique (ce qui vaut pour le tout vaut pour la 
partie) ou comme analogie, voire comme basé sur des liaisons de coexistence. » 
\bigskip
(1) Quintilien, Vol. II, liv. V, chap. X, § 89. 
\bigskip
P 258 : « Nous pourrions même, avec quelque apparence de raison, ramener tous les groupes de 
schèmes  à  l'un  d'eux  qui  serait  considéré  comme  fondamental,  comme  sous-jacent  à  tous  les 
\bigskip
\bigskip
\bigskip
129 
\bigskip
autres. Mais ce serait déformer les premiers résultats de notre analyse au profit d'une conception 
préconçue.  Aussi  envisagerons-nous  successivement  les  divers  groupes  d'arguments  sous  leurs 
formes les plus caractérisées. » 
\bigskip
CHAPITRE, PREMIER : LES ARGUMENTS QUASI LOGIQUES 
\bigskip
§ 45 CARACTERISTIQUES DE L'ARGUMENTATION QUASI LOGIQUE 
\bigskip
P 259 : « Les arguments que nous allons examiner dans ce chapitre prétendent à une certaine force 
de  conviction,  dans  la  mesure  où  ils  se  présentent  comme  comparables  à  des  raisonnements 
formels, logiques ou mathématiques. Pourtant, celui qui les soumet à l'analyse perçoit aussitôt les 
différences  entre  ces  argumentations  et  les  démonstrations  formelles,  car  seul  un  effort  de 
réduction  ou  de  précision,  de  nature  non-formelle,  permet  de  donner  à  ces  arguments  une 
apparence démonstrative; c'est la raison pour laquelle nous les qualifions de quasi logiques. 
\bigskip
Dans  tout  argument  quasi  logique  il y  a  lieu  de  mettre  en  évidence,  d'abord,  le  schème  formel  à 
l'instar  duquel  l'argument  est  construit  et,  ensuite,  les  opérations  de  réduction  qui  permettent 
d'insérer  les  données  dans  ce  schème,  et  qui  visent  à  les  rendre  comparables,  semblables, 
homogènes. » 
\bigskip
P  259-260 :  « Notre  technique  d'analyse  peut  sembler  donner  le  primat  au  raisonnement  formel 
sur l'argumentation qui n'en serait qu'une forme approchée et imparfaite. Ce n'est pourtant pas là 
notre pensée. Tout au contraire, nous croyons que le raisonnement formel résulte d'un processus 
de simplification qui n'est possible que dans des conditions particulières, à l'intérieur de systèmes 
isolés et circonscrits. Mais étant donné l'existence admise de démonstrations formelles, de validité 
reconnue,  les  arguments  quasi  logiques  tirent  actuellement  leur  force  persuasive  de  leur 
rapprochement avec ces modes de raisonnement incontestés. » 
\bigskip
P 260 : « Ce qui caractérise l'argumentation quasi logique, c'est donc son caractère non-formel, et 
l'effort  de  pensée  que  nécessite  sa  réduction  au  formel.  C'est  sur  ce  dernier  aspect  que  portera 
éventuellement  la  controverse.  Quand  il  s'agira  de  justifier  telle  réduction  qui  n'aura  pas  paru 
convaincante par la seule présentation des éléments du discours, on aura recours le plus souvent à 
d'autres formes d'argumentation que les arguments quasi logiques. 
\bigskip
L'argumentation quasi logique se présentera d'une façon plus ou moins explicite. Tantôt l'orateur 
désignera  les  raisonnements  formels  auxquels  il  se  réfère,  se  prévalant  du  prestige  de  la  pensée 
logique,  tantôt  ceux-ci  ne  constitueront  qu'une  trame  sous-jacente.  Il  n'y  a  d'ailleurs  pas 
corrélation nécessaire entre le degré  d'explicitation des schèmes formels auxquels on se réfère et 
l'importance des réductions exigées pour y soumettre l'argumentation. 
\bigskip
Celui qui critique un argument aura tendance à prétendre que ce qu'il a en face de lui relève de la 
logique; 
logique  est,  elle-même,  souvent,  une 
argumentation  quasi  logique.  On  se  prévaut,  par  cette  accusation,  du  prestige  du  raisonnement 
rigoureux.  Cette  accusation  pourra  être  précise  (accusation  de  contradiction  par  exemple)  et  se 
situer au niveau même de l'argumentation. Elle pourra aussi être générale (accusation de tenir un 
discours  passionnel  au  lieu  d'un  discours  logique).  Dans  ce  cas,  l'auditeur  met  en  opposition  le 
discours  entendu  avec  l'image  d'un  discours  qui  lui  paraît  supérieur  et  qui  serait  composé  de 
schèmes logiques auxquels le donné serait ramené. » 
\bigskip
\bigskip
l'accusation  de  commettre  une  faute  de 
\bigskip
P  260-261 :  « Les  réductions  exigées  pour  soumettre  l'argumentation  aux  schèmes  formels 
concernent tantôt les termes du discours, que l'on traite comme des entités homogènes, tantôt les 
structures  que  l'on  assimile  à  des  relations  logiques  ou  mathématiques,  ces  deux  aspects  de  la 
réduction étant d'ailleurs liés. » 
\bigskip
\bigskip
\bigskip
\bigskip
\bigskip
130 
\bigskip
P  261 :  « Nous  analyserons,  parmi  les  arguments  quasi  logiques,  en  premier  lieu  ceux  qui  font 
appel à des structures logiques - contradiction, identité totale on partielle, transitivité; en second 
lieu ceux qui font appel à des relations mathématiques - rapport de la partie au tout, du plus petit 
au  plus  grand,  rapport  de  fréquence.  Bien  d'autres  relations  pourraient  évidemment  être 
examinées. 
\bigskip
Répétons,  à  ce  propos,  qu'un  même  argument  peut  être  compris  et  analysé  différemment  par 
différents  auditeurs  et  que  les  structures  logiques  pourraient  être  considérées  comme  mathé-
matiques et inversement. En outre presque toute argumentation quasi logique utilise aussi d'autres 
types d'arguments qui peuvent paraître, à d'aucuns,  prépondérants. Les exemples que nous don-
nerons ici sont analysés comme argumentation quasi logique parce que cet aspect y est aisément 
décelable. 
\bigskip
On  est  surpris,  à  cet  égard,  de  ce  que  l'argumentation  quasi  logique  explicitement  basée  sur  des 
structures  mathématiques,  ait  été  beaucoup  plus  en  honneur  autrefois,  et  spécialement  chez  les 
Anciens,  qu'elle  ne  l'est  aujourd'hui.  De  même  que  le  développement  de  la  logique  formalisée  a 
permis de séparer la démonstration de l'argumentation, de même le développement des sciences a 
sans  doute  contribué  à  leur  réserver  l'usage  du  calcul  et  de  la  mesure  en  montrant  mieux  les 
conditions exigées pour leur application. Ajoutons que, aux périodes où prédominent les lieux de la 
quantité,  l'emploi  des  relations  mathématiques  est  sans  doute  favorisé  et  que  la  pensée  antique 
classificatoire  est  toute  géométrique.  Quoi  qu'il  en  soit,  les  arguments  quasi  logiques  étaient 
autrefois  développés  avec  une  sorte  de  joie,  de  virtuosité,  qui  en  mettent  les  modalités  bien  en 
évidence. » 
\bigskip
§ 46. CONTRADICTION ET INCOMPATIBILITE 
\bigskip
P 262 : « L'assertion, au sein d'un même système, d'une proposition et de sa négation, en rendant 
manifeste  une  contradiction  qu'il  contient,  rend  le  système  incohérent,  et  par  là  inutilisable. 
Mettre  a  jour  l'incohérence  d'un  ensemble  de  propositions,  c'est  l'exposer  à  une  condamnation 
sans  appel,  obliger  celui  qui  ne  veut  pas  être  qualifié  d'absurde  à  renoncer  au  moins  à  certains 
éléments du système. 
\bigskip
Quand les énoncés sont parfaitement univoques, comme dans des systèmes formels, oh les seuls 
signes  suffisent,  par  leur  combinaison,  à  rendre  la  contradiction  indiscutable,  on  ne  peut  que 
s'incliner devant l'évidence. Mais cela n'est pas le cas quand il s'agit d'énoncés du langage naturel, 
dont  les  termes  peuvent  être  interprétés  de  différentes  façons.  Normalement,  quand  quelquun 
soutient  simultanément  une  proposition  et  sa  négation,  nous  pensons  qu'il  ne  désire  pas  dire 
quelque  chose  d'absurde,  et  nous  nous  demandons  comment  il  faut  interpréter  ce  qu'il  dit  pour 
éviter  l'incohérence.  En  effet,  il  est  rare  que  le  langage  utilisé  dans  l'argumentation  puisse  être 
considéré comme entièrement univoque tel celui d'un système formalisé. La contradiction logique, 
discernable d'une façon purement formelle, fait corps avec le système, et est indépendante de notre 
volonté et des contingences, car elle est inéluctable dans le cadre des conventions admises. Il n'en 
est pas ainsi dans l'argumentation, où les prémisses ne sont que rarement entièrement explicitées 
et,  quand  elles  le  sont,  rarement  définies  d'une  façon  entièrement  univoque;  le  champ  et  les 
conditions  d'application  y  varient  avec  les  circonstances,  dont  font  partie  d'ailleurs  les  décisions 
elles-mêmes des participants au débat. » 
\bigskip
P  262-263 :  « Toutes  ces  raisons  font  que,  sauf  dans  des  cas  tout  à  fait  exceptionnels  -  quand  il 
arrive  à  l'orateur  d'emprunter  quelques  chaînons  de  son  raisonnement  à  un  système  formel  -  il 
n'est  pas  l'une  contradiction,  dans  le  Système  de  permis  de  faire  état  l'adversaire.  D'habitude 
l'argumentation  s'efforcera  de  montrer  que  les  thèses  que  l'on  combat  mènent  à  une 
incompatibilité, qui ressemble en ceci à une contradiction, qu'elle consiste en deux assertions entre 
lesquelles  il  faut  choisir,  à  moins  de  renoncer  à  l'une  et  à  l'autre.  Les  thèses  incompatibles  ne  le 
sont pas pour des raisons purement formelles comme des assertions contradictoires. Quoique l'on 
\bigskip
\bigskip
\bigskip
131 
\bigskip
les  tiers,  qui  sont 
\bigskip
s'efforce souvent de la présenter comme conforme à la raison ou à la logique, c'est-à-dire comme 
nécessaire,  l'incompatibilité  dépend  soit  de  la  nature  des  choses,  soit  d'une  décision  humaine. 
Aussi, un  des  moyens  de  défense qui sera  opposé  à  l'argumentation  quasi  logique  faisant  état  de 
contradictions sera de montrer qu'il s'agit non de contradiction, mais d'incompatibilité, c'est-à-dire 
que l'on mettra en évidence la réduction qui seule a permis l'assimilation à un système formel du 
système attaqué, lequel est loin de présenter, en fait, la même rigidité. «  
\bigskip
P  263 :  « Le  cas  où  l'incompatibilité  dépend  d'une  décision  personnelle  paraît  le  plus  éloigné  de 
celui de la contradiction formelle, parce que, au lieu de s'imposer, cette incompatibilité est posée, 
et  que  l'on  peut  espérer  qu'une  décision  nouvelle  la  lèvera  éventuellement.  Le  chef  de 
gouvernement  qui  pose  la  question  de  confiance,  à  propos  d'un  problème  particulier,  crée  une 
incompatibilité  entre  son  maintien  en  fonctions,  et  le  rejet  de  la  solution  qu'il  préconise.  Un 
ultimatum  crée  une  incompatibilité  entre  le  refus  de  céder  et  le  maintien  de  la  paix  entre  deux 
États.  Les  dirigeants  d'un  groupe  peuvent  décider,  ou  constater  à  un  moment  donné,  qu'il  y  a 
incompatibilité  entre  l'appartenance  à  leur  groupe  et  celle  à  un  autre  groupe,  alors  que  les 
dirigeants de ce dernier peuvent ne pas s'en aviser ou affirmer le contraire. » 
\bigskip
P  263-264 :  « A  certains  points  de  vue,  il  est  donc  possible  de  décider  de  l'existence  d'une 
incompatibilité,  mais  pour 
incapables  de  modifier  cette  décision, 
l'incompatibilité posée peut avoir un aspect objectif, dont il faut tenir compte, comme d'une loi de 
la  nature.  Vouloir  ignorer  cette  obligation  où  l'on  se  trouve  de  choisir  peut  conduire  à  de  graves 
mécomptes. Comme le dit joliment La Bruyère : 
\bigskip
La neutralité entre des femmes qui nous sont également amies, quoiqu'elles aient rompu pour des 
intérêts où nous n'avons nulle part, est un point difficile : il faut choisir souvent entr'elles, ou les 
perdre toutes deux (1). » 
\bigskip
(1) La Bruyère, Œuvres Bibl. de la Pléiade, Caractères, Des femmes, 50, p. 142.  
\bigskip
La  neutralité  entre  États,  en  temps  de  guerre,  ou  de  forte  tension,  n'est  pas  moins  difficile  à 
observer.  Comme  l'a  remarqué  E.  Dupréel,  dans  son  chapitre  relatif  à  la  logique  des  conflits  :  « 
Tout différend tend à s'étendre aux tiers, qui le développent en prenant parti (2). » 
\bigskip
Des  incompatibilités  peuvent  résulter  de  l'application  à  des  situations  déterminées  de  plusieurs 
règles  morales  ou  juridiques,  de  textes  légaux  ou  sacrés.  Alors  que  la  contradiction  entre  deux 
propositions  suppose  un  formalisme  ou  du  moins  un  système  de  notions  univoques, 
l'incompatibilité  est  toujours  relative  à  des  circonstances  contingentes,  que  celles-ci  soient 
constituées  par  des  lois  naturelles,  des  événements  particuliers  ou  des  décisions  humaines.  C'est 
ainsi que, selon William Pitt, l'adoption d'une certaine motion rendrait incompatibles deux aspects 
de la paix souhaitée : 
\bigskip
... les qualificatifs « prompte et honorable » deviennent alors incompatibles. Nous devons dans ce 
cas choisir l'un des termes de l'alternative ; si nous adoptons la motion, nous ne pouvons avoir une 
paix « prompte et honorable » (3). 
\bigskip
(2) E. Dupréel, sociologie générale, p. 143. 
(3) William Pitt, Orations on the French war, p. 116 (15 février 1796). 
\bigskip
§ 47. PROCEDES PERMETTANT D'EVITER UNE INCOMPATIBILITE 
\bigskip
P 264-265 : « Les incompatibilités obligent à un choix qui est toujours pénible. Il faudra sacrifier 
l'une  des  deux  règles,  l'une  des  deux  valeurs  -  à  moins  de  renoncer  aux  deux,  ce  qui  entraîne 
souvent de nouvelles incompatibilités - ou bien il faut recourir à des techniques variées permettant 
de lever les incompatibilités et que nous pourrons qualifier de compromis, au sens le plus large du 
\bigskip
\bigskip
\bigskip
132 
\bigskip
terme,  mais  qui  le  plus  souvent  entraînent  aussi  un  sacrifice.  Aussi  la  vie  nous  offre-t-elle  des 
exemples  nombreux  et  importants  de  comportement  visant  essentiellement  lion  pas  à  lever  une 
incompatibilité  entre  deux  règles,  ou  entre  une  conduite  et  une  règle,  mais  à  éviter  que  cette 
incompatibilité puisse se présenter. » 
\bigskip
P  265 :  « Comme  les  incompatibilités  ne  sont  pas  formelles,  mais  n'existent  qu'en.  égard  à 
certaines situations, l'on comprend que trois attitudes fort différentes puissent être adoptées dans 
la façon de traiter les problèmes que cette confrontation des règles et des situations peut poser au 
théoricien et à l'homme d'action. » 
\bigskip
P  265-266 :  « La  première,  que  l'on  pourrait  appeler  logique,  est  celle  où  l'on  se  préoccupe,  à 
l'avance,  de  résoudre  toutes  les  difficultés  et  tous  les  problèmes  qui  peuvent  surgir,  dans  les 
situations les plus variées, que l'on s'efforce d'imaginer, suite à l'application de règles, de lois et de 
normes auxquelles on accorde son adhésion. C'est normalement l'attitude du savant qui s'efforce 
de  formuler  des  lois  qui  lui  semblent  régir  le  domaine  qu'il  étudie,  et  dont  il  voudrait  qu'elles 
rendent compte de tous les phénomènes susceptibles de s'y produire. C'est aussi l'attitude normale 
de celui qui élabore une doctrine juridique ou éthique et qui se propose de résoudre, si pas tous les 
cas  d'application,  du  moins  le  plus  grand  nombre  de  ceux  dont,  en  pratique,  on  pourrait  avoir  à 
s'occuper.  Celui  qui,  dans  la  conduite  de  sa  vie,  imitera  les  théoriciens  auxquels  nous  venons  de 
faire  allusion,  sera  traité  d'homme  logique,  dans  le  sens  dans  lequel  on  dit  que  les  Français  sont 
logiques et les Anglais pratiques et réalistes. L'attitude logique suppose que l'on parvient à clarifier 
suffisamment les notions dont on se sert, à préciser suffisamment les règles que l'on admet, pour 
que les problèmes pratiques puissent être résolus sans difficulté par voie de simple déduction. Cela 
implique d'ailleurs que l'imprévu a été éliminé, que l'avenir a été maîtrisé, que tous les problèmes 
sont devenus soluble techniquement. » 
\bigskip
P 266 : « A cette attitude s'oppose celle de l'homme pratique, qui ne résout les problèmes qu'au fur 
et  à  mesure  qu'ils  se  présentent,  qui  repense  ses  notions  et  ses  règles  en  fonction  des  situations 
réelles et des décisions indispensables à son action. Ce sera, contrairement à celle des théoriciens, 
l'attitude des hommes de la pratique, qui ne désirent pas s'engager plus qu'il ne faut, qui désirent 
se  laisser,  aussi  longtemps  que  possible,  toute  la  liberté  d'action  que  les  circonstances  leur 
permettent, qui désirent pouvoir s'adapter à l'imprévu et à l'expérience future. C'est normalement 
l'attitude  du  juge  qui,  sachant  que  chacune  de  ses  décisions  constitue  un  précédent,  cherche  à 
limiter leur portée autant qu'il le peut, à les énoncer sans  dépasser dans ses attendus ce qu'il est 
nécessaire  de  dire  pour  fonder  sa  décision,  sans  étendre  ses  formules  interprétatives  à  des 
situations dont la complexité pourrait lui échapper. 
\bigskip
La troisième enfin des attitudes, que nous qualifierons de diplomatique, en pensant à l'expression 
«  maladie  diplomatique  »,  est  celle  où  ne  désirant  pas,  du  moins  à  un  moment  et  dans  des 
circonstances  déterminées,  se  mettre  en  opposition  avec  une  règle  ou  résoudre,  d'une  façon  on 
d'une autre, le conflit né de l'incompatibilité entre deux règles pouvant s'appliquer à une situation 
particulière, on invente des procédés pour éviter que l'incompatibilité apparaisse, on pour remettre 
à un moment plus opportun les décisions à prendre. En voici quelques exemples. » 
\bigskip
P 266-267 : « Proust nous rappelle, après Saint-Simon, de quels subterfuges les nobles se servaient 
pour  éviter  de  résoudre  de  délicats  problèmes  de  préséance  qu'aucune  tradition  établie  ne 
permettait de trancher d'une façon satisfaisante : 
\bigskip
Dans certains cas, devant l'impossibilité d'arriver à une entente, on préfère convenir que le fils de 
Louis XIV, Monseigneur, ne recevra chez lui tel souverain étranger que dehors, en plein air, pour 
qu'il  ne  soit  pas  dit  qu'en  entrant  dans  le  château  l'un  a  précédé  l'autre;  et  l'Electeur  palatin, 
recevant le duc de Chevreuse à dîner, feint, pour ne pas lui laisser la main, d'être malade et dîne 
avec lui mais couché, ce qui tranche la difficulté (1). » 
\bigskip
\bigskip
\bigskip
133 
\bigskip
 
(1) M. Proust, Œuvres complètes, vol. 8 : Le côté de Guermantes, III, p. 70. 
\bigskip
P 267 : « Il est de règle, au japon, de ne recevoir ses visiteurs qu'en habits décents. Si le fermier est 
surpris  dans  son  travail  par  un  visiteur  inattendu,  l'arrivant  fera  semblant  de  ne  pas  le  voir, 
jusqu'au moment où il aura changé de vêtements, ce qui pourra se faire dans la chambre même où 
le visiteur attend (2). 
\bigskip
L'on  voit,  dans  ce  cas,  comme  dans  le  précédent,  quel  rôle  joue  la  fiction  comme  technique 
permettant  d'éviter  une  incompatibilité.  La  fiction  est  un  procédé  consistant  dans  une  feinte, 
admise  par  les  parties,  les  convenances,  ou  le  système  social,  qui  permet  de  se  conduire,  et 
spécialement  de  raisonner,  comme  si  certains  faits  s'étaient  ou  ne  s'étaient  pas  produits, 
contrairement  à  la  réalité.  Quand  la  feinte  n'est  qu'unilatérale,  nous  avons  affaire  au  mensonge. 
Ceux qui évitent de prendre des décisions désagréables sont souvent obligés de mentir aux autres, 
et de se mentir à eux-mêmes. Parfois se taire n'a d'autre but qu'éviter une décision relative à une 
incompatibilité. Citons encore une fois Proust : 
\bigskip
Tenez, savez-vous, Madame [dit le due de Guermantes à la princesse de Parme], j'ai bien envie de 
ne  pas  même  dire  à  Oriane  que  vous  m'avez  parlé  de  Mme  de  Souvré.  Oriane  aime  tant  votre 
Altesse, qu'elle ira aussitôt inviter Mme de Souvré, ce sera une visite de Plus, etc. (3). 
\bigskip
Le  due,  en  feignant  de  ne  pas  dire  à  sa  femme  que  la  princesse  de  Parme  a  parlé  de  Mme  de 
Souvré, évite une incompatibilité ; il parlera effectivement sans doute de cette démarche, mais il 
dispense  sa  femme  de  devoir  choisir  entre  son  aversion  pour  Mme  de  Souvré,  et  sa  déférence 
envers la princesse de Parme. » 
\bigskip
(2) R. Benedict, The Chrysanthemum and the Sword, p. 156. 
(3) M. Proust, Le côté de Guermantes, 111, p. 90, 
\bigskip
P 267-268 : « La fiction, le mensonge, le silence, servent à éviter une incompatibilité sur le plan de 
l'action, pour ne pas devoir la résoudre sur le plan théorique. L'hypocrite fait semblant d'adopter 
une règle de conduite conforme à celle des autres pour éviter de devoir justifier une conduite qu'il 
préfère et qu'il adopte en réalité. On  a souvent dit  que l'hypocrisie était un hommage que le vice 
rend à la vertu: il faudrait préciser que l'hypocrisie est un hommage à une valeur déterminée, celle 
que l'on sacrifie, tout en feignant de la suivre, parce que l'on refuse de la confronter avec d'autres 
valeurs. L'incompatibilité est ainsi levée dans l'action, mais c'est évidemment au prix de nouvelles 
incompatibilités, celle entre une conduite hypocrite et une conduite franche et sincère, celle entre 
une  pensée  plus  ou  moins  systématisée  et  une  pensée  qui  se  dispense  de  chercher  des  solutions 
défendables. On pourrait rappeler ici le rapprochement que fait V. Jankélévitch entre l'aumône et 
le  mensonge;  «  l'aumône,  comme  le  mensonge,  recule  le  problème  sans  le  résoudre  ;  ajourne  la 
difficulté  en  l'alourdissant  »  (1).  Cette  dernière  remarque  nous  paraît  évidente,  toutefois,  il  faut 
bien se rendre compte que c'est de difficultés nouvelles qu'il s'agit : on sait le poids que représente 
pour le menteur le maintien de la cohérence de son univers fictif. Le problème actuel a, lui, bel et 
bien  été  résolu.  A  ce  titre,  le  mensonge  ne  se  distingue  guère  de  toutes  les  solutions  que  nous 
rencontrerons : elles aussi posent de nouveaux problèmes, mais dont la solution peut n'être point 
aussi urgente que l'était celle du problème résolu. » 
\bigskip
(1) V. Jankélévitch, Traité des vertus, p. 435. 
\bigskip
P  268-269 :  « Alors  que  l'hypocrisie  consiste  à  laisser  croire  que  l'on  adopte  une  conduite 
conforme à celle que l'on attendait de vous, c'est-à-dire à laisser croire que l'on a tranché dans un 
certain sens, d'autres techniques au contraire consistent à laisser croire que l'on n'a pas tranché. La 
maladie  diplomatique  peut  servir  à  éviter  de  prendre  certaines  décisions,  mais  elle  sert  aussi  à 
\bigskip
\bigskip
\bigskip
134 
\bigskip
masquer le fait qu'une décision a été prise : décidé à ne pas se rendre à telle réception, l'intéressé 
feint  d'être  dans  l'incapacité  -  pour  cause  de  maladie,  d'absence  -  de  choisir  si  oui  ou  non  il  s'y 
rendra. » 
\bigskip
P  269 :  « Sartre  a  développé  une  théorie  de  la  mauvaise  foi,  comme  étant  «  un  certain  art  de 
former des concepts contradictoires » (1). Ces concepts « unissent en eux une idée et la négation de 
cette  idée  ».  Il  ressort  assez  el  airement  des  exemples  qu'il  donne,  que  l'on  n'est  pas  dans  le 
domaine  du  contradictoire  et  que  la  mauvaise  foi  de  Sartre  c'est  le  refus  de  reconnaître  des 
incompatibilités : témoin l'exemple de la femme à qui on dit des paroles spiritualistes et à qui on 
prend  la  main.  Partant  de  ce  refus,  Sartre  développera  une  conception  de  la  mauvaise  foi  qui 
s'applique  à  la  conviction  elle-même  (2)  et  sur  laquelle  nous  ne  nous  étendrons  pas.  Mais  la 
distinction  qu'il  établit  au  départ,  entre  la  facticité,  ce  que  les  paroles  et  gestes  signifient,  et  la 
transcendance,  ce  vers  quoi  elles tendent,  et  que  la  mauvaise foi refuse  de  coordonner,  peut  être 
utile pour décrire certaines incompatibilités et le refus de les reconnaître. » 
\bigskip
(1) J.-P. Sartre, L'être et le néant, p. 95. 
 (2) Ibid., p. 109. 
\bigskip
P  269-270 :  « Les  incompatibilités  diffèrent  des  contradictions  parce  qu'elles  n'existent  qu'en 
fonction des circonstances : il faut que deux règles, pour entrer en un conflit qui impose un choix, 
soient  applicables  simultanément  à  une  même  réalité.  A  partir  du  moment  où  l'on  peut  diluer 
l'incompatibilité dans le temps, où il paraît possible d'appliquer les deux règles successivement, et 
non pas au même moment, le sacrifice de l'une d'entre elles pourrait être évité. C'est la raison pour 
laquelle l'attitude, que nous avons qualifiée de pratique, ne cherche pas à résoudre, à l'avance, tous 
les conflits possibles. L'attitude diplomatique s'efforce de retarder leur solution, pour ne pas devoir 
faire  immédiatement  un  sacrifice  considéré  comme  pénible,  espérant  que  des  circonstances 
ultérieures permettront soit d'éviter le choix soit de prendre la décision en meilleure connaissance 
de cause. Mais nous l'avons déjà dit, et nous le répétons, il se peut qu'éluder une incompatibilité 
actuelle en crée de nouvelles, et plus graves, dans l'avenir. » 
\bigskip
§ 48. TECHNIQUES VISANT A PRESENTER DES  THESES COMME COMPATIBLES OU 
\bigskip
INCOMPATIBLES 
P 270 : « Puisque deux propositions ne sont pas incompatibles mais le deviennent, par suite d'une 
certaine  détermination  de  notions  par  rapport  à  des  circonstances  particulières,  les  techniques 
permettant  de  présenter  des  énoncés  comme  incompatibles  et  les  techniques  visant  à  rétablir  la 
compatibilité sont parmi les plus importantes de toute argumentation. 
\bigskip
Deux  propositions  sont  dites  contradictoires,  dans  un  système  formalisé,  quand,  l'une  étant  la 
négation  de  l'autre,  on  suppose  que  chaque  fois  que  l'une  d'entre  elles  peut  s'appliquer  à  une 
situation,  l'autre  le  peut  également.  Présenter  des  propositions  comme  contradictoires,  c'est  les 
traiter comme si, en étant la négation l'une de l'autre, elles faisaient partie d'un système formalisé. 
Montrer l'incompatibilité de deux énoncés, c'est montrer l'existence de circonstances qui rendent 
inévitable le choix entre les deux thèses en présence. » 
\bigskip
P 270-271 : « Toute formulation qui, dans l'énoncé de propositions, tendra à les présenter comme 
étant  la  négation  l'une  de  l'autre  pourra  suggérer  que  les  attitudes  qui  y  sont  liées  sont 
incompatibles.  Le  monde  «  où  il  y  a  de  l'être»  et  celui  sans  être,  sont  pour  G.  Marcel  les 
présupposés ontologiques de deux modes de vie, celui de la personnalité et celui de la fonction, l'un 
« plein », l'autre « vide », qui, décrits comme incompatibles, paraissent l'avoir été à juste titre à 
cause  de  ces  présupposés  même  (1).  D'autre  part,  affirmer  qu'il  y  a  eu  choix  aidera  à  présenter, 
rétrospectivement  pour  ainsi  dire,  comme  incompatibles,  les  thèses  qui  peuvent  avoir  influencé 
celui-ci. » 
les  deux  règles  différentes 
\bigskip
(1) G. Marcel, Position et approches concrètes du mystère ontologique. 
\bigskip
P  271 :  « On  présentera  donc  des  thèses  comme  incompatibles  en  mettant  l'accent  -  dans 
l'ensemble de ce à quoi elles sont liées - sur le point où elles peuvent le plus facilement se traduire 
par  une  affirmation  et  une  négation.  Mais  la  mise  en  opposition  de  thèses  n'est  jamais 
indépendante des conditions de leur application. 
\bigskip
Une  des  techniques  pour  poser  des  incompatibilités  consiste  à  affirmer  que  de  deux  thèses  qui 
s'excluent,  au  moins  l'une  est  toujours  d'application,  ce  qui  rendrait  le  conflit  inévitable  avec 
l'autre  thèse  à  condition  qu'elles  s'appliquent,  toutes  deux,  à  un  même  objet.  Les  deux  thèses 
deviendront  compatibles  si  une  division  dans  le  temps,  on  une  division  quant  à  l'objet,  permet 
d'éviter  le  conflit.  Deux  affirmations  d'une  même  personne,  à  des  moments  différents  de  sa  vie, 
peuvent  être  présentées  comme  incompatibles  si  tous  les  énoncés  de  cette  personne  sont  traités 
comme formant un seul système ; si l'on traite les diverses périodes de sa vie comme n'étant pas 
solidaires l'une de l'autre, l'incompatibilité disparaît. Des énoncés de divers membres d'un groupe 
seront traités comme incompatibles, si le groupe est considéré comme un tout et les thèses de tous 
ses  membres  comme  formant  un  système  unique;  si  l'on  peut  montrer  que  l'un  des  énoncés  ne 
représente pas un point de vue autorisé, l'incompatibilité n'existe plus. Il n'y a pas d'inconvénient 
en  principe  à  ce  que  des  règles  différentes  régissent  le  comportement  des  membres  de  groupes 
distincts.  Une  difficulté  se  produira  si  un  membre  commun  à  ces  deux  groupes  se  trouve  placé 
dans  une  situation  où 
lui  prescrivent  des  comportements 
incompatibles. » 
\bigskip
P 271-272 : « Il est parfaitement possible qu'un chef d'État, désireux de sauvegarder la paix, puisse 
v parvenir sans permettre qu'une atteinte soit portée à l'honneur national. Mais il se peut que les 
deux normes qu'il s'impose dans la direction des affaires politiques deviennent incompatibles dans 
une  situation  déterminée.  Quelle  sera  cette  situation  attentatoire  à  l'honneur  national  ?  Des 
hommes  politiques  pourront  différer  d'avis  à  ce  sujet:  leur  liberté  de  décision  est  corrélative  du 
caractère vague des notions utilisées pour décrire la situation. » 
\bigskip
P 272 : « Celui qui s'interdit de tuer un être vivant peut être acculé à une incompatibilité s'il admet 
également  qu'il  faut  soigner  les  malades  souffrant  d'une  infection.  Va-t-il  ou  lion  se  servir  de  la 
pénicilline  qui  risque  de  détruire  un  grand  nombre  de  microbes  ?  Pour  éviter  l'incompatibilité 
entre les deux règles qu'il désire observer, il sera peut-être obligé de préciser certains termes, de 
façon que la situation particulière  devant laquelle il se trouve ne tombe plus sous l'application de 
l'une  d'elles.  De  même  que  l'extension  du  champ  d'application  des  règles  augmente  les  risques 
d'incompatibilités, la restriction de ce champ les diminue. 
\bigskip
Bentham accuse de sophisme ceux qui s'opposent à toute création d'office nouveau en arguant du 
danger  d'accroître  l'influence  du  gouvernement.  En  effet,  d'après  lui,  le  système  entier  du 
gouvernement serait détruit si l'on s'était avisé d'appliquer d'une façon constante cet argument (1). 
Le  sophisme  résulte  de  l'incompatibilité  de  cet  argument,  étendu  non  seulement  à  toutes  les 
propositions  nouvelles,  mais  aussi  à  toutes  les  situations  déjà  existantes,  avec  le  maintien  d'une 
forme  quelconque  de  gouvernement.  Mais  pour  mettre  à  jour  cette incompatibilité,  Bentham  est 
obligé  d'étendre  le  champ  d'application  de  l'argument  bien  au  delà  de  ce  que  ses  adversaires 
avaient jamais prétendu. » 
\bigskip
(1) Bentham, Œuvres t. 1 : Traité des sophismes politiques, p. 479. 
\bigskip
P  272-273 :  « C'est  souvent  par  l'extension  à  des  cas  qui  auraient  échappé  à  l'attention  de 
l'adversaire que l'on prétend mettre en évidence des incompatibilités : on objectera, à qui ne veut 
pas  admettre  qulune  vérité  soit  dans  l'esprit  si  l'esprit  n'y  a  jamais  pensé,  que  par  extension,  les 
vérités  auxquelles  on  ne  pense  plus  seraient  dès  lors  également  étrangères  à  l'esprit  (1)  ;  on 
\bigskip
\bigskip
\bigskip
136 
\bigskip
assimilera la naissance des dieux à leur mort, pour accuser d'impiété aussi bien ceux qui affirment 
que les dieux naissent que ceux qui affirment qu'ils meurent (2). » 
\bigskip
(1) Leibniz, Oeuvres éd. Gerhardt, 5e vol. : Nouveaux essais sur l'entendement, pp. 79-80. 
(2) Aristote, Rhétorique, II, chap. 23, 1399 b, 5. 
\bigskip
P 273 : « Ces extensions lie sont point simple généralisation, mais mettent en jeu, très visiblement, 
une identification dont nous aurons à reparler (3). C'est sur elle que portera l'accent lorsque Locke 
écrit : 
\bigskip
Il  sera  très  difficile  de  faire  admettre  à  des  hommes  de  sens  que  celui  qui,  l'oeil  sec  et  l'esprit 
satisfait,  livre  son  frère  aux  exécuteurs  pour  être  brûlé  vif,  est  sincèrement  et  de  tout  coeur 
préoccupé de sauver ce frère des flammes de l'enfer dans le monde de l'au-delà (4). 
\bigskip
Certaines normes peuvent être incompatibles par le fait que l'une d'elles réglemente une situation 
que l'autre exclut. Ruth Benedict signale que les prisonniers japonais étaient très complaisants lors 
des interrogatoires, parce qu'ils n'avaient pas reçu d'instructions concernant ce qu'ils pouvaient, ou 
non,  révéler  quand  ils  seraient  faits  prisonniers.  Elle  remarque  que  cela  était  dû  à  l'éducation 
militaire  japonaise  qui  obligeait  les  soldats  à  se  battre  jusqu'à  la  mort (5).  Cette  conception  était 
incompatible avec l'enseignement de règles de conduite à observer par les prisonniers. 
\bigskip
(3) V. plus loin § 53 : Arguments de réciprocité. 
(4) Locke, The second treatise of civil government and A letter concerning toleration, p. 137. 
(5) Ruth Benedict Tite Chrysanthemum and the Sword, pp 30 et 41. 
\bigskip
P  273-274 :  « Il  serait  certes  permis  de  s'étendre  sur  bien  d'autres  cas  d'incompatibilité.  Nous 
aimerions  exposer  encore  quelques  situations  particulièrement  intéressantes  où  l'incompatibilité 
n'oppose pas, l'une à l'autre, des règles différentes, mais une règle à des conséquences résultant du 
fait même qu'elle a été affirmée : nous qualifierons  cette sorte d'incompatibilités, qui se présente 
sous  des  modalités  diverses,  du  nom  générique  d'autophagie.  La  généralisation  d'une  règle,  son 
application sans exception, conduirait à empêcher son application, à la détruire. Pour prendre un 
exemple chez Pascal : 
\bigskip
Rien  ne  fortifie  plus  le  pyrrhonisme  que  ce  qu'il  y  en  a  qui  ne  sont  point  pyrrhoniens  :  si  tous 
l'étaient, ils auraient tort (1). » 
\bigskip
(1) Pascal, Oeuvre Bibl. de la Pléiade, Pensées, 185 (81), p. 871 (374 éd. Brunschvigh). 
\bigskip
P 274 : « La rétorsion, que l'on appelait au moyen âge la redarguitio elenchica, constitue l'usage le 
plus célèbre de l'autophagie : c'est un argument qui tend à montrer que l'acte par lequel une règle 
est attaquée, est incompatible avec le principe qui soutient cette attaque. La rétorsion est souvent 
utilisée,  depuis  Aristote,  pour  défendre  lexistence  de  principes  premiers  (2).  C'est  ce  que  Ledger 
Wood appelle très j ustement « method of affirmation by attempted denial " (3). 
\bigskip
Ainsi, à celui qui objecte au principe de non-contradiction, on rétorque que son objection même, 
par le fait qu'il prétend affirmer la vérité et en tirer la conséquence que son interlocuteur affirme le 
faux,  présuppose  le  principe  de  non-contradiction  :  l'acte  implique  ce  que  les  paroles  nient. 
][,'argument  est  quasi  logique  parce  que,  pour  mettre  en  évidence  l'incompatibilité,  il  faut  une 
interprétation  de  l'acte  par  lequel  l'adversaire  s'oppose  à  une  règle.  Et  cette  interprétation, 
condition de la rétorsion, pourrait, elle-même, faire l'objet de controverses (4). 
\bigskip
(2) Cf. G. Isaye, La justification critique par rétorsion, Revue philosophique de Louvain, mai  1954 
pp. 205-233. Cf. ail-si Dialectica, 21, p. 32. 
\bigskip
\bigskip
\bigskip
137 
\bigskip
(3) Ledger Wood, The Analysis of Knowledge, pp. 194 et suiv. 
(4) CI. à ce sujet F. Gonseth, Dialectica, 21, p. 61, et H Feigl, De Principiis non disputandum... ?, 
dans Philosophical Analysis, edited by Max Black, P. 125 
\bigskip
P  274-275 :  « Un  cas  comique  de  l'application  de  la  rétorsion,  et  qui  suggère  les  possibilités  d'y 
échapper,  est  fourni  par  l'histoire  du policier  qui,  dans  un  théâtre  de  province,  au  moment  où  le 
public s'apprêtait à chanter La Marseillaise, monte sur la scène pour annoncer qu'est interdit tout 
ce qui ne figure pas sur l'affiche. « Et vous, interrompt l'un des spectateurs, êtes-vous sur l'affiche ? 
» Dans cet exemple le policier, par son affirmation, contrevient à un principe qu'il pose, alors que 
dans  les  cas  de  rétorsion,  on  présuppose  un  principe  que  l'on  rejette,  mais  la  structure  de 
l'argument est la même. » 
\bigskip
P  275 :  « Une  autre  situation  qui  peut  mener  à  l'autophagie,  est  celle  où  l'on  n'oppose  pas  un 
énoncé à l'acte par lequel il est affirmé, mais où l'on applique la règle à elle-même : l'autophagie 
résulte de l'auto-inclusion. Aux  positivistes qui affirment que toute proposition est analytique ou 
de nature expérimentale, on demandera si ce qu'ils viennent de dire est une proposition analytique 
ou  résultant  de  l'expérience.  Au  philosophe  qui  prétend  que  tout  jugement  est  un  jugement  de 
réalité ou un jugement de valeur, on demandera quel est le statut de son affirmation.  A celui qui 
argumente  pour  rejeter  la  validité  de  tout  raisonnement  non  démonstratif,  on  demandera  quelle 
est la valeur de sa propre argumentation. Toute auto-inclusion ne conduit pas à l'autophagie, mais 
elle oblige à réfléchir à la valeur du cadre classificatoire que l'on se propose d'établir, et aboutit par 
là à une augmentation de conscience ; souvent l'auteur prendra les devants soit pour montrer que 
l'auto-inclusion  ne  crée  aucune  difficulté  soit  pour  indiquer  les  raisons  qui  empêchent  l'auto-
inclusion de se produire. 
\bigskip
Une autre forme encore d'autophagie est celle qui oppose une règle aux conséquences qui semblent 
en  découler.  Dans  ses  Sophismes  anarchiques,  Bentham  critique  la  constitution  française  qui 
justifie les insurrections : 
\bigskip
Mais  les  justifier,  c'est  les  encourager...  justifier  la  destruction  illégale  d'un  gouvernement,  c'est 
saper tout autre gouvernement, sans en excepter celui même qu'on veut substituer au premier. Les 
législateurs  de  la  France  imitaient,  sans  y  songer,  l'auteur  de  cette  loi  barbare  qui  conférait  au 
meurtrier d'un prince le droit de lui succéder au trône (1). » 
\bigskip
(1) Bentham, Oeuvres, t. 1 : Sophismes anarchiques, p. 524. 
\bigskip
Tomberait  sous  le  coup  de  la  même  objection  toute  théorie,  professée  par  un  infirme,  et 
préconisant  la  suppression  des  infirmes.  Nous  pouvons  placer  dans  cette  même  catégorie 
d'arguments la réplique d'Epictète à Epicure qui prend parti pour l'abandon des enfants: 
\bigskip
Pour  moi,  je  crois  bien  que  même  si  ta  mère  et  ton  père  avaient  deviné  que  tu  dirais  pareilles 
choses, ils ne t'auraient pas exposé (1). 
\bigskip
Tous ces cas d'autophagie affaiblissent une thèse en montrant les incompatibilités que révèle une 
réflexion sur des conditions ou des conséquences de son affirmation. Ni ici, ni  dans les autres cas 
d'incompatibilité,  on  n'est  acculé  à  l'absurde,  à  une  contradiction  purement  formelle.  Mais 
néanmoins on ne peut négliger de tenir compte de ces arguments si l'on ne veut pas s'exposer au 
ridicule. C'est le ridicule, et non l'absurde (2), qui est l'arme principale de l'argumentation : aussi 
est-il indispensable de consacrer à cette notion un développement plus important. 
\bigskip
§ 49. LE RIDICULE, ET SON ROLE DANS L'ARGUMENTATION 
\bigskip
P 276 : « Le ridicule est ce qui mérite d'être sanctionné par le rire, celui que E. Dupréel dans son 
excellente  analyse,  a  qualifié  de  «  rire  d'exclusion  »  (3).  Ce  dernier  est  la  sanction  de  la 
transgression d'une règle admise, une façon de condamner une conduite excentrique, que l'on ne 
juge pas assez grave ou dangereuse pour la réprimer par des moyens plus violents. » 
\bigskip
(1) Epictète, Entretiens, liv. 1, 23, 9 7. 
(2)  Cf.  l'usage  de  ces  termes  chez  Pascal,  Bibl.  de  la  Pléiade,  Pensées,  4  (213),  p.  823  (273  éd. 
Brunschvicg). 
(3) E. Dupréel, Essais pluralistes (Le problème sociologique du rire), p. 41. 
\bigskip
P 276-277 : « Une affirmation est ridicule dès qu'elle entre en conflit, sans justification, avec une 
opinion admise. Est d'emblée ridicule celui qui pèche contre la logique ou se trompe dans l'énoncé 
des faits, à condition qu'on ne le considère pas comme un aliéné ou un être qu'aucun acte ne risque 
de disqualifier parce qu'il ne jouit pas du moindre crédit. Il suffit d'une erreur de fait, constate La 
Bruyère, pour jeter un homme sage dans le ridicule (1). La crainte du ridicule, et la déconsidération 
qu'il entraîne, a été souvent utilisée comme moyen d'éducation ; celui-ci est tellement puissant que 
des psychiatres ont même souligné le danger de son usage pour l'équilibre de l'enfant, guetté par 
l'anxiété  (2).  Normalement  le  ridicule  est  lié  à  ce  qu'une  règle  a  été  transgressée  ou  combattue 
d'une  façon  inconsciente  (3),  par  ignorance  soit  de  la  règle  elle-même,  soit  des  conséquences 
désastreuses d'une thèse ou d'un comportement. Le ridicule s'exerce en faveur de la conservation 
de  ce  qui  est  admis;  un  simple  changement  d'avis  injustifié,  c'est-à-dire  une  opposition  à  ce  que 
l'on avait soi-même énoncé, pourra y exposer. » 
\bigskip
(1) La Bruyère, Bibl. de la Pléiade, Les caractères, Des jugements, 47, p. 379.  
(2) Harry Stack Sullivan, The Interpersonal Theory of Psychiatry, p. 268 ; cf. d'autre part pour le 
rapport entre anxiété et incompatibilité, pp. 170, 190, 346 et sur l'inattention sélective permettant 
d'éviter ces incompatibilités, A. H. STANTON, Sullivan's Conceptions, dans Patrick Mullahy, The 
Contributions o/ Harry Stock Sullivan, p. 70. 
(3) Platon ne l'ignore point, qui prévoit opportunément le rire que susciteront certaines de ses 
propositions heurtant violemment les usages établis, Platon, La République, liv. V, 452, 457 b, 473 
c. 
P 277 : « Le ridicule est l'arme puissante dont dispose l'orateur contre ceux qui risquent d'ébranler 
son argumentation, en refusant, sans raison, d'adhérer à l'une ou l'autre prémisse de son discours. 
C'est  elle  aussi  qu'il  faut  utiliser  contre  ceux  qui  s'aviseraient  d'adhérer,  ou  de  continuer  leur 
adhésion,  à  deux  thèses  jugées  incompatibles,  sans  s'efforcer  de  lever  cette  incompatibilité  :  le 
ridicule  ne  touche  que  celui  qui  se  laisse  enfermer  dans  les  mailles  du  système  forgé  par 
l'adversaire. Le ridicule est la sanction de l'aveuglement, et ne se manifeste qu'à ceux pour lesquels 
cet aveuglement ne fait pas de doute. » 
\bigskip
P  277-278 :  « Sera  ridicule  non  seulement  celui  qui  s'oppose  à  la  logique  ou  à  l'expérience,  mais 
encore celui qui énoncera des principes dont les conséquences imprévues le mettent en opposition 
avec  des  conceptions  qui  vont  de  soi  dans  une  société  donnée,  et  qu'il  n'oserait  pas  heurter  lui-
même.  L'opposition  au  normal,  au  raisonnable,  peut  être  considérée  comme  un  cas  particulier 
d'opposition à une norme admise. Il est par exemple risible de ne pas proportionner ses efforts à 
l'importance de leur objet (1). » 
\bigskip
(1) Platon, La République, liv. VI, 504 e. Pour l'argumentation par double hiérarchie cf. notre § 
76. 
\bigskip
P  278 :  « Dire  d'un  auteur  que  ses  opinions  sont  inadmissibles,  parce  que  les  conséquences  en 
seraient  ridicules,  est  une  des  plus  fortes  objections  que 
l'on  puisse  présenter  dans 
l'argumentation.  C'est  ainsi  que  La  Bruyère,  dans  ses  dialogues  sur  le  quiétisme,  ridiculise  cette 
doctrine  en  montrant  que  ses  adhérents  devraient  s'opposer  aussi  bien  au  devoir  de  charité  qu'à 
l'exercice de dévotions, conséquences auxquelles aucun chrétien ne pourrait souscrire  (2). Quand, 
en 1877, en Belgique, le ministre catholique de la justice décide de ne pas poursuivre, malgré la loi 
\bigskip
\bigskip
\bigskip
139 
\bigskip
pénale  protégeant  la  liberté  de  l'électeur,  les  curés  qui  menaçaient  des  peines  de  l'enfer  leurs 
ouailles qui voteraient pour le parti libéral, le tribun Paul Janson ridiculise le ministre : ce dernier, 
en mettant en doute le sérieux de pareilles menaces, commettait « une véritable hérésie religieuse 
» (3). 
\bigskip
(2) La Bruyère, Bibl. de la Pléiade, Dialogues sur le quiétisme, I, p. 532 ; V, p. 576. 
(3) Paul Janson, Discours parlementaires, vol. I, p. 19 (6 juin 1877). 
\bigskip
P 278-279 : « Souvent cette ridiculisation s'obtient par de savantes constructions basées sur ce que 
l'on s'efforce de critiquer. De même que, en géométrie, le raisonnement par l'absurde commence 
par supposer vraie une proposition A pour montrer que ses conséquences sont contradictoires avec 
ce à quoi on a consenti par ailleurs, et passer de là à la vérité de non-A, de même, l'argumentation 
quasi logique par le ridicule la plus caractérisée, consistera à admettre momentanément une thèse 
opposée  à  celle  que  l'on  veut  défendre,  à  développer  ses  conséquences,  à  montrer  leur 
incompatibilité  avec  ce  à  quoi  l'on  croit  par  ailleurs,  et  à  prétendre  passer  de  là  à  la  vérité  de  la 
thèse  que  l'on  soutient.  C'est  ce  que  tentait  Whately  lorsque,  dans  un  pamphlet  anonyme,  il 
commençait  par  admettre  comme  fondé  le  type  d'objections  alléguées  contre  la  véracité  des 
Écritures,  en  développait  les  conséquences  et  aboutissait  à  nier  l'existence  de  Napoléon. 
L'argumentation qui, en ridiculisant les procédés de la critique biblique, visait à rendre confiance 
dans le texte des Écritures, n'eut pas le succès qu'il en espérait, mais parut spirituelle (1). » 
\bigskip
(1) Cf. B. D. D. Whately, Elements of Rhetoric, Part I, chap. III, § 6, note, P. 100. 
\bigskip
P  279 :  « L'assomption  provisoire  par  laquelle  commence  ce  genre  de  raisonnement  peut  se 
traduire par une figure, l'ironie. Par l'ironie « on veut faire entendre le contraire de ce qu'on dit » 
(2). Pourquoi ce détour ? C'est que nous avons affaire en réalité à une argumentation indirecte. En 
voici un bel exemple pris chez Démosthène : 
\bigskip
Vraiment le peuple d'Oréos a eu fort à se réjouir de s'être mis entre les mains des amis de Philippe 
et d'avoir écarté Euphraeos ! Ils ont eu à se réjouir, les Erétriens, d'avoir renvoyé vos députés et de 
s'être donnés à Clitarque ! Les voici esclaves, on les fouette et on les égorge (3) ! 
\bigskip
L'ironie est pédagogique (4) parce que si le peuple d'Oréos et les Érétriens ne peuvent plus rien, le 
peuple d'Athènes, lui, peut encore choisir. Rappelons à ce propos l'émotion causée en Belgique, en 
1950,  par  un  discours  où  Paul  Reynaud  avait  parlé  de  «  la  neutralité  qui  a  si  bien  réussi  à  deux 
reprises  à  la  Belgique  »  (5).  L'orateur  déclara  qu'il  n'avait  pas  voulu  critiquer  la  Belgique,  mais 
montrer que la neutralité n'était pas une garantie, c'est-àdire qu'il accordait que, pour la Belgique, 
il y avait eu, en faveur de la neutralité, contrainte matérielle ou psychologique, réservant le ridicule 
à ses auditeurs français encore libres de décider. » 
\bigskip
(2) Dumarsais, Des Tropes, p. 131. 
(3) Démosthène, Harangues et plaidoyers politiques, t. II : Troisième Philippique, § 66. 
(4) Pour son rôle dans le dialogue platonicien, cf. B. Schaerer, Le mécanisme de l'ironie dans ses 
rapports avec la dialectique, Rev. de métaph. et de morale, juill. 1941. 
(5) Cf. Journal Le Soir du 3 juin 1950. 
\bigskip
P  280 :  « L'ironie  suppose  toujours  des  connaissances  complémentaires  au  sujet  de  faits,  de 
normes.  Dans  l'exemple  cité  par  Dumarsais  :  «  je  le  déclare  donc,  Quinaut  est  un  Virgile  »  (1), 
l'affirmation serait incompatible avec les  normes admises et bien connues. L'ironie ne peut donc 
être utilisée dans les cas où l'on doute des opinions de l'orateur. Ceci donne à l'ironie un caractère 
paradoxal  :  si  on  l'emploie,  c'est  qu'il  v  a  utilité  à  argumenter;  mais  pour  l'employer  il  faut  un 
minimum  d'accord.  C'est  là  sans  doute  ce  qui  fait  dire  à  Baroja  que  l'ironie  a  un  caractère  plus 
\bigskip
\bigskip
\bigskip
140 
\bigskip
social  que  l'humour  (2).  Cet  apparent  paradoxe  n'est  qu'un  des  aspects,  poussé  à  l'extrême,  de 
toute argumentation. 
\bigskip
L'ironie  est  d'autant  plus  efficace  qu'elle  s'adresse  à  un  groupe  bien  délimité  (3).  C'est  la 
conception que l'on se fait des convictions de certains milieux qui seule peut nous faire deviner si 
tels textes sont ou non ironiques (4). 
\bigskip
L'usage  de  l'ironie  est  possible  dans  toutes  les  situations  argumentatives.  Néanmoins  certaines 
paraissent  y  convier  particulièrement.  Vayson  de  Pradenne  constate  que,  dans  les  controverses 
archéologiques, les défenseurs de l'authenticité se servent volontiers de l'ironie : ainsi Th. Reinach 
décrit  une  société  de  faussaires,  prenant  ses  décisions  à  la  majorité,  qui  serait  le  fabricant  de  la 
tiare de Saïtapharnès (5). On comprend que l'ironie soit surtout le procédé de la défense puisque, 
pour être comprise, l'ironie exige une connaissance préalable des positions de l'orateur or celles-ci 
ont été mises en évidence par l'attaque. » 
\bigskip
(1) Boileau, satire IX, cité par Dumarsais, Des Tropes, p. 132. 
(2) Pio Baroja, La caverna del humorismo, p. 96. 
(3) Cf. Auerbach, Mimesis, pp. 213-214, excellente analyse d'un passage ironique de Boccace. 
(4) Par exemple, lettre de Zhdanov à  Staline, dans J. Huxley,  Soviet  genetics and  world science, 
pp. 230-234 (Postscript II). 
(5) Vayson De Pradenne, Les fraudes en archéologie préhistorique, p. 538. 
\bigskip
P  280-281 :  « S'il  est  vrai  que  le  ridicule  joue,  dans  l'argumentation,  un  rôle  analogue  à  celui  de 
l'absurde  dans  la  démonstration,  néanmoins  -  et  c'est  bien  la  preuve  que  l'argumentation  n'est 
jamais contraignante - on peut braver le ridicule, en se mettant carrément en opposition avec une 
règle habituellement admise. Celui qui brave le ridicule sacrifie celle-ci et encourt condamnation 
de  la  part  du  groupe.  Mais  ce  sacrifice  peut  n'être  que  provisoire,  si  le  groupe  consent,  soit  à 
admettre des exceptions, soit à modifier la règle. » 
\bigskip
P 281 : « Il faut de l'audace pour braver le ridicule, une certaine capacité de surmonter l'anxiété, 
mais  cela  ne  suffit  pas  pour  réussir  :  pour  ne  pas  sombrer  dans  le  ridicule,  il  faut  un  prestige 
suffisant, et on n'est jamais sûr qu'il le sera. En effet, en bravant le ridicule que suscite l'opposition 
injustifiée  à  une  norme  admise,  on  engage  toute  sa  personne,  solidaire  de  cet  acte  périlleux,  on 
lance un défi, on provoque une confrontation de valeurs dont l'issue est incertaine. » 
\bigskip
P  281-282 :  « Ceux  qui  adoptent  un  nom  injurieux  et  s'en  glorifient,  ceux  qui  lancent  une  mode 
nouvelle  ou  qui  refusent,  comme  Ghandi,  de  se  plier  aux  coutumes  de  l'Occident,  quand  ils  s'y 
trouvent,  ceux  qui  adhèrent  à  des  opinions  ou  adoptent  des  façons  de  se  conduire  sortant  de 
l'ordinaire, cesseront d'être ridicules quand on leur emboîtera le pas. Le prestige du chef se mesure 
à  sa  capacité  d'imposer  des  règles  qui  semblent  ridicules,  et  de  les  faire  admettre  par  ses 
subordonnés (1). Pour qu'un énoncé contraire à l'opinion couramment admise devienne une thèse 
qui mérite discussion, il faudrait qu'il jouisse de l'appui d'un philosophe notable  (2). Un prestige 
surhumain serait nécessaire pour s'opposer aux faits ou à la raison: d'où la portée du credo quia a 
bsurdum.  Normalement  l'argumentation,  oeuvre  d'humains,  ne  s'oppose  qu'à  ce  qui  n'est  pas 
considéré  comme  objectivement  valable.  Les  opinions  dont  elle  traite  ne  sont  pas  absolument 
indiscutables, les autorités qui les posent ou les combattent ne sont pas tout à fait inattaquables et 
les solutions qui seront acceptées, en fin de compte, ne sont pas connues d'avance. » 
\bigskip
(1) Cf. Isocrate, Discours, t. I : Busiris, § 26. 
(2) Aristote, Topiques, liv. 1, chap. 2, 104 b, 
\bigskip
P 282 :  « La façon la  plus fréquente de combattre une règle ou une norme admise ne consistera 
pas simplement dans un conflit de forces, dans le fait d'opposer au prestige dont jouit la règle celui 
des  adversaires  de  celle-ci.  Normalement,  on  justifiera  cette  opposition,  on  trouvera  des  raisons 
pour lesquelles dans certaines circonstances, dans  des situations déterminées, la règle devrait ne 
pas être appliquée : on en restreindra la portée et le sens, grâce à une argumentation appropriée, 
dont  résultera une  rupture  des  liaisons  admises,  un  remaniement  de  notions.  Nous  examinerons 
longuement  ces  procédés  d'argumentation  dans  la  partie  de  notre  traité  consacrée  aux 
dissociations. » 
\bigskip
§ 50. IDENTITE ET DEFINITION DANS L'ARGUMENTATION 
\bigskip
Une  des  techniques  essentielles  de  l'argumentation  quasi  logique  est  l'identification  de  divers 
éléments  qui  sont  l'objet  du  discours.  Tout  usage  de  concepts,  toute  application  d'une 
classification, tout recours à l'induction implique une réduction de certains éléments à ce qu'il y a 
en  eux  d'identique  ou  d'interchangeable;  mais  nous  ne  qualifierons  cette  réduction  de  quasi 
logique  que  lorsque  cette  identification  d'êtres,  d'événements  ou  de  concepts  n'est  considérée  ni 
comme tout à fait arbitraire ni comme évidente, c'est-à-dire quand elle donne ou peut donner lieu 
à  une  justification  argumentative.  Nous  distinguerons,  parmi  les  procédés  d'identification,  ceux 
qui  visent  à  une  identité  complète  et  d'autres  qui  ne  prétendent  qu'à  une  identité  partielle  des 
éléments confrontés. » 
\bigskip
P 282-283 : « Le procédé le plus caractéristique d'identification complète consiste dans l'usage des 
définitions.  Celles-ci,  quand  elles  ne  font  pas  partie  d'un  système  formel,  et  qu'elles  prétendent 
néanmoins  identifier  le  definiens  avec  le  definiendum,  seront  considérées,  par  nous,  comme  de 
l'argumentation quasi logique. Que ces définitions puissent être fondées sur l'évidence de rapports 
notionnels, nous ne pouvons l'admettre, car cela supposerait la  clarté parfaite de tous les termes 
confrontés. » 
\bigskip
P  283 :  « Pour  qu'une  définition  ne  nous  suggère  pas  cette  identification  des  termes  qu'elle 
présente  comme  équivalents,  il  faut  qu'elle  insiste  sur  leur  distinction,  telles  ces  définitions  par 
approximation  ou  par  exemplification  où  l'on  demande  expressément  au  lecteur  de  fournir  un 
effort de purification ou de généralisation lui permettant de franchir la distance qui sépare ce que 
l'on définit des moyens utilisés pour le définir. 
\bigskip
Parmi les définitions qui mènent à l'identification de ce qui est défini avec ce qui le définit, nous 
distinguerons, avec Arne Naess (1), les quatre espèces suivantes : 
\bigskip
1)  Les  définitions  normatives,  qui  indiquent  la  façon  dont  on  veut  qu'un  mot  soit  utilisé.  Cette 
norme peut résulter d'un engagenient individuel, d'un ordre destiné à d'autres, d'une règle dont on 
croit qu'elle devrait être suivie par tout le monde ; 
\bigskip
2)  Les  définitions  descriptives  qui  indiquent  quel  est  le  sens  accordé  à  un  mot  dans  un  certain 
milieu à un certain moment; 
\bigskip
3)  Les  définitions  de  condensation  qui  indiquent  des  éléments  essentiels  de  la  définition 
descriptive ; 
\bigskip
4)  Les  définitions  complexes  qui  combinent,  de  façon  variée,  des  éléments  des  trois  espèces 
précédentes. 
\bigskip
Ces diverses définitions seraient soit des prescriptions soit des hypothèses empiriques concernant 
la synonymie du definiendum et du definiens. » 
\bigskip
(1) Cf. A. Naess, Interpretation and Preciseness, chap. TV. 
P  283-284 :  « Parmi  les  définitions  normatives,  seules  celles  qui  se  présentent  comme  une  règle 
obligatoire  sont  susceptibles  d'être  appuyées  ou  combattues grâce  à  l'argumentation;  il  en  est  de 
même  de  définitions  de  condensation,  dont  on  peut  se  demander  dans  quelle  mesure  les 
indications qu'elles fournissent sont ou non essentielles. Quant aux définitions descriptives, elles 
jouiront, aussi longtemps qu'elles ne sont pas contestées, du statut d'un fait. » 
\bigskip
P 284 : « Toutes ces définitions, et les possibilités argumentatives qu'elles fournissent, sont encore 
méconnues par la plupart des logiciens dont la  pensée continue à se mouvoir dans le cadre de la 
dichotomie classique des définitions réelles et nominales, les premières étant traitées comme des 
propositions susceptibles d'être vraies ou fausses, les secondes comme étant purement arbitraires. 
\bigskip
Voici un texte caractéristique de J. St. Mill, et auquel souscriraient encore bon nombre de logiciens 
contemporains : 
\bigskip
Les assertions relatives à la signification des mots, parmi lesquelles les plus importantes sont les 
définitions ont une place, et une place indispensable, en philosophie. Mais comme la signification 
des  mots  est  essentiellement  arbitraire,  les  assertions  de  cette  classe  ne  sont  susceptibles  ni  de 
vérité ni de fausseté, et, par conséquent, ni de preuve ni de réfutation (1). 
\bigskip
Mill  opte  pour  le  caractère  nominal,  donc  conventionnel  et  arbitraire,  des  définitions,  qui  par  là 
même, échapperaient à toute preuve comme à toute tentative de réfutation. Mais en est-il vraiment 
ainsi  ?  S'il  est  exact  que  les  définitions  sont  arbitraires,  dans  le  sens  qu'elles  ne  s'imposent  pas 
nécessairement,  faut-il  considérer  qu'elles  soient  arbitraires,  dans  un  sens  bien  plus  fort,  qui 
prétendrait qu'il n'y aurait pas de raison pour choisir l'une ou l'autre définition, et qu'il n'y aurait 
donc aucune possibilité d'argumenter en leur faveur ? Or, non seulement on trouve chez Mill une 
série  de  raisonnements  tendant  à  faire  prévaloir  ses  définitions  de  la  cause,  de  l'inférence,  de 
l'induction, mais on trouve même, dans son ouvrage consacré à l'utilitarisme, une définition de la 
preuve assez large pour couvrir des raisonnements de cette espèce (2). » 
\bigskip
(1) J. St. Mill, Système de logique déductive et inductive, vol. I (liv. 11, chap. 1, § 1), pp. 176-177. 
(2) ID., L'utilitarisme, p. 8. 
\bigskip
P  285 :  « Ce  qui  fait  croire  au  caractère  conventionnel  des  définitions,  c'est  la  possibilité 
d'introduire de toutes pièces dans tous les langages, même usuels, des symboles nouveaux. Mais si 
ces signes nouveaux sont appelés à remplir entièrement ou partiellement le rôle de termes anciens, 
le caractère arbitraire de leur définition est illusoire - même s'il s'agit de symboles créés ad hoc. Il 
l'est plus encore si definiens et definiendum sont tous deux empruntés au langage usuel. Lorsque 
Keynes,  dans  ses  ouvrages  (1),  propose  une  série  de  définitions  techniques  (2),  celles-ci  peuvent 
s'éloigner  tellement  de  l'idée  que  le  sens  commun  se  fait  des  notions  ainsi  définies  qu'elles 
apparaissent  comme  conventionnelles.  L'auteur  les  modifiera  même  d'un  ouvrage  à  l'autre  (3). 
Mais  quand  il  définit  d'une  part  l'épargne,  d'autre  part  l'investissement,  de  façon  a  ce  que  ses 
observations  et  analyses  aboutissent  à  montrer  que  leur  égalité  est  plus  essentielle  que  leurs 
divergences  passagères,  l'intérêt  de  son  raisonnement  résulte  de  ce  que  nous  rapprochons  les 
termes définis par lui des notions usuelles, ou déjà précisées par les économistes, que son analyse 
contribue à éclairer. » 
\bigskip
(1)  J.  M.  Keynes,  A  treatise  on  money,  1930  ;  The  general  theory  of  employment  interest  and 
money, 1936. 
(2)  Qui  pourraient  être  rapprochées  de  la  notion  carnapienne  d'  « explication »,  Cf.  C.  G. 
HEMPEL, Fundamentals of Concept Formation in Empirical Science, pp. 11-12. 
(3) Voir à ce propos, The general theory.... pp. 60-61. 
\bigskip
\bigskip
\bigskip
\bigskip
143 
\bigskip
P 285-286 : « Une théorie peut se prétendre purement conventionnelle, et vouloir fonder sur cette 
prétention  le  droit  de  définir  ses  signes  comme  bon  lui  semble,  mais  dès  qu'elle  vise  à  une 
confrontation avec  le réel, dès que l'on se propose  de l'appliquer à des situations antérieurement 
connues, le problème de l'identification des notions qu'elle définit avec celles du langage naturel ne 
peut  être  éludé.  La  difficulté  que  l'on  a  cherché  à  éviter  n'a  pu  être  que  transposée  sur  un  autre 
plan. C'est là tout le problème du formalisme : ou bien ce dernier fournira un système isolé, non 
seulement  de  ses  applications,  mais  même  d'une  pensée  vivante  qui  doit  le  comprendre  et  le 
manier,  c'est-à-dire  l'intégrer  dans  des  structures  mentales  préexistantes,  ou  bien  il  aura  à  être 
interprété et opérera des identifications qui relèveront de l'argumentation quasi logique. Même si 
ces identifications ne sont pas contestées, pendant une certaine période de l'évolution scientifique, 
il serait dangereux pour un progrès ultérieur de la pensée, de les considérer comme nécessaires et 
de leur accorder ce caractère d'évidence que l'on prête aux affirmations qu'il n'est plus permis de 
remettre en discussion. C'est une des raisons de notre adhésion au principe de révisibilité, défendu 
avec tant de vigueur par F. Gonseth (1). » 
\bigskip
(1) F. Gonseth, Dialectica, 6, pp. 123-124. 
\bigskip
P  286 :  « Le  caractère  argumentatif  des  définitions  apparaît  nettement  quand  on  se  trouve  en 
présence  de  définitions  variées  d'un  même  terme  d'un  langage  naturel  (ou  même  de  termes 
considérés  comme  équivalents  dans  différentes  langues  naturelles).  En  effet,  ces  définitions 
multiples constituent soit des éléments successifs d'une définition descriptive - mais alors l'usager 
d'un  terme  doit  faire  son  choix  parmi  elles  -  soit  des  définitions  descriptives  opposées  et 
incomplètes,  des  définitions  normatives  ou  de  condensation  qui  sont  incompatibles.  Certains 
auteurs,  pour  se  faciliter  la  tâche  et  parfois  pour  éviter  des  discussions  inopportunes,  se 
contenteront  de  fournir  non  pas  les  conditions  suffisantes  et  nécessaires  mais  uniquement  les 
conditions  suffisantes  de  l'application  d'un  terme  (2);  mais  l'énoncé  de  ces  conditions,  joint  à  ce 
que l'on sait par ailleurs du terme en question, constitue néanmoins le choix d'une définition. » 
\bigskip
(2) Exemple chez Morris, Signs, Language and Beehavior, p. 12, et n. G, 1). 251). 
\bigskip
P  286-287 :  « Le  caractère  argumentatif  des  définitions  se  présente  toujours  sous  deux  aspects 
intimement  liés  mais  qu'il  faut  néanmoins  distinguer  parce  qu'ils  concernent  deux  phases  du 
raisonnement  :  les  définitions  peuvent  être justifiées,  valorisées,  à  l'aide  d'arguments ;  elles  sont 
elles-mêmes des arguments. Leur justification pourra se faire par les moyens les plus divers : l'un 
aura recours à l'étymologie (1), l'autre proposera de substituer une définition par les conditions à 
une définition par les conséquences ou vice versa (2). Mais ceux qui argumentent en  faveur d'une 
définition voudront tous que celle-ci influe, par l'un ou l'autre biais, sur l'usage de la notion que, 
sans  leur  intervention,  on  eût  été  enclin  à  adopter,  et  surtout  sur  les  relations  de  la  notion  avec 
l'ensemble  du  système  de  pensée,  cela  toutefois  sans  faire  oublier  complètement  les  usages  et 
relations anciennes. Or il en va de même lorsque la définition est donnée comme allant de soi ou 
comme imposée, telle la définition légale, et que les raisons qui militent en sa faveur ne sont pas 
explicitées.  L'usage  de  la  notion  que  l'on  veut  modifier  est  généralement  ce  qu'on  appelle  l'usage 
normal de celle-ci. De sorte que la définition d'une notion empruntée au langage naturel soulève 
implicitement les difficultés inhérentes à la double définition. » 
\bigskip
(1) Cf. Quintillien, Vol. II, liv. V, chap. X, § 55 ; cf. J. PAULHAN, La preuve par l'étymologie. 
(2) Cf. définition du miracle chez S. Weil, L'enracinement, pp. 227, et suiv. 
\bigskip
P  287-288 :  « Quand,  au  début  de  son  Éthique,  Spinoza  définit  la  cause  de  soi  comme  «  ce  dont 
l'essence  enveloppe  l'existence,  ou  (sive)  ce  dont  la  nature  ne  peut  être  connue  que  comme 
existante »; quand il définit la substance comme « ce qui est en soi et est conçu par soi, c'est-à-dire 
(hoc est)  ce dont le concept peut être formé sans avoir besoin du concept d'autre chose » (3), les 
mots  sive  et  hoc  est,  affirment  le  caractère  interchangeable  de  deux  définitions  différentes  d'une 
\bigskip
\bigskip
\bigskip
144 
\bigskip
même notion. En fait, il s'agit d'une identification entre trois notions, la troisième étant fournie par 
l'usage du terme tel qu'il était en vigueur du temps de Spinoza, spécialement chez les cartésiens. 
Normalement,  une  pareille  identification  exige,  si  pas  une  démonstration,  du  moins  une 
argumentation  pour  la  faire  admettre.  Quand  cette  identification  est  simplement  posée,  l'on  se 
trouve devant le cas type d'un procédé quasi logique. Mais ce que Spinoza fait explicitement, et qui 
peut donc être remarqué et critiqué même par un logicien qui ne s'attacherait qu'au texte, sans le 
confronter  avec  l'usage  usuel  des  notions,  pourrait,  d'une  façon  plus  délicate  à  constater,  être 
retrouvé chez tous ceux qui définissent les mots du langage d'une façon qui semble univoque, alors 
que le lecteur ne peut s'empêcher d'identifier aussi le mot ainsi défini, avec le même mot tel que la 
tradition linguistique l'a précisé. » 
\bigskip
(3) Spinoza, Éthique, liv. I. déf. I et III. 
\bigskip
P  288 :  « Nos  remarques  tendent  à  montrer  que  l'usage  argumentatif  des  définitions  suppose  la 
possibilité de définitions multiples, empruntées à l'usage ou créées par l'auteur, entre lesquelles il 
est indispensable de faire un choix. Elles montrent aussi que les termes mis en relation sont eux-
mêmes  en  interaction  constante,  non  seulement  avec  un  ensemble  d'autres  termes  du  même 
langage  ou  d'autres  langages  qui  peuvent  être  mis  en  rapport  avec  le  premier,  mais  aussi  avec 
l'ensemble  des  autres  définitions  possibles  du  même  terme.  Ces  interactions  ne  peuvent  être 
éliminées, et sont même essentielles généralement pour la portée des raisonnements. Cependant, 
le choix étant fait, qu'il soit présenté comme allant de soi ou qu'il soit défendu par des arguments, 
la  définition  utilisée  est  considérée  comme  expression  d'une  identité,  voire  comme  la  seule 
satisfaisante  en  l'occurrence,  et  les  termes  de  l'équivalence,  détachés  en  quelque  sorte  de  leurs 
liens et de leur arrière-plan, sont considérés comme logiquement substituables : aussi l'usage de la 
définition,  pour  faire  avancer  un  raisonnement,  nous  paraît-il  le  type  même  de  l'argumentation 
quasi logique. » 
\bigskip
§ 51. ANALYCITE, ANALYSE ET TAUTOLOGIE 
\bigskip
P 288-289 : « Une définition étant admise, on peut considérer comme analytique, l'égalité établie 
entre  les  expressions  déclarées  synonymes;  mais  cette  analycité  aura,  dans  la  connaissance,  le 
même  statut  que  la  définition  dont  elle  dépend.  L'on  voit  immédiatement  que  si,  par  jugement 
analytique, posant l'égalité de deux expressions, on veut concevoir un jugement permettant de les 
substituer,  chaque  fois,  l'une  à  l'autre,  sans  que  la  valeur  de  vérité  des  propositions  où  ces 
expressions  apparaissent  soit  modifiée,  l'analycité  d'un  jugement  ne  peut  être  affirmée  avec 
constance,  sans  risque  d'erreur,  que  dans  une  langue  où  de  nouveaux  usages  linguistiques  ne 
menacent plus de s'introduire, c'est-à-dire en définitive, dans un langage formalisé. » 
\bigskip
P  289 :  « Malgré  ces  restrictions,  nous  assistons,  spécialement  en  Grande-Bretagne,  au 
développement d'un mouvement philosophique, inauguré par G. Moore, pour lequel l'analyse des 
propositions  constitue  la  tâche  principale.  J.  Wisdom  a  pu  distinguer  trois  espèces  d'analyses  : 
l'analyse matérielle, l'analyse formelle et l'analyse philosophique (1). Les analyses matérielle (ex. : 
« A est enfant de B » signifie que « A est fils ou fille de B ») et formelle (ex. : « Le roi de France est 
chauve» équivaut à « il y a un être et un seul qui est roi de France et qui est chauve ») resteraient à 
un  même  niveau  du  discours,  alors  que  l'analyse  philosophique  -que  pour  cette  raison  L.  S. 
Stebbing appelle directionnelle (2) (ex. : « La forêt est très dense » équivaut à « les arbres de cette 
région sont très proches l'un de l'autre ») - se dirigerait dans un certain sens ; pour Stebbing, elle 
conduirait vers des faits fondamentaux, pour J. Wisdom, vers les données sensorielles. » 
\bigskip
(1) J. Wisdom, Logical constructions, Mind, 1931 à 1933 et A. H. S. Coom-Betennant, Mr. Wisdorn 
on philosophical Analysis, Mind, oct. 1936. 
(2) L. S. Stebbing, The method of analysis in metaphysies, Proceedings o/ the Aristotelian Society, 
vol. XXXIII, 1932-33. 
\bigskip
\bigskip
\bigskip
\bigskip
145 
\bigskip
P 289-290 : « Les distinctions établies par Wisdom nous semblent déjà présupposer une attitude 
philosophique. Il faut, au contraire, du point de vue argumentatif, souligner que toute analyse est 
directionnelle,  en  ce  sens  qu'elle  s'opère  dans  une  certaine  direction.  Le  choix  de  celle-ci  est 
déterminé  par  la  recherche  de  l'adhésion  de  l'interlocuteur.  En  effet,  sauf  dans  les  traités  de 
logique, on ne pratique pas d'analyse sans se proposer un but précis (1). Quand il s'agira d'analyse 
technique  conforme  aux  exigences  d'une  discipline,  elle  se  dirigera  vers  les  éléments  que  cette 
discipline considère comme fondamentaux; une analyse non-technique s'adaptera à l'auditoire et 
peut  donc  prendre  les  directions  les  plus  variées,  selon  les  objets  d'accord  admis  par  ce  dernier. 
Vouloir  imposer  à  l'auditoire  philosophique  des  critères  du  fait  ou  de  la  vérité,  qu'il  devrait 
admettre sans discussion, c'est déjà faire état d'une philosophie particulière, et raisonner dans les 
cadres établis par elle (2). » 
\bigskip
1) Cf. K. Britton, Communication, p. 139.  
(2) Cf. La discussion des faits et des vérités au § 16. 
\bigskip
P  290 :  « Toute  analyse,  dans  la  mesure  où  elle  ne  se  présente  pas  comme  purement 
conventionnelle, peut être considérée comme une argumentation quasi logique, utilisant soit des 
définitions soit une procédure par énumération, qui limite l'extension d'un concept aux éléments 
dénombrés. 
\bigskip
C'est  pourquoi,  en  dehors  d'un  système  formel,  l'analyse  ne  peut  jamais  être  définitive  ni 
exhaustive. Max Black reproche avec raison à Moore de ne pas indiquer de méthode pour réaliser 
les analyses qu'il préconise, ni pour reconnaître leur exactitude (3). En réalité, cette exactitude ne 
pourrait  même  pas  être  postulée,  si  l'on  prétend  reproduire  le  sens  des  notions  d'un  langage 
naturel. 
\bigskip
Si l'analyse paraissait indiscutable, entièrement assurée, ne pourrait-on lui faire le reproche de ne 
rien nous apprendre de neuf ? Toute argumentation quasi logique, dont on reconnaît le caractère 
évident et même nécessaire, risque ainsi, au lieu d'être critiquée comme faible et non-concluante, 
d'être attaquée comme manquant totalement d'intérêt, parce que ne nous enseignant rien de neuf : 
une pareille affirmation sera qualifiée de tautologie, parce que résultant du sens même des termes 
utilisés. » 
\bigskip
(3 ) Max Black. Philosophical Analysis, Introduction. 
\bigskip
P 291 : « Voici un texte de Nogaro, illustrant ce procédé : 
\bigskip
Pendant  longtemps  les  classiques  affirmèrent  ne  la  dépréciation,  ou  baisse  de  la  valeur  de  la 
monnaie, provoque la hausse des prix, sans prendre garde que baisse de valeur de la monnaie (par 
rapport  aux  marchandises)  et  hausse  des  prix  sont  deux  expressions  inversées  d'un  même 
phénomène, et qu'il y a là, par conséquent, non une relation de cause à effet, mais une tautologie 
(1). 
\bigskip
il 1 accusation de tautologie revient à présenter une affirmation comme résultat d'une définition, 
d'une convention purement linguistique, et ne nous enseignant rien quant aux liaisons empiriques 
qu'un phénomène peut avoir avec d'autres, et pour l'étude desquelles une recherche expérimentale 
serait  indispensable.  Elle  suppose  que  les  définitions  sont  arbitraires,  dépourvues  d'intérêt 
scientifique, et indépendantes de l'expérience. Mais dans la mesure où il n'en est pas ainsi, où les 
définitions sont liées à une théorie qui peut apporter des vues originales, l'accusation de tautologie 
perd de son poids. Au point que Britton assimile loi naturelle et tautologie. Il donne l'exemple d'un 
métal  inconnu,  nouvellement  défini  à  l'aide  de  certains  tests  qui  permettent  d'en  déceler  la 
présence  ultérieurement,  on  l'isole  et  on  détermine  son  point  de  fusion  la  nouvelle  propriété  est 
incorporée  dans  la  définition  et  y  prend  une  importance  primordiale  :  «  La  grande  découverte, 
\bigskip
\bigskip
\bigskip
146 
\bigskip
écrit Britton, est devenue une simple tautologie (2). » Une fois devenue tautologique, l'affirmation 
s'intègre  dans  un  système  déductif,  peut  être  considérée  comme  analytique  et  nécessaire,  et  ne 
semble plus liée aux aléas d'une généralisation empirique. » 
\bigskip
(1) B. Nogaro, La valeur logique des théories économiques, pp. 12-13.  
(2) K. Britton, communication, P. 179. 
\bigskip
P 291-292 : « C'est ainsi que la qualification de tautologie, appliquée à une proposition, isole celle-
ci du contexte qui a permis l'élaboration des notions sur lesquelles elle porte. Quand on intègre ces 
dernières  dans  la  pensée  vivante  qui  a  permis  leur  élaboration,  on  constate  qu'elles  ne  se 
caractérisent ni par la nécessité propre à un système formalisé ni par la trivialité dont on les accuse 
dans  une  discussion  non  formelle,  mais  que  leur  statut  est  lié  à  celui  des  définitions  qui  leur 
servent de fondement. » 
\bigskip
P  292 :  « Lorsque,  dans  une  discussion  non  formelle,  la  tautologie  paraît  évidente  et  voulue, 
comme  dans  les  expressions  du  type  «  un  sou  est  un  sou  »,  «  les  enfants sont  les  enfants  »,  elle 
devra être considérée comme une figure. On utilise alors une identité formelle entre deux termes 
qui ne peuvent être identiques si l'énoncé doit avoir quelque intérêt. L'interprétation de la figure, 
que nous appellerons tautologie apparente exige donc un minimum de bonne volonté de la part de 
l'auditeur. 
\bigskip
Ces énoncés ont depuis longtemps suscité l'attention des théoriciens du style. Voyant que les deux 
termes devaient avoir une signification différente, ils ont fait de ces tautologies des cas particuliers 
d'autres  figures  :  selon  Vico,  dans  la  figure  appelée  ploce  «(  Corydon  depuis  ce  temps  m'est 
Corydon ») le même terme est pris pour signifier la personne et pour signifier le comportement (ou 
la chose et ses propriétés) (1) ; selon Dumarsais dans « un père est toujours un père » le second 
terme  est un  substantif  pris  adjectivement  (2) ;  selon  Baron,  c'est  une  syllepse  oratoire,  l'un  des 
mots étant au sens propre, l'autre au figuré (3). » 
\bigskip
(1) Vico, Delle instituzioni oratorie, p. 142. 
(2) Dumarsais, Des Tropes, p. 173. 
(3) Baron, De la Rhétorique, p. 337. 
\bigskip
P  292-293 :  « Moins  soucieux  des  figures,  les  modernes  analysent  ce  genre  d'expressions  en 
fonction  de  leurs  préoccupations.  Parmi  les  remarques  les  plus  intéressantes,  citons  celles  de 
Morris  qui  souligne  la  distinction  entre  mode  formel  et  fonction  d'évaluation  (4),  celles  de 
Hayakawa pour qui c'est une façon d'imposer des directives de classification (1) et enfin celles de J. 
Paulhan,  qui  a  fort  bien  perçu  la  valeur  argumentative  de  pareilles  expressions,  mais  y  verrait 
volontiers un paradoxe de la raison (2). » 
\bigskip
(4) Ch. Morris, Signs, Language and Behavior, p. 171. 
(1) S. I. Hayakawa, Language in Thought and Action, pp. 213-214.  
(2) J. PAULHAN, Entretien sur des faits divers, p. 143. 
\bigskip
P 293 : « Ces propositions, parce que tautologiques, incitent à la distinction entre les termes. Mais 
il  serait  faux  de  croire  que  le  sens  exact  de  ceux-ci  soit  fixé  d'avance  ni,  surtout,  que  la  relation 
entre les termes soit toujours la même. La formule d'identité nous met sur la voie d'une différence 
mais  ne  spécifie  pas  sur  quoi  doit  porter  notre  attention.  Elle  n'est  qu'une  manière  formelle  du 
procédé  qui  consiste  à  valoriser  positivement  ou  négativement  quelque  chose  par  un  pléonasme, 
dont l'Ana de Madame Apremont nous donne un joli exemple : 
\bigskip
Quand je vois tout ce que je vois, je pense ce que je pense (3). 
\bigskip
\bigskip
\bigskip
\bigskip
147 
\bigskip
Ici, comme dans la répétition, c'est le second énoncé du terme qui porte la valeur (4). 
\bigskip
Notons que l'obligation de différencier les termes, au lieu de naître du souci de donner un sens à 
une  tautologie  exprimant  une  identité  peut  naître  d'une  autre  figure  quasi  logique,  basée  sur  la 
négation d'un terme Par lui-même, donc sur une contradiction : « Un sou n'est pas un sou » peut 
jouer  le  même  rôle  que  «  un  sou  est  un  sou  ».  L'identité  des  contradictoires  est  à  mettre  sur  le 
même plan, par exemple la célèbre maxime d'Héraclite : 
\bigskip
\bigskip
Nous entrons et n'entrons pas dans le même fleuve (5) » 
\bigskip
\bigskip
\bigskip
(3) M. Jouhandeau, Ana de Madame Apremont, p. 61.  
(4) Cf. § 42 : Les figures du choix, de la présence et de la communion.  
(5) V. plus loin § 94 :Enoncés incitant à la dissociation. 
\bigskip
P  293-294 :  « Les  tautologies  et  les  contradictions  ont  un  aspect  quasi  logique  parce  que,  au 
premier  abord,  on  traite  les  termes  comme  univoques,  comme  susceptibles  de  s'identifier,  de 
s'exclure.  Mais,  après  interprétation,  les  différences  surgissent.  Celles-ci  peuvent  être  connues 
préalablement  à  J'argumentation.  Dans  l'antanaclase  il  ne  s'agit  plus  que  d'un  emploi  de 
l'homonymie : 
\bigskip
Être aimé m'est cher à condition de ne pas coûter cher (1) » 
\bigskip
(1) Vico, Delle instituzioni oratorie, p. 142.  
\bigskip
P  294 :  « Ici  la  connaissance  des  usages  linguistiques  fournit  immédiatement  la  solution.  Mais 
dans les tautologies d'identité la différence n'est généralement pas fixée. En suivant sans doute des 
modèles déjà connus, nous pouvons créer une grande variété de différenciations et établir entre les 
termes une grande variété de relations. 
\bigskip
Si certaines de ces identités peuvent jouer le rôle de maximes «( une femme est une femme », peut 
être manière de poser que toutes les femmes se valent, mais aussi de poser qu'une femme doit se 
conduire comme une femme), elles n'acquièrent leur signification argumentative que lorsqu'elles 
s'appliquent à une situation concrète, qui seule donne à ces notions la signification particulière qui 
convient. » 
\bigskip
§ 32. LA REGLE DE JUSTICE 
\bigskip
« Les arguments que nous allons examiner dans ce paragraphe et dans le suivant concernent, non 
pas  une  réduction  complète  à  l'identité  des  éléments  que  l'on  confronte  les  uns  avec  les  autres, 
mais une réduction partielle permettant de les traiter comme interchangeables à un point de vue 
déterminé. 
\bigskip
La règle de justice exige l'application d'un traitement identique à des êtres ou à des situations que 
l'on intègre à une même catégorie. La rationalité de cette règle et la validité qu'on lui reconnaît se 
rattachent  au  principe  d'inertie,  duquel  résulte  notamment  l'importance  que  l'on  accorde  au 
précédent (2). » 
\bigskip
(2) Cf. § 27 : Accords propres à chaque discussion. 
\bigskip
P 294-295 : « Pour que la règle de justice constitue le fondement d'une démonstration rigoureuse. 
les  objets  auxquels  elle  s'applique  auraient  dû  être  identiques..  c'est-à-dire  complètement 
interchangeables.  Mais,  en  fait,  ce  n'est  jamais  le  cas.  Ces  objets  diffèrent  toujours  par  quelque 
aspect,  et  le  grand  problème,  celui  qui  suscite  la  plupart  des  controverses,  est  de  décider  si  les 
différences  constatées  sont  ou  ne  sont  pas  négligeables  ou,  en  d'autres  termes,  si  les  objets  ne 
\bigskip
\bigskip
\bigskip
148 
\bigskip
diffèrent  pas  par  les  caractères  que  l'on  considère  comme  essentiels,  c'est-à-dire  les  seuls  dont  il 
faille  tenir  compte  dans  l'administration  de  la  justice.  La  règle  de  justice  reconnaît  la  valeur 
argumentative de ce que l'un d'entre nous a appelé la justice formelle, d'après laquelle les « êtres 
d'une même catégorie essentielle doivent être traités de la même façon » (1). La justice formelle ne 
précise ni quand deux objets font partie d'une même catégorie essentielle ni quel est le traitement 
qu'il  faut  leur  accorder.  En  fait,  dans  toute  situation  concrète,  une  classification  préalable  des 
objets  et  l'existence  de  précédents  quant  à  la  façon  de  les  traiter  sera  indispensable.  La  règle  de 
justice fournira le fondement permettant de passer des cas antérieurs à des cas futurs, c'est elle qui 
permettra de présenter sous la forme d'une argumentation quasi logique l'usage du précédent. » 
\bigskip
(1) Ch. Perelman, De la justice, p. 27. 
\bigskip
P  295 :  « Voici  un  exemple  de  l'utilisation  de  la  règle  de  justice  dans  l'argumentation  ;  nous 
l'empruntons à Démosthène : 
\bigskip
Prétendraient-ils par hasard qu'une convention si elle est contraire à notre ville, est valable, alors 
que, si elle lui sert de garantie, ils refusent de la reconnaître ? Est-ce là ce qui vous semble juste ? 
Quoi ? Si quelque chose de ce qui a été juré est favorable à nos ennemis, mais nuisible pour nous, 
ils  en  affirmeront  la  validité;  et  si,  au  contraire,  il  s'y  trouve  une  stipulation  à  la  fois  juste  et 
avantageuse  pour  nous,  mais  défavorable  pour  eux,  ils  se  croient  obligés  de  la  combattre  sans 
relâche ! (2). » 
\bigskip
(2) Démosthène, Harangues, t. II : Sur le traité avec Alexandre § 18, 
\bigskip
P  295-296 :  « Si  ni  les  Athéniens,  ni  leurs  adversaires,  lie  jouissent  d'une  situation  privilégiée,  la 
règle  de  justice  demande  que  le  comportement  des  uns  et  des  autres,  comme  parties  dans  une 
convention, ne soit pas différent. L'appel à cette règle présente un aspect de rationalité indéniable. 
Quand on fait état de la cohérence d'une conduite, on fera presque toujours allusion au respect de 
la règle de justice. » 
\bigskip
P 296 : « Celle-ci suppose l'identification partielle des êtres, par leur insertion dans une catégorie, 
et l'application du traitement prévu pour les membres de cette catégorie. Or, c'est sur chacun de 
ces points que la critique pourrait porter, et empêcher le caractère contraignant de la conclusion. 
\bigskip
Tout  le  roman  de  Gheorghiu,  La  vingt-cinquième  heure,  est  une  protestation  contre  la 
mécanisation  des  hommes,  leur  désindividualisation  par  leur  insertion  dans  des  catégories 
administratives. Voici un passage où son humour macabre se révolte contre pareille réduction : 
\bigskip
Ces fractions d'hommes qui n'ont plus que des morceaux de chair, reçoivent la même quantité de 
nourriture que les prisonniers en parfaite possession de leur corps. C'est une grande injustice. je 
propose  que  ces  prisonniers  reçoivent  des  rations  alimentaires  proportionnelles  à  la  quantité  de 
corps qu'ils possèdent encore (1). 
\bigskip
Pour montrer le caractère arbitraire de toutes les classifications administratives, l'auteur introduit 
une suggestion bouffonne, propose de faire intervenir un élément horrible, la mutilation ; il tend 
ainsi à ridiculiser les classifications établies qui ne traitent pas les hommes comme des individus, 
selon un esprit de charité et d'amour, mais comme des membres interchangeables d'une classe. » 
\bigskip
(1) C. Virgil Gheorghiu, La vingt-cinquième heure, p. 274. 
\bigskip
P  296-297 :  « L'autre  critique,  celle  qui  concerne  la  manière  d'utiliser  la  règle  de  justice,  peut 
s'illustrer  en  partant  du  raisonnement  par  lequel  Locke  espère  inviter  ses  concitoyens  à  plus  de 
tolérance : 
\bigskip
\bigskip
\bigskip
149 
\bigskip
 
Aucun  homme  ne  se  plaint  du  mauvais  gouvernement  des  affaires  de  son  voisin.  Aucun  homme 
n'est irrité contre un autre pour une erreur commise en ensemençant son champ ou en mariant sa 
fille.  Personne  ne  corrige un  prodigue  qui  consomme  son  patrimoine  dans  les  tavernes...  Mais  si 
quelqu'homme  ne  fréquente  pas  l'Eglise,  s'il  ne  conforme  pas  là  sa  conduite  exactement  aux 
cérémonies habituelles, ou s’il n’amène pas ses enfants pour les faire initier aux mystères sacrés de 
telle ou telle congrégation, cela cause immédiatement un tumulte (1). » 
\bigskip
(1) Locke, The second treatise of civil goverrunent and A letter concerning toleration, p. 136. 
\bigskip
P  297  :  “Locke  voudrait  que  l'on  applique  la  même  règle  pour  les  affaires  religieuses  et  pour  les 
affaires civiles, et il se sert de la tolérance usuelle, à son époque, en ce qui concerne ces dernières, 
pour  inciter  à  la  même  tolérance  dans  les  questions  religieuses.  Mais,  aujourd'hui,  on  reculerait 
devant l'assimilation de ces situations différentes, de peur qu'elle ne conduise à une intervention 
de l'État  dans les affaires de conscience, analogue au dirigisme qui caractérise plusieurs secteurs 
de la vie économique. L'application de la règle de justice, suite à une assimilation préalable de deux 
espèces de situations, peut conduire à des résultats fort différents de ceux qu'on aurait souhaités. 
La règle, purement formelle, suppose, pour son application, une prise d'appui dans le concret, liée 
à des opinions et des accords rarement indiscutables. » 
\bigskip
§ 53. ARGUMENTS DE RECIPROCITE 
\bigskip
Les arguments de réciprocité visent à appliquer le même traitement à deux situations qui sont le 
pendant l'une de l'autre. L'identification des situations, nécessaire pour que soit applicable la règle 
de justice, est ici indirecte, en ce sens qu'elle requiert l'intervention de la notion de symétrie. 
\bigskip
Une relation est symétrique, en logique formelle, quand sa converse lui est identique, c'est-à-dire 
quand la même relation peut être affirmée entre b et a qu'entre a et b. L'ordre de l'antécédent et du 
conséquent peut donc être interverti. » 
\bigskip
P  297-298 :  « Les  arguments  de  réciprocité  réalisent  l'assimilation  de  situations  en  considérant 
que certaines relations sont symétriques. Cette intervention de la symétrie introduit évidemment 
des difficultés particulières dans 1 1 application de la règle de justice. Mais, par ailleurs, la symétrie 
facilite  l'identification  entre  les  actes,  entre  les  événements,  entre  les  être.,,,  parce  qu'elle  met 
l'accent sur un certain aspect qui paraît s'imposer en raison même de la symétrie mise en évidence. 
Cet aspect est ainsi présenté comme essentiel. » 
\bigskip
P 298 : « Parmi les exemples d'arguments, qu'Aristote considérait déjà comme tirés des « relations 
réciproques », nous trouvons celui du publicain Diomédon disant an sujet des impôts : 
\bigskip
Si les vendre n'est pas honteux pour vous, les acheter ne l'est pas non plus pour nous (1). 
\bigskip
Quintilien  fournit  comme  exemple  du  même  genre  de  propositions  «  qui  se  confirment 
mutuellement » : 
\bigskip
Ce qu'il est honorable d'apprendre, il est honorable aussi de l'enseigner (2). 
\bigskip
Par  un  raisonnement  de  même  nature,  La  Bruyère  condamne  les  chrétiens  qui  assistent  aux 
spectacles, puisque les comédiens sont damnés pour donner ces mêmes spectacles (3). 
\bigskip
Ces  arguments  de  réciprocité,  basés  sur  les  rapports  entre  l'antécédent  et  le  conséquent  d'une 
même relation, paraissent, plus que n'importe quels autres arguments quasi logiques, être à la fois 
formels  et  fondés  dans  la  nature  des  choses.  La  symétrie  est  supposée  le  plus  souvent  par  la 
qualification même des situations. » 
\bigskip
\bigskip
\bigskip
150 
\bigskip
 
(1) Aristote, Rhétorique, liv. 11, chap. 23, 1397 a. 
(2) Quintilien, Vol. II, liv. V, chap. X, § 78. 
(3 ) La Bruyère, Bibl. de la Pléiade, Caractères, De quelques usages, 21, p. 4:12. 
\bigskip
P 298-299 : « Cette influence de la qualification est manifeste dans certains arguments où elle est 
seule à commander la symétrie invoquée, tel cet argument de Rousseau : 
\bigskip
Point de mère, point d'enfant. Entre eux les devoirs sont réciproques ; et s'ils sont mal remplis d'un 
côté, il-, seront négligés de l'autre (1). » 
\bigskip
(1) Rousseau, Emile, p. 18. 
\bigskip
P 299 : « Les arguments de réciprocité peuvent aussi résulter de la transposition des points de vue, 
transposition qui permet de reconnaître, à travers leur symétrie, l'identité de certaines situations. 
\bigskip
La  possibilité  d'effectuer  pareilles  transpositions  est  considérée  par  Piaget  et,  à  sa  suite,  par 
certains  psychiatres  comme  une  des  aptitudes  humaines  primordiales  (2).  Elle  permet  de 
relativiser  des  situations  que  l'on  avait  considérées  jusque-là  comme  privilégiées,  si  pas  comme 
uniques. Puisque nous trouvons étranges les moeurs des Persans, ces derniers ne devraient-ils pas 
s'étonner  des  nôtres  ?  Les  coutumes  ridicules  des  pays  d'utopie,  complaisamment  décrites,  nous 
amènent à réfléchir sur ceux de nos usages qui leur font pendant et à les considérer comme tout 
aussi ridicules. 
\bigskip
Notons que, sous couleur de faire droit au point de vue d'autrui, ces arguments adoptent souvent le 
point  de  vue  d'un  tiers,  par  rapport  à  qui  la  symétrie  s'établirait  ;  c'est  l'intervention  de  ce  tiers 
impartial qui permet d'éliminer certains facteurs, tels le prestige de l'un des intéressés, capables de 
fausser la symétrie. 
\bigskip
Souvent une transposition, faisant ressortir la symétrie (mettez-vous à sa place !) sert de base à ce 
que l'on considère comme une application fondée de la règle de justice : celui qui a été généreux 
dans  l'opulence,  miséricordieux  dans  la  puissance,  sera,  semble-t-il,  en  droit  de  faire  appel  à  la 
générosité et à la miséricorde, quand la fortune lui est devenue défavorable (3). » 
\bigskip
(2) J. Piaget, Le jugement et le raisonnement chez l'enfant, pp. 2.32 et suiv.; La causalité physique 
chez l'enfant, pp. 278-280 ; cf. Ch.  Odier, Les deux sources, consciente  et  inconsciente, de  la vie 
morale, pp. 263-268. 
(3) Cf. Rhétorique à Herennius, liv. 11, § 25. 
\bigskip
P  300 :  « Certaines  règles  morales  s'établissent  en  fonction  de  la  symétrie:  Isocrate  loue  les 
Athéniens de ce que: 
\bigskip
Ils exigeaient d'eux-mêmes pour leurs inférieurs les mêmes sentiinents qu'ils demandaient à leurs 
supérieurs (1). 
\bigskip
Les préceptes de morale humaniste, qu'il s'agisse d'énoncés judéo-chrétiens «( Ne fais pas à autrui 
ce que tu ne voudrais pas que l'on te  fît ») ou de l'impératif catégorique de Kant «( Agis de telle 
sorte  que  la  maxime  de  ta  volonté  puisse  en  même  temps  toujours  valoir  comme  principe  d'une 
législation  universelle  »),  supposent  que  l'individu  ni  ses  règles  d'action  ne  peuvent  prétendre  à 
une  situation  privilégiée,  qu'il  est  au  contraire  régi  par  un  principe  de  réciprocité,  qui  paraît 
rationnel, parce que quasi logique. 
\bigskip
\bigskip
\bigskip
\bigskip
151 
\bigskip
Ce  principe  de  réciprocité,  fondé  sur  une  symétrie  de  situations,  peut  servir  d'argument,  même 
quand la situation à laquelle on se réfère n'est présentée que comme une hypothèse. C'est ainsi que 
Démosthène,  incitant  les  Athéniens  à  l'action  contre  Philippe,  imagine  ce  que  ce  dernier  aurait 
entrepris contre eux s'il avait été à leur place : 
\bigskip
... le mal qu'il vous ferait s'il le pouvait, ne serait-il pas honteux pour vous de ne pas le lui infliger 
quand vous en avez l'occasion, et cela faute d'oser ? (2). 
\bigskip
Ailleurs,  il  demande  aux  Athéniens  de  considérer  l'hypothèse  où,  Eschine  étant  l'accusateur  et 
Philippe  le  juge,  lui  Démosthène  se  conduirait  comme  Eschine,  et  de  juger  Eschine  comme  lui-
même eût été jugé par Philippe (3). » 
\bigskip
(1) Isocrate, Discours, t. II : Panégyrique d'Athènes, § 81. 
(2) Démosthène, t. I : Première Olynthienne, § 24. 
(3) Démosthène, t. III : Sur l'ambassade, § 214. 
\bigskip
P  300-301 :  « Le  portrait  du  diplomate  tracé  par  La  Bruyère,  dont  le  dessein  serait  toujours  la 
tromperie,  correspond  à  une  vue  assez  commune.  Mais  les  feintes  complaisamment  décrites  ne 
sont que des manières de se servir des symétries de situation ; la tâche du diplomate est d'arriver à 
ses fins avec de bonnes raison : l'argument de réciprocité, s'il n'est pas toujours exprimé, est l'un 
des  pivots  d'une  diplomatie  s'exerçant  d'égal  à  égal  ;  or  c'est  à  ce  cas  idéal  que  correspond  la 
description classique de La Bruyère (1). » 
\bigskip
Parfois  l'identification  de  situations  résulte  de  ce  que  deux  actes,  tout  en  étant  distincts,  ont 
concouru à un même effet : 
\bigskip
« Moi, j'ai accusé; vous, vous avez condamné », est une réplique célèbre de Domitius Afer (2). 
\bigskip
Deux conduites complémentaires, dans ce sens qu'elles constituent toutes les deux une condition 
nécessaire à la réalisation d'un effet déterminé, peuvent donner lieu à l'utilisation de l'argument de 
symétrie.  Un  exemple  de  cette  façon  d'argumenter  se  trouve  dans  la  démarche  du  ministre  des 
États-Unis à La Haye, de passage à Bruxelles pendant les premières semaines de la révolution de 
I83o,  en  vue  d'obtenir,  du  gouvernement  belge,  la  sortie  des  marchandises  appartenant  aux 
neutres  et  entreposées  à  Anvers.  Pour  être  efficace,  cette  autorisation  aurait  également  dû  être 
accordée par le roi de Hollande. De là l'argumentation du diplomate américain disant aux Belges: « 
Si vous accordez l'autorisation et si le roi de Hollande la refuse, quel prestige moral il en résultera 
pour  vous!  Si  vous  refusez  l'autorisation,  et  si  le  roi  de  Hollande  l'accorde,  quel  prestige  en 
découlera pour lui (3) ! » L'argumentation quasi logique devient possible à condition d'oublier tout 
ce qui différencie les situations et de les réduire à ce qui les rend symétriques. » 
\bigskip
(1) LA Bruyère, Bibl. de la Pléiade, Les Caractères, Du Souverain ou de la République, 12, pp. 295 
et suiv. 
(2) Quintilien, Vol. II, liv. V, chap. X, § 79. 
(3)  D'après  la  lettre  du  ministre  W.  P.  Preble  à  Martin  van  Buren,  secrétaire  d'Etat,  du  16  nov. 
1830,  reproduite  dans  Sophie  Perelman,  Introduction  aux  relations  diplomatiques  entre  la 
Belgique et les Etats-Unis, Bull. de la Commission royale d'Histoire, Bruxelles, 1949, p. 209. 
\bigskip
P 302 : « Des argumentations quasi logiques peuvent utiliser un autre type de symétrie résultant 
de ce que deux actions, deux conduites, deux événements, sont présentés comme l'inverse l'un de 
l'autre. On en conclut que ce qui s'applique à l'un - moyens nécessaires pour le réaliser, évaluation, 
nature de l'événement - s'applique à l'autre. 
\bigskip
Voici un passage du Pro Oppio, cité par Quintilien 
qu'il  n'a  pu faire  venir  malgré  eux  dans  la  province,  comment  a-t-il  pu  les y retenir  malgré 
eux ? (1). 
\bigskip
La pensée connue de Pascal : 
\bigskip
Peu de chose nous console, parce que peu de chose nous afflige (2). 
\bigskip
tire sa force de persuasion de cette même symétrie. 
\bigskip
De même Calviii, en partant du dogme de la rédemption du genre humain par la mort du Christ, y 
trouve  un  argument  lui  permettant  de  préciser  la  portée  du  dogme  du  péché  originel,  dont  le 
sacrifice du Christ devait combattre les effets : 
\bigskip
Que  babilleront  icy  les  Pelagielis,  que  le  peché  a  esté  espars  au  monde  par  l'imitation  d'Adam  ? 
N'avons-nous donc autre profit de la grace de Christ, sinon qu'elle nous est proposée en exemple 
pour  eusuyvre  ?  Et  qui  pourroit  endurer  tel  blaspheme  ?  Or  il  n'y  a  nulle  doute  que  la  grace  de 
Christ ne soit nostre par communication, et que par icelle nous n'ayons vie; il s'ensuit pareillement 
que l'une et l'autre a esté perdue en Adam, comme nous les recouvrons en Christ : et que le peché 
et la mort ont esté engendrez en nous par Adam, comme ils sont abolis par Christ (3). » 
\bigskip
(1) Quintilien, vol. II, liv. V, chap. X, § 76. 
(2) Pascal, Bibl. de la Pléiade, 175 (25*), p. 869 (136 éd. Brunschvicg). 
(3) Calvin, Institution de la religion chrétienne, liv. 11, chap. 1, 6. 
\bigskip
P  302-303 :  « Certain  usage  de 
l'argument  de  réciprocité,  parce  qu'il  conduit  à  des 
incompatibilités,  oblige  à  reconsidérer  la  situation  dans  son  ensemble.  Pascal  nous  y  conviera  à 
propos des jésuites : 
\bigskip
Vous  pensez  beaucoup  faire  en  leur  faveur  de  montrer  qu'ils  ont  de  leuxs  Pères  aussi  conformes 
aux maximes évangéliques que les autres y sont contraires; et vous concluez de là que ces opinions 
larges n'appartiennent pas à toute la Société. je le sais bien : car si cela était, ils n'en souffriraient 
pas  qui  y  fussent  si  contraires.  Mais  puisqu'ils  en  ont  aussi  qui  sont  dans  une  doctrine  si 
licencieuse,  concluez-en  de  même,  que  l'esprit  de  la  Société  n'est  pas  celui  de  la  sévérité 
chrétienne; car, si cela était, ils n'en souffriraient pas qui y fussent si opposés (1). » 
\bigskip
(1) Pascal, Bibl. de la Pléiade, Les Provinciales, Cinquième lettre, p. 473, 
\bigskip
P  303 :  « La  plupart  des  exemples  que  les  Anciens  nous  donnent  d'argumentation  Par  les 
contraires  aboutissent  à  une  généralisation  en  partant  d'une  situation  particulière  et  en  exigeant 
que l'on applique le même traitement à la situation symétrique : 
\bigskip
S'il n'est pas juste de se laisser aller à la colère envers qui nous a fait du mal contre son gré, celui 
qui nous a fait du bien parce qu'il y était forcé n'a droit à aucune reconnaissance (2). 
\bigskip
Nous trouvons un argument analogue dans un traité du XVIIIe siècle : 
\bigskip
Comment  soutenir  que  sur  une  preuve  suffisante  le  juge  doive  condamner  l'innocent  dont  en 
particulier il connoîtroit l'innocence; 
\bigskip
et  que  faute  de  preuves  suffisantes,  il  ne  doive  pas  absoudre  le  Coupable,  quand  même  en  son 
particulier il auroit connoissance de son crime (3) ? 
\bigskip
\bigskip
\bigskip
\bigskip
153 
\bigskip
L'usage  de  l'argument  de  réciprocité  est  à  la  base  d'une  généralisation  fréquente  en  philosophie, 
comme  celle  qui  affirme  que  tout  ce  qui  naît  meurt,  passant  ainsi  de  la  naissance  d'un  être  à  sa 
contingence (4). Montaigne en tire une leçon de morale : 
\bigskip
C'est pareille folle de pleurer de ce que d'icy à cent ans nous ne vivrons pas, que de pleurer de ce 
que nous ne vivions pas il y a cent ans (5). » 
\bigskip
(2) Aristote, Rhétorique, II, chap. 23, 1397 a. 
(3) Gibert, Jugemens des savans sur les auteurs qui ont traité de la Rhétorique, vol. III, p. 154. 
(4)  Cf.  Quintilien,  Vol.  11,  liv.  V,  chap.  X,  §  79,  et  Aristote,  Rhétorique,  II,  chap.  23,  1399  b., 
enthymème  XVII.  Cf.  §  48:  Techniques  visant  à  présenter  des  thèses  comme  compatibles  ou 
incompatibles. 
(5) Montaigne, Bibl. de la Pléiade, Essais, liv. I, chap. XX, p. 105. 
\bigskip
P 304 : « Cette leçon est-elle valable ? Y a-t-il ici abus de l'argument de symétrie ? Quelles sont les 
limites au delà desquelles l'application de cet argument devient inadmissible ? On peut se rendre 
compte nettement de leur transgression quand l'usage de cet argument produit un effet comique. 
Voici une des rares histoires qui semble avoir provoqué le rire de Kant : 
\bigskip
A  Surate,  un  Anglais  débouche  une  bouteille  d'ale,  qui  mousse  abondamment.  A  un  Indien  qui 
s'en étonne, il demande ce qu'il trouve là de si étrange. «Ce qui me frappe, ce n'est pas que tout 
cela s'échappe ainsi, répond l'indigène, mais c'est que vous ayez pu l'y faire entrer (1) » 
\bigskip
Cette histoire comique rappelle le passage du Pro Oppio cité plus haut ; elle en paraît la caricature. 
\bigskip
Laurence Sterne exploite d'une façon consciente cette même veine, le comique de l'argumentation, 
dans un passage de son Tristram Shandy : 
\bigskip
Eh! s'écria Kysarcius, qui a jamais eu l'idée de coucher avec sa grand-mère ? 
\bigskip
-  Ce  jeune  homme,  répliqua  Yorick,  dont  parle  Selden,  et  qui,  non  seulement  en  eut  l'idée  mais 
encore la justifia devant son père en se basant sur la loi du talion : «Vous couchez, lui dit-il, avec 
nia mère, pourquoi ne coucherais-je pas avec la vôtre ? » C'est un Argumentum commune, ajouta 
Yorick (2). » 
\bigskip
(1) Cité d'après Ch. Lalo, Esthétique (lu rire p. 159. 
(2) L. Sterne, Vie et opinions de Tristram Shandy, liv. IV. chap. XXIX, p. 275 
\bigskip
P  304-305 :  « Les  arguments  de  réciprocité,  on  le  voit  dans  ces  exemples  comiques,  ne  peuvent 
donc pas toujours être utilisés, car l'identification des situations, valable du point de vue où l'on se 
place,  peut  néanmoins  négliger  des  différences  essentielles.  Le  rejet  de  cette  sorte  d'arguments 
résultera  de  la  preuve  de  l'asymétrie  de  deux  situations.  Déjà  Aristote  soulignait  certains 
paralogismes de réciprocité à propos des actions subies et exécutées (3) ; d'autres montreront qu'il 
y a des notions qui ne peuvent s'appliquer normalement qu'à certaines situations, telle, selon Ryle, 
la  notion  d'acte  volontaire,  que 
indûment  étendue  des  actes 
répréhensibles aux actes méritoires (1). Souvent, on rejettera la symétrie parce que l'on attachera à 
l'une  des  situations  une  valeur  éminente  :  ce  qui  contribue  à  un  bien  est  généralement  moins 
apprécié que ce qui évite un mal. » 
\bigskip
(3) Aristote, Rhétorique, II chap. 23, 1397 a. 
(1) G. Ryle, The concept of mind, pp. 71-74. 
\bigskip
\bigskip
les  philosophes  auraient 
\bigskip
\bigskip
\bigskip
154 
\bigskip
P  305 :  « Les  conditions  d'application  de  pareilles  argumentations  ne  sont  donc  pas  purement 
formelles  :  elles  résultent  d'une  appréciation  sur  l'importance  des  éléments  qui  distinguent  des 
situations,  jugées  pourtant  symétriques  à  un  point  de  vue  déterminé.  Parfois  la  symétrie  de 
situation  est  complaisamment  évoquée  dans  le  seul  but  de  pouvoir  la  nier.  Ainsi  dans  ce  propos 
recueilli par Jouhandeau : 
\bigskip
Lévy, si j'avais su que vous étiez si riche, je ne vous aime pas, mais c'est vous, au lieu de Raymond, 
qui  m'auriez  épousée  et  je  vous  aurais  trompé  avec  lui,  jusqu'au  jour  où,  à  force  de  vous  voler, 
quand  nous  aurions  pu  être  heureux  ensemble  sans  vous,  je  vous  aurais  quitté,  mais  tout  s'est 
tourné autrement : je suis sa femme et vous auriez beau être encore plus riche, ni pour or ni pour 
argent, mon Raymond, je ne le tromperais avec vous (2). 
\bigskip
§ 54. ARGUMENTS DE TRANSITIVITE 
\bigskip
La  transitivité  est  une  propriété  formelle  de  certaines  relations  qui  permet  de  passer  de 
l'affirmation  que  la  même  relation  existe  entre  les  termes  a  et  b,  et  entre  les  termes  b  et  c,  à  la 
conclusion qu'elle existe entre les termes a et c : les relations d'égalité, de supériorité, d'inclusion, 
d'ascendance, sont des relations transitives. » 
\bigskip
(2) M. Jouhandeau, Un monde, p. 251. 
\bigskip
P  305-306 :  « La  transitivité  d'une  relation  permet  des  démonstrations  en  forme,  mais  quand  la 
transitivité  est  contestable  ou  quand  son  affirmation  exige  des  aménagements,  des  précisions, 
l'argument de transitivité est de structure quasi logique. C'est ainsi que la maxime «les amis de nos 
amis  sont  nos  amis  »  se  présente  comme  l'affirmation  que  l'amitié  est,  pour  qui  proclame  cette 
maxime, une relation transitive. Si l'on élève des objections  - basées sur l'observation, ou sur une 
analyse de la notion d'amitié - le défenseur de la maxime pourra toujours répliquer que c'est ainsi 
qu'il  conçoit  la  véritable  amitié,  que  les  vrais  amis  doivent  se  conduire  conformément  à  cette 
maxime. » 
\bigskip
P  306 :  « Celle-ci  nous  offre,  par  ailleurs,  un  bon  exemple  de  la  diversité  des  schèmes 
argumentatifs qui peuvent être en cause : au lieu d'un transfert du type a R b, b R c, donc a R c, on 
peut  y  voir  un  transfert  du  type  a  =  b,  b  =  c,  donc  a  =  c  (en  supposant  que  l'amitié  établit  une 
égalité entre certains partenaires - et cette égalité même peut être conçue non comme une relation 
mais comme l'appartenance à une classe) ; on peut y voir encore un transfert du type a R b, c R b, 
donc a R c (en supposant que l'amitié est une relation transitive et de plus symétrique). C'est sous 
ce  dernier  aspect  que  l'amitié  apparaît  lorsque  cette  exclamation  est  mise  dans  la  bouche  d'un 
jeune homme chassé tour à tour par son père et par son oncle, frères ennemis, pour avoir secouru 
l'un, puis l'autre : 
\bigskip
Qu'ils s'aiment l'un l'autre ! L'un et l'autre m'a aimé (1). 
\bigskip
Les arguments combinant transitivité et symétrie semblent avoir exercé beaucoup d'attrait sur les 
rhéteurs latins. Dans la même controverse, un autre défenseur du fils donne comme argument : 
\bigskip
J'ai bien mérité de votre père à tous deux, quoique son âge m'ait empêché de le connaître; lui aussi 
me doit un bienfait : j'ai donné du pain à ses deux fils (2). » 
\bigskip
(1) Sénèque, Controverses et suasoires, t. I : Controverses, liv. 1, 1, § 7. 
(2) Ibid., § 8. 
\bigskip
P 306-307 : « Le père et l'oncle étant antagonistes, on préfère lie point s'arrêter à eux comme relais 
des relations de bienveillance : l'argument suppose deux relations transitives et symétriques entre 
\bigskip
\bigskip
\bigskip
155 
\bigskip
le fils et le père, le fils et l'oncle, des relations de même nature entre le père et le grand-père, l'oncle 
et le grand-père, pour conclure à une relation de même nature entre le grand-père et le petit-fils. » 
\bigskip
P 307 : « La plupart de ces arguments peuvent non seulement être interprétés à l'aide de différents 
schèmes quasi logiques mais peuvent aussi être soutenus par des arguments basés sur la structure 
du  réel  (par  exemple  des  relations  de  moyen  à  fin,  le  bien  de  nos  amis  étant  notre  but,  nous 
apprécions tout ce qui peut les aider). Il semble cependant que, en première instance, on assiste à 
la  mise  en  oeuvre  du  schème  quasi  logique.  Ce  n'est  qu'à  la  réflexion,  si  le  raisonnement  quasi 
logique  est  contesté,  que  viendraient,  d'abord  une  justification  de  celui-ci,  et  ensuite  sans  doute 
des  arguments  basés  sur  le  réel  et  capables  d'appuyer  les  mêmes  affirmations.  La  force  de 
beaucoup d'arguments résulte de ce qu'une validité relative, précaire, douteuse, est soutenue par 
celle - tout aussi précaire bien entendu - d'arguments d'un autre genre. A partir du moment où l'on 
explicite le schème quasi logique les arguments adjuvants sont supprimés : la mise en forme, tout 
en fournissant un aspect contraignant, fait ainsi apparaître le raisonnement comme plus pauvre et 
plus faible qu'il ne l'est dans la réalité pratique. On sera tenté de croire que sa valeur est illusoire, 
quand on a dénoncé les réductions opérées, mais cela parce que l'on a détaché l'argument, par sa 
formalisation,  des  autres  arguments  qui  pouvaient  le  relayer  et  qui  sont  peut-être  pour  certains 
auditeurs, et à certains moments, dominants.  
\bigskip
Dans  l'exemple  ci-après,  on  voit  que,  à  l'interprétation  quasi  logique,  peut  se  superposer  une 
interprétation par les conséquences : 
\bigskip
... tandis que vous tenez pour vos meilleurs alliés ceux qui ont juré d'avoir même ennemi et même 
ami que vous, vous considérez qu'-entre les politiques ceux que vous savez certainement dévoués 
aux ennemis de la ville sont les plus dignes de confiance (1). » 
\bigskip
(1) Démosthène, Harangues, t. I : Pour la liberté des Rhodiens, § 33. 
\bigskip
P 308 : « Le caractère quasi logique est ici accentué par le passage, explicite, d'une relation à une 
autre, de la maxime « les ennemis de nos ennemis sont nos amis » à la conclusion « les amis de 
nos ennemis sont nos ennemis ». 
\bigskip
Les arguments basés sur les rapports d'alliance ou d'antagonisme entre personnes et entre groupes 
prennent  aisément  une  allure  quasi  logique,  les  mécanismes  sociaux  sur  lesquels  ils  s'appuyent 
étant bien connus et admis par tous.  E. Dupréel a même tenté de systématiser ce qu'il appelle la 
logique  des  conflits,  en  formulant,  à  ce  propos,  cinq  théorèmes  qui  ne  portent,  il  tient  à  le 
souligner, que sur des probabilités (1). Ces théorèmes concernent la propagation des antagonismes 
et la formation des alliances ; leur énoncé, quoique présenté sous forme d'équations algébriques, 
nous paraît relever de l'argumentation quasi logique. 
\bigskip
Ces  raisonnements  sont  appliqués  à  toutes  solidarités  et  antagonismes  et  pas  seulement  aux 
relations  entre  personnes  et  groupes:  les  rapports  entre  valeurs  sont  souvent  présentés  comme 
engendrant de nouvelles relations entre valeurs, sans que l'on ait recours à une autre justification 
que la transitivité, combinée, s'il le faut, avec la symétrie. » 
\bigskip
(1) E. Dupréel, Sociologie générale, pl). 140-145. 
\bigskip
P 308-309 : « L'usage de relations transitives est précieux dans les cas où il s'agit d'ordonner des 
êtres, des événements, dont la confrontation directe ne peut avoir lieu. Sur le modèle de certaines 
relations  transitives  comme  plus  grand  que,  Plus  lourd  que,  plus  étendu  que,  on  établit  entre 
certains  êtres  dont  les  caractères  ne  peuvent  être  connus  qu'à  travers  leurs  manifestations,  des 
relations  que  l'on  considère  comme  transitives.  Ainsi,  si  le  joueur  A  a  battu  le  joueur  B  et  si  le 
joueur B a battu le joueur C, on considère que le joueur A est supérieur au joueur C. Il se pourrait 
\bigskip
\bigskip
\bigskip
156 
\bigskip
que, dans une rencontre effective, le joueur C batte le joueur A. Mais cette rencontre est souvent 
impossible  à  réaliser  ;  le  système  des  épreuves  éliminatoires  l'exclut  en  tout  cas.  L'hypothèse  de 
transitivité est indispensable si l'on veut se passer d'une confrontation directe de tous les joueurs. 
Le classement qui résulte de ces relations transitives n'est d'ailleurs rendu possible que parce que 
l'on raisonne sur la personne en se basant sur certaines de ses manifestations. » 
\bigskip
P  309 :  « Une  relation  transitive  se  nourrir  de  semble  sous-tendre  l'énoncé  ci-après  qui  vise  à 
mettre en évidence une incompatibilité : 
\bigskip
Se pourrait-il que, le régime végétal étant reconnu le meilleur pour l'enfant, le régime animal fût le 
meilleur pour la nourrice ? Il y a de la contradiction à cela (1). 
\bigskip
Ce raisonnement est presque comique, parce que le terme de «nourrice » évoque une transitivité, 
étrangère sans doute à la pensée de Rousseau, qui ne peut oublier que le lait de la nourrice n'est 
pas une nourriture végétale. » 
\bigskip
(1) Rousseau, Emile, pp. 35-36. 
\bigskip
P  309-310 :  « L'une  enfin  des  relations  transitives  les  plus  importantes  est  la  relation 
d'implication.  La  pratique  argumentative  n'utilise  pas  toutes  les  implications  que  peut  définir  la 
logique formelle. Mais elle utilise largement la relation de conséquence logique. Le raisonnement 
syllogistique est essentiellement fondé sur la transitivité. Rien d'étonnant que les auteurs anciens 
aient essayé de mettre sous la forme  syllogistique les arguments qu'ils rencontraient : les termes 
d'enthymème  et  d'épichérème  correspondent,  grosso  modo,  aux  arguments  quasi  logiques 
présentés sous forme de syllogisme. Aristote qualifie d'enthymème (2) et Quintilien d'épichérème 
(3)  le  syllogisme  de  la  rhétorique.  Nous  n'entrerons  pas  dans  le  détail  de  leur  terminologie  -  il 
faudrait  sans  doute  montrer  l'influence  que  la  logique  stoïcienne  exerça  sur  les  modifications  de 
celle-ci  (1)  -  mais  nous  tenons  à  insister  sur  le  fait  que  l'assimilation  de  certains  arguments  au 
raisonnement  formel  jouait,  en  gros,  le  rôle  des  arguments  quasi  logiques  ;  c'est  d'ailleurs  de  la 
même façon qu'il faut comprendre les tentatives des juristes de mouler leurs raisonnements dans 
la forme syllogistique. Notre étude des raisonnements quasi logiques permettra de voir que ceux-ci 
sont beaucoup plus variés qu'on ne pouvait le croire. » 
\bigskip
(2) Aristote, Rhétorique, liv. 1, chap. 1, 1335 a ; liv. Il, chap. 22, 1395 b. 
(3) Quintilien, Vol. II, liv. V, chap. XIV, § 14. 
(1) Cf. Cicéron, Topiques, § 54 et suiv. 
\bigskip
P  310 :  « Notons  à  ce  propos  que  la  chaîne  syllogistique,  en  tant  que  relation  de  conséquence 
logique, est l'une des chaînes transitives qui semble présenter le plus d'attrait pour l'argumentation 
quasi  logique;  mais  le  syllogisme  peut  mettre  en  oeuvre  des  relations  d'égalité,  de  rapport  de  la 
partie  au  tout.  La  relation  transitive  d'implication  n'est  elle-même  que  la  résultante  d'autres 
relations  transitives.  Des  chaînes  transitives  peuvent  ainsi  se  bâtir  sur  des  relations  de 
conséquence logique, elles-mêmes diverses : c'est le cas normal de la plupart des raisonnements. » 
\bigskip
P 310-311 : « Il y a cependant un type de raisonnement qui, à cet égard, est caractéristique, que l'on 
trouve abondamment dans les écrits chinois et auquel certains auteurs donnent le nom  de sorite 
(nom  que  d'autres  réservent  au  paradoxe  du  tas  de  blé,  mot  grec  ;  nous  appellerons  l'un  sorite 
chinois, l'autre sorite grec, pour la commodité, réservant la question du rapport qui peut les unir) 
(2). En voici un exemple, pris au Tà Hio : 
\bigskip
Les  Anciens  qui  voulaient  faire  remplir  par  l'intelligence  son  rôle  éducateur  dans  tout  le  pays 
mettaient d'abord de l'ordre dans leur principauté; voulant mettre de l'ordre dans leur principauté, 
ils réglaient d'abord leur vie familiale ; voulant régler leur vie familiale, ils cultivaient d'abord leur 
\bigskip
\bigskip
\bigskip
157 
\bigskip
personne; désirant cultiver leur personne, d'abord ils rectifiaient leur coeur ; voulant rectifier leur 
coeur, ils cherchaient la sincérité dans leurs pensées ; cherchant la sincérité dans leurs pensées, ils 
s'appliquaient  d'abord  à  la  science  parfaite;  cette  science  parfaite  consiste  à  acquérir  le  sens  des 
réalités (1). » 
\bigskip
(2) Cf. § 66 : L'argument de la direction. 
(1) Le Tà Hio, Première Partie, § 4. Traduction proche de celles de J. Legge, The sacred books of 
the east, vol. XXVIII, pp. 411-412, et The chinese classics, vol. 1, 2e éd., pp. 357-358. Les autres 
interprétations, notamment de G. Pauthier, Les Sse Chou ou les quatre livres de philosophie 
morale et politique de la Chine, 1, pp. 21-23, gardent la marche générale du raisonnement. Mais les 
commentateurs, anciens et modernes, discutent pour savoir quel cri est le point central. 
\bigskip
P 311 : « Ce raisonnement est très strict dans sa forme, en ce sens que le dernier terme de chaque 
proposition est le premier ternie de la suivante - en chinois, le rythme accuse en outre les relations 
entre  propositions.  On  montre  qu'il  y  a  une  chaîne  possible  entre  la  valeur  que  l'on  prône  (la 
connaissance des choses) et les autres valeurs auxquelles on tient. Mais le passage de condition à 
conséquence est basé, à chaque étape, sur des relations différentes. Aussi la transitivité n'est elle, à 
nos yeux d'occidentaux tout au moins, que lâche et peu formelle. » 
\bigskip
§ 55. L'INCLUSION DE LA PARTIE DANS LE TOUT 
\bigskip
La relation d'inclusion donne lieu à deux groupes d'arguments qu'il y a intérêt à distinguer : ceux 
qui  se bornent  à  faire  état  de  cette inclusion  des  parties  dans  un  tout,  et  ceux  qui font  état  de  la 
division du tout en ses parties et des relations entre parties qui en résultent. » 
\bigskip
P 311-312 : « Les arguments quasi logiques du premier groupe, qui se bornent à confronter le tout 
avec  une  de  ses  parties,  n'attribuent  aucune  qualité  particulière  ni  à  certaines  parties  ni  à 
l'ensemble : on le traite comme semblable à chacune de ses parties ; on n'envisage que les rapports 
permettant  une  comparaison  quasi  mathématique  entre  le  tout  et  ses  parties.  Ceci  permet  de 
présenter des argumentations fondées sur le schème « ce qui vaut pour le tout vaut pour la partie 
», par exemple cette affirmation de Locke : 
\bigskip
Rien de ce qui n'est  pas permis par la loi à toute l'Eglise, ne peut, par aucun droit ecclésiastique, 
devenir légal pour aucun de ses membres (1). » 
\bigskip
(1) Locke, The second treatise of civil government and A letter concerning toleration, p. 135. 
\bigskip
P 312 : « Le plus souvent la relation du tout à ses parties est traitée sous l'angle quantitatif : le tout 
englobe la partie et, par conséquent, est plus important qu'elle ; souvent la valeur de la partie sera 
considérée  comme  proportionnelle  à  la  fraction  qu'elle  constitue  par  rapport  au  tout.  C'est  ainsi 
qu'Isocrate  utilise  l'argument  de  la  supériorité  du  tout  sur  ses  parties  pour  magnifier  le  rôle  des 
éducateurs des princes : 
\bigskip
Les  maîtres  qui  font  l'éducation  des  particuliers  ne  rendent  service  qu'à  leurs  élèves  ;  mais 
quiconque inclinerait vers la vertu les maîtres e la masse, rendrait service à la fois aux uns et aux 
autres, à ceux qui détiennent la puissance et à ceux qui sont sous leur autorité (2). 
\bigskip
Bien  des  raisonnements  philosophiques,  surtout  ceux  des  rationalistes,  sont  fondés  sur  une 
pareille argumentation. C'est là, en définitive, pour H. Poincaré, ce qui détermine la supériorité de 
l'objectif sur le subjectif : 
\bigskip
Ce que nous appelons la réalité objective, c'est, en dernière analyse, ce qui est commun à plusieurs 
êtres pensants, et pourrait être commun à tous ; ... (3). » 
\bigskip
\bigskip
\bigskip
\bigskip
158 
\bigskip
(2)  Isocrate,  Discours,  t.  II  A  Nicoclés,  §  8  ;  cf.  aussi  t.  II  :  Panégyrique  d'Athènes,  §  2  ;  t.  II  : 
Archidamos, 54 ; t. III ; Sur l'échange, § 79. 
(3) H. Poincarré, La valeur de la science, Introduction, p. 65. Cf. § 16 : Les faits et les vérités. 
\bigskip
P  312-313 :  « Un  type  de  raisonnement  basé  sur  l'inclusion,  fréquemment  utilisé,  concerne  le 
rapport entre ce qui comprend et ce qui est compris, au double sens du mot. Sous sa forme la plus 
simple,  il  consistera  à  déclarer  le  menteur  supérieur  à  ceux qu'il trompe, parce  que  « il  sait  qu'il 
ment » : les connaissances de ses interlocuteurs ne sont qu'une partie des siennes. Sous une forme 
plus subtile, c'est le schème qu'utilise Platon pour justifier la supériorité de l'ami de la sagesse sur 
l'ami  des  honneurs  et  l'ami  du  gain  (1).  En  philosophie,  nous  aurons  la  supériorité  de  celui  qui 
comprend l'autre, de celui (lui connaît, explique l'autre, sans que le contraire soit vrai. Ainsi, pour 
Merleau-Ponty, l'empirisme est atteint d'une sorte de cécité mentale, il est  
\bigskip
le  système  le  moins  capable  d'épuiser  l'expérience  révélée,  alors  que  la  réflexion  [c'est-à-dire  le 
criticisme] comprend sa vérité subordonnée en la mettant à sa place (2). » 
\bigskip
(1) Platon, La République, liv. IX, 582 b-583 b. 
(2) Merleau-Ponty, Phénoménologie de la perception, p. 33. 
\bigskip
P  313 :  « Le  philosophe,  par  rapport  ,tu  savant  '  surtout  le  philosophe  critique,  est  ainsi tenté  de 
s'attribuer souvent une supériorité dérivant de ce que son objet embrasse  la science, concerne les 
principes  qui  la  gouvernent,  alors  que  cette  dernière  ne  constitue  qu'une  partie  des  intérêts  de 
l'homme. Cela implique que l'on réduit la science ou les connaissances du spécialiste à n'être plus 
qu'une  partie  de  ce  que  soi-même  on  comprend.  Maintes  pensées  de  Pascal  sur  la  supériorité  de 
l'honnête homme expriment ce point de vue (3). Mais cela suppose, entre la partie et le tout, une 
sorte  d'homogénéité,  qu'il  suffit  de  refuser  pour  mettre  en  doute  cette  supériorité  du  non-
spécialiste.  Ce  refus  demande  cependant  un  effort  d'argumentation  assez  poussé,  alors  que  le 
schème  quasi  logique  entraîne  aisément  la  valorisation  du  tout,  de  ce  qui  comprend,  de  ce  qui 
explique la partie. » 
\bigskip
(3) Pascal, Bibl. de la Pléiade, Pensées,39, 40,41, 42 (129*, 440, 11, 49),pp. 832-33 (34, 35, 36, 37, 
éd. Brunschvicg). 
\bigskip
P 313-314 : « Les arguments dérivés de l'inclusion de la partie dans le tout, permettent de poser le 
problème  de  leurs  relations  avec  les  lieux  de  la  quantité,  que  nous  avons  examinés  parmi  les 
prémisses de l'argumentation. Les arguments quasi logiques sont toujours à notre disposition pour 
justifier  les  lieux  de  la  quantité,  si  ceux-ci  étaient  mis  en  discussion.  Par  ailleurs,  les  lieux  de  la 
quantité peuvent servir de prémisses à une argumentation d'allure quasi logique. Ce qui fait que, 
en  présence  d'un  raisonnement,  on  peut  tantôt  le  considérer  comme  l'application  d'un  lieu  de  la 
quantité, tantôt comme une argumentation quasi logique. 
\bigskip
Considérons ce passage de VI. Jankélévitch : 
\bigskip
L'économie opère selon la succession comme la diplomatie selon la coexistence; et comme celle-ci 
déterminait  le  sacrifice  de  la  partie  au  tout,  de  l'intérêt  local  à  l'intérêt  total,  ainsi,  par  ses 
aménagements temporels, celle-là détermine le sacrifice du présent au futur et de l'instant fugitif à 
la  plus  longue  durée  possible.  Peux-tu  vouloir  sans  absurdité  que  le  plaisir  d'une  seconde 
compromette les intérêts supérieurs de toute une vie ? (1). 
\bigskip
On  pourrait  se  demander  s'il  s'agit  ici  d'un  lieu  de  la  quantité  on  bien  d'un  raisonnement  quasi 
logique  :  c'est  uniquement  l'allusion  à  l'absurdité  d'un  certain  choix  qui  nous  fait  pencher  vers 
cette deuxième hypothèse. En effet, à un lieu de la quantité on pourrait toujours opposer un lieu de 
la qualité qui ne permettrait pas de considérer la partie et le tout comme homogènes ; c'est ce que 
\bigskip
\bigskip
\bigskip
159 
\bigskip
le  même  auteur  note,  quelques  pages  plus  loin,  quand  il  écrit,  pour  marquer  la  supériorité  de  « 
l'Aujourd'hui » sur un « Demain quelconque » : 
\bigskip
L'événement voluptueux, par son effectivité même, recèle un élément irrationnel et quodditatif que 
toutes les bonnes raisons de la raison ne suffisent pas à déterminer (2). 
\bigskip
Homogénéité, hétérogénéité des éléments que l'on compare ? Rien, si ce n'est une argumentation, 
confrontant les lieux et les raisons, et les éprouvant devant une conscience ainsi éclairée, ne nous 
permettra de décider, et de justifier cette décision à nos yeux et au regard d'autrui. » 
\bigskip
(1) V. Jankélévitch, Traité des Vertus, P. 18.  
(2) Ibid., p. 28. 
\bigskip
§ 56. LA DIVISION DU TOUT EN SES PARTIES 
\bigskip
P  315 :  « La  conception  du  tout  comme  la  somme  de  ses  parties  sert  de  fondement  à  une  série 
d'arguments  que  l'on  peut  qualifier  d'arguments  de  division  ou  de  partition,  tel  l'enthymème 
d'Aristote : 
\bigskip
Tous les hommes commettent l'injustice à trois fins (celle-ci, ou celle-ci, on celle-ci) ; et pour deux 
raisons le délit était impossible; quant à la troisième, les adversaires eux-mêmes n'en font pas état 
(1). 
\bigskip
On peut en rapprocher l'enthymème suivant : 
\bigskip
Un autre se tire des parties, comme, dans les Topiques, quelle espèce de mouvement est l'âme : ou 
celle-ci ou cette autre (2). 
\bigskip
Pourquoi Aristote dit-il que le deuxième enthymème se tire des parties, alors que le premier se tire, 
d'après  lui,  de  la  division  :  à  première  vue  ils  paraissent  indiscernables.  Pourtant,  quand  on  se 
reporte aux Topiques, selon les indications d'Aristote, on constate que le dernier enthymème vise 
surtout la division du genre en espèces (3) : 
\bigskip
Il faut examiner si, suivant l'une des espèces du mouvement, l'âme peut se mouvoir : si elle peut, 
par exemple, augmenter, ou se corrompre, ou devenir, ou avoir telle autre espèce de mouvement 
(4) » 
\bigskip
(1) Aristote, Rhétorique, liv. II, chap. 23, IX, 1398 a.  
(2) Ibid., liv. II, chap. 23, XII, 1399 a. 
(3) Aristote, Topiques, liv. II, chap. 4, III a.  
(4) Ibid., liv. II, chap. 4, III b. 
\bigskip
P 315-316 : « Dans ce dernier exemple on se trouve devant une argumentation qui, quoique proche 
de l'argument de division, en diffère néanmoins parce qu'elle s'appuie nettement sur le rapport qui 
existe entre le genre et les espèces : pour pouvoir affirmer quelque chose du genre, il faut que cela 
se  confirme  dans  l'une  des  espèces  ;  ce  qui  ne  fait  partie  d'aucune  espèce,  ne  fait  pas  partie  du 
genre.  On  retrouve  un  écho  de  la  distinction  entre  ces  deux  enthymèmes  chez  Cicéron,  quoique 
avec un vocabulaire assez différent de celui d'Aristote, à propos de la définition par énumération 
des parties (partes) on par analyse portant sur les espèces (formae) (1). Quintilien la reprend aussi 
en  insistant  sur  le  fait  que  le  nombre  de  parties  est  indéterminé,  celui  des  formes  par  contre  est 
déterminé : on ne peut dire de combien de parties se compose un État, mais on sait qu'il v a trois 
formes d'État, suivant que l'État est soumis au pouvoir du peuple, de quelques homines, d'un seul 
(2).» 
\bigskip
\bigskip
\bigskip
\bigskip
160 
\bigskip
(1) Cicéron, Topiques, §§ 28 à 30. 
(2) Quintilien, Vol. II, liv. V, chap. X, § 6:3. 
\bigskip
P  316 :  « On  voit  un  effort  constant  pour  distinguer  ce  que  -  à  en  juger  par  cet  effort  -  on  était 
enclin  à  confondre.  Nous  considérerons,  quant  à  nous,  que  dans  l'argument  par  division,  les 
parties  doivent  pouvoir  être  dénombrées  d'une  façon  exhaustive,  mais  qu'elles  peuvent  être 
choisies comme on le veut et de façon fort variée, à condition d'être susceptibles, par leur addition, 
de  reconstituer  un  ensemble  donné.  Dans  l'argumentation  par  espèces,  il  s'agit  de  divisions  sur 
lesquelles  on  est  d'accord,  qui  préexistent  à  l'argumentation,  qui  semblent  naturelles,  et  qu'il  ne 
faut  pas  nécessairement  énumérer  de  manière  exhaustive  pour  pouvoir  argumenter.  L'argument 
par  les  espèces,  qui  suppose  une  communauté  de  nature  entre  les  parties  et  l'ensemble,  peut  se 
rattacher  aux  arguments  d'inclusion  dont  nous  avons  traité  au  paragraphe  précédent.  Mais  il  se 
mue le plus souvent en argument par division, car on envisage les espèces comme reconstruisant 
par leur addition le genre. C'est pourquoi nous en traitons ici, au même titre que de l'argument par 
division. 
\bigskip
Pour  utiliser  efficacement  l'argument  par  division,  il  faut  que  l'énumération  des  parties  soit 
exhaustive, car, nous dit Quintilien : 
\bigskip
... si, dans les points énumérés, nous omettons une seule hypothèse, tout l'édifice s'écroule et nous 
prêtons à rire (3). » 
\bigskip
(3) Quintilien, Vol.II., liv. V, chap. X, § 67. 
\bigskip
P 317 : « Ce conseil de prudence attire notre attention sur le fait que l'argument par division n'est 
pas  purement  formel,  car  il  exige  une  connaissance  des  rapports  que  les  parties  entretiennent 
effectivement  avec  le  tout,  dans  le  cas  particulier  en  question.  Cette  technique  d'argumentation 
suppose  d'ailleurs  que  les  classes  formées  par  la  subdivision  d'un  ensemble  soient  dépourvues 
d'ambiguïté  :  or  ce  n'est  pas  toujours  le  cas.  Si  l'on  cherche  les  motifs  d'un  crime,  et  que  l'on  se 
demande si le meurtrier a agi par jalousie, par haine ou par cupidité, non seulement on n'est pas 
sûr d'avoir épuisé tous les motifs d'action, mais on n'est pas sûr d'être à même de répondre sans 
ambiguïté à chacune des questions particulières que soulève ce raisonnement. Ce dernier nécessite 
une  structure  univoque  et,  pour  ainsi  dire,  spatialisée  du  réel,  dont  seraient  exclus  les 
chevauchements,  les  interactions,  la  fluidité,  qui,  au  contraire,  ne  sont  jamais  absents  des 
arguments  que  nous  examinerons  plus  loin  dans  le  chapitre  consacré  aux  raisonnements  fondés 
sur la structure du réel. 
\bigskip
Que peut nous apporter l'argumentation par division ? En principe tout ce qui se tire d'opérations 
d'addition, de soustraction, et de leurs combinaisons. » 
\bigskip
P 317-318 : « L'effort peut tendre essentiellement à prouver l'existence de l'ensemble ;  c'est le cas 
dans  l'induction  aristotélicienne  et  dans  une  série  d'argumentations  par  énumération  de  parties. 
Notons à cet égard que ces formes d'argumentation peuvent donner lieu à figures : nous avons cité 
un  exemple  d'amplification  Par  congérie,  emprunté  à  Vico,  où  l'énumération  des  parties  a  pour 
effet d'augmenter la présence (1). Selon les cas, un même énoncé peut être argument par division 
ou amplification: prouver qu'une ville est tout entière détruite, à quelqu'un qui le nie, peut se faire 
en énumérant exhaustivement les quartiers endommagés. Mais si l'auditeur ne conteste pas le fait, 
ou ne connaît pas la ville, la même énumération sera figure argumentative de la présence. » 
\bigskip
(1) Cf. § 42 : Les figures du choix, de la présence et de la communion. 
\bigskip
P  318 :  « Dans  les  arguments  par  division  les  plus  caractéristiques,  l'effort  tend  à  prouver 
l'existence ou la non-existence d'une  des parties. On argumente par exclusion. Voici un exemple, 
\bigskip
\bigskip
\bigskip
161 
\bigskip
pris à Bergson, où ce dernier se demande quelle force peut jouer, dans la morale aspiration, le rôle 
que remplit la pression du groupe dans la morale sociale : 
\bigskip
Nous n'avons pas le choix. En dehors de l'instinct et de l'habitude, il n'y a d'action directe sur le 
vouloir que celle de la sensibilité (1). 
\bigskip
Le  même  schème  peut  sous-tendre  l'usage  des  tables  de  présence  et  d'absence,  tel  qu'il  est 
recommandé par Bacon et par Mill, encore que plus souvent elles doivent se rattacher aux tables 
de variations concomitantes (2). 
\bigskip
L'argument  par  division  est  à  la  base  du  dilemme,  forme  d'argument  où  l'on  examine  deux 
hypothèses pour en conclure que, quelle que soit celle que l'on choisit, on aboutit à une opinion, 
une conduite, de même portée, et cela pour l'une des raisons suivantes : ou bien elles conduisent 
chacune  à  un  même  résultat,  ou  bien  elles  conduisent  à  deux  résultats  de  même  valeur 
(généralement  deux  événements  redoutés),  ou  bien  elles  entraînent,  dans  chaque  cas,  une 
incompatibilité avec une règle à laquelle on était attaché. 
\bigskip
Nous emprunterons à Pascal un exemple de la première espèce de dilemme : 
\bigskip
Que  pouvaient  faire  les  juifs,  ses  ennemis  ?  S'ils  le  reçoivent,  ils  le  prouvent  par  leur  réception, 
car les dépositaires de l'attente du Messie le reçoivent; s'ils le renoncent, ils le prouvent par leur 
renonciation (3). » 
\bigskip
(1) Bergson, Les deux sources de la morale et de la religion, p. 35. 
(2) Cf. § 76 : L'argument de double hiérarchie. 
(3) Pascal, Bibl. de la Pléiade, Pensées, 521 (37), p. 979 (762 éd. Brunschvicg). 
\bigskip
P  319 :  « Pour  que  les  deux  cornes  du  dilemme  aboutissent  au  même  résultat,  il  faut  admettre 
l'équivalence  des  moyens  de  preuve  dont  on  fait  état,  car  dans  le  premier  cas  on  se  fonde  sur 
l'autorité des Juifs, dans le deuxième sur l'autorité des Écritures : mais si les deux autorités sont 
équivalentes, le raisonnement inverse qui ferait, lui, jouer les deux solutions contre le Messie, ne 
serait-il  pas  également  admissible  ?  Les  Anciens  avaient  examiné  pareille  réfutation  du  dilemme 
sous le nom de conversion (1). 
\bigskip
Que deux possibilités contradictoires conduisent à une même conclusion semble bien plus résulter 
d'une idée préconçue en faveur de cette dernière que de l'argumentation que l'on présente. C'est la 
raison pour laquelle pareil dilemme est souvent attribué à l'adversaire pour suggérer sa mauvaise 
foi. Dans la controverse au sujet de l'authenticité de la tiare de Saïtapharnès, Héron de Villefosse, 
défenseur de la tiare, s'exclame : 
\bigskip
Quand M. Furtwängler retrouve ou croit retrouver sur un monument antique une des figures ou un 
des motifs de la tiare, il déclare pour cette raison que la tiare est fausse ; quand il ne retrouve pas 
d'exemple  du  même  motif  ou  des  mêmes  figures...,  il  déclare  également  que  la  tiare  est  fausse. 
C'est un procédé de discussion tout à fait extraordinaire (2). 
\bigskip
La  deuxième  espèce  de  dilemme  tend  à  limiter  le  cadre  du  débat  à  deux  solutions,  toutes  deux 
désagréables,  mais  entre  lesquelles  le  choix  paraît  inévitable  ;  le  reste  de  l'argumentation 
consistera dans la preuve que la solution proposée constitue le moindre mal : 
\bigskip
Enfin,  Athéniens,  il  est  une  chose  que  vous  ne  devez  pas  perdre  de  vue  :  vous  avez  le  choix 
aujourd'hui entre ceci et cela, attaquer Philippe chez lui ou être attaqués par lui chez vous... Quant 
à  montrer  quelle  différence  il  y  a  entre  faire  la  guerre  chez  lui  et  la  faire  chez  nous,  est-ce 
nécessaire ? (3). 
\bigskip
\bigskip
\bigskip
162 
\bigskip
 
(1) Cicéron, De Inventione, liv. 1, § 83. 
(2) Vayson De Pradennes, Les fraudes en archéologie préhistorique, p. 533. 
(3) Démosthène, Harangues, t. 1 : Première Olynthienne, §§ 25, 27. 
\bigskip
P  320 :  « Nous  trouvons  un  exemple  comique  de  la  troisième  espèce  de  dilemme  dans  les 
réflexions que Sterne attribue aux jurisconsultes de Strasbourg devant le nez d'un étranger : 
\bigskip
Vrai, opinèrent-ils, un liez si monstrueux eût-été civilement intolérable; faux, il eût constitué une 
violation  plus  grave  et  plus  impardonnable  encore  des  droits  de  la  société  qu'il  cherchait  à 
tromper par son apparence abusive. 
La seule objection à ce dilemme fut que s'il prouvait quelque chose, c'était que le nez de l'étranger 
n'était ni vrai ni faux (1). 
\bigskip
Pour  réduire  une  situation  à  un  dilemme,  il  faut  que  les  deux  branches  en  soient  présentées 
comme  incompatibles,  parce  que  se  rapportant  à  une  situation  sur  laquelle  le  temps  n'a  pas  de 
prise et qui, par là même, exclut la possibilité d'un changement. Ce caractère statique du dilemme 
se marque bien dans les exemples suivants. Le premier, que la Rhétorique à Herennius attribue à 
un auteur laborieux, est l'argumentation d'une fille qu'un père voudrait séparer malgré elle, de son 
mari : 
\bigskip
Tu me traites, mon père, avec une rigueur que je ne mérite pas. En effet, si tu juges Chresphonte 
un méchant homme, pourquoi me le donnais-tu pour mari ? Si c'est, au contraire, un homme de 
bien, pourquoi nie forcer, malgré moi, malgré lui, à quitter un tel homme ? (2). » 
\bigskip
(1) L. Sterne, Vie et opinions de Tristram Shandy, p. 220. 
(2) Rhétorique à Herennius, liv. 11, § 38. 
\bigskip
P 320-321 : « L'autre dilemme est celui dont Démosthène veut accabler Eschine : 
\bigskip
Quant à moi, je demanderais volontiers à Eschine si, au moment où cela se passait et où la cité 
était remplie d'émulation, de joie et d'éloges, il s'associait aux sacrifices et à la satisfaction de la 
majorité, ou s'il restait chez lui, chagriné, gémissant, irrité du bonheur publie. S'il était présent et 
se  faisait  remarquer  au  milieu  des  autres,  n'agit-il  pas  maintenant  de  façon  scandaleuse,  ou 
plutôt  sacrilège,  quand,  ayant  pris  lui-même  les  dieux  à  témoins  de  l'excellence  de  ces  actes,  il 
prétend  vous  faire  voter  qu'ils  n'étaient  pas  excellents,  à  vous  qui  avez  juré  par  les  dieux  ?  S'il 
n'était pas présent, ne mérite-t-il pas mille fois la mort, puisqu'il souffrait de voir ce qui causait la 
joie des autres ? (1). » 
\bigskip
(1) Démosthène, Harangues et plaidoyers politiques, t. IV : Sur la couronne, § 217. 
\bigskip
P 321 : « La réduction de la situation à un schème quasi logique, qui exclut à la fois les nuances et 
l'influence du changement, permet de cerner l'adversaire dans l'alternative du dilemme, dont il ne 
pourra sortir qu'en faisant état d'un changement ou de nuances, qu'il s’agit chaque fois de justifier. 
\bigskip
Puisque  l'argument  par  division  suppose  que  l'ensemble  des  parties  reconstitue  le  tout,  que  les 
situations  envisagées  épuisent  le  champ  du  possible,  quand  les  parties  ou  les  possibilités  se 
limitent à deux, l'argument se présente comme une application du tiers exclu. On se sert de cette 
forme de la division lorsque, dans un débat, on limite les solutions à deux : celle de l'adversaire et 
celle que l'on défend soi-même. Après avoir ridiculisé la thèse de l'adversaire - que l'on crée parfois 
de toutes pièces pour les besoins de la cause - on se prévaut de celle que l'on propose comme étant 
la  seule  possible.  Une  technique  quelque  peu  différente  consiste  à  présenter  une  thèse  comme 
fournissant  la  réponse  au  problème,  toutes  autres  hypothèses  étant,  en  bloc,  rejetées  dans 
\bigskip
\bigskip
\bigskip
163 
\bigskip
l'indéterminé. Seule la thèse que l'on développe jouit de la présence. Parfois, après l'avoir exposée, 
on  s'adresse  aux  auditeurs,  en  leur  demandant  s'ils  ont  une  meilleure  solution  à  présenter.  Cet 
appel,  classiquement  qualifié  d'argument  ad  ignorantiam,  tire  sa  force  essentiellement  de 
l'urgence, car il exclut un délai de réflexion : le débat se circonscrit à la thèse présentée et à celle 
que  l'on  pourrait  éventuellement  lui  opposer  sur-le-champ.  Par  là,  cet  argument,  pour  être 
utilisable, place les interlocuteurs dans un cadre limité qui rappelle celui du dilemme. » 
\bigskip
P  322 :  « Les  arguments  par  division  impliquent  évidemment  tous,  entre  les  parties,  certains 
rapports  qui font  que  leur  somme  soit  à  même  de  reconstituer  l'ensemble.  Ces  rapports  peuvent 
être  liés  à  une  structure  du réel  (par  exemple  celui  entre  les  différents  quartiers  d'une  ville)  ;  ils 
peuvent aussi être de nature surtout logique. La négation joue à cet égard un rôle essentiel ; c'est 
elle  qui  semble  garantir  que  la  division  est  exhaustive.  Ainsi,  dans  cet  argument  éristique 
(Kunstgriff 13) que Schopenhauer expose en ces termes : 
\bigskip
Pour  faire  admettre  une  proposition  par  l'adversaire,  il  faut  y  joindre  son  contraire  et  laisser  à 
l'adversaire  le  choix  ;  ce  contraire  sera  formulé  d'une  façon  assez  crue  pour  que,  ne  voulant  pas 
être  paradoxal,  l'interlocuteur  accepte  notre  proposition,  qui  par  rapport  à  l'autre  paraît  très 
plausible.  Par  exemple,  pour  qu'il  concède  que  l'on  doit  faire  tout  ce  que  le  père  ordonne,  on 
demandera : « Faut-il en toutes choses obéir ou désobéir à ses parents ? » (1). 
\bigskip
La  thèse  proposée  comme  repoussoir  est  formée  par  la  négation  de  l'autre  -  ou  tout  au  moins  la 
négation de certains de ses éléments. L'artifice est manifeste. Mais notons que Pascal n'hésite pas à 
recommander  l'argumentation  par  division  entre  les  deux  possibilités,  que  constituent  une 
proposition et son contraire : 
\bigskip
... toutes les fois qu'une proposition est inconcevable, il faut en suspendre le jugement et ne pas la 
nier à cette marque, mais en examiner le contraire ; et si on le trouve manifestement faux, on peut 
hardiment affirmer la première, tout incompréhensible qu'elle est (2). 
\bigskip
Utilisée  comme  le  fait  Pascal,  pour  prouver  l'infinie  divisibilité  de  l'espace,  cette  argumentation 
quasi logique est elle-même basée sur une division exhaustive d'un ensemble donné. 
\bigskip
Semblable argumentation est d'habitude considérée comme allant de soi. La Bruyère écrit : 
\bigskip
L'impossibilité où je suis de prouver que Dieu n'est pas me découvre son existence (3). » 
\bigskip
(1) Schopenhauer,  éd. Piper Nul. Eristische Dialektik, p. 41-1. 
(2) Pascal,  Bibl. de la Pléiade, De l'esprit géométrique, p. 369. 
(3) La Bruyère, Bibl. de la Pléiade, Caractères, Des esprits forts, 13, p. 47-1. 
\bigskip
P  323 :  « C'est  sur  ce  type  de  raisonnement  qu'est  basée  généralement,  comme  l'a  montré  E. 
Dupréel, la notion de nécessité en philosophie (1). 
\bigskip
La  disjonction  affirmée  entre  deux  termes  qui  ne  sont  pas  formellement  contradictoires  indique 
souvent  que  l'orateur  assume,  par  là,  qu'il  identifie  une  des  branches  de  l'alternative  avec  la 
négation de l'autre. Quand Gide, dans l'exemple cité plus haut  (2) dit à propos de  la Bible et des 
Mille  nuits  et  une  nuit  :  «  On  peut  aimer  ou  ne  comprendre  point  »,  il  identifie  «  ne  pas 
comprendre  »  et  «  ne  pas  aimer  »,  disqualifiant  d'ailleurs,  par  là,  cette  branche  négative  de 
l'alternative. 
\bigskip
Et lorsque H.. Lefebvre écrit 
\bigskip
\bigskip
\bigskip
\bigskip
164 
\bigskip
La  pensée  logique  et  scientifique  est  objective...  ou  n'est  rien.  De  même,  pensée  universelle...  ou 
elle n'est rien (3). 
\bigskip
il semble, par l'identification entre « rien » et « pas objective », « pas universelle », donner valeur 
absolue à la définition proposée. 
\bigskip
Cette  identification  de  la  branche  négative  à  quelque  chose  de  méprisable  -  peut  elle-même  être 
réalisée  par  un  dilemme.  Locke  s'opposant  aux  guerres  de  religion  et  à  l'intolérance  des  prêtres 
chrétiens, écrit : 
\bigskip
Si quelqu'un qui professe être le ministre de la parole de Dieu, le prédicateur de l'Evangile de la 
Paix,  enseigne  le  contraire,  ou  bien  il  ne  comprend  pas  ou  bien  il  néglige  les  devoirs  de  sa 
vocation, et devra en rendre compte un jour au Prince de la paix (4) 
\bigskip
Pareil  dilemme  peut  être  utilisé  comme  figure.  La  Rhétorique  à  Herennius  donne  l'exemple  ci-
après d'hésitation (dubitatio) : 
\bigskip
A  cette  époque  la  république  a  subi  un  grand  préjudice  du  fait  des  consuls,  faut-il  dire  par  leur 
sottise, par leur perversité, ou par l'un et par l'autre ? (5). » 
\bigskip
(1) E. Dupréel, Essais pluralistes (De la nécessité), p. 77. 
(2) Cf. § 32 : Le choix des qualifications. 
(3) H. Lefebvre, A la lumière du matérialisme dialectique, I, p. 43. 
(4) J. Locke, The second treatise of civil government and A letter concerning toleration, p. 136. 
(5) Rhétorique à Herennius, liv. IV, § 40. 
\bigskip
P  324 :  « Il  ne  s'agit  pas  d'une  simple  hésitation  au  sujet  d'une  qualification;  c'est  figure  de 
présence  bien  plus  que  figure  de  choix.  Nous  savons  déjà  que  l'amplification  est  perçue  comme 
figure  lorsqu'elle  use  de  schèmes  argumentatifs  caractérisés  (1).  Ici  le  dilemme  se  résout  en 
disjonction non exclusive. 
\bigskip
Le rapport entre les deux parties formant un tout peut aussi être celui de complémentarité. 
\bigskip
Sera  complémentaire  d'une  notion  ce  qui  est  indispensable  pour  expliquer,  justifier,  permettre 
l'emploi  d'une  notion  :  c'est  ce  que  E.  Dupréel  appelle  une  notion-béquille  (2).  Mais  sera  aussi 
complémentaire  ce  qui,  ajouté  à  la  notion,  reconstitue  toujours  un  tout,  quelles  que  soient  les 
fluctuations dans l'application  de celle-ci. Ces deux  aspects  de la complémentarité sont d'ailleurs 
liés. 
\bigskip
L'évêque Blougram montre que croyance et incroyance sont complémentaires : 
\bigskip
Tout ce que nous avons gagné par notre incroyance  
Est une vie de doute diversifiée par la foi  
Pour une vie de foi diversifiée par le doute  
Nous appelions l'échiquier blanc, - nous l'appelons noir (3). 
\bigskip
Une  affirmation  et  sa  négation  sont,  en  un  sens,  toujours  complémentaires.  Mais  en  mettant 
l'accent  sur  la  complémentarité,  on  élimine  l'idée  d'opposition  et  d'indispensable  choix  pour 
aboutir au contraire à l'idée que le choix est indifférent. » 
\bigskip
(1) Cf. § 42 : Les figures du choix, de la présence et de la communion.  
(2) E. Dupréel, Esquisse d'une philosophie des valeurs, pp. 68-69.  
(3) R. Browning, Poems, Bishop Blougram's Apology, p. 140. 
\bigskip
\bigskip
\bigskip
165 
\bigskip
All we have gained then by our unbelief  
Is a life of doubt diversified by faith,  
For one of faith diversified by doubt  
We called the chess-board white, - we call it black. 
Cf. § 43 : Le statut des éléments d'argumentation et leur présentation. 
\bigskip
P  325 :  « Les  négations  utilisées  dans  les  dilemmes  pourraient  par  ce  biais,  se  rattacher  à  la 
complémentarité. 
\bigskip
L'importance  qu'a  la  manière  dont  est  perçu  le  rapport  entre  parties  formant  un  tout,  est 
particulièrement marquée dans les arguments a Pari et a contrario, bien connus dans la tradition 
juridique. Ils traitent de l'application ou de la non-application, à une autre espèce du même genre, 
de ce qui a été affirmé pour une espèce particulière. Prenons un exemple. Une loi édicte certaines 
dispositions  relatives  aux  fils  héritiers  ;  grâce  à  l'argument  a  Pari  on  cherche  à  étendre  ces 
dispositions  aux  filles  ;  l'argument  a  contrario,  par  contre,  permet  de  prétendre  qu'elles  ne 
s'appliquent pas aux personnes de sexe féminin. Dans le premier cas, la loi est considérée comme 
un exemple d'une règle qui concerne le genre tout entier; dans le deuxième, elle est conçue comme 
une exception à une règle sous-entendue concernant le genre. 
\bigskip
L'argument a Pari est perçu comme une identification; l'argument a contrario comme division. Il 
faut  cependant  noter  que  dans  la  mesure  où  l'identification  a  pari  est  affirmée  comme  étant 
l'assimilation  de  deux  espèces  d'un  même  genre,  elle  donne  prise  à  l'argument  a  contrario  : 
l'argument  quasi  logique  suscite  l'argument  quasi  logique  de  l'adversaire  ;  dans  la  mesure  où 
l'identification se fait par d'autres moyens, elle risque moins de susciter la réplique par l'argument 
a contrario. 
\bigskip
Comment faut-il interpréter les textes légaux ou les décisions de la jurisprudence ? A Priori, il n'est 
pas  possible  de  le  savoir.  Seul  le  contexte,  l'appréciation  de  la  situation,  la  détermination  du  but 
poursuivi  par  les  dispositions  légales  ou  les  décisions  jurisprudentielles,  permettra  dans  chaque 
cas de faire prévaloir l'une ou l'autre technique argumentative, de préférer l'identification de deux 
espèces  à  leur  opposition,  ou  inversement.  Cette  conclusion  met  bien  en  évidence  la  différence 
entre l'argumentation quasi logique et la démonstration formelle. 
\bigskip
§ 57. LES ARGUMENTS DE COMPARAISON 
\bigskip
P  326 :  « L'argumentation  ne  saurait  aller  bien  loin  sans  recourir  à  des  comparaisons,  où  l'on 
confronte plusieurs objets pour les évaluer l'un par rapport à l'autre. En ce sens, les arguments de 
comparaison  devront  être  distingués  aussi  bien  des  arguments  d'identification  que  du 
raisonnement par analogie. 
\bigskip
En affirmant « Ses joues sont rouges comme  des pommes », aussi bien qu'en affirmant « Paris a 
trois  fois  plus  d'habitants  que  Bruxelles  »I  «  Il  est  plus  beau  qu'Adonis  »,  nous  comparons  des 
réalités  entre  elles,  et  cela  d'une  façon  qui  semble  bien  plus  susceptible  de  preuve  qu'un  simple 
jugement  de  ressemblance  ou  d'analogie.  Cette  impression  tient  à  ce  que  l'idée  de  mesure  est 
sousjacente  dans  ces  énoncés,  même  si  tout  critère  pour  réaliser  effectivement  la  mesure  fait 
défaut;  par  là  les  arguments  de  comparaison  sont  quasi  logiques.  Ils  sont  présentés  souvent 
comme des constatations de fait, alors que le rapport d'égalité ou d'inégalité affirmé ne constitue 
souvent qu'une prétention de l'orateur. Ainsi : 
\bigskip
Le crime est le même ou de voler l'Etat ou de faire des largesses contraires à l'intérêt publie (1). 
\bigskip
est  une  affirmation  qui  rapproche  de  ce  qui  est  un  délit  avéré,  une  action  qui  n'est  pas  qualifiée 
légalement, et dont l'égalité avec la précédente n'est donc pas préalable à l'argumentation. 
\bigskip
\bigskip
\bigskip
\bigskip
166 
\bigskip
L'idée de mesure, sous-jacente aux arguments de comparaison, se traduit cependant souvent par 
l'énoncé de certains critères. » 
\bigskip
(1) Cicéron, De Oratore, liv. II, § 172. 
\bigskip
P  326-327 :  « Les  comparaisons  peuvent  avoir  lieu  par  opposition  (le  lourd  et  le  léger),  par 
ordination  (ce  qui  est  plus  lourd  que)  et  par  ordination  quantitative (en  l'occurrence  la  pesée  au 
moyen d'unités de poids) (2). Lorsqu'il s'agit de notions empruntées à l'usage commun, les critères 
sont généralement complexes : la décomposition d'une notion, telle que l'opèrent les statisticiens 
(par  exemple  la  mesure  du  degré  d'instruction  en  se  basant  sur  le  nombre de  personnes  sachant 
lire,  le  nombre  de  bibliothèques,  de  publications,  etc.)  constitue  un  effort  pour  faire  droit  aux 
divers  éléments  mesurables.  La  combinaison  des  critères  est réalisée  de  manières  diverses.  Ainsi 
chez saint Thomas nous trouvons la combinaison suivante : 
\bigskip
Les  êtres  inférieurs,  en  effet,  sont  naturellement  incapables  d'atteindre  une  complète  perfection, 
mais ils atteignent un degré médiocre d'excellence au moyen de quelques  mouvements. Ceux qui 
leur  sont  supérieurs  peuvent  acquérir  une  complète  perfection  au  moyen  d’un  grand  nombre  de 
mouvements.  Supérieurs  encore  aux  précédents  sont  les  êtres  qui  atteignent  leur  complète 
perfection  par  un  petit  nombre  de  mouvements,  le  plus  haut  degré  appartenant  à  ce  qui  la 
possèdent sans exécuter de mouvements pour l'acquérir (1). » 
\bigskip
(2) Hempel et Oppenheim, Der Typusbegriff im Lichte der neuen Logik. 
(1) E. Gilson, Le thomisme p. 281 (cf. Sum. theol. I, 77, 2, ad Resp.). 
\bigskip
P  327 :  « Cette  graduation  qui  combine  les  degrés  de  perfection  et  les  moyens  utilisés  pour 
l'acquérir  permet  à  saint  Thomas  de  mettre  au  sommet  Dieu,  puis  les  anges,  les  hommes,  les 
animaux. Cette construction métaphysique s'appuie sur une analogie où joue également ce double 
critère  :  il  s'agit  des  degrés  de  la  santé,  comparés  en  tenant  compte  du  résultat  obtenu  et  des 
remèdes nécessaires pour l'obtenir. 
\bigskip
 Les critères entrent souvent en conflit: le procédé de saint Thomas établit, pour chaque palier de 
l'un des critères, une hiérarchie basée sur le second critère  (2),  ce qui implique évidemment une 
prédominance  du  premier  sur  le  second.  Et  pourtant,  la  nécessité  même  d'introduire  ce  dernier 
montre que le premier était reconnu insuffisant. » 
\bigskip
(2) Cf. 1 20 : Les hiérarchies. 
\bigskip
P  327-328 :  « Les  combinaisons  les  plus  diverses  peuvent  être  envisagées  elles  ne  refléteront 
jamais  la  complexité  des  notions  non  formalisées.  Bien  plus,  dans  beaucoup  de  cas  il  y  a  une 
proportionnalité inverse entre les critères appelés à entrer en ligne de compte : le mérite est, pour 
le sens commun, fait de disposition innée au bien et de sacrifice : une hiérarchisation des mérites 
doit envisager ces facteurs incompatibles. » 
\bigskip
P 328 : « Dès qu'il y a comparaison entre éléments non intégrés dans un système, les termes de la 
comparaison, quelle qu'elle soit, interagissent l'un sur l'autre et ceci de deux manières. 
\bigskip
D'une part, le niveau absolu du terme étalon pourra influer sur la valeur des ternies appartenant à 
la  même  série  et  qui  lui  sont  comparés.  Cet  effet  est  observé  dans  la  perception;  notons  que  les 
répétitions  des  termes  confrontés  Y  concourent  toutes  semble-t-il  à  situer  un  niveau  neutre 
d'adaptation (1). Il en va de même sans doute dans l'argumentation où les termes déjà mis en avant 
constituent un arrière-fonds qui influe sur les évaluations nouvelles. 
\bigskip
\bigskip
\bigskip
\bigskip
167 
\bigskip
D'autre  part,  la  confrontation  peut  rapprocher  deux  termes  que  l'on  était  fondé  à  considérer 
comme incommensurables. La comparaison entre Dieu et les hommes jouera à la fois à l'avantage 
du  terme  inférieur  et  au  détriment  du  terme  supérieur.  Les  partisans  de  l'amour  divin,  tout  en 
méprisant l'amour terrestre, ne peuvent que valoriser celui-ci par la comparaison qu'ils établissent 
entre les deux : 
\bigskip
[L'âme], dira Plotin, purifiée des souillures de ce inonde et préparée à retourner vers son père, elle 
est dans la joie. Pour ceux qui ignorent cet état, qu'ils imaginent d'après les amours d'ici-bas ce que 
doit être la rencontre de l'être le plus aimé, les objets que nous aimons ici sont mortels et caducs ; 
nous n'aimons que des fantômes instables; et nous ne les aimons pas réellement ; ils ne sont pas le 
bien que nous cherchons (2). » 
\bigskip
(1) Cf- H. Helson, Adaptation-level as a basis for a quantitative theory of frames of reference, The 
Psychol. Review, nov 1948 , p. 302.  
(2) Plotin, t. VI, 2, Ennéade Vl, 9, § 9, p. 185. 
\bigskip
P 328-329 : « Cette valorisation de l'inférieur, des orateurs tels que Bossuet la soulignent même, 
pour en tirer des effets argumentatifs : 
\bigskip
... les souverains pieux, veulent bien que toute leur gloire s'efface en présence de celle de Dieu; et 
bien loin de s'offenser que l'on diminue leur puissance dans cette vue, ils savent qu'on ne les révère 
jamais plus profondément que lorsqu'on ne les rabaisse qu'en les comparant avec Dieu (1). » 
\bigskip
(1) Bossuet, Sermons, t. II : Sur l'ambition, p. 395. 
\bigskip
P 329 : « De même, c'est un honneur pour un poète médiocre que d'être déclaré très inférieur à un 
maître renommé : dès lors il entre, même si ce n'est pas pour y occuper une place en vue, dans la 
confrérie des poètes illustres. 
\bigskip
Par  contre,  tout  ce  qui  est  confronté  avec  des  objets  très  inférieurs  ne  peut  que  souffrir  de  ce 
rapprochement;  c'est  pourquoi  Plotin,  après  avoir  marqué  la  supériorité  de  l'Un  sur  toute  autre 
réalité, mais craignant la dévaluation qui en résulte pour lui, ajoute : 
\bigskip
Ecartons donc de lui toute chose; ne disons pas même que les choses dépendent de lui et qu'il est 
libre;... il ne doit avoir absolument aucun rapport à rien;... (2). 
\bigskip
Pour disqualifier quelqu'un, un procédé efficace est de le confronter avec ce qu'il méprise, même si 
c'est pour concéder qu'il est supérieur. Il reste que les êtres comparés font désormais partie d'un 
même groupe. 
\bigskip
Les  interactions  entre  termes  d'une  comparaison  peuvent  être  dues  à  la  perception  de  liaisons 
réelles entre ce que l'on confronte. Mais peu importe leur origine. Elles ont pour conséquence que 
dans les comparaisons, quand c'est la mise en évidence d'une distance qui est visée, il faut un effort 
constant  pour  rétablir  celle-ci.  Seules  des  conventions  de  mesure  précises  peuvent  assurer  la 
persistance des rapports évoqués. » 
\bigskip
(2) Plotin, t. VI, 2, Ennéade VI 8, § 8, p. 143. 
\bigskip
P 329-330 : « Les arguments de comparaison ne laissent cependant pas de considérer ces rapports 
comme établis et comme transposables. En voici un exemple comique. Une jolie fille et une vieille 
personne revêche attendent le bus. La seconde refuse avec indignation une cigarette : 
\bigskip
Fumer en rue ? je préférerais encore embrasser le premier venu. 
\bigskip
\bigskip
\bigskip
168 
\bigskip
lui  accorde  une 
\bigskip
la  tautologie,  on 
\bigskip
- Moi aussi, mais en attendant en peut en griller une (1). » 
\bigskip
(1) D'après le journal Le Soir, du 20-6-1950. 
\bigskip
P 330 : « Le comique provient de ce que la même hiérarchie préférentielle est située dans une tout 
autre région de l'échelle des valeurs. 
\bigskip
Le  choix  de  termes  de  comparaison  adaptés  à  l'auditoire,  peut  être  un  élément  essentiel  de 
l'efficacité  d'un  argument,  même  lorsqu'il  s'agit  de  comparaison  numériquement  précisable  :  il  y 
aura  avantage,  dans  certains  cas,  à  décrire  un  pays  comme  neuf  fois  aussi  grand  que  la  France 
plutôt que de le décrire comme une demi-fois aussi grand que le Brésil. 
\bigskip
Ce  sont  les  caractères  du  terme  de  référence  qui  donnent  leur  aspect  particulier  à  une  série 
d'arguments. » 
\bigskip
P 330-331 : « Une forme typique de comparaison est celle qui fait état de la perte non subie, pour 
apprécier les avantages d'une solution adoptée. A ceux qui lui demandaient ce qu'il avait gagné par 
la  guerre, Pitt  répondait:  tout  ce  que nous  aurions  perdu  sans  elle  (2).  Le  terme  de  référence  est 
hypothétique  mais,  grâce  à 
importance  réelle  bien 
qu'indéterminable.  Souvent,  cependant,  il  est  nécessaire  d'évaluer  ce  terme  de  référence,  lequel 
sera présenté d'une façon favorable aux conclusions de l'argumentation. Le même Pitt critique ses 
adversaires qui, pour mesurer les inconvénients de la guerre, décrivent d'une façon enthousiaste la 
prospérité  qu'elle  a  fait  disparaître,  et  que,  en  son  temps,  ils  semblaient  peu  apprécier  (3).  En 
général, les tableaux de l'âge d'or, passé ou futur, du paradis perdu ou espéré, qu'il s'agisse du bon 
vieux  temps  ou  du  bonheur  que  l'on  trouve  ailleurs,  servent  à  disqualifier  l'âge  et  le  pays  dans 
lequel on vit. Par contre, la description enthousiaste de la situation présente servira à écarter tout 
effort pour l'améliorer et même pour la modifier; l'accroissement relatif de félicité serait minimum, 
la perte de félicité considérable. Toute persuasion par la menace sera donc d'autant plus efficace 
que l'état dont on jouit est valorisé. » 
\bigskip
 (2) W. Pitt, Orations on the French war, p. 123 (9 déc. 1795), 
(3) Ibid., p. 133 (10 mai 1796). 
\bigskip
P 331 : « L'argument de comparaison peut se manifester également par l'usage du superlatif. Celui-
ci s'exprime en considérant quelque objet, soit comme supérieur à tous les êtres d'une série, soit 
comme  incomparable  et,  par  là,  unique  en  son  genre.  N'oublions  pas  que  cette  dernière 
qualification  nécessite  une  tentative  préalable  de  comparaison  dont  on  reconnaît  l'impossibilité. 
C'est  ainsi  que  l'unicité  peut  résulter  elle-même  du  superlatif  comme,  chez  Leibniz,  l'unicité  des 
vérités contingentes est fondée sur le principe du meilleur. Cette technique permet d'individualiser 
les êtres en les qualifiant au moyen du superlatif, procédé que Giraudoux n'a pas hésité à utiliser 
fréquemment (1). » 
\bigskip
Par ailleurs les jugements faisant état du superlatif sont bien plus impressionnants, en partie par 
leur aspect quasi logique, que des jugements plus modérés. Ils dispensent souvent de montrer que 
la comparaison porte sur quelque chose qui a une valeur : témoin l'abondance de superlatifs dans 
les  écrits  publicitaires.  Leur  caractère  péremptoire  dispense  aussi  plus  aisément  de  preuve. 
L'accusation  d'avoir  commis  «  l'acte  le  plus  infâme  »  sera  généralement  moins  étayée  de  preuve 
que  celle  d'avoir  «mal  fait  ».  Cette  hiérarchie  peut  être  exprimée  en  dehors  de  la  forme 
comparative, par le simple usage de notions telles qu' «exécrable », « miraculeux ». La Bruyère a 
noté l'aspect péremptoire de ces termes (2). 
\bigskip
Le superlatif peut aussi être suggéré par certains procédés d'amplification, tel celui que Quintilien 
nous donne comme exemple : 
\bigskip
\bigskip
\bigskip
169 
\bigskip
 
C'est ta mère que tu as frappé. Que dire de plus ? C'est ta mère que tu as frappé (3). » 
\bigskip
(1) Cf. Y. Gandon, Le démon du style, P. 140. 
(2) LA Bruyère, Oeuvres, Bibl. de la Pléiade, Caractères, De la société et do la conversation, 19, p. 
176. 
(3) Quintilien, Vol. III, liv. VIII, chap. IV. § 7. 
\bigskip
P 332 : « Lorsqu'il n'y a plus rien qui, ajouté, puisse augmenter la gravité d'une infraction, puisse 
souligner  l'importance  d'un  acte,  il  n'est  plus  nécessaire  de  le  comparer  à  n'importe  quel  autre 
pour en reconnaître la prééminence. 
\bigskip
Parfois l'effet du superlatif sera renforcé par une restriction particulière : on semble avoir vraiment 
procédé  à  une  comparaison  effective,  le  superlatif  ne  doit  pas  être  considéré  comme  une  simple 
manière de parler, comme figure de style. Virgile à propos de Lausus dit : 
\bigskip
Le plus beau de tous, en exceptant le corps de Turnus le Laurentin (1). 
\bigskip
Ici le superlafif ne coïncide pas avec l'unique ; par contre l'impression d'unicité est produite quand 
on ne trouve, pour parler d'un être, que lui-même comme terme de comparaison; on forme, pour 
ainsi dire, une classe à part, incomparable. Comme le dit La Bruyère, 
\bigskip
V... est un peintre, C... un musicien, et l'auteur de Pyrame est un poète; mais Mignard est Mignard, 
Lulli est Lulli, et Corneille est Corneille (2). 
\bigskip
Toute  comparaison,  nous  le  voyons  par  cet  exemple,  est  par  quelque  biais,  disqualifiante,  parce 
qu'elle  fait  fi  de  cette  unicité  des  objets  incomparables.  Traiter  sa  patrie,  sa  famille,  comme  une 
patrie, une famille, c'est déjà la priver d'une partie de son prestige ; de là le caractère quelque peu 
blasphématoire du rationalisme, qui se refuse de considérer les valeurs concrètes dans leur unicité. 
C'est  la  raison  pour  laquelle  tout  amour,  dans  la  mesure  où  il  résulte  d'une  comparaison 
aboutissant au choix du meilleur objet sur lequel il puisse se porter, sera suspect et peu apprécié. Il 
y a des sentiments qui excluent tout choix, aussi flatteur qu'il puisse être. » 
\bigskip
(1) Quintilien, Vol. III, liv. VIII, chap. IV, § 6 (Enéide, chant VII, 649-650). 
(2) LA Bruyère, Oeuvres Bibl. de la Pléiade, Caractères, Du mérite personnel, 24, p. 118. 
\bigskip
P  333 :  « L'idée  même  de  choix,  de  bon  choix,  implique  toujours  comparaison.  Néanmoins  les 
expressions relatives au choix montrent bien le va et vient entre le domaine du comparatif et celui 
de l'absolu. « Nous avons fait un bon choix » signifie souvent le contentement, la volonté de ne pas 
comparer. L'idée que quelque chose est bon, surtout si ce quelque chose existe, et que joue l'inertie, 
s'exprime volontiers par l'idée que c'est le meilleur, que l'on ne pouvait trouver mieux, c'est-à-dire 
un  superlatif.  Une  justification  implicite  serait  que  l'objet  paraît  susceptible  de  soutenir  maintes 
comparaisons. On peut rapprocher de ces affirmations relatives au bon choix bien des affirmations 
portant sur une quantité (par exemple tel chiff re d'affaires obtenu par telle publicité). On suggère 
que  cette  quantité  est  supérieure  à  celles  auxquelles  on  pourrait  la  comparer.  Par  contre,  si  un 
événement  bénéficie  de  grandes  manchettes  dans  les  journaux,  on  pourra  minimiser  son 
importance en soulignant que, chaque jour, un événement est mis en vedette la valeur absolue se 
réduit à une valeur comparative. 
\bigskip
Ces interprétations semblent passer, en deux temps, de la valeur absolue à la valeur comparative, 
ou inversement. Toutefois c'est là le fruit d'une analyse. Par contre, il y a  des argumentations en 
deux temps qui opèrent ce passage de façon explicite. Blougram se base sur ce que, une fin étant 
choisie,  le  moyen  doit  être  le  meilleur  possible  (1).  Mais  cette  comparaison  faite  sur  le  plan  des 
\bigskip
\bigskip
\bigskip
170 
\bigskip
moyens  réagit sur  l'ensemble  de  la  situation  :  elle  valorise  en  l'occurrence,  à  la  fois  la  religion  la 
plus efficace pour agir sur le monde, et la fin choisie, à savoir agir sur le monde. Nous retrouverons 
ces interactions à l'occasion de l'étude des arguments basés sur les structures du réel. » 
\bigskip
(1) E. Browning, Poems, Bishop Blougram's Apology, p. 141. 
\bigskip
P  333-334 :  « Toutes  ces  analyses  tendent  à  montrer  combien  les  arguments  de  comparaison 
diffèrent de confrontations entre valeurs effectivement mesurables, dont la place dans une série ou 
dans un système aurait été fixée une fois pour toutes ; néanmoins c'est leur rapprochement avec 
des structures mathématiques qui fournit une grande partie de leur force persuasive. » 
\bigskip
§ 58. L'ARGUMENTATION PAR LE SACRIFICE 
\bigskip
P 334 : « L'un des arguments de comparaison le plus fréquemment utilisés est celui qui fait état du 
sacrifice que l'on est disposé à subir pour obtenir un certain résultat. 
\bigskip
Cette  argumentation  est  à  la  base  de  tout  système  d'échanges,  qu'il  s'agisse  de  troc,  de  vente,  de 
louage  de  services  -  bien  qu'elle  ne  soit  certainement  pas  seule  en  cause  dans  les  relations  de 
vendeur  à  acheteur.  Mais  elle  n'est  pas  réservée  au  domaine  économique.  L'alpiniste  qui  se 
demande s'il est prêt à faire l'effort nécessaire pour gravir une montagne recourt à la même forme 
d'évaluation. » 
\bigskip
P 334-335 : « Dans toute pesée, les deux termes se déterminent l'un par l'autre. Aussi Sartre a-t-il 
raison de dire que nous ne pouvons jamais savoir si le monde, par les obstacles qu'il nous présente, 
nous renseigne sur lui ou sur nous. C'est nous qui, librement, fixons les limites de nos efforts (1). 
Mais  pour  pouvoir  faire  état  de  cet  effort,  il  faut  que  celui-ci  puisse  être  décrit,  ou  connu  par 
ailleurs, d'une manière qui, tout au moins provisoirement, paraisse suffisante. Il faut à cet égard se 
garder  de  certaines  illusions.  Ainsi  la  distinction  de  Klages  entre  les  facultés  quantitatives,  qui 
seraient  mesurables  en  comparant  divers  individus,  et  les  mobiles  qui,  eux  se  mesureraient 
réciproquement chez un même individu (2), distinction féconde peut-être, ne doit pas faire oublier 
que  cette  évaluation  réciproque  n'est  utilisable  qu'à  condition  de  savoir  si,  chez  tel  individu,  un 
mobile donné est réputé d'intensité normale, ou d'en posséder une quelconque estimation. » 
\bigskip
(1) J.-P.Sartre, L'être ei le néant, p. 569. 
(2) L. Klages, Notions fondamentales de la caractérologie, dans Le diagnostic du caractère, p. 16. 
\bigskip
P  335 :  « Dans  l'argumentation  par  le  sacrifice,  celui-ci  doit  mesurer  la  valeur  attribuée  à  ce 
pourquoi le sacrifice est consenti. C'est l'argument dont se sert Calvin pour garantir l'importance 
que les protestants contrairement aux catholiques - attribuent à leur religion : 
\bigskip
Mais  comment  qu'ils  se  moquent  de  l'incertitude  d'icelle,  s'ils  avoyent  à  signer  la  leur  de  leur 
propre sang, et aux despens de leur vie, on pourroit voir combien ils la prisent. Nostre fiance est 
bien autre, laquelle ne craind ne les terreurs de la mort, ne le jugement de Dieu (1). 
\bigskip
C'est, dirigé contre les catholiques, l'argument bien connu fondé sur l'existence de confesseurs de 
la foi ; et l'absence de sacrifice sert à mesurer la faible importance accordée à une chose que l'on 
prétend, par ailleurs, révérer. 
\bigskip
Si,  dans  l'argument  du  sacrifice,  la  pesée  est  le  fait  de  l'individu  qui  consent  au  sacrifice,  la 
signification  de  ce  dernier  aux  yeux  d'autrui,  est  fonction  de  l'estime  pour  celui  qui  effectua  la 
pesée. Quand Pascal écrit : 
\bigskip
je ne crois que les histoires dont les témoins se feraient égorger (2) 
\bigskip
\bigskip
\bigskip
\bigskip
171 
\bigskip
il  faut  que  ces  témoins,  qui  servent  de  terme  de  référence,  jouissent  d'un  certain  prestige.  Plus 
celui-ci est grand, plus l'argument impressionne. Pauline le marque bien, disant: 
\bigskip
Mon époux en mourant m'a laissé ses lumières; 
Son sang, dont tes bourreaux viennent de me couvrir 
M'a dessillé les yeux, et me les vient d'ouvrir (3) » 
\bigskip
(1) Calvin, Institution de la religion chrétienne, AU ROY de France, P. S. 
(2) Pascal, Bibl. de la Pléiade, Pensées, 397 (159), p. 932 (593 éd. Brunschvicg) 
(3) Corneille, Polyeucte, acte V, se. V. 
\bigskip
P 336 : « A la limite, ce sera le sacrifice d'un être divin, celui qu'évoque Bossuet : 
\bigskip
Et  en  effet,  chrétiens,  Jésus-Christ,  qui  est  la  vérité  même,  n'aime  pas  moins  la  vérité  que  son 
propre corps; au contraire, c'est pour sceller de son sang la vérité de sa  parole qu'il a bien voulu 
sacrifier son propre corps (1). 
\bigskip
Les confesseurs de la foi peuvent être humbles, mais ils ne seront ni aliénés ni abjects ; leur grand 
nombre pourra suppléer au faible prestige individuel, comme dans la légende des 11 000 vierges 
qui  accompagnent  sainte  Ursule.  La  pesée  menant  au  sacrifice,  faite  en  toute  sincérité,  est 
d'ailleurs un élément susceptible d'accroître ce prestige. 
\bigskip
Toutefois, si l'objet du sacrifice est connu et que sa valeur est faible, le prestige de ceux qui se sont 
sacrifiés en sortira diminué, par une espèce de choc en retour. Isocrate, dans son éloge d'Hélène, la 
glorifie  par  les  sacrifices  que  les  Grecs  ont  acceptés  pour  la  reconquérir  (2).  Fénelon  critique  ce 
procédé : 
\bigskip
Rien n'y est prouvé sérieusement, il n'y a en tout cela aucune vérité de morale : il ne juge du prix 
des choses que par les passions des hommes (3). 
\bigskip
C'est  que  le  sacrifice  des  Grecs  lui  semble  futile,  à  cause  de  la  futilité  de  son  objet;  mais  la 
technique  de  la  preuve  ne  diffère  en  rien  de  celle  des  confesseurs  de  la  foi,  de  celle  utilisée  par 
Plotin pour valoriser l'état mystique : 
\bigskip
Elle  [l'âme]  n'échangerait  rien  contre  lui  [le  Premier]  lui  promît-on  le  ciel  tout  entier,  parce 
qu'elle sait bien qu'il n'y a rien de meilleur et de préférable à lui... Tout ce qui lui faisait plaisir 
auparavant,  dignités,  richesse,  beauté,  science,  tout  cela,  elle  le  méprise  et  elle  le  pouvoir, 
richesse, si elle n'avait rencontré des biens meilleurs ? (4). » 
\bigskip
(1) Bossuet, Sermons, vol. II : Sur la parole de Dieu, p. 157.  
(2) Isocrate, Discours, t. I : Eloge d'Hélène, §§' 48 et suiv. 
(3) Fénélon, Oeuvres, éd. Lebel, t. XXI : Dialogues sur l'éloquence, p. 75.,  
(4) Plotin, t. VI, 2, Ennéade VI, 7, § 34, p. 108. 
\bigskip
P 337 : “Mais notons-le, pour que la valeur de l'Un se prouve par la grandeur du sacrifice, il faut 
que  l'ascétisme  qui  en  résulte  repose  sur  une  appréciation  positive  préalable  des  biens  de  ce 
monde, sans quoi le renoncement ne serait guère probant. Une grave objection peut toujours être 
faite à l'argument par le sacrifice. L'accent mis par la psychologie contemporaine sur l'ambivalence 
des sentiments permet de la formuler en termes extrêmes : celui qui sacrifie son fils à l'honneur ne 
nourrissait-il pas envers lui une haine inconsciente ? La valeur de l'honneur ne se trouverait alors 
en  rien  rehaussée  par  cette  immolation.  La  mesure  par  le  sacrifice  suppose  constants,  et  insérés 
dans un cadre quasi formel, des éléments qui, en fait, sont sujets à variations. La preuve, c'est que 
la  conception  que  l'on  se fait  d'un  même  sacrifice  peut,  en  pratique,  être  fort  différente  selon  les 
\bigskip
\bigskip
\bigskip
172 
\bigskip
conclusions que l'on en veut tirer. S'il s'agit de confier ou non un poste à une personnalité que tous 
les  participants  au  débat  ont  en  estime,  ceux  qui  favorisent  ce  candidat  pourront  faire  état  de 
l'humiliation  que  celui-ci  éprouverait  en  cas  d'échec  ;  les  adversaires,  par  contre,  chercheront  à 
minimiser l'inconvénient qui en résulterait pour lui. Et le fait même d'y renoncer n'agit-il pas, par 
une espèce de choc en retour, pour modifier la valeur de ce à quoi l'on renonce ? Nous sommes en 
plein  dans  l'argumentation  quasi  logique  parce  que  le  terme  de  référence  ne  constitue  pas  une 
grandeur fixe, mais est en interaction constante avec d'autres éléments. 
\bigskip
La valeur de la fin que l'on poursuit par le sacrifice se transforme également, au cours de l'action, 
de par les sacrifices consentis eux-mêmes. Simone Weil écrit, fort justement, à ce propos : 
\bigskip
… des souffrances trop grandes par rapport aux impulsions du coeur peuvent pousser à l'une ou 
l'autre attitude; ou on repousse violemment ce à  quoi on  a trop donné, ou on s'y accroche avec 
une sorte de désespoir (1). 
\bigskip
(1) Simone Weil, L'enracinement, p. 114. 
\bigskip
P  338 :  « Dans  le  premier  cas,  on  ne veut  plus  être  dupe,  à  l'avenir,  et  on  détourne  les  autres  de 
cette valeur décevante ; dans le second, on grandit la fin, de manière que sa grandeur dépasse le 
sacrifice  :  on  se  trouve  en  présence  d'un  autre  argument  que  nous  analyserons  plus  loin  sous  le 
nom d'argument du gaspillage (1). 
\bigskip
L'argument du sacrifice, utilisé d'une façon hypothétique, peut servir à mettre en évidence le prix 
que  l'on  accorde  à  quelque  chose;  mais  il  s'accompagne  très  souvent  de  l'affirmation  que  pareil 
sacrifice, que l'on serait prêt à assumer, est, ou superflu, parce que la situation ne l'exige pas, ou 
inefficace, parce qu'il ne permettrait pas d'arriver au but escompté (2). 
\bigskip
Le  sacrifice  inutile,  qui  n'est  pas  pure  hypothèse,  mais  tragique  réalité,  peut  conduire  à  la 
déconsidération  de  ceux  qui  l'ont  accompli.  A  propos  des  morts  tombés  lors  d'une  offensive 
repoussée, voici la réflexion d'un camarade de combat : 
\bigskip
...  pour  tout  dire,  ils  nous  étaient  moins  sympathiques;  c'étaient  des  morts  ingrats  et  qui 
n'avaient pas réussi. Ferrer précisa cela, en disant: « ceux qu'il faut recommencer (3). » 
\bigskip
Le  pathétique  du  sacrifice  inutile  inspire  à  Bossuet  des  effets  poignants,  dans  son  sermon  sur  la 
compassion de la Vierge. La mère de Dieu se résignait au sacrifice de son fils, espérant sauver les 
hommes, mais ne peut supporter la douleur que lui cause l'impénitence des chrétiens : 
\bigskip
... quand je vous vois perdre le sang de mon Fils en rendant sa grâce inutile, ... (4). 
\bigskip
A  l'évaluation  par  le  sacrifice  consenti  se  rattachent  les  techniques  d'évaluation  par  le  sacrifice 
entraîné,  de  la  faute  par  la  sanction,  la  riposte  ou  le  remords,  du  mérite  par  la  gloire  ou  la 
récompense, de la perte par le regret. » 
\bigskip
(1) Cf. § 65 : L'argument du gaspillage. 
(2)  Cf.  Epictète,  Entretiens,  liv.  I,  4,  §  27,  p.  20,  B.  Crossman,  Palestine  .Mission,  with  Speech 
delivered in the House of Commons, 1er July 1946, p. 250. 
(3) J. Paulhan, Le guerrier appliqué, pp. 132-133. 
(4) Bossuet, Sermons, vol. II : Sur la compassion de la Sainte Vierge, p. 645. 
\bigskip
P  339 :  « En  raison  de  leur  aspect  séquentiel,  les  arguments  qui  en  font  état  se  rattachent  aux 
arguments  basés  sur  la  structure  du  réel.  Mais  ils  constituent  aussi  une  pesée  ;  et  l'on  s'efforce 
\bigskip
\bigskip
\bigskip
173 
\bigskip
souvent,  pour  la  rendre  plus  aisée,  de  donner  à  l'un  des  éléments  à  mettre  sur  le  plateau  de  la 
balance une structure homogène, afin de pouvoir en donner une description quantitative. 
\bigskip
La gravité de la sanction fait connaître celle de la faute : la damnation du genre humain est mieux 
connue que le péché originel dans la théologie chrétienne ; les malheurs de job font seuls mesurer 
sa culpabilité. 
\bigskip
La riposte fait connaître l'importance d'un acte 
\bigskip
C'est  quasi  le  propos  de  la  parolle  de  Dieu,  que  jamais  elle  ne  vient  en  avant,  que  Satan  ne 
s'esveille et escarmouche (1). 
\bigskip
L'intensité du regret mesure la valeur de la chose perdue. On trouverait une application curieuse 
de  cette  argumentation  dans  les  fantasmes  de  l'enterrement  qui,  selon  Odier,  seraient  un 
mécanisme puissant de sécurisation : l'abandonnien imagine son propre enterrement et mesure sa 
valeur à l'intensité des regrets que provoque sa mort (2). 
\bigskip
L'argument  quasi  logique  du  sacrifice  peut  s'appliquer  aussi  à  tout  le  domaine  des  rapports  de 
moyen à fin (3), le moyen étant un sacrifice, un effort, une dépense, une souffrance. L'aspect quasi 
logique est surtout marqué lorsque, pour valoriser telle chose on transforme autre chose en moyen 
apte à la produire et à la mesurer. Ainsi Isocrate, dans le Panégyrique d'Athènes : 
\bigskip
A mon avis, c'est quelque dieu qui a lait naÎtre cette guerre par admiration pour leur courage, pour 
empêcher  que  de  telles  natures  fussent  méconnues  et  qu'ils  ne  finissent  leur  vie  dans  l'obscurité 
(4). 
\bigskip
(1) Calvin, Instilution de la religion chrétienne, Au Roy de France, p. 13. 
(2) Ch Odier, L'angoisse et la pensée magique, p. 214. 
(3) Cf. § 64 : Les fins et les moyens. 
(4) Isocrate, Discours, t. II : Panégyrique d'Athènes, § 84. 
\bigskip
P 340 : « Il est très net aussi lorsque l'importance de l'enjeu est mesurée au déploiement de forces 
sollicitées. Paul Janson reproche à ses adversaires catholiques d'avoir utilisé cette technique pour 
convaincre  les  populations  de  ce  que  leur  foi  serait  mise  en  péril  par  le  vote  de  la  loi  scolaire  de 
1879. 
\bigskip
On est fatigué sans doute de prier Dieu ; on décide de s'adresser à ses saints et les voici tous mis en 
réquisition aux fins d'intervenir pour que la gauche ne vote pas cette loi de malheur (1). 
\bigskip
Le  très  vieil  argument,  éternellement  repris,  de  la  difficulté  d'expression,  est  également  mesure 
quasi logique : 
\bigskip
... il n'est pas moins difficile de louer les gens qui surpassent en vertus tous les autres que ceux qui 
n'ont rien fait de bon : d'un côté on ne dispose d'aucun exploit, de l'autre il n'y a pas de mot qui 
convienne (2). 
\bigskip
Tous ces arguments ne jouent que si la valeur que l'on mesure n'est pas sujette à une autre pesée, 
plus convaincante. Sinon, l'argument par le sacrifice peut devenir comique, comme dans l'anecdote 
de l'employeur qui, interrogeant un candidat pour un poste s'étonne : « Vous demandez un salaire 
très  élevé,  pour  un  homme  sans  expérience!  »  «  Le  travail  est  tellement  plus  difficile,  répond  le 
candidat, quand on ne sait pas comment s'y prendre » (3). 
\bigskip
\bigskip
\bigskip
\bigskip
174 
\bigskip
Puisque  l'argumentation  par  le  sacrifice  permet  d'évaluer,  comme  dans  tout  argument  de 
comparaison,  l'un  des  termes  par  l'autre,  la  manière  dont  la  confrontation  se  réalise  peut  elle-
même donner lieu à une argumentation intéressante. 
\bigskip
De Jankélévitch : 
\bigskip
Le diable n'était fort que de notre faiblesse, qu'il soit donc faible de notre force (4). » 
\bigskip
(1) P. Janson, Discours parlementa ires, vol. 1, p. 124, séance de la Chambre des Représentants, 26 
février 1880. 
(2) Isocrate, Discours, t. II : Panégyrique d'Athènes, § 82. 
(3) Fun Fare, Reader's Digest, 1949, p. 62. 
(4) V. Jankélévitch, Traité des vertus, p. 795. 
\bigskip
P 341 : « De Bossuet : 
\bigskip
Malheureux, si vos liens sont si forts que l'amour de Dieu ne les puisse rompre; malheureux, s'ils 
sont si faibles que vous ne vouliez pas les rompre pour l'amour de Dieu (1) ! 
\bigskip
Dans le premier exemple on se borne à mentionner un renversement possible : un des termes, le 
diable, est censé rester valeur constante. Mais chez Bossuet, aucun des deux termes n'est constant : 
la différence entre eux subsiste, de même sens, dans deux mesures différentes. L'emploi des verbes 
« pouvoir » et « vouloir » indique que, dans le premier cas on mesure la force des passions, dans le 
deuxième, la faiblesse de l'amour de Dieu, par le sacrifice que l'on refuse d'accomplir. 
\bigskip
Il semble bien que la mesure par le sacrifice soit liée souvent à l'idée d'une limite mobile entre deux 
éléments.  Lorsque  ceux-ci  forment  une  totalité  fixe,  l'argument  du  sacrifice  rejoint  parfois 
l'argument par division. C'est le cas lorsque deux caractères sont tels que, pour arriver à un résultat 
donné,  la  quantité  de  l'un  varie  en  sens  inverse  de  celle  de  l'autre.  Le  sacrifice  mesure  alors 
l'importance attribuée au complémentaire. 
\bigskip
Aristote  s'est  servi  de  cette  mesure  d'un  bien,  par  le  sacrifice  de  l'autre,  dans  ce  passage  des 
Topiques : 
\bigskip
Et si de deux choses nous répudions l'une afin de paraître avoir l'autre, celle qui est préférable est 
celle  que  nous  voulons  paraître  avoir  :  ainsi  nous  nions  que  nous  sommes  laborieux  pour  qu'on 
nous croie bien doués naturellement (2). » 
\bigskip
(1) Bossuet, Sermons, vol. II : Sur l'ardeur de la pénitence, p. 588. 
(2) Aristote, Topiques, liv. III, chap. 2,118 a 20. 
\bigskip
P  341-342 :  « La  complémentarité  se  présente  parfois  comme  compensation.  Cela  suppose  aussi 
une totalité constante à laquelle on se réfère. Mais l'idée de compensation est plus complexe que 
celle de complémentarité. Elle suppose avant tout une série d'évaluations réciproques. La faiblesse 
peut ainsi devenir mesure de l'élection : 
\bigskip
...  un  sens  exquis  de  sa  propre  faiblesse  l'avait  merveilleusement  réconfortée  et  consolée,  car  il 
semblait qu'il fût en elle comme le signe ineffable de la présence de Dieu... (1). » 
\bigskip
(1) G. Bernanos, La joie, p. 35. 
\bigskip
\bigskip
\bigskip
\bigskip
175 
\bigskip
P  342 :  « La  faiblesse  n'est  valeur  que  dans  une  éthique  compensatoire.  Mais  elle  devient  aussi, 
pour  le  lecteur,  argument  en faveur  de  cette  éthique  compensatoire.  Cela  peut  devenir  argument 
aux yeux de toute une civilisation. 
\bigskip
Ces  arguments  de  complémentarité,  de  compensation,  liés  à  une  idée  de  totalité,  sont 
généralement  utilisés  pour  promouvoir  une  certaine  stabilité.  Montesquieu  argumente  en  faveur 
du  système  bicaméral  en  montrant  qu'il  faut  compenser  la  faible  puissance  numérique  des  gens 
distingués  par  la  naissance,  les  richesses  ou  les  honneurs,  en  augmentant  la  puissance  de  leurs 
votes (2). Son raisonnement n'est fondé ni sur une hiérarchie de classes, ni sur l'expérience : il est 
fondé sur le maintien d'un équilibre. 
\bigskip
L'élément compensatoire peut devenir une mesure de l'imperfection de. celui qu'il doit compléter. 
Ainsi, pour saint Thomas, Dieu introduit sa ressemblance dans les choses. Mais 
\bigskip
il  est  évident  qu'une  seule  espèce  de  créatures  ne  réussirait  pas  à  exprimer  la  ressemblance  du 
créateur... s'il s'agit au contraire, d'êtres finis et créés, une multiplicité de tels êtres sera nécessaire 
pour  exprimer  dans  le  plus  grand  nombre  d'aspects  possibles  la  perfection  simple  dont  ils 
découlent (3). 
\bigskip
Ici  encore,  notons-le,  l'argument  est  basé  sur  une  totalité,  parfaite  cette  fois,  donc  invariable,  et 
que l'élément compensatoire doit, au mieux, tenter de reconstituer. » 
\bigskip
(2) Montesquieu, De l'esprit des lois, liv. XI, chap. VI, p. 267. 
(3) E. Gilson, Le thomisme, pp. 215-216. (cf. Cont. Gent., II, 45, ad Cum enim, Sum, theol., I, 47, 1, 
ad Resp.). 
\bigskip
P 342-343 : « Les éléments compensatoires peuvent parfois être tous deux de même nature. C'est 
par  un  jeu  de  compensation  que  Bertrand  Russell,  désirant  lutter  contre  toute  violence,  et 
reconnaissant toutefois la nécessité de certaines contraintes, tente de lever l'incompatibilité entre 
ces deux attitudes. » 
\bigskip
P 343 : « Il y a probablement une fin et seulement une, pour laquelle l'usage de la violence par un 
gouvernement est bienfaisante, et c'est de diminuer le montant total de violence dans  le monde 
(1). 
\bigskip
On raisonne comme si la violence, dans le monde, formait un ensemble auquel nulle addition n'est 
légitime, si ce n'est compensée par une diminution au moins égale. En réalité la force qui s'utilise 
entre en déduction de la violence future, non encore connue. 
\bigskip
Pour  terminer,  insistons  encore  sur  ce  que  l'argumentation  par  le  sacrifice,  et  celles  qui  lui  sont 
liées, rapproche les termes confrontés et établit une interaction entre eux. Dans une de ses lettres, 
saint Jérôme s'adresse à Pammachius qui, à la mort de sa femme, a distribué ses biens aux pauvres 
: 
\bigskip
Ceteri  mariti  super  tumulos  conjugum  spargunt  violas,  rosas,  lilia,  floresque  purpureos  :  et 
dolorem  pectoris  his  officiis  consolantur.  Panimachius  noster  sanctam  favillam  ossaque 
veneranda, eleemosynae balsamis rigat (2). 
\bigskip
Auerbach,  qui  cite  ce  passage,  note  très  justement  que  les  fleurs,  qui  ne  sont  pas  répandues, 
embaument  cependant.  Le  critique  veut  ainsi  attirer  notre  attention  sur  le  style  fleuri  de  saint 
Jérôme (3). Mais sa remarque a, selon nous, une portée beaucoup plus générale. Elle s'applique à 
la  plupart  des  sacrifices.  Même  si  elles  n'avaient  pas  été  énumérées  avec  cette  complaisance,  les 
fleurs auxquelles on renonce eussent déjà embaumé. l'expression d'Auerbach, « die Blumen duften 
\bigskip
\bigskip
\bigskip
176 
\bigskip
mit  »  convient  pour  nous  rappeler  que,  dans  l'argumentation  quasi  logique,  l'interaction  des 
termes est constante.é 
\bigskip
(1) Bertrand Russell, Political ideals, d'après S. I. Hayakawa, Language in Thought and Action, p. 
139. 
(2) Saint-Jérôme, Epistolae, liv. XVI, 5; Patrologie latine, I. XXII, col. 642. 
(3) E. Auerbach, Mimesis, p. 70. 
\bigskip
§ 59. PROBABILITES 
\bigskip
P  344 :  « L'utilisation  croissante  des  statistiques  et  du  calcul  des  probabilités,  dans  tous  les 
domaines de la recherche scientifique, ne doit pas faire oublier l'existence d'argumentations, non-
quantifiables,  basées  sur  la  réduction  du  réel  à  des  séries  ou  collections  d'êtres  ou  d'événements 
semblables par certains aspects et différenciés par d'autres. Ainsi Isocrate, dans le plaidoyer contre 
Euthynous : 
\bigskip
Même si Nikias... pût et voulût accuser faussement, on peut voir facilement qu'il ne se serait pas 
attaqué  à  Euthynous.  En  effet  ceux  qui  veulent  agir  de  la  sorte  ne  commencent  pas  par  leurs 
amis... s'agit-il de se plaindre ? On peut choisir parmi tous. S'agit-il de voler ? On peut frustrer 
seulement qui vous a confié (1). 
\bigskip
Puisque le hasard n'explique pas suffisamment l'action de Nikias, Isocrate suggère qu'il y faut une 
autre raison, à savoir le bien-fondé de l'accusation. 
\bigskip
La  technique  du  calcul  des  probabilités  permet  de  nos  jours  à  Lecomte  du  Noüy  de  montrer,  de 
façon  analogue,  que,  étant  donnée  la  très  faible  probabilité  pour  que  se  forment  sur  la  terre  des 
molécules  aussi  complexes  que  les  molécules  protéiniques  nécessaires  à  la  vie,  il  faut  une  autre 
hypothèse pour expliquer leur apparition (2). 
\bigskip
Tous ces raisonnements, qui semblent progresser du passé vers le présent, partent d'une situation, 
d'un fait actuels, dont ils soulignent le caractère remarquable et dont ils accroissent également la 
valeur et l'intérêt argumentatifs. » 
\bigskip
(1) Isocrate, Discours, t. 1: Contre Euthynous, §§ 8, 10. 
(2) Lecomte Du Noüy, L'homme et sa destinée, pp. 37 et suiv. 
\bigskip
P  344-345 :  « Un  autre  groupe  important  d'arguments  se  réfère  à  la  notion  de  variabilité  et  aux 
avantages  que  présente,  à  cet  égard,  un  ensemble  plus  étendu.  D'Isocrate  encore,  citons  cet 
argument en faveur de l'accès des jeunes aux délibérations : 
\bigskip
Puisque la qualité de nos jugements diffère non par le nombre de nos années, mais en raison  de 
nos tempéraments et de notre faculté d'application, pourquoi ne pas faire obligatoirement appel à 
l'expérience des deux générations afin que vous ayez la possibilité de choisir dans tous les discours 
tenus les plus utiles conseils (1) ? » 
\bigskip
(1) Isocrate, Discours, t. II : Archidamos, § 4. 
\bigskip
P 345 : « De même, dans le Phèdre, Lysias insère, entre autres, cet argument tendant à donner la 
préférence à celui qui n'aime pas sur celui qui aime : 
\bigskip
...  s'agit-il  pour  toi,  entre  ceux  qui  aiment  d'élire  celui  qui  aime  le  mieux  ?  c'est  sur  un  petit 
nombre que tu auras à faire ce choix; est-ce, parmi tout le reste, l'homme qui te sera le plus utile ? 
ton choix porte sur un grand nombre. J'en conclus que tu as beaucoup plus d'espoir, au milieu de 
cette multitude, de mettre la main sur l'homme qui mérite ta propre amitié (2). 
\bigskip
\bigskip
\bigskip
177 
\bigskip
 
Ce genre d'argumentation pourrait être rattaché aux rapports entre le tout et les parties. Mais les 
parties, ce sont bien ici les fréquences d'une variable, l'utile ; et l'argument vise l'accroissement de 
dispersion de cette variable. 
\bigskip
Basée sur la variabilité également, mais pour en tirer des conclusions quelque peu différentes, cette 
argumentation de Locke contre la tyrannie des princes dans le domaine religieux : 
\bigskip
S'il n'y a qu'une vérité, un chemin pour aller au ciel, quel espoir y-a-t-il que plus de gens y seront 
conduits  s'ils  n'ont  d'autre  règle  que  la  religion  du  prince,  et  sont  mis  dans  l'obligation 
d'abandonner la lumière de leur propre raison... l'étroit chemin serait fortement rétréci ; un seul 
pays serait dans le vrai... (3). » 
\bigskip
(2) Platon, Phèdre, 231 d. 
(3) Locke, The second treatise of civil government and A letter concerning toleration, p. 128. 
\bigskip
P 345-346 : « Il est à remarquer que, dans ce raisonnement, on suppose que, pour reconnaître le 
bon  chemin,  chaque  individu  a  égale  compétence.  On  préconise  donc  de  renoncer  à  un  système 
certainement mauvais, en faveur d'un système qui sera vraisemblablement plus avantageux, sans 
qu'il y ait confrontation explicite. » 
\bigskip
P  346 :  « L'argumentation  quasi  logique  par  le  probable  prend  tout  son  relief,  lorsqu'il  y  a  des 
évaluations  basées,  à  la  fois,  sur  l'importance  des  événements  et  sur  la  probabilité  de  leur 
apparition,  c'est-à-dire  sur  la  grandeur  des  variables  et  leur  fréquence,  sur  l'espérance 
mathématique.  Le  type  en  serait  le  pari  de  Pascal  (1).  Ce raisonnement  confronte  les  chances  de 
gain et de Perte combinées avec la grandeur de l'enjeu, en considérant comme quantifiables tous 
les  éléments  en  cause.  Notons  immédiatement,  à  cet  égard,  que  lorsqu'elles  font  intervenir  les 
probabilités,  les  comparaisons  sont  sujettes  à  toutes  les  interactions  signalées  aux  paragraphes 
précédents ; l'introduction des probabilités leur confère seulement une dimension supplémentaire; 
que le sacrifice concerne quelque chose que l'on n'a, en tout état de cause, qu'une chance sur deux 
de  conserver,  tout  ce  que  nous  avons  dit  de  l'argumentation  par  le  sacrifice  n'en  subsiste  pas 
moins. » 
\bigskip
(1) Pascal, Bibl. de la Pléiade, Pensées, 4.51 (7), p. 955 (233 éd. Brunschvieg). 
\bigskip
P  346-347 :  « L'application  du  calcul  des  probabilités  à  des  problèmes  de  conduite  est  le  plus 
souvent - il faut le dire - énoncée comme un souhait. Leibniz reprenant la classification de Locke 
relative  aux  degrés  de  l'assentiment,  aurait  voulu  refondre  l'art  de  conférer  et  de  disputer  en 
rendant  ces  degrés  proportionnels  aux  degrés  de  probabilité  de  la  proposition  envisagée.  La 
distinction établie par les juristes entre les différentes espèces de preuve  - preuve entière, preuve 
plus  que  à  demi-pleine,  à  demi-pleine  et  autres  constituait,  selon  lui,  un  effort  en  ce  sens,  qu'il 
suffirait  de  poursuivre  (2).  Bentham  énonce  des  ambitions  analogues,  notamment  en  ce  qui 
concerne la force probante des témoignages (3). Maints écrivains, à l'heure actuelle, spécialement 
ceux  qui  continuent  de  manière  plus  ou  moins  directe  la  tradition  utilitariste,  recourent  aux 
raisonnements de probabilité pour expliquer l'ensemble de notre conduite (1). Les théoriciens des 
fonctions  de  décision,  de  leur  côté,  tentent  de  formuler  les  problèmes  de  choix,  de  manière  à 
pouvoir  les  soumettre  à  ce  calcul.  Rien  ne  s'y  oppose,  malgré  les  difficultés  techniques, 
admirablement  surmontées  déjà  -  à  condition  que  sur  un  problème  précis  soient  donnés  des 
critères précis de choix, et notamment de ce que l'on considère comme un risque acceptable. D'où 
maints  exposés  de  ces  techniques  mathématiques s'accompagnent  d'un  renouveau  des  ambitions 
leibniziennes (2). » 
\bigskip
( 2) Leibniz, éd. Gerhardt, 5e vol. : Nouveaux essais sur l'entendement, pp. 1 L, et suiv. 
\bigskip
\bigskip
\bigskip
178 
\bigskip
(3) BENTHAM, Œuvres, t. II : Traité des preuves judiciaires, chap. XVII, p. 262. 
(1) Cf. 1. J. Good, Probability and the weighing of evidence. 
(2) Cf. notamment Irwin D. J. Bross, Design for decision. 
\bigskip
P 347 : « En fait, dans chaque discussion particulière oh l'on argumente par le probable, on pourra 
voir  surgir  -  à  moins  qu'il  ne  s'agisse  de  domaines  scientifiques  conventionnellement  délimités  - 
des objections tendant à dénoncer les réductions qui ont dû être opérées pour insérer la question 
dans le schème proposé. J. Stuart Mill a déjà souligné que ce n'est pas sur une grossière mesure de 
fréquence  que  l'on  fonde  sa  confiance  dans  la  crédibilité  d'un  témoin.  Dans  le  domaine  de  la 
conjecture, l'application de la règle de critique historique selon laquelle un texte a d'autant plus de 
chance de ne pas être altéré qu'il est séparé de l'original par un moins grand nombre de copies (3), 
sera  tempérée  par  tout  ce  que  l'on  croira  deviner  par  ailleurs  au  sujet  de  ces  copies. 
L'argumentation  donnera  plus  de  prise  encore  aux  objections  lorsqu'il  s'agit  de  problèmes  de 
conduite.  Ces  objections  ne  seront,  elles-mêmes,  bien  entendu,  jamais  contraignantes,  mais  elles 
pourront se développer sur des plans très divers. » 
\bigskip
(3) Cf. L.- E. Halkin, lnitiation à la critique historique, p. 22. 
\bigskip
P 347 : « On montrera notamment que le raisonnement par les probabilités n'est qu'un instrument 
qui  demande,  pour  être  appliqué,  une  série  d'accords  préalables.  Cela  semble  avoir  échappé  à 
Leibniz  lorsqu'il  a  proposé  -  le  premier  selon  Keynes  -  que  l'espérance  mathématique  soit 
appliquée dans les problèmes de jurisprudence: si deux personnes réclament une certaine somme, 
on répartira celle-ci suivant les probabilités de leurs droits (1). Le raisonnement est fondé sur une 
certaine  conception  de  ce  qui  est  équitable,  laquelle  est  loin  d'être  admise  nécessairement,  car 
habituellement  on  accordera  toute  la  somme  à  celui  dont  les  prétentions  paraîtront  le  mieux 
fondées.  C'est  sur  l'intervention  de  ce  facteur,  indépendant  du  calcul,  que  Van  Dantzig  attire 
l'attention, en analysant deux problèmes posés à Pascal par le chevalier de Méré (2). Alors que le 
premier pourrait être entièrement résolu grâce au calcul, le second (quel est le partage équitable de 
la mise entre deux joueurs qui n'achèvent pas une partie et dont on fournit la situation) suppose 
que  l'on  s'entende  sur  la  signification  des  mots  «  partage  équitable  »  ;  on  peut  notamment 
imaginer que celui-ci sera proportionnel aux chances des joueurs ou bien aussi que celui qui a les 
chances les plus grandes recevra toute la mise. » 
\bigskip
(1) Cf. Keynes, A treatise on probability, p. 311, note. 
(2) D. Van Dantzig, Blaise Pascal en de betekenis der wiskundige denkwijze voor de studie van de 
menselyke samenleving, p. 12. 
\bigskip
P 348-349 : « Par ailleurs, on montrera que l'argumentation par le probable entraîne la réduction 
des  données  -  même  quand  il  n'est  pas  question  de  quantifier  celles-ci  -  à  des  éléments  qui 
paraissent  plus  facilement  comparables  :  ce  n'est  qu'à  condition  de  substituer  aux  notions 
philosophiques  et  morales  de  bien  et  de  mal,  des  notions  qui  semblent  plus  précises  et  mieux 
déterminables, telles que le  Plaisir  et la douleur,  que les utilitaristes pouvaient espérer fonder la 
morale sur un calcul. D'autres espèces de réduction sont possibles, mais elles aboutissent toujours 
à  un  monisme  de  valeurs  permettant,  par  quelque  biais,  de  rendre  homogènes  les  éléments  que 
l'on  compare.  C'est  ainsi  que  les  moralistes  de  Port-Royal,  pour  lutter  contre  la  casuistique 
probabiliste  des  jésuites  -  qui  tendait  à  excuser  certains  actes  lorsque  quelque  conséquence 
favorable pouvait éventuellement en découler -introduisirent l'idée qu'il fallait envisager à la fois le 
bien  et  le  mal,  et  la  probabilité  que  l'un  et  l'autre  avait  de  se  produire  (1).  C'est  là  un  argument 
considérable  contre  le  probabilisme  des  jésuites.  Mais  pareille  confrontation  des  conséquences 
n'est  possible  que  si  elles  se  situent  dans  un  même  ordre,  sans  quoi  une  conséquence  favorable, 
même de probabilité infime, peut entraîner la décision. Or la distinction des ordres n'est pas une 
distinction  qui  va  de  soi  ;  elle  résulte  généralement  d'une  argumentation.  L'introduction  par 
Pascal,  dans  son  pari,  de  la  notion  d'infini,  peut  être  assimilée  à  l'introduction  d'une  notion 
\bigskip
\bigskip
\bigskip
179 
\bigskip
d'ordre.  Elle  rend  le  gain  possible  tellement  supérieur  à  la  mise  qu'aucune  hésitation  ne  peut 
subsister;  mais  elle  empêche  également  toute  confrontation  effective  et  reporte  tout  le  poids  de 
l'argument sur cette notion d'ordre. » 
\bigskip
(1) Cf. Keynes, A treatise on probability, p. 308. 
\bigskip
P  349 :  « Enfin,  sur un  plan  plus  technique,  on  montrera  que  la  complexité  des  éléments  dont  il 
faut  tenir  compte  peut  être  poussée  de  plus  en  plus  loin  :  grandeur  d'un  bien,  probabilité  de 
l'acquérir,  amplitude  de  l'information  sur  laquelle  est  basée  cette  probabilité,  degré  de  certitude 
avec  lequel  nous  savons  que  quelque  chose  est  un  bien.  Ces  éléments  résulteront  chacun  d'un 
ensemble  de  raisonnements  qui  sont  la  plupart  du  temps  de  type  quasi  logique.  Et,  par  le  fait 
même de la discussion, certains des éléments importants sur lesquels elle porte, tel celui de réalité, 
pourront faire l'objet de remaniements. » 
\bigskip
P  349-350 :  « Rappelons  que,  dans  une  argumentation  concrète,  les  énoncés  du  discours  eux-
mêmes  font  l'objet  de  raisonnements  spontanés  qui  interfèrent  avec  les  raisonnements  énoncés. 
En  l'occurrence,  des  raisonnements  par  le  probable,  portant  sur  la  véracité  de  l'orateur,  seront 
fréquents. Ils pourront, chez certains auditeurs, se compliquer de réflexions sur le fondement des 
probabilités qui, à leur tour, interféreront avec les arguments énoncés. » 
\bigskip
P 350 : « En général, l'application de raisonnements basés sur les probabilités aura pour effet, quel 
que  soit  le  fondement  théorique  que  l'on  attribue  aux  probabilités,  de  donner  aux  problèmes  un 
caractère empirique. Ces raisonnements quasi logiques pourront modifier l'idée que l'on se fait de 
certains domaines. Selon Cournot, la philosophie du probable aurait été retardée par la découverte 
même du calcul des probabilités, parce que celui-ci s'avérait inapplicable à la philosophie (1). En 
tout  cas,  l'usage  de  certaines  formes  de  raisonnement  ne  peut  manquer  d'exercer  une  action 
profonde sur la conception même des données qui en sont l'objet. » 
\bigskip
(1)  Cournot,  Essai  sur  les  fondements  de  nos  connaissances  et  sur  les  caractères  de  la  critique 
philosophique, vol. 1, pp. 171-172. 
\bigskip
CHAPITRE II LES ARGUMENTS BASÉS SUR LA STRUCTURE DU RÉEL 
\bigskip
§ 60. GENERALITES. 
\bigskip
P  352-353 :  « Alors  que  les  arguments  quasi  logiques  prétendent  à  une  certaine  validité  grâce  à 
leur  aspect  rationnel,  qui  dérive  de  leur  rapport  plus  ou  moins  étroit  avec  certaines  formules 
logiques  ou  mathématiques,  les  arguments  fondés  sur  la  structure  du  réel  se  servent  de  celle-ci 
pour établir une solidarité entre des jugements admis et d'autres que l'on cherche à promouvoir. 
Comment se présente cette structure ? Sur quoi est fondée la croyance en son existence ? Ce sont 
des questions qui ne sont pas censées se poser, aussi longtemps que les accords qui sous-tendent 
l'argumentation  ne  soulèvent  pas  de  discussion.  L'essentiel  est  qu'ils  paraissent  suffisamment 
assurés  pour  permettre  le  développement  de  l'argumentation.  Voici  un  passage  où  Bossuet 
s'efforce d'accroître le respect dû à la parole des prédicateurs : 
\bigskip
Le temple de Dieu, chrétiens, a deux places augustes et vénérables, je veux dire l'autel et la chaire. 
... Il y a une très-étroite alliance entre ces deux places sacrées, et les oeuvres qui s'y accomplissent 
ont un rapport admirable... . C'est à cause de ce rapport admirable entre l'autel et la chaire que 
quelques docteurs anciens n'ont pas craint de prêcher aux fidèles qu'ils doivent approcher de l'un 
et  de  l'autre  avec  une  vénération  semblable  ...  .  ...  Celui-là  n'est  pas  moins  coupable  qui  écoute 
négligemment la sainte parole que celui qui laisse tomber par sa faute le corps même du Fils de 
Dieu (1). » 
\bigskip
(1) Bossuet, Sermons, vol. II : Sur la parole de Dieu, pp. 143-45. 
\bigskip
\bigskip
\bigskip
180 
\bigskip
 
P 352 : « En établissant une solidarité entre la prédication et la communion, Bossuet ne croit pas 
un  instant  que  le  prestige  de  celle-ci  puisse  en  souffrir  ;  il  sait,  à  la  fois,  que  ses  auditeurs 
admettront la solidarité de fait qu'il établit entre l'autel et la chaire, et quelle est l'intensité de leur 
vénération pour le corps du Christ. 
\bigskip
Une  façon  de  mettre  en  évidence  la  solidarité  entre  des  éléments  divers  consiste  à  les  présenter 
comme des parties indissociables d'un même tout : 
\bigskip
Est-ce donc que l'Évangile de Jésus-Christ n'est qu'un assemblage monstrueux de vrai et de faux, 
et  qu'il  en  faut  prendre  une  partie  et  rejeter  l'autre  ?  Totus  veritas  :  Il  est  tout  sagesse,  tout 
lumière, et tout vérité (2). 
\bigskip
Parfois cette solidarité est le résultat d'une volonté humaine, mais qui paraît inébranlable ; c'est à 
prendre ou à laisser : 
\bigskip
« Si la fille te plaît » dit le vieux Charmidès, dans une comédie de Plaute, « la dot qu'elle t'apporte 
doit te plaire aussi. Au reste, tu n'auras pas ce que tu veux, si tu ne prends pas ce dont tu ne veux 
pas (3). » 
\bigskip
En l'occurrence, l'argument est comique, parce que la dot ne paraît guère moins désirable que la 
fille: c'est que, normalement, la solidarité sert à vaincre une résistance, à entraîner l'adhésion à ce 
qu'on ne vent pas, pour obtenir ce que l'on veut. » 
\bigskip
(2) Bossuet, Sermons, vol. II : Sur la soumission (lue à la parole de Jésus-Christ. p. 133. 
(3) Plaute, Comédies, vol. VII : Trinummus, aete V, scène II, vv. 1159-1160. 
\bigskip
P  352-353 :  « Dans  le  présent  chapitre  nous  analyserons  successivement  différents  ty  es 
d'arguments,  classés  selon  les  structures  du  réel  auxquelles  ils  s'appliquent,  et  que  l'on  peut 
retrouver  dans  l'usage  commun.  C'est  dire  que  nous  nous  défendons  de  toute  prise  de  position 
ontologique.  Ce  qui  nous  intéresse  ici,  ce  n'est  pas  une  description  objective  du  réel,  mais  la 
manière dont se présentent les opinions qui le concernent ; celles-ci pouvant d'ailleurs être traitées 
soit comme des faits, soit comme des vérités, soit comme des présomptions (1). » 
\bigskip
(1) Cf. § 16 : Les faits et les v érités ; § 17 : Les présomptions. 
\bigskip
P  353 :  « Nous  examinerons,  pour  commencer,  les  arguments  s'appliquant  à  des  liaisons  de 
succession,  qui  unissent  un  phénomène  à  ses  conséquences  ou  à  ses  causes,  ainsi  que  les 
arguments  s'appliquant  à  des  liaisons  de  coexistence,  qui  unissent  une  personne  à  ses  actes,  un 
groupe  aux  individus  qui  en  font  partie  et,  en  général,  une  essence  à  ses  manifestations.  Nous 
verrons,  ensuite,  dans  quelle  mesure  le  lien  symbolique,  qui  rattache  le  symbole  à  ce  qu'il 
symbolise,  constitue  une  liaison  de  coexistence.  Nous  terminerons  ce  chapitre  par  l'analyse 
d'arguments plus complexes, auxquels ces liaisons peuvent servir de base, à savoir, les arguments 
de double hiérarchie, ainsi que ceux relatifs aux différences de degré ou d'ordre. 
\bigskip
Nous  sommes,  soulignons-le,  convaincus  de  ce  que  les  différents  types  de  liaison  énumérés 
n'épuisent pas la richesse de la pensée vivante, et de ce que, d'un type de liaison à l'autre, il existe 
des nuances, des contaminations. L'orateur peut concevoir une certaine réalité selon divers types 
de  liaison.  Rien  ne  nous  garantit  d'ailleurs  que  ces  liaisons  soient  toujours  perçues  de  la  même 
manière par l'orateur et par son auditoire. 
\bigskip
Enfin,  dans 
liaison 
argumentative, à ce qui justifie le « donc », variera suivant ce qu'en dit l'orateur, et aussi suivant 
\bigskip
le  discours  envisage  comme  réalité, 
\bigskip
la  signification  attribuée  à 
\bigskip
la 
\bigskip
\bigskip
\bigskip
181 
\bigskip
les  opinions  de  l'auditeur  à  ce  sujet.  Si  l'orateur  prétend  que  pareille  liaison  est  contraignante, 
l'effet argumentatif pourra en être renforcé ; il pourra néanmoins être diminué par cette prétention 
même, à partir du moment où l'auditeur la trouve insuffisamment fondée et la rejette. » 
\bigskip
A) LES LIAISONS DE SUCCESSION 
\bigskip
§ 61. LE LIEN CAUSAL ET L'ARGUMENTATION 
\bigskip
P  354 :  « Parmi  les  liaisons  de  succession,  le  lien  causal  joue,  sans  conteste,  un  rôle  essentiel,  et 
dont  les  effets  argumentatifs  sont  aussi  nombreux  que  variés.  Dès  l'abord,  on  voit  qu'il  doit 
permettre des argumentations de trois types :   
\bigskip
a) Celles qui tendent à rattacher l'un à l'autre deux événements successifs donnés, au moyen d'un 
lien causal ; 
\bigskip
b)  Celles  qui,  un  événement  étant  donné,  tendent  à  déceler  l'existence  d'une  cause  qui  a  pu  le 
déterminer ; 
\bigskip
c) Celles qui, un événement étant donné, tendent à mettre en évidence l'effet qui doit en résulter. 
\bigskip
Si une armée, dotée d'un excellent service de renseignements, remporte des succès, on peut vouloir 
en déceler la cause dans l'efficacité du service en question ; on peut, de ses succès actuels, inférer 
qu'elle  possède  un  bon  service  de  renseignements  ;  on  peut  aussi,  sur  l'efficacité  de  ce  dernier, 
étayer sa confiance dans des succès futurs. 
\bigskip
Nous  réservons  l'examen  du  premier  de  ces  trois  types  d'arguments  aux  paragraphes  où  nous 
analyserons  l'argumentation  par  l'exemple  et  les  problèmes  que  pose  le  raisonnement  inductif  ; 
nous  nous  limiterons,  pour  le  moment,  aux  argumentations  qui,  grâce  à  l'intervention  du  lien 
causal,  visent,  à  partir  d'un  événement  donné,  à  augmenter  ou  à  diminuer  la  croyance  en 
l'existence d'une cause qui l'expliquerait ou d'un effet qui en résulterait. Le terme événement doit, 
d'ailleurs, être pris au sens le plus large. En effet le rapport entre un principe et ses conséquences 
est souvent traité comme une liaison de succession faisant partie de la structure du réel. » 
\bigskip
P 354-355 : « Le policier, qui cherche à identifier l'assassin, dans un meurtre commis en l'absence 
de témoins et de tout indice révélateur, orientera ses investigations vers ceux qui avaient quelque 
intérêt à la mort de la victime et qui, d'autre part, avaient pu matériellement commettre le crime. 
On  suppose  que  le  crime  a  eu  non  seulement  une  cause,  mais  aussi  un  motif  :  une  accusation, 
fondée  sur  des  présomptions,  aura  à  montrer  à  la  fois  le  comment  et  le  pourquoi  de  l'acte 
délictueux.  Comment  ou  pourquoi  domineront  l'argumentation  selon  l'interprétation  que  l'on 
donne  à  certains  événements  difficiles  à  expliquer  :  dans  The  Ring  and  the  Book,  une  moitié  de 
Rome  prétend  que  Guido  Franceschini  dormait  au  moment  du  départ  de  sa  femme,  parce  que 
celle-ci l'avait drogué (1); l'autre moitié de Rome suggère que Guido simulait le sommeil, pour ne 
pas devoir intervenir (2). » 
\bigskip
(1) R. Browning, The Ring and the Book, p. 56. 
(2) Ibid. p. 97 who knows? Sleeping perhaps, silent for certain,... 
\bigskip
P 355 : « L'argumentation par la cause suppose, quand il s'agit d'actes humains, que ceux-ci sont 
raisonnables. On admettra difficilement que quelqu'un ait agi d'une certaine façon, si l'accusateur 
n'explique pas les raisons du comportement allégué ; il faudrait même qu'il explique pourquoi l'on 
aurait commis tel acte, et pas tel autre qui semble préférable : 
\bigskip
Dans la Médée de Carcinos, nous dit Aristote, ses accusateurs prétendent qu'elle a tué ses enfants, 
qu'on  ne  voit  plus  nulle  part;  ...  elle  répond  pour  sa  défense  que  ce  n'est  pas  ses  enfants,  mais 
\bigskip
\bigskip
\bigskip
182 
\bigskip
Jason qu'elle aurait tué ; c'eût été une erreur de sa part de ne pas le faire, en admettant qu'elle 
eût  fait  l'autre  chose.  Ce  lieu  d'enthymème  et  ce  lieu  spécial  font  toute  la  matière  de  l'ancienne 
Techné de Théodore (3). 
\bigskip
Cette  argumentation,  pour  être  efficace,  exige  un  accord  entre  les  interlocuteurs  concernant  les 
motifs d'action et leur hiérarchisation. » 
\bigskip
(3) Aristote, Rhétorique, II, chap. 23, XXVII, 1400 b. 
\bigskip
P  355-356 :  « C'est  en  raison  de  pareils  accords  que  peuvent  se  dérouler  des  argumentations  qui 
visent  à  écarter  tout  ce  qui  paraît  trop  peu  probable  pour  s'être  produit.  Quand  un  événement 
s'impose  néanmoins  comme  incontestable,  il  s'agira  de  le  situer  dans  un  cadre  qui  explique  son 
apparition  :  celui  qui,  à  un  jeu  de  hasard,  gagne  un  nombre  de  fois  anormalement  élevé,  sera 
soupçonné  de  tricher,  ce  qui  rendrait  sa  réussite  moins  invraisemblable.  De  même,  des 
témoignages concordants devront trouver une autre explication que le simple hasard : si le risque 
de  collusion  a  été  écarté,  il  faudra  bien  reconnaître  qu'ils  renvoient  à  un  événement  réellement 
constaté. » 
\bigskip
P 356 : « Le lien causal joue un rôle important dans le raisonnement historique qui  fait appel à la 
probabilité rétrospective : 
\bigskip
Tout historien, écrit Aron après Weber, pour expliquer ce qui a été, se demande ce qui aurait pu 
être (1). 
\bigskip
Il  s'agit  d'éliminer,  dans  une  construction  purement  théorique,  la  cause,  considérée  comme 
condition  nécessaire  de  la  production  du  phénomène,  pour  envisager  les  modifications  qui 
résulteraient de cette élimination. Parfois l'accent est mis surtout sur cette modification de l'effet : 
le défenseur d'un savant convaincu d'espionnage dira que, sans la guerre, son client, au lieu d'être 
sur le banc des accusés, eût fait figure d'un candidat au prix Nobel (2). 
\bigskip
Une caricature de l'argumentation par le lien causal, de la preuve d'un événement par sa cause et 
réciproquement,  se  rencontre  dans  un  admirable  épisode  du  second  Don  Quichotte.  Parlant  des 
enchantements  auxquels  le  héros  prétend  avoir  assisté  dans  la  caverne  de  Montesinos,  Sancho 
Pança, incrédule, s'écrie : 
\bigskip
Oh!  Dieu  saint!...  Est-il  possible  que  pareilles  choses  se  passent  dans  le  monde,  et  que  les 
enchanteurs et les enchantements aient tant de force, qu'ils aient pu changer le bon sens de mon 
maître en une si extravagante folie (3). » 
\bigskip
(1) H. Aron, Introduction à la philosophie de l'histoire, p. 164. 
(2) Curtis-Bennett, Défenseur du D' Fuchs, ait procès d'avril 1950. 
(3) Cervantès, EI ingenioso hidalgo Don Quijote de la Mancha, vol. Vl, II, chap. XXIII, pp. 112-113. 
\bigskip
P  357 :  « Le  comique  naît  ici  de  l'antinomie  entre  les  réflexions  sur  la  cause  en  partant  d'une 
certaine  interprétation  de  l'événement  et  les  considérations  sur  l'événement  en  partant  d'une 
certaine interprétation de la cause. 
\bigskip
A 
l'eff  et. 
L'argumentation se développe, dans ce cas, d'une façon analogue : l'événement garantit certaines 
conséquences; des conséquences prévues, si elles se réalisent, contribuent à prouver l'existence de 
l'événement qui les conditionne. 
\bigskip
\bigskip
la  cause  correspond,  dans  d'autres  circonstances,  celle  de 
\bigskip
la  recherche  de 
\bigskip
\bigskip
\bigskip
183 
\bigskip
Attirons, pour finir, l'attention, sur les raisonnements tirés de la validité universelle du principe de 
causalité ou de celui, correspondant, de responsabilité. En partant du principe que tout événement 
a une cause, on argumente en faveur de l'éternité de l'univers, qui n'aurait jamais commencé. De 
même, du fait que tout acte est considéré comme la récompense ou la punition d'un acte antérieur, 
les Hindous concluent à l'éternité de l'âme, sans quoi elle serait « dotée d'un karman dont elle ne 
serait pas l'auteur responsable » (1). 
\bigskip
(1) Annambhatta, Le compendium des topiques, p. 46. 
\bigskip
§ 62. L'ARGUMENT PRAGMATIQUE 
\bigskip
Des transferts de valeur entre éléments de la chaîne causale s'effectuent en allant de la cause vers 
l'effet,  de  l'effet  vers  la  cause.  Dans  le  premier  cas  cependant,  celui  de  la  relation  que  nous 
appellerons  descendante,  le  lien  entre  termes  -  surtout  quand  il  s'agit  de  personnes  -  est  fourni 
normalement  non  par  la  relation  causale,  mais  par  une  relation  de  coexistence  (2).  Ainsi  la 
dévaluation d'une norme, en montrant qu'elle dérive d'une coutume primitive, de l'homme, parce 
qu'il  descend  des  animaux,  la  valorisation  de  l'enfant,  en  raison  de  la  noblesse  des  parents, 
s'opèrent  plus  par  une  relation  de  coexistence,  par  l'idée  d'essence,  que  par  une  relation  de 
succession. » 
\bigskip
(2) Cf. § 68 : La personne et ses actes. 
\bigskip
P  358 :  « Nous  appelons  argument  pragmatique  celui  qui  permet  d'apprécier  un  acte  ou  un 
événement en fonction de ses conséquences favorables ou défavorables. Cet argument joue un rôle 
à  tel  point  essentiel  dans  l'argumentation,  que  certains  ont  voulu  y  voir  le  schème  unique  de  la 
logique des jugements de valeur : pour apprécier un événement il faut se reporter à ses effets. C'est 
à ceux-ci que Locke, par exemple, se réfère pour critiquer le pouvoir spirituel des Princes: 
\bigskip
On ne pourra jamais établir ou sauvegarder ni la paix, ni la sécurité, ni  même la simple amitié 
entre homines, aussi longtemps que prévaudra l'opinion que le pouvoir est fondé sur la Grâce et 
que la religion doit être propagée par la force des armes (1), 
\bigskip
Pour les utilitaristes, tels que Bentham, il n'y a pas d'autre façon satisfaisante d'argumenter : 
\bigskip
Qu'est-ce que donner une bonne raison en fait de loi ? C'est alléguer des biens ou des maux que 
cette loi tend à produire... Qu'est-ce que donner une fausse raison ? C'est alléguer, pour ou contre 
une loi, toute autre chose que ses effets, soit en bien, soit en mal (2). 
\bigskip
L'argument pragmatique semble se développer sans grande difficulté, car le transfert sur la cause, 
de  la  valeur  des  conséquences,  se  produit,  même  sans  être  recherché.  Cependant,  celui  qui  est 
accusé d'avoir commis une mauvaise action, peut s'efforcer de rompre le lien causal et de rejeter la 
culpabilité sur quelqu'un d'autre ou sur les circonstances (3). S'il réussit à se disculper il aura, par 
le  fait  même,  reporté  le  jugement  défavorable  sur  ce  qui  paraîtra,  à  ce  moment,  la  cause  de 
l'action. » 
\bigskip
(1) Locke, The treatise of civil government and A letter concerning toleration, P. 135. 
(2) Bentham, Oeuvres, t. I : Principes de législation, chap. XIII, p. 39. 
(3) Cf. De Inventione. liv. II, §86 ; Rhétorique à Herennius, liv. II, § 26. 
\bigskip
P  358-359 :  « L'argument  pragmatique  qui  permet  d'apprécier  quelque  chose  en  fonction  de  ses 
conséquences,  présentes  ou  futures,  a  une  importance  directe  pour  l'action  (1).  Il  ne  demande, 
pour être admis par le sens commun, aucune justification. Le point de vue opposé, chaque fois qu'il 
est défendu , nécessite au contraire une argumentation ; telle l'affirmation que la vérité doit être 
\bigskip
\bigskip
\bigskip
184 
\bigskip
préconisée,  quelles  qu'en  soient  les  conséquences,  parce  qu'elle  possède  une  valeur  absolue, 
indépendante de celles-ci. » 
\bigskip
(1)  Cf.  H.  Feigl,  De  Principis  non  disputandum  ?....  dans  Philosophical  Analysis,  edited  by  Max 
Black,  P.  122,  Sur  l'opposition  entre  iustificatio  actionis,  qu'il  appelle  vindication  et  justificatio 
cognitionis ou validation. 
\bigskip
P  359 :  « Les  conséquences,  source  de  la  valeur  de  l'événement  qui  les  entraîne,  peuvent  être 
observées  ou  simplement  prévues,  elles  peuvent  être  assurées  ou  purement  hypothétiques  ;  leur 
influence , exercera sur la conduite, ou uniquement sur le jugement. La liaison  entre une cause et 
ses  conséquences  peut  être  perçue  avec  tant  d'acuité  qu'un  transfert  émotif  immédiat,  non 
explicité, s'opère de celles-ci sur celles-là, de telle sorte que l'on croie tenir à quelque chose pour sa 
valeur propre, alors que ce sont les conséquences qui, en réalité, importent (2). 
\bigskip
L'argumentation par les conséquences peut s'appliquer, soit à (les liaisons communément admises, 
vérifiables  ou  non,  soit  à  des  liaisons  qui  ne  sont  connues  que  d'une  seule  personne.  Dans  ce 
dernier  cas,  l'argument  pragmatique  pourra  être  utilisé  pour  justifier  le  comportement  de  cette 
personne.  C'est  ainsi  que,  dans  son  livre  sur  les  névroses  d'angoisse  et  d'abandon,  Odier  résume 
comme suit le raisonnement du superstitieux : 
\bigskip
Si nous sommes treize à table, si j'allume trois cigarettes avec une seule allumette, eh bien! je suis 
inquiète et ne vaux plus rien... Si j'exige au contraire que nous soyons douze, ou refuse d'allumer la 
troisième cigarette, alors je suis rassurée et recouvre toutes mes facultés. Donc cette exigence et ce 
refus  sont  légitimes  et  raisonnables.  En  un  mot  :  ils  sont  logiques,  et  je  suis  logique  avec  moi-
même (3). » 
\bigskip
(2) Cf. remarques de D. Van Dantzig dans Democray in a world of tensions, edited by H. Mc Keon, 
p. 55. 
(3) Ch. Odier, L'angoisse ei la pensée magique, p. 121. 
\bigskip
P  360 :  « A  partir  du  moment  où  une  liaison  fait-conséquence  est  constatée  l'argumentation  est 
valable,  quel  que  soit  le  bien-fondé  de  la  liaison  elle-même.  Remarquons  que  le  superstitieux 
rationalise  sa  conduite,  la  rationalisation  consistant  dans  l'invocation  d'arguinents  qui  puissent 
être  admis  par  l'interlocuteur.  Le  superstitieux  sera  justifié  si  l'interlocuteur  reconnaît  l'utilité 
d'une  conduite  qui  évite  à  son  auteur  un  malaise  on  une  déficience  psychique.  En  général, 
l'argument  pragmatique  ne  peut  se  développer  qu'à  partir  de  l'accord  sur  la  valeur  des 
conséquences. Une argumentation, basée le plus souvent sur d'autres techniques, sera appelée à la 
rescousse,  lorsqu'il  s'agira,  en  cas  de  contestation,  de  discuter  l'importance  des  conséquences 
alléguées. 
\bigskip
L'argument pragmatique ne se borne pas à transférer une qualité donnée de la conséquence sur la 
cause. Il permet de passer d'un ordre de valeurs à un autre, de passer d'une valeur inhérente aux 
fruits à une autre valeur inhérente à l'arbre, de conclure à la supériorité d'une conduite en partant 
de l'utilité de ses conséquences. Il peut aussi, et c'est alors qu'il paraît philosophiquement le plus 
intéressant,  considérer  les  bonnes  conséquences  d'une  thèse  comme  preuve  de  sa  vérité.  Voici, 
chez Calvin, un exemple de cette façon de raisonner, à propos des rapports du libre arbitre et de la 
Grâce : 
\bigskip
Mais  afin  que  la  verité  de  ceste  question  nous  soit  plus  facilement  eselaircie,  il  nous  faut 
preinierement mettre un but, auquel nous adressions toute nostre dispute. Or voicy le moyen qui 
nous gardera d'errer, c'est de considerer les dangers qui sont d'une part et d'autre (1). 
\bigskip
\bigskip
\bigskip
\bigskip
185 
\bigskip
Un usage caractéristique de l'argument pragmatique consiste à proposer le succès comme critère 
d'objectivité,  de  validité;  pour  beaucoup  de  philosophies  et  de  religions,  le  bonheur  se  présente 
comme l'ultime justification de leurs théories, comme l'indice d'une conformité avec le réel, d'un 
accord avec l'ordre universel. » 
\bigskip
(1) Calvin, Institution de la religion chrétienne, liv. 11, chap. 11, § 1. 
\bigskip
P  361 :  « Le  stoïcisme  n'hésite  pas  à  se  servir  de  pareil  argument.  Même  des  philosophies 
existentialistes, qui se prétendent antirationalistes, se résolvent à voir dans l'échec d'une existence 
l'indice évident de son caractère « non-authentique ». Le théâtre contemporain insiste volontiers 
sur  cette  idée  (1).  Le  même  argument  sert  dans  les  traditions  les  plus  variées,  depuis  celle  pour 
laquelle la meilleure cause se reconnaît au triomphe de son paladin, jusqu'au réalisme hégélien qui 
sanctifie l'histoire, en lui conférant le rôle de juge ultime. C'est par ce biais que la réalité est gage 
de la valeur, que ce qui a pu naître, se développer, survivre, se présente comme réussite, comme 
promesse de succès futur, comme preuve de rationalité et d'objectivité. » 
\bigskip
L'argument pragmatique est donné souvent comme une simple pesée de quelque chose au moyen 
de ses conséquences. Mais il est très difficile de réunir en un faisceau toutes les conséquences d'un 
événement  et,  d'autre  part,  de  déterminer  la  part  qui  revient  à  un  événement  unique  dans  la 
réalisation de l'effet. 
\bigskip
Pour que le transfert de valeur s'opère clairement, on tentera de montrer qu'un certain événement 
est condition nécessaire et suffisante d'un autre. Voici un exemple de pareille argumentation; elle 
vise à déprécier les biens terrestres, donc périssables : 
\bigskip
Il t'est dur d'avoir perdu ceci ou cela ? Ne cherche donc pas à perdre ; car c'est chercher à perdre 
que de vouloir acquérir ce qui ne se peut conserver (2). 
\bigskip
(1) cf. G. Marcel, Un homme de Dieu ; C.A. Puget, La peint- capitale. 
(2)  Guigues  Le  Chartreux.  Meditaciones,  chap.  II,  Patrol.  latine,  t.  CLIII,  col.  610  b,  cité  dans 
E.Gilson, L'esprit de la philosophie médiévale, p. 268. 
\bigskip
P 361-362 : « Cependant, hors les cas où cause et effet peuvent être considérés comme la définition 
l'un  de  l'autre  -  nous  avons  affaire  alors  à  une  argumentation  quasi  logique  -  l'événement  à 
apprécier ne sera qu'une cause partielle, ou une condition nécessaire. Pour pouvoir transposer sur 
lui  tout 
l'influence  des  causes 
complémentaires, en les considérant comme des occasions, des prétextes, des causes apparentes. » 
\bigskip
P  362 :  « Par  ailleurs,  quand  il  s'agit  de  transférer  la  valeur  d'un  effet  sur  la  cause,  jusqu'à  quel 
chaînon de la chaîne causale peut-on remonter ? Quintilien constate que : 
\bigskip
En remontant ainsi de cause en cause et en les choisissant, on peut arriver où l'on veut (1). 
\bigskip
Mais plus on remonte haut, plus le refus de l'adversaire sera aisé. En imputant les conséquences à 
une cause trop éloignée, on risque de détruire toute possibilité de transfert. 
\bigskip
Une  autre  complication  de  l'argument  pragmatique  résulte  de  l'obligation  où  l'on  se  trouve,  de 
tenir  compte  d'un  grand  nombre  de  conséquences,  bonnes  ou  mauvaises.  L'existence  de 
conséquences divergentes, faisait tout l'objet de la Techné de Callippe, nous dit Aristote, qui retient 
l'exemple suivant : 
\bigskip
L'éducation expose à l'envie, ce qui est un mal, et rend savant, ce qui est un bien (2). 
\bigskip
\bigskip
l'effet,  il  faudra  diminuer 
\bigskip
le  poids  de 
\bigskip
l'importance  et 
\bigskip
\bigskip
\bigskip
186 
\bigskip
Moyen  sûr  d'entretenir  la  controverse,  cette  considération  des  conséquences  favorables  et 
défavorables semble trouver une solution dans le calcul utilitariste. Mais à une pareille philosophie 
ont été opposées des objections de principe. » 
\bigskip
(1) Quintilien, Vol. II, liv. V, chap. X, § 84.  
(2) Aristote, Rhétorique. II, chap. 23, XIII, 1399 a. 
\bigskip
P 362-363 : « Les adversaires de l'argument pragmatique revendiqueront le droit de choisir, parmi 
les  conséquences,  celles  qu'ils  retiendront  comme  dignes  d'être  prises  en  considération,  étant 
donné  l'objet  du  débat.  Bien  plus,  l'argument  pragmatique  est  critiqué  par  les  tenants  d'une 
conception  absolutiste  on  formaliste  des  valeurs,  et  spécialement  de  la  morale.  Ceux-ci 
reprocheront à l'argument pragmatique de réduire la sphère de l'activité morale ou religieuse à un 
commun  dénominateur  utilitaire,  faisant  disparaître  ainsi  ce  qu'il  y  a  précisément  de  spécifique 
dans les notions de devoir, de faute, ou de péché. Montaigne note à ce propos : 
\bigskip
...  car  cette  sentence  est  justement  receüe,  qu'il  lie  faut  pas  juger  les  conseils  par  les  evenemens. 
Les Carthaginois punissoient les mauvais advis de  leurs capitaines, encore  qu'ils fussent corrigez 
par  une  heureuse  issue.  Et  le  peuple  Romain  a  souvent  refusé  le  triomphe  à  des  grandes  et  tres 
utiles victoires par ce que la conduitte du chef ne respondoit point à son bon heur (1). » 
\bigskip
(1) Montaigne, Bibl. de la Pléiade, Essais, liv. 111, chap. VlIL pp. 904-90.5.  
\bigskip
P 363 : « Ces réflexions, opposées à l'argument pragmatique, supposent que les valeurs morales ou 
religieuses ne sont pas discutées, que les règles du vrai et du faux, du bien et du mal, de l'opportun 
et  de  l'inopportun,  sont  reconnues  par  ailleurs,  indépendamment  de  leurs  conséquences,  ou  du 
moins de leurs conséquences actuelles et immédiates. 
\bigskip
S.  Weil  s'indigne  de  ce  que  plusieurs  arguments  en  faveur  du  christianisme  soient  de  l'espèce  « 
publicité pour pilules Pink », et du type « avant l'usage-après l'usage ». Ils consistent à dire : 
\bigskip
Voyez comme les hommes étaient médiocres avant le Christ... (2). » 
\bigskip
(2) S. WEIL, L'enracinement, p. 213. 
\bigskip
P  363-364 :  « Mais  l'argument  est-il  mauvais,  parce  qu'il  réussit  dans  un  domaine  commercial  ? 
Calvin  ni  Pascal  n'y  répugnaient  point.  Et  Leibniz,  en  précurseur  inattendu  du  pragmatisme, 
n'hésite pas à juger les procédés d'argumentation eux-mêmes en fonction de leurs conséquences : 
\bigskip
Or  cette  vérité  de  l'immatérialité  de  l'Anie  est  sans  doute  de  conséquence.  Car  il  est  infiniment 
plus avantageux à la religion et à la morale, sur tout dans le temps où nous sommes (où bien des 
gens ne respectent gueres la revelation toute seule et les miracles) de monstrer que les aines sont 
immortelles naturellement, et que ce  seroit un miracle si elles ne le fussent pas, que de soutenir 
que  nos  aines  doivent  mourir  naturellement,  mais  que  c'est  en  vertu  d'une  grace  miraculeuse 
fondée dans la seule promesse de Dieu qu'elles ne meurent point. Aussi saiton depuis longtemps 
que ceux qui ont voulu détruire la religion  naturelle  et reduire le tout à la revelée, comme si la 
raison ne  nous enseignoit rien  là dessus, ont passé pour suspects, et ce n'est pas tousjours sans 
raison (1). » 
\bigskip
(1) Leibniz, Oeuvres, éd. Gerhardt, 5e vol. : Nouveaux essais sur l'entendement, pp. 60-61. 
\bigskip
\bigskip
\bigskip
187 
\bigskip
§ 63. LE LIEN CAUSAL COMME RAPPORT D'UN FAIT A SA CONSEQUENCE OU D'UN 
\bigskip
MOYEN A UNE FIN 
P 364-365 : « Un même événement sera interprété, et valorisé différemment, selon l'idée que l'on 
se  forme  de  la  nature,  délibérée  on  involontaire,  de  ses  conséquences.  Les  cris  du  nourrisson 
attirent  l'attention  de  sa  mère,  mais à  un  moment  donné  ils  deviennent  un  moyen  en  vue  de  cet 
effet ; de la signification qu'elle leur attribuera dépendra très souvent la réaction de la mère. D'une 
façon générale, le fait de considérer ou non, une conduite comme un moyen en vue d'une fin, peut 
entraîner les conséquences les plus importantes et peut donc, pour cette raison, constituer l'objet 
essentiel  d'une  argumentation.  Selon  que  l'on  conçoit  la  succession  causale,  sous  l'aspect  de  la 
relation «fait-conséquence » ou «moyenfin », l'accent est mis tantôt sur le premier, tantôt sur le 
second  des  deux  termes  :  si  l'on  veut  minimiser  un  effet,  il  suffit  de  le  présenter  comme  une 
conséquence  ;  si  l'on  veut  en  grandir  l'importance,  il  faut  le  présenter  comme  une  fin.  La 
valorisation est-elle due à ce que, dans le premier cas, on oppose l'unicité du fait à la pluralité de 
ses conséquences, dans le deuxième, l'unicité de la fin à la multiplicité des moyens, peu importe. 
En  tout  état  de  cause,  cette  considération  autorise  une  double  critique  contre  l'argument 
pragmatique  :  elle  révèle  que  la  valeur  des  conséquences  n'est  pas  une  grandeur  fixe,  et  d'autre 
part, elle semble donner raison à ceux qui insistent sur la disqualification qu'entraîne l'usage de cet 
argument  pour  tout  ce  qui  apparaît  uniquement,  dès  lors,  comme  moyen  en  vue  de  résultats  à 
obtenir. » 
\bigskip
P 365 : « La distinction des fins et des conséquences permet de n'imputer à un auteur que certains 
des effets de ses actes. C'est ainsi que saint Thomas justifie l'existence du mal dans l'univers : 
\bigskip
La  forme  principale  que  Dieu  se  propose  manifestement  dans  les  choses  créées  est  le  bien  de 
l'ordre universel. Mais l'ordre de l'univers requiert, et nous le savons déjà, que certaines d'entre 
les choses soient déficientes. Dieu est donc cause des conceptions et des défauts de toutes choses, 
mais seulement en conséquence de ce qu'il veut causer le bien de l'ordre universel, et comme par 
accident.  En  résumé,  l'effet  de  la  cause  seconde  déficiente  peut  être  imputé  à  la  cause  première 
pure de tout défaut, quant à ce qu'un tel effet contient d'être et de perfection, mais non quant à ce 
qu'il contient de mauvais et de défectueux (1). 
\bigskip
L'ironie consiste parfois à renverser l'interprétation d'un même événement : 
\bigskip
Comme les habitants de Tarragone, raconte Quintilien, annonçaient à Auguste qu'un palmier avait 
poussé sur son autel: «On voit bien, répondit-il, que vous y allumez souvent du feu (2). » 
\bigskip
Auguste  interprète  les  faits  non  comme  un  signe  miraculeux,  mais  comme  la  conséquence  d'une 
négligence. 
\bigskip
Un  même  fait  ayant  plusieurs  conséquences,  il  arrive  que  celles-ci  se  combattent  et  que  les 
conséquences  non  souhaitées  en  viennent  à  prévaloir  sur  les  fins  désirables  d'une  conduite, 
laquelle peut ainsi paraître d'une maladresse comique. Voici une histoire qui fit beaucoup rire Kant 
: 
\bigskip
Un  riche  héritier  a  payé  largement  ses  gens  pour  faire  digne  figure  aux  obsèques  de  son  défunt 
parent. Mais voilà que ces marauds, plus on les paye pour être tristes, plus ils en deviennent joyeux 
(3) ! » 
\bigskip
(1) E. Gilson, Le thomisme, p. 223. (Cf. Sum. theol., 1, 49, 2, ad Resp.).  
(2) Quintilien, Vol. II, liv. VI, chap. III, § 77. 
(3) Cité d'après Ch. Lalo, Esthétique du rire, p. 159. 
\bigskip
\bigskip
\bigskip
\bigskip
188 
\bigskip
P 365-366 : « Paulhan analyse comme « prévision du passé » (4) des expressions telles « assassin 
pour cent francs », qui résultent de la transformation du schème « fait-conséquence » en schème « 
moyen-fin  ».  L'on  constate  le  caractère  dévaluant,  et  choquant,  de  cette  transformation.  Mais  la 
même  transformation  paraît  moins  choquante  lorsqu'il  s'agit  d'intégrer  dans  les  fins  d'une 
entreprise  beaucoup  de  ses  conséquences  importantes  quoique  involontaires.  C'est  le  cas, 
notamment, lorsqu'une guerre entraîne des conséquences qui dépassent les prévisions, et que l'on 
affirme, par après, que le pays avait pris les armes aux fins de défendre son existence. » 
\bigskip
(4) J. Paulhan, Entretien sur des laits divers, p. 54 ; cf. sur ce point Lefebve, Jean Paulhan, pp. 91 
et suiv. 
\bigskip
P  366 :  « Pour  soutenir  une  interprétation  fait-conséquence  à  l'encontre  d'une  interprétation 
moyen-fin,  diverses  techniques  seront  utilisées.  On  montrera,  par  exemple,  que  l'événement 
survenu ne pouvait être une fin, vu le peu d'intérêt qu'il présentait en l'occurrence, le peu de cas 
que  l'on  en  a  fait,  d'avantages  que  l'on  en  a  tirés,  ou  bien  on  montrera  que  le  fait  qui  devait 
l'entraîner  n'était  pas  un  moyen,  puisque  c'était  déjà  une  conséquence  d'un  fait  déterminé.  C'est 
ainsi  que  A.  Smith,  pour  prouver  que  la  division  du  travail  n'a  pas  été  voulue  par  les  hommes 
comme moyen en vue de certaines fins, la présente comme la conséquence du goût qu'ont ceux-ci 
pour l'échange des biens (1). » 
\bigskip
(1) A. Smith, The wealth of nations, p. 13. 
\bigskip
P 366-367 : « La transformation d'un fait en moyen détruit souvent les heureux effets qu'il pouvait 
entraîner  :  on  le  disqualifie  sous  le  nom  de  «procédé  ».  Nous  avons  cité  ailleurs  ce  passage  de 
Proust qui illustre notre propos : 
\bigskip
De  même,  si  un  homme  regrettait  de  ne  pas  être  assez  recherché  par  le  monde,  je  ne  lui 
conseillerais pas de faire plus de visites, et d'avoir encore un plus bel équipage; je lui dirais de ne 
se rendre à aucune invitation, de vivre enfermé dans sa chambre, de n'y laisser entrer personne, 
et qu'alors on ferait queue devant sa porte. Ou plutôt je ne le lui dirais pas. Car c'est une façon 
assurée  d'être  recherché  qui  ne  réussit  que  comme  celle  d'être  aimé,  c'est-à-dire  si  on  ne  l'a 
nullement  adoptée  pour  cela,  si,  par  exemple,  on  garde  toujours  la  chambre  parce  qu'on  est 
gravement malade, ou qu'on croit l'être, ou qu'on y tient une maîtresse enfermée et qu on préfère 
au monde... (1) » 
\bigskip
(1)  M.  Proust,  A  la  recherche  du  temps  perdu,  vol.  12  :  La  prisonnière,  II,  p.  210,  cité  dans 
Rhétorique et philosophie, p. 30 (Logique et rhétorique). 
\bigskip
P  367 :  « Pour  écarter  l'accusation  de  procédé,  il  faut  fournir  une  meilleure  explication  de  la 
conduite  :  on  dira  qu'elle  est  conséquence  d'un  fait  indépendant  de  lu  volonté,  on  moyen  en  vue 
d'une autre fin que celle qui est en cause. Ainsi le culte de la spontanéité en art, ou la présentation 
de l'art comme moyen à des fins sociales ou religieuses, sont des façons variées de montrer que les 
techniques de l'artiste ne constituent pas des procédés, accusation qui a discrédité la rhétorique au 
XIXe siècle (2). 
\bigskip
Quand  un  acte,  dont  la  fin  est  pourtant  avérée,  produit  des  conséquences  que  l'on  ne  peut  pas 
négliger,  et  qui  sont  ce  qui  importe  surtout  aux  tiers,  ceux-ci  peuvent  ne  voir  dans  l'acte  en 
question qu'un moyen en vue de ces conséquences. On se rappelle, dans César de Pagnol, comment 
le médecin écarte du lit de Panisse le prêtre qui apporte les Saintes Huiles : 
\bigskip
...  Et  l'employé  des  trams,  celui  qui  avait  eu  la  jambe  coupée  par  sa  remorque  ?  Après  la 
transfusion du sang, il avait une gueule possible. Mais tu es venu: ça n'a pas traîné ! Quand il t'a 
\bigskip
\bigskip
\bigskip
189 
\bigskip
vu, il s'est cru mort, et il est mort de se croire mort... Alors permets-moi de te dire que ton rôle 
n'est pas de tuer nies malades. J'en tue déjà assez tout seul, et salis le faire exprès (3). 
\bigskip
Ce décrochage entre un acte et sa fin normale, au profit de ses conséquences, peut devenir si 
habituel que le lien ancien passe à l'arrière-plan. La chasse, qui avait pour but de chercher de la 
nourriture, est devenue avant tout moyen en vue de maintenir certaines distinctions sociales (4). » 
\bigskip
(2) Cf. § 96 : La rhétorique comme procédé. 
(3) Pagnol, César, p. 60. 
(4) R. Amy, Homines et bêles, pi). 106 et suiv. (Rev. de l'Inst. de Sociol., 1954, n°1, pp. 166 et suiv.). 
\bigskip
P 367-368 : « Si une fin entraîne elle-même certaines conséquences ultérieures, celles-ci pourront 
être  prises  en  guise  de  fin  véritable.  Une  ridiculisation  de  l'agent  pourra  en  résulter,  lorsque  les 
deux phases des événements se détruisent l'une l'autre, comme dans ce passage de Cicéron : 
\bigskip
Ce  n'est  pas  un  exil  misérable  que  ton  iniquité  m'a  infligé,  mais  un  retour  glorieux  qu'elle  m'a 
préparé (1). 
\bigskip
P 368 : « Bien des antithèses seraient de ce type. 
\bigskip
§ 64. LES FINS ET LES MOYENS 
\bigskip
La logique des valeurs, dans ses premières élaborations, a supposé une nette distinction des fins et 
des moyens, les fins ultimes correspondant à des valeurs absolues. Mais, dans la pratique, il existe 
une interaction entre les buts que l'on poursuit et les moyens mis en oeuvre pour les réaliser. Les 
buts se constituent, se précisent et se transforment, au fur et à mesure de l'évolution de la situation 
dont font partie les moyens disponibles et acceptés ; certains moyens peuvent être identifiés à des 
fins,  et  peuvent  même  devenir  des  fins,  en  laissant  dans  l'ombre,  dans  l'indéterminé,  dans  le 
possible, ce à quoi ils pourraient servir. 
\bigskip
Les techniques modernes de la publicité et de la propagande ont exploité à fond la plasticité de la 
nature  humaine  qui  permet  de  développer  de  nouveaux  besoins,  de  faire  disparaître  ou  de 
transformer  des  besoins  anciens.  Ces  changements  confirment  que  seules  restent  invariables  et 
universelles  les  fins  énoncées  d'une  façon  générale  et  imprécise,  et  que  c'est  par  l'examen  des 
moyens que s'effectue souvent l'élucidation de la fin (2). » 
\bigskip
Des  fins  apparaissent  comme  désirables,  parce  que  des  moyens  de  les  réaliser  sont  créés  ou 
deviennent facilement accessibles. » 
\bigskip
(1) Cicéron, Paradoxa stoicorum, IV, § 29. 
(2) Cf. W. Barnes, Ethics without Propositions, Aristotelian Society, Supp. Vol. XXII, p. 16. 
\bigskip
P 369 : « Pour engager les pécheurs à la pénitence, Bossuet insiste sur ce que Dieu leur fournit le 
moyen de se sauver : 
\bigskip
... il [Dieu] ne refuse rien aux pêcheurs de ce qui leur est nécessaire. Ils ont besoin de trois choses : 
de la miséricorde divine, de la puissance divine, de la patience divine... (1). 
\bigskip
Des  fins  paraissent  d'autant  plus  souhaitables  que  leur  réalisation  est  facile.  Aussi  est-il  utile  de 
montrer  que si,  jusqu'à  présent,  on  n'a  pas  obtenu  de  succès,  c'est  que  l'on  avait  ignoré  les  bons 
moyens, ou que l'on avait négligé de s'en servir. Notons, à ce propos, que l'impossible et le difficile 
ou leurs opposés, le possible et le facile, ne concernent pas toujours l'impossibilité et la difficulté 
techniques, mais aussi morales, ce qui s'oppose à des exigences, ce qui entraînerait des sacrifices 
\bigskip
\bigskip
\bigskip
190 
\bigskip
que l'on  ne serait pas  disposé à  assumer. Ces deux points de vue, qu'il est utile de distinguer, ne 
sont pas, comme l'ont montré les analyses de Sartre (2), indépendants l'un de l'autre. 
\bigskip
Dans certains  cas,  le moyen peut devenir une fin que l'on poursuivra pour elle-même. Goblot en 
donne un bon exemple pris à la vie sentimentale : 
\bigskip
On  aime  déjà  quand  on  devine  dans  l'aimé  une  source  de  félicités  inépuisables,  indéterminées, 
inconnues... Alors l'aimé est encore un moyen, un moyen unique et impossible à remplacer de fins 
innombrables et indéterminées... On aime véritablement, on aime son ami pour lui-même, comme 
l'avare aime son or, quand, la fin ayant cessé d'être considérée, c'est le moyen qui est devenu la fin, 
quand la valeur de l'aimé, de relative, est devenue absolue (3). » 
\bigskip
(1) Bossuet, sermons, vol. il : Sur la pénitence, p. 71. 
(2) J.-P. Sartre, L'être et le néant, pp. 531 et suiv., 562 et suiv. 
(3) E. Goblot, La logique des jugements de valeur, pp. 55-56. 
\bigskip
P 369-370 : « Dans la vie sociale, le plus souvent, c'est l'accord sur un moyen, apte à réaliser des 
fins divergentes, et qui ne sont pas également appréciées par tous, qui conduit à détacher ce moyen 
des fins qui lui conféraient sa valeur, et à le constituer en une fin indépendante (1). C'est d'ailleurs 
la  meilleure technique  pour  magnifier  cet  accord  que  d'y  voir  un  accord  sur  des fins,  c'est-à-dire 
sur ce qui paraît l'essentiel. Insister sur ce que l'accord ne concerne qu'un moyen menant à des fins 
divergentes,  c'est  insister  sur  le  caractère  provisoire,  précaire,  somme  toute  secondaire,  de  cet 
accord. » 
\bigskip
(1) Cf. Ch. L. Stevenson, Ethics and Language, p. 193.  
\bigskip
P  370 :  « Dans  ce  même  esprit,  pour  montrer  qu'à  l'avenir  le  bien-être,  la  joie  au  travail,  du 
producteur, devraient être d'une importance primordiale, S. Weil voudrait qu'ils soient considérés 
comme une fin en soi, et non comme simple moyen d'accroître la production : 
\bigskip
jusqu'ici les techniciens n'ont jamais eu autre chose en vue que les besoins de la fabrication. S'ils se 
mettaient  à  avoir  toujours  présents  à  l'esprit  les  besoins  de  ceux  qui  fabriquent,  la  technique 
entière de la production devrait être peu à peu transformée (2). 
\bigskip
L'appel à un changement de fin a quelque chose de généreusement révolutionnaire. 
\bigskip
Le  processus  inverse,  qui  transformerait  une  fin  cri  moyen,  a  quelque  chose  de  dévaluant,  de 
dépréciatif.  C'est  contre  la  réduction  de  la  morale  à  une  simple  technique  en  vue  d'une  fin,  si 
importante soit-elle, que s'insurge Jankélévitch, car l'essentiel, ce n'est pas le but, mais la manière, 
« c'est l'intervalle qui est tout » : 
\bigskip
Vous dites : il n'est pas nécessaire de souffrir, mais de guérir... Dans cette identification de l'activité 
morale  aux  techniques,  qui  ne  reconnaît  la  philosophie  de  l'approximation  pharisienne,  c'est-à-
dire de la tricherie ? Certes si on peut uerir sans chirurgie ni cautères, il n'y a pas à se gêner. Mais 
en morale il est dit que nous travaillerons dans la douleur et que l'anesthésie sera la plus grave des 
tricheries puisqu'elle méconnaît ce moyen qui est la fin elle-même (3). » 
\bigskip
(2) S. Weil, L'enracinement, p. 57.  
(3) V. Jankélévitch, Traité des vertus, p. 38. 
\bigskip
P 370-371 : « Pour éviter de disqualifier les valeurs dont il traite, sans laisser pourtant échapper un 
argument  efficace,  à  savoir  leur  utilité  comme  moyen  pour  une  fin  par  ailleurs  reconnue  bonne, 
maint  orateur  mentionnera  cette  utilité,  tout  en  soulignant  le  caractère  superfétatoire  de 
\bigskip
\bigskip
\bigskip
191 
\bigskip
l'argument,  avouant  ne  s’en  servir  qu'en  raison  de  l'auditoire  auquel  il  s'adresse.  Relevons,  à  ce 
propos, que la mention,  devant certains auditoires, et en certaines circonstances,  de valeurs trop 
élevées risque de les ravaler au rang de moyen. » 
\bigskip
P 371 : « Notons aussi que le fait de choisir entre valeurs, de discriminer celles (lue l'on favorise, 
amène souvent à traiter les valeurs, ou àparaître les traiter, comme des moyens. Ainsi d'Ignace de 
Loyola suppliant le pape de ne pas donner à un Jésuite de charge épiscopale : 
\bigskip
je ne voudrais pas que la cupidité et l'ambition nous enlèvent tout ce qui a grandi jusqu'à présent 
par la charité et le mépris du monde (1). 
\bigskip
Quand  deux  activités  sont  confrontées  l'une  avec  l'autre,  sera  présentée  comme  moyen  celle  que 
l'on voudra subordonner à l'autre et, par là, dévaloriser, comme dans la maxime : il faut manger 
pour  vivre  et  non  pas  vivre  pour  manger.  Des  argumentations  piquantes  résultent  souvent  du 
renversement ainsi réalisé. Celui-ci est rendu possible chaque fois que la chaîne causale présente 
une succession continue de deux éléments alternés. D'où la recherche et la construction de pareils 
schèmes en vue même de l'argumentation. Souvent l'interaction entre éléments s'exprimera par de 
telles  alternances,  ce  qui  permettra  de  traiter  comme  fin  ce  qui  rencontre  le  plus  aisément 
l'adhésion. » 
\bigskip
(1) Rivadeneira, Vida del bienaventurado padre Ignacio de Loyola, p. 277. 
\bigskip
P  372 :  « Il  arrive  cependant  qu'une  activité  soit  valorisée  comme  moyen.  Cette  valorisation  ne 
résulte pas de la transformation d'une fin en moyen, mais de l'importance instrumentale que l'on 
reconnaît à quelque chose dont la valeur était complètement négligée ou même négative. Voici un 
texte  où  Démosthène  hésite  à  parler  de  lui-même  et  à  faire  son  propre  éloge,  mais  il  s'y  décide 
parce qu'il s'agit d'un moyen efficace : 
\bigskip
je  sais  fort  bien,  Athéniens,  que  rappeler  ce  qu'on  a  dit  et  parler  de  soi-même,  quand  on  ose  le 
faire,  est  un  moyen  de  succès  auprès  de  vous;  néanmoins,  ce  moyen  nie  paraît  à  moi  de  si 
mauvais  goût  et  si  indiscret  qu'en  nie  voyant  contraint  d'en  user,  j'hésite.  Mais  quoi  ?  Il  me 
semble que vous jugerez mieux de ce que je vais dire, si je vous rappelle brièvement quelques-unes 
des choses que j'ai dites antérieurement (1). 
\bigskip
On évitera avec soin de se louer soi-même, 
\bigskip
à  moins  qu'il  n'en  doive  résulter  un grand  avantage  pour  nous  on  pour ceux  qui  nous  écoutent 
(2). 
\bigskip
N'oublions pas que, s'il est vrai que la fin valorise les moyens, elle ne les justifie pas toujours, car 
l'usage  de  ceux-ci  peut  être  condamnable  en  soi,  on  avoir  des  conséquences  désastreuses,  dont 
l'importance  peut  dépasser  celle  de  la  fin  recherchée.  Néanmoins  une  fin  noble,  attribuée  à  un 
crime,  diminuera  le  dégoût  que  l'on  éprouve,  non  seulement  à  l'égard  du  criminel,  mais  aussi  à 
l'égard de son acte : le meurtre politique, le crime de l'idéaliste, même quand ils sont punis  plus 
sévèrement que le crime crapuleux, ne sont pas l'objet d'une condamnation morale sans réticence. 
\bigskip
Le  choix  de  certaine  fin  permet  de  valoriser  une  action  que,  par  ailleurs,  on  a  coutume  de 
condamner.  C'est  ainsi  que  Claudel,  loin  de  présenter  la  femme  comme  l'instrument  du  péché 
originel, voit en elle une condition de la Rédemption (3). » 
\bigskip
(1) Démosthène, Harangues, t. II : Sur la paix, § 4. 
(2)  Plutarque,  Oeuvres  morales,  t.  II  :  Comment  on  peut  se  louer  soi-même  sans  s'exposer  à 
l'envie, p. 600. 
\bigskip
\bigskip
\bigskip
192 
\bigskip
(3) Cité par S. de Beauvoir, Le deuxième sexe, 1, p. 343, 
\bigskip
P  372-373 :  « C'est  entre fins  diversement  situées  dans  le  temps  que  souvent  le  choix  s'effectue  ; 
mais  il  existe  bien  d'autres  manières  de  substituer  une  fin  à  une  autre,  de  les  subordonner.  La 
distinction stoïcienne entre le but de l'action et la fin de l'agent, situe ces deux tins dans le présent, 
mais fait de la première un moyen pour la seconde (1). Le remplacement d'une fin apparente par 
une  fin  réelle  (2)  aura  un  effet  argumentatif  d'autant  plus  assuré  que  la  substitution  surprendra 
plus  vivement  l'auditoire.  On  rapporte  que  Harry  Stack  Sullivan  détournait  certains  malades 
mentaux  du  suicide,  en  leur  montrant  que  le  désir  de  suicide  n'était  chez  eux  qu'un  effort  pour 
renaître autrement (3). » 
\bigskip
(1) Cf. V. Goldschmidt, Le système stoïcien et l'idée de temps, pp. 146-149. 
(2) Cf. § 92 ; Le rôle des couples philosophiques et leurs transformations. 
(3) Mary J. White, dans The Contributions of Harry Stock Sullivan, edited by Patrick Mullhaly, P. 
147. 
\bigskip
P 373 : « La substitution de fins, cri vue de valoriser le moyen, peut se ramener au choix de la lin la 
plus  favorable  à  l'argumentation,  sans  que  l'on  prétende  au  primat  de  l'une  d'elles.  On 
argumentera, comme le dit Quintilien : 
\bigskip
...  en  invoquant  quelque  avantage  pour  l'Etat,  pour  beaucoup  d'hommes,  même  pour  notre 
adversaire, quelquefois pour nous..... C'est encore une défense rentrant dans la question d'utilité, 
que de soutenir que l'acte en question en a évité un pire (4). » 
\bigskip
Tout ce que nous venons de dire de la valorisation du moyen, grâce à la fin, peut être répété, avec 
changement de signe, à propos de ce qui est considéré comme obstacle à la réalisation de cette fin. 
\bigskip
Pour qu'un moyen soit valorisé par la fin, il faut évidemment qu'il soit efficace ; mais ceci ne veut 
pas dire que ce sera le meilleur. La détermination du meilleur moyen est un problème technique, 
exigeant la mise en oeuvre de données diverses et le recours à des argumentations de tous genres. 
Le moyen qui l'emporte qui demande le moins de sacrifice pour la fin escomptée jouit d'une valeur 
inhérente, cette fois, à cette supériorité. » 
\bigskip
(4) Quintilien, Vol, III, Liv. VII, chap. IV, §§ 9, 12. 
\bigskip
P 374 : « Le danger qu'il peut y avoir à traiter quelque chose comme moyen, se trouve ainsi accru 
du fait que l'on peut toujours trouver un moyen plus efficace pour un but donné. 
\bigskip
La détermination du meilleur moyen dépend évidemment de la définition précise du but poursuivi. 
Par  ailleurs,  celui  qui  argumente  en  fonction  du  meilleur  moyen,  sera  tenté  de  diviser  les 
problèmes, de façon à éliminer toutes les considérations de valeurs autres que celles relatives à la 
fin  en  vue.  C'est  dans  cette  voie  que  s'orientent  certaines  disciplines  techniques.  Par  contre,  le 
raisonnement journalier peut rarement se prévaloir de pareil schématisme. 
\bigskip
Comme  la  discussion  technique  au  sujet  du  moyen  le  meilleur  dépend  d'un  accord  sur  le  but, 
tantôt  on  demandera  à  l'interlocuteur  un  accord  précis  relatif  à  celui-ci,  tantôt  on  attribuera  à 
l'interlocuteur 'Lin but qu'il n'oserait désavouer et en fonction duquel seront discutés les moyens. 
Par ailleurs, si un moyen est reconnu inefficace pour un but proclamé, celui qui tient à ce moyen, 
celui  qui  l'utilise,  pourra  toujours  être  soupçonné  et  accusé  de  rechercher  un  but  inavoué. 
L'affirmation de l'inefficacité d'un moyen intéresse donc souvent bien plus la discussion sur les fins 
que le problème technique du meilleur moyen. 
\bigskip
\bigskip
\bigskip
\bigskip
193 
\bigskip
Un cas éminent du problème technique du meilleur moyen est celui des arguments, considérés en 
tant  que  moyen  de  persuasion.  Rien  ne  permet  d'affirmer  qu'il  existe  un  argument  qui  soit  le 
meilleur pour tous. Comme le dit sainte Thérèse : 
\bigskip
Certaines personnes font des progrès par la considération de l'enfer, d'autres du ciel, qui s'affligent 
en pensant à l'enfer; d'autres par celle de la mort (1). 
\bigskip
D'où  le  rapport  étroit  entre  le  problème  technique  de  l'argumentation  efficace  et  celui  des 
auditoires. » 
\bigskip
(1) Santa Teresa de Jésus, Vida, p. 115. 
\bigskip
P 375 : « Le discours lui-même peut devenir, nous le savons, objet de réflexion. Il petit être traité 
comme  fait  engendrant  des  conséquences,  comme  conséquence,  comme  moyen,  comme  fin.  Les 
réflexions  de  l'auditeur  à  ce  propos  ne  seront  pas  sans  modifier  parfois  fortement  l'effet  que 
produit  le  discours.  Et,  d'une  manière  plus  précise,  l'hypothèse  que  n'importe  quel  acte 
intentionnel  doit  avoir  une  raison  d'être,  qu'il  constitue  un  moyen  en  vue  d'une  certaine  fin, 
justifiera le rejet de toute interprétation du discours qui rendrait celui-ci ridicule on inutile. C'est la 
conception  qui  sert  de  fondement  aux  arguments  ab  absurdo  et  ab  inutili  sensu,  utilisés  dans  la 
théorie de l'interprétation (1). » 
\bigskip
§ 65. L'ARGUMENT DU GASPILLAGE 
\bigskip
Les  arguments  qui  suivent  se  réfèrent  à  la  succession  des  événements,  des  situations,  d'une 
manière qui, sans exclure nécessairement l'idée de causalité, ne met pas celle-ci à l'avant-plan. 
\bigskip
L'argument  du  gaspillage  consiste  à  dire que,  puisque  l'on  a  déjà  commencé  une  oeuvre,  accepté 
des sacrifices qui seraient perdus en cas de renoncement à l'entreprise, il faut poursuivre dans la 
même direction. C'est la justification fournie par le banquier qui continue à prêter à son débiteur 
insolvable espérant, en fin de compte, le renflouer. C'est l'une des raisons qui, selon sainte Thérèse, 
incite à faire oraison, même en période de «sécheresse ». On abandonnerait tout, écrit-elle, si ce 
n'était : 
\bigskip
que l'on se souvient que cela donne agrément et plaisir au Seigneur du jardin, que l'on prend garde 
à  ne  pas  perdre  tout  le  service  accompli  et  aussi  au  bénéfice  que  l'on  espère  du  grand  effort  de 
lancer souvent le seau dans le puits et le retirer sans eau (2). » 
\bigskip
(1) Berriat, Saint-Prix, Manuel de logique juridique, p. 47-18. 
(2) Santa Teresa de Jésus, Vida, p. 96. 
\bigskip
P  375-376 :  « Par  un  certain  biais,  les  arguments  du  possible  et  du  facile  peuvent  se  rattacher  à 
l'argument du gaspillage ; ce n'est pas l'intéressé, mais la divinité, ou la nature, ou la fortune qui 
semble  s'être  donné  une  peine  qu'il  ne  faut  pas  mépriser.  D'oit  aussi  le  conseil  d'emboîter  le  pas 
pour  favoriser  une  évolution  déjà  commencée  :  on  invite  à  ne  pas  entraver  ces  forces  naturelles, 
sociales, qui se sont déjà manifestées et qui constituent une sorte de mise de fonds. » 
\bigskip
P 376 : « Bossuet se sert de l'argument pour reprocher aux pécheurs impénitents, de galvauder le 
sacrifice de Jésus en ne profitant pas des possibilités de salut qu'il a offertes (1).  
\bigskip
On pourrait rapprocher de cet argument tous ceux qui font état d'une occasion à ne pas rater, d'un 
moyen qui existe et dont il faut se servir. 
\bigskip
On emploiera le même argument pour inciter quelqu'un, doué d'un talent, d'une compétence, d'un 
don exceptionnel, à l'utiliser dans la plus large mesure possible. Pour une raison analogue, Volkelt 
\bigskip
\bigskip
\bigskip
194 
\bigskip
refuse  d'identifier  deux  mots  existants  de  la  langue  :  ce  serait  gaspiller  la  richesse  des  moyens 
d'expression (2). 
\bigskip
De  même  encore,  on  éprouve  du  regret  à  voir  un  effort  presque  réussi,  une  oeuvre  presque 
parfaite, ne point trouver leur couronnement. C'est ce qu'exprime Polyeucte, à propos de Pauline 
\bigskip
Elle a trop de vertus our n 1 etre pas chrétienne 
Avec trop de mérite il vous plut la former, 
\bigskip
Pour ne vous pas connaître et ne vous pas aimer, Pour vivre des enfers esclave infortunée... (3). 
\bigskip
Est particulièrement apprécié, ce dont la présence viendrait compléter heureusement un ensemble, 
que l'on peut alors envisager comme étant dans la nature même des choses. Dans une conception 
optimiste de l'univers, l'idée de gaspillage incite à compléter des structures, en y intégrant ce dont 
l'absence est ressentie comme un manque (4). » 
\bigskip
(1) Bossuet, sermons, vol. II : Sur la pénitence, p. 72. 
(2) J . Volkelt, Gewissheit and Wahrheit p. 169, n. 1. 
(3) Corneille, Polyeucte, acte IV, sc. 111. 
(4) CI. § 74 : L'acte et l'essence. 
\bigskip
P 377 : « Le sentiment du manque peut jouer un rôle, même quand on ne sait pas exactement en 
quoi consiste l'occasion perdue. Cet aspect prenant de l'argument est bien exprimé par le héros de 
Quand le navire... : 
\bigskip
«  Manquer»,  «Ce  que  vous  manquez  ».  je  réentendais  ces  mots-là.  Je  m'avouais  qu'ils  étaient 
poignants. Passer près de quelque chose. Etre à deux doigts de quelque chose. Le manquer. Même 
sans savoir ce qu'on manque, on arrive très bien à sentir le tragique essentiel de la situation où 
on est (1). 
\bigskip
Du moment que la conviction de manquer quelque chose est établie, elle vient renforcer la valeur 
propre de ce qui est ainsi galvaudé. 
\bigskip
Un cas important du manque est celui de l'ignorance. On considère que, par la faute de celle-ci, se 
perdent réalisations de la nature, efforts, souffrance. Dans le sonnet d'Arvers résonne le tragique 
lié au gaspillage : 
\bigskip
Et celle qui l'a fait n'en a jamais rien su. 
\bigskip
Aussi  trouvera-t-on  dans  l'argument  du  gaspillage  un  incitant  à  la  connaissance,  à  l'étude,  à  la 
curiosité, à la recherche. 
\bigskip
L'argument du gaspillage rappelle celui du sacrifice inutile. Le sacrifice est mesure de la valeur qui 
le détermine, mais si cette valeur est minime, le sacrifice est déprécié à son tour. Dans Le guerrier 
appliqué, 
\bigskip
Sièvre,  blessé,  dit  simplement,  stoïquement  :  «  Il  faut  ce  qu'il  faut.»  «  Que  l'on  eût  de  bonnes 
raisons  pour  se  battre  »,  commente  Jacques  Maast,  «  il  n'avait  fallu  rien  de  moins  (que  cette 
blessure) pour lui faire entrer la chose dans la tête (2). » 
\bigskip
Le sacrifice, réalisé et accepté, augmente et valorise les raisons du combat, incite à le continuer. » 
\bigskip
(1) J. Romains, Psyché, Ili : Quand le navire... pp. 194-195. 
\bigskip
\bigskip
\bigskip
195 
\bigskip
(2) M.-J. Lefebve, Jean Paulhan, p. 165 (Le guerrier appliqué, pp. 122 et 125). 
\bigskip
P  377-378 :  « C'est  par  un  processus  analogue  que  certains  tortionnaires  nazis  ont  tenté 
d'expliquer comment ils en sont arrivés à la bestialité dans le traitement de leurs prisonniers : les 
premières  douleurs  infligées  à  un  homme  font  de  l'agent  un  sadique,  si  on  ne  continue  pas  à 
torturer la victime jusqu'au moment où elle parle. » 
\bigskip
P 378 : « A l'argument du gaspillage peut être rattachée la préférence accordée à ce qui est décisif. 
On  sera  tenté  de  donner  sa  voix  à  un  candidat  si  l'on  croit  que  ce  vote  peut  emporter  le  succès. 
L'argument ne consiste pas à dire qu'il faut suivre le vainqueur, mais a conseiller d'agir de manière 
à  ce  qu'il  y  ait,  grâce  à  l'acte  posé,  un  vainqueur.  L'action  qui,  vu  les  circonstances,  pourra  avoir 
pleine portée, qui ne devra pas être considérée comme un gaspillage, sera  de ce fait valorisée, ce 
qui milite en faveur de son accomplissement. 
\bigskip
En sens inverse, on dévalue une action en insistant sur son caractère  superfétatoire ; tout  ce qui 
est  superfétatoire  est,  comme  tel,  déclassé.  Alors  que  l'argument  du  gaspillage  incite  à  continuer 
l'action commencée jusqu'à la réussite finale, celui du superfétatoire incite à s'abstenir, un surcroît 
d'action  étant  de  nul  eff  et. C'est  ainsi  que,  pour  Leibniz,  si  l'on  imagine un  auteur  intelligent  de 
l'univers, il faut que cette intelligence ne paraisse pas superfétatoire : 
\bigskip
Quand on est serieusement dans ces sentimens qui donnent tout à la necessité de la matiere on à 
un  certain  hazard....  il  est  difficile  qu'on  puisse  reconnoistre  un  auteur  intelligent  de  la  nature. 
Car  l'effect doit  répondre  à  sa  cause,  et  même  il  se  connoist  le  mieux  par la  connoissance  de  la 
cause, et il est déraisonnable d'introduire une intelligence souveraine ordonnatrice des choses, et 
puis au lieu d'employer sa sagesse, ne se servir que des proprietés de la niatiere pour expliquer 
les phenomenes (1). 
\bigskip
En  axiomatique,  la  recherche  de  l'indépendance  des  axiomes  se  justifie  par  la  même  raison  :  un 
système est moins élégant s'il contient un axiome superfétatoire. » 
\bigskip
(1) Leibniz, Oeuvres ed. Gerhardt, vol. 4 ; Discours de métaphysique, pp. 445-446. 
\bigskip
P 379 : « En économie politique, la dévalorisation des biens destinés en partie à des besoins quasi 
superflus, est dénoncée par la théorie de l'utilité marginale. Cette dévalorisation a parfois servi de 
fondement  à  une  argumentation  en  faveur  du  socialisme  :  il  s'agissait  de  promouvoir  un  régime 
qui,  par  leur  plus  égale  répartition,  valorise  les  richesses  -  et  qui  détourne  de  leur  inutile 
accumulation en certaines mains. » 
\bigskip
§ 66. L'ARGUMENT DE LA DIRECTION 
\bigskip
La  liaison  causale,  le  rapport  entre  la  fin  et  les  moyens,  ont  été  envisagés  antérieurement  d'une 
façon  globale  et  statique.  Mais  il  est  possible  de  décomposer  la  poursuite  d'une  fin  en  plusieurs 
étapes  et  d'envisager  la  manière  dont  la  situation  se  transforme  :  le  point  de  vue  sera  à  la  fois 
partiel  et  dynamique.  On  constate  que,  bien  souvent,  il  y  a  intérêt  à  ne  pas  confronter 
l'interlocuteur avec tout l'intervalle qui sépare la situation actuelle de la fin ultime, mais à diviser 
cet  intervalle  en  sections,  en  plaçant  des  jalons  intermédiaires,  en  indiquant  des  fins  partielles 
dont la réalisation ne provoque pas une aussi forte opposition. En effet, si le passage du point A en 
C soulève des difficultés, il se petit qu'on puisse ne pas voir d'inconvénient à passer du point A en 
B, d'où le point C apparaîtra dans une tout autre perspective : appelons cette technique le procédé 
des étapes. La structure du réel conditionne le choix de celles-ci, mais elle ne l'impose jamais. 
\bigskip
L'argument de direction consiste, essentiellement, dans la mise en garde contre l'usage du procédé 
des étapes : si vous cédez cette fois-ci, vous devrez céder un peu plus la prochaine fois, et Dieu sait 
où  vous  allez  vous  arrêter.  Cet  argument  intervient,  d'une  façon  régulière,  dans  les  négociations 
\bigskip
\bigskip
\bigskip
196 
\bigskip
entre  États,  entre  représentants  patronaux  et  ouvriers,  lorsque  l'on  ne  veut  pas  paraître  céder 
devant la force, la menace ou le chantage. » 
\bigskip
P  379-380 :  « Chaque  fois  qu'un  but  peut  être  présenté  comme  un  jalon,  une  étape  dans  une 
certaine direction, l'argument de la direction peut être utilisé. Cet argument répond à la question : 
où  veut-on  en  venir  ?  En  effet,  régulièrement,  pour  faire  admettre  une  certaine  solution,  qui 
semble au premier abord, désagréable, l'on divise le problème. Si l'on veut amener quelqu'un, qui y 
répugne, à prononcer un discours, à une certaine occasion, on lui montrera d'abord qu'un discours 
doit être prononcé, et puis on cherchera le meilleur orateur ou, inversement, on lui montrera que, 
si un discours doit être prononcé,  cela ne peut être que par lui, puis, qu'il est indispensable qu'il 
soit prononcé. » 
\bigskip
P 380 : « Il se peut cependant que la division soit non seulement inutile, mais même nuisible. C'est 
le cas si M. X. aime beaucoup à prendre la parole en publie. Il v aura intérêt alors à lui proposer, en 
une fois, de prononcer un discours en une certaine circonstance. 
\bigskip
La  manière  dont  s'opérera  la  division  dépend  de  l'opinion  que  l'on  se  forme  de  la  plus  ou  moins 
grande  facilité  de  franchir  telles  étapes  déterminées  :  il  est  rare  que  l'ordre  dans  lequel  on  les 
envisage soit tout à fait indifférent. En effet, une première étape étant franchie, les interlocuteurs 
se  trouvent  devant  une  nouvelle  configuration  de  la  situation,  qui  modifie  leur  attitude  devant 
l'issue  finale.  Dans  certains  cas,  l'un  des  caractères  de  cette  nouvelle  situation  sera  de  permettre 
l'emploi de l'argument du gaspillage, la première étape étant considérée comme une mise de fonds. 
\bigskip
Pourrait être assimilée à un procédé par étapes toute argumentation en plusieurs temps. Toutefois, 
celle-ci  ne sera  dénoncée  comme procédé, et  ne sera combattue par l'argument de  direction, que 
lorsque,  à  chaque  phase  de  l'argumentation,  est  sollicitée  une  décision  et  que  celle-ci  est 
susceptible de modifier la manière d'envisager une décision ultérieure. » 
\bigskip
P 380-381 : « Il convient, par ailleurs, de distinguer l'argument de la direction, de l'appréhension 
du  précédent,  qui  lui  ressemble  sur  ce  point  qu'elle  s'oppose  à  une  action  par  crainte  de  sa 
répercussion  sur  d'autres  actions  dans  l'avenir.  Mais  alors  que  l'appréhension  du  précédent 
concerne  d'autres  actions  de  même  espèce,  l'argument  de  la  direction,  évoque  des  actions  qui, 
quoique  différentes  de  l'action  en  cause,  entraîneront  un  changement  dans  le  même  sens.  il  v  a, 
néanmoins, des cas qui se situent entre l'appréhension du précédent et l'argument de direction : 
ceux  où  l'on  fait  état  d'une  récursivité,  d'une  même  opération  qui  se  répète,  mais  qui  s'effectue 
dans une situation modifiée. Pareille récursivité est souvent invoquée pour mettre en garde contre 
certaines constructions. Ainsi G. Ryle, pour critiquer la doctrine intellectualiste selon laquelle un 
acte intelligent serait celui qui est précédé d'une activité théorique intelligente, nous dit que cette 
exigence sera suivie d'une série d'autres : 
\bigskip
Devons-nous dire alors que les réflexions de l'agent sur la manière intelligente de se comporter, 
exigent qu'il réfléchisse d'abord à la meilleure manière de réfléchir à la façon d'agir ? (1). » 
\bigskip
P  381 :  « Le  procédé  des  étapes  peut  devenir  un  argument  positif  en  faveur  d'une  mesure 
considérée  comme  première  dans  une  direction  que  l'on  souhaite.  Il  arrive  cependant  que  cette 
argumentation  ne  soit  qu'une  feinte,  une  manœuvre  dilatoire,  que  l'on  fasse  semblant  de 
considérer  une  réforme,  une  mesure,  comme  un  jalon  dans  une  direction,  alors  que  l'on  est 
secrètement  décidé  à  ne  pas  continuer,  ou  du  moins  à  ne  le  faire  qu'avec  une  «  sage  »  lenteur. 
Bentham examine, parmi les sophismes dilatoires, celui de la marche graduelle. Il consiste, écrit-il 
à  vouloir  séparer  ce  qui  devrait  faire  un  tout,  à  rendre  la  mesure  nulle  ou  inefficace  en  la 
morcelant... La marche graduelle est escortée de toutes les épithètes flatteuses, elle est tempérée, 
elle est paisible, elle est conciliante (2). 
\bigskip
\bigskip
\bigskip
\bigskip
197 
\bigskip
En pareil cas, le fait de présenter comme un jalon ce qui, dans l'esprit de ses promoteurs, était, si 
possible,  une  mesure  finale,  n'avait  d'autre  but  que  de  la  valoriser  aux  yeux  des  partisans  de 
réformes plus drastiques. » 
\bigskip
(1) G. Ryle, The concept of mind, p. 31.  
(2) Bentham, Oeuvres t.I : Traité des sophismes politiques, p. 463. 
\bigskip
P 382 : « L'argument de la direction vise toujours à rendre une étape solidaire de développements 
ultérieurs. Celui qui se défend contre cet argument prétend isoler la mesure envisagée, veut qu'on 
l'examine  en  elle-même,  suppose  qu'elle  n'entraînera  pas  de  changement  dans  la  situation 
d'ensemble, et affirme que celle-ci pourra être considérée, une fois la mesure prise, dans le même 
état  d'esprit  que  précédemment.  Il  faut  donc,  pour  que  l'argument  de  la  direction  puisse  être 
combattu,  que  l'action  envisagée  présente  un  intérêt  en  elle-même,  puisse  être  appréciée 
indépendamment de la direction dans laquelle elle engage. On peut se demander si le grand art en 
éducation  intellectuelle  ou  morale,  ne  réside  pas  dans  le  choix  d'étapes  présentant  chacune  un 
intérêt propre, indépendant du fait qu'elles facilitent le passage à une étape ultérieure. L'ordre des 
arguments dans un discours aura à tenir compte de cette même considération. 
\bigskip
L'argument de direction, celui de la pente savonneuse, ou du doigt dans l'engrenage, insinue qu'il 
n'y aura pas moyen de s'arrêter en chemin. Le plus souvent l'expérience seule du passé permet de 
départager, à ce point de vue, les antagonistes. 
\bigskip
Voici un bon exemple de son utilisation, à propos de l'expérimentation sur les animaux : 
\bigskip
La médecine expérimentale chez les animaux admettait que pour l'utilité de la médecine humaine 
on pouvait sacrifier l'animal. Bientôt l'idée se fit jour que pour l'utilité de l'ensemble de l'humanité, 
on pouvait sacrifier quelques êtres humains. Bien sûr, au début, cette  idée soulevait des défenses 
internes  fortes,  mais  l'habitude  vient  à  bout  de  tout.  On  commence  par  admettre  l'idée  de 
l'expérimentation  sur  des  condamnés  à  mort,  puis  l'idée  fut  émise  de  l'expérimentation  sur  les 
prisonniers de droit commun, et enfin l'idée fut conçue de l'expérimentation sur ses ennemis ! La 
marche des idées est comme on e voit extrêmement redoutable et en même temps très insidieuse 
(1). 
\bigskip
(1) H. Baruk, Le psychiatre dans la société, La Semaine des hôpitaux de Paris, 23e année, n°74, pp. 
3046-47. 
\bigskip
P 383 : « En invoquant l'accoutumance, le Dr. Baruk fournit une raison en faveur de la thèse qui 
forme  l'essentiel  de  l'argument  de  direction,  à  savoir  que  l'on  n'est  pas  maître  de  son 
comportement ultérieur, que l'on ne saura pas s'arrêter à une étape donnée de l'évolution dans une 
certaine direction. 
\bigskip
L'argument  de  la  direction  implique  donc,  d'une  part,  l'existence  d'une  série  d'étapes  vers  un 
certain  but,  le  plus  souvent redouté  et,  d'autre  part,  la  difficulté,  si  pas  l'impossibilité  de  arrêter, 
une fois que l'on est engagé dans la voie qui y mène. Les ripostes à cet argument porteront donc 
sur l'un ou l'autre de ces points. 
\bigskip
Une  première  riposte  à  l'argument  de  direction  consiste  dans  l'indication  de  développements,  à 
partir de la première étape, différents de ceux qui semblaient à craindre. On fait état de l'ambiguïté 
de développement et, par suite, de l'arbitraire qu'il y a à ne voir qu'une seule direction possible. 
\bigskip
Cette riposte peut d'ailleurs soulever d'autres objections et spécialement la crainte de ne pas savoir 
oh l'on va ; on redoute les conséquences imprévisibles d'un premier ébranlement : 
\bigskip
\bigskip
\bigskip
\bigskip
198 
\bigskip
Mais  la  nouveauté,  Philonous,  la  nouveauté!  C'est  là  qu'est  le  danger.  Les  nouvelles  opinions 
doivent toujours être désavouées; elles ébranlent les intelligences humaines et personne ne sait où 
elles aboutiront (1). » 
\bigskip
(1) Berkeley, Oeuvres choisies, t. II: Les trois dialogues entre Hylas et Philonous, 3e dial., p. 171. 
\bigskip
P 383-384 : « On peut aussi montrer qu'entre l'étape en discussion et les suivantes qui seraient à 
craindre,  il  y  a une  différence  qualitative.  C'est  ain  si  que  B.  S. Chlepner  insiste  sur  la  différence 
qu'il y a entre la nationalisation de certaines entreprises et l'économie socialiste, vers laquelle elle 
semble mener : 
\bigskip
On  peut  dès  lors  soutenir  que  la  nationalisation  d'une  entreprise  soit  même  d'une  branche 
industrielle entière, ne constitue pas une mesure socialiste, du moment que le reste de l'économie 
reste basé sur le principe de  l'initiative privée, de l'économie pour le marché,  et  que la branche 
nationalisée elle-même se soumet à la discipline du marché, notamment en couvrant ses frais par 
ses ventes et non par les subsides de l'état. 
\bigskip
...  Le  seul  point  que  nous  voulions  mettre  en  évidence,  c'est  qu'entre  une  économie  socialisée  et 
une  économie  dont  certaines  branches  ont  été  nationalisées,  il  y  a  plus  qu'une  différence 
quantitative; l'atmosphère est différente, ou du moins pourrait être différente (1). » 
\bigskip
(1)  B.  S.  Chlepner,  Réflexions  sur  le  problème  des  nationalisations,  Vivue  de  l'Institut  de 
Sociologie, 1949, p. 219. 
\bigskip
P  384 :  « Le  second  genre  de  ripostes  concerne  la  possibilité  de  l'arrêt  après  une  certaine  étape. 
D'habitude, l'arrêt sera garanti par la création d'un cadre formel, juridique, qui empêcherait d'aller 
au-delà de ce qui a été décidé. Le tout est de savoir dans quelle mesure un formalisme est à même 
de  s'opposer  à  une  évolution  naturelle.  Une  manière  habituelle  aussi  de  prévoir  l'arrêt,  c'est  de 
faire état d'un équilibre de forces qui empêcherait d'aller indéfiniment dans une certaine direction 
: on suppose l'existence d'un pluralisme, autorisant l'espoir d'une résistance qui croîtrait au fur et à 
mesure que l'on continue dans un certain sens ; c'est l'argument qui convient aux adversaires des 
solutions extrêmes. » 
\bigskip
P  384-385 :  « Enfin,  un  autre  argument  consiste  à  montrer  que  l'on  se  trouve  déjà  sur  la  pente 
fatale  que  l'on  redoute,  et  qu'il  est  indispensable  de  faire  un  premier  pas,  dans  une  certaine 
direction, pour pouvoir, après lui, s'arrêter. C'est l'argument préféré de Démosthène. A ceux qui ne 
voulaient  pas  secourir  Mégalopolis,  menacée  par  Sparte,  parce  qu'elle  était  une  alliée  de  Thèbes, 
Démosthène réplique : 
\bigskip
Si  les  Lacédémoniens  prennent  Mégalopolis,  Messène  sera  en  danger.  S'ils  prennent  encore 
Messène, je dis que nous ferons alliance avec les Thébains. Alors n'est-il pas plus avantageux et 
plus  honorable  d'accueillir  spontanément  les  alliés  de  Thèbes  et  de  ne  pas  nous  prêter  aux 
convoitises  des  Lacédémoniens,  que  d'hésiter  à  sauver  un  peuple,  parce  qu'il  est  allié  des 
Thébains,  de  le  sacrifier,  au  risque  d'avoir  un  jour  à  sauver  les  Thébains  eux-mêmes,  non  sans 
nous mettre nous-mêmes en danger ? (1). » 
\bigskip
(1) Démosthène, Harangues, f. 1 ; Pour les Mégalopolitains, §§ 20-21. 
\bigskip
P  385 :  « Il  faut  donc,  selon  Démosthène,  faire  un  pas,  pour  ne  pas  être  entraîné  à  en  faire  un 
autre, beaucoup plus grave. 
\bigskip
On  peut  se  demander  si  les  deux  genres  de  ripostes,  celles  qui  mettent  l'accent  sur  la  nature  du 
chemin et celles qui portent sur la possibilité de s'arrêter, peuvent se combiner, à l'intention d'un 
\bigskip
\bigskip
\bigskip
199 
\bigskip
auditeur  unique.  Il  semble  bien  que  oui.  Celui-ci  sera  utilement  rassuré  par  une  argumentation 
d'ensemble,  lui  montrant  qu'il  s'agit  d'autre  chose  que  ce  qu'il  craignait  et  lui  montrant  la 
possibilité de s'arrêter. 
\bigskip
L'argument  de  la  direction  peut  prendre  diverses  formes  l'une  de  celles-ci  est  l'argument  de  la 
Propagation.  Il s'agit de mettre en garde contre certains phénomènes qui, par l'intermédiaire de 
mécanismes  naturels  ou  sociaux,  auraient  tendance  à  se  transmettre  de  proche  en  proche,  à  se 
multiplier, et à devenir, par cette croissance même, nocifs. 
\bigskip
Si  le  phénomène  initial  est,  lui-même,  considéré  déjà  comme  un  mal,  on  aura  recours  le  plus 
souvent à la  notion  de  contagion. C'est ainsi que Pitt conseille d'écraser dans l'oeuf les principes 
révolutionnaires : 
\bigskip
Si  jamais  les  principes  du  jacobinisme  devaient  triompher  dans  les  îles  françaises  des  Indes 
occidentales pourrions-nous espérer sauvegarder les nôtres de la contagion (2) ?  
\bigskip
Dans  l'argument  de  contagion,  il  y  a  donc  collusion  entre  deux  points  de  vue  dévaluants,  ce  que 
l'on redoute comme jalon est, en même temps, stigmatisé comme un mal. » 
\bigskip
(2) W. Pitt, Orations on the French war, p. 61, 30 déc. 1794. 
\bigskip
P 385-386 : « La perspective est toute différente dans l'argument de la  vulgarisation.  On met en 
garde contre la propagation qui dévaluerait, en le rendant commun et vulgaire, ce qui est distingué 
parce  que  rare,  limité,  secret.  A  l'inverse,  mais  dans  une  perspective  analogue,  l'argument  de  la 
consolidation met en garde contre les répétitions qui donnent pleine signification et valeur à ce qui 
n'était qu'ébauche, balbutiement, fantaisie, et qui deviendra mythe, légende, règle de conduite. » 
\bigskip
P 386 : « Enfin, il y a une série de variantes de l'argument de direction qui mettent l'accent sur le 
changement de nature entre les premières étapes et l'aboutissement. Le type peut en être pris dans 
le sorite grec, où le passage du tas de blé au tas moins un grain, toujours renouvelé, aboutit à ce qui 
n'est plus un tas. Le changement pourra être interprété comme un véritable changement de nature, 
ou  comme  la  révélation  de  la  véritable  nature  des  premiers  pas.  Peu  importe.  Il  faut  y  prêter 
attention. Ainsi : 
\bigskip
Chaque concession faite à l'ennemi et à l'esprit de facilité en entraÎnait une autre. Celle-ci n'était 
pas plus grave que la première, mais les deux, bout à bout, formaient une lâcheté. Deux lâchetés 
réunies faisaient le déshonneur (1). 
\bigskip
Le comique de ces changements de nature donne lien à des plaisanteries, comme celle de Cicéron, 
disant de la famille des Lentulus, où régulièrement les enfants étaient plus petits que les parents, 
qu'elle mourrait à force de naissances (2). 
\bigskip
Tous  ces  développements,  qu'ils  soient  marqués  par  l'idée  de  contagion,  de  vulgarisation,  de 
consolidation,  de  changement  de  nature,  montrent  qu'un  phénomène,  inséré  dans  une  série 
dynamique,  acquiert  une  signification  différente  de  celle  qu'il  aurait,  pris  isolément.  Cette 
signification varie selon le rôle qu'on lui fait jouer dans cette série. » 
\bigskip
(1) A. Camus, Actuelles, P. 57.  
(2) Quintilien, Vol. II, liv. VI, chap. 111, § 67. 
\bigskip
§ 67. LE DEPASSEMENT 
\bigskip
P 387 : « A l'encontre  de l'argument  de direction, qui fait craindre qu'une action ne nous engage 
dans un engrenage dont on redoute l'aboutissernent, les arguments du dépassement insistent sur 
\bigskip
\bigskip
\bigskip
200 
\bigskip
la possibilité d'aller toujours plus loin dans un certain sens, sans que l'on entrevoie une limite dans 
cette direction, et cela avec un accroissement continu de valeur. Comme le dit une paysanne, dans 
un recueil de Jouhandeau : « Plus c'est bon, meilleur c'est »(1). Ainsi Calvin affirme que jamais on 
n'exagère dans la direction qui attribue toute gloire, toute vertu à Dieu : 
\bigskip
Mais nous ne lisons point qu'il y en ait eu de repris pour avoir trop puisé de la source d'eaux vives 
(2). 
\bigskip
On  peut,  en  le  présentant  sous  ce  jour,  défendre  un  comportement  que  les  auditeurs  seraient 
tentés de blâmer, mais que l'on situera dans le prolongement de ce qu'ils approuvent et admirent : 
par  exemple,  le  fanatisme  nationaliste  ou  religieux  aux  yeux  des  patriotes  ou  des  croyants.  On 
peut, d'ailleurs, se servir du dépassement pour dévaluer un état, une situation, dont on aurait pu se 
contenter, mais auquel un état plus favorable est censé pouvoir succéder. A ceux qui estimaient la 
situation  militaire  assez  bonne  pour  entamer  des  négociations  de  paix  avec  la  France,  Pitt 
répondait : 
\bigskip
Que nous soyons plus en sécurité aujourd'hui, non seulement je l'admets, mais je prétends même 
que les perspectives s'améliorent de jour en jour, et que cette sécurité est de plus en plus assurée 
(3). » 
\bigskip
(1) M. Jouhandeau, Un monde, P. 231.  
(2) Calvin, Institution (le la religion chrétienne, Au Roy de France, p. 7.  
(3) W. Pitt, Orations on the French war, p. 93 (27 mai 1795). 
\bigskip
P  387-388 :  « Ce  qui  vaut,  ce  n'est  pas  de  réaliser  un  certain  but,  d'arriver  à  une  certaine  étape, 
mais de continuer, de dépasser, de transcender, dans le sens indiqué par deux ou plusieurs jalons. 
L'important,  n'est  pas  un  but  bien  défini  :  chaque  situation  sert,  au  contraire,  de  jalon  et  de 
tremplin permettant de poursuivre indéfiniment dans une certaine direction. » 
\bigskip
P 388 : « Cette forme de raisonnement n'est pas seulement utilisée pour promouvoir une certaine 
conduite,  mais  aussi,  surtout  dans  des  ouvrages  philosophiques,  pour  définir  certaines  notions  « épurées » en partant de conceptions de sens commun que l'on présente comme un point de départ. 
C'est ainsi que Sartre, à partir d'une notion de la mauvaise foi, qui s'inspire, au premier abord, du 
sens commun, aboutit, à force de dépassement, à une conception qui en est bien éloignée, et selon 
laquelle  tout  engagement  dans  le  social  et  dans  le  rationnel  serait  plus  ou  moins  empreint  de 
mauvaise foi (1). 
\bigskip
De même, Claparède, dans une amusante analyse, à laquelle nous avons déjà fait allusion ailleurs, 
montre  comment  le  sens  du  mot  «associationnisme  »  évolue  toujours  un  peu  plus  dans  une 
certaine  direction.  Cette  évolution  rappellerait  l'attitude  de  ce  révolutionnaire  par  tempérament, 
qui se définit, non par un programme déterminé, mais par le fait d'être toujours plus à gauche (2). 
\bigskip
Pour  fonder  cette  conception  d'une  direction  illimitée,  et  dont  les  termes  sont  hiérarchisés,  on 
présentera  au  bout  un  idéal  inaccessible,  mais  dont  les  termes  réalisables  constituent  des 
incarnations  de  plus  en  plus  parfaites,  de  plus  en  plus  pures,  de  plus  en  plus  proches  du  terme 
ultime  (3)  ;  elles  en  seraient  le  «miroir»,  1'«image»,  c'est-à-dire  qu'il  y  a,  de  l'idéal  à  elles,  un 
mouvement  descendant  qui  garantit  le  caractère  inaccessible  de  celui-ci,  quels  que  soient  les 
progrès réalisés. » 
\bigskip
(1)  J.  -P.  Sartre,  L'être  et  le  néant,  p.  109.  Cf.  §  48:  Techniques  visant  à  présenter  des  thèses 
comme compatibles ou incompatibles. 
(2)  Claparède,  La  genèse  de  l'hypothèse,  p.  45  ;  cf.  Ch.  Perelman,  et  L.  Olbrechts-Tyteca,  Les 
notions et l'argumentation, pp. 260-261 et § 35 : Usages argumentatifs et plasticité des notions. 
\bigskip
\bigskip
\bigskip
201 
\bigskip
(3) Cf. Plotin, Ennéades, 1, 2, § 6. 
\bigskip
P 388-389 : « Dans d'autres cas, l'idéal ne se conçoit que grâce aux termes inférieurs, auxquels on 
s'oppose et que l'on dépasse. C'est ainsi que, pour Lecomte du : 
\bigskip
L'homme reste donc biologiquement un animal. Nous verrons, par la suite, que cet état de choses 
était nécessaire, car c'est en luttant contre ses instincts qu'il s'humanise (1). » 
\bigskip
P 389 : « Souvent cette technique est utilisée pour transformer les arguments contre en arguments 
pour, pour montrer que ce qui était considéré jusqu'à présent comme un obstacle, est en réalité un 
moyen pour arriver à un stade supérieur, comme la maladie qui rend l'organisme plus résistant, en 
l'immunisant. 
\bigskip
La  réfutation  de  l'argumentation  par  le  dépassement  se  trouve  dans  la  constatation  qu'il  est 
impossible  d'aller  indéfiniment  dans  la  direction  préconisée,  soit  parce  que  l'on  aboutit  à  un 
absolu, soit parce que l'on aboutit à une incompatibilité. Aboutir à un terme absolu, parfait, c'est 
reconnaître qu'il faut renoncer à la progression. Pascal, adoptant le point de vue cartésien dans sa 
manière de traiter les définitions, affirme que 
\bigskip
en poussant les recherches de plus en plus, on arrive nécessairement à des mots primitifs qu'on 
ne  peut  plus  définir,  et  à  des  principes  si  clairs  qu'on  n'en  trouve  plus  qui  le  soient  davantage 
pour servir à leur preuve (2). » 
\bigskip
(1) Lecomte Du Nouÿ, L'homme et sa destinée, P. 100. 
(2) Pascal, Oeuvre, Bibl. de la Pléiade, De l'esprit géométrique et de l'art de persuader, p. 362. 
\bigskip
P 389-390 : « Il n'est plus question, dans ces conditions, de poursuivre encore un idéal, d'accroître 
une  valeur,  la  perfection  obtenue  s'opposant  à  la  perfectibilité.  D'autre  part,  ce  qui  peut  aussi 
s'opposer  à  la  progression  continue,  au  dépassement,  c'est  que  l'on  soit  conduit  au  ridicule, 
résultant  de  l'incompatibilité  avec  des  valeurs  auxquelles  on  répugne  à  renoncer  :  il  faut  donc 
chercher un équilibre qui permette d'harmoniser des valeurs qui, à la limite, entreraient en conflit. 
Mettre en garde contre les excès auxquels peut mener la fidélité illimitée à une maxime, à une ligne 
de conduite, c'est toujours faire intervenir d'autres valeurs dont on exige le respect. C'est ainsi que 
les  stoïciens  mettent  en  garde  contre  l'excès  de  mépris  pour  le  corps  qui  mènerait  à  nu  suicide 
déraisonnable. C'est ainsi qu'un théologien qui prétend que les voies de Dieu sont impénétrables, 
est  obligé  de  limiter  cette  affirmation  d'une  façon  ou  de  l'autre,  à  moins  de  rendre  la  théologie 
impossible. Il dira, par exemple, que les voies de Dieu sont impénétrables à la lumière naturelle, ou 
qu'elles sont impénétrables sans révélation. » 
\bigskip
P  390 :  « Dans  l'argumentation  ayant  recours  au  dépassement,  ce  qui  intéresse  souvent  les 
auditeurs, bien plus que le terme ultime dans une direction donnée, toujours fuyant, c'est la valeur 
que cette argumentation confère à certains termes situés en deçà, et sur lesquels porte en réalité le 
débat. 
\bigskip
C'est  ce  qui  ressort  nettement  de  l'examen  des  figures  destinées  àréaliser  le  dépassement.  Nous 
songeons avant tout à l'hyperbole et à la litote. 
\bigskip
L'hyperbole est une manière de s'exprimer outrancière. Comme le disait Dumarsais : 
\bigskip
Nous nous servons de mots qui, à les prendre à la lettre, vont au delà de la vérité, et représentent 
le  plus  ou  le  moins,  pour  faire  entendre  quelque  excès  en  grand  ou  en  petit.  Ceux  qui  nous 
entendent, rabattent de notre expression ce qu'il en faut rabattre (1)... » 
\bigskip
\bigskip
\bigskip
\bigskip
202 
\bigskip
(1) Dumarsais, Des Tropes, p. 98. 
\bigskip
P 390-391 : « L'hyperbole diffère de l'argumentation habituelle par le dépassement en ce que, elle 
n'est  pas  justifiée  ni  préparée,  mais  lancée  brutalement  :  son  rôle,  cependant,  est  de  donner  une 
direction à la pensée, de l'orienter dans l'appréciation de cette direction, et, seulement par un choc 
en  retour,  de  donner  une  indication  sur  le  terme  qui  importe.  D'où  la  marge  énorme  de  liberté 
dans les énoncés, qu'il s'agisse de simples affirmations de fait, comme dans cet exemple, pris dans 
l'Énéide : 
\bigskip
Deux pies jumeaux menacent le ciel (1)  
\bigskip
ou de comparaison, comme dans cet autre exemple, pris dans l'oraison funèbre de Condé : 
\bigskip
semblable,  dans  ses  sauts  hardis  et  dans  sa  légère  démarche  a  ces  animaux  vigoureux  et 
bondissants, il ne s'avance que par vives et impétueuses saillies, et n'est arrêté ni par montagnes 
ni par précipices (2). » 
\bigskip
(1) Cité par Quintilien, Vol. III, liv. VIII, chap. Vl, § 68 (Enéide, chant I, vv. 162-163). 
(2)  Cité  par  Saint-Aubin,  Guide  pour  la  classe  de  rhétorique,  p.  90.  (Texte  Bibl.  de  la  Pléiade, 
Oraison funèbre de Louis de Bourbon, prince de Condé, p.216.) 
\bigskip
P 391 : « Les hyperboles qui utilisent des expressions concrètes, ne visent pas, ainsi qu'Erdmann 
l'a déjà remarqué, à faire image (3). Leur rôle est de donner une référence qui, dans une direction 
donnée, attire l'esprit, pour ensuite l'obliger à revenir quelque peu en arrière, à la limite extrême 
de ce qui lui paraît compatible avec son idée de l'humain, du possible, du vraisemblable, avec tout 
ce qu'il admet par ailleurs. » 
\bigskip
(3) K. O. Erdmann, Die Bedeutung des Wortes, p.224 
\bigskip
P  391-392 :  « Dumarsais,  qui  ne  voit  dans  l'hyperbole  que  l'élément  exagération,  et  non  le 
dépassement,  lequel  nous  semble  essentiel,  répugne  à  cette  façon  de  s'exprimer  propre  «  aux 
orientaux  »  et  «aux  jeunes  gens».  Il  préconise  de  ne  l'utiliser  qu'avec  des  précautions  oratoires 
telles  que  «  pour  ainsi  dire  »,  «  si  l'on  peut  parler  ainsi  »,  qui  n'en  feraient  plus  qu'une  simple 
figure  de  style.  Or  ces  précautions  oratoires,  celui-là  même  qui  les  utilise,  ne  vent  pas  qu'on  les 
prenne trop au sérieux. Car un dépassement est bien ce que vise l'hyperbole, quand elle a, ce qui 
est presque toujours le cas, un but argumentatif. Telle cette maxime d'Audiberti, citée par Paulhan 
comme exemple d'hyperbole : 
\bigskip
\bigskip
Rien ne sera que ce qui fut (1) 
\bigskip
\bigskip
et qui, par le dépassement, donne valeur positive au passé. » 
\bigskip
(1) J. Paulhan, Les figures ou la rhétorique décryptée, Cahiers du Sud, p. 370. 
\bigskip
P  392 :  « Signalons  que  les  anciens  distinguaient  souvent  deux  genres  d'hyperboles,  considérés 
comme très différents, l'amplification et l'atténuation. Un exemple de ce dernier genre serait : 
\bigskip
\bigskip
Ils n'ont plus que la peau et les os (2) 
\bigskip
\bigskip
Mais par son caractère abstrait, la maxime d'Audiberti, qui pourrait être interprétée de l'une ou de 
l'autre manière, nous montre bien que l'amenuisement et le grossissement sont, dans l'hyperbole, 
un seul et même processus de dépassement. 
\bigskip
\bigskip
\bigskip
\bigskip
203 
\bigskip
La litote, elle, est généralement définie par contraste avec l'hyperbole, comme étant une façon de 
s'exprimer qui semble affaiblir la pensée (3). L'exemple classique en est « va je ne te hais point » 
de  Chimène  (4).  Dumarsais  cite  notamment  encore  «il  n'est  pas  sot  »,  «  Pythagore  n'est  pas  un 
auteur méprisable », « je ne suis pas difforme ». 
\bigskip
Si la litote peut être opposée à l'hyperbole, c'est parce que, pour établir une valeur, elle prend appui 
en deçà de celle-ci au lieu de le prendre dans le dépassement. » 
\bigskip
(2) Quintilien, Vol. III, liv. VIII, chap. Vl, 73. 
(3) Dumarsais, Des Tropes, p. 97. 
(4) Corneille, Le Cid, acte III, se. IV. 
\bigskip
P 392-393 : « Le plus souvent, la litote s'exprime par une négation. Sans doute est-il des litotes à 
forme  d'assertion,  telles  «  c'est  assez  bon»,  lorsque  cette  expression  désigne  une  valeur  très 
appréciée. Mais c'est dans la litote par négation que nous serions tentés de voir le mécanisme type 
de cette figure. Le terme mentionné, et repoussé, doit servir de tremplin pour que la pensée prenne 
la direction voulue. On suggère que ce terme eût pu normalement être admis comme adéquat, dans 
ces  circonstances,  et  étant  donné  les  informations  dont  disposait  l'auditeur.  Chimène  affirme 
qu'elle  aurait  dû  haïr,  qu'il  eût  été  normal  de  haïr,  et que  son  auditeur  pourrait  le  croire.  C'est  à 
partir  de  cette  négation  du  normal  que  la  pensée  est  dirigée  vers  d'autres  termes.  Or  le  terme 
repoussé est souvent lui-même une hyperbole. Dans « Pythagore n'est pas un auteur méprisable » 
l'effet de surprise est causé par cette hyperbole, évoquée pour être aussitôt rejetée. » 
\bigskip
P 393 : « Plus encore que l'hyperbole, la litote exige que l'auditeur connaisse un certain nombre de 
données qui le guideront dans son interprétation. « Il n'est pas sot » peut être pris dans un sens 
statique  ou  comme  élan  vers  une  direction.  D'où  l'intérêt  qu'il  y  a  à  user  de  litotes  basées  sur  le 
rejet d'une hyperbole. 
\bigskip
Les  relations  entre  ces  deux  figures  sont  donc  beaucoup  plus  complexes,  pensons-nous,  qu'il  n'y 
paraiît  communément.  L'hyperbole  aurait  souvent  pour fonction  de  préparer  la  litote,  dont,  sans 
elle, l'intention pourrait nous échapper. Cette dernière n'est donc pas toujours, comme on le dit, 
une confession à mi-voix (1). 
\bigskip
Remarquons,  à  ce  propos,  que  la  litote  peut  se  transformer  en  ironie,  par  suppression  de  la 
négation.  D'un  même  homme  difforme,  dont  par  litote  on  disait,  «  ce  n'est  pas  un  Adonis  »  on 
pourra dire, par ironie « c'est un Adonis ». Dans le premier cas, nous avons un mouvement de la 
pensée, le long d'une échelle de valeurs, dans l'autre, une confrontation entre une qualification et 
une  réalité  perçue.  Dans  le  premier  cas,  c'est  la  direction  qui  domine,  dans  le  second,  on  ne 
souhaite  pas  que  l'esprit  revienne  aussitôt  en  arrière,  mais  qu'il  constate  le  ridicule  né  d'une 
incompatibilité. 
\bigskip
L'hyperbole, souvent involontairement comique, peut produire cet effet d'une façon préméditée. » 
\bigskip
(1) Cl.-L. Estève, Etudes philosophiques sur l'expression littéraire, p. 87. 
\bigskip
P 394 : « Citons cette boutade rapportée par le pseudo-Longin : 
\bigskip
Il  possédait  une  terre  à  la  campagne,  qui  n'était  pas  plus grande  qu’une  épître  de  Lacédémonien 
(1). 
\bigskip
Il  s'agit  ici  du  comique  de  l'argumentation.  Sans  l'existence  d'hyperboles  sérieuses,  l'auteur 
aurait-il conçu ce trait amusant ? 
\bigskip
\bigskip
\bigskip
204 
\bigskip
B) LES LIAISONS DE COEXISTENCE 
\bigskip
§ 68. LA PERSONNE ET SES ACTES 
\bigskip
Alors  que,  dans  les  liaisons  de  succession,  les  termes  confrontés  se  trouvent  sur  un  même  plan 
phénoménal,  les  liaisons  de  coexistence  unissent  deux  réalités  de  niveau  inégal,  l'une  étant  plus 
fondamentale, plus explicative que l'autre. C'est le caractère plus structuré de l'un des termes qui 
distingue cette espèce de liaison, l'ordre temporel des éléments étant tout à fait secondaire : nous 
parlons de liaisons de coexistence, non pas pour insister sur la simultanéité des termes, mais pour 
opposer cette sorte de liaisons du réel aux liaisons de succession où l'ordre temporel est primordial 
(2). La liaison de coexistence fondamentale, en philosophie, est celle qui rattache une essence à ses 
manifestations.  Il  nous  semble  cependant  que  le  prototype  de  cette  construction  théorique  se 
trouve dans les rapports qui existent entre une personne et ses actes. C'est par l'examen de  cette 
relation que nous commencerons notre analyse (3). » 
\bigskip
(1) Longin, Traité (lu sublime, chap. XXXI, p. 151. 
(2) Semblablement, A. Angyal, Foundations for a science of personality, chap. VIII. 
(3) Pour les § 68 à 71, et. Ch. Perelman et L. Olbrechts-Tyteca, Rhétorique et philosophie, pp. 49 a 
84 : (Acte et personne dans l'argumentation). 
\bigskip
P 394-395 : « La construction de la personne humaine, que l'on sous-tend aux actes, est liée à une 
distinction entre ce que l'on considère comme important, naturel, propre à l'être dont on parle, et 
ce  que  l'on  considère  comme  transitoire,  manifestation  extérieure  du  sujet.  Cette  liaison  entre  la 
personne  et  ses  actes  ne  constituant  pas  un  rapport  nécessaire,  ne  possédant  pas  les  mêmes 
caractères de stabilité que la relation qui existe entre un objet et ses qualités, la simple répétition 
d'un acte peut entraîner, soit une reconstruction de la personne, soit une adhésion renforcée à la 
construction antérieure. » 
\bigskip
P 395 : « Il va sans dire que la conception de la personne peut varier beaucoup selon les époques et 
selon la métaphysique que l'on adopte. L'argumentation des primitifs se servirait d'une idée de la 
personne  beaucoup  plus  large  que  la  nôtre:  en  feraient,  salis  doute,  partie  toutes  les 
appartenances,  l'ombre,  le  totem,  le  nom,  les  fragments  détachés  du  corps,  entre  lesquels  et 
l'ensemble  de  la  personne,  nous  n'établirions,  le  cas  échéant, qu'une  liaison  symbolique.  Un  seul 
exemple,  la  beauté  d'une  femme,  suffit  pour  montrer  comment  un  même  phénomène  peut  être 
considéré,  soit  comme  partie  intégrante  de  la  personne,  de  son  essence,  soit  comme  une  de  ses 
manifestations transitoires, c'est-à-dire un simple acte. 
\bigskip
En rattachant un phénomène à la structure de la personne, on lui accorde un statut plus important 
:  c'est  dire  que  la  manière  de  construire  la  personne  pourra  faire  l'objet  d'accords  limités, 
précaires,  particuliers  à  un groupe  donné,  accords  susceptibles  de  révision sous  l'influence  d'une 
nouvelle conception religieuse, philosophique ou scientifique. 
\bigskip
L'idée de « personne» introduit un élément de stabilité. 'fout argument sur la personne fait état de 
cette  stabilité  :  on  la  présume,  en  interprétant  l'acte  en  fonction  de  la  personne,  on  déplore  que 
cette stabilité n'ait pas été respectée, quand on adresse à quelqu'un le reproche d'incohérence ou 
de changement injustifié. Un grand nombre d'argumentations tendent à prouver que la personne 
n'a pas changé, que le changement est apparent, que ce sont les circonstances qui ont changé, etc. 
(1). 
\bigskip
(1) Cf. N. Leites, The third international on ils changes of policy, dans l'ouvrage collectif édité par 
H. Lasswell, Language of politics. 
\bigskip
P 396 : « Toutefois, la stabilité de la personne n'est jamais complètement assurée : des techniques 
linguistiques contribueront à accentuer l'impression de permanence; la plus importante est l'usage 
\bigskip
\bigskip
\bigskip
205 
\bigskip
du nom propre. La désignation de la personne par certains traits (votre avare de père), l'hypostase 
de  certains  sentiments  (celle  dont  la  fureur  poursuivit  votre  enfance),  peuvent  également  y 
concourir.  La  qualification,  l'épithète  (ce  héros,  Charlemagne  à  la  barbe  fleurie)  visent  à  rendre 
immuable  certains  caractères,  dont  la  stabilité  renforce  celle  du  personnage.  C'est  grâce  à  cette 
stabilité qu'un  mérite  acquis,  ou que l'on  va  acquérir,  peut  être  attribué  à quelqu'un  d'une  façon 
intemporelle. Comme le remarque justement Kenneth Burke : 
\bigskip
Un  héros  est  tout  d'abord  un  homme  qui  accomplit  des  choses  héroïques;  et  son  «  héroïsme  » 
réside dans ses actes. Mais ensuite, un héros peut être un homme avec des potentialités d'action 
héroïque. Les soldats qui s'en vont à la guerre sont des héros dans ce sens... Ou un homme peut 
être considéré comme un héros parce qu'il a accompli des actes héroïques, alors que dans son état 
actuel, il peut être, en tout cas, trop vieux ou trop faible pour les accomplir (1). 
\bigskip
Mais  cette  stabilité  de  la  personne,  qui  la  fait  quelque  peu  ressembler  à  une  chose,  avec  ses 
propriétés déterminées une fois pour toutes, s'oppose à sa liberté, sa spontanéité, sa possibilité de 
changer. Aussi est-on beaucoup plus enclin de stabiliser les autres que soi-même : 
\bigskip
les autres peuvent avoir, et ils ont souvent, des qualités très supérieures aux miennes, mais leurs 
qualités  adhèrent  à  eux  beaucoup  plus  que  mes  défauts  n'adhèrent  à  moi  :  s'ils  sont  généreux, 
intelligents, travailleurs, séduisants, ils le resteront comme ils resteront avares, bêtes, paresseux, 
ennuyeux, s'ils sont ainsi faits. Moi pas. je ne suis pas poète; mais dans une seconde peut-être le 
deviendrai-je. L'ouvrage que je n'ai Pas pu faire, rien ne s'oppose à ce que je le fasse demain. Cette 
plasticité, Sylvia aussi la possédait, mélange de fait et de doute (2). » 
\bigskip
(1) K. Burke, A Grammar of motives, p. 42.  
(2) E. Berl, Sylvia, p. 86. 
\bigskip
P  397 :  « C'est  accorder  à  Sylvia,  vue  pour  la  première  fois,  un  véritable  privilège,  que  de  lui 
reconnaître  cette  plasticité  que  chacun  s'accorde  spontanément  tout  en  la  déniant  tout  aussi 
spontanément  à  autrui.  Toute  mise  en  péril  de  cette  faculté  de  renouvellement  est  ressentie  de 
façon désagréable. D'où, sans doute, le malaise que l'on éprouve à entendre des amis parler, même 
avec éloge, de la conduite que l'on va tenir (1). 
\bigskip
L'existentialisme, en mettant l'accent sur la liberté de la personne, qui l'opposerait nettement aux 
choses,  a  pu  élaborer  une  ontologie  originale.  Des  pages,  qui  semblent  d'une  métaphysique 
compliquée, affirment uniquement que l'on refuse de voir dans le rapport de la personne à ses 
actes une simple réplique du rapport entre un objet et ses propriétés (2). L'objet, défini à partir de 
ses propriétés, fournit le modèle d'une conception de la personne, stabilisée à partir de certains de 
ses actes, transformés en qualités, en vertus, que l'on intègre dans une essence invariable. Mais  si 
la  personne  ne  possédait  pas  le  pouvoir  de  se  transformer,  de  se  modifier,  de  se  convertir,  de 
tourner de quelque manière le dos àson passé, la formation éducative serait un leurre, la morale 
n'aurait  pas  de  sens,  et  les  idées  de  responsabilité,  de  mérite  et  de  culpabilité,  liées  à  celle  de  la 
liberté  de  la  personne,  devraient  être  abandonnées  an  profit  d'une  simple  appréciation 
pragmatique des comportements. » 
\bigskip
( 1) Cf. .J. Paulhan, Entretien sur des laits divers, p. 67.  
(2) Cf. J.-P. Sartre, L'être et le néant, pp. 158 et suiv. 
\bigskip
P  397-398 :  « Dans  l'argumentation,  la  personne,  considérée  comme  support  d'une  série  de 
qualités, l'auteur d'une série d'actes et de jugements, l'objet d'une série d'appréciations, est un être 
durable  autour  duquel  se  groupe  toute  une  série  de  phénomènes  auxquels  il  donne  cohésion  et 
signification.  Mais,  comme  sujet  libre,  la  personne  possède  cette  spontanéité,  ce  pouvoir  de 
changer et de se transformer, cette possibilité d'être persuadée et de résister à la persuasion, qui 
\bigskip
\bigskip
\bigskip
206 
\bigskip
font  de  l'homme  un  objet  d'étude  sui  generis  et,  des  sciences  humaines,  des  disciplines  qui  ne 
peuvent se contenter de copier fidèlement la méthodologie des sciences naturelles. » 
\bigskip
P 398 : « C'est ainsi, pour prendre un exemple, que la morale et le droit ont  besoin des notions de 
personne et d'acte dans leur liaison et leur indépendance relative. La morale et le droit jugent à la 
fois l'acte et l'agent : ils ne pourraient se contenter de prendre en considération un seul de ces deux 
éléments. Du fait même qu'on le j tige lui, l'individu, et non ses actes, on admet qu'il est solidaire 
des actes qu'il a commis. Mais cependant, si l'on s'occupe de lui, c'est en raison d'actes, que l'on est 
capable de qualifier indépendamment de sa personne. Tandis que les notions de responsabilité, de 
mérite et de culpabilité sont relatives à la personne, celles de norme, de règle, se préoccupent avant 
tout de l'acte.  Toutefois cette dissociation de  l'acte et  de la personne  n'est jamais que partielle et 
précaire. On pourrait concevoir le mérite d'une personne indépendamment de ses actes, mais ce ne 
serait  possible  que  dans  une  métaphysique  où  la  référence  aux  actes  serait  fournie  dans  le 
contexte. D'autre part, si les règles prescrivent ou interdisent certains actes, leur portée morale ou 
juridique  réside  dans  le  fait  qu'elles  s'adressent  à  des  personnes.  Les  termes  de  la  relation  acte-
personne  sont  assez  indépendants  pour  permettre, quand  il  le  faut,  de  se  servir  de  chacun  d'eux 
isolément,  et  ils  sont  suffisamment  liés  pour  que  leur  intervention  conjointe  caractérise  des 
domaines entiers de la vie sociale. » 
\bigskip
§ 69. INTERACTION DE L'ACTE ET DE LA PERSONNE 
\bigskip
Après  ces  considérations  d'ordre  général,  nous  examinerons  successivement  l'influence  des  actes 
sur la conception de la personne, celle de la personne sur ses actes, pour terminer en signalant des 
situations où l'interaction est si marquée que l'analyse munie ne saurait donner le primat à l'un ou 
l'autre élément. » 
\bigskip
P  398-399 :  « La  réaction  de  l'acte  sur  l'agent  est  de  nature  à  modifier  constamment  notre 
conception  de  la  personne,  qu'il  s'agisse  d'actes  nouveaux  qu'on  lui  attribue,  ou  d'actes  anciens 
auxquels  on  se  réfère.  Les  uns  et  les  autres  jouent  un  rôle  analogue  dans  l'argumentation, 
quoiqu'une prépondérance soit accordée aux actes  les plus récents. Sauf dans des  cas limite, que 
nous  examinerons  dans  un  paragraphe  ultérieur,  la  construction  de  la  personne  n'est  jamais 
terminée, pas même à sa mort. Mais il va de soi que, plus un personnage recule dans l'histoire, plus 
l'image qu'on s'en forme devient rigide. Comme l'a bien remarqué R. Aron: 
\bigskip
L'autre,  présent,  nous  rappelle  sans  cesse  sa  capacité  de  changer,  absent  il  est  prisonnier  de 
l'image  que  nous  nous  sommes  faite  de  lui.  ...  Si  nous  distinguons  encore  en  nos  amis  ce  qu'ils 
sont de ce qu'ils font, cette distinction s'efface a mesure q ommes s'enfoncent dans le passé (1). » 
\bigskip
(1) R. Aron, Introduction à la philosophie de l'histoire, p. 80. 
\bigskip
P  399 :  « La  personne  coïnciderait  alors  avec  l'ensemble  structuré  de  ses  actes  connus  ;  plus 
précisément, dirons-nous, la relation entre ce qu'il faut considérer comme essence de la personne, 
et  les  actes  qui  n'en  sont  que  la  manifestation,  est  définie  une  fois  pour  toutes.  Pourtant  cette 
rigidité  n'est  que  relative  :  non  seulement  de  nouveaux  documents  peuvent  déterminer  une 
révision  mais,  en  dehors  de  tout  fait nouveau,  une  évolution  de  l'opinion  publique,  ou  une  autre 
conception  de l'histoire, peuvent modifier la conception du  personnage, par l'intégration dans sa 
structure d'actes négligés auparavant ou par la minimisation d'actes jugés importants jusque-là. » 
\bigskip
P 399-400 : « L'acte ne peut pas être considéré comme un simple indice, révélateur du caractère 
intime  de  la  personne,  lequel  serait  invariable,  mais  inaccessible  sans  l'intermédiaire  de  l'acte. 
Nous  sommes  quelque  peu  choqués  par  ce  passage  d'Isocrate,  qui  assimile  les  hommes  à  des 
champignons vénéneux : 
\bigskip
\bigskip
\bigskip
\bigskip
207 
\bigskip
Le meilleur serait en effet, si un signe distinguait les hommes vicieux, de les châtier avant qu'ils 
n'aient fait du tort à un de leurs concitoyens. Mais puisqu'on ne peut les reconnaître avant qu'ils 
n'aient fait du mal à quelqu'un, du moins quand ils sont découverts, convient-il que tout le monde 
les déteste et les regarde comme des ennemis de tous (1). » 
\bigskip
(1) Isocrate, Discours, t. 1 : Contre Lokhitès, § 14. 
\bigskip
P 400 : « Il en résulte que la punition devrait être proportionnelle, non pas à la gravité de l'offense, 
mais à la méchanceté de la nature qu'elle révèle. 
\bigskip
Dans  notre  conception  habituelle,  un  acte  est,  plutôt  qu'un  indice,  un  élément  permettant  de 
construire  et  de  reconstruire  notre  image  de  la  personne,  de  classer  celle-ci  dans  des  catégories 
auxquelles s'appliquent certaines qualifications, comme dans le célèbre passage de Pascal : 
\bigskip
Il  n'y  a  que  trois  sortes  de  personnes  :  les  unes  qui  servent  Dieu,  l'ayant  trouvé  ;  les  autres  qui 
s'emploient  à  le  chercher,  ne  l'ayant  pas  trouvé  ;  les  autres  qui  vivent  sans  le  chercher  ni  l'avoir 
trouvé. Les premiers sont raisonnables et heureux ; les derniers sont fous et malheureux; ceux du 
milieu sont malheureux et raisonnables (2). 
\bigskip
La  valeur  que  nous  attribuons  à  l'acte  nous  incite  à  attribuer  une  certaine  valeur  à  la  personne, 
mais  il  ne  s'agit  pas  d'une  valorisation  indéterminée.  Au  cas  où  un  acte  entraîne  un  transfert  de 
valeur, celui-ci est corrélatif d'un remaniement de notre conception de la personne, à laquelle nous 
attribuerons,  d'une  façon  explicite,  ou  implicite,  certaines  tendances,  aptitudes,  instincts  ou 
sentiments nouveaux. » 
\bigskip
(2) Pascal, Oeuvre Bibl. de la Pléiade, Pensées, 364 (61), p. 922 (257 éd. Brunschvicg).  
\bigskip
P  400-401 :  « Par  acte,  nous  entendons  tout  ce  qui  peut  être  considéré  comme  émanation  de  la 
personne,  que  ce  soient  des  actions,  des  modes  d'expression,  des  réactions  émotives,  des  tics 
involontaires,  ou  des  jugements.  Ce  dernier  point  est,  pour  notre  propos,  essentiel.  En  effet,  en 
accordant  une  certaine  valeur  à  un  jugement,  on  porte  par  là  même  une  appréciation  sur  son 
auteur; parfois d'ailleurs le jugement permet de juger le juge : 
\bigskip
Philanthe a du mérite, de l'esprit, de l'agrément, de l'exactitude sur son devoir, de la fidélité et de 
l'attachement  son  maître,  et il  en  est médiocrement  considéré  ;  il  ne  plaît pas,  il  n'est  pas goûté. 
Expliquez-vous : est-ce Philanthe, ou le grand qu'il sert, que vous condamnez ? (1) » 
\bigskip
(1) La Bruyère, Oeuvres, Bibl. de la Pléiade, Caractères, Des Grands, 8, p. 270. 
\bigskip
P 401 : « Le jugement sur le juge suppose un certain accord quant à la valeur de l'objet dont le juge 
a  traité  ;  c'est  en  mettant  en  cause  cet  accord  que  l'on  peut  arriver  à  modifier  le jugement  sur  le 
juge.  Par  contre,  lorsqu'on  prétend  juger  une  personne  aux  expressions  qu'elle  utilise,  le 
déplacement de la discussion sur l'objet est beaucoup plus difficile. Théodore Reinach relève chez 
Furtwängler,  dans  la  controverse  au  sujet  de  la  tiare  de  Saïtapharnès,  les  expressions  «  fraude 
grossière », « inventions méprisables », et conclut : 
\bigskip
Des jugements aussi excessifs jugent surtout celui qui les émet (2). 
\bigskip
Ici,  la  disqualification  de  l'adversaire  paraît  liée  à  un  défaut  d'impartialité;  dans  d'autres  cas,  on 
l'accusera de légèreté. Sans doute ne peut-on accuser de partialité ou de légèreté dans l'expression, 
que s'il y aaccord au sujet de l'objet. Toutefois, le plus souvent, on se réfère non à celui-ci, mais à 
une norme généralement admise de mesure, de bienséance, qui permettrait de disqualifier, en tout 
\bigskip
\bigskip
\bigskip
208 
\bigskip
état de cause, l'adversaire qui s'en écarte. D'où le danger bien connu de défendre une bonne cause 
par des expressions trop violentes. » 
\bigskip
(2) A. Vayson De Pradenne, Les fraudes en archéologie préhistorique, pp. 536537. 
\bigskip
P 401-402 : « Il est rare que la réaction de l'acte sur la personne se limite à une valorisation ou à 
une  dévalorisation  de  cette  dernière.  Le  plus  souvent  la  personne  sert,  pour  ainsi  dire,  de  relais 
permettant de passer des actes connus aux actes inconnus, de la connaissance d'actes passés à la 
prévision  d'actes  futurs.  Cette  technique  est  constamment  utilisée,  notamment  dans  les  débats 
judiciaires. Parfois ce procédé concernera des actes de même nature (qui n'a jamais été séditieux 
ne machinera pas de renverser des royaumes) (1), parfois il permettra de passer de certains actes à 
d'autres semblables (qui a porté un faux témoignage, n'hésitera pas à amener des faux témoins en 
sa  faveur)  (2),  parfois  il  se  se  compliquera  d'un  argument  a  fortiori  (qui  a  tué,  n'hésitera  pas  à 
mentir) (3). » 
\bigskip
(1) Calvin, institution de la religion chrétienne, Au Roy de France, p. 15. 
2) Isocrate, Discours, t. 1: Contre Callimakhos, §51. 
3) Cf. aussi Quintillien, Vol. Ii, liv. V, chap. X. § 87. 
\bigskip
P  402 :  « Les  actes  qui  servent  de  prémisse  peuvent  être  habituels;  ils  peuvent  être  rares  : 
l'important est qu'on les considère comme caractéristiques. Pour que l'acte unique ne réagisse pas 
sur  la  personne  il  faudra  des  techniques  particulières  dont  nous  parlerons  plus  loin  ;  les  erreurs 
accumulées  de  l'adversaire  peuvent  servir  àle  disqualifier  ;  une  seule  erreur  peut  aussi  y  être 
propice. 
\bigskip
Cette garantie d'un acte par un autre s'applique également aux opinions d'une personne. S. Weil, 
pour marquer sa défiance envers le thomisme, imprégné de pensée aristotélicienne, s'en prend à ce 
qu'Aristote a dit au sujet de l'esclavage: 
\bigskip
... bien que nous repoussions cette pensée d'Aristote, nous sommes forcément amenés dans notre 
ignorance à en accueillir d'autres qui ont été en lui la racine de celle-là. Un homme qui prend la 
peine d'élaborer une apologie de l'esclavage n'aime pas la justice. Le siècle où il vit n'y fait rien 
(4). 
\bigskip
Ce  qui  est  invoqué  ici,  sans  doute  est-ce  la  cohérence  entre  certaines  idées;  mais  c'est  par 
l'intermédiaire de la personne que cette cohérence est postulée ; car notre « ignorance » fait que 
nous ne pouvons la saisir autrement. » 
\bigskip
4) S. Weil, L'enracinement, p. 207. 
\bigskip
P  402-403 :  « Les  actes  passés  et  l'effet  qu'ils  produisent  en  arrivent  à  acquérir  une  sorte  de 
consistance,  former  un  passif  extrêmement  nuisible  ou  un  actif  très  estimable.  La  bonne 
renommée  dont  on  jouit  doit  être  prise  en  considération,  et  Isocrate  ne  manque  pas  de 
l'invoquer pour défendre ses clients : 
\bigskip
[je] serais le plus malheureux des hommes si après avoir dépensé beaucoup de mon argent pour 
l'État, je passais pour convoiter celui des autres et pour ne tenir aucun compte de votre mauvaise 
opinion, quand on me voit faire moins de cas, non seulement de nia fortune. mais même de ma 
vie que de la bonne renommée que vous accordez (1). » 
\bigskip
(1) Isocrate, Discours, t. 1 : Contre Callimakhos, ~ 6.1. 
\bigskip
\bigskip
\bigskip
\bigskip
209 
\bigskip
P 403 : « Avoir en, autrefois, souci de la bonne renommée devient un garant de ce que l'on ne ferait 
rien  qui  puisse  en  déterminer  la  perte.  Les  actes  antérieurs,  et  la,  bonne  renommée  qui  en  est 
résultée,  deviennent  une  sorte  de  capital  qui  s'est  incorporé  à  la  personne,  un  actif  que  l'on  a  le 
droit d'invoquer pour sa défense. 
\bigskip
Souvent l'idée que l'on se fait de la personne, au lieu de constituer un aboutissement, est plutôt le 
point de départ de l'argumentation et sert, soit à prévoir certains actes inconnus, soit à interpréter 
d'une certaine façon les actes connus, soit à transférer sur les actes le jugement porté sur l'agent. 
Une caricature de ce dernier procédé nous est tracée par La Bruyère : 
\bigskip
...  certaines  femmes  qui  ne  juraient  que  par  vous  et  sur  votre  parole,  qui  disaient  :  «  Cela  est 
délicieux; qu'a-t-il dit ? (2) » 
\bigskip
(2) La Bruyère, Oeuvres Bibl. de la Pléiade, Caractères, De la Société et de la conversation, 66, pp. 
188-189. 
\bigskip
P 403-404 : « Ce mécanisme de transfert ne suit pas nécessairement un ordre chronologique : la 
valorisation  peut  porter,  tout  aussi  bien,  sur  des  actes  antérieurs  au  moment  où  la  personne  a 
acquis une valeur éminente. « Quel génie ne sauve ses enfances ? » dit fort bien Malraux (3). Et, en 
fait, celui qui juge les oeuvres de jeunesse d'un grand artiste ne peut s'empêcher d'y voir les signes 
avantcoureurs de ce qui fera sa grandeur future. L'auteur d'œuvres géniales, créées à des époques 
diverses, est un génie : cette qualification rattache les actes à une qualité stable de la personne, qui 
rayonne ami bien sur les années antérieures à la période de production de chefs-d'œuvre que sur 
les années qui suivent. Il ne suffit plus de dire que le passé garantit l'avenir, mais que la structure 
stable de la personne permet de préjuger de ses actes ; cette réaction de la personne sur l'acte se 
manifeste  le  mieux  quand  une  qualification,  une  épithète,  met  particulièrement  en  évidence  ce 
caractère de stabilité. » 
\bigskip
(3) A. Malraux, Saturne, Essai sur Goya, p. 18. 
\bigskip
P 404 : « Pascal utilise ce transfert de la personne à l'acte pour établir le dilemme suivant : 
\bigskip
L'Alcoran  dit  que  saint  Matthieu  était  homme  de  bien.  Donc,  il  était  faux  prophète,  ou  en 
appelant  gens  de  bien  des  méchants,  ou  en  ne  demeurant  pas  d'accord  de  ce  qu'ils  ont  dit  de 
Jésus-Christ (1). 
\bigskip
D'une façon parallèle, tel névrosé, dont parle Odier, est incapable de soutenir un point de vue dans 
une discussion : 
\bigskip
Comment pourrait-il valoriser ses idées sans s'être au préalable valorisé lui-même ? (2). 
\bigskip
Souvent, un acte ambigu ne prend sa signification et sa portée que grâce à ce que l'on sait de son 
auteur.  C'est  ainsi  que,  dans  son  Éloge  d'Hélène,  Isocrate  rapporte  que  Thésée  a  enlevé  Hélène 
alors qu'elle n'était pas encore dans la fleur de l'âge, et il ajoute : 
\bigskip
A coup sûr, si l'auteur de ces exploits était un homme perdu dans la foule et non pas une nature 
exceptionnelle, mon discours ne montrerait pas encore avec évidence s'il est un éloge d'Hélène ou 
une  attaque  contre  Thésée...  Il  me  semble  convenable  de  parler  ici  de  lui  plus  longuement,  car 
j'estime que pour donner toute l'autorité nécessaire à ceux qui entreprennent l'éloge d'Hélène, le 
mieux est de montrer que ses amis et ses admirateurs ont été plus dignes euxmêmes d'admiration 
que les autres hommes (3). 
\bigskip
Suit un long éloge de Thésée. » 
\bigskip
\bigskip
\bigskip
210 
\bigskip
 
(1) Pascal, Œuvre Bibl. de la Pléiade, Pensées, 401 (457), p. 933 (597 éd. Brunschvicg). 
(2) Ch. Odier, L'angoisse et la pensé magique, p. 128. 
(3) Isocrate, Discours, t. 1 : Eloge d'Hélène, § 21, 22. 
\bigskip
P  404-405 :  « Il  y  a  plus.  Dans  certains  cas,  ce  que  nous  savons  de  la  personne,  non  seulement 
nous permet d'apprécier l'acte, mais constitue le seul critère pour le qualifier. C'est ainsi que, pour 
Pascal : 
\bigskip
Il  y  a  bien  de  la  différence  entre  n'être  pas  pour  Jésus-Christ  et  le  dire,  ou  n'être  pas  pour 
Jésus-Christ et feindre d'en être. Les uns peuvent faire des miracles, non les autres (1)... » 
\bigskip
(1) Pascal, Œuvre, Bibl. de la Pléiade, Pensées, 751 (461), P. 1065, (836 éd. Brunschvicg). 
\bigskip
P  405 :  « Les  miracles  provenant  d'ennemis  de  J.  C.  sont  possibles,  car  ils  sont  clairement 
diaboliques; quant aux autres, ils sont impossibles, car Dieu ne permettrait pas que l'on trompe les 
fidèles. 
\bigskip
L'intervention  de  la  personne,  comme  contexte  servant  à  1  interprétation  de  l'acte,  se  réalise 
souvent par le truchement de la notion d'intention, laquelle a pour fonction, à la fois, d'exprimer et 
de justifier la réaction de l'agent sur l'acte. » 
\bigskip
P  405-406 : « Lorsque  l'on  passe  de  la  connaissance  de  ses  actes  antérieurs  à  des  considérations 
sur  ses  actes  futurs,  le  rôle  de  la  personne  est  important,  mais  elle  n'intervient  que  comme  un 
chaînon privilégié dans l'ensemble des faits que l'on invoque. Par contre, dès qu'intervient l'appel à 
l'intention, on met l'accent essentiellement sur la personne et son caractère permanent. L'intention 
est,  en  effet,  liée  à  l'agent,  en  est  l'émanation,  résulte  de  son  vouloir,  de  ce  qui  le  caractérise 
intimement. L'intention d'autrui n'étant pas connue directement, on ne peut la présumer que par 
ce que l'on sait de la personne dans ce qu'elle a de durable. Parfois l'intention est révélée grâce à 
des actes répétés et concordants, mais il est des cas où seule l'idée que l'on a de l'agent permet de la 
déterminer.  Le  même  acte,  accompli  par  quelqu'un  d'autre,  sera  considéré  comme  différent  et 
autrement  apprécié,  parce  qu'on  le  croira  accompli  dans  une  intention  différente.  Le  recours  à 
l'intention  constituera  alors  le  noeud  de  l'argumentation  et  subordonnera  l'acte  à  l'agent,  dont 
l'intention permettra de comprendre  et d'apprécier l'acte. C'est ainsi que Calvin, en rappelant les 
afflictions  de  Job,  qui  peuvent  être  attribuées  simultanément  àDieu,  à  Satan,  et  aux  hommes, 
trouvera  que  Dieu  a  bien  agi,  Satan  et  les  hommes,  par  contre,  d'une  façon  condamnable,  parce 
que  leurs  intentions  n'étaient  pas  pareilles  (1).  Or  l'idée  que  nous  avons  de  celles-ci  dépend 
essentiellement de ce que nous savons des agents. » 
\bigskip
(1) Calvin, Institution de la religion chrétienne, liv. 1, chap. XVIII, § 1. 
\bigskip
P 406 : « Toute l'argumentation morale basée sur l'intention est une morale de l'agent, à opposer à 
une  morale  de  l'acte,  bien  plus  formaliste.  L'exemple  ci-dessus,  parce  qu'il  fait  intervenir  des 
agents aussi caractérisés que Dieu et Satan, montre fort bien le mécanisme de ces arguments, mais 
il n'est point de controverse morale où l'on ne s'en serve. Les intentions de l'agent, les motifs qui 
ont  déterminé  son  action,  seront  souvent  considérés  comme  la  réalité  qui  se  cache  derrière  des 
manifestations  purement  extérieures,  et  qu'il  faut  chercher  à  connaître  à  travers  les  apparences, 
car  ce  sont  eux,  en  fin  de  compte,  qui  auraient  seuls  de  l'importance.  L'ambassadeur  d'un  pays 
asiatique,  invité,  dans  un  restaurant  américain,  à  prendre  place  dans  une  chambre  réservée,  est 
flatté  de  cette  marque  de  distinction,  mais  il  proteste  avec  indignation  lorsqu'il  apprend  que,  en 
réalité, dans cette ville où règne la ségrégation raciale, on l'a pris pour un noir. 
\bigskip
\bigskip
\bigskip
\bigskip
211 
\bigskip
Cette technique d'interprétation par l'intention, Permettrait de juger l'agent, et pas seulement telle 
ou telle de ses œuvres. Les deux façons de juger, celle qui se rapporte à un critère formel et celle 
qui le dépasse, peuvent donner lieu à des jugements opposés. Comme le dit A. Lalande : 
\bigskip
on  parle,  non  sans  raison,  d'erreurs  intelligentes  :  Descartes  en  est  plein;  de  crimes  ou  de  délits 
honorables,  comme  saint  Vincent  de  Paul  trichant  pour  les  pauvres...  Un  roman  ou  un  paysage 
manqués font quelquefois dire : « Cela ne vaut rien, mais c'est d'un artiste » (2). » 
\bigskip
(2) A. Lalande, La raison et les normes, pp. 196-197. 
\bigskip
P  407 :  « Comment  prouver  l'existence  de  l'intention  alléguée  ?  En  établissant  notamment  des 
correspondances  entre  des  actes  divers  d'une  même  personne  et  en  suggérant  qu'une  même 
intention les avait déterminés : 
\bigskip
…  tous  savent  en  effet  que  les  mêmes  hommes  ont  causé  la  destruction  de  la  démocratie  et  le 
bannissement de mon père (1). 
\bigskip
Au delà des faits, l'énoncé insinue l'existence d'une même intention politique. 
\bigskip
La  recherche  de  la  véritable  intention  est  un  des  problèmes  centraux  du  théâtre  contemporain. 
Parfois le personnage tâtonne, les partenaires l'éclairent peu à peu sur la signification de ses actes. 
Dans  le  Chemin  de  crête  de  G.  Marcel,  ni  le  personnage  central,  ni  les  partenaires,  ni  les 
spectateurs n'arrivent à démêler les intentions ; seule une connaissance de l'agent réservée à Dieu 
pourrait donner aux actes leur signification indubitable. 
\bigskip
C'est l'ambiguïté des comportements humains, quand on les interprète en fonction de l'intention, 
qui marque un des points essentiels  par lesquels toute science de l'homme diffère profondément 
des  sciences  naturelles.  De  là,  d'ailleurs,  l'effort  des  behavioristes  pour  éliminer  ce  facteur 
d'incertitude  et  de  subjectivisme,  mais  au  prix  de  quelle  déformation  de  l'objet  même  que  l'on 
étudie  ?  La  psychanalyse  a  préféré  courir  le  risque  d'erreur  plutôt  que  de  renoncer  à  l'étude  de 
l'homme complet. 
\bigskip
La réaction de la personne sur ses actes est influencée par l'un des facteurs auxquels la psychologie 
sociale a accordé la plus grande importance, celui du prestige. » 
\bigskip
(1)  Isocrate,  Discours,  t.  1  ;  Sur  l'attelage,  §  4  ;  cf.  §  31  :  L'interprétation  du  discours  et  ses 
problèmes. 
\bigskip
P 407-408 : « Le prestige est une qualité de la personne qui se reconnaît à ses effets. C'est ce qui 
permet  à  E.  Dupréel  de  le  définir  comme  la  qualité  de  ceux  qui  entraînent  chez  les  autres  la 
propension à les imiter; il est donc lié de près au rapport de supériorité d'individu à individu et de 
groupe à groupe (1) ; il désigne, pour Lippitt et ses collaborateurs, ceux qui, dans leur milieu, sont 
les  plus  aptes  à  devenir  les  dirigeants,  à  obtenir  des  autres  qu'ils  fassent  ce  qu'ils  désirent  (2). 
Psychologues et sociologues s'attachent à en reconnaître les formes (3), à en déceler les origines, à 
le  décrire  comme  la  résultante  d'un  champ  de  forces,  à  établir  les  rapports  entre  le  prestige 
attribué  à  autrui  et  à  soi-même.  Ce  qui  nous  intéresse  dans  ces  travaux,  c'est  que  la  plupart  des 
éléments d'analyse que l'on introduit sont aussi les facteurs qui, dans l'argumentation, permettent 
de  défendre  le  prestige,  de  l'expliquer,  de  le  valoriser.  La  description  sociologique rejoint  le  plus 
souvent  la  pratique  argumentative.  Si,  dans  certains  cas,  on  postule  ou  croit  observer  une 
discordance  entre  raisons  alléguées  et  origine  réelle  du  prestige,  c'est  en  fonction  des  premières 
que  se  fait,  auprès  des  membres  d'un  groupe  concret,  toute investigation  relative  aux  critères  du 
prestige, lesquels diffèrent de groupe à groupe. » 
\bigskip
\bigskip
\bigskip
\bigskip
212 
\bigskip
(1) E. Dupréel, sociologie générale, P. 66. 
(2) R. Lippitt, N. Polansky et S. Rosen, The dynamics of power, Human relations, vol. V, n. 1, 1952. 
(3) Cf. notamment B. Srokvis, Psychologie der suggestie en autosuggestie pp. 36 et suiv. 
\bigskip
P  408 :  « Cependant,  sauf  si  le  prestige  est  mis  en  doute,  on  n'a  pas  coutume  de  le  justifier.  Il 
s'exerce en bien comme en mal : 
\bigskip
L'exemple des Grands, dit Gracian, est si bon Rhétoricien, qu'il persuade jusqu'aux choses les plus 
infâmes (4). 
\bigskip
Par contre, une personne peut être mal famée au point que tout ce qu'elle dit et tout ce qu'elle fait 
en devient marqué d'un signe négatif, est dévalué par sa solidarité avec la personne. » 
\bigskip
(4) B. Gracian, L'homme de cour, p. 217. 
\bigskip
P  408-409 :  « Ce  phénomène,  si  caractéristique  de  la  psychologie  sociale,  explique  ce  qui,  au 
premier abord, aurait pu paraître étrange, et que nous appellerons la polarisation des vertus et des 
vices. Voici comment Méré la décrit : 
\bigskip
Ne voyons-nous pas que le mérite nous semble de plus grand prix en un beau corps, qu en un corps 
mal  fait  ?  comme  aussi  quand  le  mérite  est  bien  reconnu  nous  en  trouvons  la  personne  plus 
aimable. La mesme chose arrive de ce qui ne tombe que sous les sens;  lorsqu'on est satisfait du 
visage, le son de la voix en paroist plus agréable (1). » 
\bigskip
(1) Chev. De Méré, Œuvres t. II : Des agrémens, p. 20. 
\bigskip
P 409 : « Les personnages des romans populaires, tout blancs, ou tout noirs, ne font qu'exagérer 
une  tendance  spontanée  de  l'esprit,  propice  à  écarter  certains  scrupules  dans  l'action.  Cette 
polarisation des vertus et des vices peut s'étendre aux aspects sociaux de la personne : le mérite se 
lie à la situation sociale privilégiée, tout se divise en camps opposés. Comme l'écrit Walter White : 
\bigskip
J'étais un nègre, je faisais partie de ce qui, dans l'histoire, s'oppose au bien, à ce qui est juste, a la 
lumière (2). 
\bigskip
La technique argumentative se sert de ces liaisons. Le panégyrique unifie, dans un éloge commun, 
tous les aspects de la personne, qui sont valorisés les uns par les autres. 
\bigskip
Mais ces techniques basées sur la solidarité, sont assez pauvres si l'on ne les envisage pas comme 
une  interaction  continue  de  l'acte  et de  la  personne.  C'est  cette  dernière qui  produit un  véritable 
effet de boule de neige. Ainsi, l'argumentation par le sacrifice (3) gagnera en force grâce au prestige 
accru  de  ceux  qui  se  sont  sacrifiés  :  le  sang  des  martyrs  atteste  d'autant  mieux  la  valeur  de  la 
religion à laquelle il a été sacrifié que les confesseurs de la foi jouissent d'un plus grand prestige 
préalable, mais celui-ci ne pourra que grandir suite à leur immolation. » 
\bigskip
(2) W. White, Deux races se rencontrent en moi, Echo, juin 1948, p. 417. 
(3) Cf. § 58 : L'argumentation par le sacrifice. 
\bigskip
P 409-410 : « L'effet boule de neige se marque à l'extrême lorsque toute l'idée que l'on se fait de la 
personne  dérive  de  certains  actes  et  réagit  néanmoins  sur  l'opinion  que  l'on  se  fait  de  ceux-ci. 
Ainsi,  dans  la  question  des  faux  autographes  présentés  par  M.  Chasles  à  l'Académie,  chaque 
objection des adversaires, une fois surmontée, incite Chasles à augmenter sa confiance en celui qui 
lui fournit les documents ; tandis que cette confiance augmente la valeur de ceux-ci. D'autre part, 
le  faussaire,  qui  paraît  à  Chasles  impossible  à  imaginer,  acquiert  cependant  à  ses  yeux  des 
\bigskip
\bigskip
\bigskip
213 
\bigskip
capacités telles que, lorsque des chiffres empruntés à la troisième édition des Principes de Newton 
apparaissent dans une soi-disant lettre de Pascal, Chasles affirme que 
\bigskip
le  faussaire  prétendu  aurait  été  trop  intelligent  pour  commettre  la  faute  de  copier  sur  la 
troisième édition des Principes (1). » 
\bigskip
(1) Vayson De Pradenne, Les fraudes en archéologie préhistorique, pp. 398-399.  
\bigskip
P 410 : « Ce cas extrême d'interaction, abolissant tout sens critique, n'est possible que parce que 
des interprétations des documents, tantôt comme vrais, tantôt comme faux, réagissent toutes deux 
pour accroître la confiance en ceux-ci, par l'intermédiaire d'une conception de la personne, basée 
uniquement sur ces documents. » 
\bigskip
§ 70. L'ARGUMENT D'AUTORITE 
\bigskip
Beaucoup d'arguments sont influencés par le prestige; c'est le cas, nous l'avons vu, de l'argument 
par le sacrifice. Mais il existe une série d'arguments, dont toute la portée est conditionnée par le 
prestige.  La  parole  d'honneur,  donnée  par  quelqu'un  comme  unique  preuve  d'une  assertion, 
dépendra de l'opinion que l'on a de lui comme homme d'honneur; le respect qu'inspire l'intégrité 
de  Brutus  est  le  principal  fondement  de  son  argumentation  dans  le  jules  César  de  Shakespeare 
(2). » 
\bigskip
(2) Shakespeare, Julius Caesar, acte III, sc. II. 
\bigskip
P 410-411 : « La Rhétorique à Herennius relève, comme exemple d'argumentation faible, basée sur 
ce qu'on va faire et non sur ce qu'il convient de faire, ces phrases mises par Plaute dans la bouche 
du vieux radoteur Megaronides : 
\bigskip
C'est  une  chose  désagréable  de  reprendre  un  ami  pour  une  faute,  mais  c'est  quelquefois  utile  et 
agréable : car moi-même je reprendrai aujourd'hui mon ami pour celle qu'il a commise (1). » 
\bigskip
(1)  Rhétorique  à  Herennius,  liv.  11,  §  35;  Cf.  PLAUTE,  Trinummus,  acte  I.  sc.  1,  vv.  23-27  ;  cité 
également dans Cicéron, De Inventione, liv. I, § 95. 
\bigskip
P 411 : « Si l'argumentation est faible, et même comique, ce n'est point en raison du schème qui la 
sous-tend,  mais  parce  que  c'est  une  argumentation  par  le  modèle,  employée  en  dehors  de  ses 
conditions d'application, alors que le modèle manque totalement de prestige (2). 
\bigskip
L'argument  de  prestige  le  plus  nettement  caractérisé  est  l'argument  d'autorité,  lequel  utilise  des 
actes ou des jugements d'une personne ou d'un groupe de personnes comme moyen de preuve en 
faveur d'une thèse. 
\bigskip
L'argument  d'autorité  est  le  mode  de  raisonnement  rhétorique  qui  fut  le  plus  vivement  attaqué 
parce que, dans les milieux hostiles à la libre recherche scientifique, il fut le plus largement utilisé 
et cela d'une manière abusive, péremptoire, c'est-à-dire en lui accordant une valeur contraignante, 
comme si les autorités invoquées avaient été infaillibles : 
\bigskip
Quiconque, dit Locke, soutient ses prétentions au moyen de telles autorités, croit qu'il doit, par là, 
l'emporter, et est prêt à qualifier d'impudent toute personne qui oserait s'opposer à elles. C'est là, 
je pense, ce que l'on peut appeler argumentum ad verecundiam (3). » 
\bigskip
(2) Cf §' 80 : Le modèle et l'antimodèle. 
(3) Locke, An Essay concerning human understanding, p. 581 (liv. IV, chap. XVII, § 19). 
\bigskip
\bigskip
\bigskip
\bigskip
214 
\bigskip
P  411-412 :  « Certains  penseurs  positivistes  ont  attaqué  cet  argument  -  dont  ils  reconnaissent 
l'énorme  importance  dans  la  pratique  en  le  traitant  de  frauduleux,  tel  Pareto,  pour  qui  cet 
argument  serait  à  considérer  «comme  un  moyen  de  donner  un  vernis  logique  aux  actions 
non-logiques  et  aux  sentiments  dont  elles  tirent  leur  origine  »  (1).  Ce  serait  donc  un 
pseudo-argument  destiné  à  camoufler  l'irrationnel  de  nos  croyances,  en  les  faisant  soutenir  par 
l'autorité de personnes éminentes, le consentement de tous ou du plus grand nombre. » 
\bigskip
(1) V. Pareto, Traité de sociologie générale, I. chap. IV, 4 583, p. 312. 
\bigskip
P  412 :  « Pour  nous,  au  contraire,  l'argument  d'autorité  est  d'une  importance  extrême  et,  s'il  est 
toujours,  dans  une  argumentation  particulière,  permis  de  contester  sa  valeur,  on  ne  peut,  sans 
plus, l'écarter comme irrelevant, sauf dans des cas spéciaux que nous aurons l'occasion d'examiner 
dans  le  paragraphe  suivant.  On  a  attaqué  l'argument  d'autorité  au  nom  de  la  vérité.  Et  en  effet, 
dans la mesure où toute proposition est considérée comme vraie  ou fausse, l'argument d'autorité 
ne trouve plus de place légitime dans notre arsenal intellectuel. Mais est-ce toujours le cas, et peut-
on réduire tous les problèmes de droit, par exemple, à des problèmes scientifiques, où il ne s'agit 
que de vérité ? C'est au nom d'une pareille conception que tel auteur traitant de logique juridique 
voit un sophisme dans l'argument d'autorité, auquel il assimile le précédent : 
\bigskip
Un précédent judiciaire exerce une influence inévitable, quoique fâcheuse, sur le juge saisi d'une 
demande... les auteurs doivent garder leur indépendance et chercher la vérité par la logique (2). 
\bigskip
Mais  n'est-ce  pas  une  fâcheuse  illusion  que  de  croire  que  les  juristes  s'occupent  uniquement  de 
vérité,  et  pas  de  justice  ni  de  paix  sociale  ?  Or  la  recherche  de  la  justice,  le  maintien  d'un  ordre 
équitable,  de  la  confiance  sociale,  ne  peuvent  négliger  les  considérations  fondées  sur  l'existence 
d'une tradition juridique, et qui se manifeste aussi bien dans la doctrine que dans la jurisprudence 
:  pour  attester  l'existence  d'une  pareille  tradition,  le  recours  à  l'argument  d'autorité  est 
inévitable. » 
\bigskip
(2) Berriat Saint-Prix, Manuel de logique juridique, pp. 77, 85, 89. 
\bigskip
P 412-413 : « Par contre, c'est quand ce recours semble superflu que naît volontiers le comique de 
l'argument  d'autorité.  Telle  cette  réplique  d'un  enfant  à  sa  grande  soeur  qui  s'enquérait  de  la 
manière dont la princesse Elisabeth savait qu'elle allait avoir un bébé : 
\bigskip
Mais, elle set lire, n'est-ce pas ? Cela se trouvait dans tous les journaux (1). » 
\bigskip
(1) Fun Pare, 1949, p. 21. 
\bigskip
P 413 : « Souvent on semble attaquer l'argument d'autorité, alors que c'est l'autorité invoquée qui 
est  mise  en  question.  Le  même  Pascal  qui  se  moque  de  l'argument  d'autorité,  quand  il  s'agit  de 
l'autorité des «gens de condition » (2), n'hésite pas à invoquer celle de saint Augustin (3) ; Calvin 
récuse celle de l'Église, mais admet celle des prophètes. 
\bigskip
Comme les autorités se contredisent, on peut évidemment, tel Descartes, vouloir toutes les écarter 
au  profit  d'autres  moyens  de  preuve;  le  plus  souvent,  on  se  contente  d'énumérer  les  autorités 
auxquelles  on  peut  se  fier,  ou  d'indiquer  celles  auxquelles  on  accordera  la  préférence  en  cas  de 
conflit (cf. la loi des citations de Théodose). En tout état de cause, celui qui invoque une autorité 
s'engage : il n'est pas d'argument d'autorité qui n'ait de répercussion sur celui qui l'emploie. 
\bigskip
Les  autorités  invoquées  sont  fort  variables  :  tantôt  ce  sera  «  l'avis  unanime  »  ou  «  l'opinion 
commune », tantôt certaines catégories d'hommes, « les savants », « les philosophes », « les Pères 
\bigskip
\bigskip
\bigskip
215 
\bigskip
de l'Église », « les prophètes » ; parfois l'autorité sera impersonnelle : « la physique », « la doctrine 
», « la religion », « la Bible » parfois il s'agira d'autorités nommément désignées. » 
\bigskip
(2) Pascal, Bibl. de la Pléiade, Pensées, 301 (440*), p. 902 (333, éd. Brunschvicg 
(3) Ibid 625 (270*), p. 1032 (812, éd. Brunschvicg); 804 (109). P. 1083 (869 ed. Brunschvicg). 
\bigskip
P  413-414 :  « Le  plus  souvent  l'argument  d'autorité,  au  lien  de  constituer  la  seule  preuve,  vient 
compléter  une  riche  argumentation.  On  constate  alors  qu'une  même  autorité  est  valorisée  on 
dévalorisée  suivant  qu'elle  s'accorde  ou  non  avec  l'opinion  des  orateurs.  A  l'adversaire 
conservateur qui lance avec mépris : « c'est du Condorcet », l'orateur libéral opposera les dires de l' 
«illustre  Condorcet  »  (1).  C'est  suivre,  selon  Pascal,  les  divagations  de  personnes  mal  nées  que 
d'exprimer  des  pensées  méprisables  (2)  :  l'argument  d'autorité  est  ici  invoqué  non  seulement 
négativement mais pour ainsi dire à rebours, et sert autant à qualifier la source des propos qu'à s'y 
référer. » 
\bigskip
(1) 1. Janson, Discours parlementaires, 1, p. 82 (17-19 mai 1879). 
(2) PASCAL, Bibl. de la Pléiade, Pensées, 335 (C. 209-217), pp. 917, 918 (194 éd. Brunschvicg). 
\bigskip
P 414 : « La place de l'argument d'autorité dans l'argumentation est considérable. Mais il ne faut 
pas perdre de vue que, comme tout argument, il s'insère entre d'autres accords. D'une part, si l'on y 
recourt, c'est lorsque l'accord sur ce que l'on exprime risque d'être mis en discussion. D'autre part, 
l'argument  d'autorité  lui-même  peut  être  contesté.  Sur  le  premier  point,  notons  la  tendance  à 
transformer, pour les soutenir, les normes axiologiques en normes thétiques. Sur le second, notons 
que très souvent, l'argument d'autorité ne nous apparaît pas clairement comme tel, parce que nous 
pensons aussitôt à certaines justifications possibles. 
\bigskip
Quand  l'autorité  est  celle  du  grand  nombre,  à  l'argument  d'autorité,  à  proprement  parler,  est 
souvent  sous-jacent  celui  du  normal;  c'est  ainsi  que,  pour  défendre  le  point  de  vue  matérialiste, 
Lefèbvre écrira que : 
\bigskip
Le  matérialisme  met  expressément  à  la  base  de  sa  théorie  de  la  connaissance  cette  conviction 
naïve,  pratique,  de  tous  les  êtres  hu  ains  [que  les  choses  existent  indépendamment  de  notre 
sensation]. 
\bigskip
Il  parlera  par  ailleurs  de  «  l'homme  normal,  qui  n'a  pas  passé  par  un  asile  d'aliénés  ou  par  un 
cercle de philosophes idéalistes » (3). 
\bigskip
L'autorité du grand nombre peut se manifester par la qualification, comme lorsque Plotin nous dit 
: 
\bigskip
De fait ceux qui les possèdent [les vertus civiles], sont réputés divins (4). » 
\bigskip
(3) H. Lefebvre, A la lumière du matérialisme dialectique, I, p. 29. 
(4) Plotin, Ennéades, 1, 2. § 1. 
\bigskip
P  415 :  « Aussi,  toute  appellation  de  «  sage  »,  «  docte  »,  présentée  comme  notoire,  sert-elle  en 
quelque sorte de garantie, par le grand nombre, à une autorité particulière. 
\bigskip
Souvent,  avant  d'invoquer  une  autorité,  on  la  confirme,  on  la  consolide,  on  lui  donne  le  sérieux 
d'un témoin valable. En effet, plus l'autorité est importante, plus son propos paraît indiscutable. A 
la limite, l'autorité divine surmonte tous les obstacles que la raison pourr t lui opposer : 
\bigskip
\bigskip
\bigskip
\bigskip
216 
\bigskip
... Un maître [jésus] en qui, il paraît tant d'autorité, quoique sa doctrine soit obscure, mérite bien 
qu'on  l'en  croie  sur  sa  parole:  ipsum  audite.  ...  Vous  pouvez  reconnaître  son  autorité  en 
considérant les respects que lui rendent Moïse et Elie; c'est-à-dire, la loi et les prophètes, comme 
je l'ai expliqué. ... Ne recherchons pas les raisons des vérités qu'il nous enseigne : toute la raison, 
c'est qu'il a parlé (1). 
\bigskip
La  conclusion  fournit  l'argument  d'autorité  sous  son  aspect  péremptoire  et  absolu.  Remarquons 
néanmoins  que  cette  autorité  est  encore  attestée  par  les  respects  que  lui  ont  témoignés  d'autres 
autorités, Moïse et Élie. Sa force est révélée par les obstacles dressés sur le chemin de la croyance 
et qu'elle permet néanmoins de surmonter : c'est, sous une autre forme le credo quia absurdum. 
\bigskip
Les  autorités  que  l'on  invoque  sont  le  plus  souvent,  sauf  quand  il  s'agit  d'un  être  absolument 
parfait,  des  autorités  spécifiques  :  leur  autorité  est  reconnue  par  l'auditoire  dans  un  domaine 
particulier, et c'est uniquement dans ce domaine que l'on peut s'en servir. Mais de quelle autorité 
jouissent-elles en dehors de ce domaine ? Quelle est l'influence de l'opinion des experts quand elle 
est opposée à celle du grand nombre ? Dans quels domaines l'une ou l'autre peuventelles prévaloir 
? Ces questions ont fait l'objet, surtout en Amérique, de nombreuses recherches (2). » 
\bigskip
(1) Bossuet, Sermons, vol. il : Sur la soumission due à la parole de Jésus-Christ, pp. 117, 120, 121. 
(2) Cf. Bird, Social Psychology, pp. 284 et suiv. 
\bigskip
P  415-416 :  « Dès  qu'il  y  a  conflit  entre  autorités,  se  pose  le  problème  des  fondements  :  ceux-ci 
devraient permettre de déterminer le crédit que méritent les autorités respectives. Actuellement, le 
fondement le plus souvent allégué en faveur de l'autorité est la compétence, mais ce n'est pas le cas 
dans chaque milieu et à chaque époque. La lutte contre l'argument d'autorité qui, parfois, n'est que 
la lutte contre certaines autorités, mais en faveur d'autres, peut, par ailleurs, résulter du fait que 
l'on  désire  remplacer  le  fondement  traditionnel  de  l'autorité  par  un  fondement  différent,  ce  qui 
entraînera le plus souvent, par voie de conséquence, un changement d'autorité. » 
\bigskip
P  416 :  « Un  cas  curieux  est  celui  oh  l'argument  d'autorité  accorde  une  valeur  argumentative 
indéniable  à  des  affirmations  qui  font  état  d'une  ignorance  ou  d'une  incompréhension.  Quand  le 
maître dit à son élève : « je ne comprends pas ce que vous dites », cela signifie d'habitude « vous 
vous  êtes  mal  exprimé  »,  ou  «  vos  idées  ne  sont  pas  très  claires  sur  ce  point  ».  La  feinte 
incompétence, l'ignorance affectée, ont été dénoncées par Schopenhauer (1), par Bentham (2). On 
en trouve de jolis exemples chez Marcel Proust (3). 
\bigskip
L'incompétence du compétent peut servir de critère pour disqualifier tous ceux que l'on n'a aucune 
raison  de  croire  plus  compétents  que  celui  qui  s'est  avoué  incompétent.  Cette  forme 
d'argumentation peut avoir une portée philosophique éminente, car elle peut viser à détruire non 
seulement la compétence, en telle matière, d'un individu ou d'un groupe, mais de l'humanité tout 
entière. Lorsqu'on dénonce, chez des penseurs éminents, les déficiences de la raison, c'est souvent 
pour  bien  assurer  les  déficiences  de  la  raison  en  général,  et  seule  l'autorité  dont  ils  jouissent 
permet une pareille extrapolation. » 
\bigskip
(1) Schopenhauer, éd. Piper, vol. 6: Eristische Dialektik, p. 423, Kunstgriff 31.  
(2) Bentham, Œuvres t. I : Traité des sophismes politiques, pp. 458-459. 
(3) M. Proust, A la recherche du temps perdu, vol. 7 : Le côté de Guermantes, II, p. 73. 
\bigskip
P  416-417 :  « Cependant,  il  n'est  pas  exclu  que  ce  soient  réellement  certaines  déficiences, 
particulières à la personne, qui augmentent son autorité. On peut mettre en parallèle l'argument 
basé sur la compétence (l'avis d'un expert) et celui basé sur l'innocence (le témoignage d'un enfant, 
d'un  homme  ivre)  (1).  Lors  d'un  accident,  l'avis  de  l'expert  et  celui  de  l'enfant  peuvent  être 
invoqués  conjointement  ;  dans  les  deux  cas,  l'opinion  est  valorisée  par  les  caractères  de  la 
personne, qui sont tout différents de ceux d'un témoin quelconque. » 
\bigskip
(1) Cf. Cicéron, Topiques, § 75. 
\bigskip
P 417 : « Quant aux fondements de la compétence - car elle aussi pourra devoir être justifiée  - ils 
seront  fort  divers  ;  on  les  cherchera  dans  des  règles  de  conditionnement,  d'acquisition  des 
aptitudes,  dans  des  règles  de  vérification  des  aptitudes,  dans  des  règles  de  confirmation  de  la 
compétence. 
\bigskip
Qui  est  compétent  pour  juger,  pour  prendre  une  décision  ?  Comme  le  désaccord  sur  les 
compétences conduit souvent à laisser la question en suspens, un ordre juridique qui se préoccupe 
d'éviter les dénis de justice, devra décider quels sont, en cas de conflit, les magistrats compétents 
qui auront l'autorité pour juger et trancher le débat. » 
\bigskip
§  71.  LES  TECHNIQUES  DE  RUPTURE  ET  DE  FREINAGE  OPPOSEES  A 
\bigskip
L'INTERACTION ACTE-PERSONNE 
\bigskip
Les techniques qui rompent, ou qui freinent, l'interaction de l'acte et de la personne, doivent être 
mises en branle lorsqu'il existe une incompatibilité entre ce que nous croyons de la personne et ce 
que  nous  pensons  de  l'acte,  et  que  nous  nous  refusons  à  opérer  les  modifications  qui 
s'imposeraient, parce que nous voulons garder soit la personne à l'abri de l'influence de l'acte, soit 
celui-ci  à  l'abri  de  l'influence  de  la  personne.  Ceci  veut  dire  que  les  techniques  que  nous  allons 
exposer  ont  pour  effet  de  transformer  l'interaction,  en  action  qui  va  dans  un  sens  et  pas  dans 
l'autre. » 
\bigskip
P  418 :  « La  technique  la  plus  efficace  pour  empêcher  la  réaction  de  l'acte  sur  l'agent  est  de 
considérer  celui-ci  comme  un  être  parfait,  en  bien  ou  en  mal,  comme  un  dieu  ou  un  démon.  La 
technique  la  plus  efficace  pour  empêcher  la  réaction  de  l'agent  sur  l'acte  est  de  considérer  ce 
dernier  comme  une  vérité  ou  l'expression  d'un  fait.  Nous  qualifierons  ces  deux  procédés  de 
techniques de rupture. 
\bigskip
Dès qu'une personne, un agent, est considéré comme un être parfait, divin, l'idée que l'on se fait de 
ses  actes  va  évidemment  bénéficier  de  l'opinion  que  l'on  a  de  l'agent,  mais  l'inverse  ne  sera  plus 
vrai. Leibniz nous fournit une explication de ce processus, qu'il considère comme conforme à une « 
bonne logique des vraisemblances » (1), en imaginant : 
\bigskip
... qu'il y ait quelque chose de semblable parmy les homines à ce cas qui a lieu en Dieu. Un homme 
pourroit donner de si grandes et de si fortes preuves de sa vertu et de sa sainteté, que toutes les 
raisons  les  plus  apparentes  que  l'on  pourroit  faire  valoir  contre  luy  pour  le  charger  d'un 
prétendu  crime,  par  exemple,  d'un  larcin,  d'un  assassinat,  meriteroient  d'être  rejettées  comme 
des  calomnies  de  quelques  faux  temoins  ou  comme  un  jeu  extraordinaire  du  hasard,  qui  fait 
soubçonner  quelquesfois  les  plus  innocens.  De  sorte  que  clans  un  cas  où  tout  autre  seroit  en 
danger d'être condamné, ou d'être mis à la question (selon les droits des lieux), cet homme seroit 
absous par ses juges d'une commune voix (2). 
\bigskip
Cette  justification,  qu'il  considère  comme  rationnelle,  de  la  technique  consistant  à  refuser  tout 
effet défavorable de l'acte sur l'agent, Leibniz l'a exposée à l'aide d'un exemple humain, mais il va 
de soi que c'est quand on l'applique à Dieu que ce procédé devient inattaquable : 
\bigskip
J'ay déjà remarqué, que ce qu'on peut opposer à la bonté et à la justice de Dieu, ne sont que des 
apparen  ces,  qui  seroient  fortes  contre  un  homme,  mais  qui  deviennent  nulles,  quand  on  les 
\bigskip
\bigskip
\bigskip
218 
\bigskip
applique à Dieu, et quand on les met en balance avec les demonstrations qui nous assurent de la 
perfection infinie de ses attributs (3). 
\bigskip
(1) Leibniz, éd. Gerhardt, vol. 6 : Essais de Théodicée, p. 71. 
(2) Leibniz, ibid., pp. 70-71. 
(3) Leibniz, ibid., p. 74. 
\bigskip
P 419 : «Ce que l'on pourrait opposer à Dieu n'est ni vrai ni réel : ce qui peut être considéré comme 
incompatible avec la perfection divine est, par le fait même, disqualifié et traité d'apparence. 
\bigskip
Cette  indépendance  de  la  personne  par  rapport  à  l'acte,  nous  la  retrouvons  également  quand  il 
s'agit de démons : 
\bigskip
Cependant reconnaissons, chrétiens, que ni les sciences, ni le grand esprit, ni les autres dons de 
nature, ne sont des avantages fort considérables, puisque Dieu les laisse entiers aux diables, ses 
capitaux ennemis... (1). 
\bigskip
Au lieu de valoriser la personne, ces qualités reconnues sont dévaluées et minimisées par le fait de 
constituer  des  attributs  diaboliques :  l'interaction  acte-personne  cesse  ;  la  nature  de  la  personne 
est seule à influencer notre opinion sur la valeur de l'acte. 
\bigskip
Quand la qualité de la personne ne semble pas suffisante pour la mettre à l'abri de l'interaction, le 
recours à ce même type d'argument peut paraÎtre comique ou blasphématoire, telle cette réflexion, 
à propos de sainte Marie l'Égyptienne : 
\bigskip
Il faut être aussi sainte qu'elle pour en faire autant sans pécher (2). » 
\bigskip
(1) Bossuet, Sermons, Vol. II: Premier sermon sur les démons, p. 11. 
(2) A. France, La rôtisserie de la reine Pédauque, p. 45 (communiqué par R. Schaerer). 
\bigskip
P  419-420 :  « La  technique  de  rupture  opposée  donne  le  primat  à  l'acte,  qui  ne  dépend  plus  de 
l'opinion que l'on a de la personne : cette indépendance résulte de ce que l'acte exprime un fait ou 
énonce  une  vérité.  Le  prestige  d'aucune  personne  (l'Être  parfait  excepté)  ne  pourrait  nous  faire 
admettre  que  2 +  2 =  5,  ni  obtenir  notre  adhésion  à  un  témoignage  contraire  à  l'expérience.  Par 
contre, « une erreur de fait jette un homme sage dans le ridicule » (3) et l'on risque de perdre tout 
son prestige en soutenant ce qui est considéré comme contraire aux lois de la nature. Témoin cette 
mésaventure arrivée à l'ambassadeur hollandais qui : 
\bigskip
... entretenant le roi de Siam des curiosités de la Hollande, lui dit, entre  autres « que l'eau, dans 
son pays devenait parfois si dure, par temps froid, que les hommes pouvaient se promener à sa 
surface, et qu'elle supporterait le poids d'un éléphant s'il y en avait là ». A cela le roi répondit : « 
jusqu'à présent j'ai cru les choses étranges que vous m'avez racontées, parce que je vous prends 
pour un homme sérieux et honnête, mais maintenant je suis sùr que vous mentez (1). » 
\bigskip
(3) La Bruyère, Bibl. de la Pléiade, Caractères, Des jugements, 47, p. 379. 
(1) Locke, An Essay concerning human understandinq, liv. IV, chap. XV, § 5, p. 557. 
\bigskip
P  420 :  « Dans  ce  récit,  l'expérience,  et  les  généralisations  qu'elle  semble  autoriser,  sont 
considérées comme un fait, qui prime toute influence de la personne. L'acte de celle-ci, parce que 
jugé incompatible avec les convictions issues de l'expérience, est traité de mensonge, déconsidère 
son auteur, et porte atteinte au crédit accordé à tous ses témoignages antérieurs. 
\bigskip
\bigskip
\bigskip
\bigskip
219 
\bigskip
Un  fait  est  ce  qui  s'impose  à  tous  ;  aucune  autorité  ne  peut  rien  sur  lui.  C'est  donc  ébranler  ce 
statut de fait que de rendre quelque chose, qui devrait être indépendant de la personne, dépendant 
de la qualité de celui qui en témoigne. Rappelons, encore une fois, la célèbre anecdote du magicien, 
favori  d'un  roi  auquel  il  fit  don  de  vêtements,  que  ne  voyaient,  disait-il,  que  les  hommes 
moralement irréprochables. Ni le roi ni les courtisans n'osent avouer qu'ils ne voient rien, jusqu'au 
moment où un enfant, dans son innocence s'écrie : « Pourquoi le roi court-il tout nu ? » Le charme 
était  rompu.  Le  prestige  du  magicien  était  suffisant  pour  attribuer  à  la  perception  la  valeur  d'un 
critère de moralité, jusqu'au moment où l'innocence incontestable de l'enfant détruisit le crédit du 
magicien. » 
\bigskip
P 420-421 : « S'il est indéniable que les faits et les vérités échappent, aussi longtemps qu'ils sont 
reconnus  comme  tels,  au  domaine  de  l'argumentation  -  et  c'est  ce  qu'il  y  a  de  fondé  dans 
l'opposition  établie  par  Pareto  entre  le  domaine  logico-experimental  et  celui  de  l'autorité  (2)  - 
quand peut-on dire que l'on se trouve en  présence  d'un fait ou d'une vérité ? C'est le cas, avons-
nous  vu,  aussi  longtemps que  l'énoncé  est  considéré  comme  valable  pour  un  auditoire universel. 
Pour éviter toute discussion à ce propos, on l'encadrera dans une discipline dont on suppose admis 
les fondements, dont les critères peuvent faire l'objet d'un accord, explicite ou implicite, de portée 
universelle.  Dans  ce  cas,  et  dans  ce  cas  seulement,  la  validité  du  fait  échappe  à  tout  argument 
d'autorité : 
\bigskip
Au  point  de  vue  logico-expérimental,  la  vérité  de  la  proposition :  A  est  B,  est  indépendante  des 
qualités morales de l'homme qui l'énonce. Supposons que demain on découvre qu'Euclide fut un 
assassin, un voleur, en somme le pire homme qui ait jamais existé ; cela porterait-il le moindre 
préjudice à la valeur des démonstrations de sa géométrie ? (1). » 
\bigskip
(2) Cf. § 70 : L'argument d'autorité. 
(1) V. Pareto, Traité de sociologie générale, II § 1444, p. 817.  
\bigskip
P  421 :  « Mais  est-il  permis  d'étendre  l'exemple  de  la  géométrie  à  tous  les  domaines,  comme 
l'insinue Pareto : 
\bigskip
«  Une  certaine  proposition  A  ne  peut  être  bonne  que  si  elle  est  faite  par  un  homme  honnête;  je 
démontre que celui qui fait cette proposition n'est pas honnête, ou qu'il est aye pour la faire; donc 
j'ai démontré que la proposition A est nuisible au pays. » Cela est absurde; et celui qui use de ce 
raisonnement sort entièrement du domaine des choses raisonnables (2). 
\bigskip
Si  Pareto  a  raison  de  critiquer  cette  façon  péremptoire  de  rejeter  une  proposition  à  cause  de  la 
personnalité de celui qui l'avance, il a tort de vouloir entièrement négliger l'action de la personne 
sur  l'acte.  Nous  ne  pouvons  que  nous  ranger  à  l'avis  de  Whately,  à  propos  d'une  remarque 
analogue de Bentham : 
\bigskip
Si la mesure proposée est bonne, dit M. Bentham, deviendra-t-elle mauvaise parce que soutenue 
par un méchant homme ? Si elle est mauvaise, deviendra-t-elle bonne parce que soutenue par un 
homme de bien ? 
\bigskip
A cela, Whately réplique : 
\bigskip
Ce n'est que quand il s'agit de science pure, et encore, en discutant avec les homines de science, 
que  le  caractère  des  conseillers  (ainsi  e  tous  les  autres  arguments  probables)  doit  être 
entièrement laissé de côté (3). 
\bigskip
(2) Ibid., § 1756, pp. 1103-04. 
(3) Whately, Elements of Rhetoric, Part 11, chap. III, § 4, pp. 162, 164, 
\bigskip
\bigskip
\bigskip
220 
\bigskip
 
P 422 : « S'il est vrai que les faits et les vérités échappent à toute influence de la personne, il ne faut 
pas  abuser  de  cette  technique  de  rupture  en  accordant  cette  qualité  éminente  à  des  énoncés  sur 
lesquels non seulement il n'y a pas accord, mais qui, de plus, échappent à tout critère reconnu qui 
permettrait d'établir, à leur propos, l'unanimité qui seule garantirait leur statut de fait ou de vérité. 
\bigskip
Des  techniques  scientifiques  ou  pratiques  visent  à  l'objectivité  en  détachant  l'acte,  soit  pour  le 
décrire soit pour le juger, de l'agent qui l'a posé. Le béhaviorisme en est un exemple; un autre est 
fourni par tous les concours où l'on juge les concurrents sur des performances mesurables ou du 
moins, où l'œuvre est jugée, sans que soit révélé le nom de son auteur. En droit, un grand nombre 
de  dispositions  visent  à  qualifier  des  actes,  sans  tenir  compte  de  la  personne  qui  les  commet,  et 
sans  se  préoccuper  de  son  intention.  Ce  formalisme  est  plus  rare  en  éthique,  mais  la  morale 
japonaise semble néanmoins en fournir certains exemples (1). 
\bigskip
Ces façons de procéder présentent, bien des fois, d'incontestables avantages, dont le principal est 
de  faciliter  l'accord  sur  des  critères;  mais  il  ne  faut  jamais  oublier  qu'il  ne  s'agit  là  que  de 
techniques qui, parfois, se révèlent grosses d'inconvénients auxquels il faut ensuite porter remède. 
La meilleure preuve en est les tentatives récentes, en droit pénal, visant à l'individualisation de la 
peine. 
\bigskip
Les cas où l'action de l'acte sur la personne on de la personne sur l'acte est complètement rompue 
sont relativement rares, dans la pratique argumentative, car ce sont des cas limite. La plupart des 
techniques qui y sont utilisées visent, non pas à supprimer, mais à restreindre cette action : c'est 
pourquoi nous les appellerons des techniques de freinage. » 
\bigskip
(1) R. Benedict, The Chrysanthemum and the Sword, Patterns of Japanese Culture, p. 151. 
\bigskip
P  422-423 :  « L'une  de  ces  techniques  est  le  préjugé  ou,  mieux  peut-être,  la  prévention.  On 
interprète  et  on  juge  l'acte  en  fonction  de  l'agent,  celui-ci  fournissant  le  contexte  qui  permet  de 
mieux comprendre celui-là. Grâce à quoi se maintient une adéquation entre l'acte et la conception 
que nous avions de la personne. Remarquons d'ailleurs que, si le préjugé suffit pour détourner la 
menace  d'une  incompatibilité,  il  n'est  pas  là  même  de  lever  cette  dernière  quand  elle  est  trop 
manifeste. » 
\bigskip
P  423 :  « La  prévention,  le  préjugé,  favorable  ou  défavorable,  ayant  pour  effet,  bien  souvent, 
d'aveugler  sur  la  valeur  de  l'acte,  de  transférer  sur  celui-ci  d'autres  valeurs  venant  de  l'agent,  se 
garder du préjugé serait opérer une rupture salutaire entre l'acte et la personne. Mais, si nous nous 
plaçons  au  point  de  vue  qui  nous  paraît  primordial,  celui  de  la  permanence  de  la  personne,  le 
préjugé  se  présente  comme  une  technique  de  frein,  une  technique  qui  s  1  oppose  aux 
renouvellements  incessants  de  la  conception  que  nous  nous  faisons  d'une  personne,  et  qui 
contribue éminemment à sa stabilité. Tandis que le prestige peut être considéré comme le facteur 
qui assure l'action de la personne sur l'acte, qu'il a un rôle actif, positif, la prévention corrige une 
incompatibilité,  elle intervient  lorsque  la  personne  a  besoin  d'être  abritée. Prestige  et  prévention 
peuvent agir dans le même sens, mais ils jouent à des moments différents de l'argumentation. 
\bigskip
Pour éviter de donner l'impression que l'on juge certains actes en fonction de la personne, que l'on 
est  en  proie  au  préjugé,  il  faudra,  maintes  fois,  avoir  recours  à  des  précautions.  L'une  d'elles 
consiste  à  faire  précéder  un  avis  défavorable  sur  l'acte,  de  certains  éloges  de  la  personne,  et 
inversement.  Ces  éloges  porteront  parfois  sur  d'autres  actes  de  la  même  personne,  mais  visent  à 
louer  celle-ci  et  doivent  témoigner  de  notre  impartialité.  L'éloge  de  l'adversaire  est  donc  le  plus 
souvent autre chose qu'une formule de politesse : il exerce un effet argumentatif. » 
\bigskip
\bigskip
\bigskip
\bigskip
221 
\bigskip
P  423-424 :  « Lorsque,  entre  l'acte  et  l'image  que  l'on  s'est  faite  de  la  personne,  il  y  a  une 
discordance  si  flagrante  que  le  préjugé  ne  peut  arriver,  par  une  interprétation  satisfaisante,  à 
l'abolir, divers procédés peuvent être utilisés pour empêcher néanmoins l'acte d'exercer ses effets 
sur la personne. » 
\bigskip
P 424 : « On pourra établir entre domaines d'activité une séparation telle, que l'acte dépendant de 
certains  d'entre  eux  soit  considéré  comme  irrelevant  pour  l'idée  que  l'on  se  fait  de  la  personne. 
Dans différentes sociétés, et dans différents milieux, la détermination des domaines qui comptent 
ne  se  fera  pas  de  la  même  façon  :  la  constance  dans  le  travail,  la  fidélité  conjugale,  la  piété  ou 
l'irréli gion, par exemple, peuvent dans certains cas, être déterminants pour l'image de la personne 
et, dans d'autres, être relégués dans les domaines négligés. L'étendue de ces domaines inactifs fait 
l'objet d'un accord, le plus souvent tacite, et permet même de caractériser un groupe social. Il va de 
soi  que  le  domaine  des  actes  irrelevants  peut  varier  suivant  les  personnes  :  tels  actes,  sans 
importance dans le chef du Prince, seront jugés essentiels pour l'idée que l'on se fait de personnes 
d'un rang moindre, et inversement; il en sera de même des actes couvrant une certaine période de 
la vie, l'enfance par exemple. Pour Schopenhauer, ce sont les actes mineurs qui doivent déterminer 
notre  image  de  la  personne.  En  effet,  les  autres,  les  actes  surveillés  en  raison  même  de  leur 
retentissement possible, auraient, d'après lui, une valeur représentative beaucoup moindre (1). On 
pourra aussi ne retenir de la diversité des actes qu'un aspect particulier ; parfois on fractionne la 
personne en fragments, sans interaction les uns sur les autres ; parfois on contrecarre l'influence 
de l'acte sur la personne, en figeant celle-ci dans un stade déterminé de son existence, comme ce 
personnage de Jouhandeau qui dit à son client : 
\bigskip
Je suis dans le passé. ... ce n'est que ma momie, monsieur, qui raccommode vos chaussures (2). 
\bigskip
(1) Schopenhauer, éd. Brockhaus, vol. 6: Parerga und Paralipomena, II, Zur Ethik, § 118, p. 245. 
(2) M. Jouhandeau, Un monde, p. 35. 
\bigskip
P  425 :  « A  côté  de  ces  techniques  d'une  portée  générale,  et  dont  nous  sommes  très  loin  d'avoir 
épuisé  l'innombrable  richesse,  il  existe  des  techniques  d'une  portée  plus  restreinte,  qui  ne 
s'appliquent qu'à certains actes déterminés. L'une d'elles est le recours à la notion d'exception. On 
excipera  du  caractère  exceptionnel  de  l'acte  pour  diminuer  son  retentissement  sur  l'image  de  la 
personne. 
\bigskip
Parfois on décrira un acte comme maladroit, inefficace, pour suggérer que la personne  ne s'y est 
pas mise tout entière, de toutes ses forces, avec le meilleur d'elle-même, et que donc il n'en est pas 
une véritable manifestation. 
\bigskip
En  sens  inverse,  pour  que  l'acte  ne  souffre  pas  de  l'image  que  l'on  se  fait  de  la  personne,  on 
prétendra que l'acte n'émane pas d'elle, qu'elle n'est qu'un porte-parole, un témoin : 
\bigskip
Des prédicateurs corrompus, se demande Bossuet (1), peuvent-ils porter la parole de vie éternelle 
? 
\bigskip
Et il répond, en reprenant une comparaison de saint Augustin: 
\bigskip
Le  buisson  porte  un  fruit  qui  ne  lui  appartient  pas,  mais  qui  n'en  est  pas  moins  le  fruit  de  la 
vigne, quoiqu'il soit appuyé sur le buisson... Ne dédaignez pas ce raisin, sous prétexte que vous le 
voyez parmi des épines ; ne rejetez pas cette doctrine, parce qu'elle est environnée de mauvaises 
moeurs : elle ne laisse pas de venir de Dieu... 
\bigskip
\bigskip
\bigskip
\bigskip
222 
\bigskip
Le fait d'attribuer l'acte non à son auteur, mais à la bonne fortune, d'attribuer un jugement à des 
tiers, à un « on » impersonnel, et bien d'autres procédés connus, visent, pour les raisons les plus 
diverses, à diminuer la solidarité entre l'acte et la personne. » 
\bigskip
(1) Bossuet, sermons, vol. II : Sur les vaines excuses des pécheurs, p. 489. 
\bigskip
P  425-426 :  « Toutes  ces  techniques  sont  appliquées,  à  profusion,  dans  les  débats  judiciaires, 
spécialement au pénal. Les traités de rhétorique des Anciens n'omettent presque jamais de signaler 
que le coupable peut, dans la déprécation, avouer le crime, mais implorer la pitié au nom de son 
passé  (2).  On  cherche  à  accroître  la  solidarité  de  la  personne  avec  ses  actes  louables,  et  à  la 
diminuer avec les actes pour lesquels elle est jugée. Le rôle de l'orateur sera de faire admettre une 
image de la personne propre à éveiller la pitié des juges. » 
\bigskip
(2) Cf. Rhétorique à Herennius, liv. I, § 24. 
\bigskip
§ 72. LE DISCOURS COMME ACTE DE L'ORATEUR 
\bigskip
P 426 : « Dans les rapports entre l'acte et la personne, le discours, comme acte de l'orateur, mérite 
une attention particulière, à la fois parce que, pour beaucoup, le discours est la manifestation, par 
excellence,  de  la  personne,  et  parce  que  l'interaction  entre  orateur  et  discours  joue  un  rôle  très 
important dans l'argumentation. Qu'il le veuille ou non, qu'il utilise ou non lui-même des liaisons 
du  type  acte-personne,  l'orateur  risque  d'être  considéré,  par  l'auditeur,  en  liaison  avec  son 
discours.  ©
\bigskip
Il est vrai, comme l'a fait remarquer Pareto, que la moralité d'Euclide n'influence en rien la validité 
de ses démonstrations géométriques, mais si celui qui nous recommande un candidat espère tirer 
de la nomination on de l'élection de ce dernier un avantage personnel appréciable, le poids de sa 
recommandation s'en ressentira inévitablement (1). N'oublions pas, en effet, que la personne est le 
contexte le plus précieux pour apprécier le sens et la portée d'une affirmation, surtout lorsqu'il ne 
s'agit pas d'énoncés intégrés dans un système plus ou moins rigide, pour lesquels la place occupée 
et le rôle joué dans le système fournissent des critères suffisants d'interprétation. » 
\bigskip
(1) Cf. Ch. L. Stevenson, Ethics and Language, p. 128. 
\bigskip
P 427 : « Même les paroles d'autrui, reproduites par l'orateur, changent de signification, car celui 
qui les répète prend toujours à leur égard une position, d'une certaine manière, nouvelle, ne fût-ce 
que  par  le  degré  d'importance  qu'il  leur  accorde.  Cela  est  vrai  des  énoncés  figurant  dans  les 
arguments d'autorité. C'est vrai aussi des mots d'enfant: Lewis Carroll a raison de signaler à un de 
ses amis que (les remarques blasphématoires, qui sont innocentes, faites par des enfants, perdent 
ce  caractère,  répétées  par  des  adultes  (1).  Dans  un  sens  opposé,  une  remarque  injurieuse,  et  qui 
aurait mérité un rappel à l'ordre du député coupable, perd en gravité aux yeux de qui suppose qu'il 
s'agit d'une citation (2). 
\bigskip
Signalons,  à  ce  sujet,  une  intéressante  étude  américaine  (3),  qui  critique  les  procédés 
habituellement  utilisés  en  psychologie  sociale  pour  déterminer  l'influence  du  prestige.  On 
demande d'abord aux sujets dans quelle mesure ils sont d'accord avec une série de jugements ; on 
leur présente, plus tard, les mêmes jugements en fournissant des références quant à leurs auteurs. 
Les résultats obtenus ne prouvent point, comme on le croit généralement que, les sujets modifient 
leur appréciation uniquement en fonction du prestige accordé à l'auteur, tous les autres éléments 
étant restés invariables. En fait, l'énoncé n'est pas le même, quand il émane de tel auteur ou de tel 
autre, il change de signification; il n'y a pas simple transfert de valeurs, mais réinterprétation dans 
\bigskip
\bigskip
\bigskip
223 
\bigskip
un nouveau contexte, fourni par ce que l'on sait de l'auteur présumé. Il en résulte que l'influence 
reconnue  au  prestige,  et  au  pouvoir  de  suggestion  qu'il  exerce,  se  manifeste  d'une  façon  moins 
irrationnelle et moins simpliste qu'on ne l'a cru. » 
\bigskip
(1) L. Carroll, Alice's Adventures in Wonderland, Introduction, P. xi. 
(2) Débat à la Chambre des Communes du 4 octobre 1949, d'après N. Y. Herald Tribune du 5 oct., 
éd. de Paris. 
(3)  S.  E.  Asch,  The  doctrine  of  suggestion,  prestige  and  Imitation  In  social  psychology, 
Psychological  Review,  vol.  55,  pp.  250  à  276.  Cf.  aussi,  du  même  auteur,  Social  psychology,  pp. 
387 à 449. 
\bigskip
P  428 :  « Tenant  compte  des  rapports  qui  existent  entre  l'opinion  que  l'on  a  de  l'orateur  et  la 
manière  dont  on  juge  son  discours,  les  maîtres  anciens  de  rhétorique  en  ont,  depuis  longtemps, 
tiré  des  conseils  pratiques,  recommandant  aux  orateurs  de  donner  une  impression  favorable  de 
leur  personne,  de  s'attirer  l'estime,  la  bienveillance,  la  sympathie  de  leur  auditoire  (1)  ;  leurs 
adversaires  devaient,  par  contre,  s'efforcer  de  les  dévaluer,  en  attaquant  leur  personne  et  leurs 
intentions. 
\bigskip
L'orateur  doit,  en  effet,  inspirer  confiance  :  sans  elle,  son  discours  ne  mérite  pas  créance.  Pour 
réfuter une accusation, Aristote conseille : 
\bigskip
. accuser à notre tour, quiconque nous accuse, car ce serait l'absurdité même que l'accusateur fût 
jugé indigne de confiance et que ses paroles méritassent confiance (2). 
\bigskip
Ceux qui sont présumés indignes de confiance ne sont même pas admis à la  barre des témoins, et 
des règles de la procédure judiciaire, fort explicites, visent à assurer leur exclusion. » 
\bigskip
(1) Cf. § 104 : Ordre du discours et conditionnement de l'auditoire.  
(2) Aristote, Rhétorique, liv. III, chap. XV, § 7, 1416 a. 
\bigskip
P 428-429 : « Aujourd'hui, le conseil de réfuter son adversaire par des attaques  ad Personam, s'il 
peut être suivi dans certains cas bien particuliers - quand il s'agit de disqualifier un témoin dévoyé 
risquerait la plupart du temps de déconsidérer celui qui l'appliquerait. Le prestige de la science et 
de ses méthodes  de vérification a diminué le crédit de toute argumentation qui s'écarte du sujet, 
qui  attaque  l'adversaire  plutôt que  son  point  de  vue;  mais  cette  distinction  ne  joue  que  dans  des 
matières  où  des  critères  reconnus  permettent  de  séparer  le  discours,  de  l'orateur,  grâce  à  des 
techniques de rupture. Dans beaucoup de matières et spécialement quand il s'agit d'édification, la 
personne de l'orateur joue un rôle éminent : 
\bigskip
Un  clerc  mondain  ou  irréligieux,  s'il  monte  en  chaire,  est  déclamateur.  Il  y  a  au  contraire  des 
hommes saints, et dont le seul caractère est efficace pour la persuasion : ils paraissent, et tout un 
peuple qui doit les écouter est déjà ému et comme persuadé par leur présence; le discours qu'ils 
vont prononcer fera le reste (1). » 
\bigskip
P 429 : « Les mêmes paroles produisent un tout autre effet, selon celui qui les prononce : 
\bigskip
Le même langage, dit très justement Quintilien, est souvent libre chez tel orateur, insensé chez tel 
autre, arrogant chez un troisième (2). 
\bigskip
Les  fonctions  exercées,  tout  comme  la  personne  de  l'orateur,  constituent  un  contexte  dont 
l'influence  est  indéniable  :  les  membres  du  jury  apprécieront  de  façon  fort  différente  les  mêmes 
remarques prononcées par le juge, l'avocat ou le procureur. 
\bigskip
\bigskip
\bigskip
\bigskip
224 
\bigskip
Si  la  personne  de  l'orateur  fournit  un  contexte  au  discours,  ce  dernier,  d'autre  part,  détermine 
l'opinion  que  l'on  aura  d'elle.  Ce  que  les  Anciens  appelaient  l'éthos  oratoire  se  résume  à 
l'impression que l'orateur, par ses propos, donne de lui-même (3) : 
\bigskip
Ne prête pas ton appui pour de mauvaises affaires, dit Isocrate, et ne t'en fais pas l'avocat ; tu 
donnerais l'impression de commettre toi aussi les actes dont tu prendrais la défense chez autrui 
(4). » 
\bigskip
(1) La Bruyère, Bibl. de la Pléiade, Caractères, De la chaire, 24, p. 464. 
(2) Quintilien, Vol. IV, liv. XI, chap. 1er, § 37. 
(3) Aristote, Rhétorique, liv. I, chap. 2, 1356 a; liv. II, chap. 21, 1395 b; Topiques, liv. VIII, chap. 9, 
160 b; Cicéron, Partitiones oratoriae, § 22; Quintilien, Vol. II, liv. VI, chap. 11, §§ 8 et suiv.; cf. W. 
Süss, Ethos, Studien zur älteren griechischen Rhetorik. 
(4) Isocrate, Discours, t. I : A Demonicos, § 37. 
\bigskip
P  429-430 :  « Bien  qu'il  soit  souhaitable  que  le  discours  contribue  à  la  bonne  opinion  que 
l'auditoire  peut  se  former  de  l'orateur,  c'est  assez  rarement  qu'il  est  permis  à  ce  dernier,  pour  y 
parvenir, de faire son propre éloge. Les cas où ce procédé est admissible ont été minutieusement 
examinés par Plutarque (5) : ils se résument aux situations oh cet éloge ne constitue qu'un moyen 
indispensable  pour  atteindre  un  but  légitime  (6)  ;  dans  tous  les  cas  où  la  vanité  semble  le 
déterminer,  l'éloge  de  soi-même  produit  un  effet  déplorable  sur  les  auditeurs.  Platon  présentait 
tous  les  sophistes  comme  des  vantards  parce  que,  préoccupé  de  vérité  plus  que  d'adhésion,  il  ne 
voyait pas en quoi le prestige de l'orateur pouvait importer à l'affaire ; mais dès que l'on envisage 
ces  procédés  sous  l'angle  de  l'argumentation,  on  peut  leur  trouver  une  justification  qui  les  rend 
moins déplaisants. » 
\bigskip
(5)  Plutarque,  Œuvres  morales,  t.  II  :  Comment  on  peut  se  louer  soi-même  sans  s’exposer  à 
l'envie. 
(6) Cf. § 64 : Les fins et les moyens. 
\bigskip
P  430 :  « Aujourd'hui,  l'éloge  que  ferait  l'orateur  de  sa  propre  personne  nous  paraîtrait  le  plus 
souvent déplacé et ridicule. Ordinairement le président de la séance assume ce rôle, mais dans la 
plupart  des  cas  l'orateur  est  connu,  soit  parce  qu'il  parle  devant  un  auditoire  familier,  soit  parce 
que  l'on  sait  qui  il  est,  grâce  à  la  presse  et  à  toutes  les  formes  modernes  de  publicité.  La  vie  de 
l'orateur, dans la mesure oil elle est publique, constitue un long préambule à son discours (1). 
\bigskip
A  cause  de  l'interaction  constante  entre  le  jugement  que  l'on  porte  sur  l'orateur  et  celui  que  l'on 
porte  sur  son  discours,  celui  qui  argumente  expose  constamment  quelque  peu  son  prestige,  qui 
s'accroît  ou  décroît  selon  les  effets  de  l'argumentation.  Une  argumentation  honteuse,  faible  ou 
incohérente, ne peut que nuire à l'orateur; la vigueur du raisonnement, la clarté et la noblesse du 
style, disposeront, par contre, en sa faveur. A cause de la solidarité entre le discours et l'orateur, la 
plupart des discussions, spécialement devant témoins, ressemblent quelque peu à un duel, où l'on 
cherche  moins  l'accord  que  la  victoire  :  on  connaît  les  abus  auxquels  a  conduit  l'éristique.  Mais 
rechercher des victoires, ce n'est pas seulement aspiration puérile ou manifestation d'orgueil, c'est 
aussi un moyen pour l'orateur de s'assurer de meilleures conditions pour persuader. » 
\bigskip
(1) Cf. § 104 : Ordre du discours et conditionnement de l'auditoire. 
\bigskip
P 430-431 : « L'orateur aura à coeur de se concilier l'auditoire, soit en montrant sa solidarité avec 
lui,  soit  en  lui  témoignant  son  estime,  soit  en  s'abandonnant  à  son  intégrité.  Une  figure,  la 
permissio,  terme  que  l'on  traduit  souvent  par  concession,  est  illustrée  par  ce  passage  de  la 
Rhétorique à Herennius : 
\bigskip
\bigskip
\bigskip
\bigskip
225 
\bigskip
Puisque tout m'a été enlevé, et qu'il me reste seulement mon âme et mon corps, ces biens mêmes... 
je les remets à vous (1)... » 
\bigskip
(1) Rhétorique à Herennius, liv. IV, § 39. 
\bigskip
P 431 : « On parle de figure, parce que l'orateur tantôt ne peut pas se dérober à la sentence, tantôt 
n'a point l'intention de s'y soumettre réellement. 
\bigskip
L'orateur,  ayant  souvent  à  assumer  le  rôle  de  mentor,  de  celui  qui  conseille,  réprimande,  dirige, 
doit veiller à ne pas provoquer chez son public un sentiment d'infériorité et d'hostilité à son égard : 
il faut que l'auditoire ait l'impression de décider en pleine liberté. Jouhandeau, dans de fort belles 
pages,  explique  la  discrétion  divine  par  le  respect  de  Dieu  pour  le  moi  humain  :  malgré  sa 
puissance,  Dieu  renoncerait  à  tout  ce  qui  peut  sembler  une  atteinte  à  notre  indépendance  de 
jugement, au point de vouloir paraître absent (2). 
\bigskip
Toutes  les  techniques  favorisant  la  communion  de  l'orateur  avec  l'auditoire  atténueront 
l'opposition  entre  eux,  laquelle  est  néfaste  quand  le  rôle  de  l'orateur  est  de  persuader.  Le 
cérémonial,  technique  de  distinction,  qui  rehausse  l'éclat  de  l'orateur,  pourra  lui-même  être 
favorable  à  la  persuasion,  si  les  auditeurs  le  considèrent  comme  un  rituel  auquel,  eux  aussi, 
participent. 
\bigskip
Lorsqu'il  s'agit  de  communiquer  des  faits,  la  personne  de  l'orateur  semble  beaucoup  moins 
engagée  que  lorsqu'il  s'agit  d'émettre  des  appréciations.  Mais,  même  dans  ce  cas,  l'attitude  de 
l'orateur  peut  manifester  son  estime  envers  le  publie  :  prudence,  restrictions,  -refus  de  se 
prononcer sur un point où l'on est pourtant compétent, raccourcis dans l'exposé (3), peuvent être 
autant d'hommages à l'auditoire. » 
\bigskip
(2) M. Jouhandeau, Essai sur soi-même, p. 146.  
(3) Cf. C. K. Ogden et 1. A. Richards, The meaning of meaning, p. 225. 
\bigskip
P  432 :  « Lorsqu'il  s'agit  d'initiation  à  une  discipline,  le  sentiment  d'infériorité  de  l'auditoire  ne 
joue pas, parce que celui-ci a, au préalable, le désir de s'assimiler cette discipline. Le rôle du maître 
le rapproche, quoi qu'on puisse en penser, beaucoup plus du prêtre que du propagandiste (1). 
\bigskip
Notons,  pour  terminer,  que  la  solidarité  entre  acte  et  personne  existe  aussi  dans  le  chef  de 
l'auditoire.  Nous  savons  déjà  que  la  valeur  des  arguments  sera  évaluée  selon  celle  des  auditoires 
qui leur donnent crédit (2). Inversement, un auditoire peut être loué ou blâmé d'après le genre de 
discours  qui  ont  son  audience,  le  genre  d'orateurs  qu'il  écoute  volontiers,  le  genre  de 
raisonnements  qui  trouvent  son  agrément.  Cette  solidarité  acte-personne,  dans  le  chef  de 
l'auditoire, n'est pas sans retentir sur les effets de l'argumentation. La référence à cette solidarité 
peut  se  superposer  aux  arguments  entendus,  ainsi  qu'à  la  liaison  entre  orateur  et  discours,  et 
interférer avec ces derniers éléments. » 
\bigskip
(1) Cf. 9 12 : Education et propagande. 
(2) f. § 5 Adaptation de l'orateur à l'auditoire 6 : Persuader et convaincre ; cf. aussi, 97 : Interaction 
et  force  des  arguments.  On  trouvera  chez  C.  1.  Hovland,  A.  A.  Lumsdaine  et  F.  D.  Sheffield, 
Experiments  on  Mass  Communication,  pp.  166-168,  190-194,  275-278,  une  tentative  pour 
caractériser et hiérarchiser les opinions d'après les auditoires qui les admettent. 
\bigskip
§ 73. LE GROUPE ET SES MEMBRES 
\bigskip
P 432 : « Il est  permis de considérer que la liaison entre la personne et ses actes, avec toutes les 
argumentations  qu'elle  peut  susciter,  est  le  prototype  d'une  série  de  liens  qui  donnent  lieu  aux 
mêmes interactions et se prêtent aux mêmes argumentations. Le plus banal, peut-être, de ceux-ci, 
\bigskip
\bigskip
\bigskip
226 
\bigskip
est le rapport établi entre un groupe et ses membres, ces derniers étant la manifestation du groupe, 
tout comme l'acte est l'expression de la personne. » 
\bigskip
P  432-433 :  « Remarquons,  tout  de  suite,  que  nous  ne  faisons  pas  état  ici  d'une  sociologie 
organiciste on à la Durkheim, qui aboutirait à une personnification du groupe, et qui attribuerait à 
ce  dernier  toutes  les  propriétés  de  la  personne.  Ces  théories  ne  sont  que  des  conceptions 
particulières  de  la  relation  à  laquelle  nous  faisons  allusion,  tandis  que  celle-ci  est  implicite  dans 
toute  affirmation  concernant  un  groupe,  désigné  autrement  que  par  l'énumération  de  ses 
membres. » 
\bigskip
P  433 :  « C'est  ainsi  que  nous  pouvons  répéter  ici  ce  que  nous  avons  dit  du  rapport  entre  la 
personne  et  ses  actes:  les individus influent  sur  l'image  que  nous  avons  des  groupes  auxquels  ils 
appartiennent  et,  inversement,  ce  que  nous  croyons  du  groupe  nous  prédispose  à  une  certaine 
image de ceux qui en font partie ; si une académie donne du lustre à ses membres, chacun de ceux-
ci contribue à représenter et à illustrer l'académie. 
\bigskip
La valeur d'un individu rejaillit sur le groupe, une déficience individuelle peut, dans certains cas, 
compromettre  la  réputation  du  groupe  tout  entier,  d'autant  plus  aisément  que  l'on  se  refuse  à 
utiliser des techniques de rupture. 
\bigskip
Jouhandeau rapporte cette anecdote : 
\bigskip
Elise  a  convoque  un  Marocain  pour  décharger  ses  fagots  et  celui-ci  remarque  un  Français  qui 
doit l'aider, mais l'aide si mal qu'à la fin il s'écrie, aux applaudissements d'Elise Et dire que je suis 
colonisé par « ça » (1). 
\bigskip
Inversement le prestige du groupe peut favoriser la propagation de ses idées, coutumes et modes, 
de  ses  produits  et  procédés  ;  l'on  sait  combien  l'hostilité  qu'on  témoigne  au  groupe  peut,  au 
contraire, constituer pour cette diffusion un sérieux handicap.  
\bigskip
L'argumentation concernant le groupe et ses membres est bien plus complexe que celle concernant 
la personne et ses actes, d'abord parce qu'une même personne  appartient toujours à des groupes 
multiples, mais surtout parce que la notion de groupe est plus indéterminée que celle de personne. 
L'hésitation peut porter sur les frontières du groupe, et aussi sur son existence même. » 
\bigskip
(1) Jouhandeau, Un monde, p. 251. 
\bigskip
P  434 :  « Certains  groupes  -  nationaux,  familiaux,  religieux,  professionnels  -seront  reconnus  par 
tous,  voire garantis  par  les  institutions.  Mais  d'autres  naissent  au gré  du  comportement  de  leurs 
membres:  ainsi,  à  l'école,  à  l'intérieur  de  certaines  classes  d'enfants,  peuvent  se  former  des 
subdivisions fondées sur l'âge, le sexe, la race, la religion, subdivisions plus ou moins calquées sur 
des  catégories  sociales  existantes  ;  une  opposition  peut  aussi  se  produire  entre  les  petits  et  les 
grands, qui formeront deux groupes caractérisés, dont les membres se sentent solidaires. 
\bigskip
Si la réalité du groupe dépend de l'attitude de ses membres, elle dépend autant, et parfois plus, de 
l'attitude des étrangers. Ceux-ci ont tendance à considérer qu'il y a groupe social chaque fois qu'ils 
ont un comportement différencié à l'égard de ses membres, la notion de groupe servant à décrire, à 
expliquer ou à justifier ce comportement différencié, et servant aussi à étayer les arguments dont 
nous nous occupons ici. Ce souci de  l'argumentation explique, notons-le, la tendance à constituer 
en  groupe,  pour  les  rendre  solidaires,  tous  ceux  chez  qui  on  observe  une  même  attitude,  les 
adversaires ou les partisans d'un certain point  de vue, d'une certaine personne  ou d'une certaine 
façon d'agir. Cette prétention ne sera pas toujours admise. Bref, la notion de groupe est un élément 
argumentatif éminemment sujet à controverse, instable, mais d'une importance capitale. » 
\bigskip
\bigskip
\bigskip
227 
\bigskip
 
P  434-435 :  « L'interaction  entre  l'individu  et  le  groupe  peut  être  utilisée  pour  valoriser  ou 
dévaloriser  soit  l'un  soit  l'autre.  On  insistera  sur  les  erreurs  de  certains  archéologues  pour 
disqualifier les spécialistes en cette matière (1). Inversement, si l'on ne peut pas faire son propre 
éloge, on peut se présenter comme adhérent de telle politique ou comme membre de telle Eglise, ce 
qui  est  susceptible  de  constituer  une  forte  recommandation  (2).  Remarquons-le,  c'est  là  une 
application  de  la  technique  très  efficace  qui  consiste  à  faire  passer  des  jugements  d'appréciation 
inexprimés  sous  le  couvert  de  jugements  de  fait  indiscutables  (1).  L'orateur  n'insiste  pas  sur  la 
valorisation implicite par les auditeurs de tous ceux qui appartiennent au groupe en question : c'est 
dans la mesure où elle semble aller de soi, qu'elle agit le mieux. 
\bigskip
(1) Vayson De Pradenne, Les fraudes en archéologie préhistorique, P. 314.  
(2) Whately, Elements of Rhetoric, l'art. II, chap. 111, § 3, pp. 159-160. 
(1) Cf. § 43 : Le statut des éléments d'argumentation et leur présentation. 
\bigskip
P  435 :  « L'appartenance  à  un  groupe  donné,  peut,  en  effet,  faire  préjuger  de  l'existence  de 
certaines qualités dans le chef de ses membres, et cette présomption est d'autant plus forte que le 
sentiment  de  classe  ou  de  caste  est  plus  prononcé.  C'est  ainsi  que  Racine  prend  soin  de  rendre 
Phèdre un peu moins odieuse que dans la tragédie grecque, à cause du rang qu'elle occupe : 
\bigskip
J'ai  cru,  écrit-il  dans  sa  préface  à  Phèdre,  que  la  calomnie  avait  quelque  chose  de  trop  bas  et  de 
trop noir pour la mettre dans la boue , d'une princesse qui a d'ailleurs des sentiments si nobles et si 
vertueux.  Cette  bassesse  m'a  paru  plus  convenable  à  une  nourrice  qui  pouvait  avoir  des 
inclinations plus serviles... (2). 
\bigskip
Certaines façons de se conduire sont conformes à l'idée que l'on se fait des membres d'un groupe : 
le  comportement  des  nobles  est  noble,  celui  des  vilains  est  vilain,  celui  des  chrétiens,  chrétien, 
celui  des  hommes,  humain;  le  comportement  est  souvent  décrit  par  la  dénomination  même  du 
groupe; il réagit d'autre part sur l'image que l'on se fait de celui-ci. » 
\bigskip
(2) Racine, Bibliothèque de la Pléiade, I, Préface à Phèdre, p. 763. 
\bigskip
P  435-436 :  « La  valeur  de  l'acte  dépend,  nous  le  savons,  du  prestige  de  l'individu;  la  valeur  de 
l'individu dépend de celle que l'on attribue au groupe; personne et groupe jouent par rapport aux 
actes  et  aux  individus  un  rôle  analogue,  qui  peut  se  conjuguer.  Le  groupe  s'enorgueillira  de  la 
conduite  de  ceux  qu'il  considère  comme  ses  membres,  négligera  souvent  de  s'occuper  des 
étrangers : 
\bigskip
Les  exemples  des  morts  généreuses  de  Lacédémoniens  et  autres  ne  nous  touchent  guère.  Car 
qu'est-ce que cela nous apporte ? Mais l'exemple de la mort des martyrs nous touche; car ce sont 
« nos membres » [Rom., XII, 5] (1). 
\bigskip
P 436 : « Pour la liaison individu-groupe, les techniques de rupture semblent moins élaborées que 
pour  la  liaison  acte-personne,  en  ce  sens  que  nous  ne  rencontrons  pas  de  cas  limite  où  toute 
réaction est suspendue, tels le cas de l'Être parfait ou du jugement considéré comme un fait. N'est 
groupe  parfait  au  sens  requis  ici,  ni  la  société  des  dieux  antiques,  ni  la  société  chrétienne,  ni  la 
famille princière. Ce qui se rapproche le plus de la notion de groupe parfait, c'est la notion d'une 
humanité  qui  n'aurait  pour  caractère  que  ce  qui  est  commun  à  tous  les  hommes  et  ne  serait 
influencée par le comportement de quelque nombre d'hommes que ce soit. 
\bigskip
D'autre part l'individu raisonnable, celui qui obéit seulement à l'ordre universel, ne serait-il point 
détaché de tout groupe (2), son comportement n'aurait-il pas une objectivité qui correspond à celle 
du fait ? Mais l'accord sur l'ordre universel est loin d'être, à aucun moment, assuré. » 
\bigskip
\bigskip
\bigskip
228 
\bigskip
 
(1) Pascal, Bibl. de la Pléiade, Pensées, 714 (161), p. 1053 p. 181 éd. Brunschvicg). 
(2) E. Dupréel, Essais pluralistes, pp. 71-72 (De la nécessité). 
\bigskip
P  436-437 :  « Aussi  la  seule  technique  qui  permette  de  réaliser  une  rupture  d'interaction  entre 
groupe  et  individu  est  l'exclusion  de  celui-ci  :  elle  pourra  être  appliquée  soit  par  l'individu  lui-
même, soit par les autres membres du groupe, soit par des tiers. Si quelqu'un exprime une opinion 
violemment  opposée  à  celle  des  autres  membres  du groupe,  et  que  l'on  se refuse  à  admettre  que 
cette  opinion  puisse  être  portée  sur  le  compte  du  groupe,  une  rupture  s'imposera  :  on  verra  une 
incompatibilité entre l'adhésion à certaine thèse et l'appartenance à un certain groupe. Celui qui ne 
partage plus les opinions du groupe, tout en manifestant clairement qu'il ne veut pas s'en détacher, 
devra user de dissociations opposant, par exemple, la vraie doctrine à celle de la majorité (3). Mais 
il  va  de  soi  que  la  majorité  peut  ne  pas  être  du  même  avis,  et  procéder  à  l'exclusion  du  membre 
non-conformiste.  Une  telle  procédure  peut  être  appliquée  pour  toute  espèce  d'action  jugée 
incompatible  avec  les  intérêts  ou  l'honneur  du  groupe.  Presque  toujours  l'exclusion  a  pour 
conséquence le rattachement de l'individu à un autre groupe, rattachement qui rend, dans certains 
cas, manifeste, la rupture avec le groupe précédent. » 
\bigskip
(3) Cf. § 90 : Le couple « apparence-réalité ". 
\bigskip
P  437 :  « Il  arrive  que  l'exclusion  soit  recherchée  par  l'individu  luimême  :  dans  ce  cas,  celui  qui 
possède  certains  caractères  extérieurs  servant  couramment  de  critère  pour  reconnaître 
l'appartenance  à  un  groupe,  suscitera  son  exclusion  -  aux  yeux  des  tiers  particulièrement  -  en 
s'opposant aux croyances du groupe, on en adoptant les croyances d'un autre. Il en résulte qu'une 
même critique envers un groupe aura une portée très différente selon qu'elle émane de quelqu'un 
qui reste solidaire du groupe, de quelqu'un qui veut s'en détacher, ou de personnes qui lui sont, en 
tout état de cause, extérieures. 
\bigskip
Notons que le problème du lien individu-groupe, dans l'argumentation, se complique, par rapport 
au problème acte-personne, par le fait de l'inclusion possible d'un individu dans un groupe dont il 
ne faisait pas partie jusqu'à présent. Si l'individu a défend les opinions du groupe B, il pourra être 
intégré,  par  les  tiers,  dans  ce  groupe.  Dès  lors  ses  arguments,  ses  jugements,  seront  interprétés 
comme  étant  ceux  d'un  membre  du  groupe  B,  et  non  d'un  observateur  étranger.  D'où  parfois 
l'intérêt, pour l'argumentation, de maintenir les distances entre l'individu et certains groupes qu'il 
favorise. » 
\bigskip
P 437-438 : « Un groupe qui rejette immédiatement, et quasi automatiquement, tout membre dont 
le  comportement  est  aberrant,  qui  ne  consent  jamais  à  servir  de  caution  à  ses  membres,  se 
rapproche  le  plus  de  la  situation  de  la  personne  parfaite.  Mais  cela  exige une  critique  constante, 
aussi  sévère,  au  moins,  que  celle  des  tiers  ;  et  cela  entraîne,  malgré  tout,  une  modification  du 
groupe, ne fût-ce que dans sa composition. Cette modification peut être perçue comme une simple 
opération mathématique, mais elle le sera, bien plus souvent, comme un remaniement. » 
\bigskip
P  438 :  « Plus  fréquentes  que  les  techniques  de  rupture  sont  les  techniques  de  freinage.  Un  des 
progrès  du  droit  a  consisté  à  remplacer  la  responsabilité  collective  par  la  responsabilité 
individuelle,  en  permettant  de  ne  pas  mettre  au  passif  du  groupe  les  actes  que  la  législation 
condamne et poursuit ; mais ce n'est qu'une technique juridique, que peut répudier un moraliste 
ou un sociologue. 
\bigskip
Les techniques de freinage, d'usage plus étendu, seront le recours au préjugé et à l'exception. Cette 
dernière technique s'utilisera avec d'autant plus de succès que les individus passeront pour moins 
représentatifs  du  groupe  :  si  les  chefs,  les  délégués  ou  porte-parole  officiels  sont  considérés 
souvent comme incarnant le groupe, c'est parce qu'il est plus difficile d'écarter  leurs avis ou leurs 
\bigskip
\bigskip
\bigskip
229 
\bigskip
opinions  comme  exceptionnels.  On  a  souligné  que  Bismarck,  dans  ses  discours  parlementaires, 
combattait les partis dans la personne de leurs chefs (1). 
\bigskip
Parfois  on  prétendra  que  les  affirmations  ridicules  ou  sottes  d'un  individu  ne  peuvent  pas  être, 
sans  sophisme,  attribuées  au  groupe  (2),  ce  qui  revient  à  exiger  de  l'auditeur  qu'il  procède  à  un 
triage, et ne considère point comme représentatif l'individu dont les affirmations sont erronées ou 
insoutenables. 
\bigskip
Une autre technique de freinage destinée à montrer que l'individu ne représente pas le groupe, ne 
s'identifie  avec  aucun  groupe  déterminé,  est  de  le  solidariser  pour  une  part  de  lui-même  avec 
certains d'entre eux, pour une part avec d'autres. Selon Bernanos : 
\bigskip
L'homme de l'Ancien Régime avait la conscience catholique, le coeur et le cerveau monarchistes, et 
le tempérament républicain (3). » 
\bigskip
(1) H. Wunderlich, Die Kunst der Rede in ihren Hauptzügen an den Reden Bismarcks dargestellt, 
p. 85. 
(2) Cf. Bentham, Œuvres, t. I : Traité des sophismes politiques, p. 473. 
(3) G. Bernanos, Scandale de la vérité, p. 27. 
\bigskip
P 439 : « Toutes ces techniques de freinage ne sont pas sans retentir sur les deux composantes de 
la  liaison  individu-groupe.  Le  recours  à  l'exception  ne  tend  pas  seulement  à  freiner  l'action 
qu'exerce le comportement de l'individu sur l'image que l'on se fait du groupe. Il peut aussi avoir 
pour effet de valoriser ou de dévaloriser l'individu, en le présentant comme unique, de provoquer 
exprès un effet de surprise. 
\bigskip
Cette exemption du défaut commun est d'autant plus estimée, que personne ne s'y attend (1). 
\bigskip
Plus le préjugé contre le groupe est défavorable, plus l'exception semble difficile à concevoir, plus 
les membres du groupe qui ne 
\bigskip
désirent  pas  tomber  sous  le  coup  de  la  condamnation  générale,  devront  œuvrer  pour  qu'on  leur 
reconnaisse ce statut exceptionnel. De là ces remarques désabusées d'un noir : 
\bigskip
J'ai souvent entendu ce raisonnement. Ma mère ne m'a-t-elle pas maintes fois répété que c'est déjà 
assez  mal  que je  sois  noir  pour  éviter  de  commettre  la  plus  petite faute  ? Oui,  je sais que  tout  le 
monde, Blanc et Noir, est d'accord sur le fait qu'un Nègre, appelant si peu d'indulgence de par sa 
couleur, n'est tolérable que dans la mesure où il se comporte comme un saint (2). 
\bigskip
( 1) B.Gracian, L'homme de cour, p. 8 (Maxime IX : Dementir les défauts de sa nation.) 
(2) J. Zobel, La rue Cases-Nègres, p. 292). 
\bigskip
§ 74. AUTRES LIAISONS DE COEXISTENCE, L'ACTE ET L'ESSENCE 
\bigskip
P 439-440 : « Les mêmes interactions que nous avons constatées dans les rapports de l'acte et de 
la personne, de l'individu et du groupe, se retrouvent chaque fois que des événements, des objets, 
des  êtres,  des  institutions,  sont  groupés  d'une  façon  compréhensive,  qu'on  les  considère  comme 
caractéristiques  d'une  époque,  d'un  style,  d'un  régime,  d'une  structure.  Ces  constructions 
intellectuelles  s'efforcent  d'associer  et  d'expliquer  des  phénomènes  particuliers,  concrets, 
individuels, en les traitant comme manifestations d'une essence dont d'autres événements, objets, 
êtres ou institutions sont également l'expression. L'histoire, la sociologie, l'esthétique, constituent 
le champ de prédilection des explications de ce type : les événements caractérisent une époque, les 
œuvres  un  style,  les  institutions  un  régime;  même  les  comportements  et  la  manière  d'être  des 
hommes peuvent être expliqués non seulement par leur appartenance à un groupe mais aussi par 
\bigskip
\bigskip
\bigskip
230 
\bigskip
l'époque ou le régime dont ils relèvent : parler de l'homme du moyen âge on d'un comportement 
capitaliste,  c'est  essayer  de  montrer  comment  cet  homme,  ce  comportement,  participent  à  une 
essence et l'expriment, et comment, à leur tour, ils permettent de la caractériser. » 
\bigskip
P 440 : « La notion d'essence, élaborée en philosophie, est néanmoins familière à la pensée du sens 
commun,  et  ses  rapports  avec  tout  ce  qui  l'exprime  sont  conçus  sur  le  modèle  du  rapport  de  la 
personne  avec  ses  actes.  Nous  avons  vu  comment,  à  partir  de  certains  actes  caractéristiques,  on 
arrive  à  qualifier  quelqu'un  de  héros,  à  stabiliser  les  aspects  d'une  personne  (1).  Par  un  procédé 
analogue on arrive, à partir d'un verbe, d'un adjectif ou d'une expression désignant une relation, à 
former  des  essences  («  le  joueur  »,  «  le  patriote  »,  «  la  mère  »),  caractérisant  certaines  classes 
d'êtres dont elles expliquent le comportement. 
\bigskip
Chaque  fois  que,  au  lieu  de  pouvoir  s'interpréter  l'un  par  l'autre,  l'acte  et  l'essence  semblent 
s'opposer, on appliquera des procédés qui permettront de justifier l'incompatibilité : l'homme qui 
n'est  pas  de  son  époque  sera  un  précurseur  ou  un  attardé,  l'œuvre  qui  présente  des  caractères 
différents du style de l'auteur aura été élaborée sous une influence étrangère ou manifestera déjà 
des signes de dégénérescence, ne sera plus une expression aussi pure du style en question; ce qui 
ne  correspond  pas  à  l'image  de  l'essence  sera  exceptionnel,  et  cette  exception  sera  justifiée  par 
l'une ou l'autre des innombrables explications concevables. 
\bigskip
(1) Cf. § 68 : La personne et ses actes. 
\bigskip
P 441 : « Le recours à la notion d'essence permettra de rattacher des  événements variables à une 
structure  stable,  qui  seule  aurait  de  l'importance  :  la  Philosophia  Perennis  en  est  un  exemple 
classique.  Le  recours  à  la  notion  d'essence  peut  aussi  être  implicite  et  servir  à  rendre  compte  de 
certains  changements  :  par  exemple,  les  modifications  de  tarif  douanier  d'un  pays  seront 
considérées comme  le résultat de la volonté  de maintenir une certaine structure économique (1). 
La politique devient la structure économique en acte: les variations de cette politique, expliquées 
par des causes occasionnelles, ne sont que des accidents. 
\bigskip
Notons,  à  ce  propos,  que  ce  qui  correspond  à  l'essence,  en  dehors  des  phénomènes  biologiques, 
peut être déterminé, dans la plupart des cas, avec une liberté qui dépasse celle de la liaison acte-
personne. Mais il va de soi que c'est par rapport à cette essence, quelle que soit la manière de la 
préciser,  que  joueront  tous  les  phénomènes  de  rupture  et  de  freinage  visant  à  rétablir  une 
compatibilité entre l'essence et ses manifestations. 
\bigskip
Deux notions intéressantes, celles d'abus et de manque sont corrélatives à la notion d'essence, qui 
exprime  la  façon  normale  dont  les  choses  se  présentent.  Il  suffira  de  mentionner  l'abus  ou  le 
manque pour que l'auditeur se réfère à une essence implicitement supposée. 
\bigskip
Ainsi la maxime « il ne faut pas argumenter de l'abus contre l'usage », est souvent, selon Bentham 
(2),  un  moyen  sophistique  pour  ne  pas  tenir  compte  des  mauvais  effets  d'une  institution.  On 
considère comme abus les mauvais effets qui en dérivent, et comme usage ce qu'elle est idéalement 
dans l'esprit de ses promoteurs, et qui correspondrait donc à son essence. » 
\bigskip
(1) Cf. J. Weiler, Problèmes d'économie internationale, vol. II, pp. 282 à 300.  
(2) Bentham, Œuvres, t. I : Traité des sophismes politiques, pp. 479-480. 
\bigskip
P 441-442 : « C'est ce qui était intentionnel, admet-on souvent, qui détermine l'essence. Le reste, 
ce qui contrevient à cette visée, est considéré comme abus, comme accident. Ce lien entre intention 
et essence est manifeste dans ce passage de Bossuet: 
\bigskip
\bigskip
\bigskip
\bigskip
231 
\bigskip
Vous  trouverez  étrange  peut-être  que  je  donne  de  si  grands  éloges  aux  anges  rebelles  et 
déserteurs;  mais  souvenez-vous,  s'il  vous  plaît,  que  je  parle  de  leur  nature,  et  non  pas  de  leur 
malice ; de ce que Dieu les a faits et non pas de ce qu'ils se sont faits eux-mêmes (1). » 
\bigskip
(1) Bossuet, Sermons, vol. II : Premier sermon sur les démons, p. 6. 
\bigskip
P 442 : « L'usage normal est conforme à l'essence ; l'abus doit être détaché de celle-ci, sous peine 
de  la  modifier  profondément.  Toutefois,  aussi  longtemps  que  l'on  utilise  le  terme  «  abus  »,  c'est 
signe que l'on veut préserver l'essence, que le débat ne porte pas sur elle. Si des libéraux, partisans 
du  capitalisme,  sont  néanmoins  en  faveur  du  contrôle  des  bénéfices,  ils  diront  que  c'est  pour 
corriger  un  des  abus  du  capitalisme,  pour  maintenir  une  structure  économique  essentiellement 
saine. Les socialistes soutiendront cette même mesure pour brimer le capitalisme qui produit, par 
son seul fonctionnement, de révoltantes inégalités. Par contre, le libéral, adversaire de la mesure, 
dira qu'elle risque de modifier profondément la structure du régime; le communiste, adversaire de 
la même mesure, dira qu'il ne s'agit que d'une mesure illusoire, que ce n'est qu'un palliatif qui ne 
modifie  rien  à  l'essentiel  du  régime. Qui  a raison,  en  l'occurrence  ?  Il  est  difficile  de  le  dire  sans 
avoir une idée précise de ce qu'est l'essence du capitalisme, chacun concevant cette notion de façon 
à  justifier  son  propre  point  de  vue  :  ce  que  l'on  considère,  traditionnellement,  comme  des 
jugements de valeur, détermine des structures conceptuelles qui permettent de préciser le sens et 
la  portée  de  ce  que  l'on  appelle  des  jugements  de  fait.  Quand  révolution  et  réforme  sont 
caractérisées, non par les moyens employés, mais par l'ampleur des changements d'un système, la 
même discussion peut se reproduire : elle portera sur l'essence du système modifié. » 
\bigskip
P 442-443 : « Ajoutons que, sur le plan de la connaissance, à la notion d'abus, correspond celle de 
« déformation ». Ainsi, selon Chester Bowles, les Indiens ont une vue déformée du capitalisme (1). 
Cette idée se rattache d'ailleurs non seulement à celle d'abus mais aussi à celle de manque. » 
\bigskip
(1) Chester Bowles, Ambassador's Report, p. 106. 
\bigskip
P 443 : « De même que l'abus, le manque ne peut être invoqué que si l'on a une notion, vague ou 
précise,  de  l'essence  par  rapport  à  laquelle  il  se  détermine.  Le  critère  permettant  de  prouver  ce 
manque est entièrement subordonné à la conception que l'on se forme de l'essence. On trouverait 
par  exemple  une  application  curieuse  de  l'idée  de  manque  dans  les  descriptions  que  la 
psychanalyse fait de la femme : les caractères de celle-ci sont interprétés comme une réaction au 
manque  d'organes  génitaux  externes,  ce  qui  implique  que  l'homme  est  considéré  comme 
représentant l'essence (2). 
\bigskip
Le manque, plus que la négation dont il peut être rapproché, est caractéristique de l'argumentation 
sur  les  valeurs,  sur  ce  qui  doit  être  fait.  La  notion  de  manque  ne  peut  se  réduire,  comme  la 
négation,  à  des  caractères  formels,  réversibles  et  statiques,  car  elle  se  définit  par  rapport  à  une 
norme, qu'il s'agisse de normal ou d'idéal. Elle correspond à ce que J.-P. Sartre appelle la négation 
interne, par opposition à la négation externe. 
\bigskip
Par négation interne, écrit-il, nous entendons une relation telle entre deux êtres que celui qui est 
nié de l'autre qualifie l'autre par son absence même, au coeur de son essence (3). » 
\bigskip
(2) Cf. Viola Klein, The feminine character. History of on Ideology, pp. 72, 83. 
(3) J.-P. Sartre, L’être et le néant, p. 223. 
\bigskip
P 443-444 : « Lorsque l'essence est considérée comme ne pouvant, en aucun cas, être remaniée, le 
manque,  perçu  comme  une  déception,  peut  suggérer  que  ce  vide  sera  comblé;  on  en  tirera 
argument pour prétendre que l'on a quelque chose à attendre : 
\bigskip
\bigskip
\bigskip
\bigskip
232 
\bigskip
-  il  voit  bien  qu'il  n'est  pas  possible  que  notre  nature,  qui  est  la  seule  que  Dieu  a  faite  à  sa 
ressemblance, soit la seule qu'il abandonne au hasard ; ainsi, convaincu  par raison qu'il doit y 
avoir  de  l'ordre  parmi  les  homines,  et  voyant  par  expérience  qu'il  n'est  pas  encore  établi,  il 
conclut nécessairement que l'homme a quelque chose à attendre (1). » 
\bigskip
(1) Bossuet, sermons, vol. II: Sur la Providence p. 208. 
\bigskip
P  444 :  « Ce  qui  est  de  trop  se  définit  également  par  rapport  à  l'essence,  soit  par  rapport  à  une 
essence déterminée, soit par rapport à une essence quelconque ; ce qui est de trop, dans ce dernier 
sens,  ne  pouvant  être  expliqué  par  aucune  structure,  par  aucun  ordre,  n  1  aura  ni  poids  ni 
signification : 
\bigskip
La  conscience  existe  comme  un  arbre,  comme  un  brin  d'herbe.  Elle  somnole,  elle  s'ennuie...  Et 
voici le sens de son existence : c'est qu'elle est conscience d'être de trop... (2). 
\bigskip
Les  techniques  pour  suggérer  qu'il  y  a  manque  ou  qu'il  y  a  quelque  chose  de  trop  seront  très 
diverses. L'une d'elle sera le souhait : celui-ci pourra dévaluer la personne à qui il est adressé, en 
évoquant une essence a laquelle elle ne se conformerait pas. La plus sûre parade, dit Sterne, est la 
suivante : 
\bigskip
... le souhaité doit se lever brusquement et souhaiter au souhaiteur quelque chose en échange de 
valeur équivalente... (3). 
\bigskip
Parfois la simple qualification, en évoquant l'essence, peut faire comprendre combien la réalité s'en 
éloigne : on rendra par là manifeste, une imperfection qui sans cet élément de référence passerait 
peut-être inaperçue. Antoine présentera Brutus comme un ami de César, afin de montrer combien 
il  a  failli  à  ce qui  est  l'essence  de  l'amitié  (4).  Parfois  les  modes  d'expression  seront  utilisés  pour 
suggérer  le  manque  :  un  style  passionné  pourra  faire  comprendre  que  la  scène  décrite  l'est 
beaucoup trop peu. 
\bigskip
On  retrouve  ces  techniques  dans  l'allusion  et  l'ironie,  la  première  se  référant  implicitement,  la 
seconde explicitement à l'essence qui sert de critère de dévaluation. » 
\bigskip
(2) J.-P. Sartre, La nausée, p. 213.  
(3) Sterne, vie et opinions de Tristram Shandy, liv. III, chap. 1, p. 135. 
(1) Shakespeare, Julius Caesar, acte Ill, se. Il. 
\bigskip
P 445 : « Pour finir ce paragraphe par une remarque qui renforcera notre conception selon laquelle 
les diverses liaisons de coexistence résultent de la généralisation on plutôt de la transposition du 
rapport  acte-personne,  nous  observerons  que  les  catégories  d'essence  et  de  personne  peuvent 
servir  à  l'interprétation  des  mêmes  phénomènes.  Chaque  fois  que  l'on  utilise  notamment  les 
arguments  par  le  manque,  c'est  la  notion  d'essence  qui  est  appliquée,  même  à  la  personne.  Par 
contre chaque fois que l'on désire rendre stables, concrets et présents un groupe, une essence, on 
se servira de la  Personnification.  Cette figure argumentative permet de stabiliser les contours du 
groupe, d'en rappeler la cohésion. Elle peut aussi s'appliquer à certains traits de l'individu, comme 
dans cette phrase de Démosthène : 
\bigskip
En réalité, c'est votre mollesse et votre négligence que Philippe a vaincues, mais il n'a pas vaincu 
la république (1)... 
\bigskip
Nous avons ici deux espèces de personnification, d'une part celle de la mollesse et de la négligence, 
d'autre  part,  celle  de  la  République.  La  première  est  une  technique  de  rupture;  elle  a  pour  effet 
d'isoler,  en  en  faisant  des  êtres  distincts,  les  défauts  qu'ont  montrés  les  citoyens  d'Athènes,  de 
mettre par là ces derniers à l'abri de l'effet trop dévaluant de leurs actes et de permettre qu'ils se 
considèrent, malgré ces tares momentanées, comme les membres de la république insoumise. La 
personnification de celle-ci, d'autre part, renforce son importance en tant que groupe, plus stable 
que  les  individus  qui  n'en  sont  que  la  manifestation,  et  nettement  opposée  aux  accidents  et 
vicissitudes, que causent les événements. 
\bigskip
La  personnification  sera  souvent  soulignée  par  l'emploi  d'autres  figures.  Par  l'apostrophe  on 
s'adressera  à  ce  qui  est  personnifié  et  devenu  ainsi  capable  d'être  pris  comme  auditeur;  par  la 
Prosopopée on en fera un sujet discourant et agissant. » 
\bigskip
(1) Démosthène, Harangues, t. II : Troisième Philippique, §5. 
\bigskip
§ 75. LA LIAISON SYMBOLIQUE 
\bigskip
P  446 :  « Nous  croyons  utile  de  rapprocher  la  liaison  symbolique  des  liaisons  de  coexistence.  En 
effet  le  symbole,  pour  nous,  se  distingue  du signe,  parce  qu'il  n'est  pas  purement  conventionnel; 
s'il  possède  une  signification  et  une  valeur  représentative,  cette  signification  et  cette  valeur  se 
tirent  de  ce  qu'il  semble  exister,  entre  le  symbole  et  ce  qu'il  évoque,  un  rapport  que,  faute  d'un 
meilleur  terme,  nous  qualifierons  de  rapport  de  participation.  C'est  la  nature  quasi  magique,  en 
tout cas irrationnelle, de ce rapport, qui distingue la liaison symbolique des autres liaisons, aussi 
bien  de  succession  que  de  coexistence.  De  même  que  celles-ci,  le  lien  symbolique  est  envisagé 
comme faisant partie du réel, mais il ne se réfère pas à une structure définie de ce dernier. Par le 
fait que, très souvent, symbole et symbolisé ne font pas partie de ce qui, par ailleurs, est considéré 
comme  une  même  couche  de  réalité,  comme  un  même  domaine,  leur  relation  pourrait  être 
considérée comme analogique; mais on détruirait, par là même, ce qu'il y a d'impressionnant dans 
la  liaison  symbolique,  car  pour  qu'elle  joue  son  rôle,  il  faut  que  symbole  et  symbolisé  soient 
intégrés  dans  une  réalité  mythique  ou  spéculative,  où  ils  participent  l'un  à  l'autre (1).  Dans  cette 
nouvelle  réalité,  il  existe une  liaison de  coexistence  entre  les  éléments  de  la  relation  symbolique, 
même quand, en fait, le symbole est séparé du symbolisé par un intervalle temporel. » 
\bigskip
(1) Selon Cassirer, dans la vision mythique, la partie s'identifiant au tout, symbole et symbolisé 
sont indiscernables. Cf. E. Cassirer, The philosophy of symbolic forms, vol. II : Mythical thought. 
\bigskip
P 446-447 : « Il en va ainsi lorsque l'on traite certaines personnes et certains événements comme « 
figure » d'autres personnes et d'autres événements. Entre Adam, ou Isaac, on joseph, et le Christ, 
dont ils sont considérés comme la préfigure, il n'y a pas de lien de succession sur le mode causal, 
mais  un  rapport  indéfinissable  de  coexistence,  une  participation  qui  se  situerait  dans  la  vision 
divine du réel. » 
\bigskip
P  447 :  « La  liaison  symbolique  entraîne  des  transferts  entre  symbole  et  symbolisé.  Lorsque  la 
croix,  le  drapeau,  la  personne  royale,  sont  envisagés  comme  Symboles  du  christianisme,  de  la 
patrie,  de  l'État,  ces réalités  suscitent  un  amour  on  une  haine,  une  vénération  ou  un  mépris,  qui 
seraient incompréhensibles et ridicules si, à leur caractère représentatif, n'était attaché un lien de 
participation.  Celui-ci  est indispensable  pour  susciter  la  ferveur  patriotique  ou  religieuse (1).  Les 
cérémonies  de  communion  exigent,  en  effet,  un  support  matériel  sur  lequel  puisse  se  concentrer 
l'émotion, que la seule idée abstraite pourrait plus difficilement susciter et nourrir. Ce lien entre le 
support  et  la  chose  qu'il  figure  n'est  pas  fourni  par  une  liaison  admise  par  tous,  c'est-à-dire 
objective, mais par une liaison que reconnaissent uniquement les membres du groupe : la croyance 
en ces structures de participation est, elle-même, un aspect de la communion entre eux. » 
\bigskip
(1) Harold D. Lasswell, Language of politics, Introduction, p. 11. 
\bigskip
P 447-448 : « La constatation de ces liens immatériels, de ces harmonies et solidarités invisibles, 
caractérise une conception poétique ou religieuse, en un mot, romantique, de l'univers. Les auteurs 
\bigskip
\bigskip
\bigskip
234 
\bigskip
romantiques  avaient,  on  le  sait,  une  certaine  prédilection  à  décrire  les  événements  de  façon  telle 
qu'émotions  humaines  et  milieu  physique  semblaient  participer  l'un  à  l'autre.  Même  un  auteur 
aussi réaliste que Balzac n'a pas échappé à cette vision romantique des choses, comme le prouve ce 
portrait de Mme Vauquer, dans Le Père Goriot: 
\bigskip
Sa face vieillotte, grassouillette, du milieu de laquelle sort un nez à bec de perroquet ; ses petites 
mains  potelées,  sa  personne  dodue  comme  un  rat  d'église,  son  corsage  trop  plein  et  qui  flotte, 
sont en harmonie avec cette salle où suinte le malheur, où s'est blottie la spéculation, et dont Mme 
Vauquer  respire  l'air  chaudement  fétide  sans  en  être  écœurée.  Sa  figure  fraîche  comme  une 
première  gelée  d'automne,  ses  yeux  ridés,  dont  l'expression  passe  du  sourire  prescrit  aux 
danseuses à Fainer renfrognement de l'escompteur, enfin toute sa personne explique la pension, 
comme la pension implique sa Personne. Le bagne ne va pas sans l'argousin, vous n'imagineriez 
pas 'un sans l'autre (1). » 
\bigskip
(1) Cité par Auerbach, Mimesis, p. 416. Cf. aussi E. Poe, The Fall of the house Of Usher; Villiers De 
l’Isle Adam, L'intersigne. 
\bigskip
P  448 :  « Notons  qu'il  existe  souvent pour  l'orateur  une  grande  liberté  de  choix  dans  les  liaisons 
utilisées. Ainsi, alors que tout semble indiquer que dans la  Divine Comédie  les âmes ici-bas sont 
considérées comme figure de ce qu'elles seront dans l'au-delà (2), c'est là une manière de concevoir 
le  rapport  entre  vie  présente  et  vie  future  qui  est  loin  de  s'imposer.  Dans  le  cas  de  Balzac,  pour 
interpréter les rapports entre individu et milieu, des liaisons précises auraient pu être invoquées : 
liaisons  causales,  liaisons  acte-essence.  Mais  ce  n'est  que  dans  un  cadre  présenté,  par  simple 
description  et  sans  justification,  comme  unitaire,  lorsqu'une  liaison  de  participation  est  postulée 
entre les personnes et le milieu, que le moindre événement peut prendre une valeur symbolique.  
\bigskip
Les actes symboliques joueront un tout autre rôle et auront une tout autre signification que ceux 
qui ne le sont pas : ils réagissent d'une manière plus violente sur les êtres qui en sont solidaires, 
qui  en  sont  responsables.  Les  techniques  de  rupture  ou  de  freinage  entre  acte  et  personne  ne 
pourront  être  utilisées,  lorsque  l'acte est  considéré  comme  symbolique,  parce  que  ces  techniques 
impliquent une certaine rationalité. » 
\bigskip
(2) Cf. Auerbach, Mimesis, pp. 183 à 196. 
\bigskip
P 448-449 : « Il importe donc,  dans l'argumentation, de  savoir dans quelle  mesure une chose, et 
tout  ce  qui  la  touche,  est  pourvue  de  cette  nature  symbolique.  Or  il  y  a  moyen,  étant  donné  le 
caractère indéterminé et indéfini objectivement de la liaison symbolique, de conférer à toute chose, 
à tout acte, à tout événement, une valeur symbolique, et de modifier, par là, sa signification et son 
importance.  L'aspect  symbolique  d'un  acte  sera  admis  d'autant  plus  facilement  que  toute  autre 
interprétation sera moins plausible. » 
\bigskip
P  449 :  « Certains  indices  peuvent  devenir  symboliques  d'une  situation,  d'une  manière  de  vivre, 
d'une classe sociale, comme le fait de posséder une voiture d'une certaine marque ou de porter un 
chapeau haut-de-forme. De même si un individu, membre d'un groupe, est devenu symbole de ce 
dernier,  son  comportement  sera  considéré  comme  plus  important,  parce  que  plus  représentatif, 
que celui d'autres membres du même groupe. Cet individu symbolique, qui représente le groupe, 
sera  parfois  choisi  pour  jouer  ce  rôle  :  tantôt  parce  qu'il  est  le  meilleur  dans  un  domaine 
déterminé,  comme  le  champion  de  boxe,  par  exemple,  tantôt  parce  qu'il  est  un  individu  moyen, 
que rien ne distingue, même pas son nom, tel le soldat inconnu. 
\bigskip
Celui qui est porte-parole du groupe est, par là même, admis comme représentatif. Se considérer, 
ou  être  considéré,  comme  symbole  du  groupe  est  un  fait  qui  peut  exercer  une  influence 
déterminante sur la conduite. Tout recours, dans l'argumentation, à la notion d'honneur est lié à 
\bigskip
\bigskip
\bigskip
235 
\bigskip
l'idée que l'individu est symbole d'un groupe. L'honneur varie avec le groupe et suppose d'ailleurs 
une  certaine  supériorité  de  celui-ci.  Si  l'on  parle  de  l'honneur  de  la  personne,  c'est  comme 
représentant symbolique du groupe des humains. Le serment sur l'honneur n'est pas une référence 
à la valeur de l'individu, mais à sa relation symbolique avec le groupe. 
\bigskip
La conduite d'un individu peut déshonorer le groupe ; si elle déshonore aussi l'individu, c'est parce 
qu'elle  entraîne  son  exclusion  du  groupe,  et,  à  la  limite,  de  celui  même  des  humains.  On  le 
considère  comme  un  pestiféré,  dont  on  craint  la  contamination  symbolique.  Cela  se  traduira 
juridiquement par la mort civile; dans certains cas la pression morale poussera au suicide. » 
\bigskip
P  449-450 :  « Le  recours  au  symbole peut  jouer  un  rôle  éminent  aussi bien  dans  la  présentation 
des prémisses que dans l'ensemble de l'argumentation. Tout ce qui concerne le symbole est censé 
concerner le symbolisé. Et encore que le rapport entre eux ne soit pas strictement réversible (1)  - 
mais c'est là un caractère que nous avons observé  dans toutes les liaisons, hors certaines liaisons 
formelles  de  l'argumentation  quasi  logique  -  le  symbole  se  modifie  par  son  usage  en  tant  que 
symbole.  Quelle  que  soit  la  genèse  du  lien  symbolique,  admis  généralement,  entre  le  lion  et  le 
courage, chaque nouvel usage de ce lien, dans l'argumentation, confère au lion certains caractères 
et une certaine valeur attachés au courage. » 
\bigskip
(1) Cf. Silvio, Ceccato, Divagazioni di animal semioticum, Sigma, 1947. 
\bigskip
P  450 :  « Le  symbole  est  généralement  plus  concret,  plus  maniable,  que  le  symbolisé,  ce  qui 
permettra de concentrer en attitude concernant le symbole  - tel le fait de saluer le drapeau  - une 
attitude  concernant  le  symbolisé  qui  exigerait,  pour  être  comprise,  de  longs  développements.  La 
technique du boue émissaire simplifie les comportements, par l'utilisation du rapport symbolique 
de participation entre individu et groupe. 
\bigskip
Le  symbole  est  non  seulement  plus  maniable,  mais  il  peut  s'imposer  avec  une  présence  que 
n'aurait pas le symbolisé : le drapeau que l'on voit, ou que l'on décrit, peut flotter, claquer au vent, 
se  déployer.  Le  symbole,  malgré  ses  liens  de  participation,  garde  une  certaine  individualité  qui 
permet les manipulations les plus variées. « Il n'y a plus de Pyrénées » n'évoque pas seulement une 
idée  politique,  mais  aussi  les  fatigues,  les  dangers  d'une  frontière,  la  masse  d'efforts  nécessaire 
pour l'annihiler. » 
\bigskip
P 450-451 : « Tout symbole peut être utilisé comme signe et servir de moyen de communication, à 
condition  qu'il  s'intègre  dans  un  langage  compris  par  les  auditeurs.  Mais  la  liaison  symbolique 
n'étant ni conventionnelle, ni basée sur une structure du réel universellement connue et admise, la 
signification d'un symbolisme peut être réservée aux seuls initiés, rester pour d'autres entièrement 
incompréhensible;  ce  qui  était  symbole  perdra  complètement  ce  caractère  si  cette  initiation  fait 
défaut. » 
\bigskip
P  451 :  « Il  se  peut  cependant  que,  après  avoir  perdu  leur  aspect  symbolique,  certaines  réalités 
continuent  à  être  utilisées  comme  signes,  comme  moyens  de  communication  purement 
conventionnels. Elles seront, pour ainsi dire, 'désacralisées et joueront alors un rôle tout différent 
dans la vie spirituelle. Le symbole devenu signe désigne plus adéquatement l'objet signifié qu'il ne 
le  faisait  auparavant,  il  est  mieux  adapté  aux  besoins  de  la  communication,  parce  qu'il  a  perdu 
certains des aspects qui lui étaient propres, qui lui conféraient une réalité indépendante de celle du 
symbolisé ; mais cet avantage du symbole devenu signe est compensé par le fait que l'action sur le 
signe n'entraîne plus l'action sur le signifié. 
\bigskip
N'oublions  pas  toutefois  que,  comme  toute  liaison,  la  liaison  symbolique  peut  s'appliquer  au 
discours  lui-même.  Qu'il  ait  ou  non  une  origine  symbolique,  le  signe  verbal  peut  être  considéré 
comme ayant un lien magique avec le signifié : le discours agit sur ce qu'il énonce. ]D'autre part, 
\bigskip
\bigskip
\bigskip
236 
\bigskip
l'action sur le signe pourra symboliser l'action sur le signifié : telle négligence dans l'énoncé d'un 
nom  propre,  la  suppression  de  certaines  terminaisons,  la  substitution  de  certaines  consonnes  à 
d'autres, autant d'actions qui peuvent agir d'une manière indirecte, volontairement ou non, sur la 
conception que l'auditeur se fait du signifié. 
\bigskip
La  précarité  de  la  liaison  symbolique,  jointe  à  son  pouvoir  evocateur  et  à  sa  force  émotive,  tient 
sans  doute  au  fait  qu'elle  n'est  guère  sujette  à  justification.  Les  symboles  exercent  une  action 
indéniable sur ceux qui reconnaissent la liaison symbolique, mais n'en ont aucune sur les autres : 
ils  sont  caractéristiques  d'une  culture  particulière,  mais  ne  peuvent  servir  pour  l'auditoire 
universel, ce qui confirme leur aspect irrationnel. » 
\bigskip
P  451-452 :  « Cependant  si  les  liaisons  symboliques  sont  extrêmement  variées,  si  elles  sont 
précaires  et  particulières,  ce  qui  ne  l'est  pas,  c'est  J'existence  même  de  symboles  et  l'importance 
qu'on  leur  accorde. La  valeur  symbolique  in  abstracto  peut  donc,  contrairement  aux  symboles 
particuliers,  constituer  l'objet  d'une  argumentation  rationnelle,  d'une  argumentation  visant  à 
l'universel.  Il  en  est  de  même  en  ce  qui  concerne  toute  argumentation  demandant  que  l'on  lie 
néglige pas, que l'on lie sous-estime pas des liaisons symboliques propres à certains milieux, quand 
on s'adresse à eux : ce qu'on exige, dans ce cas, c'est tout simplement le respect d'un fait, qui est le 
rôle joué par des symboles déterminés dans une certaine société. » 
\bigskip
P  452 :  « Les  figures  de  substitution,  métonymie  et  synecdoque,  ont  été,  selon  les  auteurs, 
diversement décrites et définies (1). 
\bigskip
Ce qui nous paraît mériter attention, autant que le rapport structurel entre les termes substitués 
l'un  à  l'autre,  est  de  voir  s'il  existe  entre  eux  un  lien  réel  et  de  voir  quel  il  est.  Sur  ce  plan,  une 
distinction importante entre figures de substitution apparaîtra. 
\bigskip
Liaison symbolique, semble-t-il, dans cette métonymie empruntée à Fléchier par Dumarsais : 
\bigskip
Cet homme [Macchabée] ... qui réjouissoit Jacob par ses vertus et par ses exploits (2). 
\bigskip
« Jacob » pour désigner le peuple juif, « John Bull » pour désigner l'Angleterre, « chemises noires 
» pour désigner les fascistes, autant de symboles. De même « le sceptre » pour l'autorité royale, « 
le chapeau » pour le cardinalat, « Mars » pour la guerre, et peutêtre même « la bouteille » pour le 
vin, « une Perse » pour un tissu venant de Perse, et « un Philippe » pour une monnaie à l'effigie de 
Philippe. » 
\bigskip
(1)  Cf.  notamment  Baron,  De  la  Rhétorique  pp.  341  à  345;  Cl.-L.  Estève,  Etudes  philosophiques 
sur l'expression littéraire, pp. 223-225. 
(2) Dumarsais, Des Tropes, p. 53. Cf. Fléchier, Oraison funèbre de Turenne, p. 4. 
\bigskip
P 452-453 : « Dans les synecdoques, par contre, telles « la voile » pour le  navire, « les mortels » 
pour les hommes, nous verrions que le terme substitué n'est plus uni à celui qu'il remplace par un 
lien symbolique, mais qu'il marque un aspect caractéristique de l'objet désigné : tantôt parce qu'il 
en est une partie, suffisante pour le reconnaître (la voile) ; tantôt parce qu'il en est le genre, mais 
un  genre  permettant  de  le  caractériser  de  la  façon  la  plus  pertinente  (les  mortels  par  opposition 
aux dieux). » 
\bigskip
P 453 : « Il va de soi que, si l'on porte attention surtout à la liaison entre termes, on pourra hésiter 
souvent entre l'interprétation comme métonymie ou comme synecdoque. Notons seulement (lue, 
si toutes les figures sont soumises à certaines convenances culturelles (on serait ridicule, prétend 
Dumarsais, si l'on disait qu'une armée navale était composée de cent mâts) (1), les figures basées 
\bigskip
\bigskip
\bigskip
237 
\bigskip
sur  la  liaison  symbolique  sont  les  plus  précaires  -  à  moins  de  devenir  signe  et  de  perdre  leur 
caractère de figure. » 
\bigskip
(1) Dumarsais, Des Tropes, P. 85. 
\bigskip
§  76.  L'ARGUMENT  DE  DOUBLE  HIERARCHIE  APPLIQUE  AUX  LIAISONS  DE 
\bigskip
SUCCESSION ET DE COEXISTENCE 
Les  hiérarchies,  tout  comme  les  valeurs,  font  partie  des  accords  qui  servent  de  prémisses  au 
discours ; mais on peut aussi argumenter à leur propos, se demander si une hiérarchie est fondée, 
où situer un de ses termes, montrer que tel terme devrait occuper telle place plutôt que telle autre. 
\bigskip
Divers arguments pourront être utilisés à cet effet. Toutefois le plus souvent on se basera sur une 
corrélation  entre  les  termes  de  la  hiérarchie  discutée  et  ceux  d'une  hiérarchie  admise  :  on  aura 
recours  à  ce que  nous  qualifions  d'argument  de  double  hiérarchie.  Parfois  même  on  présente  les 
hiérarchies  comme  liées  au  point  que  l'une  d'entre  elles  sert  de  critère  ou de  définition  à  l'autre. 
Quand on entend affirmer que tel homme est plus fort que tel autre, parce qu'il soulève des poids 
plus lourds, on ne sait Pas toujours si cette dernière hiérarchie sert de fondement ou de critère à la 
première. » 
\bigskip
P 454 : « L'argument de double hiérarchie est souvent implicite. En effet, derrière toute hiérarchie 
on voit se profiler une autre hiérarchie ; ce recours est naturel et se produit spontanément parce 
que nous nous rendons compte que c'est ainsi que l'interlocuteur tenterait sans doute de soutenir 
son affirmation. Au point que la méditation sur les hiérarchies amène souvent à nier qu'il puisse 
exister  des  hiérarchies  simples.  Il  faut  se  garder  de  croire  cependant  que  la  hiérarchie  que 
l'interlocuteur  utiliserait  comme  justification  est  nécessairement  celle  à  laquelle  nous  pensons. 
Quand on demande pourquoi telle information paraît sous un plus grand titre que telle autre, on 
pourrait  dire  qu'elle  est  plus  importante,  plus  intéressante,  plus  inattendue,  mais  on  voit  que  la 
hiérarchie qui devrait fonder celle des titres reste implicite et vague. 
\bigskip
La  double  hiérarchie  exprime  normalement  une  idée  de  proportionnalité,  directe  ou  inverse,  ou 
tout au moins un lien terme à terme. Cependant, dans beaucoup de cas, la liaison se réduit, quand 
on l'examine de près, à l'idée d'une corrélation statistique, où les termes hiérarchisés de l'une des 
suites sont reliés à une moyenne de termes appartenant à l'autre. C'est le cas par exemple quand 
on  conclut  de  la  taille  respective  de  deux  hommes  à  la  longueur  respective  probable  de  leurs 
membres. 
\bigskip
Mais il va de soi que maintes hiérarchies ne peuvent être ni décrites ni fondées à l'aide d'éléments 
homogènes,  quantifiables  ou  mesurables.  Or  c'est  quand  on  se  trouve  en  face  de  hiérarchies 
qualitatives que l'argumentation, ne pouvant être remplacée par la mesure ou le calcul, joue le plus 
grand rôle et que, pour soutenir ces hiérarchies, on aura recours à d'autres, souvent empruntées au 
monde  physique.  On  se  servira  par  exemple  des  notions  de  profondeur,  hauteur,  grandeur, 
consistance. » 
\bigskip
P  454-455 :  « La  hiérarchie  quantitative  qui  semble  sous-tendre  l'autre  est  peut-être  réglée  elle-
même par une hiérarchie de valeurs ; ainsi, lorsque saint Anselme conclut que la liberté de lie pas 
pouvoir  pécher  est  plus  grande  que  la  liberté  de  pouvoir  ou  ne  pas  pouvoir  pêcher,  la  hiérarchie 
d'intensité  dérive  de  ce  que  nous  attribuons  plus  de  valeur  à  la  première  liberté  (1).  Certaines 
maximes,  telles  «  qui  peut  le  plus  peut  le  moins»,  qui  développent  une  argumentation  quasi 
logique  --  l'inclusion  de  la  partie  dans  le  tout  -ne  peuvent  se  justifier  ou  s'appliquer  que  par  le 
recours à des doubles hiérarchies dont la plupart sont, malgré les apparences, qualitatives. » 
\bigskip
(1) Saint Anselme, De libero arbitrio, chap. 1, Patrol. latine, t. CLVIII, col. 490 C-491 A. 
\bigskip
\bigskip
\bigskip
\bigskip
238 
\bigskip
P  455 :  « A  vrai  dire,  peu  importe,  pour  leur  usage,  la  genèse  de  beaucoup  de  ces  doubles 
hiérarchies.  On  fera  cependant,  afin  de  justifier  leur  emploi,  un  effort  pour  découvrir,  entre  les 
deux hiérarchies, un rapport basé sur le réel, par le recours notamment à la notion de symbole. Ou 
bien on s'efforcera de voir entre les deux séries une liaison plus étroite encore, les deux ne formant 
qu'une même réalité : ainsi pour Cassirer, les références spatiales sont une forme indispensable à 
la constitution des  objets de pensée (2) ; pour beaucoup de contemporains, tels Sartre, Merleau-
Ponty, Minkowski, les qualités morales et les qualités physiques ont une seule et même racine de 
signification (3), et lorsque Gabriel Marcel affirme que la vie du croyant est supérieure à celle de 
l'incroyant  parce  que  plus  pleine,  il  souligne  explicitement  que  cette  expression  est  à  prendre  au 
sens de plénitude « métaphysique » (4), excluant ainsi, par principe, toute référence que ce soit à 
un récipient plus ou moins rempli ou à une matière plus ou moins dense. 
\bigskip
Toutes les liaisons fondées sur la structure du réel, qu'elles soient de succession ou de coexistence, 
pourront  servir  à  relier  deux  hiérarchies,  l'une  à  l'autre,  et  à  fonder  l'argument  de  double 
hiérarchie. » 
\bigskip
(2) E. Cassirer, The philosophy of symbolic forms, vol. 1, p. 199. 
(3)  J.-P.  Sartre,  L'être  et  le  néant,  pp.  695-96;  M.  Merleau-Ponty,  Phénoménologie  de  la 
perception,  p.  329  ;  E.  Minkowski,  Le  langage  et  le  vécu,  dans  le  vol.  Semantica  de  Archivio  di 
Filosofia, 1955, pp. 358, 362. 
(4) G. -Marcel, Le monde casse' et Position et approches concrètes du mystère ontologique, p. 259. 
\bigskip
P 456 : « C'est la relation de cause à  effet qui permettra de hiérarchiser les variations de volume 
d'un corps d'après la variation de la température. Inversement, une hiérarchie des fins peut nous 
aider à établir une hiérarchie des moyens, conformément à la remarque d'Aristote : 
\bigskip
De deux agents de production, est préférable celui dont la fin est meilleure (1). 
\bigskip
Un  être  raisonnable  ne  peut  que  se  conformer  à  cette  double  hiérarchie.  De  là,  la  force  de 
l'argument de Leibniz, repris des Évangiles : 
\bigskip
...  ayant  soin  des  passereaux,  il [Dieu]  ne  negligera  pas  les  créatures  raisonnables  qui  luy  sont 
infiniment plus cheres (2)... . 
\bigskip
Bossuet se sert du même argument dans ses sermons : 
\bigskip
Vous  vous  êtes  tant  de  fois  surmontés  vous-mêmes  pour  servir  à  l'ambition  et  à  la  fortune, 
surmontez-vous quelque fois pour servir à Dieu et à la raison (3). 
\bigskip
Il  l'utilise  ailleurs  en  faisant  état  d'une  hiérarchie  des  fins  basée  elle-même,  non  sur  leur  valeur 
propre, mais sur la facilité à les atteindre : 
\bigskip
S'il  [le  démon]  se  roidit  avec  tant  de  fermeté  contre  Dieu,  bien  qu'il  sache  que  tous  ses  efforts 
seront  inutiles;  que  n'entreprendra-t-il  pas  contre  nous,  dont  il  a  si  souvent  expérimenté  la 
faiblesse ? (4). 
\bigskip
Cet argument est du même ordre que le lieu d'Aristote : 
\bigskip
De deux moyens, le plus désirable est celui qui est le plus rapproché de la fin (5). » 
\bigskip
(1) Aristote, Topiques, liv. III, chap. 1, 116 b, 25-30. 
(2) Leibniz, éd. Gerhardt, vol. 4 : Discours de métaphysique, XXXVII, p. 463. 
(3) Bossuet, sermons, vol. II, Sur l'efficacité de la pénitence, p. 567. 
\bigskip
\bigskip
\bigskip
239 
\bigskip
(4) Bossuet, Sermons, vol. II, Premier sermon sur les démons, p. 16. 
(5) Aristote,  liv. III, chap.1, 116 b, 20-25. 
\bigskip
P 456-457 : « Plus souvent que sur des liaisons de succession, la double hiérarchie est fondée sur 
des liaisons de coexistence. C'est ainsi que la hiérarchie des personnes entraîne une hiérarchisation 
de  leurs  sentiments,  de  leurs  actions,  de  tout  ce  qui  émane  d'elles.  C'est  ce  qu'exprime  ce  lieu 
d'Aristote : 
\bigskip
L'attribut qui appartient à un sujet meilleur et plus honorable est aussi préférable : par exemple, 
ce qui appartient à Dieu est préférable à ce qui appartient à l'homme, et ce qui appartient à l'âme 
à ce qui appartient au corps (1). » 
\bigskip
(1) Aristote,  Topiques, liv. III, chap. 1, 116 b, 
\bigskip
P 457 : « La célèbre réplique d'Antigone n'en constitue qu'une application : 
\bigskip
Et je n'ai pas cru que tes édits pussent l'emporter sur les lois lion écrites et immuables des Dieux, 
puisque tu n'es qu'un mortel (2). 
\bigskip
L'attitude d'Antigone est légitime, l'attitude opposée est ridicule : 
\bigskip
C'est une plaisante chose à considérer, de ce qu'il y a des gens dans le inonde qui, ayant renoncé à 
toutes  les  lois  de  Dieu  et  de  la  nature,  s'en  sont  fait  eux-mêmes  auxquelles  ils  obéissent 
exactement, comme par exemple les soldats de Mahomet les voleurs, les hérétiques, etc. (3). 
\bigskip
Toute  cette  argumentation,  pour  être  efficace,  suppose  évidemment  un  accord  préalable  sur  la 
hiérarchie des personnes. Quand Iphicrate ayant demandé à Aristophon s'il livrerait des vaisseaux 
pour de l'argent, et sur sa réponse négative, s'exclame : 
\bigskip
Toi, un Aristophon, tu ne les livrerais pas, et moi, un Iphicrate, je le ferais ! (4). 
\bigskip
cet argument n'a de valeur que pour celui qui lie doute pas de la supériorité morale d’Iphicrate. 
\bigskip
L'argumentation par double hiérarchie reçoit chez Aristote quelques applications curieuses, basées 
sur  les  rapports  qui  existent,  dans  sa  métaphysique,  entre  une  essence  et  ses  incarnations.  Il 
n'hésitera pas à dire que  
\bigskip
Si A est, absolument, meilleur que B, la meilleure des choses contenues dans A est meilleure que la 
meilleure  de  celles  contenues  dans  B  :  par  exemple  si  l'homme  est  meilleur  que  le  cheval,  le 
meilleur homme sera -meilleur aussi que le meilleur cheval, » 
\bigskip
(2) Antigone, trad. Leconte de Lisle, p. 249. 
(3) Pascal, Bibl. (le la Pléiade, Pensées, 286 (157*), p. 898 (393 éd. Brunschvieg). 
(4) Aristote,, Rhétorique, II, chap. 23, 1398 (Y. 
\bigskip
P 458 : « et inversement : 
\bigskip
...  si  le  meilleur  homme  est  meilleur  que  le  meilleur  cheval,  alors,  absolument,  l'homme  est 
meilleur que le cheval (1). 
\bigskip
Il  raisonnera  de  la  même  façon  sur  la  taille  des  hommes  et  des  femmes  (2),  en  admettant 
implicitement  que  la  dispersion,  au  sens  statistique,  dans  les  divers  groupes  reste  toujours  la 
même. 
\bigskip
\bigskip
\bigskip
240 
\bigskip
 
Par ailleurs, dans la biologie contemporaine également, la relation de coexistence, bien plutôt que 
la  relation  causale,  soustend  les  rapports  entre  hiérarchies  de  divers  caractères  chez  un  même 
individu  -  par  exemple  taille  et  poids  -  ou  encore  entre  hiérarchie  des  espèces  et  hiérarchie  d'un 
caractère donné - par exemple, place dans la lignée évolutive, et poids du cerveau. 
\bigskip
Les doubles hiérarchies sont souvent utilisées pour extrapoler l'une des hiérarchies : 
\bigskip
S'il  plaît  aux  barbares  de  vivre  au  jour  le  jour,  nos desseins  à  nous  doivent  envisager  l'éternité 
des siècles (3). 
\bigskip
Il est pourtant malaisé de dire si l'extrapolation ne concerne strictement que l'une des hiérarchies. 
La durée à prévoir est extrapolée jusqu'à couvrir l'éternité, mais l'homme n'est-il pas lui aussi porté 
ici au delà de sa condition ? Que l'extrapolation puisse concerner les deux hiérarchies ressort très 
bien de l'exemple suivant: 
\bigskip
La  conscience  est  généralement  emprisonnée  dans  le  corps;  elle  est  concentrée  dans  les  centres 
du  cerveau,  du  coeur  et  du  nombril  (mental,  émotif  et  sensoriel).  Quand  vous  sentez  qu'elle  ou 
quelque chose d'elle s'élève et se fixe au-dessus de la tête, ... c'est le mental en vous qui monte en 
cette place, y prend contact avec quelque chose de supérieur au mental ordinaire (4)... » 
\bigskip
(1) Aristote, Topiques, liv. III, chap. 2, 117 b. 
(2) Aristote, Rhétorique, liv. I, chap. 7, 1363 b. 
(3) Cicéron, De Oratore, liv. II, § 169. 
(4) Shrî Aurobindo, Le guide du yoga, p. 90. 
\bigskip
P  458-459 :  « L'extrapolation  peut  consister  également  à  passer  des  degrés  positifs  aux  degrés 
négatifs d'une qualité ou d'une situation ou inversement. 11 semble bien que ce soit l'argument de 
double hiérarchie qui forme la base de ce que les Anciens appelaient « l'argument des contraires » 
et dont voici deux exemples : 
\bigskip
Etre tempérant est bon, attendu qu'être intempérant est nuisible ; Si la guerre est cause des maux 
présents, c'est avec la paix qu'il faut les réparer (1). » 
\bigskip
(1) Aristote, Rhétorique, liv. II, chap. 23, 1397 ei ; cf. Quintilien, VOI. II, liv. V, chap. X, § 73. 
\bigskip
P  459 :  « Ces  arguments,  dont  l'analyse  en  fonction  des  sujets  et  prédicats  paraît  artificielle  et 
sujette  à  caution,  se  justifient  si  l'on  admet  une  double  hiérarchie  qui  s'étende  sur  les  degrés 
négatifs  aussi  bien  que  positifs  d'une  qualité  ou  d'une  situation,  la  mention  de  termes  opposés 
n'étant qu'une facilité linguistique, destinée à indiquer, approximativement, la position respective 
des termes. 
\bigskip
L'argument  de  double  hiérarchie  permet  d'appuyer  une  hiérarchie  contestée  sur  une  hiérarchie 
admise ; il rend par là les plus grands services quand il s'agit de justifier des règles de conduite. Ce 
qui est préférable devant être préféré, la détermination de celui-là nous dicte notre comportement. 
Si  certaines  lois  sont  préférables  à  d'autres,  c'est  à  elles  qu'il  faut  obéir  et  non  aux  autres  ;  si 
certaines vertus sont objectivement supérieures, il faut s'efforcer de les acquérir dans sa vie. C'est 
par  le  biais  de  doubles  hiérarchies  que  des  considérations  métaphysiques  fournissent  un 
fondement à l'éthique, comme dans cet exemple caractéristique de Plotin : 
\bigskip
Puis donc que l'Un est l'objet de notre recherche et que nous examinons le pri  de toutes choses, le 
Bien  et  le  Premier,  il  ne  faut  pas  s'éloigner  des  objets  qui  sont  au  voisinage  des  premiers,  et 
\bigskip
\bigskip
\bigskip
241 
\bigskip
tomber jusqu'aux derniers de tous ; il faut se ramener soi-même des objets sensibles qui sont les 
derniers de tous jusqu'aux premiers objets (2) ; ... 
\bigskip
A une hiérarchie ontologique correspondra une hiérarchie éthique des conduites. » 
\bigskip
(2) Plotin, t. Vl, IIe Partie ; Ennéade VI, 9, § 3. 
\bigskip
P 460 : « Par un contre-coup assez compréhensible, si l'on n'est pas disposé à admettre des règles 
de conduite qui résultent de l'admission d'une double hiérarchie, celle-ci sera elle-même battue en 
brèche.  C'est  le  sens  de  la  remarque  d'Iphicrate  qui  dit  à  ceux  qui  voulaient  contraindre  aux 
liturgies son fils, trop jeune mais grand pour son âge : 
\bigskip
Si l'on considérait les enfants grands comme des hommes, l'on décréterait que les hommes petits 
sont des enfants (1). 
\bigskip
L'on  voit,  par  cet  exemple,  que  l'argumentation  par  la  double  hiérarchie  est  parfois  utilisée  pour 
conduire au ridicule; on montre que les énoncés de l'adversaire impliquent une double hiérarchie 
inadmissible. 
\bigskip
La  réfutation  d'une  double  hiérarchie  se  fait,  soit  en  contestant  une  des  hiérarchies,  soit  en 
contestant la liaison établie entre elles - ce qui suppose un changement dans la vision du réel qui a 
été  proposée  -,  soit  en  montrant  qu'une  autre  double  hiérarchie  vient  combattre  les  effets  de  la 
première. Par contre l'acceptation d'une double hiérarchie confirme généralement la structure du 
réel qui a été invoquée pour unir les deux séries. 
\bigskip
A  cet  égard,  les  tables  de  présence  et  d'absence,  qui  pourraient  être  considérées  comme  un  cas 
particulier de doubles hiérarchies limitées aux degrés o et i, peuvent, d'un autre point de vue, être 
considérées  comme  un  cas  très  général,  se  référant  à  des  liaisons  dont  la  structure  n'est  pas 
précisée et que l'observation doit permettre d'élaborer. 
\bigskip
L'argument  par  double  hiérarchie  est  à  la  base,  nous  semble-t-il,  de  certaines  techniques 
d'amplification bien connues. A preuve cet exemple donné par Quintilien : 
\bigskip
C'est  par  leurs  armes qu'on  nous  donne  à  juger  de  la  taille  des  anciens  héros,  témoin  le  bouclier 
d'Ajax et la lance Pélias d'Achille (2). » 
\bigskip
(1) Aristote, Rhétorique, II, chap. 23, XVI, 1399 a. 
(2) Quintilien, Vol. III, liv. VIII, chap. IV, § 24. 
\bigskip
P  461 :  « Une  autre  technique  consiste,  prenant  appui  sur  la  corrélation  entre  une  hiérarchie 
d'actes  et  celle  de  leurs  qualifications,  à  opérer  un  déplacement  de  la  seconde  hiérarchie  tout 
entière  ;  c'est  ainsi  pensons-nous  que  peuvent  s'interpréter  le  mieux,  dans  toute  leur  généralité, 
certains moyens décrits par les Anciens : 
\bigskip
Après  avoir  présenté  des  actes  particulièrement  atroces  sous  le  jour  le  plus  odieux,  nous  les 
atténuons  à  dessein,  pour  que  ceux  qui  les  ont  suivis  paraissent  plus  noirs.  C'est  ce  qu'a  fait 
Cicéron, disant dans titi passage bien connu : « Mais, pour l'accusé que je poursuis, ce sont des 
peccadilles » (1). 
\bigskip
Si le déplacement des qualifications s'effectue dans le sens du grossissement, il est normal que l'on 
ne parvienne plus à trouver de mots pour désigner les crimes les plus atroces : 
\bigskip
\bigskip
\bigskip
\bigskip
242 
\bigskip
C'est  une  indignité  de  jeter  dans  les  fers  un  citoyen  romain,  un  crime  de  le  battre  de  verges, 
presque un parricide de le mettre à mort ; comment appellerai-je l'action de le mettre en croix ? 
(2). 
\bigskip
L'une  des  deux  hiérarchies  semble  donc  incapable  de  suivre  l'autre.  On  peut  prétendre que  cette 
carence est définitive, et désigner comme l'inexprimable, l'incomparable, les termes qui dépassent 
un  certain  degré  de  la  hiérarchie  donnée  et  présentent,  pour  ainsi  dire,  des  valeurs  d'un  autre 
ordre. » 
\bigskip
(1) Quintilien, ibid., liv. VIII, chap. IV, § 19.  
(2) Ibid., liv. VIII, chap. IV, § 4. 
\bigskip
P  461-462 :  « Presque  tous  les  arguments  par  double  hiérarchie  peuvent  être  traités  comme 
arguments a fortiori : le dessein n'est pas alors de trouver la place exacte  d'un élément dans une 
hiérarchie  à  l'aide  d'une  autre  hiérarchie,  mais  de  déterminer  une  limite  a  quo.  Ainsi  dans 
l'argument  de  Leibniz  cité  plus  haut,  on  affirme  que  les  soins  que  Dieu  donnera  aux  hommes 
seront  au  moins  aussi  adéquats  que  ceux  donnés  aux  passereaux.  Si  les  dieux  ne  sont  pas 
omniscients,  à  plus  forte  raison  les  hommes  (3)  ;  les  sacrifices  que  s'impose  un  parent  éloigné 
devraient être a fortiori assumés par un parent plus proche (1). » 
\bigskip
(3) Aristote, Rhétorique, II, chap. 23, IV, 1397 b. 
(1) Cf. M. Proust, A la recherche du temps perdu, vol. 8 : Le côté de Guermantes, III, p. 234. 
\bigskip
P 462 : « On réservera cependant le nom d'argument a fortiori, au sens strict, à certains arguments 
où la limite est renforcée par une autre double hiérarchie dont elle fait aussi partie. Comme dans 
ce texte d'Isocrate : 
\bigskip
N'est-il pas honteux qu'un seul d'entre nous ait suffi autrefois pour sauver les villes des autres et 
qu'aujourd'hui  la  totalité  de  notre  peuple  soit  incapable  et  n'essaie  même  pas  de  sauver  notre 
propre patrie (2) ? 
\bigskip
Aujourd'hui les arguments a fortiori sont souvent énoncés plus discrètement : 
\bigskip
je crois qu'une grande puissance doit être magnanime. M comme, dans une certaine mesure, ce 
gouvernement est dans son tort, il devrait montrer plus de magnanimité (3). 
\bigskip
La  troisième hiérarchie  qui  entre  en  jeu,  et que  nous  appellerons  confirmative,  n'est  pas  dérivée 
terme  à  terme  de  la  première,  comme  ce  pourrait  être  le  cas  dans  des  enchaînements  de 
hiérarchies  tels  :  Dieux,  hommes  -  lois  divines,  lois  humaines  -obéissance  aux  lois  divines, 
obéissance  aux  lois  humaines.  Elle  ne  lui  est  donc  pas  entièrement  parallèle,  mais  jouit  d'une 
indépendance relative. S'il s'agit de fixer une conduite, on la rattachera à des éléments divers, tels 
cause, effets, conditions, qui permettent de constituer plusieurs doubles hiérarchies agissant dans 
le  même  sens.  Dans  l'exemple  d'Isocrate,  la  plus  grande  importance  du  but  poursuivi  et  la 
supériorité  des  moyens  dont  on  disposait,  viseront  à  accroître  la  honte  résultant  de  la 
confrontation des deux situations. » 
\bigskip
(2) Isocrate, Discours, t. II : Archidamos, § .54. 
(3) R. Crossman, Palestine Mission, with Speech delivered in the House of Coinmous, 1st July 
1946, p. 254. 
\bigskip
P  462-463 :  « Certaines  antithèses,  notamment  la  figure  appelée  contraire  dans  la  Rhétorique  à 
Herennius qui « étant donné deux choses opposées, emploie l'une à prouver l'autre brièvement et 
facilement (1) », ne sont rien d'autre que l'argument a fortiori. Voici l'un des exemples cités : 
\bigskip
\bigskip
\bigskip
243 
\bigskip
 
Celui que tu as courut ami perfide, comment penser qu'il pourra être ennemi loyal? » 
\bigskip
(1) Rhétorique à Herennius, liv. IV, § 25. 
\bigskip
P  463 :  « Ce  qui  incite  semble-t-il  à  y  voir  une  figure  c'est  le  balancement  de  la  phrase  ;  mais  il 
s'agit de figure argumentative au premier chef. 
\bigskip
Appliqués  au  discours  lui-même,  les  arguments  de  double  hiérarchie  pourront  servir  à  le  situer, 
soit par des liaisons de succession, soit par des liaisons de coexistence ; elles porteront sur ses buts, 
sur les moyens qu'il utilise, sur l'orateur dont il émane, sur l'auditoire auquel il s'adresse, lesquels 
éléments  peuvent  tous  faire  partie  de  hiérarchies.  L'une  des  principales  serait  le  classement  des 
auditoires  selon  leur  extension.  Il  n'est  pas  impossible  que  pareille  hiérarchie  se  présente 
spontanément à l'esprit des auditeurs, qu'elle influence leur jugement sur le discours et modifie ses 
effets. 
\bigskip
§ 77. ARGUMENTS CONCERNANT LES DIFFERENCES DE DEGRE ET D'ORDRE 
\bigskip
En examinant l'argument de double hiérarchie, nous avons insisté sur le fait que les hiérarchies qui 
lui  servent  de  fondement  peuvent  être  quantitatives  on  qualitatives  ,  il  arrive  même  que  l'une 
d'elles  soit  qualitative  mais  que  l'autre  soit  quantitative,  comme  dans  les  corrélations  établies  en 
physique, entre les couleurs et les longueurs d'onde, par exemple. » 
\bigskip
P  463-464 :  « Les  hiérarchies  quantitatives  ne  présentent  entre  leurs  termes  que  des  différences 
numériques,  des  différences  de  degré  ou  d'intensité,  sans  qu'il  y  ait  entre  un  terme  et  le  suivant 
une coupure due au fait que l'on passe à un autre ordre. » 
\bigskip
P  464 :  « L'importance  de  cette  distinction  entre  degré  et  ordre  est  bien  marquée  par  ce  mot  de 
Ninon  de  Lenclos  à  qui  lui  racontait  que  saint  Denis  décapité  aurait  parcouru  trois  kilomètres 
portant sa tête : « Il n'y a que le premier pas qui coûte. » La réponse est spirituelle parce qu'elle 
souligne la valeur éminente d'une différence d'ordre par rapport à une différence de degré. 
\bigskip
L'introduction  de  considérations  relatives  à  l'ordre,  qu'elles  résultent  de  l'opposition  entre  une 
différence  de  degré  et  une  différence  de  nature,  ou  entre  une  différence  de  modalité  et  une 
différence de principe, a pour effet de minimiser les différences de degré, d'égaliser plus ou moins 
les termes qui ne diffèrent entre eux que par l'intensité, et d'accentuer ce qui les sépare des termes 
d'un  autre  ordre.  Par  contre,  la  transformation  de  différences  d'ordre  en  différences  de  degré 
produit l'effet inverse ; elle rapproche les uns des autres des termes qui semblaient séparés par une 
borne infranchissable, et met en valeur les distances entre les degrés. 
\bigskip
Voici un texte où Cicéron reprend certaines idées stoïciennes : 
\bigskip
Il ne faut pas juger les mauvaises actions par leur résultat, mais ,par le vice qu'elles supposent. 
La  matière  de  la  faute  peut  être  plus  ou  moins  considérable,  mais  la  faute  en  elle-même...  ne 
comporte  ni  le  plus  ni  le  moins.  Qu'un  pilote  perde  un  vaisseau  chargé  d'or  ou  une  barque 
chargée de paille; il y aura quelque différence dans la valeur perdue, aucune dans l'impéritie du 
pilote... il en est de faire le mal comme de sortir des bornes : une fois en dehors, la faute est faite ; 
si loin que vous alliez au delà de la barrière, vous n'ajouterez rien au tort de l'avoir franchie (1). » 
\bigskip
(1) Cicéron, Paradoxa stoicorum, III-, § 20. 
\bigskip
P 464-465 : « Le refus de hiérarchiser les fautes d'après leurs conséquences, la décision de ne tenir 
compte  que  des  vices  du  sujet,  tend  à  établir  entre  les  actions  une  hiérarchie  axiologique 
caractérisée  par  une  coupure  brusque  entre  ce  qui  est  permis  et  ce  qui  est  interdit.  La  plus  ou 
\bigskip
\bigskip
\bigskip
244 
\bigskip
moins grande gravité de la faute est une considération sans importance : toutes appartiennent à un 
même ordre ; ce qui compte avant tout, c'est la qualité de la nature humaine révélée par l'acte en 
question. » 
\bigskip
P 465 : « Le passage suivant de la Troisième Philippique rappelle celui de Cicéron : 
\bigskip
Philippe, lui, s'emparait de Serrion et de Doriscos, il chassait vos troupes du fort de Serreion et du 
Mont-Sacré, où votre stratège les avait postées. Que faisait-il donc en agissant ainsi ? Car déjà il 
avait  juré  la  paix.  Qu'on  ne  me  dise  pas  :  «  Mais  qu'est-ce  que  ces  places  ?  »  ou  «  quelle 
importance  ont-elles  pour  nous  ?  »  Si  ces  places  sont  petites,  si  elles  sont  pour  vous  salis 
importance, c'est là une autre question ; mais le respect du serment, mais le droit, que l'infraction 
commise soit petite on grande, ont toujours même valeur (1).  
\bigskip
On voit que cette technique d'égalisation est souvent utilisée lorsque l'on craint que certaine chose, 
à ses degrés inférieurs, semble peu digne d'attention : pour y obvier on fait Participer ceux-ci à la 
valeur qui s'attacherait normalement aux degrés les plus élevés. En plaçant ainsi la question sur le 
terrain des principes, on ne l'apprécie plus uniquement d'un point de vue utilitaire. L'affirmation 
d'une distinction fondamentale s'oppose à la stricte application de l'argument pragmatique. 
\bigskip
Peut-être faut-il voir un emploi de cette technique d'égalisation dans certains procédés de défense : 
on reconnaîtra, on dévoilera une minime partie des faits, en escomptant qu'une différence de degré 
sera considérée, le cas échéant, comme moins grave qu'une différence de nature et que l'on sera à 
l'aise devant le reproche d'avoir menti, d'avoir passé sous silence. 
\bigskip
En transformant une différence de nature en différence de degré, on rapproche ce qui pouvait 
paraître relever d'ordres incommensurables. » 
\bigskip
(1) Démosthène, Harangues, t. II : Troisième Philippique, §§ 15, 16. 
\bigskip
P 466 : « Voici un passage significatif de Bergson : 
\bigskip
La différence est profonde [entre la science antique et la science moderne]. Elle est même radicale 
par un certain côté. Mais, du point de vue d'où nous l'envisageons, c'est une différence de degré 
plutôt que de nature. L'esprit humain a passé du premier genre de connaissance au second par 
perfectionnement  graduel,  simplement  en  cherchant  une  précision  plus  haute.  Il  y  a  entre  ces 
deux  sciences  le  même  rapport  qu'entre  la  notation  des  phases  d'un  mouvement  par  l'oeil  et 
l'enregistrement beaucoup plus complet de ces phases par la photographie instantanée (1). 
\bigskip
Pomponazzi  rejette  toute  distinction  d'ordre  entre  le  spirituel  et  le  matériel  et,  par  là,  l'un  des 
fondements de l'immortalité de l'âme en prétendant que la nature procède d'une façon graduelle, 
que déjà les formes inférieures, même végétales, ont une âme plus ou moins élaborée : 
\bigskip
Il y a des animaux intermédiaires entre les plantes et les animaux, comme les éponges marines, 
fixes comme les plantes mais sensibles à la manière des animaux. Il y a le singe, dont on ne sait 
s'il est bête on homme; il y a l'âme intellective intermédiaire entre le temporel et l'éternel (2). 
\bigskip
On  obtient  le  même  effet  au  moyen  d'une  hypothèse  évolutionniste,  qui  ne  peut  traiter  l'espèce 
humaine comme d'un autre ordre que le reste du règne animal. » 
\bigskip
(1) Bergson, L'évolution créatrice, p. 359. 
(2) Cf. E. Garin, L'umanesimo italiano, pp. 175-177. 
\bigskip
\bigskip
\bigskip
\bigskip
245 
\bigskip
P  466-467 :  « Lorsque  l'on  se  trouve  en  présence  de  deux  domaines  d'ordre  différent, 
l'établissement de degrés à l'intérieur de l'un d'eux a souvent pour but d'atténuer la coupure. On 
prépare ainsi la réduction d'une différence d'ordre à une différence de degré. La hiérarchisation à 
l'intérieur de l'un des domaines s'effectue, en effet, de manière à ce que son degré extrême f orme 
transition  entre  les  deux  domaines  :  ainsi,  le  certain  et  l'incertain  se  rejoignent  plus  aisément  à 
partir  du  moment  oit  il  y  a  des  degrés  à  l'intérieur  de  l'incertain  ;  de  même,  on  prépare  le 
rapprochement entre jugements de valeur et jugements de réalité en établissant une graduation à 
l'intérieur des jugements de valeur (1). Cette technique peut d'ailleurs être utilisée tantôt au profit 
de l'un tantôt au profit de l'autre des deux ordres : 
\bigskip
Les sciences de la nature ont crû un  bon  bout vers les sciences de l'esprit. A la suite de  quoi les 
différences sont peut-être plus de degré que de principe (2). » 
\bigskip
(1) Par exemple F. L. Polak, Kennen en Keuren in de Sociale Wetenschappen, pp. 95, 180. 
(2) Ibid., p. 171. 
\bigskip
P 467 : « Selon que l'on  prétend être en présence  d'une différence d'ordre  ou d'une différence de 
degré,  on  portera  ou  non  attention  à  ce  qui  aurait  pu  provoquer,  expliquer,  garantir  le  saut  d'un 
ordre  à  un  autre,  ou  tout  au  moins  en  témoigner.  Souvent  donc  les  arguments  relatifs  aux 
différences  d'ordre  préparent  on  supposent  des  considérations  sur  le  phénomène  qui  marque  la 
coupure : la mutation, l'émergence, rendront compte du bond d'un ordre à l'autre dans la chaîne 
évolutive ; la conversion religieuse fera passer l'individu de l'ordre de la nature à celui de la grâce. 
En général, cet événement-clé est obscur, imprévisible, irrationnel ; la réduction des différences de 
nature à des différences de degré tend à faire l'économie de pareils éléments, à cantonner l'esprit 
dans ce qui est connu, familier, rationnel. » 
\bigskip
P  467-468 :  « Parmi  les  suites,  celle  du  temps  qui  se  déroule  joue  un  rôle  très  important.  Les 
phénomènes auxquels cette suite sert de guide, prennent un aspect continu, homogène, et souvent 
aussi  quantifiable  :  durée,  accroissement,  vieillissement,  oubli,  perfectionnement,  peuvent  se 
quantifier  en  fonction  du  temps  écoulé.  Mais  on  découpe  souvent  les  phénomènes  successifs  de 
manière  à  les  rendre  hétérogènes.  Nous  avons  déjà  fait  allusion  à  ce  que  certaines  périodes 
historiques  sont  considérées  comme  des  essences,  dont  les  phénomènes  particuliers  ne  seraient 
que  la  manifestation  (3).  Ces  essences  jouent,  au  point  de  vue  qui  nous  occupe  ici,  le  rôle  de 
natures, de principes. C'est dire que, chaque fois que l'on utilisera pareilles essences, on sera enclin 
à accentuer le rôle des événements source ou témoin de la discontinuité : révolution, guerres, fait 
du  prince,  penseur  de  marque,  bref  tout  phénomène  capable  de  justifier  la  scission  entre  deux 
phases de l'histoire. Inversement, chaque fois que l'on renoncera à certaines essences, on réduira le 
rôle de ces événements. » 
\bigskip
(3) Cf. § 74 : L'acte et l'essence. 
\bigskip
P 468 : « Pour minimiser l'idée que l'on se fait d'un phénomène lié à une coupure, on sera amené, 
non  seulement  à  remplacer  la  différence  d'ordre  par  une  différence  de  degré,  mais  aussi  à 
introduire  de  nouvelles  différences  d'ordre  que  l'on  jugera  plus  importantes.  Luttant  contre  la 
crainte de la mort, Montaigne nous montre toute notre vie comme une succession de « saults » qui 
nous y mène, et dont les derniers moments ne sont même pas les plus pénibles : 
\bigskip
...  nous  ne  sentons  aucune  secousse,  quand  la  jeunesse  meurt  en  nous,  qui  est  en  essence  et  en 
verité une mort plus dure que n'est la mort entiere d'une vie languissante, et que n'est la mort de 
la vieillesse. D'autant que le sault n'est pas si lourd du mal estre au non estre, comme il est d'un 
estre doux et fleurissant à un estre penible et douloureux (1). » 
\bigskip
(1) Montaigne, Bibl. de la Pléiade, Essais, liv. I, chap. XX, p. 104. 
\bigskip
\bigskip
\bigskip
246 
\bigskip
 
P 468-469 : « En divisant la vie en plusieurs époques, qui meurent l'une après l'autre, il superpose 
à  l'image  de  la  mort,  graduelle  et  insensible,  une  division  en  ordres,  différente  de  l'opposition  « 
vie-mort  »,  et  diminue,  par  là,  la  coupure  que  cette  dernière  semble introduire.  Par  contre,  ceux 
qui  insistent  sur  l'importance  de  la  mort,  et  qui  voudraient  en  faire  le  centre  de  nos 
préoccupations, ne pourront qu'écarter toutes autres distinctions et hiérarchies comme n'étant que 
vanité : 
\bigskip
Ainsi l'homme, dira Bossuet, petit en soi et honteux de sa petitesse, travaille à s'accroître et à se 
multiplier dans ses titres, dans ses possessions, dans ses vanités : tant de fois comte, tant de fois 
seigneur,  possesseur  de  tant  de  richesses,  maître  de  tant  de  personnes,  ministre  de  tant  de 
conseils, et ainsi du reste : toutefois, qu'il se multiplie tant qu'il lui plaira, il ne faut toujours pour 
l'abattre  qu'une  seule  mort.  ...  il  ne  s'avise  jamais  de  se  mesurer  à  son  cercueil,  qui  seul 
néanmoins le mesure au juste (1). » 
\bigskip
(1) Bossuet, sermons, vol. II : Sur l'honneur, p. 173. 
\bigskip
P 469 : « Il semble résulter de ce qui précède, qu'il y a une opposition bien nette entre des séries 
quantitatives et des hiérarchies entre termes qui relèvent de deux ordres différents. Mais, en fait, il 
se peut que, à un moment donné, une différence purement quantitative entraîne le passage à des 
phénomènes  d'un  autre  ordre.  C'est  ainsi  que,  pour  reprendre  un  exemple  que  nous  avons  cité 
ailleurs  (2),  lors  de  la  discussion,  pendant  les  années  qui  ont  suivi  la  guerre,  du  plan  américain 
d'aide à l'Europe (plan Marshall), les promoteurs ont prétendu qu'une réduction des crédits de 25 
%  transformerait  ce  qui  était  conçu  comme  un  programme  de  reconstruction  en  un  programme 
d'assistance : un changement quantitatif entraînerait un changement de la nature même du plan. Il 
va  de  soi  que  cette  affirmation  tendait  à  obtenir  un  minimum  de  crédits,  en  dessous  duquel  les 
objectifs visés ne seraient plus atteints. 
\bigskip
Qu'un changement quantitatif puisse entraîner un changement de  nature, les raisonnements que 
les  Grecs  qualifiaient  de  sorite  l'ont  mis  depuis  longtemps  en  évidence  (3).  A  partir  d'un  certain 
moment les grains additionnés aux grains forment un tas, des cheveux arrachés l'un après l'autre 
transforment un homme chevelu en chauve. Mais à quel moment fixer la limite, difficile a établir et 
indispensable  ?  Il  n'y  a  pas  de  critère  objectif  à  cet  égard,  il  faut  une  décision  ;  quand  elle  sera 
prise,  la  coupure  acquerra  une  importance  que  la  seule  détermination  quantitative  ne  pourrait 
justifier. » 
\bigskip
(2)  Cf.  Ch.  Perelman  et  L.  Olbrechts-Tyteca,  Rhétorique  et  philosophie,  P.  35  (Logique  et 
rhétorique). 
(3) Cf. § 66 : L'argument de la direction. 
\bigskip
P 470 : « L'existence de certains concepts facilitera la coupure. Ainsi les aspects négatifs et positifs 
d'une  hiérarchie,  lorsqu'ils  sont  indiqués  par  un  terme  et  sa  négation  -  tels  tempérance-
intempérance,  tolérance-intolérance  -  seront  souvent  interprétés  comme  une  différence  d'ordre 
(1). 
\bigskip
Toute élaboration conceptuelle originale modifie d'une façon on de l'autre les hiérarchies admises, 
en ramenant une distinction d'ordre à une différence de degré ou, inversement, en remplaçant une 
hiérarchisation  par  une  autre,  jugée  plus  fondamentale.  Ces  façons  diverses  de  structurer  et  de 
restructurer le réel exercent des effets indéniables sur les évaluations et la manière de les fonder. 
\bigskip
(1) Cf. plus haut, dans § 76, l'argument des contraires. 
\bigskip
\bigskip
\bigskip
247 
\bigskip
CHAPITRE III LES RAISONS QUI FONDENT LA STRUCTURE DU RÉEL 
\bigskip
A) LE FONDEMENT PAR LE CAS PARTICULIER 
\bigskip
§ 78. L'ARGUMENTATION PAR L'EXEMPLE 
\bigskip
P 471 : « Dans les paragraphes qui suivent, nous analyserons les liaisons qui fondent le réel par le 
recours au cas particulier. Celui-ci peut jouer des rôles fort divers : comme exemple, il permettra 
une  généralisation  ;  comme  illustration,  il  étayera  une  régularité  déjà  établie;  comme  modèle,  il 
incitera à l'imitation. Nous examinerons successivement ces trois types d'arguments. 
\bigskip
L'argumentation par l'exemple implique - puisqu'on a recours à elle -quelque désaccord au sujet de 
la règle particulière que l'exemple est appelé à fonder, mais cette argumentation suppose un accord 
préalable sur la possibilité même d'une généralisation à partir de cas particuliers, ou tout au moins 
sur les effets de l'inertie (1). Ce dernier accord pourra être mis en doute à un moment donné, mais 
ce  n'est  pas  à  l'aide  d'une  argumentation  par  l'exemple  que,  à  ce  niveau-là  de  la  discussion,  on 
militera.  Aussi  le  problème  philosophique  de  l'induction  ne  relève-t-il  point  de  notre  présent 
propos. » 
\bigskip
(1) CI. § 27 : Accords propres à chaque discussion. 
\bigskip
P 471-472 : « Quand un phénomène est-il introduit dans le discours à titre d'exemple, c'est-à-dire 
comme  l'amorce  d'une  généralisation  ?  En  faveur  de  quelle  règle  l'exemple  cité  constitue-t-il  un 
argument ? Voici les deux questions qui se posent tout naturellement. » 
\bigskip
P  472 :  « Toute  description  d'un  phénomène  ne  doit  pas  être  considérée  comme  devant  servir 
d'exemple.  Pour  certains  théoriciens  de  l'histoire,  celle-ci  aurait  justement  pour  caractère  de 
s'attacher à ce qui, dans les événements étudiés, est unique, en raison de la place particulière qu'ils 
occupent  dans  une  série  dont  l'ensemble  forme  nu  processus  continu,  caractérisé  par  ces 
événements eux-mêmes. 
\bigskip
En  sciences,  les  cas  particuliers  sont  traités  soit  comme  des  exemples  devant  mener  à  la 
formulation  d'une  loi  ou  la  détermination  d'une  structure,  soit  comme  échantillons,  c'est-à-dire 
illustration  d'une  loi  ou  d'une  structure  reconnues.  En  droit,  invoquer  le  précédent,  c'est  traiter 
celui-ci comme un exemple fondant une règle, nouvelle au moins sous certains de ses aspects (1). 
Par  ailleurs  une  disposition  juridique  est  souvent  envisagée  comme  un  exemple  de  principes 
généraux, reconnaissables à partir de cette disposition. » 
\bigskip
(1) Cf. § 2 : La règle de justice. 
\bigskip
P  472-473 :  « Dans  bien  des  circonstances,  l'orateur  manifeste  clairement  son  intention  de 
présenter  les  faits  comme  des  exemples;  mais  il  n'en  est  pas  toujours  ainsi.  Certaines  revues 
américaines  se  plaisent  à  raconter  la  carrière  de  tel  grand  industriel,  de  tel  homme  politique  ou 
d'une  étoile  de  cinéma,  sans  en  tirer  explicitement  de  leçon.  Ces  faits  sont-ils  simplement  une 
contribution  à  l'histoire  ou  à  la  petite  histoire,  servent-ils  d'exemples  pour  une  genéralisation 
spontanée,  sont-ce  des  illustrations  de  quelques  recettes  bien  connues  pour  réussir  socialement, 
veut-on  proposer  les  héros  de  ces  récits  comme  modèles  prestigieux,  et  contribuer  ainsi  à 
l'éducation  du  publie  ?  Rien  ne  permet  de  le  dire  avec  certitude;  probablement  le  récit  doit-il 
remplir,  et  remplit-il  effectivement,  pour  différentes  catégories  de  lecteurs,  tous  ces  rôles 
simultanément. » 
\bigskip
P  473 :  « Néanmoins,  quand  des  phénomènes  particuliers  sont  évoqués  les  uns  à  la  suite  des 
autres,  surtout  s'ils  offrent  quelque  similitude,  on  sera  enclin  à  y  voir  des  exemples,  alors  que  la 
\bigskip
\bigskip
\bigskip
248 
\bigskip
description d'un phénomène isolé eût été prise plutôt pour une simple information. Un procureur, 
comme personnage de théâtre, peut passer pour un homme particulier, non représentatif ; si, dans 
la même pièce, deux procureurs sont mis en scène, leur comportement paraîtra exemplatif de celui 
de toute une profession (1). La seule mise d'un événement au pluriel est significative à cet égard : 
\bigskip
C'est  grâce  à  lui  [au  pluriel],  dit  Caillois  dans  un  intéressant  commentaire,  que  s'effectue  la 
promotion  poétique,  la  généralisation  qui,  donnant  à  l'événement  inimaginable  une  valeur 
d'archétype,  lui  permet  de  prendre  place  dans  les  annales  humaines.  L'auteur  n'agit  pas 
autrement quand il parle des Colisées, des Castilles ou des Florides, quand il écrit que «la terre 
contait ses rois René » (Vents, IV, 5) ou quand il multiplie sans la nommer l'infiniment unique île 
de Pâques (Vents, IV, 2) (2). 
\bigskip
Pour  nous  assurer que  nous  sommes en  présence  d'une  argumentation  par  l'exemple,  rien  de  tel 
cependant  que  les  exposés  où  elle  se  présente  en  forme.  Le  cas  extrême  serait  la  phrase  à  cinq 
membres des anciens logiciens indiens : 
\bigskip
Le mont est flambant 
Parce que fumant 
Tout ce qui est fumant est flambant, de même que le foyer; 
De même celui-ci, 
Donc ainsi (3). 
\bigskip
Lorsque, par contre, il ne tire lui-même aucune conclusion des faits qu'il allègue, nous ne sommes 
jamais assurés que l'orateur souhaite que ses énoncés soient considérés comme exemples. » 
\bigskip
(1) Cf. M. Aymé, La tête des autres. 
(2) R. Caillois, Poétique de Saint-John Perse, p. 152. 
(3) Annambhatta, Le compendium des topiques, pp. 128 et suiv. 
\bigskip
P  474 :  « Schopenhauer  mentionne  un  stratagème  consistant  à  tirer  de  ce  que  dit  l'orateur 
certaines  conclusions  qui  vont  à  l'encontre  de  sa  pensée  (1)  :  traiter  comme  exemple  ce  que 
l'orateur n'entendait point ainsi, peut être manière de l'embarrasser fortement. 
\bigskip
L'emploi  de  l'argumentation  par  l'exemple,  bien  qu'ouvertement  proclamé,  tend  souvent  à  nous 
faire passer de celui-ci à une conclusion également particulière, sans qu'aucune règle soit énoncée : 
c'est ce qu'on appelle l'argumentation du particulier ait particulier : 
\bigskip
Il faut faire des préparatifs militaires contre le Grand Roi et ne pas le laisser asservir l'Egypte; 
en effet Darius ne passa point en Europe avant d'avoir pris l'Egypte, et, quand il l'eut prise, il y 
passa  ;  et,  plus  tard,  Xerxès  n'entreprit  rien  avant  de  l'avoir  conquise,  et,  quand  il  s'en  fut 
emparé, il passa en Europe, de sorte que, si le prince dont il s'agit la prend, il passera en Europe; 
aussi ne faut-il pas le laisser faire (2). 
\bigskip
Tout comme le passage de l'exemple à la règle, cette forme de raisonnement fait appel à l'inertie. 
Les  notions  utilisées  pour  décrire  le  cas  particulier  servant  d'exemple  jouent  d'ailleurs 
implicitement  le  rôle  de  la  règle  permettant  le  passage  d'un  cas  a  un  autre.  Ce  curieux 
raisonnement de S. Weil pourra nous éclairer : 
\bigskip
De  même  que  la  seule  manière  de  témoigner  du  respect  à  celui  qui  souffre  de  la  faim  est  de  lui 
donner à manger, de même le seul moyen de témoigner du respect à celui qui s'est mis hors la loi 
est de le réintégrer dans la loi en le soumettant au châtiment qu'elle prescrit (3). » 
\bigskip
\bigskip
\bigskip
\bigskip
249 
\bigskip
(1)  Schopenhauer,  éd.  Brockhaus.  vol.  6  :  Parerga  und  Paralipomena,  Zweiter  Band,  Zur  Logik 
und Dialektik, § 26 (Achtes Stratagem), p. 31. 
(2) Aristote, Rhétorique, liv. II, chap. 20, 1393 b. 
(3) S. Weil, L'enracinement, p. 25. 
\bigskip
P 474-475 : « La règle, implicite dans cette argumentation, est que le seul moyen de témoigner du 
respect  à  un  être  est  de  lui  donner  ce  qui  lui  manque;  mais  tandis  que  l'exemple  de  l'affamé  ne 
prête pas à contestation, parce que points de vue objectif et subjectif coïncident, vu que l'affamé « 
souffre  de  la  faim  »,  l'application  de  la  règle  au  cas  du  criminel  fait  prévaloir  le  point  de  vue 
objectif, sans se préoccuper outre mesure des désirs de celui à qui va notre sollicitude. » 
\bigskip
P  475 :  « La  critique  de  cette  argumentation  du  particulier  au  particulier,  qui  est  caractéristique 
des dialogues socratiques, sera centrée sur le matériel conceptuel grâce auquel se fait le passage de 
l'une à l'autre des situations envisagées. 
\bigskip
Quelle  que  soit  la  manière  dont  l'exemple  est  présenté,  dans  quelque  domaine  que  se  déroule 
l'argumentation,  l'exemple  invoqué  devra,  pour  être  pris  comme  tel,  jouir  du  statut  de  fait,  au 
moins provisoirement; le grand avantage de son utilisation est de porter l'attention sur ce statut. 
Ainsi la plupart des propos d'Alain partent d'un récit concret que l'auditeur n'a aucune raison de 
mettre en doute. Le rejet de l'exemple, soit parce qu'il est contraire à la vérité historique, soit parce 
que  l'on  peut  opposer  des  raisons  convaincantes  à  la  généralisation  proposée,  affaiblira 
considérablement l'adhésion à la thèse que l'on voulait promouvoir. En effet, le choix de l'exemple, 
en  tant  qu'élément  de  preuve,  engage  l'orateur,  comme  une  espèce  d'aveu.  On  a  le  droit  de 
supposer que la solidité de la thèse est solidaire de l'argumentation qui prétend l'établir. » 
\bigskip
P  475-476 :  « Quelle  est  la  généralisation  qui  peut  être  tirée  de  l'exemple  ?  A  cette  question  se 
rattache étroitement celle de savoir quels sont les cas qui peuvent être considérés comme exemples 
de  la  même  règle.  En  effet,  c'est  par  rapport  à  une  certaine  règle  que  des  phénomènes  sont 
interchangeables et, d'autre part, l'énumération de ces derniers permet de dégager le point de vue 
auquel ils  ont  été  assimilés  l'un  à  l'autre.  C'est  la raison  pour  laquelle,  quand  il  s'agit  de  clarifier 
une règle aux cas d'application variables, il est utile d'en fournir des exemples aussi différents que 
possible, car on indique de cette façon que, en l'occurrence, ces différences n'importent pas. Ainsi 
dans ce passage de Berkeley : 
\bigskip
J'observe  en  outre  que  le  péché  ou  turpitude  morale  ne  consiste  pas  dans  l'action  physique 
extérieure  ou  mouvement,  c'est  l'écart  intérieur  de  la  volonté  hors  des  lois  de  la  raison  et  de  la 
religion. C'est clair, puisque tuer un ennemi à la bataille et mettre légalement à mort un criminel 
ne sont pas considérés comme péchés; pourtant l'acte extérieur est exactement le même que dans 
le cas du meurtre (1). » 
\bigskip
(1)  Berkeley,  Œuvres  choisies,  t.  II:  Les  trois  dialogues  entre  Hylas  et  Philonous,  Troisième 
dialogue, p. 157. 
\bigskip
P 476 : « En multipliant les exemples, Berkeley précise sa pensée comme par un commentaire. La 
systématisation  de  ce  procédé  mène  aux  règles  classiques  concernant  la  variation  des  conditions 
dans  l'induction,  dont  l'application  peut  aboutir  à  dégager  un  principe  d'une  portée  tout  à  fait 
générale. Ainsi le principe du levier peut être employé sous une telle variété de formes qu'à peine y 
aurait-il une caractéristique physique commune à chacune d'elles (2). 
\bigskip
Au lieu de multiplier seulement des exemples différents, on renforce parfois l'argumentation par 
l'exemple au moyen d'arguments de double hiérarchie, ce qui permet de raisonner a fortiori. C'est 
ce que nous appellerons le recours à l'exemple hiérarchisé : 
\bigskip
\bigskip
\bigskip
\bigskip
250 
\bigskip
[Tous les peuples honorent les sages] : par exemple, les Pariens ont honoré Archiloque nonobstant 
ses  diffamations  ;  et  les  Chiotes  Homère,  qui  pourtant  n'était  pas  leur  concitoyen;  et  les 
Mytiléniens Sappho, bien que ce fût une femme ; et les Lacédémoniens Chilon... bien qu'ils eussent 
très peu de goût pour les lettres (3)... 
\bigskip
Il semble que Whately, dans l'argument de « l'approche progressive » ne recommande autre chose 
que le recours à l'exemple hiérarchisé (4). » 
\bigskip
(2) Cf. à ce propos M. Polanyi, The Logis of liberty, p. 21.  
(3) Aristote, Rhétorique, II, chap. 23, X, 1398 b. 
(4) Whately, Elements of Rhetoric, Part 1, chap. II, p. 60. 
\bigskip
P  476-477 :  « Le  choix  de  l'exemple  le  plus  probant  parce  que  sa  réalisation  est  le  plus  difficile, 
peut donner lieu à caricature. Si, pour prouver que les chagrins peuvent faire blanchir en une nuit 
les  cheveux  de  certaines  victimes  on  raconte  que  cet  accident  peu  commun  est  arrivé  à  un 
marchand qui se désespérait de la perte de ses marchandises en mer, et que c'est sa perruque qui a 
subitement  grisonné  (1),  on  obtient  lin  effet  qui  relève  du  comique  de  l'argumentation.  Bien  des 
histoires marseillaises ne sont que des exemples hiérarchisés, que l'on veut trop convaincants. » 
\bigskip
(1) Cité d'après Ch. Lalo, Esthétique du rire, pp. 159-160. 
\bigskip
P 477 : « Remarquons à ce propos que les exemples interagissent, en ce sens que la mention d'un 
exemple  nouveau  modifie  la  signification  des  exemples  déjà  connus;  elle  permet  de  préciser  le 
point  de  vue  sous  lequel  les  faits  antérieurs  devaient  être  considérés.  En  droit  notamment,  alors 
que  l'on  réserve  parfois  le  nom  de  précédent  à  la  première  décision  prise  selon  une  certaine 
interprétation  de  la  loi,  la  portée  de  ce  jugement  peut  ne  se  dégager que  peu  à  peu,  à  la  suite  de 
décisions  ultérieures.  Aussi  le  fait  de  se  contenter  d'un  seul  exemple  dans  l'argumentation 
semble-t-il indiquer que l'on ne perçoit aucun doute quant à la façon de le généraliser. La situation 
est à peu près la même, à ce point de vue, lorsque l'on mentionne en bloc, à l'aide d'une formule 
unique,  telle  que  «  on  voit  souvent  que...  »  des  cas  nombreux.  Sans  doute  peut-on  les  présumer 
quelque peu différents l'un de l'autre mais, en vue de la généralisation, ils sont traités comme un 
exemple unique. La multiplication des cas non différenciés est, par contre, importante lorsque, au 
lieu de viser la généralisation, on cherche à déterminer la fréquence d'un événement, et à conclure 
à  une  certaine  probabilité  que  l'on  aura  de  l'observer  ultérieurement.  Ici  aussi  d'ailleurs  le 
caractère indifférencié des événements suppose cependant une variabilité des conditions ; aussi le 
choix  des  cas  en  observation  devra-t-il  être  fait  de  telle  façon,  que  l'on  soit  assuré  du  caractère 
représentatif des échantillons prélevés sur le réel. » 
\bigskip
P 478 : « Dans beaucoup d'énoncés, notamment dans le passage de Berkeley cité plus haut, un rôle 
essentiel  est  joué  également  par  le  cas  invalidant,  l'exemplum  in  contrarium,  qui  empêche  une 
généralisation indue, en montrant qu'elle est incompatible avec lui, et qui indique donc dans quelle 
direction seulement la généralisation est permise. 
\bigskip
C'est  l'infirmation  de  la  règle,  par  le  cas  invalidant,  et  le  rejet  ou  la  modification  subséquents  de 
celle-ci, qui selon Karl Popper, fournira le seul critère rendant possible le contrôle empirique d'une 
loi naturelle (1). 
\bigskip
Mais le cas invalidant, même incontesté, aboutit-il toujours au rejet de la loi ? Oui, sans doute, si 
l'on  entend  par  là  un  énoncé  applicable  à  un  ensemble  de  cas,  englobant  également  le  cas 
invalidant. Cela suppose, à la limite, que celui-ci était prévisible avant la formulation de la règle, ce 
qui n'aurait aucun sens. En réalité, un cas particulier, observé, ne peut jamais être en contradiction 
absolue avec un jugement dont l'universalité est empirique. Il ne peut que le renforcer on l'affaiblir 
(2). La loi pourra toujours être maintenue, en lui attribuant une portée légèrement  différente, qui 
tiendrait compte du nouveau cas. 
\bigskip
On  pourra  également  la  maintenir  en  restreignant  son  champ  d'application,  en  recourant,  par 
exemple, à la notion d'exception : la relation entre les événements liés par la loi cesse, comme en 
grammaire  ou  en  linguistique,  d'être  absolue.  Parfois  on  cherchera  à  remplacer  une  loi 
déterministe par une corrélation plus ou moins forte. » 
\bigskip
(1) Karl Popper, Logik der Forschung, spécialement les pages 12 à 14. 
(2) Cf. F. Waissmann, Verifiability, dans A. FLEw, Essays on Logic and Language, p. 125. 
\bigskip
P  478-479 :  « Ces  deux  solutions  supposent  admis,  et  même  théoriquement  dénombrables,  les 
événements  qui  exigent  des  aménagements  ou  des  adoucissements  de  la  loi.  Il  faudra  trouver 
d'autres solutions quand ce dénombrement ne peut être imaginé. Dans ces cas, on laissera souvent 
subsister la règle, mais on précisera les catégories d'événements auxquels elle n'est pas applicable. 
Mentionnons, comme procédé de cette espèce, le recours à la notion de miracle. L'existence du fait 
miraculeux  n'entraîne  pas  la  modification  de  la  loi  naturelle  ;  tout  au  contraire,  pour  qu'il  y  ait 
miracle,  il  faut  que  le  fait  et  la  loi  coexistent  chacun  dans  son  domaine.  Une  autre  technique 
consistera à transformer la règle qui pourrait être menacée  en règle conventionnelle. C'est ce que 
l'on  tente  de  faire  lorsqu'on  considère  le  déterminisme  comme  une  règle  de  méthode  et  non 
comme une loi scientifique (1), ou lorsque l'on établit des présomptions légales. » 
\bigskip
(1) Cf. F. Kaufmann, Methodology of the social sciences. 
\bigskip
Une  bonne  part  de  l'argumentation  consiste  à  amener  les  auditoires  à  penser  le  fait  invalidant, 
c'est-à-dire à reconnaître que les faits qu'ils admettent contreviennent à des règles qu'ils admettent 
aussi. Certaines expériences d'Eliasberg nous enseignent qu'il y a interaction entre la perception de 
faits  invalidants  et  la  conscience  de  la  règle.  Une  enfant  doit  trouver  des  cigarettes  placées  sous 
certaines cartes (bleues) ; lorsque se dessine une tendance à choisir les cartes bleues, on introduit 
une épreuve où il n'y a pas de cigarettes sous  l'une de ces cartes. La règle est amenée dès lors au 
niveau de la conscience claire et l'enfant ne tarde pas à la formuler (2). On ne s'étonnera donc pas 
qu'il soit possible, dans l'argumentation, de se servir de cas invalidants non seulement pour faire 
rejeter  la  règle,  mais  aussi  pour  la  dégager.  Ce  sera  notamment  le  cas  en  droit,  où  des  lois 
concernant l'exception font seules connaître une règle qui, par ailleurs, n'a jamais été énoncée. » 
\bigskip
(2)  P.  GUILLAUME,  Manuel  de  psychologie,  p.  274.  (À.  aussi  W.  G.  Eliasberg,  Speaking  and 
Thinking, dans Symposium : Thinking and Speaking, edited by G. Révész, pp. 98-102. 
\bigskip
P  479-480 :  « Dans  l'argumentation  par  l'exemple,  le  rôle  du  langage  est  essentiel.  Quand  deux 
phénomènes sont subsumés sous un même concept, leur assimilation semble résulter de la nature 
même  des  choses,  tandis  que  leur  différenciation  semble  nécessiter  une  justification.  C'est 
pourquoi,  sauf  dans  les  disciplines  où  l'usage  de  concepts  est  concomitant  d'une  technique  qui 
précise  leur  champ  d'application,  ceux  qui  argumentent  adapteront  souvent  les  notions  utilisées 
aux  besoins  de  leur  exposé.  L'argumentation  par  l'exemple  fournit  un  cas  éminent  où  le  sens  et 
l'extension des notions sont influencés par les aspects dynamiques de leur emploi. D'ailleurs cette 
adaptation,  cette  modification  des  notions  paraît  le  plus  souvent  si  naturelle,  si  conforme  aux 
besoins de la situation, qu'elle passe à peu près complètement inaperçue. » 
\bigskip
P  480 :  « L'utilisation  du  langage  pour  l'assimilation  de  cas  divers  joue  un  rôle  d'autant  plus 
important que le souci de subsumer les exemples sous une même règle, sans modifier celle-ci, est 
plus  puissant.  Ce  sera  certainement  le  cas  en  droit.  L'assimilation  de  nouveaux  cas  àl'occasion 
d'une décision judiciaire, n'est pas simplement un passage du général au particulier, elle contribue 
également au fondement de la réalité juridique, c'est-à-dire des normes, et nous savons déjà que de 
\bigskip
\bigskip
\bigskip
252 
\bigskip
nouveaux  exemples  réagissent  sur  les  anciens,  modifiant  leur  signification.  On  a  souligné  avec 
raison  que,  grâce  àce  que  l'on  a  appelé  la  Projection,  cette  assimilation  de  cas  nouveaux,  non 
prévisibles au moment où la loi fut élaborée, ou non pris en considération, pouvait se faire assez 
aisément,  sans  le  recours  à  aucune  technique  de  justification  (1).  Le  langage  précède  souvent  le 
juriste; à son tour la décision du juriste - car le langage lui facilite la tâche mais ne lui impose pas 
de décision - pourra réagir sur le langage, faire notamment que  deux mots, qui auraient pu, à un 
moment donné, être considérés comme homonymes, seront interprétés comme relevant d'un seul 
concept. » 
\bigskip
(1) Cf. notamment R. L. Drilsma, De woorden der wet of de wil van de wetgever, pp. 116 et suiv. 
\bigskip
§ 79. L'ILLUSTRATION 
\bigskip
P 481 : « L'illustration diffère de l'exemple en raison du statut de la règle qu'ils servent à appuyer. 
\bigskip
Tandis  que  l'exemple  était  chargé  de  fonder  la  règle,  l'illustration  a  pour  rôle  de  renforcer 
l'adhésion à une règle connue et admise, en fournissant des cas particuliers qui éclairent l'énoncé 
général,  montrent  l'intérêt  de  celui  -ci  par  la  variété  des  applications  possibles,  augmentent  sa 
présence  dans  la  conscience.  S'il  y  a  des  situations  où  l'on  peut  hésiter  quant  à  la  fonction  que 
remplit tel cas particulier introduit dans une argumentation, la distinction proposée nous semble 
cependant  importante  et  significative,  car,  le  rôle  de  l'illustration  étant  différent  de  celui  de 
l'exemple,  son  choix  sera  soumis  à  d'autres  critères.  Alors  que  l'exemple  doit  être  incontestable, 
l'illustration,  dont  ne  dépend  pas  l'adhésion  à  la  règle,  peut  être  plus  douteuse,  mais  elle  doit 
frapper vivement l'imagination pour s'imposer à l'attention. 
\bigskip
Aristote déjà avait distingué deux emplois de l'exemple suivant que l'on dispose ou non de principe 
d'ordre général. (Usage comme élément d'induction, usage comme témoignage.) Mais, d'après lui, 
le rôle des cas particuliers serait différent selon qu'ils précèdent ou suivent la règle à laquelle ils se 
rapportent. Ce qui ferait que : 
\bigskip
si  on  les  place  en  tête,  il  faut  nécessairement  en  produire  plusieurs;  en  épilogue,  même  un  seul 
suffit; car un témoin honnête, fût-il seul, est efficace (1). » 
\bigskip
(1) Aristote, Rhétorique, liv. II, chap. 20, 1394 a. 
\bigskip
P  481-482 :  “L'ordre  du  discours  n'est  cependant  pas  un  facteur  essentiel. Les  exemples  peuvent 
suivre  la  règle  qu'ils  doivent  prouver,  les  illustrations  d'une  règle  parfaitement  admise  peuvent 
précéder  son  énoncé  ;  tout  au  plus  l'ordre  incitera-t-il  à  considérer  un  fait  comme  exemple  ou 
comme illustration - et Aristote a raison d'avertir que l'exigence de l'auditeur sera plus grande dans 
la première interprétation. » 
\bigskip
P 482 : « Bacon, soulignant très fortement qu'il ne s'agit pas d'une question concernant l'ordre du 
discours  mais  bien  son  contenu,  affirme,  lui,  que  les  exemples  doivent  être  détaillés  dans  leur 
usage  inductif,  parce  que  les  circonstances  peuvent  jouer  un  rôle  capital  dans  le  raisonnement, 
tandis  que  dans  leur  usage  «  servile »  ils  peuvent  être  rapportés  succinctement  (1).  Sur  ce  point 
nous  ne  suivrons  pas  Bacon,  car  l'illustration,  visant  à  donner  la  présence,  devra  parfois  être 
développée  et  contenir  des  détails  frappants  et  concrets,  dont  au  contraire  l'exemple  sera 
prudemment  dépouillé  pour  éviter  que  la  pensée  ne  soit  distraite  ou  ne  dévie  du  but  que  se 
propose  l'orateur.  L'illustration  risque  beaucoup  moins  que  l'exemple,  d'être  mal  interprétée, 
puisque nous sommes guidés par la règle, connue et parfois très familière. 
\bigskip
Whately dit très nettement que certains exemples ne sont pas introduits pour prouver, mais pour 
rendre  clair,  «  for  illustration  »(2).  Il  discute  à  ce  propos  un  passage  du  De  Officiis  où  Cicéron 
soutient  que  rien  ne  peut  être  expédient  qui  soit  déshonorant,  et  donne  en  exemple  le  dessein 
\bigskip
\bigskip
\bigskip
253 
\bigskip
attribué à Thémistocle de brûler la flotte alliée, dessein qui, selon Cicéron, et contrairement à l'avis 
d'Aristide, n'eût pas été expédient, et cela parce qu'il était injuste (3). Whately remarque que cette 
dernière  affirmation  aurait  constitué  une  pétition  de  principe,  si  on  y  voyait  un  exemple  devant 
fonder la règle, puisqu'il la présuppose; mais qu'il n'en est rien s'il s'agit d'une application destinée 
à illustrer sa portée. » 
\bigskip
(1) Bacon, Of the advancement of learning, liv. II, XXIII, 8, p. 197. 
(2) Whately, Elements of Rhetoric, Part 1, chap. III, p. 78. 
3) Cicéron, De Officiis liv. III, XI, § 49. 
\bigskip
P  482-483 :  « Bien  que  subtile,  la  nuance  entre  exemple  et  illustration  n'est  pas  négligeable,  car 
elle permet de comprendre que, non  seulement le cas particulier ne sert pas toujours à fonder la 
règle, mais que parfois la règle est énoncée pour venir à l'appui des cas particuliers qui paraissent 
devoir  la  corroborer.  Dans  leurs  contes  fantastiques,  Poe  et  Villiers  de  l'Isle-Adam  commencent 
souvent  leur  récit  par  l'énoncé  d'une  règle,  dont  celui-ci  ne  serait  ensuite qu'une  illustration  :  ce 
procédé vise à renforcer la crédibilité des événements. » 
\bigskip
P 483-484  : « Quand, au début de la deuxième partie du Discours de la Méthode, Descartes s'avise 
de considérer que  
\bigskip
souvent il n'y a pas tant de perfection dans les ouvrages composés de plusieurs pièces, et faits de 
la main de divers maîtres, qu'en ceux auxquels un seul a travaillé,  
\bigskip
il fait suivre cet énoncé d'une énumération de cas particuliers. Le bâtiment construit par un seul 
architecte  est  plus  beau,  une  ville  plus  ordonnée  ;  une  constitution,  œuvre  d'un  seul  législateur, 
tout  comme  la  vraie  religion,  «  dont  Dieu  seul  a  fait  les  ordonnances  »est  incomparablement 
mieux réglée ; les raisonnements d'un homme de bon sens, concernant les choses qui ne sont que 
probables,  sont  plus  proches  de  la  vérité  que  la  science  des  livres;  les  jugements  de  ceux  qui 
n'auraient été conduits que par la raison, dès leur naissance, seraient plus purs et plus solides que 
ceux  des  hommes  gouvernés  par  plusieurs  maîtres.  D'après  A  Gilson  (1),  Descartes  donne  ces 
exemples pour soutenir sa proposition de la supériorité de ce qui a été fait par un seul et justifier le 
dessein  qu'il  a  conçu  de  reconstruire,  à  partir  des fondements  qu'il  en  propose,  tout  le  corps  des 
sciences. Mais les différents cas cités constituent-ils tous des exemples ? A les regarder de près, il 
semble plutôt que les deux derniers soient des illustrations d'une règle déjà établie au moyen des 
exemples qui les précèdent; en effet, si l'idée qu'ils avaient du beau, de l'ordonné, du systématique, 
permettait  aux  contemporains  de  Descartes  d'admettre  la  valeur  de  ses  réflexions  concernant 
l'édifice,  la  ville,  la  constitution  on  la  religion,  ses  deux  dernières  affirmations  étaient  nettement 
paradoxales  et  ne  pouvaient  être  envisagées  avec  quelque  faveur  que  si  l'on  voyait  en  elles  des 
illustrations d'une règle admise, car elles supposent une conception et un critère de la vérité et de 
la  méthode  qui  constituent  l'originalité  de  la  pensée  cartésienne.  Dans  une  énumération,  les  cas 
particuliers  qui  visent  à  étayer  une  règle  ne  jouent  pas  tous  le  même  rôle,  car,  si  les  premiers 
doivent être indiscutables, pour peser de tout leur poids dans la controverse, les suivants jouissent 
déjà  du  crédit  accordé  aux  précédents,  et  les  derniers  peuvent  ne  servir  que  d'illustrations.  Ceci 
explique  non  seulement  que  tous  les  cas  ne  soient  pas  sur  le  même  plan,  et  que  l'ordre  de  leur 
présentation ne soit pas réversible, mais aussi que le passage de l'exemple à l'illustration s'effectue 
bien des fois d'une façon insensible,  et que des controverses soient possibles quant à la façon de 
comprendre et de qualifier l'usage de chaque cas particulier et ses rapports avec la règle. » 
\bigskip
(1) Descartes, Discours de la méthode, éd. Gilson, p. 55, n. 1. 
\bigskip
P 484-485 : « Parce que l'illustration vise à accroître la présence, en concrétisant à l'aide d'un cas 
particulier  une  règle  abstraite,  on  a  souvent  tendance  à  y  voir  une  image, «  a  vivid  picture  of  an 
abstract  matter  »  (1).  Or  l'illustration  ne  tend  pas  à  remplacer  l'abstrait  par  le  concret,  ni  à 
\bigskip
\bigskip
\bigskip
254 
\bigskip
transposer  les  structures  dans  un  autre  domaine  comme  le  ferait  l'analogie  (2).  Elle  est 
véritablement  un  cas  particulier,  elle  corrobore  la  règle,  qu'elle  peut  même,  comme  dans  le 
proverbe,  servir  à  énoncer  (3).  Ce  qui  est  vrai,  c'est  que  l'illustration  est  souvent  choisie  pour  le 
retentissement  affectif qu'elle  peut  avoir.  Celle  qu'utilise  Aristote  dans  le  passage  ci-après  offrait 
sans  doute  ce  caractère  ;  il  s'agit  pour  lui  d'opposer  le  style  périodique  au style  coordonné,  dont 
l'inconvénient est de n'avoir pas de fin en lui-même : 
\bigskip
... or il n'est personne qui ne désire voir nettement la fin en tout. C'est ce qui explique qu'arrivés 
aux bornes du stade où l'on tourne, les coureurs halètent et succombent, tandis qu'auparavant, 
tant qu'ils avaient le but sous les yeux, ils ne sentaient pas leur fatigue (1). » 
\bigskip
(1) R. 11. Thouless, How to think straight, p. 103.  
(2) Cf. § 82 Qu'est-ce que l'analogie.  
(3) Cf. § 40 Forme du discours et communion avec l'auditoire. 
(1) Aristote, Rhétorique, III, chap. IX, 1, § 2, 1409 a.  
\bigskip
P  485 :  « Très  souvent  l'illustration  aura  pour  objet  de  faciliter  la  compréhension  de  la  règle,  à 
l'aide  d'un  cas  d'application  indiscutable.  C'est  le  rôle  qu'elle  remplit fréquemment  chez  Leibniz, 
comme dans le passage que voici : 
\bigskip
... il faut qu'il [le mal moral] ne soit admis ou permis, qu'en tant u'il est regardé comme une suite 
certaine  d'un  devoir  indispensable  :  e  sorte  que  celuy  qui  ne  voudroit  point  permettre  le  peché 
d'autruy,  manqueroit  luy  même  à  ce  qu'il  doit  ;  comme  si  un  officier  qui  doit  garder  un  poste 
important, le quittoit, surtout dans un temps de danger, pour empêcher une querelle dans la ville 
entre deux soldats de la garnison prêts à s'entretuer (2). 
\bigskip
A  l'instar  de  l'exemple  hiérarchisé,  nous  rencontrerons  l'illustration  surprenante,  inattendue, 
prestigieuse, qui doit servir par là même à faire apprécier la portée de la règle. Méré illustre ainsi 
l'affirmation que seul est aimé celui qui est aimable : 
\bigskip
Quand je pense que le Seigneur aime celuy-cy, et qu'il hait celuy-là sans qu'on sçache pourquoy; je 
n'en trouve point d'autre raison qu'un fonds d'Agrémens qu'il voit dans l'un et qu'il ne trouve pas 
dans l'autre, et je suis persuadé que le meilleur moyen, et peut-estre le seul pour se sauver c'est de 
luy plaire (3). » 
\bigskip
(2) Leibniz, Œuvres éd. Gerhardt, 6e vol. : Essais de Théodicée, p. 117.  
(3) Chev. De Méré, Œuvres, t. II : Des agrémens, P. 29. 
\bigskip
P 485-486 : « L'illustration inadéquate ne joue pas le même rôle que le cas invalidant, parce que la 
règle n'étant pas mise en question, l'énoncé de l'illustration inadéquate retentit plutôt sur celui qui 
en fait état, et témoigne de son incompréhension, de sa méconnaissance du sens de la règle. » 
\bigskip
P 486 : « Néanmoins l'illustration volontairement inadéquate peut constituer une forme  d'ironie. 
En disant, d'une haleine : « Il faut respecter ses parents; quand l'un d'eux vous gronde, répliquez-
lui vivement », on met en doute le sérieux de la règle. 
\bigskip
Cet  emploi  ironique  de  l'illustration  inadéquate  est  surtout  frappant  par  référence  à  des 
qualifications. On remarquera à ce propos que la « règle » au sens où nous en traitons, c'est tout 
énoncé  général  par  rapport  à  ce  qui  en  est  une  application.  La  qualification  donnée  à  une 
personne,  peut  être  considérée  comme  une  règle  dont  ses  comportements  fourniraient  des 
illustrations (1). Antoine use de l'illustration volontairement inadéquate lorsque, tout en ne cessant 
de répéter que Brutus est un homme honorable, il énumère ses actes d'ingratitude et de trahison 
(2)  ;  Montherlant  en  use,  dans  Les  jeunes  Filles,  lorsque  faisant  assurer  par  Costals  que  Andrée 
\bigskip
\bigskip
\bigskip
255 
\bigskip
Hacquebaut est intelligente, il nous convainc à  chaque page de sa stupidité (3). Certaines figures 
classiques, telles l'antiphrase, ne seraient souvent qu'une application de ce même procédé. 
\bigskip
Tout comme l'exemple permet non seulement de fonder une règle, mais aussi de passer d'un cas 
particulier  à  un  autre,  la  comparaison,  lorsqu'elle  n'est  pas  une  évaluation  (4),  est  souvent  une 
illustration  d'un  cas  au  moyen  d'un  autre,  tous  deux  étant  considérés  comme  des  applications 
d'une même règle. Voici un exemple typique de son usage : 
\bigskip
Ce sont les difficultés qui révèlent les hommes. Aussi, quand survient une difficulté, souviens-toi 
que Dieu, comme un maître de gymnase, t'a mis aux prises avec un jeune et rude partenaire (5). » 
\bigskip
(1) Cf. aussi, an § 74, la qualification Comme expression (le l'essence. 
(2) Shakespeare, Julius Caesar, acte 111, scène Il. 
(3) Cf. S. De Beauvoir, Le deuxième sexe, vol. 1, 1). 315. 
4) Cf. § )7 : Les arguments de comparaison. 
(5) Epictète, Entretiens, liv. 1, 24, § 1. 
\bigskip
P 487 : « La référence à une règle, bien que tout à fait implicite, est certaine aussi, et il s'agit donc 
bien d'une illustration, dans une phrase comme celle-ci : 
\bigskip
Pas de mort, pas de mourants, c'est la partie du champ de bataille, proche de l'ambulance, que 
l'on nettoie par propreté. Les premières meules, les premières haies sont vides de blessés, comme 
de leurs fruits dans un verger les branches basses (1). 
\bigskip
Certaines comparaisons illustrent une qualification générale à l'aide d'un cas concret, bien connu 
des auditeurs : il s'agit d'expressions telles « fier comme Artaban », « riche comme Crésus »... Ces 
expressions  devraient  transférer  sur  celui  auquel  on  les  applique  quelque  chose  du  caractère 
éminent de l'illustration choisie, mais elles tendent rapidement au « cliché », avant tout au plus la 
portée d'un superlatif. 
\bigskip
Quel  rôle  le  cas  particulier.  fictif,  l'expérimentation  mentale,  jouent-ils  dans  l'argumentation  ? 
Mach, Rignano, Goblot, Ruyer et Schuhl, entre autres, ont analysé ce problème, qui se pose surtout 
à  propos  de  l'illustration  (2).  C'est  en  effet,  quand  la  règle  est  suffisamment  connue,  qu'une 
situation  qui  doit  l'illustrer  peut  être  le  plus  aisément  construite,  telle,  pour  illustrer  la  règle 
prescrivant de désigner par le sort les chefs responsables, l'histoire des marins choisissant, par voie 
de tirage au sort, le capitaine auquel sera confiée la direction du navire (3). Ne confondons pas, à 
ce  propos,  cas  fictifs  et  cas  forgés  par  l'auteur,  pour  les  besoins  de  la  cause,  mais  qui  auraient 
parfaitement pu se produire. » 
\bigskip
(1) J. Giraudoux, Lectures, pour nue ombre, p. 216. 
(2)  Cf.  E.  Mach,  Erkenntnis  Und  Irrtum  ;  Rignano,  Psychologie  du  raisonneaient;  E.  Goblot, 
Traité de logique;  R. Ruyer, L'utopie et les utopies; P.-M. Schuhl, I., Le merveilleux la pensée et 
l'action 
(3) Platon, République, liv. VI1, 488 b-489 cl ; Cf. Aristote, Rhétorique, il, chap. 20,1393 b; B. D. 
D. Whately, Elements of Rhetoric, Part I, chap. II, p. 69. 
\bigskip
P  487-488 :  « L'auteur  de  la  Rhétorique  à  Herennius  explique  pourquoi  il  juge  préférable  de 
composer  lui-même  les  textes  qui  doivent  illustrer  ses  règles  de  rhétorique  plutôt  que  de  les 
emprunter,  comme  le  faisaient  les  Grecs,  aux  grands  écrivains  (1).  Le  cas  forgé  est  lié  plus 
étroitement à la règle que le cas observé -, il indique mieux que celui-ci que la réussite est possible 
à  qui  se  conforme  à  la  règle,  et  en  quoi  consiste  celle-ci.  Cependant  cette  garantie  est  en  partie 
illusoire. Le cas forgé est semblable à une expérience montée dans un laboratoire scolaire. Mais il 
\bigskip
\bigskip
\bigskip
256 
\bigskip
se peut qu'il ait été forgé bien plus à l'instar d'un modèle prestigieux qu'en application de la règle 
qu'il est censé illustrer. » 
\bigskip
(1) Rhétorique à Herennius, liv. IV, §§ 1 à 10. 
\bigskip
§ 80. LE MODELE ET L'ANTIMODELE 
\bigskip
P  488 :  « Quand  il  s'agit  de  conduite,  un  comportement  particulier  peut,  non  seulement  servir  à 
fonder ou à illustrer une règle générale, mais inciter à une action qui s'inspire de lui. 
\bigskip
Il  existe  des  conduites  spontanées  d'imitation.  Aussi  la  tendance  à  l'imitation  a-t-elle  été 
considérée souvent comme un instinct, et de la plus haute conséquence aux yeux du sociologue (2). 
Par  ailleurs  on  connaît  la  place  attribuée  par  la  psychologie  contemporaine  aux  processus 
d'identification (3). Nous-mêmes avons insisté sur le rôle de l'inertie, sur le fait que la répétition 
d'une même conduite n'a pas, contrairement à la déviation et au changement, à être justifiée, et sur 
l'importance qui, par là, s'attache au précédent (4). 
\bigskip
Mais  l'imitation  d'une  conduite  n'est  pas  toujours  spontanée,  Il  arrive  que  l'on  y  soit  invité. 
L'argumentation se fondera soit sur la règle de justice (5), soit sur un modèle auquel on demandera 
de se conformer, comme dans l'exemple d'Aristote : 
\bigskip
S'il a été beau pour les Augustes Déesses de se soumettre à la sentence de l'Aréopage, il n'en serait 
pas de même pour Mixidémidès (6) ! » 
\bigskip
(2) G. Tarde, Les lois de l'imitation, E. Dupréel, Sociologie générale, pp. 66 et suiv. 
(3)  Cf.  notamment  un  très  intéressant  exemple  où  l'identification  est  verbalisée  chez  M.-A. 
Sechehaye, Journal d'une schizophrène, p. 118. 
(4) Cf. § 27 Accords propres à chaque discussion. 
(5) Cf. § 52 La règle de justice. 
(6) Aristote, Rhétorique, liv. II, chap. 21, XI, 1398 b. 
\bigskip
P 489 : « Peuvent servir de modèle des personnes ou des groupes dont le prestige valorise les actes. 
La  valeur  de  la  personne,  reconnue  au  préalable,  constitue  la  prémisse  dont  on  tirera  une 
conclusion préconisant un comportement particulier. On n'imite pas n'importe qui; pour servir de 
modèle, il faut un minimum de prestige (1). D'après Rousseau, 
\bigskip
Le singe imite l'homme qu'il craint, et n'imite ras les animaux qu'il méprise ; il juge bon ce que 
fait un être meilleur que lui (2). 
\bigskip
Si  l'on  a  servi  de  modèle,  c'est  qu'on  possède  donc  un  certain  prestige,  dont  la  preuve  peut  être 
fournie par cet effet même (3) : 
\bigskip
Donne ta propre pondération en exemple aux autres, écrit Isocrate àNicoclès, en te rappelant que 
les  moeurs  d'un  peuple  ressemblent  à  celles  de  qui  le  gouverne.  Tu  auras  un  témoignage  de  la 
valeur  de  ton  autorité  royale  lorsque  tu  constateras  que  tes  sujets  ont  acquis  une  plus  grande 
aisance et des moeurs mieux policées grâce à ton activité (4). 
\bigskip
D'habitude le modèle glorifié est proposé à l'imitation de tous, parfois il s'agit d'un modèle réservé 
à  un  petit  nombre  ou  seulement  à  soi-même,  parfois  c'est  un  patron  (pattern)  à  suivre  dans 
certaines circonstances : conduisez-vous, dans cette situation, en bon père de famille, aimez votre 
prochain  comme  vous-même,  considérez  comme  vraies  uniquement  des  propositions  conçues 
aussi clairement et aussi distinctement que la proposition « je pense donc je suis » (5). 
\bigskip
(1) Cf. § 70 : L'argument d'autorité. 
\bigskip
\bigskip
\bigskip
257 
\bigskip
(2) J,-J. Rousseau, Emile, p, 95. 
(3) Cf. un émouvant emploi de la valorisation comme modèle dans Marie DE Vivier : Le mal que je 
l'ai fait, p. 155, « 0 Sébastien Galois, que l'éternité te ressemble ou ne soit pas. » 
(4) Isocrate, Discours, t. II . A Nicoclés, § 31 ; et. aussi Panégyrique d'Athènes, § 39. 
(5) Descartes, Discours de la méthode, p. 87. 
\bigskip
P  489-490 :  « Un  homme,  un  milieu,  une  époque,  seront  caractérisés  par  les  modèles  qu'ils  se 
proposent  et  la  manière  dont  ils  les  conçoivent.  Il  est  significatif  de  constater,  pour  marquer  la 
révolution intellectuelle qui s'est produite en France au tournant du XVIIe siècle, que Pierre de La 
Ramée,  dans  l'élaboration  de  sa  dialectique,  cherchera  des  modèles  chez  les  poètes,  les  orateurs, 
les  philosophes  et  les  juristes  (1)  tandis  que  Descartes  se  propose  lui-même  en  modèle  à  ses 
lecteurs (2). » 
\bigskip
(1) P. Ramus, Dialecticae libri duo Audomari Talari praelectionibus illustrati, 1566, note, p. 9. 
(2)  Descartes,  Méditations,  préface  ;  cf.  dans  le  même  sens  E.  Husserl,  La  crise  des  sciences 
européennes et la phénoménologie transcendantale, p. 143. 
\bigskip
P 490 : « Le modèle indique la conduite à suivre ; il sert aussi de caution àune conduite adoptée. 
Pour justifier ses sarcasmes envers les jésuites, Pascal se réclamera d'une série de Pères de l'Église 
et de Dieu lui-même, qui n'ont pas hésité à fustiger l'erreur (3). 
\bigskip
Le fait de suivre un modèle reconnu, de s'y astreindre, garantit la valeur de la conduite; l'agent que 
cette attitude valorise peut donc, à son tour, servir de modèle : le philosophe sera propose comme 
modèle à la cité parce que lui-même a pour modèle les dieux (4) ; sainte Thérèse sera inspiratrice 
de la conduite des chrétiens, parce qu'elle-même avait comme modèle, Jésus (5). 
\bigskip
Ajoutons  cependant  que  l'indifférence  au  modèle  peut,  elle-même,  être  donnée  en  modèle  :  on 
propose  en  modèle  celui  qui  est  capable  de  se  dérober  aux  tentations  de  l'imitation.  Le  fait  qu'il 
peut y avoir une argumentation par le modèle sur ce plan de l'originalité montre nettement que les 
modes  d'argumentation  s'appliquent  aux  circonstances  les  plus  diverses,  c'est-à-dire  que  la 
technique  argumentative  n'est  pas  liée  à  telle  situation  sociale  définie  ni  au  respect  de  telles  ou 
telles valeurs. » 
\bigskip
(3) Pascal, Bibl. de la Pléiade, XIe Provinciale, pp. 551-553. 
(4) Platon, République, liv. Vl, 500 c, d. 
(5) Don Quichotte sera pour d'aucuns un modèle parce qu'il était capable de suivre, avec passion, 
le modèle qu'il s'était choisi. 
\bigskip
P  490-491 :  « Par  ailleurs,  sur  qui  est  modèle  et  inspirateur  des  autres,  pèse  une  obligation  qui 
sera, le plus souvent,  déterminante de sa conduite. C'est de cet argument, avons-nous vu, que se 
sert Isocrate pour éduquer Nicoclès. Ce même thème forme l'essentiel d'une pièce contemporaine, 
où l'aîné de deux frères, parce qu'il constitue le modèle de l'autre, voit sa conduite inspirée par lui : 
\bigskip
C'est en lui que je confronte l'image  que je connais de moi à celle qu'il s'en est formée, et que  je 
modèle l'une sur l'autre. Sans lui, je ne suis rien, car c'est par lui que je me prouve (1). » 
\bigskip
(1) C.-A. Puget, La peine capitale, acte II, p. 64. 
\bigskip
P 491 : « Le modèle doit surveiller sa conduite, car le moindre de ses 
\bigskip
écarts  en  justifiera  mille  autres,  très  souvent  même  à  l'aide  d'un  argument  a  fortiori.  Pascal  eut 
raison de constater que : 
\bigskip
\bigskip
\bigskip
\bigskip
258 
\bigskip
L'exemple de la chasteté d'Alexandre n'a pas tant fait de continents que celui de son ivrognerie a 
fait  d'intempérants.  Il  n'est  pas  honteux  de  n'être  pas  aussi  vertueux  que  lui,  et  il  semble 
excusable de n'être pas plus vicieux que lui (2). 
\bigskip
L'être prestigieux sera décrit en fonction de son rôle de modèle, on mettra en évidence tel ou tel de 
ses caractères ou de ses actes, on adaptera même son image ou sa situation pour que l'on puisse 
plus aisément s'inspirer de sa conduite : 
\bigskip
Un honneste homme, écrira le chevalier de Méré, doit vivre à peu pres comme un grand Prince 
qui  se  rencontre  en  un  païs  étranger  sans  sujets  et  sans  suite,  et  que  la  fortune  reduit  à  se 
conduire comme un honneste particulier (3). 
\bigskip
Attribuer à des êtres supérieurs une certaine qualité permet, si le fait est admis, d'argumenter par 
le modèle, et, s'il est contesté, 
\bigskip
de  valoriser  cette  qualité  comme  étant  en  tout  cas  digne  d'être  attribuée  au  modèle.  Ainsi  selon 
Isocrate : 
\bigskip
... on raconte que les dieux, eux aussi, sont gouvernés par Zeus, leur roi. Si le récit de ces faits est 
exact,  il  est  évident  que  les  dieux  eux-mêmes  jugent  cette  institution  préférable  aux  autres.  Si 
personne  ne  connaît  la  vérité  absolue  et  si  chacun  fait  appel  à  ses  propres  conjectures  pour 
concevoir  ainsi  la  vie  des  dieux,  c'est  la  preuve  que  tous  nous  mettons  au  premier  rang  la 
monarchie (4)... » 
\bigskip
(2) Pascal, Bibl. de la Pléiade, Pensées, 182 (227), p. 870 (103 éd. Brunschvicg).  
(3) Chev. De Méré, Œuvres complètes, t. Il : Des agrémens, p. 21. 
(4) Isocrate, Discours, t. II : Nicoclès, § 26. 
\bigskip
P 492 : « De même, Montesquieu fait dire par Usbek : 
\bigskip
Ainsi, quand il n'y aurait pas de Dieu, nous devrions toujours aimer la justice; c'est-à-dire faire 
nos efforts pour ressembler à cet être dont nous avons une si belle idée, et qui, s'il existait, serait 
nécessairement juste (1). 
\bigskip
Bien  que  servir  de  modèle  soit  une  marque  de  prestige,  le  rapprochement  causé  par  l'imitation 
entre  le  modèle  et  ceux  qui  s'en  inspirent  et  qui,  presque  toujours,  lui  sont  inférieurs,  peut 
dévaluer  quelque  peu  le  modèle.  Nous  avons  déjà  vu  que  toute  comparaison  entraîne  une 
interaction entre les termes (2). De plus, en le vulgarisant, on enlève au modèle la valeur qui tenait 
à  sa  distinction  :  le  phénomène  de  la  mode,  avec  tous  ses  avatars,  s  1  explique,  on  le  sait,  par  le 
désir,  propre  à  la  masse,  de  se  rapprocher  de  ceux  qui  donnent  le  ton  et  par  le  désir  de 
différenciation et de fuite propre à ceux que l'on copie. Le même Isocrate qui conseille à Nicoclès 
de servir de modèle à la foule, lui demandera de s'en distinguer : 
\bigskip
...  tu  ne  peux  pas  [chef  suprême  d'une  foule]  avoir  les  mêmes  sentiments  que  tout  le  monde, 
convaincu  que  tu  n'apprécies  pas  la  gravité  des  affaires  et  la  sagesse  des  hommes  d'après  les 
plaisirs qu'ils te procurent, mais que tu les mets à l'épreuve d'après leur valeur pratique (3). 
\bigskip
La foule est devenue ici antimodèle. » 
\bigskip
(1) Montesquieu, Lettres persanes, LXXXIV, Usbek à Rhédi, p. 58.  
(2) Cf. § 57 : Les arguments de comparaison.  
(3) Isocrate, Discours, t. II : A Nicoclès, § 50. 
\bigskip
\bigskip
\bigskip
\bigskip
259 
\bigskip
P 492-493 : « Si la référence à un modèle permet de promouvoir certaines conduites, la référence à 
un  repoussoir,  à  un  antimodèle  permet  d'en  détourner.  Pour  certains  esprits,  tels  Montaigne, 
l'action de l'antimodèle est la plus efficace : 
\bigskip
Il en peut estre aucuns de ma complexion, qui m'instruis mieux par contrarieté que par exemple, 
et par fuite que par suite. A cette sorte de discipline regardoit le vieux Caton, quand il dict que les  
sages  ont  plus  à  apprendre  des  fols  que  les  fols  des  sages  ;  et  cet  ancien  joueur  de  lyre,  que 
Pausanias recite avoir accoustumé contraindre ses disciples d'aller ouyr un mauvais sonneur qui 
logeoit vis à vis de luy, où ils apprinsent à hayr ses desaccords et fauces mesures (1)... » 
\bigskip
(1) Montaigne, Bibl. (le la Pléiade, Essais, liv. III, chap. VIII, p. 893. 
\bigskip
P 493 : « L'effet de repoussoir est-il obtenu grâce à l'argument de l'antimodèle on  parce que l'on 
apprécie  l'acte  par  ses  conséquences,  qui  sont  déplorables  ?  Il  y  a  là  deux  argumentations 
différentes,  quoiqu'une  interaction  entre  elles  soit  inévitable  :  juge-t-on  l'agent  par  ses  actes  ou 
inversement ? Ce n'est que dans le deuxième cas que nous verrons l'effet de l'antimodèle, tel que le 
décrit le chevalier de Méré : 
\bigskip
je remarque aussi qu'on ne fuit pas seulement ceux qui déplaisent, mais qu'on hait tout ce qui leur 
appartient, et qu'on ne leur veut ressembler que le moins qu'on peut. S'ils loüent la paix, ils font 
souhaiter la guerre; s'ils sont devots et reglez, on veut estre libertin et des-ordonné (2). » 
\bigskip
(2) Chev. De Méré, (Euvres complètes, t. II : Des agrémens, IV. 30-31. 
\bigskip
P  493-494 :  « A  première  vue,  tout  ce  que  nous  avons  dit  du  modèle  peut  s'appliquer,  mutatis 
mutandis,  à  l'antimodèle.  Parfois  on  sera,  au  moment  d'une  délibération,  incité  à  choisir  un 
comportement  parce  qu'il  est  opposé  à  celui  de  l'antimodèle;  la  répulsion  ira  parfois  jusqu'à 
provoquer  le  changement  d'une  attitude  antérieurement  adoptée,  pour  la  seule  raison  que  c'est 
également  celle  de  l'antimodèle  (3).  Un  trait  important  distingue  pourtant  cette  forme 
d'argumentation  de  celle  par  le  modèle  :  alors  que,  dans  cette  dernière,  on  propose  de  se 
conformer,  fût-ce  de  façon  maladroite,  à  quelqu'un,  et  que  donc  la  conduite  à  adopter  est 
relativement bien connue, dans l'argument de l'antimodèle on incite à se distinguer de quelqu'un, 
sans que l'on puisse toujours en inférer une conduite précise. C'est souvent par référence implicite 
à un modèle qu'une certaine détermination de cette conduite sera possible : s'éloigner de Sancho 
Pança  ne  se  conçoit  que  pour  qui  connaît  la  figure  de  don  Quichotte;  la  vue  de  l'ilote  ne  peut 
déterminer une conduite que pour qui connaît le comportement d'un spartiate aguerri. » 
\bigskip
(3) Cf. G. Marcel, Rome n'est plus dans Rome, acte III, sc. IV. 
\bigskip
P  494 :  « Parce  qu'il  détourne  de  ce  qu'il  fait,  l'antimodèle,  en  adoptant  une  conduite,  la 
transforme,  de  manière  volontaire  ou  involontaire,  en  parodie  et  parfois  en  provocation.  C'est  le 
cas des démons dont parle Bossuet : 
\bigskip
J'apprends aussi de Tertullien que non-seulement les démons se faisaient présenter devant leurs 
idoles des voeux et des sacrifices, le propre tribut de Dieu, mais qu'ils les faisaient parer des robes 
et des ornements dont se revêtaient les magistrats, et faisaient porter devant eux les faisceaux et 
les bâtons d'ordonnance, et les autres marques d'autorité publique; parce qu'en effet, dit ce grand 
personnage,  «les  démons  sont  les  magistrats  du  siècle»  ...  Et  à  quelle  insolence,  mes  frères,  ne 
s'est pas porté ce rival de Dieu ? Il a toujours affecté de faire ce que Dieu faisait, non pas pour se 
rapprocher  en  quelque  sorte  de  la  sainteté,  c'est  sa  capitale  ennemie  ;  mais  comme  un  sujet 
rebelle, qui par mépris, ou par insolence, affecte la même pompe que son souverain (1). 
\bigskip
\bigskip
\bigskip
\bigskip
260 
\bigskip
Bossuet  songe-t-il,  dans  ce  passage,  à  la  Fronde  ?  Peu  importe.  L'essentiel  est  qu'il  révèle 
clairement le mécanisme de l'argumentation par l'antimodèle. 
\bigskip
Ce dernier est souvent représenté d'une façon conventionnelle et délibérément fausse en raison de 
l'effet révulsif qu'il doit produire. Ce n'est pas au manque de connaissance de la société musulmane 
qu'il faut attribuer les traits conventionnels du Sarrasin dans la chanson de geste française (2). » 
\bigskip
(1) Bossuet, Sermons, vol. II : Premier sermon sur les démons, P. 13. 
(2) Cf. C. Mrededith Jones, The conventional Saracen of the songs of geste, Speculum, vol. XVII, 
n° 2, avr. 1942, p. 202. 
\bigskip
P  494-495 :  « Cependant  l'introduction  de  l'antimodèle,  an  lieu  de  viser  simplement  à  un  effet 
révulsif,  peut  servir  d'amorce  à  une  argumentation  a  fortiori,  l'antimodèle  représentant  un 
minimum  en  dessous  duquel  il  est  indécent  de  descendre.  D'ailleurs  comme  l'antimodèle  est 
souvent,  en  même  temps,  un  adversaire  à  combattre  et  éventuellement  à  abattre,  le  rôle  dans 
l'argumentation, d'un même être abhorré, sera complexe. On sait que la compétition développe les 
ressemblances  entre  antagonistes  (1),  qui,  à  la  longue,  s'empruntent  tous  les  procédés  efficaces  : 
c'est  parce  que  ce  sont  celles  de  l'adversaire  que  certaines  techniques  pourront  être  préconisées. 
Cependant, lorsque celui-ci est aussi l'antimodèle, on aura soin très souvent de séparer moyens et 
fins,  ou  encore  de  distinguer  temporaire  et  permanent,  indispensable  et  superflu,  licite  et  illicite 
(2). » 
\bigskip
(1) R. Dupréel, Sociologie générale, p. 157. 
(2) Cf. J. Giraudoux, à propos de la création du Commissariat à l'Information, La Française et la 
France, pp. 234 à -137, -941. 
\bigskip
P  495 :  « En  proposant  à  autrui  un  modèle  ou  un  antimodèle,  on  sousentend,  à  moins  de 
restreindre leur rôle à des circonstances particulières, que soi-même on s'efforce également de s'en 
rapprocher  ou  de  s'en  distinguer.  Cela  permet  des  réparties  comiques,  du  genre  de  celle-ci  :  au 
Père, qui dit à son fils, qui travaille mal, « A ton âge, Napoléon était premier en classe », l'enfant 
réplique «A ton âge, il était empereur. » 
\bigskip
L'argument par le modèle ou l'antimodèle peut s'appliquer spontanément au discours lui-même : 
l'orateur qui affirme sa croyance en certaines choses ne les appuie pas seulement de son autorité. 
Son  comportement  à  leur  égard,  s'il  a  du  prestige,  peut  également  servir  de  modèle,  inciter  à  se 
comporter comme il le fait ; et inversement, s'il est l'antimodèle on s'éloignera de lui. » 
\bigskip
§ 81. L'ÊTRE PARFAIT COMME MODELE 
\bigskip
P 495-496 : « Les inconvénients de l'argumentation par le modèle ou l'antimodèle se manifestent 
quand  le  modèle  comporte  des  caractères  répréhensibles  on  l'antimodèle  des  qualités  dignes 
d'imitation.  En  effet,  toute  discrimination  parmi  les  actes  du  modèle  ou  de  l'antimodèle  suppose 
un critère autre que la personne ou le groupe que l'on exalte on que l'on méprise, critère qui rend 
l'argumentation par le modèle inutilisable, car superflue ou même dangereuse. » 
\bigskip
P 496 : « Pour obvier à ces inconvénients, les auteurs sont amenés a embellir ou à noircir la réalité, 
à créer des héros et des monstres, tout bons ou tout mauvais, à transformer l'histoire en mythe, en 
légende,  en  image  d'Épinal.  Mais  même  alors,  la  multiplicité  de  modèles  ou  d'antimodèles  ne 
permet pas d'en tirer une règle de conduite unique et claire. Les objets empruntés à l'expérience ne 
peuvent, pour cette raison, d'après Kant, être considérés comme des modèles (ou des archétypes) : 
\bigskip
Qui  (comme  beaucoup  l'ont  fait  réellement),  voudrait  donner  pour  type  à  la  source  des 
connaissances ce qui ne peut jamais servir que d'exemple, celui-là ferait de la vertu un fantôme 
\bigskip
\bigskip
\bigskip
261 
\bigskip
équivoque, variable suivant les temps et les circonstances, et incapable de servir jamais de règle 
(1). 
\bigskip
Au contraire, tout être incarné doit être confronté, d'après Kant, non pas simplement avec l'idée de 
la vertu, mais avec un idéal, tel celui du sage stoïcien : 
\bigskip
...  un  homme  qui  n'existe  que  dans  la  pensée,  mais  qui  correspond  pleinement  à  l'idée  de  la 
sagesse.  De  même  que  l'idée  donne  la  règle,  'idéal  sert,  en  pareil  cas,  de  prototype  à  la 
détermination complète de la copie et nous n'avons, pour juger nos actions, d'autre règle que la 
conduite de cet homme divin que nous portons en nous et auquel nous nous comparons pour nous 
juger  et  pour  nous  corriger  ainsi,  mais  sans  jamais  pouvoir  en  atteindre  la  perfection.  Ces 
idéaux,  bien  qu'on  ne  puisse  leur  attribuer  de  réalité  objective  (d'existence),  ne  doivent  pas, 
cependant,  être  regardés  comme  des  chimères;  ils  fournissent,  au  contraire,  à  la  raison  une 
mesure qui lui est indispensable, puisqu'elle a besoin du concept de ce qui est absolument parfait 
dans son espèce pour apprécier et pour mesurer, en s'y référant, Jusqu'à quel point l'imparfait se 
rapproche et reste éloigné de la perfection (2). » 
\bigskip
(1) E. Kant, Critique de la raison pitre, p. 305.  
(2) E. Kant, Critique de la raison pure, pp. 476-477. 
\bigskip
P 496-497 : « Kant se rend compte de l'importance du modèle pour la conduite, mais croit que ce 
modèle  n'est  qu'un  idéal  que  chaque  homme  porte  en  lui,  sans  que  les  bornes  naturelles  en 
permettent une réalisation dans un exemple phénoménal. » 
\bigskip
P 497 : « Cet archétype, que Kant trouve dans « cet homme divin que nous portons en nous », les 
religions  le  fournissent  aux  hommes  grâce  à  l'idée  ou  à  l'image  qu'elles  présentent  de  Dieu,  de 
l'Être age parfaitement bon ou du moins de son représentant et porteparole sur terre. Tarde a déjà 
eu  l'occasion  de  montrer  l'importance  de  jésus,  de  Mahomet,  de  Bouddha,  comme  modèles  pour 
l'humanité (1). Ce rôle est rempli d'autant plus facilement que ces êtres, quelle que soit leur qualité 
surnaturelle,  se  conduisent  néanmoins  comme  des  hommes  vivant  avec  d'autres  hommes.  D'un 
certain  point  de  vue,  l'incarnation  de  la  divinité  serait  déjà  une  correction  du  modèle  pour  le 
rapprocher  de  ceux  qu'il  faut  édifier.  Néanmoins  nous  constatons  que  ceux  qui  utilisent  cette 
forme  d'argumentation  adaptent  de  façon  beaucoup  plus  directe  encore  leur  modèle  aux 
conclusions qu'ils veulent promouvoir. Citons à ce propos, quelques exemples significatifs où jésus 
est proposé en modèle. 
\bigskip
Voici deux passages où Bossuet présente jésus comme un modèle de roi absolu : 
\bigskip
Jésus-Christ, Seigneur des seigneurs, et Prince des rois de la terre, quoique élevé dans un trône 
souverainement indépendant, néanmoins, pour donner à tous les monarques, qui relèvent de sa 
puissance,  l'exemple  de  modération  et  de  justice,  il  a  voulu  luimême  s'assujettir  aux  règlements 
qu'il a faits et aux lois qu'il a établies (2). 
\bigskip
Et ailleurs : 
\bigskip
Ce  grand  Dieu  n'a  besoin  de  personne;  et  néanmoins  il  veut  gagner  tout  le  monde...  Ce  grand 
Dieu sait tout, il voit tout, et néanmoins il veut que tout le monde lui parle; il écoute tout, et il a 
toujours  l'oreille  attentive  aux  plaintes  qu'on  lui  présente,  toujours  prêt  àfaire  justice.  Voilà  le 
modèle des rois (3)... 
\bigskip
(1) Tarde, La logique sociale, p. 308. 
(2) Bossuet, Sermons vol. II sur la prédication évangélique, P. 50.  
(3) Bossuet, Sermons vol. II Sur l'ambition, p. 411. 
\bigskip
\bigskip
\bigskip
262 
\bigskip
 
P 498 : « Pour Locke, Jésus est le modèle de la tolérance, qui doit inspirer l'action de ses prêtres et 
fidèles : 
\bigskip
Si,  comme  le  Capitaine  de  notre  salut,  ils  avaient  sincèrement  souhaité  le  bien  des  âmes,  ils 
auraient  suivi  les  traces  et  le  parfait  exemple  de  ce  Prince  de  la  paix...  Cependant  nous  savons 
très bien que si des infidèles avaient dû être convertis par la force, si des soldats armés avaient 
dû arracher à leurs erreurs, ceux qui étaient aveugles ou obstinés, il était bien plus facile à lui d'y 
parvenir avec des armées de légions célestes, qu 1 a n ' importe quel fils de l'Église, aussi puissant 
soit-il, avec tous ses dragons (1). 
\bigskip
Et il termine par cet appel suprême : 
\bigskip
Dieu lui-même ne sauvera pas les hommes malgré eux (2). 
\bigskip
Le  chevalier  de  Méré,  qui  n'hésitait  pas,  nous  l'avons  vu  (3),  à  se  servir  de  la  grâce  divine  pour 
illustrer  l'importance  d'être  aimable,  soutient  que  l'amour  des  choses  agréables  nous  fut  indiqué 
par jésus : 
\bigskip
Il me semble aussi que le plus parfait modele, et celuy que nous devons le plus imiter, aimoit tout 
ce  qui  se  faisoit  de  bonne  grace,  comme  ces  excellens  parfums  qui  furent  répandus  sur  luy  :  et 
peut-on rien s'imaginer de plus agreable que ses moindres discours et ses moindres actions (4) ? 
» 
\bigskip
Il arrive même que certains récits évangéliques soient interprétés uniquement en fonction du rôle 
de  modèle  que  jésus  assume,  et  sans  lequel  ils  deviendraient  incompréhensibles,  parce 
qu'incompatibles avec la perfection divine : 
\bigskip
Jésus-Christ  voit  dans  sa  prescience  en  combien  de  périls  extrêmes  nous  engage  l'amour  des 
grandeurs  ;  c'est  pourquoi  il  fuit  devant  elles,  pour  nous  obliger  à  les  craindre;  ....  il  nous 
apprend tout ensemble que e devoir essentiel du chrétien, c'est de réprimer son ambition (5). 
\bigskip
(1) Locke, The second treatise of civil government and A letter concerninq talcration, P. 125. 
(2) Ibid., p. 137. 
(3) Cf. § 79 : L'illustration. 
(4) Chev. De Méré, Œuvres complètes, I. II : Des agrémens, p. 28.  
(5) Bossuet, Sermons, Vol. II - Sur l'ambition, p. 394. 
\bigskip
P 499 : « Lui-même, le Dieu, n'a aucune raison de fuir. C'est le modèle seul qui en a. 
\bigskip
Il  n'est  pas  jusqu'au  milieu  dans  lequel  vit  le  modèle,  qui  ne  puisse  être  invoqué  pour  favoriser 
l'action de ce dernier en le rapprochant de ceux auxquels on le propose et que l'on valorise par là à 
leurs propres yeux : 
\bigskip
Comme les petits Jocistes s'exaltent à la pensée du Christ ouvrier, les paysans devraient puiser le 
même fierté dans la part qu'accordent les paraboles de l'Évangile à la vie des champs et dans la 
fonction sacrée du pain et du vin, et en tirer le sentiment que le christianisme est une chose à eux 
(1). 
\bigskip
Ces divers exemples montrent combien l'argumentation par le modèle, même limitée à l'exaltation 
de la vie d'un seul être, est susceptible d'utilisations et d'adaptations variées, selon que tel ou tel 
aspect de l'Être parfait est mis en vedette et proposé à l'imitation des hommes. 
\bigskip
\bigskip
\bigskip
\bigskip
263 
\bigskip
L'Être parfait se prête plus que n'importe quel autre modèle à cette adaptation parce que, par sa 
qualité même et par essence, il a quelque chose d'insaisissable, d'inconnu, et que, d'autre part, il ne 
vaut  pas  seulement  pour  un  temps  et  un  lieu.  Or  dans  la  mesure  oil  l'on  croit  pouvoir utiliser  le 
modèle  indépendamment  des  circonstances,  quand  le  modèle  est  plus  qu'un  patron  (pattern) 
limité dans sa portée, l'accusation d'anachronisme devient sans fondement. Le rôle de l'interprète 
est  alors  essentiel  :  c'est  lui  qui  permet  au  modèle  indiscuté  de  servir  de  guide  dans  toutes  les 
circonstances de la vie. » 
(1) S. Weil, L’enracinement p. 82. 
\bigskip
B) LE RAISONNEMENT PAR ANALOGIE 
\bigskip
§ 82. QU'EST-CE QUE L'ANALOGIE 
\bigskip
l'analogie 
\bigskip
fait  partie  d'une 
\bigskip
P  499-500 :  « Personne  n'a  nié  l'importance  de  l'analogie  dans  la  conduite  de  l'intelligence. 
Toutefois,  reconnue  par  tous  comme  un  facteur  essentiel  d'invention,  elle  a  été  regardée  avec 
méfiance dès que l'on voulait en faire un moyen de preuve. Il est vrai que certaines philosophies, 
celles de Platon, de Plotin ou de saint Thomas, ont justifié l'usage argumentatif de l'analogie grâce 
à  la  conception  qu'elles  fournissaient  du  réel  ;  mais  cet  usage  paraissait  ainsi  lié  à  une 
métaphysique  et  solidaire  de  son  sort.  Les  penseurs  empiristes,  par  contre,  ne  voient  le  plus 
souvent, dans l'analogie, qu'une ressemblance de qualité mineure, imparfaite, faible, incertaine (1). 
On  admet,  plus  ou  moins  explicitement,  que 
série, 
identité-ressemblance-analogie, dont  elle constitue l'échelon  le moins significatif. Sa seule valeur 
serait de permettre la formulation d'une hypothèse à vérifier par induction (2). » 
\bigskip
(1) Hume, Traité de la nature humaine, t. I, p. 226. 
(2) J. St. Mill, Système de logique, vol. II, p. 90. 
\bigskip
P 500 : « Loin de nous l'idée qu'une analogie ne puisse servir de point de départ à des vérifications 
ultérieures ; mais elle ne se distingue en cela d'aucun autre raisonnement, car leurs conclusions, à 
tous, peuvent toujours être soumises à une nouvelle épreuve. Et sommes nous autorisés à dénier à 
l'analogie  toute  force  probante,  alors  que  le  seul  f  ait  d'être  capable  de  nous  faire  préférer  une 
hypothèse  à  une  autre,  indique  qu'elle  possède  valeur  d'argument  ?  Toute  étude  d'ensemble  de 
l'argumentation doit donc lui faire place en tant qu'élément de preuve. » 
\bigskip
Il nous semble que sa valeur argumentative sera le plus clairement mise en évidence si on envisage 
l'analogie comme une similitude de structures, dont la formule la plus générale serait : A est à B ce 
que C est à D. Cette conception de l'analogie se rattache à une tradition très ancienne, encore en 
usage  chez  Kant  (3),  chez  Whately  (4),  chez  Cournot  (5).  Elle  n'est  point  entièrement  oubliée, 
témoin cette opinion de M. Cazals que cite Paul Grenet dans un récent ouvrage : 
\bigskip
Ce qui fait l'originalité de l'analogie et ce qui la distingue d'une identité partielle, c'est-à-dire de la 
notion un peu banale de ressemblance, c'est qu'au lieu d'être un  rapport de ressemblance elle est 
une  ressemblance  de  rapporte  M  ce  n'est  pas  là  un  simple  jeu  de  mots,  le  type  le  plus  pur  de 
l'analogie se trouve dans une proportion mathématique (1). » 
\bigskip
(3) Kant, Prolégomènes à toute métaphysique future, pp. 146-147. 
(4) Richard D. D. Whately, Elements of Rhetoric, p. 359. Appendix to p. 67  B). 
(5) A.-A. Cournot, Essai sur les fondements de nos connaissances, vol. 1, p. 93-94. 
(1) P. Grenet, Les origines de l'analogie philosophique dans les dialogues de Platon, p. 10. Cf. Mgr 
De Solages, Dialogue su, l'analogie, p. 15. 
\bigskip
P 501 : « Nous souscrivons à ces lignes, sauf sur le dernier point. Si l'étymologie incite à retrouver 
le  prototype  de  l'analogie  dans  la  proportion  mathématique  (2),  cette  dernière  n'est  à  nos  yeux 
qu'un  cas  particulier  de  similitude  de  rapports  et  pas  du  tout  le  plus  significatif.  En  effet,  on  n'y 
\bigskip
\bigskip
\bigskip
264 
\bigskip
voit pas ce qui précisément caractérise, selon nous, l'analogie, et qui a trait à la différence entre les 
rapports que l'on confronte. 
\bigskip
Pour préciser ceci, partons d'une analogie assez simple et typique, tirée d'Aristote : 
\bigskip
De même que les yeux des chauve-souris sont éblouis par la lumière du jour, ainsi l'intelligence de 
notre âme est éblouie par les choses les plus naturellement évidentes (3). » 
\bigskip
(2) Cf. M. Dorolle, Le raisonnement par analogie, chap. I. 
(3) Aristote, Métaphysique, Liv. oc, 993 b. 
\bigskip
P 501-502 : « Nous proposons d'appeler thème l'ensemble des ternies A et B, sur lesquels porte la 
conclusion (intelligence de l'âme, évidence) et  d'appeler  phore  l'ensemble  des termes C et D, qui 
servent  à  étayer  le  raisonnement  (yeux  de  la  chauve-souris,  lumière  du  jour).  Normalement,  le 
phore  est  mieux  connu  que  le  thème  dont  il  doit  éclairer  la  structure,  ou  établir  la  valeur,  soit 
valeur  d'ensemble,  soit  valeur  respective  des  termes.  Il  n'en  est  cependant  pas  toujours  ainsi  : 
Catherine de Gênes, en fin de son Traité du Purgatoire, tâchera d'élucider son propre état d'âme, 
par analogie avec celui des âmes du Purgatoire, dont il est difficile de dire qu'il soit mieux connu, 
mais à la description duquel elle avait consacré de longs développements : 
\bigskip
Cette  forme  purgative  que  je  vois  dans  les  âmes  du  purgatoire,  je  la  sens  en  mon  âme,  surtout 
depuis  deux  ans,  et  chaque  jour  je  la  sens  et  la  vois  plus  clairement.  Mon  âme  demeure  en  ce 
corps comme dans un purgatoire (1)... » 
\bigskip
P  502 :  « Il  y  a,  en  tout  cas,  entre  thème  et  phore,  une  relation  asymétrique  qui  naît  de  la  place 
qu'ils occupent dans le raisonnement. 
\bigskip
En outre, pour qu'il y ait analogie, thème et phore doivent appartenir à des domaines différents : 
lorsque les deux rapports que l'on confronte appartiennent à un même  domaine, et peuvent être 
subsumés sous une structure commune, l'analogie fait place à un raisonnement par l'exemple ou 
l'illustration, thème et phore fournissant deux cas particuliers d'une même règle. Aussi, tandis que 
certains  raisonnements  se  présentent  indiscutablement  comme  des  analogies  (c'est  le  cas  très 
souvent  lorsque  le  phore  est  pris  au  domaine  sensible,  le  thème  au  domaine  spirituel),  d'autres 
donnent lieu à cet égard à quelque doute, tel ce passage de Colette concernant ses relations  avec 
une troupe de passereaux : 
\bigskip
...  le  temps  n'était  pas  loin  où  dans  une  petite  foule  indistincte  j'allais  découvrir  l'individu,  le 
singulier, le préféré qui me préférerait. Chaque fois le danger, avec l'animal, se fait le même pour 
nous.  Choisir,  être  choisi,  aimer  :  tout  de  suite  après  viennent  le  souci,  le  péril  de  perdre,  la 
crainte  de  semer  le  regret.  De  si  grands  mots  au  sujet  d'un passereau  ? Oui,  d'un  passereau,  Il 
n'est pas, en amour, de petit objet (2). » 
\bigskip
(1) Sainte Catherine de Gênes, Œuvres, Traité du Purgatoire, chap. XVII, p. 150. 
(2) Colette, Le fanal bleu, p. 34. 
\bigskip
P  502-503 :  « S'agit-il  d'une  analogie  avec  l'amour  humain  ?  S'agit-il  d'un  exemple  conduisant  à 
généralisation ? Les derniers mots tendraient à faire préférer cette interprétation. Sans eux  nous 
étions incités à voir en tout ce passage un développement analogique dont l'amour humain était le 
thème. Le flottement entre les deux formes de raisonnement peut d'ailleurs être, dans certains cas, 
efficace. L'impression que l'on a affaire à deux domaines différents peut dépendre des dispositions 
de  l'auditeur.  Mais  l'assimilation  ou  la  séparation  des  domaines  est  souvent  préparée  par  le 
discours : le choix des termes utilisés est essentiel, et nous pourrions reprendre, à ce propos, nos 
remarques  antérieures  concernant  les  différences  de  nature  ou  de  degré.  Tout  ce  qui  pose  une 
\bigskip
\bigskip
\bigskip
265 
\bigskip
différence  de  nature,  d'ordre,  tend  à  instituer  des  domaines  séparés  dans  lesquels  pourront  se 
situer respectivement phore et thème : l'opposition entre le fini et l'infini est une différence d'ordre 
qui sera propice au raisonnement analogique. » 
\bigskip
P 503 : « On peut se demander si, à l'intérieur d'une même discipline, se rencontrent des analogies 
à  proprement  parler.  La  question  doit,  pensons-nous,  être  résolue  affirmativement,  mais  elle  est 
fort  délicate.  Les  biologistes  utilisent  deux  notions  susceptibles  d'éclairer  ce  problème  : 
l'homologie (ex. bras et aile), l'analogie (ex. similitudes provoquées par la vie aquatique chez des 
individus de genres différents). Dans le premier cas, nous avons un thème structural qui constitue 
un système naturel, englobant et subsumant les cas individuels apparentés, système déterminé àla 
fois  par  l'anatomie,  l'embryologie,  la  paléontologie,  qui  unit  les  individus  en  un  même  domaine. 
Dans  le  second  cas  la  pensée  va  d'un  genre  animal  à  un  autre,  considérés  dans  leur  isolement 
relatif. » 
\bigskip
P  503-504 :  « En  droit,  le  raisonnement  par  analogie  proprement  dite  se  limite,  semble-t-il,  à  la 
confrontation,  sur  des  points  particuliers,  entre  droits  positifs  distincts  par  le  temps,  l'espace 
géographique ou la matière traitée. Par contre, chaque fois que l'on recherche des similitudes entre 
systèmes,  on  considère  ceux-ci  comme  exemples  d'un  droit  universel  ;  de même  chaque fois que 
l'on  argumente  en  faveur  de  l'application  d'une  règle  déterminée  à  de  nouveaux  cas,  on  affirme, 
par là-même, que l'on est à l'intérieur d'un seul domaine: aussi, la réhabilitation de l'analogie, en 
tant que procédé d'interprétation extensive, qui répond au désir de certains j juristes de voir en elle 
autre chose que le terme par lequel on disqualifie ce que l'adversaire présente comme exemple, se 
réalisera-t-elle  en  donnant  à  l'analogie  une  signification  différente  de  celle  que  nous  avons 
proposée (1). » 
\bigskip
(1) Cf. notamment N. Bobbio, L'analogia nella logica del diritto, spécialement p. 34. 
\bigskip
§ 83. RELATIONS ENTRE LES TERMES D'UNE ANALOGIE 
\bigskip
P 504 : « En disant que dans toute analogie il y a un rapport entre quatre termes, nous présentons 
évidemment  une  vue  schématisée  des  choses.  Chacun  d'eux  peut  en  effet  correspondre  à  une 
situation complexe, et c'est même là ce qui caractérise une analogie riche. 
\bigskip
Le fait qu'il s'agit de similitude de relations autorise, entre les termes du thème et ceux du phore, 
des différences aussi importantes que l'on voudra. La nature des termes est, à première vue tout au 
moins,  secondaire.  C'est  d'ailleurs,  bien  souvent,  le  rôle  qu'ils  jouent  dans  l'analogie  qui  seul 
précise leur signification. Lorsque Ézéchiel s'écrie : 
\bigskip
je mettrai dans leurs entrailles un esprit nouveau, et j'ôterai de leur corps le coeur de pierre et leur 
donnerai un coeur de chair (2). 
\bigskip
la chair est à la pierre comme la piété à l'insoumission; tandis que, dans maintes analogies, la chair 
est à l'esprit comme l'état de péché à l'état de grâce. Un même terme est donc conçu de manières 
fort différentes, pour pouvoir s'insérer dans des analogies de sens aussi opposes. 
\bigskip
Bien que l'analogie-type comporte quatre termes, il arrive assez fréquemment que leur nombre se 
réduise à trois ; l'un d'entre eux figure deux fois dans le schème, lequel devient : B est à A ce que C 
est à B. 
\bigskip
(2) Ezéchiel, XI, 19. 
\bigskip
P 505 : « En voici un exemple tiré de Leibniz : 

...  toutes  les  autres  substances  dépendent  de  Dieu  comme  les  pensées  emanent  de  nostre 
substance (1)... 
\bigskip
et cet autre, attribué à Héraclite : 
\bigskip
L'homme, au regard de la divinité, est aussi puéril que l'enfant l'est au regard de l'homme (2). 
\bigskip
Le  terme  commun  «substance  »,  «homme  »,  invite  à  situer  le  thème  dans  le  prolongement  du 
phore,  et  à  les  hiérarchiser.  Mais  la  distinction  des  domaines,  indispensable  pour  l'existence  de 
l'analogie, est néanmoins maintenue : car le terme commun, tout en étant formellement le même 
dans le thème et dans le phore, se dissocie par son usage différencié, qui le rend équivoque. Il était 
en effet à prévoir que le terme commun, puisque sa place dans le phore et dans le thème le met en 
relation  avec  des  termes  appartenant  à  deux  domaines  différents,  prenne  par  le  fait  même  des 
significations plus ou moins divergentes. 
\bigskip
On  pourrait  en  conclure  que  toute  analogie  à  trois  termes  peut  s'analyser  en  analogie  à  quatre 
termes. Il est bon cependant de distinguer les analogies où phore et thème se mettent en quelque 
sorte  dans  le  prolongement  l'un  de  l'autre,  de  celles  où  l'accent  porte  plutôt  sur  le  parallélisme 
entre eux. En effet l'interprétation argumentative pourra en être fort différente. » 
\bigskip
(1) Leibniz, éd. Gerhardt, 4e vol. : Discours de métaphysique, XXXII, p. 457. 
(2) P. Grenet, Les origines de l'analogie philosophique dans les dialogues de Platon, p. 108, n. 367. 
(Fragments Diels, 79, Bywater, 97.) 
\bigskip
P 505-506 : « Deux analogies, empruntées à l'ouvrage de Gilson sur le thomisme, le montreront. 
Voici la première : 
\bigskip
Lorsqu'un maître instruit son disciple, il faut que la science du maître contienne ce qu'il introduit 
dans l'âme de son disciple. Or la connaissance naturelle que nous avons des principes nous vient 
de Dieu, puisque Dieu est l'auteur de notre nature. Ces principes sont donc, eux aussi, contenus 
dans la sagesse de Dieu. D'où il suit que tout ce qui est contraire à ces principes est contraire à la 
sagesse divine et, par conséquent, ne saurait venir de Dieu (1). » 
\bigskip
(1) E. Gilson, Le thomisme, p. 31 (contra Gentiles, I, 7). 
\bigskip
P 506 : « Et voici la seconde : 
\bigskip
Comme  un  enfant  qui  comprend  ce  qu'il  n'aurait  pu  découvrir,  mais  u  1  u  maître  lui  enseigne, 
l'intellect  humain  s'empare  sans  peine  '  une  doctrine  dont  une  autorité  plus  qu'humaine  lui 
garantit la vérité (2). 
\bigskip
Dans  les  deux  cas,  nous  avons  un  phore  pris  au  domaine  de  la  vie  journalière,  celui  de 
l'enseignement  ;  dans  les  deux  cas,  il  existe  une  différence  de  valeur  considérable  et  entre  les 
termes de chaque domaine et entre les deux domaines envisagés. Mais dans le premier cas, ce n'est 
pas cette différence qui surtout importe. Aussi percevons-nous plutôt le parallélisme entre les deux 
relations (la sagesse de Dieu est à la connaissance naturelle comme la science du maître à celle du 
disciple).  Dans  le  second  cas,  par  contre,  les  différences  de  valeur  importent  avant  tout.  Et  nous 
percevons  plutôt  une  analogie  à  trois  termes  hiérarchisés  (l'autorité  divine  est  pour  l'intellect 
humain  ce  que  le  maître  est  pour  l'enfant)  et  cela  bien  que  le  terme  commun  ne  soit  pas 
formellement identique «( maître », « intellect humain ») (3). 
\bigskip
Ajoutons  que,  à  côté  des  analogies  à  trois  termes  hiérarchisés,  se  rencontrent  des  analogies 
répondant au schème : A est à B ce que A est à C. En voici un joli exemple pris chez Démosthène : 
\bigskip
\bigskip
\bigskip
267 
\bigskip
 
Car  l'argent,  lorsqu'on  le  met,  comme  dans  une  balance,  à  côté  d'une  décision,  l'entraîne 
brusquement et tire à lui le raisonnement, si bien que le preneur devient incapable de calculer rien 
sainement (4). » 
\bigskip
(2) Ibid., P. 35. 
(3) Cf. chez Pascal une analogie à quatre termes et une analogie à trois termes hiérarchisés dans un 
même raisonnement. Bibl. de la Pléiade, Pensées, 452 (130), 13. 958 (234 éd. Brunschvicg). 
(4) Démosthène, Harangues t. II : Sur la paix, § II. 
\bigskip
P  506-507 :  « L'argent  bien  que  pris  uniquement  au  sens  matériel,  remplit  ici  deux  fonctions 
différentes.  L'orateur  utilise  en  quelque  sorte  une  coïncidence  heureuse  ;  c'est  elle  qui  permet  la 
fusion entre un des termes du phore et un des termes du thème. Nous verrons ultérieurement en 
quoi pareille analogie se rapproche de certaines métaphores. » 
\bigskip
P  507 :  « L'essentiel,  dans  une  analogie,  c'est  la  confrontation  du  thème  avec  le  phore  ;  elle 
n'implique pas du tout qu'il y ait un rapport préalable entre les termes de l'un et de l'autre. Mais 
quand il existe un rapport entre A et C, entre B et D, l'analogie se prête à des développements en 
tous  sens  qui  sont  l'un  des  aspects  d'une  analogie  riche.  Tarde  se  plaisait  à  développer  des 
analogies  d'une  surprenante  ampleur  où  les  relations  entre  termes  homologues  ne  le  cédaient 
presque  en  rien  aux  relations  à  l'intérieur  des  thème  et  phore  (1).  Ces  relations  entre  termes 
homologues  sont  parfois  même  à  l'avant-plan  :  l'analogie  est  pensée  avant  tout  comme  affinité 
entre termes du thème ei du phore. La similitude de structure entre domaines en est inférée. C'est 
par ce biais que Girolamo Fracastoro affirme, en plein xvie siècle, la multiplicité et la spécificité des 
agents de nos maladies infectieuses (2). » 
\bigskip
(1)  cf. notamment Tarde, La Logique sociale, pp. 98-99. 
(2) G. Fracastoro, Opera omnia, De sympathla et antipathia rerum, chap. II. De analogia rerum in 
agendo, pl). 65 b et su' De contagione, chap. 8 : De analogia contagionum, pp. 81 a et suiv. Cf. aussi 
! Sifilide, liv. I, vv. 258-306. 
\bigskip
P 507-508 : « Les doubles hiérarchies, avec les rapports complexes qui les caractérisent, rapports 
horizontaux  basés  sur  la  structure  du  réel,  rapports  verticaux  de  hiérarchisation,  se  prêtent 
particulièrement  à  l'établissement  d'analogies  riches.  La  distinction  entre  double  hiérarchie  et 
analogie est profonde selon nous ; la première est basée sur une liaison du réel ; la seconde suggère 
la  confrontation  de  relations  situées  dans  des  domaines  différents.  Mais  on  peut  très  souvent 
argumenter par analogie en répartissant les termes successifs d'une double hiérarchie, entre thème 
et phore. C'est ainsi que, la double hiérarchie concluant de la supériorité de Dieu sur les hommes à 
la supériorité de la justice divine sur la justice humaine, peut faire place à l'analogie selon laquelle 
la  justice  divine  est  par  rapport  à  Dieu  ce  que  la  justice  humaine  est  pour  les  hommes. 
Inversement,  lorsque  l'analogie  développe  deux  longues  hiérarchies  appartenant  l'une  au  phore, 
l'autre au thème, et que les deux domaines sont d'inégale valeur, l'analogie pourrait aisément faire 
place  à  une  série  de  doubles  hiérarchies.  C'est  le  cas  notamment  lorsque  Plotin  tire,  de  l'ordre 
hiérarchique qui existe dans un cortège royal, des conclusions au sujet des réalités dépendant de 
l'Un et qui en sont plus on moins proches (1). » 
\bigskip
(1) Plotin, Ennéades, V, 5, § :3. 
\bigskip
P 508 : « Bien que l'analogie soit un raisonnement qui concerne des relations, celles qui existent à 
l'intérieur du phore et à l'intérieur du thème, ce qui fait qu'elle diffère profondément de la simple 
proportion mathématique c'est que la nature des termes, dans l'analogie, n'est jamais indifférente. 
Il  s'établit,  en  effet,  entre  A  et  C,  entre  B  et  D,  grâce  à  l'analogie  même,  un  rapprochement  qui 
\bigskip
\bigskip
\bigskip
268 
\bigskip
conduit  à  une  interaction,  et  notamment  à  la  valorisation,  ou  la  dévalorisation,  des  termes  du 
thème. 
\bigskip
Voici un exemple éclairant le mécanisme de cette interaction 
\bigskip
...  et  ceste  election  d'Aymé  [due  de  Savoie],  solennellement  parfaite  par  l'authorité  du  sacré  et 
general  concile,  s'en  alla  en  fumee  :  sinon  que  ledit  Aymé  fut  appaisé  par  un  chappeau  de 
Cardinal, comme un chien abayant, par une piece de pain (2). 
\bigskip
La  dévaluation  des  termes  du  thème  est  entraînée  par  la  nature  des  termes  du  phore  ;  mais  la 
valeur de ceux-ci dérive elle-même, en partie au moins, de leur emploi dans l'analogie : l'attitude 
du chien aboyant n'est pas nécessairement l'objet d'un jugement dépréciatif. » 
\bigskip
(2) Calvin, Institution de la religion chrétienne. Au Roy de France, p. 13. 
\bigskip
P 509 : « Perdue de vue, l'interaction donne lieu à un de ces effets comiques dont Sterne était 
friand : 
\bigskip
C'est un brave [le roi William], pardieu! s'exclama mon oncle Toby et qui mérite la couronne 1 - 
Aussi dignement qu'un voleur la corde, hurla Trim [le caporal loyal] (1). 
\bigskip
L'interaction entre termes de l'analogie conduit souvent à intégrer dans la construction du phore 
des  éléments  qui  n'auraient  aucune  signification  si  l'on  ne  devait  penser  au  thème,  où  ils  en  ont 
une.  C'est  ainsi que  Locke,  pour  décrire  la  voie  qui  mène  au  salut,  se  sert d'un  phore  présentant 
une  route  conduisant  directement  àJérusalem  et  se  demande  pourquoi  le  pèlerin  doit  être 
maltraité,  parce  qu'il  ne  porte  pas  de  brodequins,  ou  parce  que  ses  cheveux  ne  sont  pas  coupés 
d'une certaine façon, ou parce que l'on suit ou ne suit pas un guide habillé de blanc ou couronné 
d'une  mitre  (2),  tous  ces  détails  n'ayant  d'importance  que  parce  qu'ils  font  penser  aux  conflits 
entre les adeptes de diverses Églises. 
\bigskip
Parfois, grâce à l'action du thème sur le phore, certains éléments de ce dernier sont modifiés. C'est 
ainsi  que  des  détails  concernant  des  personnages  de  l'Ancien  Testament,  Adam  ou  Moïse,  sont 
transformés pour mieux permettre à ces personnages de  préfigurer le Christ. Cette technique est 
attestée par Réau : 
\bigskip
Contrairement au texte de l'Exode où il est dit que Moïse retournant en Égypte chargea sa femme 
et  son  enfant  sur  son  âne,  nous  voyons  sur  un  des  panneaux  d'émail  du  rétable  de 
Klosterneuburg  (XIIe  siècle)  le  Prophète  à  califourchon  sur  l'âne  tandis  que  sa  femme  Sephora 
suit  à  pied...  Cette  variante  s'explique  tout  simplement  par  une  raison  typologique,  parce  qu'il 
s'agissait de mettre cette scène en parallèle avec l'Entrée du Christ àJérusalem. Il fallait donc que 
Moïse fût monté sur l'âne pour faire pendant à jésus dont il est la préfigure (3). » 
\bigskip
(1) Sterne, Vie et opinions de Tristram Shandy, liv. VIII, chap. XIX, p. 495.  
(2) Locke, The second treatise of civil qovernment and A letter concerning toleration, p. 138. 
(3) L. Réau, L'influence de la forme sur l'iconographie médiévale dans Formes de l'art, formes (le 
l'esprit, pp. 91-92. 
\bigskip
P 510 : « Souvent on doue les termes du phore de propriétés qui relèvent de la fantaisie mais qui 
les rapprochent du thème, tel le langage humain attribué aux animaux des fables. 
\bigskip
On voit très bien la modification apportée au phore par les besoins du thème dans cette analogie 
de Bossuet : 

Restait  cette  redoutable  infanterie  de  l'armée  d'Espagne,  dont  les  gros  bataillons  serrés, 
semblables à autant de tours, mais à des tours qui sauraient réparer leurs brèches, demeuraient 
inébranlables au milieu de tout le reste en déroute (1). ... 
\bigskip
Cette  description,  en  les  montrant  comme  des  tours  dans  une  forteresse  assiégée,  cherche  à 
caractériser  le  rôle  des  bataillons  dans  le  combat,  mais  sans  oublier  néanmoins  ce  qui  fait  leur 
supériorité. 
\bigskip
Il existe une technique, fréquemment utilisée par Plotin, pour rapprocher le phore du thème, et sur 
laquelle E. Bréhier avait attiré l'attention en la qualifiant de «correction d'images» (2): il ne s'agit 
pas d'une modification quelconque du phore, mais de sa purification, de sa modification après un 
énoncé préalable, dans le sens d'une plus grande perfection. 
\bigskip
Voici un exemple, entre bien d'autres, de ce procédé particulier: 
\bigskip
Ainsi un homme entre dans une maison richement ornée, regarde et admire toutes ces richesses, 
avant d'avoir vu le maître de la maison ; mais dès qu'il le voit, dès qu'il l'aime, ce maître qui n'est 
point une f roide statue, mais qui mérite réellement d'être contemplé, il laisse tout le reste pour le 
regarder uniquement... L'on conserverait peut être mieux l'analogie, si l'on disait que, au visiteur 
de la maison, se présente non plus un homme, mais un dieu, qui n'apparaît pas aux yeux du corps 
et remplit l'âme de sa présence (3). » 
\bigskip
(1) Bossuet,  Bibl. de la Pléiade, Oraisons funèbres Oraison funèbre de Louis de Bourbon, prince 
de Condé, p. 218. 
(2) Plotin, t. V, Notice, p. 129. 
(3) Plotin, t. VI, IIe Partie : Ennéade VI, 7, § 35. 
\bigskip
P  510-511 :  « Le  phore,  ainsi  modifié,  continue  néanmoins  à  exercer  l'action  souhaitée  et  la 
plausibilité  accordée  au  thème  n'en  est  pas  diminuée.  Souvent  cette  rectification  est  introduite  à 
titre d'hypothèse, et Plotin la fait précéder, comme dans l'exemple ci-dessus, du conditionnel « si 
l'on disait que ». 
\bigskip
P 511 : « C'est d'une hypothèse aussi que se sert Kant dans sa célèbre analogie de la colombe : 
\bigskip
La colombe légère, lorsque dans son libre vol, elle fend l'air dont elle sent la résistance, pourrait 
s'imaginer  qu'elle  réussirait  bien  mieux  encore  dans  le  vide.  C'est  justement  ainsi  que  Platon 
quitta le monde sensible parce que ce monde oppose à l'entendement trop (l'obstacles divers, et se 
risqua  au  delà  de  ce  monde,  sur  les  ailes  des  idées,  dans  le  vide  de  l'entendement  pur.  Il  ne 
remarqua pas que ses efforts ne lui faisaient point gagner de chemin, car il n'eut point, pour ainsi 
dire,  d'endroit  où  se  poser  et  de  support  sur  lequel  il  pût  se  fixer  et  appliquer  ses  forces  pour 
changer son entendement de place (1). 
\bigskip
Les  efforts  de  Platon  sont  assimilés  à  ceux  de  la  colombe,  et  décrits  en  termes  qui  rappellent 
celle-ci. Mais l'interaction se saisit, 
\bigskip
pour ainsi dire, sur le vif, parée que la démarche de la colombe n'est elle-même qu'une hypothèse 
conditionnée par le thème. 
\bigskip
Il arrive toutefois que la rectification du phore, le rende ridicule, 
\bigskip
parce  que  tout  à  fait  incompatible  avec  le  réel.  Quintilien  cite  des  expressions  qu'il  entendait 
répéter partout lorsqu'il était jeune : 

«  Même les sources des grands fleuves sont navigables », et « Un arbre vraiment fécond produit 
dès qu'il est planté » (2). 
\bigskip
(1) Kant, Critique de la raison pure, p. 45.  
(2) Quintilien, Vol. III , liv. VIII, chap. 111, § 76. 
\bigskip
P 511-512 : « Qu'est-ce sinon des analogies où le thème a réagi sur le phore de façon abusive ? Le 
désir de rapprocher le phore du thème, au lieu de rendre l'analogie convaincante, se retourne alors 
contre  l'orateur.  La  prudence  semble  s'imposer  quand  on  modifie  le  phore  :  on  peut  le  rendre 
fantastique, mais on ne peut affirmer, même à titre d'hypothèse, ce qui n'est qu'une contre-vérité. 
Il vaut mieux en pareil cas, tout en utilisant les termes du phore, faire nettement comprendre que 
la modification concerne le thème, comme dans ce passage de Bossuet où la pénitence est décrite 
par analogie avec un enfantement : 
\bigskip
Parmi ces travaux de la pénitence, Songez, mes frères, que vous enfantez; et ce que vous enfantez, 
c'est vous-même. si c'est une consolation si sensible d'avoir fait voir la lumière et donné la vie à 
un autre, qu'elle efface en un moment tous les maux passés, quel ravissement doit-on ressentir de 
s'être éclairé soi-même, et de s'être engendré soi-même pour une vie immortelle (1). » 
\bigskip
(1) Bossuet, Sermons, vol. II : Sur la pénitence, p. 83.  
\bigskip
§ 84. EFFETS DE L'ANALOGIE 
\bigskip
P 512 : « L'interaction entre thème et phore, qui résulte de l'analogie, - l'action sur le thème étant 
la  plus  marquée,  mais  l'action  inverse  n'étant,  nous  venons  de  le  voir,  nullement  négligeable,  se 
manifeste  de  deux  façons,  par  la  structuration  et  par  les  transferts  de  valeur  qui  en  dérivent  ; 
transferts  de  la  valeur  du  phore  au  thème  et  réciproquement,  transfert  de  la  valeur  relative  des  
deux termes du phore à la valeur relative des deux termes du thème. 
\bigskip
Voici une célèbre analogie d'Épictète : 
\bigskip
Qu'un enfant plonge le bras dans un vase d'une embouchure étroite, pour en tirer des figues et 
des noix, et qu'il en remplisse sa main, que lui arrivera-t-il ? Il ne pourra la retirer et pleurera. 
«Lâches-en quelques unes (lui dit-on), et tu retireras ta main. » Toi, fais de même pour tes désirs. 
Ne souhaite qu'un petit nombre de choses, tu les obtiendras (2). » 
\bigskip
(2) Epictète, Entretiens recueillis par Arrien, liv. III, chap. IX, p. 259. 
\bigskip
P 512-513 : « La conclusion normative sur la conduite de celui qui désire plus qu'il ne peut réaliser 
n'est que le transfert à son cas du jugement sur le comportement puéril de l'enfant ne parvenant 
pas à retirer du vase sa main trop remplie. Mais ce transfert résulte de ce que le comportement de 
l'adulte est reconstruit à partir du phore. Dans cet exemple, comme dans tous ceux où le phore est 
emprunté au domaine sensible et le thème au domaine spirituel, l'analogie permet de reconstruire 
le  thème  selon  une  structure  plausible,  reconstruction  d'autant  plus  utile  que  cette  structure  ne 
peut être connue directement. C'est ainsi que les discussions persistantes concernant les rapports 
du libre arbitre humain et de la grâce divine prenaient pour objet le phore de la vision, qui requiert 
des organes visuels et aussi une source de lumière : 
\bigskip
En effet, connue l'homme entouré de ténèbres très épaisses, quoique avant le sens de la vision, ne 
voit rien, car il ne peut rien voir, avant (lue ne vienne de l'extérieur la lumière, qu'il ressent même 
quand il garde les yeux fermés, et qu'il aperçoit, ainsi que tout ce qui l'entoure, quand il les ouvre; 
ainsi la volonté de l'homme, aussi longtemps qu'il se trouve dans l'ombre du péché originel et des 
siens  propres  est  entravée  par  ses  propres  ténèbres.  Mais  quand  apparaît  la  lumière  de  la 
miséricorde  divine,  non  seulement  elle  détruit  la  nuit  des  péchés  et  leur  culpabilité,  mais 
\bigskip
\bigskip
\bigskip
271 
\bigskip
guérissant la volonté malade, elle lui ouvre la vue et la rend apte à contempler cette lumière en la 
purifiant par les bonnes œuvres (1). » 
\bigskip
(1) Joannis, Scotti, Liber de Praedestinatione, IV, 8. Patrol. latine, I. CXXII, col. 374-375. 
\bigskip
P  413 :  « L'analogie  permet  de  mieux  comprendre  les  rapports  de  la  grâce  et  du  libre  arbitre  et 
l'importance respective de l'homme et de Dieu dans le péché et le salut. 
\bigskip
La valeur des termes est bien souvent déterminée par la structure de l'analogie. C'est ainsi que les 
analogies  à  trois  termes,  le  drame,  la  vie  terrestre  et  la  vie  supra-terrestre,  visent  à  enlever  tout 
sérieux à la vie terrestre par rapport à l'au-delà, en faisant d'elle une espèce de jeu, de spectacle, oh 
chacun joue son rôle en attendant que commence la vie véritable (2). » 
\bigskip
(2) Cf. Plotin, Ennéades, 111, 2, § 15. 
\bigskip
P  513-514 :  « Certains  termes,  tels  lumière,  hauteur,  profondeur,  Plein,  vide,  creux,  quoique 
empruntés au monde physique, semblent chargés de valeur au départ. Il est possible qu'il en soit 
ainsi.  Mais  peut-être  aussi  ont-ils  déjà  tant  de  fois  servi  comme  éléments  du  phore,  dans  des 
analogies  dont  le  thème  relevait  du  monde  spirituel,  qu'on  ne  peut  plus  détacher  d'eux  la  valeur 
qui  résulte  de  ce  rôle,  par  suite  de  l'interaction  avec  certains  termes  du  thème.  Parfois  on  croit 
pouvoir saisir sur le vif la manière dont s'opère le transfert de valeur, mais on n'en est jamais bien 
sûr,  témoin  cette  analogie  de  Plotin  dont  le  phore  est  constitué  par  le  rapport  du  centre  à  la 
circonférence : 
\bigskip
En dehors de lui [le Premier], se trouvent la raison et l'intelligence, qui l'entourent en le touchant 
et se suspendent à lui; ou plutôt elle n'est intelligence que parce qu'elle le touche... L'on reconnaît 
qu'un cercle tire ses propriétés de son centre, parce qu'il touche ce centre ; il en reçoit en quelque 
sorte la forme, en tant que ses rayons, convergeant au centre, sont, par celle de leur extrémité qui 
est du côté du centre, comme le centre même auquel ils aboutissent et dont ils sortent; (1)... » 
\bigskip
(1) Plotin, t. VI, II, Partie Ennéade Vl, 8, § 18. 
\bigskip
P 514 : « Si le cercle donne sa structure au monde plotinien, l'Un qui en est le principe, en arrive, 
grâce à l'interaction, et quelles que soient les raisons géométriques que l'on en fournit, à valoriser 
l'idée de centre, qui reste chargée de valeur dans notre civilisation. » 
\bigskip
P  514-515 :  « Les  paraboles,  les  paradigmes,  que  l'on  trouve  à  profusion  dans  la  Bible,  dans  les 
écrits  platoniciens,  ne  sont  pas  nécessairement  empruntés  au  domaine  matériel.  Ils  peuvent  être 
empruntés à la vie quotidienne, pour éclairer des aspects de la vie sociale, politique ou morale, leur 
donner une certaine structure et une certaine valeur. Voici une analogie de cette espèce, tirée d'un 
discours de Démosthène : 
\bigskip
Et vous savez fort bien aussi que tout ce que les Grecs subissent de la part des Lacédémoniens on 
de nous-mêmes, c'étaient du moins des fils légitimes de la Grèce qui le lui infligeaient. Il se passait 
alors ce qui a lieu dans une noble maison, où un fils légitime fait mauvais usage de sa fortune. On 
estime, certes,  qu'en  cela il mérite d'être blâmé et  accusé ; mais  qu'il n'ait  pas de droits sur ces 
biens  ou  n'en  soit  pas  l'héritier  légitime,  c'est  ce  qu'on  ne  peut  soutenir.  Au  contraire,  qu'un 
esclave  ou  un  enfant  supposé  dissipe  et  gaspille  un  bien  auquel  il  n'a  pas  droit,  combien,  par 
Hercule, cela serait jugé par tous plus ce scandaleux, plus intolérable ! Mais, au sujet de Philippe 
et de qu'il fait actuellement, oh on a d'autres sentiments ; oui, pour cet homme, qui non seulement 
n'est pas un Grec et n'a rien de commun avec les Grecs, mais n'est pas même un barbare d'une 
origine honorable (1) ... » 
\bigskip
\bigskip
\bigskip
\bigskip
272 
\bigskip
(1) Démosthène, Harangues, t. Il : Troisième Philippique, §§ 30, 31. 
\bigskip
P  515 :  « La  place  de  Philippe  dans  le  monde  grec  étant  fixée  par  l'analogie,  les  sentiments  de 
mépris et d'indignation à l'égard de sa conduite lie peuvent qu'en être renforcés, mais à condition 
évidemment  que  l'on  soit  déjà  au  préalable  convaincu  de  ce  que  Philippe  n'est  pas  tout  à  fait  un 
Grec.  A  ce  point  de  vue,  on  pourrait  établir,  entre  analogies,  des  différences  selon  le  degré 
d'adhésion préalable au thème. Certaines analogies joueraient, comme c'est le cas de l'illustration, 
un  rôle  de  renfort  ;  d'autres,  qui  devraient  jouir  par  elles-mêmes  d'une  plus  grande  force  de 
persuasion,  rempli  raient  un  rôle  plus  voisin  de  celui  de  l'exemple.  Mais  n'oublions  pas  que  ce 
rapprochement avec illustration et exemple n'est lui-même qu'une analogie. 
\bigskip
Un  des  effets  de  l'analogie  est  de  contribuer  à  déterminer  l'un  ou  les  deux  termes  du  thème. Cet 
usage  est  le  plus  fréquent  dans  les  analogies  à  trois  termes,  dont  la  structure  serait  :  B  est  à  X, 
comme C est à B. Pour faire comprendre la nature du verbe divin, Plotin se sert de l'analogie que 
voici : 
\bigskip
Comme  le  langage  parlé,  comparé  au  langage  intérieur  de  l'âme  se  fragmente  en  mots,  le 
langage de l'âme qui traduit le Verbe divin est fragmentaire si on le compare au Verbe (2). » 
\bigskip
(2) Plotin, Ennéades, 1, 2, § 3. 
\bigskip
P 515-516 : « Il arrive néanmoins que les deux termes du thème soient inconnus, et que seules les 
relations  supposées  entre  le  domaine  du  thème  et  celui  du  phore  permettent  d'en  préciser  la 
structure.  On  raisonne  sur  Dieu  et  ses  propriétés  en  se  basant  sur  des  rapports  connus  entre 
l'homme et ses propriétés, ainsi que sur l'idée que l'on se forme de la distance qui sépare Dieu de 
l'homme : c'est quand on admet que la bonté  divine et la bonté humaine  ne font  pas  partie d'un 
même  domaine  du  réel,  que  l'on  dira  qu'il  n'y  a  pas  entre  ces  deux  propriétés  de  rapport  de 
ressemblance, malgré leur désignation au moyen d'un même concept, mais uniquement un rapport 
analogique. » 
\bigskip
P  516 :  « L'idée  qu'il  existe  deux  domaines  est  souvent  garantie  par  des  notions  telles  image, 
ombre, projection, qui sont d'ailleurs ellesmêmes analogiques. La relation entre les deux domaines 
peut  être  telle  qu'elle  entraîne  une  inversion  de  certaines  structures.  Maritain  décrit  le  destin 
d'Israël par analogie avec celui de l'Église; il précise, en réponse à un adversaire que choque cette 
sorte d'analogie renversée : 
\bigskip
Nous  avons  dit  que  c'est  une  Église  précipitée,  et  que  sa  vocation,  devenue,  par  sa  faute, 
ambivalente, continue la nuit du monde; et nous avons averti que ces choses doivent s'entendre 
d'une  manière  analogique...  Israël  n'est  pas  surnaturellement  étranger  au  monde  de  la  même 
façon que l'Église (1)... 
\bigskip
Il arrive d'ailleurs que l'on soit obligé d'inventer le thème, parce que, ne pouvant comprendre les 
termes  du  discours  au  propre,  on  est  amené  à  leur  donner  un  sens  figuré,  à  rechercher  donc  le 
thème, à réinventer l'analogie qui donnerait au discours son sens véritable : 
\bigskip
Quand la parole de Dieu, qui est véritable, est fausse littéralement, elle est vraie spirituellement. 
Sede a dextris meis [Ps. CIX], cela est faux littéralement; donc cela est vrai spirituellement (2). 
\bigskip
Puisque le discours ne peut être que véridique, en raison de la qualité de celui dont il émane, il faut 
que  le  lecteur  retrouve  le  thème,  l'esprit  du  phore  qui  correspondrait  aux  intentions  de  l'auteur. 
Cette  recherche  peut  donner  lieu  à  créations  nouvelles,  en  matière  éthique,  esthétique  ou 
religieuse. » 
\bigskip
\bigskip
\bigskip
\bigskip
273 
\bigskip
(1) J. Maritain, Raison et raisons, pp. 212, 213. 
(2) Pascal, Bibl. de la Pléiade, Pensées, 555 (31), p. 1003 (687 éd. Brunschvicg). 
\bigskip
P 516-517 : « Notons que la réalité physique, ou historique, du phore n'est pas nécessairement niée 
quand l'interprétation littérale est jugée insuffisante : dire que le repos de Dieu au septième jour 
peut être interprété comme une analogie indiquant la distance qui separe le créateur de l'univers, 
le recul qu'il prend par rapport à son œuvre (1), ne préjuge pas de la réalité du récit biblique. Nous 
savons  en  effet  que  le  phore  d'une  analogie  est  souvent  pris  au  domaine  du  réel,  et  que,  d'autre 
part,  l'œuvre  de  fiction  peut  avoir  ou  n'avoir  pas  de  portée  analogique.  Tel  poème  d'amour  de 
Chaucer  sera,  pour  certains,  la  confession  voilée  d'un  amour  réel,  pour  d'autres  une  création 
analogique dont le thème serait la mort d'une princesse (2). » 
\bigskip
(1) E. Bevan, Symbolism and Belief, pp. 121-122. 
(2) Marshall W. Stearns, A note on Chaucer's attitude toward love, Speculum, vol. II, 1942, pp. 570 
à 574. 
\bigskip
P 517 : « La quête du sens analogique, qui serait le sens profond, résulte parfois non du fait que le 
sens littéral est faux, peu intéressant, mais de raisons d'une autre nature, les conventions du genre, 
de l'époque, on ce que l'on sait par ailleurs des intentions de l'auteur. 
\bigskip
Certaines techniques  peuvent,  d'autre  part,  inciter  à  considérer  lin  énoncé  comme  analogique  :  - 
l'emploi de phores multiples (3), l'emploi de phores grossiers ou naïfs (4). » 
\bigskip
(3) Cf. Richard D. D. Whately, Elements of Rhetoric, p. 361 (appendix to p. 67.) 
(4) Cf. J. Guitton, Le temps et l'éternité chez Plotin et saint Augustin, pp. 154-155. 
\bigskip
§ 85. COMMENT ON UTILISE L'ANALOGIE 
\bigskip
Les analogies jouent un rôle important dans l'invention et  l'argumentation à cause essentiellement 
des développements des prolongements qu'elles favorisent : à partir du phore, elles p mettent de 
structurer le thème, qu'elles situent dans un cadre conceptuel. Ainsi, nous dit T. Swann Harding, 
qui  songeait  d'ailleurs  avant  tout  au  rôle  du  langage,  les  savants  qui  les  premiers  ont  décrit 
l'électricité  comme  un  «courant  »  ont  pour  toujours  donné,  dans  ce  domaine,  une  forme  à  la 
science (5). » 
\bigskip
(5) T. Swann Harding, Science at the Tower of Babel, Philosophy of Science, .July 138. p. 347. 
\bigskip
P  518 :  « Cette  forme  résulte  de  ce  que  le  rapprochement  entre  phénomènes  électriques  et 
hydrauliques a donné lieu à des développements qui précisent, complètent, prolongent l'analogie 
primitive. Mais jusqu'où une analogie peut-elle être prolongée ? 
\bigskip
En tous domaines le développement d'une analogie est normal, et cela dans toute la mesure oh l'on 
en a besoin et où rien ne s'y oppose. Comme le dit très justement Richards, il n'y a pas de totalité à 
une analogie, nous pouvons en user autant que de besoin, au risque de la voir s'écrouler (1). 
\bigskip
C'est  dans  les  développements  de  l'analogie  que  son  rôle  d'invention  et  son  rôle  de  preuve  se 
séparent  :  alors  que,  en  se  plaçant  au  premier  point  de  vue,  rien  n'empêche  de  prolonger  une 
analogie  aussi  loin  que  possible,  pour  voir  ce  que  cela  donnera,  au  point  de  vue  de  sa  valeur 
probante, elle doit être maintenue dans des limites que l'on ne saurait dépasser sans dommage, si 
l'on désire renforcer une conviction. Développer une analogie, c'est parfois confirmer sa validité ; 
c'est aussi s'exposer aux coups de l'interlocuteur. 
\bigskip
Dans certains cas, l'analogie est développée sans que l'or marque la moindre rupture entre elle et 
ses prolongements, telle cette analogie où Kant compare sa philosophie à celle de Hume. 

... celui-ci [Hume] même ne pressentait en rien la possibilité de cette science formelle, ayant 
\bigskip
amené,  pour  la  mettre  en  sécurité,  la  barque  à  la  rive  (le  scepticisme)  où elle  peut  demeurer  et 
pourrir,  tandis  qu'il  m'importe  de  fournir  à  cette  barque  un  pilote  qui,  suivant  les  principes 
certains de son art, tirés de la science du globe, pourvu d'une carte maritime complète et d'une 
boussole, puisse la conduire sûrement où il lui plaira (2). » 
\bigskip
(1)  A. Richards, The Philosophy of Rhetoric, p. 133. 
(2) Kant, Prolégomènes à toute métaphysique future, p. 15. 
\bigskip
P  518-519 :  « Phore  et  thème  se  développent  ici  concuremment,  sans  que  rien  ne  sépare  les 
relations  successivement  évoquées.  Les  moments  ultérieurs  renforcent  cependant  l'analogie  de 
début  ;  il  en  est  ainsi  dans  toute  analogie  qui  se  prolonge  et  sur  le  développement  de  laquelle 
l'auteur semblait avoir compté. » 
\bigskip
P 519 : « D'autres arguments par analogie se présentent en deux phases, dont la deuxième fournit 
la conclusion principale, comme dans ce passage de La Bruyère :  
\bigskip
Les roues, les ressorts, les mouvements sont cachés; rien ne paraît d'une montre que son aiguille, 
qui  insensiblement  s'avance  et  achève  son  tour:  image  du  courtisan,  d'autant  plus  parfaite 
qu'après avoir fait assez de chemin, il revient souvent au même point d'où il est parti (1). » 
\bigskip
(1) LA Bruyère, Œuvres Bibl. de la Pléiade, Caractères, De la Cour, 65, p. 257. 
\bigskip
P  519 :  « L'expression  «  d'autant  plus  parfaite  »  est  à  retenir;  elle  indique  que  l'analogie  est 
meilleure qu'on ne l'avait supposé; et ce développement s'accompagne souvent - comme c'est le cas 
ici d'un effet inattendu et même comique. 
\bigskip
Parfois,  des  phases  dans  l'argumentation  se  marquent,  en  ce  que  l'on  profite  du  fait  qu'une 
analogie  semble  admise  pour  demander  que  l'on  en  admette  aussi  le  développement.  Dans  son 
traité d'épistémologie génétique, Piaget, après avoir montré qu'il existe une analogie entre les idées 
professées au sujet de l'évolution et celles qui concernent la théorie de la connaissance, ajoute : 
\bigskip
Si la correspondance terme à ternie entre les thèses lamarckiennes et les thèses associationnistes 
ou  empiristes  est  exacte,  il  faut  s'attendre  à  la  retrouver  entre  les  objections  elles-mêmes, 
adressées à ces deux sortes d'interprétations (2). 
\bigskip
et il s'étonne de ce que des biologistes antilamarckiens puissent soutenir un empirisme radical, 
\bigskip
comme si l'intelligence pouvait alors, contrairement au reste de l'organisme, ne posséder aucun 
pouvoir d'activité interne (3)... 
\bigskip
C'est  le  prolongement  de  l'analogie  qui  a  ici  valeur  argumentative  et  permet  de  formuler  une 
objection aux vues empiristes. » 
\bigskip
(2) J. Piaget, Introduction à l'épistémologie génétique, 111, p. 102. 
(3) J. Piaget, ibid. 
\bigskip
P 520 : « Il arrive que l'analogie, au lieu d'être prolongée par l'auteur, le soit par son critique, qui 
en tire un moyen de réfutation, d'autant plus efficace que le matériel conceptuel a été emprunté à 
l'adversaire. C'est ainsi que Berriat Saint-Prix, en face d'un juriste qui, méprisant toute référence 
au droit romain et à l'ancienne jurisprudence, prétendait décrire dans un ouvrage sur le Code Civil 
\bigskip
\bigskip
\bigskip
275 
\bigskip
« les veines, les muscles, les traits et l'âme de la loi » regrette que l'auteur n'ait pas « suivi jusqu'au 
bout sa métaphore » : 
\bigskip
... il aurait bientôt aperçu que tout être vivant reçoit son organisation d'un être antérieur qui l'a 
engendré (1). 
\bigskip
Ce  mode  de  réfutation  suppose  que  l'on  a  toujours  le  droit  de  prolonger une  analogie  au  delà  de 
l'affirmation première et que si, par suite de ce prolongement, elle se tourne *contre son auteur, ou 
devient inadéquate, c'est qu'elle l'était déjà dès le début (2). En fait, presque toujours une pareille 
réfutation serait possible, mais quelle en serait la valeur ? La réfutation n'est jamais contraignante, 
car on pourrait refuser d'admettre ce prolongement ; elle met cependant en évidence la fragilité et 
l'arbitraire de l'analogie primitive ; c'est là son principal intérêt. » 
\bigskip
(1) F. Berriat, Saint-Prix, Manuel de logique juridique, p. 66, notes. 
(2) A propos de ce que Platon considère comme faux paradigmes, cf. V. Goldschmidt, Le 
paradigme dans la dialectique platonicienne, pp. 38-39. 
\bigskip
P  520-521 :  « Il  arrive  d'ailleurs  que  l'auteur  prenne  les  devants,  montre  ce  qu'il  y  a  d'inadéquat 
dans une analogie, et développe sa thèse comme le contrepied d'une analogie possible. Cet exposé 
utilise  ce  que  les  Anciens  appelaient  similitude  par  les  contraires,  laquelle  n'est  pas,  comme 
l'affirme la Rhétorique à Herennius, un simple ornement dont on pourrait aisément se passer. On 
pourra en juger par l'exemple emprunté à ce même ouvrage : 
\bigskip
Non, à dire vrai, il n'en est pas comme dans la carrière, où celui qui reçoit le flambeau ardent est 
plus agile dans la course par relais, que celui dont il le reçoit; le nouveau général, qui reçoit une 
année, n'est pas supérieur à celui qui se retire; car c'est le coureur fatigué qui remet le flambeau 
à un. coureur tout frais; ici c'est un général expérimenté qui remet son armée à un général sans 
expérience (1). » 
\bigskip
(1) Rhétorique à Herennius, liv. IV, § 59. 
\bigskip
P 521 : « On occupe immédiatement, au sens militaire du terme, l'esprit de l'auditeur, en montrant 
ce  qu'il y  aurait  de  faux  dans  une  idée  qui  pourrait  surgir  spontanément.  Reste  à  savoir  s'il  était 
opportun de mettre en vedette cet argument encore informulé par l'interlocuteur. Oui, sans doute, 
dans  la  mesure  où  on  parvient  à  suggérer  que  la  thèse  combattue  n'est  fondée  que  sur  ce 
raisonnement par analogie que l'on prend soin de réfuter. 
\bigskip
Parfois, pour réfuter une analogie, on est amené à l'amender, en la retournant pour ainsi dire, en 
décrivant  comment  serait  le  phore,  si  le  thème  était  convenablement  conçu.  Il  ne  s'agit  pas 
seulement  d'une  correction  du  phore  visant  à  le  rendre  plus  adéquat  au  thème  au  risque  de 
l'éloigner  de  la  réalité  (2)  ;  c'est  l'ensemble  de  l'analogie  qui  est  amendé.  Le  rôle  des  termes  du 
phore devient cependant très important, parce que leur choix n'est plus libre et que ce sont eux qui 
déterminent  les  relations  qui  pourront  être  mises  en  évidence  pour  amender  l'analogie.  Nous 
voyons  cette  technique  à  l'œuvre  chez  Mill,  à  propos  d'un  passage  de  Macaulay  qui  niait  l'action 
des grands hommes, en se servant de l'analogie que voici : 
\bigskip
Le  soleil  illumine  les  collines  quand  il  est  encore  au-dessous  de  l'horizon,  et  les  hauts  esprits 
découvrent la vérité un peu avant qu'elle se manifeste à la multitude. Telle est la mesure de  leur 
supériorité.  Ils  sont  les  premiers  à  saisir  et  à  réfléchir  une  lumière  qui,  sans  leur  secours,  doit 
bientôt  devenir  visible  a  ceux  qui  sont  placés  bien  au-dessous  d'eux  (Essai  sur  Dryden,  dans  les 
Mélanges, I, 186) (3). » 
\bigskip
(2) Cf. § 84 : Effets de l'analogie. 
\bigskip
\bigskip
\bigskip
276 
\bigskip
(3) J. Stuart Mill, Système de logique, liv. VI, chap. XI, § 3, v. II, pp. 541-542. 
\bigskip
P  521-522 :  « Mill  prendra  le  contrepied  de  Macaulay,  et  amendera  l'analogie  pour  mieux  faire 
comprendre sa pensée : 
\bigskip
En poussant plus loin la métaphore, écrit-il, il s'ensuivrait que s'il n'y avait pas eu de Newton, le 
monde,  non  seulement  aurait  eu  le  système  newtonien,  mais  l'aurait  eu  aussi  vite;  absolument 
comme  le  soleil  se  serait  levé  pour  des  spectateurs  placés  dans  la  plaine,  s'il  n'y  avait  point  eu 
devant eux de montagne pour recevoir plus tôt ses premiers rayons... Les hommes éminents ne se 
contentent pas de voir briller la lumière au sommet de la colline ; ils montent sur ce sommet et 
appellent le jour; et si personne n'était monté jusque-là, la lumière, dans bien des cas, aurait pu 
ne luire jamais sur la plaine (1). » 
\bigskip
(1) J. St. Mill, ibid., p. 542. 
\bigskip
P 522 : « L'analogie amendée tire son intérêt du processus argumentatif complet dans lequel elle 
s'insère  ;  prise  en  elle-même,  l'analogie  de  Mill  paraîtra  assez  maladroite.  L'aspect  positif  (les 
hommes  montés  sur  la  colline  et  qui  appellent  le  jour)  en  est  d'ailleurs  moins  important  que 
l'aspect  négatif  (les  rayons  frappant  la  plaine  sans  écran).  L'avantage  de  cette  technique  est  que 
l'on bénéficie de l'adhésion qui avait pu être partiellement accordée à l'analogie primitive. » 
\bigskip
P 522-523 : « En philosophie, il arrive bien souvent qu'une analogie acquière, pour ainsi dire, droit 
de  cité  et  que  le  progrès  de  la  pensée  se  marque  par  les  amendements  successifs  qu'on  lui  fait 
subir. C'est ainsi que, pour opposer sa pensée à Locke, qui a comparé l'esprit à un bloc de marbre, 
Leibniz, modifie cette analogie, en la reprenant, ainsi amendée, à son propre compte : 
\bigskip
je  me  suis  servi  aussi  de  la  comparaison  d'une  pierre  de  marbre  qui  a  des  veines,  plustost  que 
d'une pierre de marbre toute unie, ou des Tablettes vuides, c'est-à-dire de ce qui s'appelle Tabula 
rasa chez les Philosophes. Car, si l'aine ressembloit à ces Tablettes vuides, les verités seroient en 
nous comme la figure d'Hercule est dans un marbre, quand ce marbre est tout à fait indifferent à 
recevoir  ou  cette  figure  ou  quelque  autre.  Mais  s'il  y  avoit  des  veines  dans  la  pierre  qui 
marquassent  la  figure  d'Hercule  preferablement  à  d'autres  figures,  cette  pierre  y  seroit  plus 
determinée, et Hercule y seroit comme inné en quelque façon, quoyqu'il faudroit du travail pour 
decouvrir  ces  veines,  et  pour  les  nettoyer  par  la  politure,  en  retranchant  ce  qui  les  empeche  de 
paroistre. Et c'est ainsi que les idées et les verités nous sont innées, comme des inclinations, des 
dispositions, des habitudes ou des virtualités naturelles, et non pas comme des actions, quoyque 
ces  virtualités  soyent  tousjours  accompagnées  de  quelques  actions  souvent  insensibles  qui  y 
repondent (1). » 
\bigskip
(1) Leibniz, Œuvres éd. Gerhardt, 5e vol.: Nouveaux essais sur l'entendement, p.45. 
\bigskip
P  523 :  « L'adaptation  à  ses  propres  thèses  d'une  analogie  de  l'adversaire  était  un  procédé 
d'argumentation  que  Leibniz  affectionnait  (2).  Il  arrive  pourtant  que  cette  technique  s'avère 
insuffisante,  que  l'on  attache  de  l'importance  à  un  aspect  du  thème  que  le  phore  est  incapable 
d'illustrer,  du  moins  si  l'on  ne  veut  pas  le  transformer  en  quelque  chose  de  fantastique. 
L'argumentation  analogique  préconisera,  dans  ce  cas,  la  substitution  à  ce  phore  d'un  autre,  jugé 
plus adéquat. C'est ainsi que Polanyi, parce que la science, à chaque étape de son progrès, donne 
l'impression  d'un  ensemble,  ne  peut  admettre  l'analogie  de  Milton,  qui  dans  son  Areopagitica, 
compare l'activité des savants à celle de chercheurs occupés, chacun pour son compte, à retrouver 
les  fragments  épars  et  cachés  d'une  statue,  pour  tenter  ensuite  de  les  joindre.  C'est  plutôt  à  un 
organisme en voie de croissance, nous dit Polanyi, qu'il faudrait comparer la science (3). 
\bigskip
\bigskip
\bigskip
\bigskip
277 
\bigskip
Une  analogie  semble  en  effet  adéquate  quand  le  phore  met  en  évidence  des  caractères  du  thème 
estimés  primordiaux  ;  son  remplacement  par  une  nouvelle  analogie  consiste,  le  plus  souvent,  à 
remplacer  une  structure  par  une  autre,  qui  met  en  relief  des  caractères  jugés  plus  essentiels. 
Admettre  une  analogie,  correspond  donc  souvent  à  un  jugement  sur  l'importance  des  caractères 
qu'elle met en évidence. C'est ce qui explique des affirmations à première vue étranges. Critiquant 
les conceptions de Wittgenstein, W. Moore s'en prend à l'analogie selon laquelle les énoncés sont 
aux faits comme les sillons d'un disque aux sons, et déclare : 
\bigskip
Si  un  énoncé  représentait  le  fait  comme  une  ligne  sur  un  disque  le  son,  alors  nous  devrions 
probablement être d'accord avec la thèse de Wittgenstein (4). » 
\bigskip
(2) Cf. Leibniz, ibid., pp. 131-132. 
(3) M. Polanyi, The logic of liberty, pp. 87-89. 
(4) W. Moore, Structure in Sentence and in Fact, Philosophy of  Science, janv. 1938, p. 87. 
\bigskip
P  524 :  « On  voit  que  l'acceptation  ou  le  rejet  de  l'analogie  paraissent  décisifs,  comme  si  un 
ensemble  de  conclusions  y  était  nécessairement  lié,  comme  si,  résumant  ce  qu'il  y  a  d'essentiel 
dans le thème, elle imposait de façon contraignante une manière de le penser (1). 
\bigskip
Certaines époques, certaines tendances philosophiques manifestent des préférences dans le choix 
du phore. Alors que les analogies spatiales avaient la faveur de la pensée classique, la nôtre préfère 
des phores plus dynamiques. Le bergsonisme se caractérise par le choix de phores empruntés à ce 
qui  est  liquide,  fluide,  mouvant,  tandis  que  la  pensée  des  adversaires  est  décrite  par  des  phores 
solides  et  statiques.  Richards  a  constaté  très  justement  que  les  métaphores  auxquelles  une 
philosophie  renonce,  dirigent  la  pensée  tout  autant  que  celles  qui  sont  adoptées  (2) ;  en  effet,  la 
pensée peut s'organiser en fonction de ce rejet. » 
\bigskip
(1)  L'application  du  raisonnement  par  analogie  au  discours  lui-même  prend  les  formes  les  plus 
variées. Ici, c'est la relation entre langue et faits qui est le thème, le phore étant pris an domaine 
familier. Souvent, le langage lui-même est un phore, dont la structure doit faire connaître celle du 
monde.  Souvent  aussi,  l'organisation  du  discours  est  thème,  éclairé  par  l'analogie  avec  lin 
organisme vivant ; cf. § 105 ; Ordre et méthode. 
(2) A. Richards, The Philosophy of Rhetoric, p. 92. 
\bigskip
P 524-525 : « On sait que le cours du temps a été rendu au moyen d'analogies spatiales, mais leur 
choix  est  fort  divers  et  plein  d'enseignements  :  parfois  le  phore  utilisé  est  le  tracé  d'une  ligne 
indéfiniment prolongée, parfois c'est un fleuve qui s'écoule, parfois les événements passent comme 
un cortège devant un spectateur, parfois ils émergent de l'obscurité, comme une rangée de maisons 
éclairées  successivement  par  le  phare  d'un  policier,  parfois  la  course  du  temps  est  celle  d'une 
aiguille  sur  un  disque  de  gramophone,  parfois  c'est  une  route  dont  on  peut  percevoir 
simultanément des fragments d'autant plus étendus que l'on jouit d'un point de vue plus dégagé : 
chaque  phore  insiste  sur  d'autres  aspects  du  thème  et  prête  à  d'autres  développements  (3).  C'est 
pourquoi  la  compréhension  d'une  analogie  est  le  plus  souvent  incomplète  si  l'on  ne  tient  pas 
compte des analogies anciennes que la nouvelle amende ou remplace. D'ailleurs la compréhension 
du  phore,  surtout  quand  ce  dernier  est  emprunté  à  un  domaine  social  ou  spirituel,  suppose  une 
connaissance  suffisante  de  la  place  qu'il  occupe  dans  une  culture  déterminée,  des,  analogies 
antérieures et sous-jacentes dans lesquelles ce phore a été utilisé, soit comme phore pour un autre 
thème,  soit  comme  thème  d'un  autre  phore  (1).  La  liaison  existant  traditionnellement  entre  la 
lumière  et  le  bien,  rend  plus  plausible  l'analogie  de  Scot  Érigène  (2)  de  même  que  celles  de 
Macaulay  et  Mill  mentionnées  plus  haut.  Nous  connaissons  le  rôle  que,  depuis  Platon,  joue 
l'analogie traitant la vie comme un spectacle (3). L'utilisation par Mauriac du phore de la  chasse 
pour  décrire  l'homme  comme  la  proie  de  Dieu  sera  interprétée  d'une  façon  plus  exacte  par  celui 
qui sait que ce même phore sert aussi à l'auteur pour décrire la femme comme gibier de l'homme 
\bigskip
\bigskip
\bigskip
278 
\bigskip
dans la poursuite amoureuse (4). On sait, du reste, la place que, dans l'étude des archétypes, Jung 
accorde à ce matériel analogique traditionnel (5). » 
\bigskip
(3) Cf. E. Bevan, Symbolism and Relief, pp. 85-94. 
(1)  Cf.  La  lecture  et  les  voyages  chez  Descartes,  Discours  de  la  méthode,  p.  47  et  chez 
Schopenhauer, éd.  Brockhaus, vol. 6 : Parerga und Paralipomena, Zweiter  Band,  Selbstdenken, § 
262, p. 525. 
(2) Cf. § 83 : Relations entre les termes de l'analogie. 
(3) Cf. V. Goldschmidt, Le système stoïcien et l'idée de temps, §§ 89-91.  
(4) Cf. N. Cormeau, L'art de François Mauriac, pp. 341-3,12. 
(5) C. G. Jung, Psychologie und Religion, p. 93. 
\bigskip
P 525-526 : „Une technique spéciale consiste à utiliser plusieurs phores pour faire comprendre un 
même thème ; on insiste par là sur l'insuffisance de chaque phore particulier, tout en imprimant 
une  direction  générale  à  la  pensée  ;  telles  sont  les  analogies  d'une  variété  déconcertante  grâce 
auxquelles Lecomte du Nouÿ expose les rapports qu'il voit entre les mécanismes de l'évolution et 
l'évolution  elle-même  (6).  L'usage  des  phores  multiples  est  pourtant  assez  délicat  ;  en  vertu  de 
l'interaction entre phore et thème, autre  le phore, autre sera le thème. Dans la mesure où l'on en 
fait  un  thème  unique,  celui-ci  risque  d'être  assez  confus.  Pour  éviter  les  interférences  entre  les 
différents phores, il sera souvent prudent -comme le fait d'ailleurs Lecomte du  - d'éviter que ceux-
ci se succèdent de trop près. Chaque phore apportant sa structure au thème, même si chacune de 
celles-ci  est  plausible,  et  même  si  au  point  de  vue  de  la  valeur  des  termes  du  thème  elles 
aboutissent  àune  même  conclusion,  leur  juxtaposition  produit  un  effet  comique,  dont  on  trouve 
d'excellents exemples dans le Don Quichotte : 
\bigskip
Car le chevalier errant sans Dame est comme l'arbre sans feuilles, l'édifice sans ciment, l'ombre 
sans corps qui la cause (1). » 
\bigskip
(6) Lecomte De Nouÿ,  L'homme et sa destinée, pp. 76 à 80. 
(1) Cervantès, EI ingenioso hidalgo Don Quijote de la Mancha, vol. VI, 11, chap. XXII, pp. 271272. 
\bigskip
P 526 : « Les analogies multiples, au lieu d'être indépendantes, peuvent se soutenir mutuellement. 
Ainsi  Locke,  pour  prêcher  la  tolérance,  passe  de  l'analogie  bien  connue  entre  les  conditions  du 
salut et les chemins qui vont au ciel  à une analogie des remèdes, de telle  manière que l'on sait à 
peine si c'est le thème ou le phore de la première analogie qui constituent le thème de la seconde. 
Voici le texte : 
\bigskip
Il n'y a qu'une voie véritable vers le bonheur éternel: mais encore, dans ce grand nombre de voies 
que suivent les hommes, peut-on se demander laquelle est la bonne. Or ni le soin de l'État, ni ses 
qualités  de  législateur  ne  font  que  le  magistrat  découvre  mieux  ce  chemin  que  l'homme  qui 
cherche  et  étudie  par  lui-même.  J'ai  un  corps  faible,  épuisé  par  une  maladie  de  langueur  à 
laquelle (je suppose) il n'y a qu'un remède, et il est inconnu. Appartient-il pour cela au magistrat 
de me prescrire un remède, parce qu'il n'y en a qu'un et parce qu'il est inconnu (2) ? » 
\bigskip
(2) Cf. Locke, The second treatise of civil government and A letter concerning toleration, P. 138. 
\bigskip
P  526-527 :  « Les  analogies  peuvent  aussi  être  greffées  l'une  sur  l'autre,  une  partie  du  phore 
devenant le point de départ d'une nouvelle analogie. Vico recourt à ce procédé pour nous décrire 
l'effet que fit la mort d'Angela Cimmino, sur la princesse de la Roccella qui venait de perdre son 
mari et  
\bigskip
dont  le  deuil  amer  et  récent  faisait  que,  pour  haut  et  grand  que  fût  son  coeur,  pareil  à un  vase 
vivant de l'or le plus pur, il était rempli d'une telle douleur que rien d'autre, pour aucune autre 
\bigskip
\bigskip
\bigskip
279 
\bigskip
raison, d'aucune manière, ne pouvait y pénétrer; cependant, si fortement le frappa le deuil de la 
mort de notre marquise, que, ainsi qu'un corps dur qui y aurait été jeté, il le fit résonner de deux 
sublimes sonnets (1). » 
\bigskip
(1) Vico, Opere, éd. Ferrari, vol. 6 ; Oraxione in morte di Angiola Cimini, p. 301.  
\bigskip
P  527 :  « Rien  dans  le  mécanisme  de  l'analogie  tel  que  nous  l'avons  décrit  ne  s'oppose  à  ces 
analogies successives. Maints traités de style parlent avec mépris des « images qui se chevauchent 
».  Si  l'on  proscrivait  celles-ci,  on  serait  étonné  de  voir  combien  de  nos  énoncés  seraient  à 
supprimer. Nous reviendrons sur ces questions à propos de la métaphore (2). » 
\bigskip
 (2) Cf. §87 : La métaphore. 
\bigskip
§ 86. LE STATUT DE L'ANALOGIE 
\bigskip
L'analogie  est  un  moyen  d'argumentation  instable.  En  effet,  celui  qui  en  rejette  les  conclusions 
tendra  à  affirmer  qu'il  n'y  a  «  même  pas  d'analogie  »,  et  minimisera  la  valeur  de  l'énoncé  en  le 
réduisant  à  une  vague  comparaison  ou  à  un  rapprochement  purement  verbal.  Mais  celui  qui 
invoque  une  analogie,  tendra  presque  invariablement  à  affirmer  qu'il  y  a  plus  qu'une  simple 
analogie. Celle-ci est ainsi coincée entre deux reniementsl celui de ses adversaires, et celui de ses 
partisans. » 
\bigskip
P 527-528 : « L'analogie est parfois dépassée avant même d'avoir été comprise comme telle. C'est 
que  la  spécificité  de  l'analogie  réside  dans  la  confrontation  de  structures  semblables  bien 
qu'appartenant à des domaines différents. Lorsque ces structures ne sont pas perçues, comme c'est 
le  cas  dans  certains  troubles  mentaux,  tout  rapprochement  entre  phore  et  thème  tendra  à 
s'expliquer par des caractères communs, notamment des ressemblances entre termes (1). D'autre 
part la distinction des domaines n'est pas toujours facile àconstater : elle dépend des critères dont 
on  se  sert  pour  l'établir.  Ce  n'est  que  dans  certaines  analogies  d'un  type  reconnu,  telles  les 
allégories et les fables, que la distinction des domaines semble hors discussion; c'est aussi le cas au 
sein  de  certaines  philosophies  où  l'usage  analogique  des  termes  et  des  structures  résulte  d'une 
préalable critériologie de l'être. » 
\bigskip
(1)  Cf.  Benary,  Studien  zur  Untersuchung  der  Intelligenz  bel  einem,  Fall  von  Seelenblindheit, 
Psychologische Forschung, 2 (3-4), 1922, pp. 257-263, 268-272. 
\bigskip
P  528 :  « Le  dépassement  de  l'analogie  sera  parfois  simplement  suggéré.  Mais  souvent  il  sera 
explicite, voire motivé, justifié. 
\bigskip
Le premier effort pour dépasser l'analogie, pour rapprocher le thème du phore vise à établir entre 
eux  un  rapport  de  participation  :  le  phore  est  présenté  comme  symbole,  comme  figure,  comme 
mythe,  réalités  dont  l'existence  même  dérive  de  leur  participation  au  thème  qu'ils  doivent 
permettre  de  mieux  appréhender.  Ayant  découvert  une  analogie  entre  certaines  hiérarchies 
surnaturelles et certains aspects de l'énergie, S. Weil dira : 
\bigskip
Ainsi ce n'est pas seulement la mathématique, c'est la science entière qui, sans que nous songions 
à le remarquer, est un miroir symbolique des vérités surnaturelles (2). 
\bigskip
De même Buber écrit : 
\bigskip
La  relation  avec  l'être  humain  est  le  véritable  symbole  de  la  relation  avec  Dieu,  dans  laquelle 
l'invocation véritable reçoit la réponse véritable (3). 
\bigskip
\bigskip
\bigskip
\bigskip
280 
\bigskip
Et  Pascal,  utilisant  largement  la  notion  de  «  figure  »  telle  que  l'avait  élaborée  la  tradition 
chrétienne, avait précisé le rôle fondamental et révélateur qu'il fallait lui attribuer : 
\bigskip
La figure, écrit-il, a été faite sur la vérité, et la vérité a été reconnue sur la figure (4). » 
\bigskip
(2) S. Weil, L'enracinement, p. 248. 
(3) M. Buber, Je et Tu, p. 151. 
(4) Pascal, Bibl. de la Pléiade, Pensées, 572 (270), p. 1013 (673 éd. Brunschvicg). 
\bigskip
P 529 : « Pareilles techniques de rapprochement entre thème et phore, tout en conservant à ceux-
ci  leur  individualité,  tendent  cependant  à  l'unification  des  domaines  :  l'idée  de  figure  suppose  la 
réalité du phore au même titre que celle du thème. 
\bigskip
Parfois  le  dépassement  de  l'analogie  se  fera  en  montrant  que  thème  et  phore  dépendent  d'un 
principe  commun.  Après  avoir  dégagé  certaines  analogies  entre  l'inertie  physique  et  la  force  de 
l'habitude, Schopenhauer poursuit : 
\bigskip
Tout ceci est plus qu'une simple analogie : c'est déjà l'identité de la chose, c'est-à-dire du Vouloir, 
à des degrés très divers de son objectivation, conformément auxquels la même loi de mouvement 
se présente différemment (1). 
\bigskip
Ce  principe  commun  pourra  être  conçu  comme  une  essence,  dont  thème  et  phore  seraient  des 
manifestations.  Lorsque  Eugenio  d'Ors  développe  d'une  manière  brillante  certaines  analogies 
entre des formes architecturales et le régime politique dans lequel elles se développent, il prétend à 
plus qu'une confrontation permettant de comprendre l'un par l'autre, tout en se défendant de faire 
de l'un la cause de l'autre (2). 
\bigskip
Maintes fois, on établira un lien indirect entre thème et phore. Si un discours plein d'antithèses et 
d'ornements  est  fait  comme  une  église  gothique,  c'est  que  tous  deux  dérivent,  dira  Fénelon,  du 
mauvais goût des Arabes (3). 
\bigskip
Il  pourra  même  se  faire,  comme  dans  certains  rapprochements  entre  ce  que  l'on  ressent  et  le 
milieu  où  l'on  se  trouve,  que  l'analogie  suggère une  action  du  phore  sur  le thème.  Dans  des  vers 
tels que 
\bigskip
Il pleure dans mon coeur 
Comme il pleut sur la ville (4) » 
\bigskip
(1) Schopenhauer, éd. Brockhaus, vol. 6 : Parerga und Paralipomena, Zweiter Band Psychologische 
Remerkungen,g 307, p. 619. 
(2) Eugenio D'Ors, Coupole et monarchie, suivi d'autres études sur la morphologie de la culture. 
(3) Fénelon, éd. Lebel, t. XXI : Dialogues sur l'éloquence, p. 76. 
(4) Verlaine, Œuvres Bibl. de la Pléiade, Romances sans paroles, III, p. 122. 
\bigskip
P 530 : « le phore peut être pris pour une cause partielle du thème et l'analogie est, par là même, 
dépassée. 
\bigskip
Ce  dépassement  sera  parfois  exprimé  par  la  transmission  d'un  élément  substantiel  du  phore  au 
thème. Ainsi selon Leibniz, 
\bigskip
Feu M. Van Helmont le fils... croyait avec quelques Rabbins le passage de l'Ame d'Adam dans le 
Messie comme dans le nouvel Adam (1). 
\bigskip
\bigskip
\bigskip
\bigskip
281 
\bigskip
Bref, pour dépasser l'analogie, on tentera, de toutes manières, de rapprocher le domaine du thème 
de  celui  du  phore.  Qu'il  s'agisse  là  d'un  processus  tout  naturel,  cela  ressort  de  l'insistance  même 
avec  laquelle  on  tente  souvent  de  se  prémunir  contre  le  dépassement.  Lorsque  Tarde  a  mis  en 
évidence les analogies entre la logique individuelle et la logique sociale (2), lorsque Odier a mis en 
évidence  les  analogies  entre  la  réflexologie  de  Pavlov  et  la  psychologie  du  moi  (3),  l'un  comme 
l'autre prévoient le dépassement qui risque d'être opéré et mettent en garde contre la Les termes 
dans lesquels Odier exprime son appréhension sont significatifs : 
\bigskip
Il  n'est  de  symptôme  névrotique  qu'on  ne  puisse,  en  fin  de  compte  décrire  en  termes  de 
physiologie, ni ramener à un choc d'énergies contraires. Mais n'oublions jamais qu'il ne s'agit là 
que d'une réduction et non pas d'une explication... . Chez l'adulte névrose, seule la psychologie du 
moi nous livre les éléments d'une explication véritable (4). » 
\bigskip
(1) Leibniz, éd. Gerhardt, 5e vol : Nouveaux essais sur l'entendement, p. 222. 
(2) G. Tarde, La logique sociale, chap. 11, pp. 87 et suiv. 
(3) Ch. Odier, L'homme esclave de son infériorité, I : Essai sur la genèse ein moi, chap. III, pp. 93 
et Suiv. 
(4) Ibid., p. 123. 
\bigskip
P  530-531 :  « Or,  si  ces  auteurs  ont  insisté  sur  l'analogie,  c'est  que  celle-ci  leur  paraissait  digne 
d'intérêt,  capable  de  nous  éclairer  sur  certains  phénomènes.  Le  dépassement  ne  pourrait  que 
renforcer la preuve de cette structure. Mais ils craignent que ce dépassement se produise au profit 
exclusif du phore. Tarde, usant d'ailleurs d'une nouvelle analogie, dira en effet : 
\bigskip
Impossible  donc  d'anéantir  la  logique  sociale  dans  la  logique  individuelle.  Leur  dualité  est 
irréductible,  mais  comme  celle  de  là  courbe  et  de  l'asymptote  qui  vont  se  rapprochant 
indéfiniment (1). » 
\bigskip
(1) G. Tarde, OP. cit., p. 114. 
\bigskip
P  531 :  « En  fait,  aussi  bien  Odier  que  Tarde  tenteront  eux  aussi  le  dépassement  ;  le  premier 
insistera  sur  ce  que  l'identité  est  «  dans  les  conséquences  »  (2),  le  second  sur  «  les  transactions 
réciproques », qu'il faudra entre les deux domaines (3). 
\bigskip
Dans  les  sciences  naturelles,  l'analogie,  comme  nous  la  concevons,  fournit  uniquement  un  point 
d'appui  à  la  pensée  créatrice.  Il  s'agit  de  dépasser  l'analogie  pour  pouvoir  conclure  à  une 
ressemblance,  à  la  possibilité  d'appliquer  au  thème  comme  au  phore,  les  mêmes  concepts.  On 
s'efforcera, en les rendant justiciables des mêmes méthodes, de réunir thème et phore en un seul 
domaine d'investigation. 
\bigskip
Ainsi, en chimie, l'observation de réactions analogiques conduira l'esprit à ranger les corps étudiés 
dans une même famille. Cournot raconte comment, frappés par certaines analogies, Gay-Lussac et 
Thénard élaborèrent l'hypothèse que la substance appelée acide muriatique oxygéné était un corps 
simple, qu'ils appelèrent chlore, comment on le rangea dans une famille naturelle avec le brome et 
l'iode, et il ajoute que, à cause de ces mêmes analogies, on est d'accord pour y faire place au fluor, 
non encore découvert (4) » 
\bigskip
(2) Ch. Odier, OP. Cit., P. 122.  
(3) G. Tarde, OP. Cit., P. 113. 
(4) Cournot, Essai sur les fondements de nos connaissances. II, pp. 237-238, 
\bigskip
P 531-532 : « L'analogie, en tant que chaînon dans le raisonnement inductif, constitue une étape en 
science, où elle sert comme moyen d'invention plus que comme moyen de preuve : si l'analogie est 
\bigskip
\bigskip
\bigskip
282 
\bigskip
féconde, thème et phore sont transformés en exemples ou illustrations d'une loi plus générale, par 
rapport à laquelle les domaines du thème et du phore sont unifiés. Cette unification des domaines 
conduit à intégrer dans une même classe la relation qui unit les termes du phore et celle qui unit 
les  termes  du  thème,  lesquelles  deviennent,  par  rapport  à  cette  classe,  interchangeables  :  toute 
asymétrie entre thème et phore a disparu. » 
\bigskip
P  532 :  « La  précarité  du  statut  de  l'analogie  tient  donc  en  grande  partie  à  ce  que  celle-ci  peut 
disparaître par son succès même. 
\bigskip
L'analogie  peut  aussi  être  exclue  de  par  les  conditions  du  raisonnement.  Nous  avons  vu  que,  en 
droit, le raisonnement par analogie occupe une place beaucoup plus limitée qu'il n'y paraît, et cela 
parce  que  lorsqu'il  s'agit  de  l'application  d'une  règle  à  de  nouveaux  cas,  nous  nous  trouvons 
d'emblée  à  l'intérieur  d'un  seul  domaine,  de  par  les  exigences  mêmes  du  droit,  puisque  nous  ne 
pouvons sortir du domaine que la règle nous assigne. 
\bigskip
D'une manière générale, le dépassement de l'analogie tend à présenter celle-ci comme le résultat 
d'une  découverte,  observation  de  ce  qui  existe,  plutôt  que  comme  le  produit  d'une  création 
originale de structuration. Dans certains cas, le problème est renversé. Il y a des philosophies qui 
voient dans l'analogie le résultat d'une différenciation au sein d'un ensemble unitaire : c'est le cas 
des  philosophies  monistes  qui  refusent  toute  distinction  des  domaines.  Ce  refus  peut  être 
considéré  comme  une  manière  extrême  de  consacrer  le  dépassement  en  le  rendant  en  quelque 
sorte  préalable  à  l'analogie.  Celle-ci  n'est  plus  que  l'explicitation  de  ce  qui  est  inclus  dans 
l'ensemble indifférencié qui la précède. Mais ces considérations philosophiques relatives au statut 
de l'analogie laissent intactes, dans la pratique, les possibilités normales d'emploi de l'analogie et 
sa tendance au dépassement. » 
\bigskip
Ajoutons  que  le  statut  de  l'analogie  est  précaire  d'une  autre  manière  encore  :  confrontation  de 
structures,  l'analogie  peut,  à  cause  de  l'interaction  entre  termes  (1),  donner  lieu  à  des 
rapprochements  qui  concernent  ceux-ci.  Cette ressemblance  des  termes  fournit  presque toujours 
des effets comiques, ce qui indique que c'est un usage abusif de l'argument par analogie. Celui qui, 
dans  l'analogie  classique  entre  l'évêque  et  ses  fidèles,  le  berger  et  ses  moutons,  verra  avant,  tout 
une  ressemblance  entre  moutons  et  fidèles,  et  qualifiera  le  fidèle  qui  prie  de  mouton  qui  bêle, 
obtiendra  un  effet  facile,  mais  en  détournant  indûment  l'analogie  de  sa  fonction.  Toutefois,  la 
distinction  entre  analogie  et  ressemblance  ne  saurait  être  absolue.  Un  élément  de  ressemblance 
entre termes semble souvent à l'origine d'une analogie, même s'il ne joue aucun rôle essentiel dans 
sa structure. » 
\bigskip
(1) CI. § 83 - Relations entre les termes d'une analogie. 
\bigskip
P 533 : « Ainsi lorsque Francis Ponge (1), dans son poème sur le lézard, suggère - à l'aide d'ailleurs 
d'un nouveau phore - une analogie entre les évolutions du lézard sur un mur, sa disparition dans 
un trou, et les phases de la création poétique, « petit train de pensées grises lequel circule ventre à 
terre » et qui « rentre volontiers dans les tunnels de l'esprit », les éléments de ressemblance entre 
le mur et la page sur laquelle on écrit (forme, couleur), sont vraisemblablement à l'origine du choix 
des termes. 
\bigskip
D'autre part, lorsqu'elle réussit, l'analogie peut aboutir à une extension du champ d'application de 
certaines  notions.  Ainsi  N.  Rotenstreich,  ayant  montré  une  analogie  entre  le  rapport  du  sujet 
concret à l'expérience, et celui de l'homme au langage, conclura en disant : 
\bigskip
Le langage doit être considéré comme de l'expérience plus étendue (2). 
\bigskip
\bigskip
\bigskip
\bigskip
283 
\bigskip
Ce  mode  de  dépassement  de  l'analogie  par  ses  termes  sera  d'autant  plus  aisé,  semble-t-il,  que 
ceux-ci  sont  plus  abstraits  et  peuvent  être  perçus  comme  exprimant  des  structures.  Il  joue  sans 
doute un rôle important dans l'évolution des notions. » 
\bigskip
(1) Francis Ponge, Le lézard. ln fine. 
(2) N. Rotenstreich, The Epistemological Status of the Concrete Subject,  Rev. int. de Philosophie 
22, pp. 414-415. 
\bigskip
P 534 : “Dans certains cas, tel celui que nous venons de relever, l'analogie influencera avant tout 
l'extension  des  notions.  Mais  elle  agit  du  même  coup  sur  leur  confusion.  Celle-ci  pourra  être 
augmentée de bien des manières par l'argumentation analogique. Lorsque Pitt établit une longue 
analogie  entre  l'heureux  état  politique  de  l'Angleterre  et  les  conditions  de  la  zone  tempérée  à  la 
surface du globe (1), nous voyous assez clairement les structures du thème et celles du phore, mais 
les idées de juste milieu, d'équilibre, ne pourront qu'être rendues plus confuses de par l'interaction 
entre phore et thème. 
\bigskip
Nous  avons  vu,  par  ailleurs,  que  des  notions  désignant  des  propriétés  du  monde  physique  se 
trouvent,  de  par  leur  usage  analogique  dans  un  milieu  culturel,  chargées  d'une  valeur  qui  fait 
désormais partie de leur signification. 
\bigskip
Que  l'analogie  puisse  modifier  les  notions  et  en  argumenter  la  confusion  nous  semble 
incontestable. Si tant d'auteurs contemporains répugnent à admettre le rôle de l'analogie dans la 
genèse  de  certaines  notions  (2),  c'est  sans  doute  par  anti-associationnisme  exaspéré.  Une 
conception  de  l'analogie  faisant  meilleure  place  à  l'interaction  entre thème  et  phore,  diminuerait 
sans  doute  ces  répugnances.  Elle  diminuerait  du  même  coup  sans  doute  la  répugnance  à 
considérer la métaphore comme dérivant de l'analogie (3). » 
\bigskip
(1) W. Pitt, Orations on the French war, pp. 3-4 (1 -1 fév. 1793). 
(2) J.-P. Sartre, L'être et le néant, pp. 695-96. 
(3) Cf. Morris B. Cohen, A Preface to Logic, p. 83. 
\bigskip
§ 87. LA METAPHORE 
\bigskip
P 534-535 : « Dans la tradition des maîtres de rhétorique, la métaphore est un trope, c'est-à-dire « 
un  heureux  changement  de  signification  d'un  mot  ou  d'une  locution  »  (4)  ;  elle  serait  même  le 
trope par excellence (5). Par la métaphore, nous dit Dumarsais, « on transporte, pour ainsi dire, la 
signification  propre  d'un  nom  à  une  autre  signification,  qui  ne  lui  convient  qu'en  vertu  d'une 
comparaison qui est dans l'esprit » (1). » 
\bigskip
(4) Quintilien, Vol. III, liv. VIII, chap. Vl, 9 1 ; ei. Volkmann, Rhetorik, der Griechen und Römer, 
p. 40. 
(5) Dumarsais, Des Tropes, pp. 167-168. 
(1) Dumarsais, Des Tropes, p. 103. 
\bigskip
P 535 : « Richards rejette à bon droit l'idée de comparaison, insistant avec finesse et vigueur sur le 
caractère vivant, nuancé, varié, des relations entre concepts exprimés en une fois par la métaphore, 
laquelle  serait  interaction  bien  plus  que  substitution  (2),  et  technique  d'invention  autant  que 
d'ornement (3). 
\bigskip
Mais  toute  conception  qui  ne  jette  pas  de  lumière  sur  l'importance  de  la  métaphore  dans 
l'argumentation  ne  peut  nous  satisfaire.  Or,  nous  croyons  que  c'est  en  fonction  de  la  théorie 
argumentative de l'analogie que le rôle de la métaphore s'éclairera le mieux. Affirmer le lien entre 
métaphore  et  analogie,  c'est  d'ailleurs  reprendre  une  tradition  ancienne,  celle  des  philosophes  et 
\bigskip
\bigskip
\bigskip
284 
\bigskip
le  rapport  analogique  est 
\bigskip
spécialement  des  logiciens,  d'Aristote  à  John  Stuart  Mill  (4).  Ce  lien  redeviendra,  pensons-nous, 
acceptable, dans la mesure où s'élaborera plus profondément la théorie de l'analogie. 
\bigskip
Nous ne pourrions mieux, en ce moment, décrire la métaphore qu'en la concevant, tout au moins 
en  ce  qui  concerne  l'argumentation,  comme  une  analogie  condensée,  résultant  de  la  fusion  d'un 
élément du phore avec un élément du thème. 
\bigskip
Aristote  nous  présente  certains  exemples  de  métaphores  où 
complètement explicité : 
\bigskip
Ce  qu'est  la  vieillesse  à  la  vie,  le  soir  l'est  au  jour.  On  dira  donc  le  soir  vieillesse  du  jour  et  la 
vieillesse soir de la vie (5) ... » 
\bigskip
(2) A. Richard, The Philosophy of Rhetoric, pp. 93 et suiv. 
(3) A. Richard, cf. A symposium on emotive meaning, The Philosophical Review, 1948, p. 146. 
(4) Aristote, Art poétique, chap. XXI, §§ 7 et suiv. ; Rhétorique, liv. III, chap. X, § 7; J. Stuart Mill, 
Système de logique, vol. 11, liv. V, chap. V, § 7, p. 375. 
(5) Aristote, Art poétique, chap. XXI, § 13. 
\bigskip
P  535-536 :  « Dans  pareils  exemples,  phore  et  thème  sont  traités  d'une  façon  symétrique,  pour 
ainsi dire scolaire, en dehors du contexte qui seul indiquerait quel est le thème et quel est le phore. 
C'est pourquoi on y voit nettement comment la métaphore peut construire une expression a partir 
d'une analogie. Dans ce cas-ci il s'agit, à partir de l'analogie " A est à B comme C est à D » d'une 
expression « C de B » pour désigner A. Nous verrons que c'est loin d'être la seule façon de réaliser 
la fusion entre thème et phore. » 
\bigskip
P 536 : « L'analogie, grâce à cette fusion, est présentée  non comme une suggestion, mais comme 
une donnée. C'est dire que la métaphore peut intervenir pour accréditer l'analogie. 
\bigskip
Il  n'est  donc  pas  étonnant  de  constater,  quand  on  examine  les  argumentations  par  analogie  que, 
souvent,  l'auteur  n'hésite  pas,  au  cours  de  son  exposé,  à  se  servir  de  métaphores  dérivées  de 
l'analogie  proposée,  habituant  ainsi  le  lecteur  à  voir  les  choses  telles  qu'il  les  lui  montre.  Il  est 
même  assez  rare  que  thème  et  phore  soient  exprimés  indépendamment  l'un  de  l'autre.  Plotin, 
après avoir parlé de la vie comme d'un spectacle, continue, en parlant de l'âme : 
\bigskip
Ensuite elle chante sa partie, c'est-à-dire elle agit et produit selon son caractère propre (1). 
\bigskip
Les  quelques  termes  pris  au  domaine  du  phore,  insérés  dans  celui  du  thème,  sont  ici  aussitôt 
explicités en termes appartenant à ce dernier. 
\bigskip
Les degrés de contamination entre thème et phore peuvent cependant être très variés. La fusion de 
termes  du  thème  et  du  phore,  qui  rapproche  leurs  deux  domaines,  facilite  la  réalisation  d'effets 
argumentatifs. Lorsque l'on s'efforcera, par le développement d'une analogie, de tirer, à partir du 
phore, des conclusions intéressant le thème, la force de l'argument sera d'autant plus grande que, 
grâce à la fusion du thème et du phore, on aura longuement décrit au préalable le phore en termes 
de thème. » 
\bigskip
(1) Plotin, Ennéades, III, 2, § 17. 
\bigskip
P  537 :  « Nous  connaissons  peu  de  textes  permettant  d'illustrer  ce  procédé  aussi  bien  que  la 
célèbre Ode à Cassandre de Ronsard : 
\bigskip
Mignonne, allons voir si la rose 
\bigskip
\bigskip
\bigskip
285 
\bigskip
Qui ce matin avait desclose 
Sa robe de pourpre au Soleil,  
A point perdu ceste vesprée  
Les plis de sa robe pourprée,  
Et son teint au vostre pareil... (1)  
\bigskip
Avant de parler de Cassandre en termes empruntés au domaine du phore (Tandis que vostre âge 
fleuronne En sa plus verte nouveauté), le poète n'hésite pas à parler de la rose comme d'une jeune 
fille, décrit les plis de sa robe, son teint, et s'indignera de la cruauté que manifeste à son égard la « 
marastre Nature ». 
\bigskip
(1) Ronsard, œuvres complètes, Bibl. de la Pléiade, vol. I : A sa maistresse, Ode XVII, pp. 419-420. 
\bigskip
P  537-538 :  « Les  métaphores  les  plus  riches  et  les  plus  significatives  sont pourtant  celles  qui  ne 
surgissent  pas,  comme  chez  Plotin,  ou  chez  Ronsard,  dans  une  analogie  en  cours  d'expression, 
mais  qui  sont  présentées  dès  l'abord,  telles  quelles,  par  l'accolement,  le  plus  souvent,  de  ternies 
supérieurs du thème et du phore (A et C), en laissant inexprimés les termes inférieurs (B et D). Ces 
termes  ne  doivent  pas  être  considérés  comme  sous-entendus,  car  il  faut  admettre  que  la  fusion, 
une fois réalisée, a créé une expression qui se suffit à elle-même; mais ils pourraient être, en cas 
d'analyse,  suppléés  de  façons  fort  diverses.  C'est  ainsi  que  la  métaphore  «  un  océan  de  fausse 
science » (2) suggère des points de vue et des attitudes différents selon que les termes B et D sont 
compris respectivement comme « un nageur » et « un savant » ; ou « un ruisseau », et « la vérité » 
; ou « la terre ferme » et « la vérité ». Toutes ces analogies, présentes simultanément à l'esprit, se 
fécondent et s'influencent mutuellement, suggèrent des développements variés, entre lesquels seul 
le  contexte  permettrait  un  choix,  rarement  dépourvu  de  toute  ambiguité  et  de  toute 
indétermination. La métaphore peut aussi opérer le rapprochement du terme B et C d'une analogie 
à trois termes (1), comme dans l'expression « la vie est un songe » ; dans ce cas c'est le terme A du 
thème  (par  exemple  «  la  vie  éternelle  »)  qui  sera  inféré  grâce  à  la  métaphore,  «  la  vie  »  étant  le 
terme commun aux deux domaines. » 
\bigskip
(2) Berkeley, Les trois dialogues entre Hylas et Philonous, Trois. dial., p. 207. 
(1) Cf. § 83 Relations entre les termes d'une analogie. 
\bigskip
P  538 :  « La  fusion  métaphorique  est  un  processus  de  rapprochement  très  différent  de  celui 
provoqué  par  la  double  hiérarchie  (2)  ou  de  celui  qu'opère  le  dépassement  de  l'analogie  par 
établissement d'une liaison symbolique entre thème et phore (3). Elle n'implique pas de rapports 
plus  étroits  entre  phore  et  thème  que  l'analogie  pure  et  simple,  mais  elle  consacre  ceux-ci. 
L'acquiescement  à  une  métaphore  est  habituel,  à  condition  que  l'on  admette  l'analogie.  On  a 
souvent  conseillé  pour  faire  accepter  la  métaphore,  de  la  préparer  (4),  ou  de  l'atténuer  par  des 
précautions  (5).  Cicéron,  et  après  lui  Quintilien,  conseillent  d'introduire  les  métaphores  trop 
hardies au moyen d'expressions telles « pour ainsi dire », « si j'ose m'exprimer ainsi ». Mais quand 
on examine les exemples cités et où ce procédé paraît utile, on voit qu'il s'agit de métaphores non 
pas trop hardies, mais qui pèchent plutôt par timidité, confrontant des domaines trop voisins, et 
risquant  par  là  d'être  méconnues,  en  ce  sens  que  l'expression  risque  d'être  prise  à  la  lettre  et  de 
devenir ridicule : 
\bigskip
...  si  l'on  avait  dit  autrefois  que  la  mort  de  Caton  laissait  le  Sénat  «orphelin  »,  la  métaphore 
aurait été un peu forcée ; au contraire «pour ainsi dire orphelin » la rendrait un peu moins dure 
(6). » 
\bigskip
(2) Cf. § 84 Effets de l'analogie. 
(3) Cf. § 86 Le statut de l'analogie. 
(4) Baron, De la Rhétorique, p. 324. 
\bigskip
\bigskip
\bigskip
286 
\bigskip
(5) Cicéron, De Oratore, liv. III, § 165 ; Quintilien, Vol. III, liv. VIII, chap. 111, § 37  ; Dumarsais, 
Des Tropes, p. 
(6) Cicéron, loc. Cit. 
\bigskip
P 538-539 : « La fusion des termes du thème et du phore peut se marquer de diverses façons, par 
une  simple  détermination  (le  soir  de la  vie,  ou,  océan  de  fausse  science),  au  moyen  d'un  adjectif 
(un  exposé  creux,  lumineux),  d'un  verbe  (elle  se  mit  à  piailler),  d'un  possessif  (notre  Waterloo). 
Parfois  même  nous  aurons  une  identification  (la  vie  est  un  songe,  l'homme  est  un  roseau)  ;  la 
copule ici n'a d'autres fonctions que de marquer la place homologue dans une relation analogique. 
La métaphore peut être renforcée par l'artifice consistant à parler de cette identification au futur. 
Ainsi,  la  métaphore  reconnue,  traditionnelle,  sert  de  point  de  départ  a  des  précisions,  à  des 
argumentations, tout comme un fait indiscutable. Témoin ce passage de La Bruyère : 
\bigskip
Dans cent ans le inonde subsistera encore en son entier : ce sera le même théâtre et les mêmes 
décorations, ce ne seront plus les mêmes acteurs. » 
\bigskip
P 539 : « Et la conclusion, quoique nous ramenant au domaine du thème, s'exprime encore par la 
métaphore : 
\bigskip
Quel fond à faire sur un personnage de comédie (1) ! 
\bigskip
Parfois, la fusion des domaines, quand elle se manifeste par la création de mots composés, tels que 
« bateau-mouche » (et que Estève qualifie de « métaphore honteuse ») (2), enrichit le langage de 
raccourcis  expressifs  ;  on  peut  même  se  demander  si  certaines  formes  verbales  comme  « 
genpillehommes » (3) ou « banksters »ne relèvent pas de la fusion métaphorique. » 
\bigskip
(1) LA Bruyère, Œuvres Bibl. de la Pléiade, Caractères, De la cour, 99, p. 266.  
(2) CI.-L. Eestève, Etudes philosophiques sur l'expression littéraire, p. 268. 
(3) Cf. Romain Rolland, Colas Breugnon, p. 27 ; cf. déjà chez Balzac gens-pille-hommes " Bibl. de 
la Pléiade, vol. VII, Les Chouans, p. 808. 
\bigskip
P 539-540 : « Ajoutons que la fusion métaphorique, même lorsqu'il s'agit d'analogies de caractère 
pittoresque, ne nous met pas en présence d'une image. « Fleur de plume », « bouquet d'ailes », « 
vaisseau d'écailles » (4) ne sont pas l'évocation d'un objet concret, réel ou fantastique, qui en toute 
sa complexe netteté, représenterait l'oiseau ou le poisson. Concevoir la métaphore comme dérivant 
de l'analogie, et l'analogie comme confrontation de relations, est la manière qui nous paraît la plus 
efficace  de  lutter,  sur  le  plan  théorique,  contre  l'erreur  dénoncée  à  juste  titre  par  Richards,  de 
considérer  la  métaphore  comme  une  image  (1).  Et,  sur  le  plan  pratique,  d'une  part  de  nous 
prémunir contre le danger des métaphores susceptibles d'être confondues avec une ressemblance 
de termes, notamment lorsqu'il s'agit de la fusion des termes A et C d'une analogie, et d'autre part 
de  nous  libérer  des  entraves  que  certains  voudraient  imposer  à  la  succession  de  métaphores 
apparemment inconciliables (2). » 
\bigskip
(4) Calderon, La vida es sueno, 1, 2. 
(1) 1. A. Richards, The Philosophy of Rhetoric, p. 16. 
2) Cf. la critique d'un poème de Lamartine, par Baron, De la Rhétorique. pp. 325-327. 
\bigskip
P 540 : « Toute analogie - hors celles qui se présentent dans des formes rigides, telles l'allégorie, la 
parabole - devient spontanément métaphore. C'est même l'absence de fusion qui nous obligerait à 
voir dans l'allégorie, dans la parabole, des formes conventionnelles oh la fusion est, par tradition, 
systématiquement  refusée.  Loin  que  l'allégorie  soit  une  métaphore  (3),  nous  aurions  en  elle  une 
double  chaîne  se  déroulant  avec  un  minimum  de  contacts.  Il  y  a  dans  l'analogie,  par  son 
prolongement même, une action qui tend à la fusion. Cette action suppose un déroulement dans le 
\bigskip
\bigskip
\bigskip
287 
\bigskip
temps, qu' une représentation non discursive est généralement incapable de rendre. C'est pourquoi 
la  peinture,  par  son  caractère  non-temporel,  doit,  soit  exprimer  uniquement  le  phore  d'une 
allégorie, qui restera toujours indépendant du thème, soit passer immédiatement à la métaphore 
au  moyen  de  la  fusion  métaphorique.  On  aboutira  à  la  création  d'êtres  bizarres  :  pour  parler  de 
l'univers en termes humains, on représentera un homme affublé d'une tête en forme de globe. Les 
dessinateurs satiriques utilisent souvent cette fusion métaphorique. » 
\bigskip
(3) Quintilien, Vol. III, liv. VIII, chap. VI, § 44. 
\bigskip
P 541 : « Elle se prête d'ailleurs admirablement à des effets comiques de toute espèce: le persiflage 
dans « César de Carnaval», dans «Mudville Milton» (pour désigner un poète de village), s'exprime 
par l'opposition de valeurs entre les termes accolés. 
\bigskip
Elle se prête aussi à un usage assez particulier, qui se confond avec celui que nous avons reconnu à 
l'hyperbole (1). Dire d'un coureur à pied qu'il fait du 120 à l'heure, est-ce métaphore ou hyperbole ? 
L'expression agit peut-être par la sommation des deux procédés ? Grâce à la métaphore, l'intrusion 
d'un nouveau domaine concourrait au dépassement hyperbolique. 
\bigskip
Rien d'étonnant, à ce que la métaphore, fusionnant les domaines, transcendant les classifications 
traditionnelles  soit,  par  excellence,  l'instrument  de  la  création  poétique  et  philosophique.  La 
célèbre pensée de Pascal « l'homme n'est qu'un roseau, le plus faible de la nature ; mais c'est un 
roseau pensant » (2), réalise la fusion du thème et du phore, en une formule inoubliable. 
\bigskip
Si on perd de vue leur -aspect métaphorique, ces formules peuvent faire naître la féerie. Le conte 
décrivant les tribulations d'un roseau pensant provoquerait l'émerveillement, en prenant à la lettre 
une expression qui ne réalisait une fusion de domaines que dans l'univers conceptuel. 
\bigskip
Les  métaphores  peuvent  aussi  faire  naître  le  comique  involontaire,  On  connaît  l'anecdote  de 
l'acheteuse  demandant  au  quincailler  de  lui  fournir  de  «  ces  rideaux  de  fer  dont  tout  le  monde 
parle aujourd'hui ». 
\bigskip
(1) Cf. § 67 : Le dépassement. 
(2) Pascal, Œuvre, Bibl. de la Pléiade, Pensées, 264 (65), p. 894 (347, ed. Brunschvicg). 
\bigskip
P  541-542 :  « S'il  est  vain  de  chercher,  dans  la  nature,  des  êtres  correspondant  aux  créations 
métaphoriques,  celles-ci  n'en  ont  pas  moins  une  action  sur  la  vie  des  notions.  La  métaphore 
n'exerce  pas  seulement  son  influence  sur  l'argumentation  en  vue  de  laquelle  elle  est  créée.  Elle 
peut contribuer notamment à la confusion des notions. Ayant utilisé la notion d'esclave dans des 
métaphores telles « esclave du patron », « esclave des passions », on sera incité à rechercher quels 
sont les éléments communs au terme «esclave » dans tous les usages qui vont interagir les uns sur 
les autres. » 
\bigskip
Les maîtres de rhétorique ont vu, dans la métaphore, un moyen de parer à l'indigence du langage 
(1).  Il  est  certain  qu'elle  peut  remplir  ce  rôle,  encore  qu'il  ne  soit  vraisemblablement  que  très 
secondaire et que la notion d'indigence soit, de ce point de vue, malaisée à élucider. 
\bigskip
Quoi  qu'il  en  soit,  l'usage  fréquent  d'une  métaphore  ne  peut  que  contribuer  à  une  assimilation 
entre  phore  et  thème,  ce  qui  suffirait  à  expliquer  que  bon  nombre  de  relations,  dans  un  milieu 
culturel donné, semblent s'appliquer tout naturellement au domaine du thème comme à celui du 
phore. Faut-il supposer, dans ce cas, qu'il s'agit d'usage métaphorique de notions qui provenaient 
d'un  des  domaines,  ou  s'agit-il  au  contraire  de  notions  applicables,  dans  leur  usage  propre,  à 
plusieurs  domaines  ?  La  réponse  à  cette  question  sera  le  plus  souvent  déterminée  par  des 
considérations  d'ordre  philosophique,  auxquelles  nous  avons  déjà  fait  plusieurs  fois  allusion  (2). 
\bigskip
\bigskip
\bigskip
288 
\bigskip
Des  exemples,  tels  celui  des  mots  «  opaque  »,  «transparent  »  appliqués  au  domaine  spirituel 
feraient  cependant  pencher  pour  la  première  hypothèse.  En  effet  il  semble  que,  quoi qu'en  aient 
certains auteurs (3), leur usage soit encore perçu comme métaphore. 
\bigskip
(1) Cicéron, De Oratore, liv. Ill, § 155; Quintilien, Vol. III, liv. VIII, chap. VI, 4 ; cf. CI.-L. Estève, 
Etudes philosophiques sur l'expression littéraire, pp. 227 et Suiv. 
(2) Cf. § 76 : L'argument de double hiérarchie ; § 86 : Le statut de l'analogie. 
(3) G. Marcel, Position et approches concrètes du mystère ontologique, p. 296 G. Bernanos, La 
joie, p. 119. 
\bigskip
§  88.  LES  EXPRESSIONS  A  SENS  METAPHORIQUE  OU  METAPHORES 
\bigskip
ENDORMIES 
P 542-543 : « Un danger des métaphores, c'est leur usure. La métaphore n'est plus perçue comme 
fusion,  comme  accolement  de  termes  empruntés  à  des  domaines  différents,  mais  comme 
application d'un vocable à ce qu'il désigne normalement : la métaphore, d'agissante est devenue « 
endormie », caractère qui marque mieux que d'autres adjectifs (méconnue, oubliée, fanée) que cet 
état  peut  n'être  que  transitoire,  que  ces  métaphores  peuvent  être  réveillées  et  redevenir 
agissantes. » 
\bigskip
P 543 : « La métaphore endormie, ou expression à sens métaphorique, parut à Whately, à la suite 
de Stewart et de Copleston, un outil très supérieur à la métaphore agissante parce qu'elle a perdu 
son  contact  avec  l'idée  primitive qu'elle  dénotait  (1)  ;  de  même  ,Stevenson  admet  que,  parce que 
son  interprétation  est  univoque,  elle  peut  fournir  une  raison,  contrairement  à  la  métaphore 
agissante qui ne serait que suggestive (2). 
\bigskip
Il nous semble, quant à nous, que leur valeur dans l'argumentation est surtout éminente à cause de 
la  grande  force  persuasive  que  possèdent  ces  métaphores;  endormies  quand,  à  l'aide  de  l'une  ou 
l'autre technique, elles sont remises en action. Cette force résulte de ce qu'elles empruntent leurs 
effets à un matériel analogique, aisément admis, car il est non seulement connu, mais intégré, par 
le langage, dans la tradition culturelle. 
\bigskip
La manière la plus usuelle de réveiller une métaphore est de développer à nouveau, à partir d'elle, 
une  analogie.  C'est  ainsi  que,  pour  réveiller  la  métaphore  endormie  dans  l'expression  «  emporté 
par  ses  passions  »,  Bossuet  développe  le  phore  qui  peut  être  considéré  comme  sous-jacent  à 
l'analogie oubliée : 
\bigskip
Voyez cet insensé sur le bord d'un fictive, qui, voulant passer à l'autre rive, attend que le fleuve se 
soit écoulé ; et il lie s'aperçoit pas qu'il coule sans cesse. Il faut passer par-dessus le fleuve ; il f aut 
marcher contre le torrent, résister au cours de nos passions, et non attendre (le voir écoulé ce qui 
ne s'écoule jamais tout à fait (3). 
\bigskip
(1) R. D. D. Whately, Elements of Rhetoric, pp. 360-61, Appendix (B) to p. 67. 
(2) Charles L. STEVENSON, Ethics and Language, p. 143.  
(3) Bossuet, Sermons vol. II : Sur l'ardeur de la pénitence, 1). 388. 
\bigskip
P  544 :  « De  même  Kant  n'hésite  pas,  en  la  développant,  à  rendre  agissante  l'expression 
métaphorique, « apporter de la lumière sur un sujet »: 
\bigskip
Il  [Hume]  n'apporta  aucune  lumière  en  cette  sorte  de  connaissance  [métaphysique],  mais  il  fit 
néanmoins  jaillir  une  étincelle  avec  laquelle  on  aurait  bien  pu  avoir  de  la  lumière,  si  elle  avait 
atteint une mèche inflammable dont la lueur eût été avec soin entretenue et augmentée (1). » 
\bigskip
\bigskip
\bigskip
\bigskip
289 
\bigskip
Dès  que  le  phore  d'une  expression  métaphorique  est  évoqué  par  un  détail  quelconque,  de 
préférence par le prolongement de l'analogie, le cliché le plus banal peut redevenir vivant : 
\bigskip
Les grands et les petits ont mêmes accidents, et mêmes fâcheries, et mêmes passions ; mais l'un 
est au haut de la roue, et l'autre près du centre, et ainsi moins agité par les mêmes mouvements 
(2). 
\bigskip
L'expression  métaphorique  que  l'on  fait  revivre  ne  doit  même  pas,  on  le  voit,  être  explicitement 
énoncée. « La roue de la fortune »n'est ici qu'un cliché sous-jacent au texte. 
\bigskip
Le  réveil  est  parfois  obtenu  rien  qu'en  accolant,  l'une  à 
l'autre,  quelques  expressions 
métaphoriques.  Lorsqu'elles  peuvent  convenir  comme  éléments  d'une  même  analogie,  elles 
agissent  l'une  sur  l'autre,  ce  qui  provoque  le  réveil  des  métaphores.  Ainsi  dans  ce  passage  de 
Démosthène : 
\bigskip
... vous leur savez gré de vous donner ce qui est à vous. Quant a eux, après vous avoir parqués 
dans la ville, ils vous mènent à cette curée et vous apprivoisent pour vous domestiquer (3). 
\bigskip
Les mots «  parquer », « curée », « apprivoiser » et « domestiquer », qui, isolément, auraient pu 
passer  pour  des  expressions  à  sens  métaphorique,  sont  perçus  comme  des  métaphores 
agissantes. » 
\bigskip
(1) Kant, Prolégomènes à toute métaphysique future, p. 10. 
(2) Pascal, Œuvre, Bibl. de la Pléiade, Pensées, 223 (442*), p. 88.5 (180 éd. Brunschvicg). 
\bigskip
P 544-545 : « La répétition d'une même expression, une fois dans le sens métaphorique l'autre fois 
au sens propre, peut aussi faire revivre une métaphore, comme dans l'expression anglaise : 
\bigskip
We have to hang together, not to hang separately.” 
\bigskip
P  545 :  « Il  suffit  d'opposer  l'expression  métaphorique  à  une  autre,  empruntée  au  domaine  du 
phore, pour obtenir le même effet : 
\bigskip
Au  lieu  d'être  une  impasse  comme  le  professe  l'ancienne  psychologie,  l'abstraction  est  un 
carrefour d'avenues (1). 
\bigskip
Parfois  le  réveil  est  produit  en  greffant  sur  l'expression  à  sens  métaphorique  une  nouvelle 
métaphore qui complète la première : 
\bigskip
Quand les rochers serrent les dents sur la langue de sable (2). 
\bigskip
La distinction entre le développement d'un même phore et son complément peut paraître délicate. 
Nous croyons pourtant qu'elle est utile pour marquer ce qu'il y a d'inattendu dans la manière de 
greffer  l'une  sur  l'autre  ces  expressions  métaphoriques.  Bergson  utilise  volontiers  ce  jeu  des 
compléments pour faire revivre les expressions les plus banales : 
\bigskip
La science moderne est fille de l'astronomie; elle est descendue du ciel sur la terre le long du plan 
incliné de Galilée, car c'est par Galilée que Newton et ses successeurs se relient à Képler (3). 
\bigskip
Il faut pourtant prendre garde à ne pas rapprocher deux expressions métaphoriques sans veiller à 
l'effet produit par leur voisinage. Il peut être d'un comique involontaire et désastreux; telle serait la 
formule : 
\bigskip
\bigskip
\bigskip
\bigskip
290 
\bigskip
Cette grosse légume, qui est la crème de la société. » 
\bigskip
(1) G. Bachelard, Le rationalisme appliqué, p. 22. 
(2) Cité par André Du Bouchet, Envergure de Reverdy, Critique 47, avril 1951, p. 315 . 
(3) Bergson, L'évolution créatrice, p. 362. 
\bigskip
P 545-546 : « Le réveil de la métaphore peut aussi être provoqué par un changement du contexte 
habituel,  par  l'emploi  de  l'expression  métaphorique  dans  des  conditions  qui,  en  lui  donnant  un 
caractère inusité, attirent l'attention sur la métaphore qu'elle contient. Une légère distorsion peut 
suffire à rendre à l'expression son pouvoir analogique. L'expression métaphorique « s'éteindre », 
qui  passe  inaperçue  dans  «  s'éteindre  doucement  »,  redeviendrait  vivante  dans  «  s'éteindre 
soudain  ».  Ce  contexte  nouveau  peut  n'être  autre  chose  que  la  personnalité  de  celui  qui  utilise 
l'expression. L'expression métaphorique stéréotypée peut recouvrer vie dans la bouche de certains 
orateurs (1), parce que l'on présume que, chez eux, elle ne peut avoir son sens banal et usuel. Le 
poète et le philosophe sont peut-être privilégiés à cet égard. » 
\bigskip
(1) Cf. .N. Reyes, El Deslinde, p. 204. 
\bigskip
P 546 : « Les expressions à sens métaphorique n'étant pas les mêmes dans les diverses langues, et 
le  degré  d'assoupissement  d'une  même  métaphore  pouvant  être  fort  différent,  la  traduction 
modifiera toujours quelque chose à cet égard. Elle aura le plus souvent pour conséquence de faire 
revivre les métaphores. Il y a plus. Un texte étranger, lu dans sa langue originale, donne souvent, si 
celle-ci n'est pas tout à fait familière au lecteur, une impression de vie et de mouvement, un plaisir 
particulier, qui proviennent de ce que l'on saisit comme métaphore vivante ce qui n'était peut-être 
que métaphore endormie. 
\bigskip
L'identité  de  milieu  culturel  permet  seule  l'assoupissement  des  métaphores.  Les  expressions  du 
langage des professions, de l'argot, nous paraissent métaphoriques, alors que, pour l'usager, c'est 
la façon normale de s'exprimer (2). Et si l'on a si souvent mentionné l'abondance des métaphores 
chez les primitifs, dans les classes inférieures, paysannes (3), cela tient peut-être, en partie, à tout 
ce qui, culturellement, les sépare de l'observateur. » 
\bigskip
(2) Cf. CI.-L. Estève, Etudes philosophiques sur l'expression littéraire, p. 206. 
(3) Dumarsais, Des Tropes, p. 2 ; BARON, De la Rhétorique, p. 308 ; S. 1. Hayakawa, Language in 
Thought  and Action, p. 121. 
\bigskip
P  546-547 :  « Le  réveil  d'une  métaphore  peut  évidemment  produire  des  thèmes  divers.  La 
connivence  entre  orateur  et  auditeur  n'est  Jamais  que  partielle  ;  ni  l'un  ni  l'autre  n'ont,  le  plus 
souvent, une idée précise de la genèse d'une expression métaphorique. La force de celle-ci vient à 
la fois de la familiarité avec elle et de la  connaissance assez imprécise de  l'analogie qui est à son 
origine. » 
\bigskip
P 547 : « Il importe même assez peu que l'expression ait réellement une origine métaphorique. En 
effet le « réveil » est un phénomène (lui a lieu dans le présent ; il suffit, pour qu'il se produise, que 
l'expression  puisse  être  perçue  métaphoriquement,  peut-être  par  analogie  avec  d'autres.  La 
catachrèse est, selon les Anciens, une figure qui « par une sorte d'abus substitue au mot propre et 
précis un mot de sens voisin et approchant » (1) ; d'aucuns, tels Vico (2), Dumarsais (3), Baron (4), 
insistent  sur  son  rapport  avec  la  métaphore;  d'autres  l'en  distinguent  soigneusement  (5).  Un 
exemple en serait « feuille de papier ». Même si pareille expression n'a rien d'une métaphore, si sa 
genèse  est  différente,  et  si  elle  est  plutôt  une  extension,  une  «  projection  »,  qu'une  fusion,  cela 
n'empêche qu'elle puisse devenir une métaphore agissante par l'usage des techniques de réveil que 
nous venons de signaler. » 
\bigskip
\bigskip
\bigskip
\bigskip
291 
\bigskip
(1) Rhétorique à Herennius, liv. IV, § 45. 
(2) Vico, Delle instituzioni oratorie, p. 137. 
(3) Dumarsais, Des Tropes, pp. 32 et suiv. 
(4) Baron, De la Rhétorique, p. 349. 
(5)  Quintilien,  Vol.  III,  liv.  VIII,  chap.  VI,  §  34;  CI.-L.  ESTÈVE,  Etudes  philosophiques  sur 
l'expression littéraire, pp. 244 et suiv. ; cf. S. Ullmann, Précis de sémantique française, pp. 253 et 
suiv. 
\bigskip
P  547-548 :  « Ces  techniques  peuvent  même  s'appliquer  à  des  expressions  tout  àfait  normales, 
notamment à certains adjectifs. Le contexte seul indique que l'auteur entend les prendre dans un 
sens analogique. Ainsi lorsque Kelsen écrit :  
\bigskip
Le  droit  des  peuples  est  encore  un  ordre  juridique  primitif.  Il  est  seulement  au  début  d'un 
développement que le droit de chaque état a déjà accompli (6). 
\bigskip
le  contexte  nous  indique  que  l'auteur  prend  l'adjectif  «  primitif  »comme  phore  indiquant  les 
possibilités de développement 
\bigskip
que possèdent les peuples primitifs. De même lorsque Bacon écrit : 
\bigskip
Ces temps-là sont les temps anciens, lorsque le monde est ancien, et non ceux que nous appelons 
anciens ordine rétrograde, en comptant à reculons à partir de nous-même (1). 
\bigskip
« temps anciens », par la contestation que l'on élève, devient une métaphore dont le phore serait la 
vie humaine, avec ce qu'elle apporte d'expérience et de sagesse. La force persuasive de pareils 
arguments réside dans l'analogie que l'on évoque. » 
\bigskip
(6) H. Kelsen, Reine Rechtslehre, p. 131, 
(1) Bacon, Of the advancement of learning, p. 35. 
\bigskip
P  548  :  “Nous  avons  insisté  au  paragraphe  précédent  sur  la  tendance  au  dépassement  propre  à 
l'analogie. La métaphore endormie est une forme de dépassement, dont on ne se rend plus compte, 
qui  est  accepté  :  le  terme  propre  et  le  terme  métaphorique  peuvent  même  en  arriver  à  être 
subsumés sous une même classe. 
\bigskip
Toutefois,  il  arrive  que  l'on  fasse  revivre  la  métaphore  pour  montrer  que  c'est  devant  une 
ressemblance  de  relations  que  l'on  se  trouve,  et  pour  opérer  un  dépassement  de  l'analogie  qui 
porte directement sur elle et non sur ses termes. Ainsi Köhler écrira : 
\bigskip
On a parfois comparé la vie à une flamme. Ceci est plus 'qu'une métaphore poétique, car du point 
de vue fonctionnel et énergétique, la vie et la flamme ont réellement beaucoup en commun (2). » 
\bigskip
(2) W. Köhler, The place of value in a world of farts, p. 320. 
\bigskip
P 548-549 : « Il développe ensuite longuement les ressemblances de structure, indiquant ainsi que 
l'analogie  est  meilleure  qu'on  n'aurait  pu  le  croire  et  ne  concerne  pas  seulement  certains  aspects 
des termes, visibles pour tous. Le même auteur dira encore : 
\bigskip
D'une façon métaphorique les ressorts de l'action humaine ont été souvent appelés des « forces ». 
Il  apparaît  que,  si  ces  ressorts  doivent  avoir  une  contrepartie,  celle-ci  ne  peut  consister  qu'en 
forces  au  sens  strict  du  mot.  D'autre  part,  si  ce  sont  réellement  des  forces,  leur  comportement 
dans un contexte de phénomènes neuraux ressemblera à la motivation humaine dans une mesure 
\bigskip
\bigskip
\bigskip
292 
\bigskip
telle  que  je  me  demande  si  structurellement  et  fonctionnellement  il  restera  quelque  différence 
(1). » 
\bigskip
(1) W. Köhler, The place of value in a world of facts, p. 357. 
\bigskip
P  549 :  « Faire  revivre  la  métaphore  a  ici  pour  but  de  permettre,  sur  nouveaux  frais,  le 
dépassement  de  l'analogie  tout  en  bénéficiant  de  l'adhésion  au  dépassement  primitif,  que  l'on 
suppose  basé  sur une  intuition.  On  se  fonde  sur  le  langage,  on  en  souligne  les  embûches,  et  l'on 
utilise néanmoins la part d'acquiescement aux thèses que l'on défend qui s'exprimait en lui. 
\bigskip
La  métaphore,  fusion  analogique,  remplit  tous  les  rôles  que  remplit  l'analogie.  Elle  s'en  acquitte 
mieux  encore  sur  certains  points,  parce  qu'elle  renforce  l'analogie  ;  la  métaphore  condensée 
l'intègre dans le langage. Mais seul le réveil de la métaphore permettra de dégager sa structure, et 
ce  premier  pas  franchi,  de  dépasser  l'analogie.  La  critique,  elle,  pourra  se  servir  de  tous  les 
procédés  utilisés  pour  rejeter  l'analogie.  Elle  arguera  souvent  en  plus  de  l'obscurité  de  la 
métaphore. 
\bigskip
Les  différences  d'attitude  possible  en  face  d'une  métaphore  montrent  que  celle-ci  peut  être 
envisagée en fonction de l'argumentation. L'étude en est, de ce point de vue, plus facile àcertains 
égards qu'elle ne l'est du point de vue de la psychologie individuelle. On a suffisamment montré les 
difficultés qu'entraîne ce dernier (2). Car il s'intéresse au créateur de la métaphore, alors que c'est 
au  cours  de  la  discussion  que  se  décidera  souvent  s'il  s'agit  ou  non  de  métaphore,  si  l'on  est  en 
présence d'ordres différents. La notion même de « sens littéral » et de « sens métaphorique » peut 
être une dissociation qui naît de la discussion, et non pas une donnée " primitive » (3). » 
\bigskip
(2) Cf. C. F. P. Stutterheim, Psychologische interpretatie van taal-verschynelen , Nieuwe Taalgids, 
XXXI, p. 265 et Flet begrip metaphoor, pp. 188 et suiv., pp. 525 et suiv. 
(3) Cf. § 93 ; L'expression des dissociations. 
\bigskip
CHAPITRE IV LA DISSOCIATION DES NOTIONS 
\bigskip
§ 89. RUPTURE DE LIAISON ET DISSOCIATION 
\bigskip
P  550 :  « Les  trois  premiers  chapitres  de  ce  livre  ont  été  consacres  a  l'étude  des  liaisons 
argumentatives qui rendent solidaires les uns des autres, des éléments que l'on pouvait considérer 
comme  indépendants  au  départ.  L'opposition  à  l'établissement  d'une  pareille  solidarité  se 
marquera  par  le  refus  de  reconnaître  l'existence  d'une  liaison.  On  montrera,  notamment,  qu'une 
liaison que l'on avait considérée comme admise, que l'on avait présumée ou souhaitée, n'existe pas, 
parce  que  rien  ne  permet  de  constater  ou  de  justifier  l'influence  que  certains  phénomènes 
envisagés auraient sur ceux qui sont en cause, et que, par conséquent, la prise en considération des 
premiers est irrelevante : 
\bigskip
Oh ! sans doute, si en supprimant ce qui est pénible à dire, de peur de vous faire de la peine, on 
supprimait du même coup les événements, il faudrait ne tenir que des discours propres à plaire 
(1). 
\bigskip
L'expérience,  réelle  ou  mentale,  la  modification  des  conditions  d'une  situation,  et,  plus 
spécialement,  en  sciences,  l'examen  isolé  de  certaines  variables,  pourront  servir  à  prouver  le 
manque de liaison. On s'efforcera, aussi, de présenter tous les inconvénients de celle-ci. » 
\bigskip
(1) Démosthène, Harangues, t. 1 : Première Philippique, § 38. 
\bigskip
P  550-551 :  « La  technique  de  rupture  de  liaison  consiste  donc  à  affirmer  que  sont  indûment 
associés  des  éléments  qui  devraient  rester  séparés  et  indépendants.  Par  contre,  la  dissociation 
\bigskip
\bigskip
\bigskip
293 
\bigskip
présuppose l'unité primitive des éléments confondus au sein d'une même conception, désignés par 
une même notion. La dissociation des notions détermine un remaniement plus ou moins profond 
des données conceptuelles qui servent de fondement à l'argumentation : il ne s'agit plus, dans ce 
cas, de rompre les fils qui rattachent des éléments isolés, mais de modifier la structure même de 
ceux-ci. » 
\bigskip
P  551 :  « Au  premier  abord  la  différence  entre  rupture  de  liaison  et  dissociation  des  notion  est 
profonde  et  immédiatement  discernable,  mais  en  réalité,  cette  distinction,  comme  les  autres 
oppositions  dites  de  nature,  peut  être  elle-même  fort  controversée.  Selon  que  les  liaisons  entre 
éléments  seront  considérées  comme  «  naturelles  »  ou  comme  «  artificielles  »,  comme  « 
essentielles » ou " accidentelles », l'un verra une dissociation des notions en ce qui, pour un autre, 
n'est que rupture de liaison. 
\bigskip
Dans  des  textes  célèbres,  Locke,  pour  qui  toute  Église  n'est  qu'une  association  volontaire  ayant 
pour but le salut de ses membres, refuse le lien établi, à son époque, entre l'État et la religion : 
\bigskip
Ni le droit ni l'art de gouverner, écrit-il, n'entraînent nécessairement une connaissance certaine 
d'autres  matières,  et  le  moins  encore  celle  de  la  vraie  religion.  Car,  s'il  en  était  ainsi,  comment 
serait-il  possible  que  les  maîtres  de  la  terre  diffèrent  si  profondément  comme  c  est  le  cas,  en 
matières religieuses (1) ? » 
\bigskip
(1) Locke The second treatise of civil government and A letter concerning toleration p. 139. 
\bigskip
P  551 :  « Pour  Locke,  le  temporel  est,  dès  l'abord,  séparé  du  spirituel,  et  l'auteur  s'oppose  à  une 
liaison  dont  il  montre  les  effets  ridicules.  Pour  un  adversaire  de  Locke,  le  temporel  implique  le 
spirituel,  et  l'effort  fait  pour  les  séparer  sera  considéré  comme  une  dissociation  d'éléments  tout 
naturellement unis. Mais cette union, Locke la reconnaît uniquement dans un régime théocratique, 
qui n'est plus celui des États modernes : en la maintenant, ceux-ci traiteraient comme règle ce qui 
n'est qu'une exception, sans rapport avec la situation actuelle (1). Ce qui n'est qu'accidentel, propre 
à un régime politique particulier, ne peut être partie intégrante de l'idée de gouvernement civil : la 
liaison entre l'État et la religion doit être récusée. » 
\bigskip
(1) Locke, ibid., pp. 148-149. 
\bigskip
P 552 : « C'est donc, en fin de compte, la situation argumentative dans son ensemble, et surtout les 
notions  sur  lesquelles  l'argumentation  prend  appui,  les  remaniements  auxquels  elle  conduit,  les 
techniques qui permettent de les opérer, qui nous indiqueront la présence d'une dissociation des 
notions et non d'un simple refus de liaison. » 
\bigskip
P 552-553 : « Ce que Rémy de Gourmont qualifiait de phénomènes d'association et de dissociation 
(2),  ce  que  Kenneth  Burke  appelle  des  identifications  (3),  ne  constituent  à  nos  yeux,  que  des 
liaisons  et  des  refus  de  liaisons,  car  les  notions,  associées  et  dissociées,  semblent  rester,  après 
l'opération,  telles  qu'elles  étaient  dans  leur  état  primitif,  comme  des  briques  récupérées  intactes 
d'un bâtiment en démolition. La dissociation des notions, comme nous la concevons, consiste dans 
un remaniement plus profond, toujours provoqué par le désir de lever une incompatibilité, née de 
la confrontation d'une thèse avec d'autres, qu'il soit question de normes, de faits ou de vérités. Des 
solutions pratiques permettent de résoudre la difficulté sur le plan exclusif de l'action, d'éviter que 
l'incompatibilité se présente, de la diluer dans le temps, de sacrifier une des valeurs qui entrent en 
conflit, ou les deux. La dissociation des notions correspond, sur ce plan pratique, à un compromis, 
mais elle conduit, sur le plan théorique, à une solution qui vaudra également dans l'avenir parce 
que,  en  restructurant  notre  conception  du  réel,  elle  empêche  la  réapparition  de  la  même 
incompatibilité.  Elle  sauvegarde,  au moins  partiellement,  les  éléments  incompatibles.  Si  l'objet  a 
disparu dans l'opération, c'est cependant aux moindres fais que celle-ci s'est réalisée, car on donne, 
\bigskip
\bigskip
\bigskip
294 
\bigskip
à ce à quoi on tient, sa juste place dans la pensée, tout en procurant à celle-ci une cohérence à l'abri 
des difficultés de même ordre. Un exemple typique en serait la solution kantienne de l'antinomie 
entre  le  déterminisme  universel  et  la  liberté  de  l'homme,  qui  dissocie  le  concept  de  causalité  en 
causalité intelligible et sensible (1), dissociation rendue possible par celle de la notion de réalité en 
réalité phénoménale et nouménale. » 
\bigskip
(2) Remy De Gourmont, La culture des idées, pp. 79 et suiv. 
(3) Kenneth Burke, A Rhetoric of motives, p. 150. 
(1) Kant, Critique de la raison pure, pp. 457 à 460. 
\bigskip
P  553 :  « Les  notions  nouvelles,  qui  résultent  de  la  dissociation,  peuvent  acquérir  une  telle 
consistance,  être  si  fortement  élaborées,  et  apparaître  comme  si  indissolublement  liées  à 
l'incompatibilité qu'elles permettent de résoudre, que présenter celle-ci dans toute sa force semble 
une autre façon de poser la dissociation. L'idée du péché originel  - qui résoud, par la dissociation 
de  la  notion  d'homme  en  «  homme  dans  l'état  de  la  création  »  et  «  homme  déchu  »,  certaines 
incompatibilités entre la bonté de Dieu et l'existence du mal, la libre volonté de l'homme et la libre 
volonté de Dieu - sera pour Pascal une raison d'insister sur l'incompatibilité entre la grandeur et la 
misère de l'homme. Parlant du mystère du péché originel il pourra affirmer : 
\bigskip
Le noeud de notre condition prend ses replis et ses tours dans cet abîme ; de sorte que l'homme 
est plus inconcevable sans ce mystère que ce mystère n'est inconcevable à l'homme (2). » 
\bigskip
(2) Pascal, Bibl. de la Pléiade, Pensées, 438 (261), p. 948 (434 éd. Brunschvicg). 
\bigskip
P 553-554 : « La solution admise semble, parfois, si assurée que l'on traitera de faute de logique, de 
sophisme, le fait de ne pas en tenir compte. C'est ainsi que l'affirmation d'une incompatibilité entre 
la  règle  hindoue  de  non-violence  et  la  coutume  védique  des  sacrifices  sanglants  est,  pour 
Candrasimha, auteur de savants commentaires de logique, un bon exemple de sophisme. Car il n'y 
a, d'après lui, cruauté que lors de la transgression d'une règle, quand on commet un acte illicite (1). 
Cette  définition  de  la  cruauté,  qui  s'éloigne  du  sens  habituel  de  ce  terme,  et  qui  résulte  d'une 
dissociation,  semble  pourtant  tellement  incontestable  à  notre  auteur  qu'il  considère  comme  une 
faute de logique le retour au sens primitif. » 
\bigskip
(1) Annabhatta, Le compendium des topiques, pp. 146-147. 
\bigskip
P 554 : « Nous montrerons, par la suite, que toute philosophie nouvelle suppose l'élaboration d'un 
appareil  conceptuel,  dont  une  partie  au  moins,  celle  qui  est  fondamentalement  originale,  résulte 
d'une dissociation des notions permettant de résoudre les problèmes que le philosophe s'est posés. 
Ceci expliquera, entre autres, le grand intérêt que mérite, d'après nous, l'étude de la technique des 
dissociations. 
\bigskip
Avant  nous,  cependant,  des  juristes  éminents  avaient  remarqué  que  le  droit  était  le  domaine  de 
prédilection du  compromis, technique de solution d'incompatibilités. Le but de l'effort juridique, 
écrit Demogue, n'est pas la synthèse logique, mais le compromis. Le progrès du droit consiste en 
l'élaboration de techniques, toujours imparfaites, permettant de concilier des exigences opposées 
(2). Reprenant ces idées, le grand juriste américain Cardozo écrira : 
\bigskip
La  conciliation  de  l'irréconciliable,  le  mélange  des  antithèses,  la  synthèse  des  oppositions,  voici 
les grands problèmes du droit (3). » 
\bigskip
(2) K. Demogue, Les notions fondamentales du droit privé, notamment, pp. 38, 75, 196, 198. 
(3) B. N. Cardozo, The Paradoxes of legal science, p. 4. 
\bigskip
\bigskip
\bigskip
\bigskip
295 
\bigskip
P  554-555 :  « Cet  effort  pour  résoudre  des  incompatibilités  se  poursuit  à  tous  les  niveaux  de 
l'activité  juridique.  Il  est  le  fait  du  législateur,  du  théoricien  du  droit,  du  juge.  Quand  le  juge  se 
trouve  dans  une  affaire  déterminée,  devant  une  antinomie  juridique,  il  ne  peut  négliger 
entièrement  l'une  des  deux  lois  au  profit  de  l'autre  ;  il  doit  justifier  sa  façon  d'agir  par  une 
délimitation  du  champ  d'application  de  chaque  loi,  par  des  interprétations  qui  rétablis  sent  la 
cohérence  du  système  juridique.  Il  introduira  des  distinctions,  destinées  à  concilier  ce  qui,  sans 
elles,  serait  inconciliable  (1).Les  «  distinguo  »  de  la  théologie  scolastique  remplissent  le  même 
office. » 
\bigskip
(1) F. Berriat, Saint-Prix, Manuel de logique juridique p. 233. 
\bigskip
P  555 :  « Quand  les  lois  ne  doivent  pas  être  appliquées,  quand  elles  ont  perdu  leur  caractère 
obligatoire, on peut, en figeant les catégories juridiques, montrer leur caractère antinomique dans 
des cas déterminés. -Mais l'effort du juriste va à l'opposé de cette rigidité. Son rôle est d'élaborer 
un système permettant de résoudre les conflits. Lorsque Napoléon, déjà Consul à vie, s'écrie à une 
séance  du  Conseil  d'Etat  :  «  Comment  concilier  l'hérédité  de  la  couronne  avec  le  principe  de  la 
souveraineté  du  peuple  »,  il  lie  demande  pas  aux  juristes  qui  l'écoutent,  de  constater  une 
contradiction, il leur demande plutôt la solution d'une incompatibilité. 
\bigskip
Il ne faut pas oublier d'ailleurs qu'une même incompatibilité peut donner lieu, pour la résoudre, à 
plusieurs  aménagements  de  concepts.  Ces  solutions  rivales  pourront,  elles-mêmes,  être  posées 
comme incompatibles. Cette lutte de solutions est particulièrement saisissable en droit, mais aussi 
en  théologie.  La  querelle  des  Églises, notamment  au  début  de  l'ère  chrétienne,  constitue  souvent 
une  confrontation  de  solutions  à  certaines  difficultés  théologiques,  solutions  qui,  elles-mêmes, 
peuvent donner lieu à un aménagement pour les concilier. 
\bigskip
Sur  le  plan  théorique,  le  compromis,  devant  les  incompatibilités,  parce  qu'il  exige  une 
structuration nouvelle du réel, est ce qui demande le plus grand effort et nécessite les plus difficiles 
justifications. Par contre, une fois établi, une fois les notions dissociées et restructurées, il tend à se 
présenter comme étant la solution inéluctable et à réagir sur l'ensemble de notions dans lequel il 
s'est inséré. » 
\bigskip
§ 90. LE COUPLE « APPARENCE-REALITE » 
\bigskip
P  556 :  « Pour  bien  comprendre  la  technique  de  la  dissociation  des  notions  et  pour  mieux  en 
apprécier  les  résultatsl  il  nous  semble  utile  d'examiner  de  plus  près  un  cas  privilégié,  celui  que 
nous  considérons  comme  le  prototype  de  toute  dissociation  notionnelle,  àcause  de  son  usage 
généralisé  et  de  son  importance  philosophique  primordiale  :  il  s'agit  de  la  dissociation  donnant 
lieu au couple «apparence-réalité ». 
\bigskip
Il  n'est  pas  douteux  que  la  nécessité  de  distinguer  l'apparence  de  la  réalité  est  née  de  certaines 
difficultés, de certaines incompatibilités entre apparences ; celles-ci ne pouvaient plus, toutes, être 
considérées  comme  exprimant  la  réalité,  si  l'on  part  de  l'hypothèse  que  tous  les  aspects  du  réel 
sont compatibles entre eux. Ie bâton, plongé partiellement dans l'eau, paraît courbé, quand on le 
regarde, et droit, quand on le touche, mais en réalité, il ne peut être simultanément courbé et droit. 
Alors que les apparences peuvent s'opposer, le réel est cohérent : son élaboration aura pour effet 
de dissocier, parmi les apparences, celles qui sont trompeuses de celles qui correspondent au réel. 
\bigskip
Cette première constatation fait immédiatement ressortir le caractère équivoque, la signification et 
la valeur indécises de l'apparence : il se peut que l'apparence soit conforme à l'objet, se confonde 
avec lui, mais il se peut aussi qu'elle nous induise en erreur à son sujet. Aussi longtemps que nous 
n'avons aucune raison d'en douter, l'apparence n'est que l'aspect sous lequel se présente l'objet, on 
entend,  par  apparence,  la  manifestation  du  réel.  Ce  n'est  que  quand  les  apparences,  parce 
qu'incompatibles,  ne  peuvent  être  acceptées  toutes  à  la  fois,  que  s'opère,  grâce  à  la  distinction 

parmi les apparences de celles qui sont trompeuses et de celles qui ne le sont pas, une dissociation 
donnant lieu au couple « apparence-réalité », dont chaque terme renvoie à l'autre d'une manière 
qu'il nous faut examiner de plus près. » 
\bigskip
P 557 : « Pour la commodité de notre analyse, et nous permettre d'en généraliser la portée, nous 
dirons que dans le couple «apparence-réalité », «apparence » constitue le terme I et «réalité», le 
terme II. Dorénavant, et pour bien montrer que ces termes sont corrélatifs, nous désignerons un 
couple issu d'une dissociation de la manière suivante : apparence  ou en général, terme I 
\bigskip
\bigskip
Le terme I, correspond à l'apparent, à ce qui se présente en premier lieu, à l'actuel, à l'immédiat, à 
ce qui est connu directement. Le terme II, dans la mesure où il s'en distingue, ne se comprend que 
par rapport au terme I : il est le résultat d'une dissociation, opérée au sein du terme I, et visant à 
éliminer  les  incompatibilités  qui  peuvent  apparaître  entre  des  aspects  de  ce  dernier.  Le  terme  II 
fournit  un  critère,  une  norme  permettant  de  distinguer  ce  qui  est  valable  de  ce  qui  ne  l'est  pas, 
parmi  les  aspects  du  terme  I  ;  il  n'est  pas  simplement  un  donné,  mais  une  construction  qui 
détermine, lors de la dissociation du terme I, une règle qui permet d'en hiérarchiser les multiples 
aspects, en qualifiant d'illusoires, d'erronés, d'apparents, dans le sens disqualifiant de ce mot, ceux 
qui ne sont pas conformes à cette règle que fournit le réel. Par rapport au terme I, le terme II sera, 
à la fois, normatif et explicatif. Lors de la dissociation, il permettra de valoriser ou de disqualifier 
tels aspects sous lesquels se présente le terme I : il permettra de distinguer, parmi les apparences, 
dont le statut est équivoque, celles qui ne sont qu'apparence, de celles qui représentent le réel (1). » 
\bigskip
(1) Ch. Perelman, Réflexions sur l'explication, Revue (le L'Institut de Sociologie, 1 , 939, n° 1, p. 39. 
\bigskip
P  557-558 :  « Ce  point  nous  paraît  essentiel,  à  cause  de  son  importance  dans  l'argumentation  : 
alors que le statut primitif de ce qui s'offre comme objet de départ de la dissociation est indécis et 
indéterminé,  la  dissociation  en  termes  I  et  II,  valorisera  les  aspects  conformes  au  terme  II,  et 
dévalorisera  les  aspects  qui  s'y  opposent  :  le  terme  I,  l'apparence,  dans  le  sens  étroit  de  ce  mot, 
n'est qu'illusion et erreur. » 
\bigskip
P  558 :  « En  fait,  le  terme  II  lie  s'accompagne  pas  toujours  d'un  critère  précis  permettant  de 
séparer les aspects du terme I : la norme qu'il fournit peut n'être que potentielle, et son principal 
effet  sera  de  hiérarchiser  les  termes  qui  résulteront  de  la  dissociation.  Quand,  pour  résoudre  les 
antinomies cosmologiques, Kant dissocie la réalité, en distinguant les phénomènes et les choses en 
soi,  le  terme  II  qu'il  construit  de  cette  façon  n'est  pas  connu,  mais  il  n'en  reste  pas  moins  que  le 
monde  phénoménal,  conditionné  par  notre  pouvoir  de  connaître,  est  dévalué  par  rapport  à  la 
réalité  des  choses  en  soi.  Le  terme  II  bénéficie  de  son  unicité,  de  sa  cohérence,  opposés  à  la 
multiplicité  et  à  l'incompatibilité  des  aspects  du  terme  I,  dont  certains  seront  disqualifiés,  et 
appelés à disparaître en fin de compte. 
\bigskip
C'est  ainsi  que,  dans  le  terme  II,  réalité  et  valeur,  sont  étroitement  liées  ;  cela  se  marque 
particulièrement  dans  toutes  les  constructions  des  métaphysiciens.  Ce  qui  permet  au  philosophe 
américain Ducasse d'écrire : 
\bigskip
Les adjectifs réel  et irréel, quand ils  sont utilisés dans l'énoncé d'une position métaphysique, ne 
désignent aucun caractère que certaines choses posséderaient, indépendamment de l'intérêt que 
les homines ont pour elles, mais sont au contraire des adjectifs d'appréciation humaine (1). 
\bigskip
Cette préférence accordée à ce qui est réel ne se manifeste pas seulement au cours de discussions 
philosophiques, mais s ' exprime dans la pensée de tous les jours, dans les circonstances les plus 
variées.  Que  réalité  et  valeur  se  conditionnent  l'une  l'autre,  l'usage  habituel  de  notre  langage  en 
témoigne, et quand Pitt dit : 
\bigskip
\bigskip
\bigskip
297 
\bigskip
  réalité 
\bigskip
\bigskip
J'essayerai, Monsieur, de limiter ce que j'ai à dire au point réel qui est à  l'examen (2)... » 
\bigskip
(1) C. J.Philosophy as a science, p. 148. 
(2) W. Pitt, Orations on the French war, p. 90 (May 27, 1793).  
\bigskip
P 559 : “cela Signifie, avant tout, qu'il se limitera à ce qui lui semble important. La recherche de ce 
qui est réel, par les métaphysicien entre constitue que l'expression systématisée de cette liaison - s, 
ne réalité et valeur, caractéristique du terme II. Pour cette raison, d'ailleurs, nous qualifierons de « 
couples philosophiques », ceux, à l'instar du couple apparence qui, - , résultent d'une dissociation  
\bigskip
des notions. » 
\bigskip
P  559 :  « L'opposition  entre  l'apparence  et  la  réalité  '  si  elle  fournit  le  prototype  d'un  couple 
philosophique, ne permet pourtant pas de réserver tous les avantages à la réalité, au détriment de 
l'apparence.  En  effet,  alors  que  celle-ci  est  donnée,  la  réalité  est  construite,  sa  connaissance  est 
indirecte,  parfois  impossible,  rarement  communicable  de  façon  exhaustive  et  indiscutable.  Cette 
réalité a le grand tort, pour certains, d'être insaisissable. Et, nous dit Gracian : 
\bigskip
Ce qui ne se voit point, est comme s'il n'étoit point (1). 
\bigskip
L'on pourrait répéter, à son propos, ce que Pascal dit de la véritable justice : 
\bigskip
Certainement,  s'il  la  connaissait,  il  n'aurait  pas  établi  cette  maxime,  la  plus  générale  de  toutes 
celles qui sont parmi les hommes, que chacun suive les mœurs de son pays ; l'éclat de la véritable 
équité aurait assujetti tous les peuples, et les législateurs n'auraient pas pris pour modèle, au lieu 
de cette justice constante, les fantaisies et les caprices des Perses et Allemands (2). » 
\bigskip
(1) B. Gracian, L'homme de cour, p. 158.  
(2) Pascal, Bibl. de la Pléiade, Pensées, 230 (69), p. 886 (294 éd. Brunschvicg). 
\bigskip
P  559-560 :  « Quand  le  critère  que  la  notion  de  réalité  fournit,  la  norme  qu'elle  détermine,  n'est 
pas en fait contesté, ou que la distinction qu'elle introduit est trop indéterminée pour donner prise 
à  controverse,  la  réalité  est,  sans  doute  aucun,  valorisée  par  rapport  à  l'apparence.  Mais  la 
dissociation même entre apparence et réalité sera rejetée par certaines philosophies qui constatent 
que des conceptions du réel s'opposent les unes aux autres, et refusent toute raison de choisir entre 
elles. Ces philosophies, dites anti-métaphysiques, positivistes, pragmatiques, phénoménologiques 
ou existentialistes, affirment que la seule réalité est celle des apparences. » 
\bigskip
P 560 : « Rien de plus net à cet égard que l'attitude de Sartre : 
\bigskip
La  pensée  moderne  a  réalisé  un  progrès  considérable  en  réduisant  l'existant  à  la  série  des 
apparitions qui le manifestent... Les apparitions qui manifestent l'existant ne sont ni intérieures 
ni  extérieures  :  elles  se  valent  toutes,  elles  renvoient  toutes  à  d'autres  apparitions  et  aucune 
d'elles n'est privilégiée. ... le dualisme de l'être et du paraître ne saurait plus trouver droit de cité 
en  philosophie.  L'apparence  renvoie  à  la  série  totale  des  apparences  et  non  à  un  réel  caché  qui 
aurait drainé pour lui tout l'être de l'existant. 
\bigskip
Tant qu'on a pu croire aux réalités nouménales, on a présenté l'apparence comme un négatif pur... 
Mais  si  nous  nous  sommes  une  fois  dépris  de  ce  que  Nietzsche  appelait  «  l'illusion  des  arrière-
mondes » et si nous ne croyons plus à l'être-de-derrière-l'apparition, celle-ci devient, au contraire, 
pleine  positivité,  son  essence  est  un  «paraître  »  qui  ne  s'oppose  plus  à  l'être,  mais  qui  en  est  la 
mesure,  au  contraire.  Car  l'être  d'un  existant,  c'est  précisément  ce  qu'il  paraît.  ...L'essence  d'un 
\bigskip
\bigskip
\bigskip
298 
\bigskip
existant.  .  c'est  la  loi  man  if  este  qui  pré  à  la  succession  de  ses  apparitions,  c'est  la  raison  de  la 
série... l'essence, comme raison de la série n'est que le lien des apparitions, c'est-à-dire elle-même 
une  apparition.  ...  Ainsi  l'être  phénoménal  se  manifeste,  il  manifeste  son  essence  aussi  bien  que 
son existence et il n'est rien que la série bien liée de ces manifestations (1). » 
\bigskip
(1) Sartre, L'être et le néant, pp. 11-12-13. 
\bigskip
\bigskip
\bigskip
P 560-561 : « Mais en s'opposant à la dissociation apparence, on laisse entier le problème posé par 
\bigskip
        réalité 
\bigskip
l'incompatibilité  des  apparences.  Le  critère,  du  choix  à  effectuer  parmi  elles,  sera  fourni  par  un 
autre couple, conçu sur le même modèle que le couple « apparence-réalité -», ce qui correspond à 
une distinction de nature ; ou bien, à son défaut, on établira une distinction purement quantitative, 
en accordant une préférence au tout sur la partie, à l'infini sur le fini, àce qui présente à un degré 
plus  élevé  la  propriété  servant  de  critère.  Nous  trouvons  dans  ce  texte  de  Merleau-Ponty  une 
expression caractéristique de cette façon de procéder : 
\bigskip
je traverse les apparences, j'arrive à la couleur on à la forme réelle, lorsque mon expérience est à 
son plus haut degré de netteté... 
\bigskip
J'ai  des  objets  visuels  parce  que  j'ai  un  champ  visuel  où  la  richesse  et.la  netteté  sont  en  raison 
inverse l'une de l'autre et que ces deux exigences, dont chacune prise à part irait à l'infini, une fois 
réunies, déterminent dans le processus perceptif un certain point de maturité et un maximum. De 
la  même  manière,  j'appelle  expérience  de  la  chose  ou  de  la  réalité  -  non  plus  seulement  d'une 
réalité-pour-la-vue ou pour-le-toucher, mais d'une réalité absolue -ma pleine coexistence avec le 
phénomène, le moment où il serait sous tous les rapports àson maximum d'articulation, et les « 
données des différents sens »sont orientées vers ce pôle unique comme mes visées au microscope 
oscillent autour d'une visée privilégiée (1). 
\bigskip
(1) Merleau-Ponty Phénoménologie de la perception, pp. 367-8. 
\bigskip
§ 91. LES COUPLES PHILOSOPHIQUES ET LEUR JUSTIFICATION 
\bigskip
P 561 : « Le couple apparence a été choisi comme prototype de dissociation notionnelle. Si le  
\bigskip
réalité 
\bigskip
processus peut être schématisé, le résultat n'en est pas, pour autant, purement formel ou verbal : la 
dissociation exprime une vision du monde, établit des hiérarchies, dont elle s'efforce de fournir les 
critères.  Cela  ne  va  pas  sans  le  concours  d'autres  secteurs  de  la  pensée.  Il  arrive  bien  souvent 
qu'une discussion concernant le terme II doive s'appuyer sur un autre couple, do-nt les termes I et 
II ne soient pas, en l'occurrence, controversés. » 
\bigskip
Tous  ces  couples  constituent  l'objet  propre  de  la  recherche  philosophique,  En  voici  quelques 
exemples, choisis parmi ceux qui se présentent le plus fréquemment dans la pensée occidentale :  
moyen,   conséquence,      acte,         accident,  occasion, relatif,    subjectif, multiplicité, normal, 
   fin        fait ou principe   personne    essence     cause     absolu     objectif         unité         norme 
\bigskip
individuel, particulier, théorie,    langage,    lettre. 
universel,    général      pratique    pensée      esprit 
\bigskip
P 562 : « Le fait que nous soyons à même d'indiquer un grand nombre de couples, en assignant à 
chacun  de  leurs  termes  une  place  déterminée,  sans  devoir,  pour  y  parvenir,  les  insérer  dans  une 
pensée systématisée - ce qui ne serait d'ailleurs pas possible pour tous les couples, car certains sont 
formés  de  façon  diamétralement  opposée  et  appartiennent  à  des  pensées  philosophiques  de 
\bigskip
\bigskip
\bigskip
299 
\bigskip
   idée        entendement     individuel   concret     nécessité      immutabilité 
\bigskip
tendances  différentes  -  est  révélateur  de  l'influence  que  les  élaborations  philosophiques  ont 
exercée  sur  la  pensée  commune,  en  la  lestant  d'une  série  de  couples,  résidus  d'une  tradition 
culturelle dominante. 
\bigskip
Toute pensée systématisée s'efforce de mettre en rapport les uns avec les autres des éléments qui, 
dans  une  pensée  non  élaborée,  constituent  autant  de  couples  isolés.  Cette  mise  en  rapport  des 
couples  est  utile  pour  éviter  des  prises  de  position  qui  aboutissent  àqualifier  les  mêmes 
phénomènes  à  l'aide  de  couples  incompatibles.  Elle  est  indispensable  quand,  au  lieu  de  se 
contenter de reprendre des dissociations admises dans un milieu culturel, le penseur original crée 
de  nouvelles  dissociations  ou  se  refuse  à  admettre  certaines  dissociations  de  ses  prédécesseurs. 
Suite  à  ce  bouleversement  et  pour  montrer  les  conséquences  de  celui-ci  en  ce  qui  concerne  les 
autres couples, le philosophe établira un système qui aboutira essentiellement à la mise en rapport 
les uns avec les autres de couples philosophiques. » 
\bigskip
P  562-563 :  « C'est  ainsi  que  dans  le  Phèdre,  la  pensée  philosophique  de  Platon  peut  s'exprimer 
par les couples : apparence, opinion, connaissance sensible,      corps,    devenir,            pluralité, 
                               réalité          science  connaissance rationnelle    âme     immutabilité       unité 
\bigskip
humain (1). » 
  divin 
\bigskip
P 563 : « L'Éthique de Spinoza aboutit aux couples : 
\bigskip
connaissance inadéquate, image,    imagination,      universel,  abstrait,  contingence, changement 
connaissance adéquate 
\bigskip
corps,  passion,  esclavage,  durée,      joie,              superstition. 
raison   action       liberté      éternité   béatitude       religion 
\bigskip
Dans ce passage de Lefebvre, on trouvera la liaison des couples :  
abstrait, métaphysique,   entendement, immobilité,   forme,      
concret      dialectique           raison          mouvement  contenu 
\bigskip
caractéristiques de la pensée marxiste :  
\bigskip
Le pouvoir de détacher du monde certains objets - par des lignes de clivage idéales ou réelles - et 
d'immobiliser, de déterminer ces objets définit, nous le savons déjà, l'intelligence ou entendement. 
Eue a e pouvoir d'abstraire, de réduire à sa plus simple expression le contenu concret. 
\bigskip
Si l'on maintient isolé par la pensée un tel objet, il s'immobilise dans la pensée, il devient de l' « 
abstraction métaphysique ». Il perd sa vérité ; en ce sens, cet objet n'est plus rien. Mais si on le 
considère  comme  un  objet  momentané,  valant  non  par  sa  forme  et  ses  contours  isolants,  mais 
par  son  contenu  objectif  ;  s'il  est  considéré  non  comme  un  résultat  définitif,  mais  comme  un 
moyen ou une étape intermédiaire pour pénétrer dans le réel ; si l'intelligence se complète par la 
raison,  alors  l'abstraction  se  légitime.  Elle  est  une  étape  vers  le  concret  retrouvé  analysé  et 
compris. Elle est concrète en un sens... La vérité de l'abstrait se trouve ainsi dans le concret. Pour 
la  raison  dialectique,  le  vrai  c'est  le  concret;  et  l'abstrait  ne  peut  être  qu'un  degré  dans  la 
pénétration  de  ce  concret  -  un  moment  du  mouvement,  une  étape,  un  moyen  pour  saisir, 
analyser,  déterminer  le  concret.  Le  vrai,  c'est  le  rationnel,  et  c'est  le  réel,  le  concret.  Ainsi,  la 
quantité  et  l'espace  géométrique  ne  sont  vrais  que  si  l'on  maintient  rationnellement  leurs 
rapports avec la qualité, avec le peuplement de l'espace en objets réels (2). » 
\bigskip
(1) Platon, Phèdre, 247 e, 248 b. 
\bigskip
\bigskip
\bigskip
300 
\bigskip
(2)  H.  Lefebvre,  A  la  lumière  du  matérialisme  dialectique.  1  :  Logique  formelle,  logique 
dialectique, pp. 83-84 (italiques de l'auteur). 
\bigskip
P 564 : « Aux couples philosophiques, résultant d'une dissociation, on pourrait opposer, d'une part 
les couples antithétiques, où le deuxième terme est l'inverse du premier, tels  haut-bas, bien-mal, 
juste-injuste, d'autre part des couples classificatoires qui, à première vue, sont dépourvus de toute 
intention  argumentative,  et  semblent  uniquement  destinés  à  subdiviser  un  ensemble  en  parties 
distinctes (le passé en époques, une étendue en régions, un genre en espèces). 
\bigskip
Il arrive certes bien souvent que ces couples se présentent comme des données, que l'on ne discute 
pas, comme des instruments permettant de structurer le discours d'une façon qui paraît objective. 
Mais,  dans  une  pensée  systématique,  les  couples  sont  mis  en  rapport  les  uns  avec  les  autres  et 
s'influencent  mutuellement,  des  termes  II  de  couples  philosophiques  seront  normalement 
rapprochés,  s'il  y  a  moyen,  de  ce  qui  dans  le  couple  antithétique  a  valeur  positive,  des  termes  I 
seront  rapprochés  de  ce  qui  a  valeur  négative,  d'où  tendance  à  la  transformation  du  couple 
antithétique  en  couple  philosophique.  D'autre  part,  dans  l'élaboration  des  couples  qui  semblent 
classificatoires, les dissociations de nature philosophique jouent fréquemment un rôle essentiel. » 
\bigskip
P 564-565 : « Dans des pages pénétrantes L. Febvre analyse la création par Michelet du concept de 
Renaissance (1). Michelet éprouvait la nécessité de discerner cette période située avant les temps 
modernes, mais hésitait entre deux conceptions, la Renaissance comme résurrection du moyen âge 
originel  ou  comme  remplaçant  le  moyen  âge.  Au  moment  où  il  opte  définitivement  pour  la 
deuxième  solution,  il  biffe  des  pages  admirables,  rédigées  en  fonction  de  la  première.  Dans  le 
premier parti, la réalité nouvelle eût déterminé un moyen âge plus pur, plus vrai, que le moyen âge 
antérieur  qui  n'en  était que  l'apparence.  Dans  le  deuxième  parti,  l'époque antérieure  constitue  le 
moyen âge authentique, qui n'est plus apparence de moyen âge, mais apparence de civilisation : le 
couple  
\bigskip
apparence     
réalité 
\bigskip
\bigskip
est  appliqué  à  une  autre  notion.  Mais,  une  fois  que  les  notions  ont  pris  une  consistance, 
indépendante de leurs origines, elles paraîtront purement classificatoires et pourront jouer un rôle, 
même dans la construction de l'historien qui voit dans les siècles qui précèdent la Renaissance un 
sommet de la civilisation. » 
\bigskip
(1) L. Febvre, Comment Jules Michelet inventa la Renaissance dans Studi in onore di Gino 
Luzzatto, t. III, pp. 1.11 
\bigskip
P 565 : « Ainsi, les notions qui résultent d'une dissociation peuvent, lancées dans la communauté 
linguistique,  paraître  indépendantes.  Celles  qui  ont  servi  de  base  à  notre  étude  des  liaisons  se 
situent toutes, indistinctement, dans des couples philosophiques. Nous plaçant à ce point  de vue, 
nous  avons,  d'ailleurs,  entamé  une  étude  systématique  qui  est  suffisamment  avancée  pour  que 
nous  puissions  affirmer  qu'elle  est  réalisable.  Il  ne  s'agit  point  là  de  construire  une  philosophie 
particulière,  mais  uniquement  d'observer  ce  qui  se  fait  dans  les  diverses  systématisations  de  la 
pensée, et dans les diverses philosophies, quelle qu'en soit la tendance. Les couples moyen,  
\bigskip
fin 
\bigskip
acte,          individu,   acte,       symbole,   particulier 
personne  groupe     essence  chose          général  
\bigskip
leurs variantes et connexions, nous fournissent les termes des liaisons les plus usuelles, bases des 
solidarités argumentatives. Le même couple de notions se présente donc, tantôt comme le résultat 
d'une dissociation, tantôt comme deux notions indépendantes entre lesquelles existent des liaisons 
caractéristiques,  une  interaction,  mais  aussi,  nous  l'avons  vu  dans  nos  chapitres  antérieurs,  des 
primautés de valeur à laquelle leur place comme terme I ou II d'un couple philosophique n'est pas 
\bigskip
\bigskip
\bigskip
301 
\bigskip
étrangère.  Chaque  fois  que  l'on  met  l'accent  sur  celle-ci,  on  marque  que  le  passage  d'un  terme 
àl'autre ne peut se faire sans restrictions. Si l'acte permet de juger la personne, la fin le moyen, et 
réciproquement, l'évocation du couple philosophique rappelle qu'il ne faut pas les confondre. » 
\bigskip
P 566 : « La connexion entre couples n'a, très souvent, pas besoin d'être explicite. Elle s'établira, et 
se justifiera s'il y a lieu, par les moyens les plus divers : liaison directe fondée sur la structure du 
réel  entre  terme  I  d'un  couple  et  terme  1  d'un  autre,  entre  terme  II  d'un  couple  et  terme  Il  d'un 
autre;  couple  considéré  comme  cas  particulier  d'un  autre;  arguments  d'allure  quasi  logique  et 
notamment affirmation de l'identité de couples ; surtout enfin rapports analogiques entre couples. 
\bigskip
A  ces  connexions  que  nous  qualifierons  d'horizontales,  qui  réalisent  des  chaînes  de  couples,  se 
superposent  des  rapports  d'autre  espèce.  En  eff  et,  dans  l'argumentation,  ce  que  l'on  qualifie 
d'apparence est généralement ce qui, pour quelqu'un d'autre, était le réel, ou était confondu avec le 
réel, sans quoi on ne lui donnerait pas ce nouveau statut. Suivant l'étendue, la nature et le rôle de 
l'auditoire  qui  était  censé  commettre  la  confusion,  l'argumentation  se  développera  sur  des  plans 
différents. Tantôt elle paraît concerner l'objet en cause, tantôt l'idée que certains se faisaient de cet 
objet,  tantôt  le  statut  que  certains  lui  accordaient  ou  étaient  censés  lui  accorder  dans  un  but 
argumentatif.  Ces  différents  plans  s'enchevêtrent  en  s'étayant  l'un  l'autre.  Donnons-en  cet 
exemple. Un des procédés utilisés dans les conflits idéologiques serait, a-t-on observé : 
\bigskip
... de traiter l'affirmation d'un idéal comme une description de faits et d'interpréter des rapports 
sur des situations réelles comme si elles étaient l'idéal poursuivi (1). » 
\bigskip
(1) B. Mckeon, Democracy in a world of tensions, p. 524. 
\bigskip
P 566-567 : « On découvre ici l'emploi, par chacun, d'un couple normal en ce qui le concerne, et 
\bigskip
        norme  
\bigskip
\bigskip
d'un couple norme en ce qui concerne l'adversaire. Mais seul celui qui dénonce ce procédé 
                      normal  
\bigskip
distingue entre fait et idéal, opère une dissociation. L'auteur, lui, a résolu par elle une difficulté, a 
effectué un choix; il se meut cependant, soi-disant, dans l'indifférencié. Et nous pourrions encore, 
à son propos, créer le couple impartialité apparente. » 
                                                      impartialité réelle 
\bigskip
P  567 :  « Les  couples  établis  sur  un  plan  peuvent  donc  donner  naissance  à  une  série  de  couples 
établis sur d'autres plans. C'est ainsi que, pour éviter une difficulté sur le plan de l'objet, on créera 
un  couple  sur  le  plan  de  l'opinion.  Lorsque  le  leader  libéral  belge  Paul  Janson  demande 
l'interdiction du travail des enfants dans les mines, il n'hésite pas àrecourir à la notion d'erreur, à 
proclamer  que  si  certains  intérêts  privés,  contrairement  à  la  doctrine  libérale,  ne  coïncident  pas 
avec l'intérêt publie, ce ne peuvent être que des intérêts apparents : 
\bigskip
... il est incontestable que lorsque les intérêts privés se trompent, lorsqu'ils s'égarent, lorsqu'ils ae 
mettent  en  contradiction  avec  l'intérêt  social  et  avec  l'intérêt  publie,  notre  devoir  est  de  le  leur 
dire et de les faire rentrer dans les justes bornes dont ils ne doivent point sortir (1). 
\bigskip
D'une  manière  générale,  la  notion  d'erreur  sert  à  affirmer  qu'il  y  a  une  règle,  qu'elle  subsiste 
malgré des observations qui semblent la démentir, et que ce qui se situe en dehors d'elle ne doit 
pas être pris en considération, ou ne doit l'être qu'avec réserve. » 
\bigskip
(1) P. Janson, Discours parlementaires, vol. 1, pp. 35-36 (séance de la Chambre des Représentants 
du 13 février 1878). 
\bigskip
\bigskip
\bigskip
302 
\bigskip
 
P  567-568 :  « Mais  l'effort  pour  faire  admettre  une  dissociation  ira  souvent  au  delà.  On  tentera 
d'expliquer  pourquoi  il  y  a  discordance  entre  termes  1  et  11,  pourquoi  notamment  à  l'unicité  du 
terme Il correspond la multiplicité, la partialité des aspects du I. La diversité des points de vue sur 
l'objet, ou les métamorphoses du terme I, seront appelés, par exemple, à justifier la multiplicité des 
apparences.  Pour  expliquer  le  surgissement  du  terme  I,  on  le  situera  dans  un  cadre  qui  le  rende 
normal.  On  fera  notamment  intervenir  le  sujet,  sa  passion,  son  impuissance,  son  ignorance,  son 
état- a péché. Gide entend ces paroles dans la bouche du Sauveur :  
\bigskip
Ne t'étonne pas d'être triste; et triste à cause de Moi. La félicité que je te propose exclut à jamais 
ce que tu prenais pour du bonheur. » 
\bigskip
P 568 : « Ce bonheur, au sens ancien, devient le terme I d'un couple bonheur : 
\bigskip
     joie 
\bigskip
\bigskip
joie, joie... je sais que le secret de votre Évangile, Seigneur, tient tout dans ce mot divin (1). 
\bigskip
Souvent,  l'explication  de  l'apparence  sera  fournie  par  l'intervention  d'un  facteur  particulier  :  les 
prestiges de l'Intellect expliquent pour Schopenhauer notre illusion de liberté (2) ; le surmoi est à 
l'origine,  pour  le  psychanalyste,  de  la  pseudo-morale  (3).  Parfois,  le  terme  1  sera  décrit  comme 
remplissant une fonction naturelle :  
\bigskip
Si  donc  l'intelligence  devait  être  retenue,  au  début,  sur une  pente  dangereuse  pour  l'individu  et 
pour  la  société,  ce  ne  pouvait  être  que  par  des  constatations  apparentes,  par  des  fantômes  de 
faits: à défaut d'expérience réelle, c'est une contrefaçon de l'expérience qu'il fallait susciter (4). 
\bigskip
Ces explications n'ont pas seulement pour but de faire admettre des couples. La dissociation fable, 
\bigskip
        réalité 
\bigskip
notamment,  n'est  pas  mise  en  cause.  Mais  l'explication  concourt  à  donner  le  critère  du  II  et  à 
insérer  le  couple  dans  un  ensemble  de  pensée.  Elle  complète  le  travail  accompli,  par 
l'établissement de connexions entre couples. » 
\bigskip
(1) A. Gide, Journal, Bibl. de la Pléiade, p. 600 (Numquid et tu... ). 
(2) Schopenhauer, éd. Brockhaus, vol. 6 : Parerga und Paralipomena, Zweiter bond, Zur Ethik, 
118, pp. 249-50. 
(3) Ch. Odier, Les deux sources, consciente et inconsciente, de la vie morale, pp. 42, 58, 59. 
(4) Bergson, Les deux sources de la morale et de la religion, p. 118.  
\bigskip
\bigskip
\bigskip
\bigskip
\bigskip
\bigskip
\bigskip
\bigskip
\bigskip
\bigskip
\bigskip
§  92.  LE  ROLE  DES  COUPLES  PHILOSOPHIQUES  ET  LEURS 
\bigskip
TRANSFORMATIONS 
P  569 :  « Si  l'emploi  de  certaines  dissociations  semble  ne  pas  apporter  grand'chose  de  neuf, 
puisque  l'on  fait  état  de  notions  très  anciennement  élaborées,  il  n'en  introduit  pas  moins  des 
remaniements  de  ces  notions,  par  leur  application  à  un  domaine  nouveau,  par  les  critères 
nouveaux adoptés pour le terme II, par la mise en rapport avec de nouveaux couples. 
\bigskip
Ainsi, un couple aussi banal que le couple apparence sera indirectement, l'objet d’un  
\bigskip
\bigskip
remodèlement constant. Une dissociation telle religion positive retentira sur le couple apparence, 
\bigskip
   réalité 
\bigskip
\bigskip
\bigskip
\bigskip
auquel elle aura pu être associée. 
\bigskip
religion-naturelle  
\bigskip
\bigskip
\bigskip
\bigskip
\bigskip
\bigskip
\bigskip
réalité 
\bigskip
\bigskip
\bigskip
303 
\bigskip
 
P  569-570 :  « L'effort  argumentatif  consistera  -  tantôt  à  tirer  parti  de  dissociations  déjà  admises 
par l’auditoire, tantôt à introduire des dissociations créées ad hoc, tantôt à présenter à un auditoire 
des dissociations admises par d'autres auditoires, tantôt à rappeler une dissociation que l'auditoire 
est censé avoir oubliée. Quant à l'opposition à une dissociation, elle portera sur les caractéristiques 
de ses termes I ou II, ou sur le principe même de la dissociation. Dans ce cas, on soutiendra qu'il 
fallait s'en tenir à une notion globale. Mais, il est très malaisé de renoncer à des termes, dont, rien 
qu'en  s'y  référant,  fût-ce  pour  les  combattre,  on  rappelle  l'existence.  La  pensée  contemporaine 
s'efforce,  dans  beaucoup  de  domaines,  à  abolir  des  couples.  C'est  au  prix  d'un  grand  effort,  car 
l'auditeur  ne  se  sentira  satisfait  que  s'il  petit,  dans  sa  pensée,  donner  une  place  aux  notions 
anciennes. Souvent on s'appuiera, pour le rejet, sur un autre couple. Le plus aisé sera de prétendre 
que la dissociation était illusoire, en se fondant sur un couple verbal ; une autre technique 
\bigskip
                                                        réel 
\bigskip
consistera à montrer que le problème, que la dissociation était destinée à résoudre, était factice, en 
se basant sur un couple factice ; ou bien même que le problème se posera de nouveau exactement 
                                       authentique 
\bigskip
dans  les  mêmes  conditions  sans  que  l'accord  provisoire  sur  la  dissociation  ait  apporté  aucun 
bénéfice dans la cohérence de la pensée (1). » 
\bigskip
(1) Cf., par exemple, G. RYLE, The concept of mind, pp. 25 et suiv. 
\bigskip
P 570 : « Mais l'effort argumentatif vise très souvent, non au rejet de couples établis, mais à leur 
renversement.  Celui-ci  portera  sur  l'un  ou  plusieurs  d'entre  eux  seulement  ;  car  l'intérêt  de  ces 
renversements vient précisément de ce qu'ils s'insèrent dans un ensemble admis par ailleurs. » 
\bigskip
Il  va  sans  dire  qu'un  renversement  de  couple  n'est  jamais  complet,  en  ce  sens  qu'une  notion  qui 
devient terme I n'est plus ce qu'elle était lorsque nous la connaissions comme terme II, un terme 
ne pouvant se concevoir que dans son rapport avec l'autre terme du couple. Presque toujours, un 
changement de terminologie indiquera la dévaluation qui s'attache à la notion devenue terme I, et 
la valorisation qui s'attache à celle qui devient terme II ; il montrera que le renversement s'insère 
dans une autre vision de la situation particulière ou du monde. En face du couple interprétation  
\bigskip
existe le couple lettre  (ce renversement est postulé couramment lorsqu'un juriste soutient une  
\bigskip
une interprétation déterminée : l'interprétation qui réussit devient terme II). En face du couple 
théorie nous aurons le couple phénomène. » 
fait 
\bigskip
P 570-571 : « L'utilisation inhabituelle de certaines notions comme terme II, même incidemment, 
marque l'originalité de la pensée. Nous connaissons fort bien le couple normal , lié au couple   
\bigskip
                       principe 
\bigskip
                     esprit 
\bigskip
lettre 
\bigskip
\bigskip
\bigskip
\bigskip
\bigskip
\bigskip
\bigskip
\bigskip
\bigskip
\bigskip
\bigskip
\bigskip
\bigskip
\bigskip
\bigskip
\bigskip
\bigskip
        norme 
\bigskip
exemple (2) ; or, dans la pièce de Salacrou, Un homme comme les autres, la réalité du héros,  
archétype 
tant cherchée, ce qui définit sa personne et lui confère, finalement, sa valeur humaine, c'est non la 
norme, mais le normal (1). » 
\bigskip
(2) Cf. KANT, Critique de la raison pure, p. 305. 
(1) A. Salacrou, Un homme comme les autres, cf. acte I, p. 242; acte II, p. 277, p. 298; acte III, pp. 
310-311, p. 325. 
\bigskip
P  571 :  « Le  renversement,  sans  changement  de  terme  aucun,  a  souvent  une  allure  provocante, 
même s'il est justifié par une distinction de domaines : 
\bigskip
\bigskip
\bigskip
304 
\bigskip
 
…le mot qui, dans l'expression en prose est le spectre d'une pensée, devient en vers la substance 
même de l'expression, où, par irisation, la pensée apparaît (2). 
\bigskip
Lorsque  le  renversement  se  marque  par  la  place  des  mots,  il  peut  prendre  allure  de  figure  : 
certaines antithèses, notamment beaucoup de commutations que certains appellent réversions (3) 
consistent à considérer un même phénomène, tantôt comme terme I, tantôt comme terme II. 
\bigskip
Il faut manger pour vivre. non pas vivre pour manger (4). 
\bigskip
Nous ne devons pas juger des règles et des devoirs par les moeurs et par les usages, mais nous 
devons juger des usages et des moeurs par les devoirs et par les règles (5). 
\bigskip
Selon  la  Rhétorique  à  Herennius,  ces  figures  «  produisent  l'impression  que  la  seconde  partie  est 
déduite de la première, quoiqu'elle la contredise (6) ». En fait il s'agit ici de mettre le phénomène 
en connexion avec des couples moyen (7), normal, et de choisir sa place dans le couple. 
\bigskip
fin  
\bigskip
    norme 
\bigskip
Certaines de ces commutations pourraient s'analyser comme une inversion de métaphore : 
\bigskip
Le poème doit être une peinture parlante, et une peinture un poème muet (8). » 
\bigskip
(2) Joël  Bousquet dans Aragon, Les yeux d'Elsa, p. 146.  
(3) Baron, De la Rhétorique, p. 360. 
(4) Cité par la Rhétorique à Herennius liv. IV, ~ 39. 
(5) Cité par Baron, De la Rhétorique, p. 360 (Bourdaloue).  
(6) Rhétorique à Herennius, ibid. 
(7) Cf. § 64 : Les fins et les movens. 
(8) Rhétorique à Herennius, ibid. ; cf. Vico, Instituzioni oratorie, p. 150. 
\bigskip
P 572 : « On envisage la place du même objet tantôt comme phore, tantôt comme thème, dans un 
couple phore. Car, notons-le, il y a moyen thème d'interpréter toute analogie comme une 
            thème 
dissociation dont le serait le terme II, les différents phores possibles étant terme I. 
\bigskip
Il est des notions qui, de par nos habitudes mentales, prennent malaisément la place de terme 1 : 
notamment la notion de réel. 
\bigskip
Cependant elle s'y insère dans le couple réel ; et les notions de « fait » ou de « donné », identifiées 
\bigskip
communément à celle de « réel » se rencontrent comme terme I, la première dans le couple fait :  
\bigskip
       droit 
Pour les hommes du parti catholique, la liberté de conscience, la liberté des cultes, la liberté de la 
presse, la liberté d'association, ne sont plus des droits naturels, inaliénables, imprescriptibles; ce 
sont des faits, de simples faits, que l'Eglise romaine tolère, parce qu'elle ne peut les empêcher (1)... 
\bigskip
la seconde dans le couple donné :  
\bigskip
         idéal 
\bigskip
\bigskip
\bigskip
\bigskip
\bigskip
\bigskip
\bigskip
\bigskip
\bigskip
\bigskip
\bigskip
\bigskip
\bigskip
       explication 
\bigskip
\bigskip
Le  donné,  dont  personne  ne  doute,  est  considéré  comme  apparence,  alors  que  ce  qui  sert 
d'explication,  quoique  rarement  aussi  certain,  est  traité  comme  caractéristique  de  la  réalité 
véritable (2). 
\bigskip
\bigskip
\bigskip
\bigskip
305 
\bigskip
(1)  P.  Janson,  Discours  parlementaires,  I,  p.  53  (séance  de  la  Chambre  des  Représentants  du  17 
mai 1878). 
(2) Ch. Perelman, Réflexions sur l'explication, Revue de l'Institut de Sociologie, 1939, no 1, p. 59. 
\bigskip
P 572-573 : « L'intérêt des renversements tient en grande partie, surtout en philosophie, à ce que 
les  notions  gardent,  par  leur  tradition,  leurs  connexions,  quelque  chose  de  ce  qu'elles  étaient 
lorsqu'elles avaient leur place ancienne dans les couples : 
\bigskip
Ce  que  vous  appelez  les  formes  vides  et  l'extérieur  des  choses,  cela  nie  paraît  les  choses  elles-
mêmes. Elles ne sont ni vides, ni incomplètes, sauf si l'on admet avec vous que la matière est une 
partie  essentielle  de  toutes  les  choses  corporelles.  Nous  nous  accordons  donc  tous  deux  sur  ce 
point  que  nous  percevons  seulement  des  formes  sensibles  :  mais  nous  différons  sur  cet  autre 
-vous soutenez que ce sont des apparences vaines et pour moi ce sont des êtres réels (1). » 
\bigskip
(1) Berkeley, Les trois dialogues entre Hylas et Philonous, 3e dial., p. 175.  
\bigskip
P 573 : « Le terme I ancien se transforme en terme II ; mais que de souvenirs, et notamment le mot 
« matière », pour indiquer la place primitive des notions. 
\bigskip
Le  maintien  de  certains  termes  traditionnels,  lors  du  renversement,  se  remarque  bien  dans  ce 
passage de Bergson : 
\bigskip
... la vie est une évolution. Nous concentrons une période de cette évolution en une vue stable que 
nous  appelons  une  forme...  il  n'y  a  pas  de  forme,  puisque  la  forme  est  de  l'immobile  et  que  la 
réalité est mouvement. Ce qui est réel, c'est le changement continuel de forme... Quand les images 
successives  ne  diffèrent  pas  trop  les  unes  des  autres,  nous  les  considérons  toutes  comme 
l'accroissement et la diminution d'une seule image moyenne, ou comme la déformation de cette 
image dans des sens différents. Et c'est à cette moyenne que nous pensons quand nous parlons de 
l'essence d'une chose, ou de la chose même (2). 
\bigskip
Bien que le point de vue de l'auteur soit tout nouveau, on relève ici l'intérêt philosophique qu'il y a 
pour lui à souligner le renversement du couple acte    en un couple essence , forme. L’originalité 
\bigskip
          essence  
\bigskip
\bigskip
\bigskip
devenir   devenir 
\bigskip
\bigskip
\bigskip
\bigskip
\bigskip
\bigskip
\bigskip
\bigskip
de la vision s'appuie sur un couple bien connu qu'elle s'attache à réfuter. L'essence cesse d'être le 
réel et devient l'apparence, en tant que théorie, forme; par contre l'acte devient terme II comme 
concret, vivant. » 
\bigskip
(2) Bergson, L'évolution créatrice, p. 327. 
\bigskip
P  573-574 :  « Une  dissociation  de  notion  met  à  l'abri  de  l'incompatibilité  qu'elle  résout.  Mais  de 
nouvelles difficultés surgissent à propos des termes ainsi établis. Aussi observe-t-on, tant dans la 
pensée pratique que philosophique, une tendance à de nouvelles subdivisions. Celles-ci porteront 
tantôt sur le I, tantôt sur le  II, et nous serons en présence de schèmes  du type ci-après, que l'on 
pourrait qualifier de dissociations en éventail : 
\bigskip
I 
II <I 
       II 
\bigskip
\bigskip
La  philosophie  de  Schopenhauer  en  fournit  d'excellents  exemples.  On  y  trouve  un  couple 
caractéristique objectivité, Objectivität :  
\bigskip
ou I < I 
             II 
\bigskip
                II 
\bigskip
\bigskip
\bigskip
306 
\bigskip
 
\bigskip
      volonté          wille 
\bigskip
\bigskip
\bigskip
Il n'y a que cette chose en soi... la volonté... elle ne connaît pas de multiplicité, est unique... La 
multiplicité des choses dans l'espace et le temps, qui dans leur ensemble sont son objectivité, ne la 
touche pas (1). 
\bigskip
Mais cette objectivité, cette représentation, se scinde elle-même en deux termes choses (au sens 
\bigskip
platonicien), tandis que le terme idées lui-même donne lieu à un couple concept   (bregriff 
\bigskip
lié au couple partiel :  
\bigskip
          intuition (Anschauung) 
\bigskip
 idées 
\bigskip
\bigskip
\bigskip
\bigskip
\bigskip
\bigskip
\bigskip
\bigskip
\bigskip
\bigskip
\bigskip
\bigskip
\bigskip
\bigskip
\bigskip
\bigskip
\bigskip
total  
\bigskip
\bigskip
Les idées sont en réalité quelque chose d'intuitif et par là... inépuisable... Le simple concept par 
contre est quelque chose de complètement déterminable, épuisable (2)... » 
\bigskip
(1) Schopenhauer, éd. Brockhaus, vol. 2 : Die Welt als Wille und Vorstellung, Erster Band, 5, 25, p. 
152. 
(2) Schopenhauer, éd. Brockhaus, vol. 3 : Die Welt als Wille und Vorstellung, Zweiter Band, Kap. 
34, p. 466. 
\bigskip
P  574-575 :  « Ces  approfondissements  successifs,  qui  permettent  de  ne  pas  sacrifier  les  résultats 
déjà obtenus, les accords acquis, les notions dont on dispose, se présentent dans tous les secteurs 
de la pensée. On peut se demander s'ils ne caractérisent pas surtout les domaines  où l'on hésite à 
opérer  un  renversement  de  couple.  Saint  Thomas  veut  s'attacher  à  la  lettre,  maintenir  donc  le 
couple interprétation. Mais, pour éviter les incompatibilités avec la science, il préférera 
                 texte 
\bigskip
l'interprétation qui « superficiellement moins littérale » (1) est rationnellement plus satisfaisante, 
introduisant ainsi un couple lettre apparente. » 
                                                     lettre réelle 
\bigskip
(1) E. Gilson, Le thomisme, p. 246. 
\bigskip
P  575 :  « Les  dissociations  nouvelles,  en  profondeur,  se  font  évidemment  selon  des  critères  qui 
peuvent être tout différents de celui de la dissociation sur laquelle elles s'insèrent. Et c'est là leur 
intérêt.  Elles  ont  cependant  pour  effet  de  rapprocher,  au  point  de  vue  valeur,  tous  les  termes  I. 
Mauriac introduit ainsi, par dissociation à l'intérieur d'un terme II, de sévères jugements : 
\bigskip
Il existe une fausse sainteté, - non au sens de la grossière imposture de Tartuffe - fausse en dépit 
ou peut-être même à cause de l'effort sincère, héroïque, de l'homme qui s'y applique (2). 
\bigskip
Les  quelques  indications  qui  précèdent  montrent  à  suffisance  le  rôle,  aussi  bien  dans  la  pensée 
philosophique  que  dans  la  pensée  journalière,  des  dissociations.  Nous  nous  bornerons  ici  à 
analyser quelques raisonnements qui mettent en rapport le couple apparence avec les couples 
\bigskip
conséquence. » 
      fait 
\bigskip
(2) F. Mauriac, Les maisons fugitives p. 19. 
\bigskip
P  575-576 :  « Quand  on  hésite  entre  des  conduites  à  adopter,  et  qu'il  s'agit  de  leur  accorder  un 
ordre de priorité, il est normal de les situer dans un ensemble que l'on systématise par rapport à 
\bigskip
  réalité 
\bigskip
\bigskip
\bigskip
307 
\bigskip
une  fin  que  l'on  s'efforce  de  réaliser  :  la  fin  devient  un  critère  permettant  d'apprécier  et  de 
hiérarchiser les moyens, elle devient normative, par rapport aux moyens, qui sont multiples, alors 
qu'elle  est  unique.  Dans  notre  activité,  le  moyen  n'est  qu'une  fin  apparente  alors  que  la  fin  est 
l'objet réel de nos préoccupations. Mais d'autre part, nous pouvons nous servir du couple moyen  
\bigskip
      fin 
comme critère du couple apparence, la réalité étant ce que nous voulons vraiment connaître,  
                                                 réalité 
\bigskip
l'apparence n'étant qu'un moyen d'y parvenir par des voies multiples et équivoques. C'est ainsi que 
Schopenhauer ne verra dans l'illusion amoureuse, qu'un moyen par lequel se manifeste la volonté 
de l'espèce, qui est la réalité profonde et qui seule importe (1). Il en résulte que la jeune fille à qui 
ses parents veulent faire épouser un riche vieillard et qui obstinément s'y refuse, par amour pour 
un jeune homme, n'est égoïste qu'en apparence, et qu'en réalité elle se sacrifie au bien de l'espèce 
(2).  Que,  en  l'occurrence,  le  désir  de  la  jeune  fille  coïncide  avec  le  bien  de  l'espèce  n'est 
qu'accidentel ; en d'autres circonstances, si par exemple, elle se refuse à avoir des enfants, il y aura 
opposition. L'amour individuel est une manifestation de la volonté de l'espèce, seule essentielle, la 
manifestation  elle-même  n'étant  qu'un  moyen,  qu'une  apparence,  qu'un  accident  purement 
contingent.  La  fin  réelle  permet  ainsi  d'opposer  le  réel  à  ses  manifestations,  apparentes, 
contingentes, accidentelles. » 
\bigskip
(1) Schopenhauer, éd. Brockhaus, vol. 3 : Die Welt als Wille und Vorstellung, Zweiter Band, Kap. 
44, p. 636. 
(2) Ibid., p. 640. 
\bigskip
P  576 :  « Traiter  quelque  chose  comme  un  moyen,  c'est  le  dévaluer,  c'est  lui  enlever  sa  valeur 
absolue, la valeur que l'on accorde à ce qui vaut en soi, à ce qui vaut comme une fin, ou comme un 
principe.  Nous  avons  vu  que  c'est  l'un  des  reproches  que  font  les  idéalistes  à  l'utilisation  de 
l'argument  pragmatique  :  en  appréciant  un  fait  en  fonction  de  ses  conséquences,  on  a  l'air  de  le 
considérer comme un moyen en vue de ces conséquences, et par là même on le dévalue (3). » 
\bigskip
(3) Cf. § 62 : L'argument pragmatique. 
\bigskip
P  577 :  « Nous  avons  vu  aussi  qu'un même  phénomène,  traité  alternativement  comme  moyen  en 
vue d'une fin ou comme un fait ou un principe entraînant une conséquence, n'a pas, dans ces deux 
cas, même valeur (1). C'est que l'on transforme en terme I d'un couple moyen ce qui était terme II 
\bigskip
fin 
\bigskip
                     fait ou principe 
\bigskip
dans le couple conséquence. Considérer quelque chose comme moyen, c'est dire que ce n'est 
\bigskip
\bigskip
qu'apparemment notre souci. La louange est une conséquence de la vertu. Si celle-ci est envisagée 
comme un moyen de briller, c'est que nous ne lui accordons qu'une valeur seconde : 
\bigskip
Ainsi l'empressement qu'il a pour l'honneur fait croire qu'il n'aime pas la vertu, et ensuite le fait 
paraître indigne de l'honneur (2). 
\bigskip
Certains  effets  ne  peuvent,  comme  on  le  voit,  être  obtenus  qu'à  condition  de  ne  pas  être 
recherchés,  ou  du  moins  de  se  présenter  comme  une  conséquence  de  faits  indépendants  de  la 
volonté  ou  d'une  conduite  déterminée  par  d'autres  préoccupations.  On  se souviendra  du  passage 
de Proust déjà cité (3). Un fait est dévalué, et ses conséquences normales ne se produisent pas, s'il 
est  perçu  comme  un  moyen  en  vue  de  ces  conséquences.  Le  moyen,  terme  I,  sera  qualifié 
péjorativement de Procédé. Nous verrons, plus loin, de quelle façon les techniques argumentatives 
elles-mêmes sont sujettes à cette disqualification. 
\bigskip
\bigskip
\bigskip
\bigskip
308 
\bigskip
La maladie diplomatique, celle à laquelle on lie croit pas, parce qu'elle sert trop bien le soi-disant 
malade, en lui permettant de présenter son absence comme une chose non délibérée, puisqu'elle 
n'est que la conséquence d'une situation de fait, est le procédé par excellence. » 
\bigskip
(1) Cf. § 63 : Le lien causal comme rapport d'un fait à sa conséquence on d'un moyen à une fin. 
(2) Bossuet, Sermons, vol. Il : Sur l'honneur du monde, p. 726. 
(3) M. Proust, A la recherche du temps perdu, vol. 12 : La prisonnière, II, p. 210 ; et. § 63. 
\bigskip
P 578 : « Dans la mesure où ce qui est moyen est allégué comme fin, il sera qualifié de prétexte : 
\bigskip
Henriette, entre nous, est un amusement,  
Un voile ingénieux, un prétexte, mon frère,  
A couvrir d'autres feux, dont je sais le mystère (1). 
\bigskip
Ce qui passait pour fin peut donc devenir terme I dans le couple conséquence ou dans le couple  
\bigskip
fait 
\bigskip
prétexte. 
fin 
\bigskip
La valorisation des phénomènes dépendra étroitement de leur place dans les couples. Ce qui n'est 
que conséquence, qui n'est pas fait ou principe, a moins d'importance : c'est pourquoi toute étude 
des  causes  de  la  criminalité,  d'oit  il  résulterait  que  celle-ci  n'est  que  la  conséquence  d'un  état  de 
choses préexistant diminue toujours quelque peu, qu'on le veuille ou non, l'indignation morale la 
plus  légitime.  Pour  les  mêmes  raisons,  on  ressent  spontanément  comme  une  dévaluation 
indéniable,  la  transformation  des  phénomènes  de  conscience  en  épiphénomènes,  la  tentative  de 
faire de l'homme un produit de son milieu. Chaignet, en appréciant les vues de Taine, qui sont une 
présentation des phénomènes culturels comme conséquence, comme terme I d'un couple  dont le 
milieu social serait le terme II, écrit spontanément : 
\bigskip
L'homme  n'est  plus  même  la  mesure  des  choses  :  il  en  est  le  jouet.  Le  génie  qu'on  s'était  plu, 
jusqu'ici, à considérer comme une force, une cause, n'est qu'un résultat ; ce n'est plus une lumière, 
c'est un reflet; ce n'est plus une voix, c'est un écho (2). » 
\bigskip
(1) Molière, Femmes savantes, 11. 3; cf. Littré au mot prétexte. 
(2) A. Ed. Chaignet, La rhétorique et son histoire, p. xv. 
\bigskip
P  578-579 :  « L'homme,  même  génial,  n'est  plus  la  réalité  qui  compte  :  c'est  un  reflet,  un 
épiphénomène,  une  apparence.  La  conséquence,  comme  le  moyen,  ne  sont  appréciés  qu'en 
fonction du fait ou de la fin dont ils dépendent : le choix de l'un ou de l'autre des couples moyen ou 
\bigskip
     fin 
\bigskip
\bigskip
conséquence aboutit donc à valoriser ou à dévaloriser le même phénomène. » 
     fait 
\bigskip
P  579 :  « La  liberté  est  la  valeur  fondamentale  des  existentialistes.  Néanmoins  quand  il  s'agit  de 
dévaluer la liberté de la femme romaine dans l'Antiquité, S. de Beauvoir n'hésite pas à transformer 
cette  liberté  en  un  moyen  et,  qui  plus  est,  en  un  moyen  pour  une  fin  sans  consistance,  en  une 
liberté « pour rien » (1). Inversement, pour valoriser le courage, normalement apprécié comme un 
moyen précieux dans l'action, Jankélévitch écrira : 
\bigskip
...  cette  vertu  est  ensemble  formelle  et  catégorique  ;  comprenez  qu'elle  est  toujours  belle  et 
toujours vertueuse, quel que soit son contenu, et ne dépend pas de la valeur de sa fin (2). 
\bigskip
\bigskip
\bigskip
\bigskip
309 
\bigskip
Voir la connaissance comme une conséquence du réel, l'action comme résultat de la connaissance, 
c'est affirmer un réalisme ontologique, le primat du théorique sur le pratique. Le pragmatisme, par 
contre,  appréciera  en  fonction  de  l'action,  qui  est  la  seule  fin  et  le  seul  critère,  aussi  bien  de  la 
connaissance que de la conception que l'on se fait du réel. C'est ainsi que E. Dupréel ne veut voir 
dans l'ontologisme même que sa fin : 
\bigskip
...  le  philosophe  va  vers  l'être  et  nous  dit  ce  qu'il  est,  à  quelle  .  Afin  que  nous  soyons  au  fait  de 
l'inévitable préalable, de tout ce qui, dans l'aménagement de nos actes et dans la position de nos 
fins, s impose à nous comme obstacle ou s'offre comme moyen. Connaître l'être, si c'est une joie de 
le découvrir, c'est aussi le principe d'une résignation. On comprend que chez le philosophe, ce soit 
une  résignation  enthousiaste,  parce  que,  étant  le  premier  résigné,  il  fera  fie,  à  l'égard  de  ceux 
qu'il renseigne, de guide et quelquefois de chef. Après avoir plié le genou devant le dieu, le prêtre 
se retourne et commande (3). » 
\bigskip
(1) S. De Beauvoir, Le deuxième sexe, I, p. 153.  
(2) V. Jankélévitch, Traité des vertus, p. 189.  
(3) E. Dupréel, Esquisse d'une philosophie des valeurs, p. 24. 
\bigskip
P 579-580 : « Une fin reconnue n'est  pas, pour autant, une fin absolue une nouvelle  dissociation 
pourra la transformer en moyen en vue d'une fin ultérieure. Cette dernière permettra de discerner, 
dans la fin primitive, qui aura perdu sa valeur de terme II, ce qui constitue un bon ou un mauvais 
moyen, c'est-à-dire ce qui, en tant que terme I, conserve une certaine valeur. » 
\bigskip
P  580 :  « Rien  ne  s'oppose  théoriquement  à  la  répétition  indéfinie  de  cette  opération,  à  cette 
transformation  de  fins  en  moyens  par  la  dissociation,  et  à  la  disqualification  qui  en  résulte.  Ce 
processus  permet  à  un  adversaire  du  rationalisme,  comme  Buber,  de  stigmatiser  la  vision  du 
monde de celui pour lequel tout n'est que technique, que rapport de moyen à fin. Pour Buber, tout 
ce qui relève de l'utile fait partie du domaine du cela : 
\bigskip
La  fonction  d'expérimentation  et  d'utilisation,  chez  l'homme,  se  développe  généralement  au 
détriment  de  l'aptitude  à  la  relation...  L'homme  de  l'arbitraire,  incrédule  jusqu'aux  moelles,  ne 
voit partout qu'incrédulité et arbitraire, choix des fins et invention des moyens. Un monde privé 
du sacrifice et de la grâce, de la rencontre et de la présence, un monde empêtré dans les fins et les 
moyens, voilà son monde (1). ... 
\bigskip
L'effort philosophique de Buber, en insistant sur la possibilité constante de transformer en terme 1 
le terme Il de tout couple moyen disqualifie ce couple tout entier et la vision du monde qui lui est 
\bigskip
fin 
\bigskip
liée, au profit d'un monde où priment la rencontre et l'amour humains, la rencontre et l'amour de 
Dieu. Il nous fournit un échantillon intéressant montrant comment la technique de la dissociation 
des  couples  philosophiques  peut  provoquer  le  rejet  des  deux  termes  du  couple,  le  reniement  du 
point de vue que le recours à ce couple présuppose, au nom d'une autre vision et d'un autre critère 
de la réalité. » 
\bigskip
(1) M. buber, Je et Tu. pp. 71, 95. 
\bigskip
§ 93. L'EXPRESSION DES DISSOCIATIONS 
\bigskip
P  580-581 :  « La  présence  de  couples  philosophiques  se  révèle,  à  qui  connaît  les  usages  d'une 
langue, par des expressions caractéristiques qui permettent au. premier coup d’œil de distinguer le 
terme I du terme II. C'est ainsi qu'à partir de l'opposition apparence, n'importe quelle notion peut 
\bigskip
être  dissociée  par  l'adjonction  des  adjectifs  «apparent  »  ou  «  réel  »,  pu  des  adverbes  « 
apparemment  »  ou  «réellement  ».  D'une  façon  générale,  chaque  fois  qu'une  dissociation  se 
\bigskip
réalité 
\bigskip
\bigskip
\bigskip
310 
\bigskip
marque  par  un  couple  de  substantifs,  les  adjectifs  et  les  adverbes  dérivés  pourront  indiquer  de 
nouvelles dissociations. Le couple nom permettra d'opposer le salaire nominal au salaire réel, et 
\bigskip
chose 
\bigskip
et le couple lettre, permettra d'écrire que lorsque la parole de Dieu est fausse littéralement, elle 
\bigskip
         esprit 
\bigskip
est vraie spirituellement (1). » 
\bigskip
(1) Cf. § 84 : Effets de l'analogie. A noter que le contenu d'une dissociation peut correspondre au 
thème et au phore d'une analogie (cf. p. 572). 
\bigskip
P 581 : « L'article défini (la solution), le démonstratif (ce monde, ille homo) peuvent indiquer qu'il 
s'agit  de  la  solution,  du  monde,  de  l'homme  véritables,  qui  seuls  comptent.  La  majuscule,  elle 
aussi, annoncera le terme II : la Guerre c'est la vraie guerre ; et Amphion chantera : 
\bigskip
Ai-je blessé, heurté, 
Charmé, peut-être, 
Le Corps secret du monde? 
…………………………………………………… 
Et touché l'Être même que nous cache 
La présence de toutes choses (2) ? » 
\bigskip
(2) P. Valéry, Poésies. Amphion, scène V, p. 283. 
\bigskip
P 581-582 : « Le terme II est généralement appelé « proprement dit ». Par contre, en accolant à un 
substantif un préfixe comme pseudo, quasi, non, on annonce la présence d'un terme I : 
\bigskip
...  le  Pseudo-athée,  en  niant  l'existence  de  Dieu  nie  l'existence  d'un  être  de  raison  qu'il  appelle 
Dieu mais qui n'est pas Dieu, - il nie l'existence de Dieu parce qu'il confond Dieu avec cet être de 
raison... le vrai athée, en niant l'existence de Dieu nie réellement, par un acte (le son intellect qui 
demande de soi à transformer toute sa table des valeurs et à descendre dans les profondeurs de 
son  être,  l'existence  de  ce  Dieu  qui  est  l'objet  authentique  de  la  raison  et  de  la  foi  et  qu'il 
appréhende dans sa notion véritable (1). » 
\bigskip
(1) J. Maritain, Raison et raisons, p. 161. 
\bigskip
P 582 : « Le terme II est celui qui est authentique, véritable, réel ; le terme 1 désigne ici, comme 
souvent, un être de raison, un fabricat illusoire, une théorie inadéquate. 
\bigskip
Sartre parlera de « quasi-multiplicité » qui est, en fait, « une unité qui se multiplie » (2). Husserl 
disqualifie les philosophies sceptiques en les traitant de« non-philosophies » : 
\bigskip
Les  luttes  spirituelles  proprement  dites  de  l'humanisme  européen  en  tant  que  tel  se  déroulent 
comme des luttes entre des philosophies, a savoir entre les philosophies sceptiques - ce sont plutôt 
des «Non-Philosophies » (Unphilosophien) qui n'en ont gardé que le nom mais pas la mission, - et 
les philosophies véritables et encore vivantes (3). 
\bigskip
A côté des préfixes disqualifiants, il existe une série d'autres expressions indiquant qu'il s'agit d'un 
terme I, depuis le mot «prétendre » jusqu'à la mise entre guillemets. 
\bigskip
Lefebvre dira, à propos des arguments de l'idéalisme 
\bigskip
L’idéalisme  ne  trouve  d'arguments  (si  l'on  peut  dire  !),  qu'en  renversant  non  seulement  le 
processus réel de la connaissance (4)... 
\bigskip
\bigskip
\bigskip
\bigskip
311 
\bigskip
Et il écrira de même : 
\bigskip
Le paradoxe, auquel est acculé l'idéalisme moderne, juge la portée de sa « critique » (5)... 
\bigskip
On  connaît  le  fréquent  usage  de  cette  forme  de  disqualification  dans  la  polémique  communiste 
(6). » 
\bigskip
(2) J.-P. Sartre, L'être et le néant, p. 179. 
(3)  E.  Husserl,  La  crise  des  sciences  européennes  et  la  phénoménologie  transcendantale,  Les 
études philosophiques, 1949, n° 2, p. 139. 
(4)  H.  Lefebvre,  A  la  lumière  du  matérialisme  dialectique,  I  :  Logique  formelle,  logique 
dialectique. p. 24. 
(5) Ibid., p. 24. 
(6) Cf. A. Koestler, dans R. Crossman, The God that failed, p. 56. 
\bigskip
Ces différentes désignations du terme I font allusion au couple opinion, subjectif, verbal, qui  
\bigskip
       vérité    objectif     réel 
\bigskip
distinguent ce qui est prétendu par certains de ce qui est réellement. » 
\bigskip
P 583 : « De telles opinions sont disqualifiées comme naïves (1), comme erreurs, illusions, mythes, 
rêveries,  préjugés,  fantaisies  ;  leur  objet  est  idole,  fantasmagorie  ;  elles  constituent  un  voile,  un 
écran, un masque, un obstacle à la connaissance de la réalité; en effet, le terme I est apparent et 
visible,  immédiatement  donné  ;  dans  la  mesure  où  il  ne  nous  révèle  pas  le  terme  II,  il  risque  de 
nous  le  cacher.  C'est  la  façon  habituelle  dont  les  philosophes  hindous  opposent  la  réalité  à 
l'apparence : 
\bigskip
L'âme, l’être psychique, est en contact direct avec la  vérité divine, mais dans l'homme l'âme  est 
masquée par le mental, par l'être vital et par la nature physique (2). 
\bigskip
De même, dans la phénoménologie de Levinas, l'image n'est pas reproduction de l'objet, mais elle 
possède une certaine opacité qui nous voile la réalité (3). 
\bigskip
Celle-ci, dans un beau poème philosophique de Girolamo Fracastoro, est présentée comme ce qui 
transparaît dans la nuit : 
\bigskip
Ne sais-tu pas que toutes choses que couvre la nuit 
Ne sont pas vraies comme elles sont, mais ombres ou 
Spectres, à travers lesquels transparaît une Figure Etrangère (4) ? 
\bigskip
La  mention  d'une  substitution  signale  souvent  la  présence  d'une  dissociation  :  on  veut  indiquer 
que le sujet, en prenant pour terme II ce qui n'est que terme I, parvient à se leurrer : 
\bigskip
Toujours  l'intellect  substitue  ses  propres  représentations,  constructions  et  opinions  à  la 
connaissance véritable (5). » 
\bigskip
(1)  E.  Husserl,  Op.  Cit.,  P.  140;  cf.  aussi  le  vers  du  poète  noir  Césaire  qui  appelle  les  blancs  ~ 
vainqueurs omniscients et naïfs », commenté par J.-P. SARTRE, Situations, III, p. 265. 
(2) Shrî Aurobindo, Le guide du yoga, p. 187. 
(3) E. Levinas, La réalité et son ombre, Les temps modernes, nov. 1948, pp. 777. 780. 
(4) G. Toffanin, Storia dell' umanesimo, p. 303. Girolamo Fracastoro, Opera omnia : Carminum 
liber unus, Ad. AL Antonium Fluminium et Galeatium Florimontium, p. 206 a. 
(5) Shrî Aurobindo, Le guide du yoga, p. 186. 
\bigskip
\bigskip
\bigskip
\bigskip
312 
\bigskip
P 584 : « Par là, le sujet se donne des raisons apparentes d'agit qu'il faut exorciser : 
\bigskip
De la soumission au Divin on ne doit pas faire une excuse, un prétexte ou une occasion pour se 
soumettre  à  ses  propres  désirs,  àses  mouvements  inférieurs,  à  son  ego  ou  a  quelque  force 
d'ignorance ou d'obscurité qui faussement se donne l'apparence du Divin (1). 
\bigskip
Pour éviter ce piège, il faut écarter toute impureté, qui produit le trouble et l'erreur. La purification 
est un processus qui permet de dégager le terme Il de ce qui n'en a que l'apparence, de ce qui n 1 en 
constitue qu'une approximation plus ou moins imparfaite. Le rôle du psychologue traitant, d'après 
Odier est de : 
\bigskip
réduire des obstacles s'opposant au libre essor spirituel... C'est une œuvre de purification (2). 
\bigskip
L'apparent, c'est le visible, qui se trouve à la surface, qui est superficiel, et par là n'est qu'un petit 
fragment de la réalité, qui veut se faire passer pour le tout : 
\bigskip
...  ce  petit  mental,  ce  petit  vital,  ce  petit  corps  que  nous  appelons  nous-mêmes  ne  sont  qu'un 
mouvement  superficiel  et  pas  du  tout  notre  vrai  «moi  ».  Tout  cela  n'est  qu'une  bribe  toute 
extérieure de personnalité, mise en avant pendant une brève existence pour le jeu de l'ignorance 
(3). 
\bigskip
Il en est de même de thèses contradictoires dans une discussion, qui se dépassent dans la synthèse 
: 
\bigskip
Les thèses en présence se découvrent alors comme incomplètes, comme superficielles, comme des 
apparences  momentanées,  des  lambeaux  de  vérité...  «Nous  donnons  le  nom  de  dialectique  au 
mouvement plus élevé de la raison dans lequel ces apparences séparées passent l'une en l'autre... 
et se dépassent (Hegel, Grande Logique, I, 108) » (4). 
\bigskip
(1) Shri Aurobindo, p. 56. 
(2) Ch. Odier, Les deux sources, consciente et inconsciente, de la nie morale, p. 31.  
(3) Shri Aurobindo, Le guide du yoga, p. 189. 
(4)  H.  Lefebvre,  A  la  lumière  du  matérialisme  dialectique,  I  :  Logique  formelle,  logique 
dialectique, p. 148. 
\bigskip
P  585 : « Le  fragmentaire  est  destiné à  disparaître,  il  n'est  que  fugitif  et  accidentel  ;  est  réel,  par 
contre,  ce  qui  est  profond,  durable,  permanent,  essentiel.  Il  est  normal  que  t  outes  les  activités 
visant à dégager le terme II dans toute sa pureté soient conçues comme une libération, comme une 
lutte  contre  les  obstacles  accumulés  par  le  terme  I.  Pour  triompher,  il  faut  traiter  tout  ce  qui 
concerne le terme 1 comme quelque chose d'étranger, d'ennemi : 
\bigskip
Quand  on  vit  dans  la  vraie  conscience,  on  sent  les  désirs  hors  de  soi,  venant  du  dehors...  La 
première condition pour se débarrasser du désir est donc d'acquérir la vraie conscience, car il est 
alors  beaucoup  plus  facile  de  asser  que  si  l'on  doit  lutter  contre  lui  comme  contre  une  partie 
constituante  de  l'être  qu'il  faudrait  rejeter  loin  de  soi.  Il  est  plus  aisé  de  se  débarrasser  d'une 
accrétion que d'amputer ce que l'on sent être un morceau de soi-même (1).  
\bigskip
Ce passage est particulièrement intéressant, parce que le rejet dans un terme I, des éléments dont 
on  veut  se  débarrasser,  est  ouvertement  préconisé  comme  étant  une  technique  plus  efficace  que 
celle qui consisterait à s'en rendre maître par contrainte morale c'est-à-dire par le sacrifice d'une 
part réelle de soi-même. » 
\bigskip
(1) Shrî Aurobindo, Le guide du yoga, p. 90. 
\bigskip
\bigskip
\bigskip
313 
\bigskip
 
P 585-586 : « Le terme I sera souvent disqualifié comme factice ou artificiel, par  opposition à  ce 
qui  est  authentique  ou  naturel.  Tarde  s'est  insurgé  contre  cette  technique  de  disqualification, 
répandue par les romantiques qui traitaient d'artificiel, tout ce qui leur déplaisait : 
\bigskip
On a la mauvaise habitude d'appeler artificiel, en toute catégorie de phénomènes sociaux, l'ordre 
établi par uni-conscience ; artificielles, les codifications durables introduites dans les langues par 
quelque  grammairien  fameux  tel  que  Vaugelas;  artificielles  les  codifications  législatives,  les 
constitutions  tout  d'une  pièce,  les  sommes  théologiques;  artificielles  surtout,  ces  grandes 
philosophies  encyclopédiques  jaillies  de  la  tête  d'un  Aristote,  d'un  Descartes,  d'un  Kant,  qui  de 
mille morceaux de science font un seul et riche vêtement - ou déguisement - du vrai ... ; artificiel 
enfin, suivant les économistes de l'ancienne école, tout régime industriel et économique qui ne se 
sera  pas  fait  comme  de  lui-même,  toute  hiérarchie  et  toute  discipline  des  diverses  productions, 
des divers intérêts, qui, même libérale et, dans une certaine mesure, individualiste, naîtrait avec 
le péché originel d'avoir été savamment élaborée par une seule tête, utilisant les travaux de mille 
esprits antérieurs (1)... » 
\bigskip
P 586 : « Cette disqualification en tant que factice, se reconnaîtra dans de très nombreux cas : les 
adversaires  de  la  métaphysique  qualifieront  ses  énoncés  de  «  fictitious  »  par  opposition  à  « 
genuine» (2) ; Pareto, adversaire du compromis, du raisonnement persuasif, qualifiera celui-ci de 
« dérivations » par rapport aux « résidus » qui sont la vraie réalité sociale (3) ; les existentialistes 
contemporains se serviront dans leurs tentatives de disqualification de l'adjectif «inauthentique », 
d'autres du terme « mécanique ». 
\bigskip
(1) Tarde, La logique sociale, pp. 203-4. 
(2) A. J. Ayer, Language, Truth and Logic, pp. 37, 40, 43, 50. 
\bigskip
P  586-587 :  « Chaque  doctrine  élabore  ses  couples  philosophiques,  le  terme  II  indiquant  ce  qui 
sert, de critère de valeur, le terme I ce qui ne satisfait pas à  ce critère. Mais nous voyons qu'il y a 
certains traits qui caractérisent le plus généralement le terme I. Nous ne résistons pas au désir de 
citer  ici,  pour  bien  le  souligner,  un  large  extrait  d'une  page  de  Nelly  Cormeau.  Le  critique  ayant 
rencontré  chez  Mauriac  plusieurs  thèmes  que  nous  grouperions  volontiers  autour  d'un  couple 
central social , et, appliquant à l'auteur lui-même les valeurs que celui-ci a mises en évidence,  
\bigskip
individuel 
s'exprime ainsi : 
\bigskip
Il  y  a,  chez  Mauriac,  quelque  chose  d'audacieux  et  d'authentique  -nous  dirions  volontiers 
d'impollué - une intégrité individuelle, un -noyau de Pureté qui ne se laisse ni intimider ni fausser 
par le monde et la vie sociale... Mais tout ce qui est superstructure purement sociale, c'est peu de 
dire que cela le laisse indifférent : nous avons vu avec quelle virulence il fustige les « convenances 
»,  les  compromissions,  les  préjugés...  Il  a  horreur  des  castes...  groupements  factices...  Le  monde 
pollue  la  pure  nature  modelée  par  le  Créateur...  Son  cadre  véritable  est  la  libre  nature...  Faut-il 
rappeler ici tous ces personnages qui, dans un salon, dans un bar  - dans l'atmosphère  frelatée et 
artificielle du « monde » - se sont sentis soudain submergés par une vague immense de désespoir 
?...  Or,  Mauriac,  est  toujours  pour  la  vérité  contre  le  mensonge,  pour  l'esprit  contre  la  tradition, 
pour  l'authenticité  des  rapports  directs  de  personne  à  personne...  Et  c'est  cette  noblesse  innée, 
cette  pureté  loyale,  cette  ingénuité  incorruptible,  cette  résolution  impavide  de  dénoncer  toute 
falsification, qui, de tonte œuvre mauriacienne, font jaillir un appel pressant à ce qui gît en nous 
de  plus  palpitant  et  de  plus  sincère...  C'est  tout  cela  aussi  -cette  authenticité  sans  lard,  cette 
absence  candide  et  hardie  de  masque  et  d'armure  -  qui  porte  Mauriac  avec  tant  d'équité  et  de 
franchise, et malgré son catholicisme absolu, à la rencontre des incroyants (1). » 
\bigskip
(3) V. Pareto, Traité de sociologie générale, II, § 1403, p. 791. 
\bigskip
\bigskip
\bigskip
314 
\bigskip
(1) Nelly Cormeau, L'art de François Mauriac, pp. 183-184. C'est nous qui Soult 
\bigskip
P 587 : « On relève ici, spontanément groupés sous la plume du critique, la plupart des expressions 
que  nous  avons  considérées  comme  caractéristiques  du  terme  II  :  impollué,  noyau,  authentique, 
vérité;  celles  qui  caractérisent  le  terme  1  :  impureté,  superstructure,  factice,  frelaté,  artificiel, 
mensonge; l'idée de masque, de fard ; l'idée que le terme I est obstacle (armure) ; enfin celle aussi 
de l'erreur (fausser). 
\bigskip
Ces quelques remarques relatives à l'énoncé des couples ne doivent point nous faire oublier qu'une 
expression  couramment  utilisée  pour  indiquer  une  dissociation  est  loin  d'avoir  toujours  cette 
signification.  Ainsi  l'adjectif  «  éternel  »  désigne  souvent  un  terme  II  :  pour  les  Allemands 
adversaires  du  3e  Reich,  l'Allemagne  éternelle  était  l'Allemagne  véritable,  par  opposition  à 
l'Allemagne nazie, transitoire, apparente; mais pour Hitler, cet adjectif accolé au mot « Allemagne 
» n'était qu'une forme du superlatif (2). » 
\bigskip
(2) V. Klemperer, L. T. I, Notizbuch eines Philologen, p. 202, p. 277. 
\bigskip
§ 94. ÉNONCES INCITANT A LA DISSOCIATION 
\bigskip
P  587-588 :  « Devant  certains  énoncés,  on  ne  peut  éviter  la  dissociation  notionnelle  si  l'on  tient, 
comme  c'est  normal,  à  l'aspect  à  la  fois  significatif  et  cohérent  de  la  pensée.  Ces  expressions 
invitent  à  dissocier  une  notion,  sans  préciser  néanmoins  de  quelle  manière  devra  s'effectuer  la 
dissociation. Le contexte indiquera ce qui doit être considéré comme terme I ou comme terme II. » 
\bigskip
P 588 : « Ainsi ce distique de Schiller : 
\bigskip
Quelle religion je professe ? Aucune de toutes celles 
Que tu me nommes. - Et pourquoi aucune? - Par religion (1) 
\bigskip
oblige  le  lecteur,  pour  comprendre  la  pensée,  à  reconnaître  au  mot  «  religion  »  deux  usages  qui 
correspondent à une dissociation sous-entendue religion apparente ou religion positive, les  
\bigskip
    religion positive           religion naturelle 
\bigskip
les  religions  que  l'on  rejette  correspondant  au  terme  I,  celle  que  l'on  professe  au  terme  II  du 
couple. 
\bigskip
Le même effort de dissociation est exigé par l'expression 
\bigskip
Si duo faciunt idem, mon est idem. 
\bigskip
ou par cette phrase de Mirabeau : 
\bigskip
Or, un roi dans ce cas, n'est plus un roi (2). 
\bigskip
Plusieurs de ces expressions constituent ce que nous avons appelé des figures quasi logiques : 
tautologie apparente, négation d'un terme par lui-même, identité des contradictoires (3). Dans des 
expressions telles « les affaires sont les affaires », « un son n'est pas un sou », le même terme ne 
peut être pris deux fois dans le même sens. Le moyen de résoudre la difficulté sera de dissocier en 
termes I et II. » 
(1)Welche Religion Ich bekenne ? Keine von allen 
Die du mir nennst. - Und warum Keine ? - Aus Religion ! 
cité par Erdmann, Die Bedeutung des Wortes, p. 61. 
(2) Cf. Timon, Livre des orateurs, p. 193. 
(3) Cf. § 51 : Analycité, analyse et tautologie. 
\bigskip
\bigskip
\bigskip
\bigskip
315 
\bigskip
P  588-589 :  « Cette  dissociation  sera  à  la  fois  le  but  visé  par  l'emploi  de  l'expression  et  sa 
justification.  Ainsi  Sartre  utilise  l'expression  «  l'être  est  ce  qu'il  est  ».  Dans  une  longue 
argumentation,  il  en  conteste  le  caractère  analytique.  L'interprétation  qu'il  en  donne  équivaut  à 
dissocier l'être en soi de l'être pour soi. La formule, dit-il : 
\bigskip
... désigne une région singulière de l'être: celle de l'être en Soi- Nous verrons que l'être du pour soi 
se définit au contraire comme étant ce qu'il n'est pas et n'étant pas ce qu'il est... le fait d'être ce 
qu'on est... est un principe contingent de l'être en soi (1). » 
\bigskip
(1) J.-P. Sartre, L'être et le néant, p. 33. 
\bigskip
P  589 :  « L'exigence  de  dissociation  pourra résulter  d'une  opposition  entre un  mot  et  ce  que  l'on 
considère, communément, comme son synonyme, telle cette constatation de Panisse : 
\bigskip
De mourir, ça ne me fait rien. Mais ça me fait peine de quitter la vie (2). 
\bigskip
Les expressions paradoxales invitent toujours à un effort de dissociation. Chaque fois qu'est accolé 
à  un  substantif  un  adjectif,  ou  un  verbe,  qui  semble  incompatible  avec  lui  (docte  ignorance,  mal 
bienheureux,  joie  amère,  penser  l'impensable,  exprimer  l'inexprimable,  les  conditions  de  la 
capitulation inconditionnelle) (3), seule une dissociation permettra la compréhension. 
\bigskip
Il  en  est  de  même  lorsque,  entre  des  notions,  est  affirmé  un  rapport  inadmissible,  par  exemple 
dans cette définition du poète par Orphée : 
\bigskip
C'est écrire sans être écrivain (4). 
\bigskip
ou dans ces vers de Plain-Chant : 
\bigskip
L'encre dont je me sers est le sang bleu d'un cygne,  
Qui meurt quand il le faut pour être plus vivant (5). » 
\bigskip
(2) M. Pagnol, César, p. 24. 
(3) Pour ce dernier paradoxe, cf. The memoirs of Cordell Hull, pp. 1570 à 78. 
(4) J. Cocteau, Orphée (film), dans Empreintes, numéro consacré à Jean Cocteau, mai-juill. 1950, 
p. 163. 
(5) J. Cocteau, Plain-Chant, ibid., en épigraphe, p. 9. 
\bigskip
P 589-590 : « On pourrait en rapprocher la maxime respectée de la vie japonaise « perdre pour 
gagner » dont les applications, décrites par Ruth Benedict, montrent que ce que l'on perd n'est que 
terme I par rapport à ce que l'on gagne (1). » 
\bigskip
(1) Ruth Benedict, The Chrysanthemum and the Sword, p. 266. 
\bigskip
P 590 : « Quant au rapport de détermination entre termes identiques, non seulement il invitera à 
une  dissociation,  mais  suggérera  que  celle-ci  approfondit  une  première  dissociation,  témoin 
l'expression «l'âme  de l'âme » dont use Jankélévitch  (2), laquelle se superpose àune  dissociation 
où âme était terme II. 
\bigskip
Des  tournures  comme  celles  que  nous  venons  de  décrire  forment  ce  que  l'on  a  appelé 
Paradoxisme,  antithèse  formulée  à  l'aide  d'une  alliance  de  mots  qui  semblent  s'exclure 
mutuellement (3), ou la figure que Vico appelle oxymoron « nier d'une chose qu'elle soit ce qu'elle 
est » (4). On les retrouve aussi très souvent dans la polyptote, usage du même mot sous plusieurs 
\bigskip
\bigskip
\bigskip
316 
\bigskip
formes  grammaticales,  dans  l'antimétathèse  ou  antimétabole  (5),  reprise  dans  deux  phrases 
successives des mêmes mots dans un rapport inversé, parfois confondue avec la commutation (6). 
\bigskip
§ 95. LES DEFINITIONS DISSOCIATIVES 
\bigskip
La définition est un instrument de l'argumentation quasi logique (7). Elle est aussi un instrument 
de la dissociation notionnelle, notamment chaque fois qu'elle prétend fournir le sens véritable, le 
sens réel de la notion, opposé à son usage habituel ou apparent. Ainsi Shrî Aurobindo, pour définir 
le  «  travail  »,  après  avoir  éliminé  des  conceptions  plus  usuelles,  nous  fournit  ce  qu'il  considère 
comme « la vérité plus profonde du travail » : 
\bigskip
Par « travail » j'entends l'action faite pour le Divin et de plus en plus en union avec le Divin - pour 
le Divin seul, et rien d'autre (8). 
\bigskip
(2) V. Jankélévitch, Traité des vertus, p. 58. 
(3) Baron, De la Rhétorique, p. 361. 
(4) Vico, Instituzioni oratorie, p. 151. 
(5) Vico, ibid., p. 150. 
(6) Cf. § 92 : Le rôle des couples philosophiques et leurs transformations. 
(7) Cf. § 50 : Identité et définition dans l'argumentation. 
(8) Shrî Aurobindo, Le guide dit yoga, pp. 207-8. 
\bigskip
P  591 :  « A.  Smith,  rejetant  les  autres  critères,  plus  aisés  à  manier,  plus  «naturels  et  évidents  », 
mais instables, par lesquels la valeur des biens est « communément estimée », avait déclaré :  
\bigskip
Le travail est la mesure réelle de la valeur d'échange de toutes les marchandises (1). 
\bigskip
En  opposant  la  nature  des  choses  à  la  signification  des  mots,  Spinoza  nous  avertit  que  ses 
définitions s'écartent de l'usage : 
\bigskip
Je  sais  que  l'usage  donne  à  ces  mots  un  autre  sens.  Mais  mon  dessein  est  d'expliquer,  non  la 
signification des mots, mais la nature des choses, et il nie suffit de désigner les passions de l'âme 
par des noms qui ne s'écartent pas complètement de la signification que l'usage leur a donnée ; 
que le lecteur en soit averti une fois pour toutes (2). 
\bigskip
rejet de la conception ancienne, comme ne correspondant pas à la réalité, est entièrement explicité 
dans ce  passage de Berkeley relatif à la possibilité de maintenir la notion de matière grâce à une 
définition nouvelle : 
\bigskip
... il n'y a pas de matière, si par ce ternie on entend une substance non pensante qui existe hors de 
l'intelligence : mais si, par matière, on entend une chose sensible dont l'existence consiste à être 
perçue, alors il y a une matière (3). 
\bigskip
Toute tentative pour faire connaître discursivement le terme II qui n'est jamais connu directement 
-  pourra  être  considérée  comme  une  définition  de  ce  terme,  c'est-à-dire  l'expression  des  critères 
qui doivent nous permettre de le cerner. Le système entier pourra ainsi servir de définition. Mais 
certaines  expressions  forment  pause,  charnière,  dans  l'enchaînement  de  la  pensée  parce  qu'elles 
sont une expression relativement condensée de ce qui caractérise un certain terme II. 
\bigskip
(1) Adam Smith, The wealth of nations, pp. 31-32.  
(2) Spinoza, Ethique, III, Partie, Appendice, 21, explication 
(3) Berkeley, Les trois dialogues entre Hylas et Philonous, 3e dial., p. 213. 
\bigskip
P 592 : « Cette formulation peut être fort diverse ; ce sera, notamment, l'énoncé d'une condition : 
\bigskip
\bigskip
\bigskip
317 
\bigskip
 
Une pensée religieuse est authentique quand elle est universelle par son orientation (1). 
\bigskip
Souvent,  affirmer  que  telle  chose  tombe  sous  tel  concept,  ou  n'y  tombe  pas,  c'est  introduire 
indirectement  une  définition  dissociative,  surtout  lorsque  l'introduction  d'un  caractère  nouveau 
devient  critère  pour  le  bon  usage  de  la  notion.  C'est  ainsi  qu'Isocrate  dit,  à  propos  des 
Lacédémoniens dont il reconnaît qu'ils furent écrasés aux Thermopyles : 
\bigskip
... il n'est pas permis de dire qu'ils furent vaincus, aucun d'eux n'ayant accepté de fuir (2.). 
\bigskip
L'extension de certains concepts correspond parfois à une redéfinition dissociative, comme dans ce 
passage de Cicéron : 
\bigskip
Non, juges, il n'y a pas seulement comme Violence celle qui s'exerce sur notre corps et notre vie ; 
il  y  en  a  une  autre,  bien  plus  grave,  celle  qui,  par  des  menaces  de  mort,  porte  la  terreur  dans 
notre esprit, et le met souvent hors de lui-même et de son état naturel (3). 
\bigskip
L'extension du concept se combine avec une minimisation de ce qui constituait le concept usuel : la 
violence sur le corps, la plus visible, deviendra aisément, si l'on n'y prend garde, terme I. » 
\bigskip
(1) S. Weil, L'enracinement, p. 84. 
(2) Isocrate, Discours, t. II : Panégyrique d'Athènes, § 92. 
(3) Cité par Quintilien, Vol. III, liv. VII, chap. III, § 17. 
\bigskip
P  592-593 :  « Un  procédé  assez  curieux  consiste  à  donner  deux  définitions,  qui,  au  lieu  d'être 
traitées comme interchangeables (4), correspondent l'une à un terme I, l'autre à un terme Il. C'est 
ainsi que Lecomte du traite de la civilisation : 
\bigskip
Premièrement,  définition  statique  :  la  Civilisation  est  l'inventaire  descriptif  de  toutes  les 
modifications apportées aux conditions morales, esthétiques et matérielles de la vie normale de 
l'homme en société, par le cerveau seul. 
Deuxièmement,  définition  dynamique:  la  Civilisation  est  le  résultat  global  du  conflit  entre  la 
mémoire  de  l'évolution  antérieure  de  l'homme,  qui  persiste  en  lui,  et  les  idées  morales  et 
spirituelles qui tendent àa lui faire oublier (1). 
\bigskip
(4) Cf. § 50 : Identité et définition dans l'argumentation. 
(1) Lecomte Du  L'homme et sa destinée, pp. 123-124. 
\bigskip
P 593 : « Il résulte clairement des commentaires que la définition dynamique est primordiale aux 
yeux de l'auteur : elle correspond au réel, à ce qui est profond ; la définition statique correspond au 
passager,  à  l'apparence  ;  grâce  à  elle,  il  est  fait  place  à  ce  qui  est  grosso  modo  la  définition 
habituelle. Mais les deux définitions sont entre elles dans un rapport correspondant à un couple  
statique, repris à la philosophie bergsonienne.  
dynamique 
\bigskip
P  593-594 :  « Stevenson,  partant  d'une  réflexion  sur  le  raisonnement  moral,  a  qualifié  les 
définitions  du genre  de  celles  dont  nous  parlons  de  «définitions  persuasives  »  (2),  parce  qu'elles 
conservent  le  sens  émotif  des  notions,  celui  qui  doit  influencer  l'interlocuteur,  tout  en  modifiant 
leur sens descriptif. Indubitablement, il s'agit pour l'orateur, dans bon nombre de cas, d'une simple 
technique  de  persuasion.  Mais  outre  que  la  distinction  entre  aspect  émotif  et  descriptif  d'une 
notion est discutable (3), la modification opérée peut résulter d'une conviction intime que l'on croit 
conforme  à  la  réalité  des  choses  et  que  l'on  serait  prêt  à  justifier.  Nous  préférons  insister  sur  la 
manière dont la définition dissocie une notion en termes I et Il, quelle que soit la raison de cette 
\bigskip
\bigskip
\bigskip
318 
\bigskip
dissociation. Remarquons d'ailleurs qu'il peut arriver, comme dans le cas de la définition du travail 
de Shrî Aurobindo, que le procédé n'ait pas pour but de transférer sur un sens nouveau une valeur 
admise,  mais  de  valoriser  une  notion,  de  lui  accorder  un  prestige  dont  elle  était  dépourvue  dans 
son usage antérieur. » 
\bigskip
(2)  Ch.  L.  Stevenson,  Ethics  and  language,  p.  210;  cf.  aussi  dans  The  Emotive  Theory  of  Ethics, 
symposium, Aristotelian Society Supplementary Vol. XXII, 1948, communications de R. Robinson, 
pp. 89-92, et de H. J. Paton, P. 112. 
(3)  Cf.  §  35  et  Ch.  Perelman,  et  L.  Olbrechts-Tyteca,  Les  notions  et  l'argumentation,  dans  vol. 
Semantica, Arch. di Filosofia, 1955, p. 254. 
\bigskip
P 594 : « Parfois la dissociation opposera un sens technique à un sens plus usuel. L'adoption d'un 
sens technique, réservé à un domaine déterminé, pourrait n'avoir guère d'influence sur le concept 
ancien et passer pour simple convention de langage. Mais il est rare qu'une discussion se déroule 
tout  entière  à  l'intérieur  d'une  science  constituée  (1).  Et  lors  de  la  confrontation  entre  notion 
technique et notion  usuelle, l'une d'elles  -  celle qui compte pour l'auditoire auquel on s'adresse  - 
pourra jouer par rapport à l'autre le rôle de terme Il. Ce sera le terme technique qui généralement 
jouira de ce privilège. Mais parfois, au contraire, il deviendra terme I, comme dans cet argument 
de Démosthène : 
\bigskip
Bien loin de dire que je ne suis pas soumis à reddition de comptes (comme cet individu le précisait 
en me calomniant), c'est toute ma vie durant, je le reconnais, que j'ai à rendre compte de tout ce 
que j'ai géré ou accompli devant vous comme homme politique (2). 
\bigskip
La  «reddition  de  comptes»,  au  sens  administratif,  technique,  fait  place  à  la  notion  morale  plus 
générale, plus essentielle et c'est en se basant sur ce dernier sens, considéré comme ternie II, que 
se déroulera en grande partie la suite de l'argumentation. 
\bigskip
Beaucoup d'antithèses sont des applications de la définition dissociative en ce qu'elles opposent au 
sens normal, que l'on pourrait croire unique, un sens qui serait plutôt celui d'un terme II. Ainsi, de 
cette antithèse spéciale que l'on appelait énantiose et dont Vico prend l'exemple chez Cicéron : 
\bigskip
C'est là non loi écrite, mais naturelle (3) » 
\bigskip
(1) Cf. § 50 : Identité et définition dans l'argumentation. 
(2) Démosthène,Harangues et plaidoyers politiques, t. IV: Sur la couronne, 
(3) Vico, Instituzioni oratorie, p. 150 (Cicéron, Pro Milone). 
\bigskip
La définition est toujours un choix (4). Ceux qui y procèdent, surtout s'il s'agit de définition 
dissociative, prétendront généralement avoir dégagé le vrai, l'unique sens de la notion, tout au 
moins le seul raisonnable, ou le seul correspondant à un usage constant. Telle Simone Weil : 
\bigskip
On  ne  peut  pas  trouver  d'autre  définition  au  mot  nation  que  l'ensemble  des  territoires 
reconnaissant l'autorité d'un même État (1). » 
\bigskip
(4) Cf. § 50 -. Identité et définition dans l'argumentation. 
(1) S. Weil, L'enracinement, p. 90. 
\bigskip
P 595 : « ou tel Schopenhauer : 
\bigskip
L'art a comme but de faciliter la connaissance des Idées du inonde (au sens platonicien, le seul que 
j'admette pour le mot Idée) (2) ... 
\bigskip
\bigskip
\bigskip
\bigskip
319 
\bigskip
Cette prétention est liée au fonctionnement du discours non formel. Elle ne sera pas étrangère à 
ceux  mêmes  qui  dénoncent  le  rôle  abusif  que  joue  la  définition  de  certains  termes  :  Crawshay 
Williams estime que les discussions sur le sens du mot « good » sont oiseuses car : 
\bigskip
Dans la mesure où le mot « bien »  a un sens objectif raisonnable en usage », il semble que l'on 
doive l'identifier avec le bonheur et le bien-être des gens, et d'autant de gens qu'il est possible (3). 
\bigskip
Il arrive que l'on fasse appel, pour justifier la définition, à l'étymologie, savante ou populaire : on 
proposera ainsi un usage de la notion que l'on prétend être primitif, authentique, c'està-dire réel, 
et que l'on dégage des falsifications ultérieures. jean Paulhan a eu avec raison l'attention attirée sur 
ce recours à l'étymologie qui lui a inspiré des remarques pertinentes (4). » 
\bigskip
(2) Schopenhauer, M. Brockhaus, vol. 3 : Die Welt als Wille und vorstellung, Zweiter Band, Kap. 
34, p. 466. 
(3) R. Crawshay-Williams, The comforts of unreason, p. 125.  
(4) J. Paulhan, La preuve par l'étymologie. 
\bigskip
P  595-596 :  « Très  proche  de  l'argumentation  par  l'étymologie  est  l'argumentation  fondée  sur  la 
syntaxe, comme dans ce passage de Sartre : 
\bigskip
Le soi ne saurait être une propriété de l'être-en-soi. Par nature, il est un réfléchi, comme l'indique 
assez  la  syntaxe  et,  en  particulier,  la  rigueur  logique  de  la  syntaxe  latine  et  les  distinctions 
strictes que la grammaire établit entre l'usage du ejus et celui du sui. ... Il indique un rapport du 
sujet  avec  lui-même  et  ce  rapport  est  précisément  une  dualité,  mais  une  dualité  particulière 
puisqu'elle exige des symboles verbaux particuliers (1 ). » 
\bigskip
(1) J.-P. Sartre, L'être et le néant, pp. 118-119. 
\bigskip
P 596 : « Et très voisine aussi, sera l'argumentation consistant à faire état d'institutions primitives 
ou  de  pratiques  élémentaires  pour  donner  à  des  concepts  actuels  leur  véritable  signification,  qui 
n'est point celle du commun (2). 
\bigskip
On  connaît  l'usage  intempéré  que  les  existentialistes,  tant  français,  qu'allemands,  ont  fait  de 
l'étymologie  pour  appuyer  leurs  thèses.  Il  arrive  pourtant  que,  dans  certaines  circonstances,  un 
auteur qui volontiers s'appuie sur l'étymologie, rejette explicitement ce lien entre le langage et le 
réel : 
\bigskip
Ces nécessités de la syntaxe nous ont obligé jusqu'ici à parler de la «conscience non positionnelle 
de  soi  ».  Mais  nous  ne  pouvons  user  plus  longtemps  de  cette  expression  où  le  «  de  soi  »  éveille 
encore l'idée de connaissance. (Nous mettrons désormais le «de» entre parenthèses. pour indiquer 
qu'il ne répond qu'à une contrainte grammaticale) (3. 
\bigskip
Pour  justifier  ce  rejet  de  la  contrainte  syntaxique,  invoquée  ailleurs,  l'auteur  aurait  sans  doute 
recours à une nouvelle dissociation, grammaire rejetant dans la grammaire, ici terme I, ce qui ne 
\bigskip
          syntaxe 
\bigskip
correspond  pas  à  la  réalité  philosophique  telle  qu'il  la  conçoit.  Le  recours  à  l'étymologie  ne  se 
bornera pas à retrouver le «bon » sens d'un mot, à le dissocier comme terme II, en liant ce recours, 
quoi qu 1 on en ait, à l'idée d'un monde qui dégénère. Parfois l'accent sera mis sur le passage d'un 
terme à un autre. Ainsi Alain : 
\bigskip
Il n'y a point du tout de pensée sans culture, et non plus sans culte, car c'est le même mot (4). » 
\bigskip
\bigskip
\bigskip
\bigskip
320 
\bigskip
(2)  EX.  G.  Bataille  :  Le  temps  de  la  révolte  (II),  Critique  56,  p.  33,  au  sujet  de  l'amok  et  de  la 
souveraineté authentique. 
(3) J.-P. Sartre, L'être et le néant, p. 20. 
(4) Alain, Histoire de mes pensées, p. 217, signalé par J. Paulhan, La preuve par l'étymologie, p. 17. 
\bigskip
P  597 :  « Reconnaître  le  rapport  entre  culte  et  culture  est  présenté  comme  la  découverte  d'une 
vérité; l'évolution sémantique est considérée comme l'œuvre oubliée d'une humanité raisonnable, 
un chemin, qui nous avait été caché par le voile de l'ignorance et qu'il faut retrouver. 
\bigskip
Parce  qu'elle  est  technique,  ou  parce  qu'elle  est  présentée  comme  seule  valable,  ou  parce  qu'elle 
s'insère  dans  un  ensemble  de  couples  philosophiques  liés  entre  eux,  la  dissociation  des  notions 
tend à rendre leur sens plus précis. Mais cet effort de précision ne réussit que dans la mesure où 
l'on se place à l'intérieur du cadre technique, où l'on fait table rase des autres acceptions, où l'on 
adhère à un système dans son entier. Pour celui au contraire qui ne se confine pas dans ces limites, 
la  définition  dissociative  introduit,  le  plus  souvent,  une  nouvelle  possibilité  d'utilisation  de  la 
notion  primitive,  qui  vient  s'ajouter  aux  usages  antérieurs  et  rend  par  là  même  la  notion  plus 
confuse. 
\bigskip
Ajoutons  que  la  définition  dissociative  d'une  notion  peut  consister  à  affirmer  qu'elle  est 
irrémédiablement  confuse,  que  son  usage  univoque  n'est  qu'illusion,  terme  I,  usage  partiel, 
momentané, et que pour résoudre les incompatibilités que ces aspects du terme I ne manquent pas 
de susciter, il n'est d'autre recours que de les distinguer soigneusement d'un terme II, qui serait la 
notion réelle, essentielle, insaisissable directement dans sa plénitude et sa confusion. 
\bigskip
§ 96. LA RHETORIQUE COMME PROCEDE 
\bigskip
Les  dissociations  ne  portent  pas  seulement  sur  les  notions  utilisées  dans  l'argumentation,  mais 
aussi  sur  le  discours  lui-même,  car  l'auditeur  pratique  à  son  sujet,  soit  spontanément,  soit  parce 
qu'il y est invité, des dissociations qui sont d'une importance capitale. » 
\bigskip
P 597-598 : « Un procédé est une manière d'opérer pour obtenir un certain résultat, tel le procédé 
de  fabrication,  moyen  technique  pour  confectionner  un  produit.  Ce  qui  se  présente  d'emblée 
comme moyen, comme procédé, est apprécié selon son efficacité, et à sa juste valeur. Mais il arrive 
très  souvent  que  le  terme  «  procédé  »  soit  disqualifiant,  qu'il  désigne  le  terme  1  d'un  couple 
philosophique et soit synonyme de fausse apparence. On entend dénoncer de cette façon ce qui se 
prétend conséquence naturelle d'un état de choses et ne serait en réalité que feinte, artifice, moyen 
imaginé en vue d'une fin, telles les larmes insincères, ou les compliments excessifs, procédés pour 
apitoyer ou pour flatter. » 
\bigskip
P 598 : « L'argumentation destinée à  autrui, l'éloquence sous toutes ses formes, eut à souffrir de 
cette  disqualification  et  y  est  constamment  exposée.  Celle-ci  peut  atteindre  tel  argument,  tel 
discours  particulier,  l'art  oratoire  tout  entier.  Les  qualifier  de  rhétorique  suffit  souvent  pour 
enlever aux énoncés leur efficacité. Beaucoup sont comme ces actes dont les effets ne se produisent 
que lorsque ce n'est pas en vue de ces effets qu'ils ont été réalisés (1). 
\bigskip
Traités  de  procédés  oratoires  ou  rhétoriques,  les  moyens  de  persuasion  sont  disqualifiés  comme 
artificiels, formels, verbaux, -nous retrouvons ici les termes I caractéristiques des couples 
artificiel forme verbal. Cette dévaluation est telle que, au discours réfléchi, prémédité, mais perçu 
naturel    fond    réel 
\bigskip
comme  procédé,  on  préférera  le  discours  spontané,  non  apprêté,  quelles  que  soient  ses 
imperfections. » 
\bigskip
(1) Cf. § 92 : Le rôle des couples philosophiques et leurs transformations. 
\bigskip
\bigskip
\bigskip
321 
\bigskip
 
P 598-599 : « Ne petit constituer un procédé ce qui est le résultat d'une poussée irrésistible. Aussi 
l'écrivain,  le  poète,  l'orateur  se  présentera-t-il  comme  étant  sous  l'emprise  d'une  Muse  qui 
l'inspire, d'une indignation qui l'anime : il sera le porte-parole d'une force qui le domine et qui lui 
dicte son langage. Cette vision romantique traduit au moyen d'un cliché, bien éculé aujourd'hui, ce 
que les maîtres de style et les grands orateurs, du pseudo-Longin àBossuet, ont toujours souligné : 
l'éloquence  la  plus  efficace  est  celle  qui  semble  être  la  conséquence  normale  d'une  situation. 
Bossuet paraphrasant saint Augustin, dira : 
\bigskip
... l'éloquence, pour être digne d'avoir quelque place dans les discours chrétiens, ne doit pas être 
recherchée  avec  trop  d'étude.  Il  faut  qu'elle  vienne  d'elle-même,  attirée  par  la  grandeur  des 
choses, et pour servir d'interprète à la sagesse qui parle (1). » 
\bigskip
(1) Bossuet, Sermons, Vol. II : Sur la parole de Dieu, p. 151. 
\bigskip
P  599 :  « Sera  perçu  comme  procédé  le  discours  ressenti  comme  ne  dérivant  pas  de  son  objet. 
Quand  les  auditeurs  communient  avec  l'orateur  dans  le  respect  ou  l'admiration  des  valeurs 
glorifiées, lors d'un discours épidictique, celui-ci sera rarement perçu comme procédé; mais il n'en 
sera pas de même pour des tiers non intéressés à ces valeurs. « Ce ne sont que des mots » est une 
accusation qu'on lance aux autres quand ils exaltent ce qui nous paraît creux, parce que ce ne sont 
pas nos valeurs (2). Lorsque Chaignet, après tant de ses prédécesseurs, répète : 
\bigskip
C'est le naturel qui persuade, tandis que l'artifice de composition et d'expression semble, quand il 
est  aperçu,  un  piège  tendu  à  la  bonne  foi  de  l'auditeur,  qui  s'indigne  de  cette  supercherie  et  en 
éprouve un mécontentement qui ne favorise pas la persuasion (3) 
\bigskip
il a incontestablement raison - notons en passant les termes de «piège », « artifice », « supercherie 
»,  caractéristiques  de  ce  que  l'on  transforme  en  terme  1  -  mais  avec  cette  restriction  que  très 
souvent, la qualification de procédé résulte d'un désaccord sur le fond. C'est parce que l'auditeur 
ne  s'imagine  pas  que  l'on  puisse  être  ému  devant  l'évocation  de  certaines  valeurs  qu'il  fait  de 
l'expression de cette émotion une feinte, un piège tendu à autrui. » 
\bigskip
(2) Cf. J. Paulhan, Les fleurs de Tarbes, p. 84.  
(3) A. Ed. Chaignet, La rhétorique et son histoire, p. 455. 
\bigskip
P 599-600 : « L'impression de procédé peut cependant être ressentie, même en cas d'accord sur les 
valeurs,  quand  l'orateur  semble  adopter  des  règles  et  des  techniques  qui,  de  par  leur  nature 
uniforme,  ou  recherchée,  ne  paraissent  pas  se  mouler  tout  naturellement  sur  l'objet.  On  ne 
trouvera  rien  à  redire  si,  dans  une  revue  de  propagande  suisse  (1)  vantant  les  différents  trajets 
touristiques,  on  lit  la  description  du  funiculaire  le  plus  long,  ou  le  plus  audacieux,  mais  que  le 
même bulletin contienne vingt réclames, chacune débutant par un autre superlatif, et la perception 
du  procédé  devient  d'un  comique  irrésistible.  Le  type  même  du  procédé,  ce  sont  les  fausses 
fenêtres, inadéquates au réel puisqu'elles n'existent que pour la symétrie: 
\bigskip
Ceux qui font les antithèses en forçant les mots sont comme ceux qui font de fausses fenêtres pour 
la symétrie : leur règle n'est pas de parler juste, mais de faire des figures justes (2). 
\bigskip
Sans que les moyens utilisés pour persuader aient rien de mécanique, ou de forcé, on de factice, la 
simple  présence  de  schèmes  argumentatifs,  de  techniques  de  persuasion  transposables, 
théoriquement, à d'autres discours, peut suffire à suggérer l'accusation de procédé. » 
\bigskip
(1) La Suisse, août-septembre 1049. 
(2) Pascal, Œuvre, Bibl. de la Pléiade ; Pensées, 49 (127), p. 834 (27 éd. Brunschvicg). 
\bigskip
\bigskip
\bigskip
322 
\bigskip
 
P  600-601 :  « Pour  que  celle-ci  puisse  être  soutenue,  il  faut  que  la  technique  argumentative, 
disqualifiée  comme  procédé,  ne  puisse  être  interprétée  mieux  encore,  comme  correspondant 
exactement  à  la  nature  même  des  choses;  or  c'est  souvent,  à  cause  de  l'ambiguïté  des  situations 
argumentatives, une question malaisée à résoudre. Lorsque les Anciens qualifiaient de couleur une 
interprétation  de  la  réalité,  favorable  à  la  thèse  que  l'on  défend  (3),  ils  supposaient  une  réalité 
objective  des  faits,  habillée,  modifiée,  par  l'orateur  -  la  terminologie  même,  ici  encore,  rappelle 
celle  bien  connue  des  termes  I  (4).  Mais  l'existence  de  cette  réalité  objective,  qui  ne  coïnciderait 
pas exactement avec l'interprétation proposée, n'est pas prouvée. Un procédé facile à déceler, non 
seulement  serait  peu  efficace,  mais  ne  servirait,  tel  un  mensonge  patent,  qu'à  confondre  son 
auteur. La rançon pourtant de la difficulté qu'il y a à dépister un procédé, c'est que tout acte aux 
conséquences favorables à l'agent, est susceptible d'être considéré comme un procédé, ce qui à la 
limite  jettera  un  soupçon  sur  n'importe  quelle  conduite  consciente  (1),  et  explique  pourquoi, 
chacun croyant connaître ses propres mobiles : 
\bigskip
On se persuade mieux, pour l'ordinaire, par les raisons qu'on a soimême trouvées, que par celles 
qui sont venues dans l'esprit des autres (2). » 
\bigskip
(3) Cf. § 30 : L'interprétation des donnée%. 
(4)  Quintilien,  Vol.  II,  liv.  IV,  chap.  IL  §  88,  §  97;  SÉNÈQUE  LE  Rhéteur,  Controverses  et 
suasoires, Introduction, p. IX. 
(1)  Schopenhauer,  éd.  Brockhaus,  vol.  6  :  Parerga  und  Paralipomena,  Zweiter  Band, 
Psychologische Remerkungen, § 340, p. 637. 
(2) Pascal, Bibl. de la Pléiade, Pensées, 43 (201 *), p. 833 (10 éd. Brunschvicg). 
\bigskip
P 601 : « La dissociation ne s'opère donc que pour lever une incompatibilité. Elle implique que l'on 
a - sur un plan qui peut d'ailleurs varier -une conception du réel, critère qui permet de déceler le 
procédé  («  arguments  véritables  »,  «  réalité  des  sentiments  de  l'orateur  »,  «réalité  des  faits 
énoncés »). Une conception de la réalité et une conception du procédé s'impliquent mutuellement; 
comme dans toute dissociation, il n'y a pas de terme I sans terme II. Mais il ne faut pas négliger le 
fait que tout ce qui favorise la perception comme procédé, l'aspect mécanique, outrancier, abstrait, 
codifié, formel, du discours suggérera la recherche d'une réalité qui s'en dissocierait. 
\bigskip
Comment réagir contre la disqualification du discours comme procédé, ou mieux encore comment 
la prévenir ? » 
\bigskip
P 601-602 : « En affirmant, avons-nous vu, que le discours est la conséquence d'un fait. Mais aussi 
par une série de techniques dont les unes tendent surtout à prévenir l'évocation d'une dissociation, 
les autres à fournir des indices garantissant que celle-ci n'est pas de mise. » 
\bigskip
P  602 :  « L'adéquation  du  style  à  l'objet  du  discours,  telle  que  la  conçoit  l'auditeur,  évite  les 
dissociations immédiatement à craindre : 
\bigskip
Ce  qui  contribue  à  persuader,  c'est  le  style  propre  au  sujet;  dans  ce  cas,  l'esprit  de  l'auditeur 
conclut  inexactement  que  l'orateur  exprime  la  vérité  parce  que  dans  de  telles  circonstances  les 
hommes sont animés de sentiments qui semblent être les siens; et même s'il n'en est pas ainsi, les 
auditeurs pensent que les choses sont telles que le dit l'orateur (1). 
\bigskip
Certains orateurs ou écrivains, pour mieux souligner ce que leur propre attitude a de sérieux et de 
sincère, l'opposent à ce qui serait procédé. Ainsi, Mirabeau, dans son discours sur la contribution 
du quart : 
\bigskip
\bigskip
\bigskip
\bigskip
323 
\bigskip
 
\bigskip
Eh! Messieurs, à propos d'une ridicule motion du Palais-Royal, d'une risible insurrection... vous 
avez  entendu  naguère  ces  mots  forcenés  :  Catilina  est  aux  portes  de  Rome,  et  l'on  délibère!  Et 
certes, il n'y avoit autour de nous ni Catilina, ni périls, ni factions, ni Rome ... Mais aujourd'hui la 
banqueroute, la hideuse banqueroute est là (2) ... 
\bigskip
Simone Weil évoque semblablement le risque de voir prendre une formule pour de la propagande : 
\bigskip
Discréditer de tels mots [la spiritualité du travail] en les lançant dans le domaine publie sans des 
précautions infinies serait faire un mal irréparable... Ils ne doivent pas être un mot d'ordre... La 
seule  difficulté  c'est  la  méfiance  douloureuse  et  malheureusement  trop  légitime  des  masses,  qui 
regardent toute formule un peu élevée comme un piège dressé pour les duper (3). 
\bigskip
Cette technique, par là même qu'elle fait allusion à l'existence de procédés, n'est pas sans danger, 
surtout lorsque l'adéquation de l'énoncé au réel n'est pas fortement garantie par ailleurs. Pourtant 
l'orateur peut parfois prendre ce risque au maximum. » 
\bigskip
(1) Aristote, Rhétorique, III, chap. VII, 4, 1408 a. 
(2) Mirabeau, L'Aîné, Collection complète des Travaux à l'Assemblée nationale, vol. 2 : Discours 
sur l'établissement de la contribution patriotique, pp. 186-187. 
(3) S. Weil, L'enracinement, pp. 88-89. 
\bigskip
P 603 : « Ainsi P.-H. Spaak, traitant de «précaution oratoire » un hommage à l'Amérique inséré au 
milieu d'un discours en faveur de l'Europe suscite volontairement la dissociation procédé ; tandis  
\bigskip
que  la  chaleur  de  l'hommage  incite  à  la  prendre  pour  réalité,  la  dissociation  permet  d'éviter 
l'accusation d'américanophilie. 
\bigskip
Un  des  avis  donné  avec  le  plus  d'insistance  par  les  maîtres  de  rhétorique  de  l'Antiquité  était  de 
faire l'éloge des qualités oratoires de l'adversaire et de cacher, de minimiser les siennes propres (1). 
Conseil suivi par Antoine : 
\bigskip
Je ne suis pas un orateur, comme l'est Brutus (2)  
\bigskip
et que ne négligeait pas Bismarck : 
\bigskip
D'ailleurs, Messieurs, l'éloquence n'est point mon affaire... je ne suis pas un orateur (dénégations 
sur tous les bancs), un avantage que j'accorde volontiers à l'orateur qui m'a précédé (3). 
\bigskip
Il est bon, non seulement de louer en paroles l'éloquence de l'adversaire, mais encore de ne jamais 
réfuter  ses  arguments  de  manière  telle  qu'il  paraisse  médiocre  avocat  (4).  Si  la  trop  grande 
réputation d'éloquence est un danger (5), et surtout la réputation d'habileté, on pourra cependant 
prendre  les  devants,  en  montrant  que  la  perte  de  persuasion  qui  en  résulte  étant  inévitable,  il 
faudra en tenir compte : 
\bigskip
...  ce  sont  mes  discours  mêmes  que  Lysimakhos  a  attaqués  calomnieusement  afin  que,  si  l'on 
trouve  que  je  suis  éloquent  '  je  paraisse  mériter  les  reproches  qu'il  m'adresse  à  propos  de  mon 
habileté, et que, si mes discours ne répondent pas à l'attente qu'il a excitée en vous, vous pensiez 
que mes actions sont plus mauvaises encore (6). 
\bigskip
(1) Quintilien, Vol. II, liv. IV, chap. I, § 8; vol. IV, liv. XI, chap. I, §§ 15, 17, 19. 
(2) Shakespeare, Julius Caesar, acte III, sc. II. 
(3) Cité par H.  Wunderlich, Die Kunst der Rede in ihren Hauptzügen an den Retient Bismarcks 
dargestellt, p. 1 
\bigskip
   réalité 
\bigskip
\bigskip
\bigskip
\bigskip
\bigskip
\bigskip
\bigskip
\bigskip
\bigskip
\bigskip
\bigskip
324 
\bigskip
(4) Cf QUINTILIEN, Vol. 11, liv. V, chap. XIII, § 37. 
(5) Cf: Cicéron, De Oratore, liv. II, § 4, au sujet de l'ignorance affichée par Crassus et par Antoine 
; Richard D. D. Whately, Elements of Rhetoric, Part II, chap. III, pp. 154-156. 
(6) ISOCRATE, Discours, t. III : Sur l'échange, § 16. 
\bigskip
P  604 :  « Tout  ce  qui  dénonce  le  talent  est  à  proscrire  si  l'on  veut  éviter  la  dissociation.  Rien  de 
plus  étudié  pour  se  prémunir  contre  elle  que  le  naturel  tant  vanté  par  les  classiques  ;  certains 
textes du chevalier de Méré sont révélateurs à cet égard : 
\bigskip
Le  bon  art  qui  fait  qu'on  excelle  à  parler,  ne  se  montre  que  sous  une  apparence  naturelle  :  il 
n'aime que la beauté simple et naïve : et quoi qu'il travaille pour mettre ses agrémens dans leur 
jour, il songe principalement à se cacher... je trouve que le plus parfait est celuy qui se remarque 
le moins, et quand les choses sentent l'art et l'étude, on peut conclure que ceux qui les disent n'ont 
guère de tous les deux, ou qu'ils ne scavent pas s'en servir (1). » 
\bigskip
Les  éléments  qui  peuvent  être interprétés  comme  indices  de  spontanéité  seront  particulièrement 
efficaces pour garantir l'adéquation au réel et aussi favoriser la persuasion : 
\bigskip
Il  échappe  à  une  jeune  personne  de  petites  choses  qui  persuadent  beaucoup,  et  qui  flattent 
sensiblement celui pour qui elles sont faites. Il n'échappe resque rien aux homines ; leurs caresses 
sont volontaires ; ils parlent, ils agissent, ils sont empressés, et persuadent moins (2). 
\bigskip
Toutefois ces indices eux-mêmes peuvent être considérés comme procédé et il est difficile, en face 
d'un  texte,  de  déterminer  sa  spontanéité  :  les  épanchements  romantiques  au  clair  de  lune  sont 
devenus rapidement un cliché difficile à prendre au sérieux. 
\bigskip
J.  Paulhan  a  finement  décrit  les  offensives  des  terroristes  et  des  contre-terroristes  en  littérature 
(3).  Il  montre  qu'il  n'y  a  pas  de  littérature  sans  rhétorique,  par  laquelle  il  entend  un  art  de 
l'expression.  Mais  les  moyens  de  cet  art  perdent  de  leur  efficacité  au  fur  et  à  mesure  qu'ils  sont 
perçus  comme  procédé.  L'argumentation  n'échappe  à  cette  dévaluation,  que  dans  la  mesure  où 
l'orateur  suggère,  des  faits  et  de  lui-même,  une  image  telle  que  l'on  n'est  pas  incité  à  faire  une 
dissociation procédé. 
\bigskip
réalité 
\bigskip
\bigskip
(1) Chevalier de Méré, Œuvres complètes, t. 1 : Les conversations (3e), p. 47. 
(2) La Bruyère, Bibl. de la Pléiade, Caractères, Des femmes, 14, p. 130. 
(3) J. Paulhan, Les fleurs de Tarbes. 
\bigskip
P  605 :  « Indices  de  maladresse,  indices  de  sincérité,  seront  pareillement  utiles  pour  éviter  la 
dissociation procédé et se confondent d'ailleurs dans bien des cas. Ils agiront tantôt par leur seule 
\bigskip
réalité 
\bigskip
\bigskip
présence, tantôt seront mis en vedette par l'orateur ou par un tiers. Toutes les imperfections qui 
semblent, à première vue, nuisibles à l'effet argumentatif, peuvent, par ce biais, favoriser celui-ci, 
et  l'un  des  avantages  de  l'improvisation  serait  de faire  naître,  de façon  spontanée,  des  indices  de 
maladresse ou de sincérité. 
\bigskip
Ces  indices  concernent  non  seulement  l'expression  formelle  mais  encore  la  nature  même  des 
arguments.  Le  choix  d'arguments  irrelevants  au  débat  mais  touchant  de  près  les  émotions  de 
l'orateur  sera,  aussi  bien  que  le  son  de  sa  voix,  indice  de  sincérité.  Le  renoncement  à  certaines 
techniques,  le  fait  de  produire  des  arguments  mal  adaptés  à  l'auditoire,  peut  n'être  pas  sans 
efficace,  et  l'argument  sur  mesure  n'est  pas  toujours,  en  fin  de  compte,  le  meilleur.  Tout  comme 
Montaigne reconnaît la sincérité de Tacite, à ce que ses récits « ne s'appliquent pas toujours aux 
\bigskip
\bigskip
\bigskip
325 
\bigskip
conclusions de ses jugements » (1), Pascal voit la preuve de la sincérité des Évangélistes dans ce 
qui semble imperfections de jésus : 
\bigskip
Pourquoi le font-ils faible dans son agonie [Luc XXII, 41-44] ? Ne savent-ils pas peindre une mort 
constante ? Oui, car le même saint  Lue peint celle de saint  Etienne plus forte que celle de Jésus 
Christ [Act. VII, 59] (2). » 
\bigskip
(1) Montaigne, Bibl. de la Pléiade, Essais, liv. III. chap. VIII, p. 913. 
(2) Pascal, Bibl. de la Pléiade, Pensées, 741 (49), p. 1063 (800 éd. Brunschvicg). 
\bigskip
P  605-606 :  « Les  indices  de  passion  peuvent  donner  lieu  à  des  figures :  l'hésitation  (3), 
l'hyperbate  ou  inversion,  qui  substitue  à  l'ordre  naturel  de  la  phrase  un  ordre  né  de  la  passion  ; 
l'absence  de  liaisons,  le  mélange  de  figures,  comme  témoignage  de  la  passion,  ont  été  fort  bien 
décrits par le pseudo-Longin (4). Étudiant les  dégradations que l'émotion inspire à la langue, un 
psychologue averti, A. Ombredane signale que : 
\bigskip
Toutes ces dégradations de la langue peuvent être recherchées par procédé  littéraire et nombre 
d'entre elles s’intègrent dans la stylistique : répétition, litanie, appauvrissement du vocabulaire, 
hyperbole, suppression du verbe, substitution de la   juxtaposition à la subordination, 
suppression des copules, rupture de la construction, etc. (1). » 
\bigskip
(3) Quintilien, Vol. III, liv. IX, Chap. 11, § 19. 
(4) Longin, Traité du sublime, chap. XVII, pp. 102-104. 
(1) A. Ombredane, L'aphasie et l'élaboration de la pensée explicite, P. 268. 
\bigskip
P 606 : « On retrouve ici bon nombre de traits, tels la parataxe, l'hyperbole, déjà rencontrés et qui, 
outre leur rôle argumentatif, peuvent être indices de sincérité. 
\bigskip
Tout ce qui fournit un argument contre la thèse que défend l'orateur, y compris la réfutation de ses 
propres hypothèses (2), se transforme en indices de sincérité, de loyauté, et augmente la confiance 
des  auditeurs.  Tout  énoncé  pénible,  les  confessions  notamment,  sera  présumé  sincère  (3).  Tout 
énoncé aussi qui menace d'aliéner l'auditoire. Et ici encore nous aurons des figures de la sincérité: 
la licence (4) et la pseudo-licence (5), parfois appelée astéisme (6). » 
\bigskip
(2) Aristote, Topiques, liv. VIII, chap. 1, 156 b. 
(3) Cf. E. Dupréel, Essais pluralistes, p. 114 (La deuxième vertu du XIXe siècle). 
(4) Rhétorique à Herennius, liv. IV, ~ 48. 
(5) Ibid., § 49. 
(6)  Cf.  Baron,  De  la  Rhétorique,  p.  365  ;  3.  Paulhan,  Les  figures  ou  la  rhétorique  décryptée, 
Cahiers du Sud, p. 371. 
\bigskip
P  606-607 :  « Puisque  toute  technique  qui  semble  contraire  au  but  à  atteindre  produit  grande 
impression, on n'hésitera pas à en user comme d'un procédé suprême : 
\bigskip
La  Vie-Humaine,  écrit  Gracian,  est  un  combat  contre  la  malice  de  l'Homme  même.  L'Homme 
adroit  y  employe  pour  armes  les  stratagèmes  de  l'intention...  Et  puis,  quand  son  artifice  est 
connu, il rafine la dissimulation, en se servant de la vérité même, pour tromper. Il change de jeu 
et de baterie, pour changer de ruse. Son artifice est de n'en avoir plus, et toute sa finesse est de 
passer de la dissimulation précédente à la candeur. Celui, qui l'observe, et qui a de la pénétration, 
connoissant l'adresse de son rival, se tient sur ses gardes, et découvre les tenèbres revêtues de la 
lumière. Il déchifre un procédé d'autant plus caché, que tout y est sincère (1). » 
\bigskip
(1) B. Gracian, L'homme de cour, p. 12. 
\bigskip
\bigskip
\bigskip
326 
\bigskip
 
\bigskip
 réalité 
\bigskip
\bigskip
P  607 :  « Ainsi  l'aveu,  sur  un  ton  ironique,  d'un  feint  amour,  donne  au  héros  des  chances  de  se 
faire  prendre  au  sérieux.  Sensibilité  moderne,  nous  explique  le  romancier  (2),  mais  surtout 
mécanisme normal de la persuasion. 
\bigskip
Quand la conduite est-elle sincère ? Quand n'est-elle qu'un procédé qui n'a de la sincérité que les 
apparences ? A défaut de critère indiscutable la dissociation procédé peut se continuer 
\bigskip
indéfiniment et contradictoirement : c'est l'emploi de cette opposition qui caractérise, semble-t-il, 
une des plus anciennes techné, celle attribuée à Corax (3), et qu'un échange de lettres parues il y a 
quelques années dans le New York Herald Tribune (4) permettra d'illustrer à suffisance. 
\bigskip
Un  correspondant  avait  adressé  au  journal  une  lettre,  d'allure  profasciste,  et  insultante  pour  les 
États-Unis.  Plusieurs  lecteurs  la  commentèrent,  l'un  d'entre  eux  y  voyant  une  forme  subtile  de 
propagande  communiste.  Mais,  se  demandent  d'autres  lecteurs,  ne  serait-ce  pas  un  fasciste  qui 
écrit une lettre dont il souhaite qu'on l'attribue à la propagande communiste pour exciter l'opinion 
contre  elle  ?  On  pourrait  continuer  le  jeu  des  interprétations  pour  attribuer,  alternativement,  à 
l'auteur, des visées procommunistes ou profascistes. » 
\bigskip
(2) J.-L. Curtis, Chers corbeaux, p. 96. 
(3)  Aristote,  Rhétorique,  II,  chap.  24,  1402  a.  Cf.  0.  Navarre,  Essai  sur  la  Rhétorique  grecque 
avant Aristote, pp. 16 et suiv. 
(4) Du 24 avril au 4 mai 1948 (Éd. de Paris). 
\bigskip
P  607 :  « Aristote  classe  ce  procédé  parmi  les  enthymèmes  apparents  et  en  fournit  l'exemple 
suivant : 
\bigskip
Si  un  homme  ne  donne  pas  prise  à  l'accusation  dirigée  contre  lui,  si  par  exemple,  un  homme 
faible est poursuivi pour sévices, sa défense sera qu'il n'est pas vraisemblable qu'il soit coupable; 
mais si l'inculpé donne prise à l'accusation, si, par exemple, il est fort, sa défense sera qu'il n'est 
pas vraisemblable qu'il soit coupable, parce qu'il était vraisemblable qu'on le croie coupable (1). » 
\bigskip
que nous trouvons déjà développé, mais avec beaucoup moins de netteté, dans le Phèdre de Platon 
(2). 
\bigskip
Il s'agit d'arguments qui opposent ce qu'Aristote appelle le vraisemblable absolu, au vraisemblable 
relatif, et qui font état d'une vraisemblance basée sur ce que nous savons du normal, ces éléments 
variant  constamment  au  cours  de  l'argumentation.  L'agent  qui  pose  un  acte,  ou  argumente,  est 
censé connaître les critères du réel que son auditoire appliquera, et agir en conséquence. 
\bigskip
Ce procédé est souvent utilisé au prétoire, comme dans ce passage d'Antiphon : 
\bigskip
Si  la  haine  que  je  portais  à  la  victime  rend  vraisemblables  les  soupçons  actuels,  n'est-il  pas 
vraisemblable  encore  que,  prévoyant  ces  soupçons  avant  le  crime,  je  nie  sois  bien  gardé  de  le 
commettre ? 
...  Et  ceux  qui  haïssaient  la  victime  autant  que  moi,  -  il  y  en  avait  plus  d'un,  -  n'est-il  pas 
vraisemblable  que  ce  soit  eux,  plutôt  que  moi,  qui  l'ont  assassinée  ?  Pour  eux  nul  doute  que  les 
soupçons  se  portassent  sur  moi,  tandis  que,  moi,  je  savais  bien  que  je  serais  incriminé  à  leur 
place (3).  
\bigskip
Le corax n'est qu'une application de la dissociation procédé au domaine de la conjecture. Il incite à 
\bigskip
         réalité 
\bigskip
\bigskip
\bigskip
327 
\bigskip
accomplir  certains  actes  précisément  parce  qu'il  sont  invraisemblables  et  diminue,  pour  des 
raisons inverses, les chances de voir accomplir les actes vraisemblables (4). Quintilien conseille de 
prendre des précautions quand le juge est notre ami car : il y a des juges sans conscience [à qui il 
arrive] de commettre une injustice pour ne pas sembler le faire (5). » 
\bigskip
(1) Aristote, Rhétorique, II, chap. 24, 1402 a. 
(2) Platon, Phèdre, 273 b-c. ar 0. Navarre, Essai sur 
(3) Antiphon, Première tétralogie, 2. 3 ; 2. 6. Cité par la Rhétorique grecque avant Aristote, p. 
139. 
(4) Cf. Aristote, Rhétorique, I, chap. 12, 1372 a. 
( 5) Quintilien, Vol. II, liv. IV, chap. 1, § 18. 
\bigskip
P 609 : « Le rebondissement du corax peut-il se poursuivre indéfiniment ? Oui. Mais à un moment 
donné  il  deviendra  très  peu  persuasif  et  même  comique,  parce  qu'il  implique  une  excessive 
capacité de prévision. Il ne faut pas oublier d'ailleurs que souvent, s'il s'agit d'un procès, il ne suffit 
pas  de  développer  le  corax  en  fonction  d'une  des  parties  seulement.  Aux  prévisions  de  l'une 
devront correspondre des prévisions inverses attribuées à l'autre. 
\bigskip
Puisque  le  corax  est  lié  à  la  connaissance  que  l'agent  peut  avoir  de  ce  qui  sera  supposé 
vraisemblable,  l'une  des  ripostes  sera  d'alléguer  l'ignorance  des  critères  servant  de  base  à  cette 
détermination, ce qui exclurait l'interprétation de la situation en fonction du procédé. Mais c'est là 
souvent démarche difficile. Le fait seul de n'avoir pu prévoir qu'il y aurait un litige, un problème de 
conjecture qui se poserait, peut cependant suffire à écarter le soupçon. 
\bigskip
Le  corax  est  un  procédé  d'argumentation  typiquement  rhétorique  parce  que  basé  sur  des 
possibilités d'interprétation multiples ; il est propre à un discours non formel et n'est imaginable 
que devant une situation ambiguë. » 
\bigskip
CHAPITRE V L'INTERACTION DES ARGUMENTS 
\bigskip
§ 97. INTERACTION ET FORCE DES ARGUMENTS 
\bigskip
P  610 :  « Nous  avons  insisté,  avant  d'entreprendre  l'étude  analytique  des  arguments,  sur  le 
caractère schématique et arbitraire de celle-ci (1). Les éléments isolés en vue de l'étude forment, en 
réalité, un tout ils sont en interaction constante et cela sur plusieurs plans interaction entre divers 
arguments  énoncés,  interaction  entre  ceux-ci  et  l'ensemble  de  la  situation  argumentative,  entre 
ceux-ci  et  leur  conclusion,  et  enfin  interaction  entre  les  arguments  contenus  dans  le  discours  et 
ceux qui ont ce dernier pour objet. 
\bigskip
Les limites au jeu d'éléments en cause sont de toutes parts indécises. 
\bigskip
En  effet,  la  description  des  arguments  appelés  à  interagir  peut  toujours  être  étendue  dans  une 
double  direction  :  par  une  analyse  plus  poussée  des  énoncés,  analyse  plus  fine  ou  encore 
diversement menée, et par la prise en considération d'un nombre croissant d'arguments spontanés 
ayant le discours pour objet. » 
\bigskip
(1) Cf. § 44 : Généralités. 
\bigskip
P  610-611 :  « Par  ailleurs  les  jugements  admis  qui  déterminent  la  situation  argumentative 
constituent toujours un ensemble aux contours mal arrêtés : extensible, suivant les domaines pris 
en considération ; mouvant, suite aux moments successifs de l'argumentation; morcelable de façon 
variée, au gré des coupes diverses que l'on y pratique. » 
\bigskip
\bigskip
\bigskip
\bigskip
328 
\bigskip
P 611 : « Enfin le discours lui-même, s'il a une unité relativement bien définie dans la plaidoirie de 
l'avocat ou le sermon du prédicateur, peut, dans les débats parlementaires ou familiaux, s'étendre 
sur plusieurs jours et résulter de l'intervention de plusieurs personnes. Il y a plus. Il arrive que la 
thèse en discussion ne soit pas conçue, par les adversaires, de la même façon, que, terme du débat 
pour l'un, elle ne soit pour l'autre qu'une étape vers une conclusion ultérieure ; d'où, le découpage 
de la réalité sur laquelle porte l'argumentation étant différent, un même avis, une même décision, 
dans un sens, ne sont pas l'exact contre-pied de l'avis ou de la décision en sens contraire. Aussi la 
fixation du point à juger est-elle l'une des préoccupations fondamentales dans un débat judiciaire ; 
c'est une tentative pour isoler celui-ci, l'insérer dans un cadre conventionnellement ou légalement 
établi. 
\bigskip
Si  imprécises  que  soient  donc  les  conditions  dans  lesquelles  se  développent  les  phénomènes 
d'interaction,  ce  sont  eux  cependant  qui  déterminent  en  grande  partie  le  choix  des  arguments, 
l'ampleur et l'ordre de l'argumentation. 
\bigskip
Pour  se  guider  dans  son  effort  argumentatif,  l'orateur  utilise  une  notion  confuse,  mais 
indispensable semble-t-il, c'est celle de force des arguments. 
\bigskip
Celle-ci  est  certainement  liée  d'une  part,  à  l'intensité  d'adhésion  de  l'auditeur  aux  prémisses,  y 
compris  les  liaisons  utilisées,  d'autre  part,  à  la  relevance  des  arguments  dans  le  débat  en  cours. 
Mais l'intensité d'adhésion, et aussi la relevance, sont à la merci d'une argumentation qui viendrait 
les combattre. Aussi la force d'un argument se manifeste tout autant par la difficulté qu'il y aurait à 
le réfuter que par ses qualités propres. » 
\bigskip
P  611-612 :  « La  force  des  arguments  variera  donc  selon  les  auditoires  et  selon  le  but  de 
l'argumentation. Aristote avait noté que « les exemples appartiennent surtout au genre délibératif, 
tandis que les enthymèmes conviennent plutôt au genre judiciaire » (1) et Whately conseille, selon 
que l'on s'adresse à des esprits désirant s'instruire ou à des adversaires dont il s'agit de réfuter les 
critiques,  d'utiliser  des  arguments  de  cause  à  effet  ou  de  se  servir  d'exemples  (2).  Ces  deux  avis 
reviennent à préconiser l'exemple, c'est-à-dire ce qui est capable de fonder des liaisons nouvelles, 
là où l'on dispose de moins de prémisses. » 
\bigskip
(1) Aristote,  Rhétorique, liv.  III, chap. XVII, § 5, 1418 a; cf. H. Volkmann,  Rhetorik der Griechen 
und Römer, p. 33. 
(2) Richard D. D. Whately Elements of Rhetoric, Part I, chap. 111, § 1, p. 74. 
\bigskip
P 612 : « Le principe majeur, en cette matière, reste toujours l'adaptation à l'auditoire, aux thèses 
qu'il  admet,  en  tenant  compte  de  l'intensité  de  cette  adhésion.  Il  ne  suffit  pas  de  choisir  des 
prémisses sur lesquelles s'appuyer : il faut prendre garde, puisque la force de l'argument tient en 
grande  partie  à  sa  résistance  possible  aux  objections,  à  tout  ce  qu'admet  l'auditoire,  même  à  ce 
dont  on  n'a  aucune  intention  de  faire  usage, mais qui  pourrait  venir  s'opposer  à  l'argumentation 
(3). 
\bigskip
Dans  la  réfutation,  mêmes  conditions.  Le  choix  y  est  en  outre  guidé  par  l'argument  que  l'on 
combat. » 
\bigskip
(3) Cf. 29 La sélection des données et la présence. 
\bigskip
P  612-613 :  « Nous  avons  souvent,  au  cours  de  notre  exposé,  indiqué  à  quelle  réfutation  telle 
argumentation  s'exposait  :  la  liaison,  au  refus  de  la  liaison  (4),  l'exemple,  à  l'exemple  invalidant 
(5), l'analogie, à la prolongation de l'analogie (6), la dissociation, au renversement du couple (7). 
Ces  modes  de  réfutation  ont  l'avantage  de  prétendre  aisément  à  la  relevance;  mais  tous  autres 
modes  de  réfutation  peuvent  être  utilisés.  Toutefois  l'objection  se  maintiendra  volontiers  dans  le 
\bigskip
\bigskip
\bigskip
329 
\bigskip
cadre adopté par l'orateur : on opposera un lieu de la qualité à un lieu de la quantité, à un lieu de 
l'ordre, ceux de l'existant (1), à la coutume, on opposera la coutume d'un autre groupe auquel on 
appartient  également.  Parmi  toutes  les  raisons  qui  pourraient  être  données  de  ne  pas  prendre  le 
deuil d'un parent, tel dira, de préférence, 
\bigskip
dans ma famille il n'est pas d'usage de porter le deuil (2), 
\bigskip
et si l'un évoque la valeur de l'utile, l'autre se fondera sur celle de la justice. » 
\bigskip
(4) Cf. 89 Rupture de liaison et dissociation. 
(5) Cf. § 78 L'argumentation par l'exemple. 
(6) Cf. § 85 Comment on utilise l'analogie. 
(7) Cf. § 92 Le rôle des couples philosophiques et leurs transformations. 
(1) Cf. § 25 : Utilisation et réduction des lieux. 
(2) G. Smets, carnet sociologique, n. 50, Revue de l'Institut de Sociologie, 19.50, 1, pp. 148-149. 
\bigskip
P  613 :  « Étant  donné  la  complexité  des  facteurs  à  prendre  en  considération,  ne  fût-ce  que  pour 
estimer si un argument possède une force quelconque, il est curieux de constater que les auteurs 
de  traités  de rhétorique  affirment  souvent  sans  hésiter,  et  comme incidemment,  que  la force  des 
arguments nous est connue, et basent leurs conseils relatifs à l'ordre du discours, à l'enchaînement 
des répliques, sur le degré de conviction que les arguments ont dû entraîner : 
\bigskip
ce  qu'il  ne  nous  sera  pas  difficile  de  savoir  puisque  nous  savons  ce  qui  la  détermine 
ordinairement (3). 
\bigskip
Illusion pourrait-on croire, après ce que nous venons de dire, et quand on se souvient que Pascal 
fut découragé lorsqu'il se proposa d'examiner les diverses manières d'agréer (4). » 
\bigskip
(3) Rhétorique à Herennius, liv. I, § 10. 
(4)  Pascal,  Œuvre,  Bibl.  de  la  Pléiade,  De  l'esprit  géométrique  et  de  l'art  de  per  suader,  pp.  375-
379. 
\bigskip
P  613-614 :  « La  seule  connaissance  de  réactions  individuelles,  et  les  études  les  plus  poussées  de 
psychologie différentielle ne pourraient en effet suffire à mesurer cette force, car il entre en elle un 
élément normatif qui peut être considéré comme une des prémisses de l'argumentation ou qui tout 
au  moins  est  inséparable  de  la  notion  de  force.  En  effet,  un  argument  fort  est-ce  un  argument 
efficace, qui détermine l'adhésion de l'auditoire, on un argument valable, qui devrait la déterminer 
?  La  force  d'un  argument  constitue-t-elle  une  qualité  descriptive  ou  normative  ?  Et  son  étude 
relève-t-elle de la psychologie individuelle et sociale ou, au contraire, de la logique ? » 
\bigskip
P 614 : « Cette distinction de deux points de vue, fondée sur une dissociation normal, ne peut être  
\bigskip
       norme 
\bigskip
absolue, - car le normal, comme la norme, ne se définissent que par rapport à un auditoire, dont 
les  réactions  fournissent  la  mesure  du  normal  et  dont  l'adhésion  fonde  les  normes  de  valeur. 
Cependant  la  distinction  est  précieuse  quand  ce  sont  les  réactions  d'un  certain  auditoire  qui 
déterminent  le  normal  et  les  conceptions  d'un  autre  qui  fournissent  le  critère  de  la  norme.  La 
supériorité de la norme sur le normal serait corrélative de celle d'un auditoire sur un autre et c'est 
à  cette  hiérarchisation  des  auditoires  que  correspond,  avons-nous  déjà  constaté,  la  distinction 
entre persuader et convaincre (1). En dissociant l'efficacité d'une argumentation de sa validité, on 
parviendra à jeter sur elle la suspicion, à diminuer son efficacité, même reconnue par celui qu'elle 
a réussi à persuader, comme dans ce dialogue entre deux personnages de Sartre : 
\bigskip
HUGO. - Que veux-tu que je pense ? je t'avais dit qu'il était chinois. 
\bigskip
\bigskip
\bigskip
330 
\bigskip
JESSICA. - Hugo ! il avait raison. 
HUGO. - Ma pauvre Jessica ! Qu'est-ce que tu peux en savoir ? lui. JESSICA. - Et toi qu'en sais-tu 
? Tu n'en menais pas large devant lui. 
HUGO. - Parbleu! Avec moi il avait beau jeu. 
JESSICA.  -  Hugo  !  Tu  parles  contre  ton  coeur.  je  t'ai  regardé  pendant  que  tu  discutais  avec 
Hoederer : il t'a convaincu. 
HUGO.  -  Il  ne  m'a  pas  convaincu.  Personne  ne  peut  me  convaincre  qu'on  doit  mentir  aux 
camarades. Mais s'il m'avait convaincu, ce serait une raison de plus pour le descendre parce que 
ça prouverait qu'il en convaincra d'autres (2). » 
\bigskip
(1) Cf. § 6 : Les auditoires d'intérêt philosophique ; persuader et convaincre. 
(2) J.-P. Sartre, Les mains sales, 5e tableau, scène 5, pp. 214-215. 
\bigskip
P  615 :  « Une  conclusion  devant  laquelle  on  ne  veut  pas  s'incliner,  fait  douter  de  la  validité  des 
arguments dont on a soi-même éprouvé l'efficacité. 
\bigskip
L'interaction  entre  le  normal  et  la  norme,  agit  dans  les  deux  sens  :  si  l'efficace,  dans  certaines 
circonstances, fournit le critère du valable, l'idée que l'on se fait du valable ne peut rester sans effet 
sur l'efficacité des techniques visant à persuader et à convaincre. 
\bigskip
Qu'est-ce qui garantit cette validité ? Qu'est-ce qui en fournit le critère ? Le plus souvent c'est une 
théorie de la connaissance, qui consiste en l'adoption de techniques qui se sont révélées efficaces 
dans  divers  domaines  du  savoir,  ou  en  la  transposition  de  techniques  ayant  réussi  dans  un 
domaine privilégié et qui fourniraient un modèle pour d'autres : d'où le conflit bien connu entre la 
reconnaissance  de  méthodologies  multiples,  efficaces  chacune  dans  un  domaine  limité,  et  la 
conception de l'unité de la science, fondée sur une méthodologie idéale, empruntée à une science 
prestigieuse et qui serait applicable à toute science digne de ce nom. Dans ce dernier cas, le critère 
de  l'évidence,  rationnelle  on  sensible,  dispensera  de  la  dissociation  entre  normal  et  normatif  : 
l'évident est simultanément efficace et valable, il convainc parce qu'il doit convaincre. C'est au nom 
de  l'évident,  devenu  le  critère  du  valable,  que  l'on  disqualifiera  toute  argumentation,  puisqu'elle 
s'avère  efficace  sans  fournir  pourtant  de  véritable  preuve  et  ne  peut  donc  relever  que  de  la 
psychologie et non de la logique, même au sens large de ce mot (1). » 
\bigskip
(1) Ch. Perelman, Rhétorique et philosophie, p. 121 (De la preuve en philosophie). 
\bigskip
P 615-616 : « Quelle que soit l'importance de ces prises de position philosophiques, quelle que soit 
la  répercussion  sur  l'appréciation  de  la  force  des  arguments,  de  son  double  aspect,  à  la  fois 
descriptif  et  normatif,  c'est-à-dire  des  notions  d'efficacité  ou  de  validité,  quelle  que  soit  la 
complexité des éléments qui interviennent, une chose est certaine, c'est que, dans la pratique, on 
distingue des arguments forts et des arguments faibles. » 
\bigskip
P 616 : « Notre hypothèse est que cette force est appréciée grâce à la règle de justice: ce qui a pu, 
dans  une  certaine  situation,  convaincre,  paraîtra  convaincant  dans  une  situation  semblable,  ou 
analogue. Le rapprochement entre situations fera l'objet, dans chaque discipline particulière, d'un 
examen et d'un raffinement constants. Toute initiation à un domaine rationnellement systématisé, 
non  seulement  fournit  la  connaissance  des  faits  et  des  vérités  de  la  branche  en  question,  de  sa 
terminologie  propre,  de  la  manière  d'user  des  instruments  dont  elle  dispose,  mais  éduque  aussi 
dans l'appréciation de la force des arguments employés en cette matière. 
\bigskip
La force des arguments dépend donc  largement d'un contexte traditionnel. Parfois l'orateur peut 
aborder  tous  les  sujets  et  se  servir  de  toute  espèce  d'arguments  ;  parfois  son  argumentation  est 
limitée par l'habitude, par la loi, par les méthodes et les techniques propres à la discipline au sein 
de 
le  niveau  de 
\bigskip
laquelle  son  raisonnement  se  développe.  Celle-ci  détermine  souvent 
\bigskip
\bigskip
\bigskip
331 
\bigskip
l'argumentation,  ce  qui  peut  être  considéré  comme  hors  discussion,  ce  qui  doit  être  considéré 
comme irrelevant dans le débat. 
\bigskip
P 616-617 : « Il va de soi que les diverses philosophies, par la détermination de la structure du réel 
et les justifications qu'elles en donnent, par les critères de la connaissance et de la preuve valables, 
par  la  hiérarchie  des  auditoires  qu'elles  adoptent,  influent  sur  la  force  de  tel  ou  de  tel  schème 
argumentatif. Le contexte philosophique donne une force accrue à certains genres d'arguments : le 
réalisme des essences favorisera toutes les formes d'argumentation qui s'appuient sur les essences, 
qu'il s'agisse d'arguments par division, ou par dissociation acte ; une vision de l'univers qui admet  
\bigskip
       essence 
\bigskip
\bigskip
l'existence de degrés de réalité hiérarchisés favorisera l'argumentation par analogie ; l'empirisme 
favorisera  les  arguments  basés  sur  des  faits  présentés  comme  indiscutables,  le  rationalisme 
favorisera  l'argumentation  au  moyen  de  principes,  le  nominalisme  le  recours  au  cas  particulier. 
Mais le philosophe, ne l'oublions pas, se servira, comme chacun de nous, des arguments les plus 
divers, quitte à attribuer à certains, dans son système - et s'il avait à prendre position à leu r égard 
- une situation subordonnée, voire à les ignorer (1). » 
\bigskip
(1)  Cf.  H.  Gouhier,  La  résistance  au  vrai  et  le  problème  cartésien  d'une  philosophie  sans 
rhétorique, dans Retorica e Barocco, pp. 85-97. 
\bigskip
§  98.  L'APPRECIATION  DE  LA  FORCE  DES  ARGUMENTS  FACTEUR 
\bigskip
D'ARGUMENTATION 
P 617 : « La force des arguments peut, elle-même, être utilisée, explicitement ou implicitement, par 
l'orateur ou par les auditeurs, comme facteur argumentatif. D'où richesse accrue des interactions 
dont il faudra tenir compte. 
\bigskip
La  surévaluation  volontaire,  par  l'orateur,  de  la  force  des  arguments  qu'il  propose  tend 
généralement à accroître celle-ci. Présenter une conclusion comme plus certaine qu'elle ne l'est à 
nos  propres  yeux,  c'est  engager  notre  personne,  c'est  user  de  son  prestige,  ajouter  dès  lors  aux 
arguments un argument supplémentaire. Cette démarche est considérée par Bentham comme un 
état mitoyen entre la mauvaise foi et la témérité (2). Sans doute l'orateur peut-il être convaincu par 
d'autres  raisons  que  celles  qu'il  trouve  dans  son  argumentation,  et  ses  mobiles  peuvent-ils  être 
honorables. Mais il devra éventuellement rendre compte de son attitude. » 
\bigskip
(2) Bentham, Œuvres t. I . Traité des sophismes politiques, p. 503. 
\bigskip
P 617-618 : « Une autre façon de surévaluer consiste à étendre les accords particuliers, obtenus en 
cours de discussion, sans que l'interlocuteur ait donné son adhésion explicite. Schopenhauer voit 
un artifice dans le fait de traiter l'adhésion aux exemples comme ayant entraîné l'accord au sujet de 
la  généralisation  qui  en  dérive  (1),  ou  encore  dans  le  fait  de  considérer  comme  acquise  une 
conclusion contestable (2). » 
\bigskip
(1) Schopenhauer, éd. Piper, vol. 6 : Eristische Dialektik, p. 413 (Kunstgriff 11). 
(2) Ibid., p. 414 (Kunstgriff 14). 
\bigskip
P 618 : « En fait, dans tout discours qui ne se déclare pas explicitement rhétorique, on surestime la 
force des arguments avancés. C'est particulièrement le cas dans l'argumentation quasi logique qui 
se donne pour démonstrative, alors qu'elle ne le serait que moyennant des prémisses sur lesquelles 
une contestation est possible. Et l'artifice dénoncé par Schopenhauer, qui consiste à conclure sans 
avoir  obtenu  l'adhésion  à  toutes  les  prémisses  (3),  n'est  que  la  forme  grossière  d'un  processus 
inéluctable. 
\bigskip
\bigskip
\bigskip
\bigskip
332 
\bigskip
Une  technique  inverse,  très  efficace,  sera  de  restreindre  la  portée  d'une  argumentation,  de 
maintenir la conclusion en  deçà  de ce à quoi l'auteur pouvait s'attendre. Ainsi Reinach, après un 
long plaidoyer en faveur de l'authenticité de la tiare de Saïtapharnès, conclut dans l'expectative : 
\bigskip
A l'heure actuelle, je pense qu'aucun archéologue n'a le droit d'être absolument affirmatif au sujet 
de la tiare. Il doit peser le pour et le contre, étudier... et attendre (4). 
\bigskip
Le  lecteur,  mis  en  confiance  par  cet  excès  de  modération,  va  spontanément  plus  loin  dans  les 
conclusions que si l'auteur avait voulu l'y conduire de force. 
\bigskip
Toutes  les  techniques  d'atténuation  (5)  donnent  une  impression  favorable  de  pondération,  de 
sincérité, et concourent à détourner de l'idée que l'argumentation est un procédé, un artifice (6). » 
\bigskip
(3) Ibid., p. 416 (Kunstgriff 20). 
(4) Vayson, De Pradenne, Les fraudes en archéologie préhistorique, p. 545. 
(5) Richard D. D. Whately, Elements of Rhetoric, Part II, chap. II, § 5, p. 137. 
(6) Cf. § 96 : La rhétorique comme procédé. 
\bigskip
P 618-619 : « Certaines figures, telles l'insinuation (7), la réticence (8), la litote (1), la diminution 
(2),  l'euphémisme,  relèvent  de  ces  techniques  d'atténuation  dans  la  mesure  où  l'on  s'attend  à  les 
voir interpréter comme l'expression d'une volonté de modération ; elles ont, sous cet aspect, une 
fonction commune, encore que, dans d'autres cas, leurs fonctions puissent être fort diverses et que 
ces  figures  trouvent  sans  doute  leur  origine  dans  des  secteurs  très  différents  de  la  pensée  et  du 
comportement. » 
\bigskip
(7) Quintilien, Vol. III, liv. IX, chap. II, §§ 65 et suiv. 
(8) Rhétorique à Herennius, liv. IV, § 67. 
(1) Cf. § 67 : Le dépassement.  
(2) Rhétorique à Herennius, liv. IV, § 50.  
\bigskip
P  619 :  « On  peut  aussi  atténuer  les  prétentions  de  l'argumentation  en  recourant  à  l'hypothèse. 
Ainsi,  l'analogie  est  souvent  présentée  comme  une  hypothèse,  ce  qui  semble  modéré,  mais  ses 
effets paraissent dès lors conduire, de façon contraignante, à la conclusion (3). 
\bigskip
Les  utopies  présentent  le  même  caractère  d'hypothèse  à  partir  de  laquelle  les  conséquences, 
cependant, se dérouleraient d'une façon entièrement rationnelle. 
\bigskip
De  même  que  l'on  peut  accroître  la  force  des  arguments  en  agissant  comme  si  leur  force  était 
supérieure à ce que l'on serait fondé à croire, ou bien encore en modérant les prétentions, de même 
on  peut,  par  des  techniques  inverses,  diminuer  la  force  des  arguments,  et  spécialement  ceux  de 
l'adversaire. L'orateur s 1 y expose lui-même souvent : ainsi l'émotion exagérée, disproportionnée 
avec  le  sujet,  le  but  à  atteindre  on  la  nature  des  arguments,  suggère  des  prétentions  qui  feront 
paraître faible toute l'argumentation. 
\bigskip
On  peut  aussi  minimiser,  d'avance  ou  rétrospectivement,  l'effet  de  certains  arguments,  en 
l'attribuant, non à leur valeur propre, mais à d'autres facteurs inhérents à la personne de l'orateur. 
Tout ce que l'on accorde à la personne sera enlevé à certaines de ses manifestations. » 
\bigskip
(3) Cf. § 85 : Comment on utilise l'analogie. 
\bigskip
P 620 : « Cette technique agit à différents niveaux. 
\bigskip
\bigskip
\bigskip
\bigskip
333 
\bigskip
A celui du jugement, on diminue la portée d'une appréciation sévère en faisant état de la sévérité 
habituelle  de  telle  personne  :  on  ne  la  considère  plus  comme  un  juge  objectif  mais  comme 
quelqu'un dont il faut escompter le coefficient de sévérité. Ce même raisonnement permettra, cela 
va sans dire, d'accorder plus d'importance au moindre éloge, à la moindre approbation. 
\bigskip
Au niveau du discours, on insistera sur les qualités de l'orateur, sur son esprit, son humour, son 
talent,  son  prestige,  sa  puissance  de  suggestion.  On  opérera  ainsi  une  dissociation  entre  la  force 
réelle, intrinsèque, des arguments, et leur puissance apparente, où se mêle ce qui tient à eux et ce 
qui  tient  à  d'autres  facteurs.  Cette  dissociation  correspond  à  d'autres  dissociations  tendant  au 
même  but,  celle  entre  auditoire  universel,  qui  échappe  aux  prestiges  de  l'orateur,  et  auditoires 
particuliers  qui  le  subissent,  celle  entre  validité  et  efficacité.  On  se  souviendra  du  dialogue  entre 
Hugo et Jessica dans Les Mains sales de Sartre, cité plus haut (1) et auquel pourraient s'appliquer 
ces trois dissociations. 
\bigskip
Au  niveau  de  la  théorie  de  l'argumentation,  enfin,  on  déniera  parfois  toute  force  aux  arguments 
eux-mêmes en attribuant leur effet à des facteurs entièrement irrationnels, ou à la seule forme des 
discours (2). 
\bigskip
Un autre moyen de diminuer la force des arguments sera de souligner leur caractère passe-partout, 
prévu, facile à trouver, déjà entendu. » 
\bigskip
(1) Cf. § 97 : Interaction et foi-ce des arguments.  
(2) Cf. la théorie des dérivations opposées aux résidus chez Pareto. 
\bigskip
P 620-621 : « Tous les maîtres de rhétorique ont insisté sur l'avantage de l'argumentation « propre 
à  la  cause  »,  qui  n'est  pas  un  simple  lieu  commun  auquel  on  pourrait  opposer  un  autre  lieu 
commun. A l'adresse d'un prêtre adultère qui s'appuyant sur une loi lui permettant de grâcier un 
coupable, voulait l'appliquer à lui-même, Quintilien propose cette réplique qu'il estime propre à la 
cause : 
\bigskip
« Ce n'est pas à un coupable que tu faisais grâce, mais à deux, puisque, toi mis hors de cause, il 
est  impossible  de  tuer  ta  complice  »  ;  cet  argument  est  fourni  par  la  loi  qui  défend  de  tuer  la 
femme adultère sans son complice (1). » 
\bigskip
(1) Quintilien, Vol. II, liv. V, chap. X, § 104. 
\bigskip
P 622 : « Ce qui caractérise l'argument propre à la cause, c'est que, contrairement aux arguments 
plus  généraux,  qui  auraient  pu  être  trouvés  spontanément  par  quiconque,  sans  le  secours  de 
l'orateur, il ajoute généralement quelque chose à notre information, ou à nos habitudes de pensée. 
Nous dirions volontiers que ce n'est souvent, se développant sur un fonds  d'arguments généraux 
non explicités, qu'un argument d'appoint, mais qui ne subit pas la dévaluation propre à tout ce qui, 
étant applicable en tous cas, connu de tous, peut être aisément considéré comme procédé (2). » 
\bigskip
(2) Cf. 96 La rhétorique comme procédé. 
\bigskip
P 621-622 : « L'argument prévu est un argument banal. C'est aussi un argument qui, bien que l'on 
en  ait  eu  connaissance,  n'a  pas  empêché  d'adopter  la  décision  que  l'on  défend  (3)  ;  d'où 
présomption que sa force n'était pas bien grande. Prévoir un argument est, en outre, témoignage 
de compétence. La prévision empêche que, une fois énoncé par l'adversaire, cet argument diminue 
la  confiance  que  l'on  a  dans  l'orateur  ;  celui-ci  ne  pourra  être  désarçonné  ;  ses  jugements  ne 
devront pas être revisés; bref l'argument prévu, qu'il soit  ensuite effectivement produit ou non, a 
perdu de sa puissance critique. Ajoutons que les raisons pour lesquelles un argument est prévisible 
contribuent  souvent  à  sa  dévaluation  :  c'est  qu'il  est  sans  doute  banal,  qu'il  peut  être  considéré 
\bigskip
\bigskip
\bigskip
334 
\bigskip
comme  un  procédé,  mais  c'est  peut-être  aussi  qu'il  tient  à  la  personne  de  l'adversaire,  à  ses 
préjugés  bien  connus,  à  ce  que  l'on  sait  de  son  caractère.  Les  méfaits  de  la  prévision  s'attachent 
d'autre part aux discours dont la conclusion, d'avance connue, ne laisse aucune place à la liberté de 
l'orateur;  d'où  certaines  difficultés  particulières  au  sermon,  au  discours  épidictique  en  général 
(1). » 
\bigskip
(3) Cf. 9 La délibération avec soi-même. 
(1) Cf. CI.-L. ESTÈVE, Etudes philosophiques sur l'expression littéraire, pp. 62 
\bigskip
P  622-623 :  « Un  argument  peut  aussi  perdre  de  sa  force,  non  parce  qu'il  fut  prévu  dans  sa 
singularité concrète, mais tout simplement parce que l'on peut montrer, en le qualifiant d'un terme 
technique,  qu'il  entre  dans  la  catégorie  des  raisonnements  aventureux,  prévus  et  classés  par  les 
théoriciens  (2).  L'auditoire,  mieux  informé,  rendu  à  même  de  reconnaître  la  banalité  de 
l'argument,  son  caractère  de  procédé,  modifiera  rétrospectivement  l'appréciation  de  sa  force. 
D'autre part celui qui l'attaque aura témoigné utilement de sa compétence. » 
\bigskip
P  623-624 :  « Les  avantages  que  l'on  reconnaît  à  l'argument  propre  à  la  cause,  à  l'argument 
imprévu, font sans doute une grande  part de la force qui s'attache à la reprise d'un argument de 
l'adversaire, pour en tirer une conclusion différente et même contraire à celle qu'il y attachait. C'est 
ainsi  que  Bossuet  dans  son  sermon  sur  l'aumône,  montre  longuement  que  le  grand  nombre  des 
enfants,  loin  de  faire  obstacle,  comme  on  pouvait  le  soutenir,  à  l'exercice de  la  charité,  devait  au 
contraire le favoriser. Il reprend notamment une exhortation de saint Cyprien : 
\bigskip
«Mais  vous  avez  plusieurs  enfants,  et  une  nombreuse  famille...  » :  C'est  ce  qui  vous  impose 
l'obligation d'une charité plus abondante; car vous avez plus de personnes pour lesquelles vous 
devez apaiser Dieu ...  
Si donc vous aimez vos enfants, si vous ouvrez sur leurs besoins la source d'une charité et d'une 
douceur véritablement paternelle, recommandez-les à Dieu par vos bonnes œuvres... 
Vous qui donnez l'exemple à vos enfants de conserver plutôt le patrimoine de la terre que celui du 
ciel, vous êtes doublement criminel, et de ce que vous n'acquérez pas à vos enfants la protection 
d'un  tel  Père,  et  de  ce  que  de  plus  vous  leur  apprenez  à  aimer  plus  leur  patrimoine  que  Jésus-
Christ même (1). 
\bigskip
(2) Cf. Schopenhauer, éd. Brockhaus, vol. 2: Die Welt als Wille und Vorstellung, Erster Band, § 9, 
P. 55 ; vol. 3 : Zweiter Band, p. 113 ; éd. Piper. vol. 6 : Eristische Dialektik, note, p. 409. 
(1) Bossuet, Sermons, vol. II, pp. 690-691. 
\bigskip
P  623 :  « Toute  réfutation  -  que  ce  soit  celle  d'une  thèse  admise,  d'un  argument  de  l'adversaire, 
d'un  argument  non  exprimé,  d'une  objection  à  un  argument  -  implique  l'attribution  à  ce  qu'on 
réfute, d'une certaine force qui convienne à l'application utile de notre effort : on estimera ce que 
l'on combat assez haut pour rendre la réfutation importante, digne d'être prise en considération, et 
cela  non  seulement  dans  un  but  de  prestige,  mais  aussi  afin  de  mieux  attirer  l'attention  de 
l'auditoire,  d'assurer  aux  arguments  que  l'on  emploie  une  certaine  force  pour  l'avenir;  et  on 
l'estimera assez bas pour rendre la réfutation suffisante. 
\bigskip
Cette évaluation de la force de ce que l'on combat peut s'énoncer plus ou moins explicitement, ou 
s'inférer  de  la  manière  dont  nous  en  traitons,  voire  par  la  façon  dont  nous  reproduisons  tel 
argument de l'adversaire. Parfois on se servira, pour décrire la puissance de certaines affirmations 
de l'adversaire, du comportement de celui-ci, de son assurance ou manque d'assurance. 
\bigskip
Ce comportement pourra être utilisé également pour en inférer la force de nos propres arguments : 
ce sera la colère d'un adversaire qui se voit coincé dans la discussion, le fait que l'adversaire utilise 
\bigskip
\bigskip
\bigskip
335 
\bigskip
des diversions, questionne au lieu de répondre (2). Faire allusion à ces réactions sera manière de 
souligner et par là d'accroître la force des arguments qui les ont provoquées. » 
\bigskip
(2) Cf. Schopenhauer, éd. Piper, vol. 6 : Eristische Dialektik, p. 424 (Kunstgrill 34). 
\bigskip
P  623-624 :  « Ces  réactions  pourront  d'ailleurs  éclairer  l'orateur,  lui  permettre  de  poursuivre 
l'argumentation  sur  le  terrain  où  l'adversaire  témoigne  qu'il  a  été  ébranlé,  et  cela  même  lorsque 
l'orateur ignore exactement ce qui a si profondément touché son interlocuteur. Car l'efficacité de 
son  propre  discours  pourra  surprendre  l'orateur,  et  influencer  son  argumentation  subséquente. 
Montaigne,  et  Pascal  à  sa  suite,  ont  souligné  la  surprise,  l'encouragement  que  peut  causer  la 
réaction  de  l'auditoire;  et  le  moyen  que  l'on  a  par  contre  de  faire  douter  l'orateur  de  ses  propres 
arguments : 
\bigskip
J'ay autrefois employé à la necessité et presse du combat des revirades qui ont faict faucée outre 
mon dessein et mon esperance je ne les donnois qu'en nombre, on les recevait en pois (1). » 
\bigskip
... qu'on ne donne pas à ce qu'il dit l'estime que son prix mérite on verra le plus souvent qu'on le 
lui fera désavouer sur l'heure, et qu'on le tirera bien loin de cette pensée meilleure qu'il ne croit, 
pour le jeter dans une autre toute basse et ridicule (2). 
\bigskip
(1) Montaigne, Bibl. de la Pléiade, Essais, liv. III, chap. VIII, p. 908. 
(2) Pascal, Œuvre, Bibl. de la Pléiade, De l'esprit géométrique et de l'art de persuader, p. 383. 
\bigskip
§ 99. L'INTERACTION PAR CONVERGENCE 
\bigskip
P  624 :  « Le  calcul  des  probabilités  a  élaboré  des  techniques  très  précises  pour  déterminer  la 
probabilité d'une conclusion fondée sur plusieurs prémisses dont la probabilité est fixée et dont les 
relations  sont  données  et,  réciproquement,  pour  évaluer  la  probabilité  de  ces  prémisses  à  partir 
d'une  conclusion  observée.  Mais  l'interaction  entre  arguments  ne  peut  être  que  bien  rarement 
traitée  de  cette  manière,  car  à  moins  d'être  insérés  dans  un  système,  ils  n'offrent  jamais  la 
précision  et  l'univocité  indispensables.  Nul  ne  méconnaît  cependant  qu'il  existe  des  interactions 
entre  arguments.  L'une  des  plus  importantes  dérive  de  ce  que,  d'une  manière  générale,  nous 
appellerons la convergence. 
\bigskip
Si  plusieurs  arguments  distincts  aboutissent  à  une  même  conclusion,  qu'elle  soit  générale  ou 
partielle, définitive ou provisoire, la valeur accordée à la conclusion et à chaque argument isolé en 
sera  accrue,  car  il  paraît  peu  vraisemblable  que  plusieurs  raisonnements  entièrement  erronés 
conduisent à un même résultat. Cette interaction entre arguments isolés convergents peut résulter 
de  leur  simple  énumération,  de  leur  exposé  systématisé,  ou  encore  d'un  «  argument  de 
convergence » explicitement allégué. » 
\bigskip
P  625 :  « La  puissance  de  cet  argument  n'est  pratiquement  jamais  méconnue.  Sans  doute  des 
discussions  théoriques  porteront-elles  sur  le  point  de  savoir  dans  quelle  mesure  la  convergence 
suffit  à  elle  seule  pour  entraîner  la  persuasion  (1),  dans  quelle  mesure  l'accroissement  de 
vraisemblance  exige  une  probabilité  initiale  minimum,  mais  il  s'agit  alors  des  fondements  de 
l'argument de convergence et non de son utilisation dans les débats. 
\bigskip
La  convergence  elle-même  est  toutefois  une  affirmation  qui  peut  toujours,  dans  un  système  non 
formel,  être  contestée,  car  elle  dépend  de  l'interprétation  donnée  aux  arguments  :  l'identité  de 
leurs conclusions n'est jamais absolue puisque celles-ci font corps avec les arguments et prennent 
leur signification de la manière dont on aboutit à elles (2). 
\bigskip
Parfois  cependant  la  convergence  est  vérifiable  expérimentalement,  c'est  la  consilience  de 
Whewell,  laquelle  constitue,  selon  lui,  le  fondement  le  plus  solide  du  raisonnement  inductif. 
\bigskip
\bigskip
\bigskip
336 
\bigskip
L'exemple  le  plus  remarquable  de  consilience  est  la  détermination,  à  l'aide  de  méthodes 
différentes, du nombre d'Avogadro. 
\bigskip
C'est aussi la base expérimentale qui caractérise la notion de congruence que l'on oppose souvent à 
la simple cohérence : lorsque quatre joueurs de cartes reçoivent successivement, au début d'un jeu, 
l'as, le roi, la dame, et le valet de coeur, la probabilité que le jeu n'a jamais été mêlé, ou qu'il a été 
réarrangé avant la distribution, très faible pour chacun des joueurs, augmente par la confrontation 
de leurs observations (3). De même, si des témoins individuellement peu dignes de foi, déposent, 
sans s'être concertés au préalable, d'une façon concordante, la valeur de chacun des témoignages 
en  sera  renforcée.  De  même  encore  la  concordance  d'opinions  entre  un  grand  nombre  de 
personnes peut renforcer les opinions individuelles. » 
\bigskip
(1) V. Pareto, Traité de sociologie générale, I, chap. IV, §§  563 et suiv., pp. 304-6.  
(2) Cf. § 31 : Clarification et obscurcissement des notions. 
(3) C. L. Lewis, Au Analysis of Knowledge and Valuation, pp. 338 et suiv. 
\bigskip
P  626 :  « La  convergence  entre  arguments  perdra  éventuellement  de  son  poids  si  le  résultat  du 
raisonnement éveille, par ailleurs, une incompatibilité qui le rend inacceptable. Et ceci nous amène 
à  souligner  un  nouveau  type  de  convergence.  C'est  celle  qui  peut  se  constater  entre  un  ensemble 
connu,  croyance  religieuse,  système  scientifique  ou  philosophique,  et  un  argument  qui  vient  le 
confirmer  :  fait  nouveau  qui  corrobore  un  système  scientifique,  interprétation  d'un  texte 
particulier qui corrobore un ensemble juridique, une conception des valeurs. 
\bigskip
L'affirmation de pareille convergence  n'a pas  nécessairement un effet favorable sur l'adhésion au 
système  et  sur  les  nouveaux  arguments  allégués.  Parfois  on  considérera  la  convergence  comme 
irrelevante  parce  que  l'auditeur  n'attache  pas  au  système  en  cause  la  même  importance  que 
l'orateur,  ou  parce  que  cette  convergence  est  considérée  comme  dépourvue  de  sens  :  pour  qui 
refuse le rapport entre science et idéologie, peu importe que les théories scientifiques de Lysenko 
soient plus ou moins conformes au matérialisme dialectique (1). Le problème de la signification de 
la  convergence  se  posera  chaque  fois  que  l'on  veut  mettre  en  rapport  des  domaines  que  l'on 
considère  comme  isolés  l'un  de  l'autre,  et  entre  lesquels,  avant  de  pouvoir  faire  état  de  la 
convergence, il faudra lever les barrières. Sans doute, à la limite, est-ce avec le corpus tout entier 
du savoir, des croyances, que l'argumentation tendra à être convergente, mais il s'agit alors d'une 
convergence  diffuse,  qui  ne  saurait  être  explicitement  alléguée  ;  la  notion  de  convergence  ainsi 
élargie  se  confond  avec  l'exigence  générale  formulée  plus  haut,  à  savoir  qu'il  y  a  lieu  de  tenir 
compte, pour donner pleine force à un argument, de tout ce qu'admet l'auditoire. » 
\bigskip
(1) J. Huxley, Soviet genetics and world science, p. 33 
\bigskip
P  626-627 :  « La  convergence  peut  aussi  exciter  la  méfiance  :  on  craindra  que  les  éléments 
nouveaux  n'aient  été  arrangés  en  vue  de  cette  convergence.  Des  plébiscites  ou  des  élections  trop 
favorables  aux  thèses  ou  aux  candidats  du  gouvernement  ont  rarement  été  considérés  comme 
l'expression sincère de l'opinion  des votants. Et l'on  n'a guère pris au sérieux l'argumentation de 
Chasles,  défendant  l'authenticité  des  autographes  qu'il  présentait  à  l'Académie  des  Sciences  de 
Paris,  en  montrant  qu'une  pièce  soutenait  l'autre  (1).  Il  était  trop  facile  de  répondre  que  les 
anciennes  et  nouvelles  pièces  apportées  au  débat  étaient  toutes  fabriquées  en  vue  de  former  un 
ensemble  cohérent.  Il  peut  se  faire  d'ailleurs  que  la  convergence  entre  arguments,  comme  celle 
entre  arguments  et  un  ensemble  doctrinal,  ne  se  perçoive  que  lorsque  chaque  élément  a  pris  sa 
place dans un complexe : la méfiance, qui n'atteignait pas tels discours isolés, les touchera à partir 
du  moment  où  ils  sont  insérés  dans  un  ensemble  trop  cohérent.  Ce  phénomène  se  perçoit 
clairement dans les réactions du publie à certaines propagandes politiques. » 
\bigskip
(1) Vayson De Pradenne, Les fraudes en archéologie préhistorique, pp. 391-395. 
\bigskip
\bigskip
\bigskip
337 
\bigskip
 
P 627 : « La méfiance envers la cohérence excessive fait que un certain degré d'incohérence est pris 
comme indice de sincérité et de sérieux. M.-L. Silberer ayant demandé à une série de sujets quelle 
était  la  plus  grande  vertu  et  quel  le  plus  grand  vice,  constate  avec  satisfaction  que  les  deux 
réponses sont rarement l'inverse l'une de l'autre et y voit la preuve que ses questions ont été prises 
au sérieux (2). 
\bigskip
La  force  persuasive  de  la  convergence  peut  donc  être  modifiée  grâce  à  une  réflexion  sur  cette 
convergence même. Il ne s'agit plus ici de l'interaction entre arguments situés sur un même plan, 
mais entre arguments étroitement dépendants l'un de l'autre, les premiers étant l'objet sur lequel 
portent les seconds. On médite notamment sur les rapports entre conclusion et arguments, on se 
demande dans quelle mesure ceux-ci sont influencés par celle-là. » 
\bigskip
(2) M.-L. Silberer, Autour d'un questionnaire, dans Le diagnostic du caractère, p. 197, 
\bigskip
P  627-628 :  « On  sait  que  la  plupart  des  hommes  admettent  plus  volontiers  les  thèses  qui  leur 
plaisent  (1).  Mais  cette  tendance  de  l'esprit  humain  au  «  wishful  thinking  »  peut  être  prise  en 
compte  et  l'on  diminuera  d'autant  la  force  des  arguments  qui  aboutissent  à  des  théories  ou 
prévisions trop conformes aux désirs. Bien plus. On montrera qu'un jugement, de par son  action 
même, tend à modifier ce qu'il décrit : le défaitiste, en temps de guerre, est celui qui non seulement 
prévoit  la  défaite  parce  qu'elle  ne  lui  répugne  pas  suffisamment,  mais  encore  celui  qui,  par 
l'affirmation  de  cette  crainte,  contribue  à  la  défaite.  On  veut,  par  l'accusation  de  défaitisme, 
l'obliger  à  prendre  connaissance,  à  la  fois  des  origines  troubles  de  son  jugement  et  des 
conséquences qui pourraient en résulter. » 
\bigskip
(1) Cf. § 14 : Argumentation et engagement. 
\bigskip
§ 100. L'AMPLEUR DE L'ARGUMENTATION 
\bigskip
P 628 : « De deux démonstrations, toutes les deux contraignantes,  partant  des mêmes prémisses 
pour  aboutir  aux  mêmes  conclusions,  la  plus  courte  paraîtra  presque  toujours  la  plus  élégante  : 
produisant les mêmes effets, entraînant le même degré de conviction, étant aussi satisfaisante et 
aussi complète, sa brièveté ne présente que des avantages. Il n'en sera pas de même dans le cas de 
l'argumentation : l'ampleur de celle-ci joue un rôle qui manifeste de façon éclatante la différence 
entre démonstration et argumentation. 
\bigskip
Dans  cette  dernière,  sauf  si  elle  se  déroule  à  l'intérieur  d'un  cadre  préalablement  donné,  les 
prémisses  peuvent  toujours  être  utilement  étayées  en  les  rendant  solidaires  d'autres  thèses 
admises.  De  même,  en  ce  qui  concerne  les  conclusions,  sauf  quand  le  point  à  juger  est  bien 
déterminé,  on  peut  les  rendre  solidaires  de  certaines  de  leurs  conséquences,  ce  qui  permet  de 
prolonger l'argumentation, en transposant l'objet du débat. » 
\bigskip
P  628-629 :  « Si  les  points  de  départ  et  l'aboutissement  d'une  argumentation  ne  sont  pas 
strictement circonscrits, les chaînons intermédiaires sont bien plus indéterminés encore. Dans une 
démonstration rigoureuse, il ne faut indiquer que les chaînons indispensables au déroulement de 
la preuve, mais il faut les indiquer tous. Dans une argumentation, il n'y a pas de limite absolue à 
l'amoncellement  utile  des  arguments,  et  il  est  permis  de  ne  pas  énoncer  toutes  les  prémisses 
indispensables au raisonnement. » 
\bigskip
P  629 :  « Les  avantages qu'offre  l'accumulation  des  arguments  sont  de  deux  sortes  :  ils tiennent, 
d'une part, aux rapports entre arguments, d'autre part, à la diversité des auditoires. 
\bigskip
Nous  avons  vu  que  des  arguments  variés,  aboutissant  à  une  même  conclusion,  se  renforcent 
mutuellement  (1).  La  recherche  de  la  convergence  entre  arguments  incitera  donc  à  accroître 
\bigskip
\bigskip
\bigskip
338 
\bigskip
l'ampleur de l'argumentation. Il en va de même de tout effort pour intégrer les arguments dans un 
réseau  plus  complet,  aux  connexions  plus  variées,  mieux  précisé,  mieux  à  l'abri  des  objections 
possibles. Cette extension de l'argumentation n'est qu'une forme nouvelle de l'effort pour s'assurer 
de plus fermes prémisses. » 
\bigskip
(1) Cf. § 99 L'interaction par convergence.  
\bigskip
P  629-630 :  « Un  cas  d'extension  mérite  une  attention  particulière  :  c'est  celui  des  arguments 
introduits comme complémentaires d'arguments antérieurs, dont ils dépendent donc étroitement. 
Toute dissociation du type apparence peut utilement être complétée, avons- nous vu, par une 
                                                   réalité  
\bigskip
explication  de  la  différence  entre  termes  I  et  II  (2).  Le  rôle  assigné  aux  idoles  par  Bacon,  à 
l'imagination et aux passions dans la philosophie rationaliste, aux préjugés dans la philosophie des 
lumières,  aux  habitudes  et  au  refoulement  dans  la  psychologie  moderne,  se  conçoit  comme 
complémentaire d'une dissociation préalable et des critères proposés pour connaître la réalité. On 
ne se bornera pas à expliquer la possibilité de l'erreur, par ces facteurs de trouble, mais on tentera 
de  combattre  ceux-ci.  C'est  à  l'usage  de  ces  arguments  complémentaires  que  songe  Fénelon 
lorsqu'il décrit la technique de l'orateur habile et expérimenté : 
\bigskip
ou bien il remonte aux principes d'ou dépendent des vérités qu'il  veut persuader ; ou bien il tâche 
de guérir les passions, qui empêchent ces vérités de faire impression (1). » 
\bigskip
 (2) Cf. § 91 Les couples philosophiques et leur justification. 
(1) Fénelon, Œuvres, éd. Lebel, t. XXI - Dialogues sur l'éloquence, p. 65. 
\bigskip
P  630 :  « Parce  que  l'argumentation  positive  n'est  pas  contraignante,  l'argumentation  négative, 
dévoilant et écartant les obstacles qui s'opposent à l'efficacité de la première, s'avère éminemment 
utile.  Notons  que  les  passions,  en  tant  qu'obstacle,  ne  doivent  pas  être  confondues  avec  les 
passions  qui  servent  d'appui  à  une  argumentation  positive,  et  qui  seront  d'habitude  qualifiées  à 
l'aide d'un terme moins péjoratif, tel que valeur, par exemple. 
\bigskip
L'argumentation  complémentaire,  expliquant  l'attrait  de  l'apparence,  du  mal  ou  de  l'erreur,  peut 
donner lieu, avec l'argumentation positive, à un argument de convergence dont Pascal ne manque 
pas d'user dans son apologie du Christianisme : 
\bigskip
…  quel  avantage  peuvent-ils  tirer,  lorsque,  dans  la  négligence  où  ils  font  profession  d'être  de 
chercher la vérité, ils crient que rien ne la leur montre, puisque cette obscurité où ils sont, et qu'ils 
objectent  à  l'Eglise,  ne  fait  qu'établir  une  des  choses  qu'elle  soutient,  sans  toucher  à  l'autre,  et 
établit sa doctrine, bien loin de la ruiner (2) ? 
\bigskip
Se  trouver  dans  l'obscurité,  ne  peut que  renforcer  l'adhésion  à  la  doctrine  de  l'Église,  puisqu'elle 
fait sa place à cet état et le prévoit expressément. » 
\bigskip
(2) Pascal, Bibl. de la Pléiade, Pensées, 335 (C. 209), p. 911 (194 éd. Brunschvicg). 
\bigskip
P  630-631 :  « L'argumentation  négative,  tendant  à  montrer  pourquoi  l'auditoire  n'a  pas  réagi 
comme il l'aurait dû aux événements ou aux discours, en viendra souvent à mettre au jour, pour les 
combattre, des arguments explicites ou implicites, qui sont censés avoir influé sur cet auditoire (3). 
On  montrera  parfois  que  l'auditeur  obéit  à  des  raisons  qu'il  ne  connaît  pas  lui-même,  ou  qu'il 
n'oserait avouer. Ceci donne à l'ampleur de l'argumentation un nouvel aspect : on ne se contente 
pas de combattre l'imagination, les passions, comme telles ; on développe les arguments qui ont pu 
séduire, que l'on rend responsables de l'attitude prise. La question est alors de savoir quels sont les 
\bigskip
\bigskip
\bigskip
339 
\bigskip
arguments  qu'il  y  a  intérêt  à  mettre  en  lumière,  que  l'auditoire  reconnaîtra  pour  siens,  et  qui 
pourtant, une fois décelés, seront aisément contrebattus. » 
\bigskip
(3) Cf. J. Wisdom, Gods, dans A. Flew, Essays on Logic and Language, pp. 199200. 
\bigskip
P  631 :  « Très  souvent  enfin  une  conséquence  dépend  d'un  certain  nombre  de  conditions,  et  l'on 
peut  examiner  celles-ci  successivement  afin  de  décider  si,  oui  ou  non,  elles  étaient  présentes.  En 
logique, la preuve de la fausseté d'une seule prémisse dispense de l'examen des autres; mais dans 
l'argumentation, cette preuve n'est jamais contraignante et l'examen critique des autres conditions 
est  rarement  superflu.  Ce  n'est  que  lorsqu'on  dispose  d'un  argument  qui  semble  difficilement 
réfutable que l'on a avantage pour mettre celui-ci bien en valeur, à serrer l'argumentation. 
\bigskip
Un exemple éminent de critique successive est, au tribunal, la double défense en droit et en fait (1). 
On pourrait en rapprocher maintes argumentations dans d'autres domaines. Lorsque Berkeley fait 
dire à Philonous : 
\bigskip
Les  innovations  en  matière  de  gouvernement  et  de  religion  sont  dangereuses  et  il  faut  les 
décourager, je l'avoue franchement. Mais existe-t-il une raison analogue pour les décourager en 
philosophie ? 
\bigskip
et plus loin : 
\bigskip
Mais  ce  n'est  pas  mon  rôle  de  plaider  pour  les  nouveautés  et  les  paradoxes...  C'est  contre  ces 
innovations et les autres analogues que je tente de défendre le sens commun (2). 
\bigskip
qu'est-ce, sinon une double défense, l'une portant sur la règle (le droit), l'autre sur son application 
(le fait) ? » 
\bigskip
(1) Cf. Quintilien, vol. II, liv. IV, chap. V, § 13. 
(2) Berkeley, Œuvres choisies, t. II: Les trois dialogues entre Hylas et Philonous, 3e dialogue, pp. 
171, 173. 
\bigskip
P 632 : « Les deux arguments ne sont pas seulement des arguments de rechange pour auditoires 
récalcitrants  :  ils  interagissent  en  ce  sens  que  chacun  sera  plus  aisément  accepté,  parce  que 
l'orateur,  n'étant  pas  dans  l'absolue  nécessité  de  s'en  servir,  est  censé  leur  accorder  une  réelle 
valeur. 
\bigskip
La  diversité  des  auditoires  suffit  cependant  à 
l'accumulation  d'arguments, 
indépendamment  de  toute  interaction  entre  eux;  et  cela  même  si  l'on  s'adresse  à  un  auditeur 
unique (1). 
\bigskip
Si on n'avoit à faire qu'à un Juge, il ne faudroit peut-être qu'une sorte d'argument. La diversité des 
Esprits  demande  des  preuves  de  plusieurs  sortes.  je  prends d'abord mon Adversaire  à  la  gorge, 
dit  un  rateur  dans  Pline  (Regulus,  Plin.  Ep.  20).  Et  moi,  dit  Pline,  qui  ne  sais  Pas  où  est  cette 
gorge, je porte des coups par tout pour la rencontrer (2). 
\bigskip
Tout ceci explique à suffisance que l'on rencontre, dans les discours, des arguments qui paraissent 
incompatibles et ne le sont pourtant pas, parce qu'ils s'appliquent à des situations différentes ou à 
des auditoires différents. Nogaro, dans sa critique de la théorie quantitative de la monnaie, n'hésite 
pas à écrire, en  deux phrases successives, que les théoriciens déductifs préfèrent ignorer les faits 
qui infirmeraient leurs théories et  
\bigskip
\bigskip
justifier 
\bigskip
\bigskip
\bigskip
340 
\bigskip
il  est  bien  rare  qu'ils  ne  trouvent  pas  dans  leur  théorie  même  les  arguments  nécessaires  pour 
écarter la contradiction que les faits prétendraient leur apporter (3). 
\bigskip
Dans  son  sermon  sur  la  prédication  évangélique,  Bossuet  nous  présente  l'Evangile,  à  la  fois, 
comme un ordre et comme un conseil : 
\bigskip
Les prédicateurs de l'Evangile font paraître la loi de Dieu dans les chaires en ces deux augustes 
qualités  :  en  qualité  de  commandement,  en  tant  qu'elle  est  nécessaire  et  indispensable;  et  en 
qualité de conseil, en tant qu'elle est utile et avantageuse (4). 
\bigskip
(1) Cf. § 4: L'auditoire comme construction de l'orateur. 
(2)  Gibert,  Juqemens  des  savans  sur  les  auteurs  qui  ont  traité  de  la  Rhétorique,  III,  p.  147.  Cf. 
Pline le Jeune, Lettres, t. I, liv. I, 20, §§ 14-16. 
(3) B. Nogaro, La valeur logique des théories économiques, p. 37. 
(4) Bossuet, Sermons, vol. II : Sur la prédication évangélique, p. 53. 
\bigskip
P  633 :  « Le  comique  de  l'argumentation  s'est  d'ailleurs  emparé  des  effets  étranges  que  peut 
produire l'accumulation de raisons diverses: 
\bigskip
Ne vous querellez pas avec une personne en colère, mais répondez gentiment. L'Ecriture Sainte le 
commande  et,  en  outre,  cela  rend  la  personne  plus  furieuse  que  tout  ce  que  vous  pourriez  faire 
d'autre (1). 
\bigskip
Les divers arguments peuvent paraître, non pas incompatibles, mais simplement superfétatoires, 
parce  que  le  fait  d'admettre  l'un  d'eux  rendrait  les  autres  inutiles.  Ainsi  les  deux  arguments 
successifs employés par W. Churchill lors d'un débat parlementaire en 1939, concernant le budget 
militaire de la Grande-Bretagne : 
\bigskip
Les partis ou les hommes politiques doivent accepter d'être renversés plutôt que de mettre la vie de 
la nation en péril. Par surcroît, il n'existe pas d'exemple dans notre histoire qu'un gouvernement se 
soit vu refuser par le Parlement et l'op es mesures de défense nécessaires (2). 
\bigskip
Ce  caractère  superfétatoire  sera  à  la  fois  plus  net  et  moins  surprenant  lorsque  les  arguments  ne 
font  que  se  répéter.  En  effet  l'ampleur  argumentative  peut  résulter  non  de  l'emploi  d'arguments 
différents qui se soutiennent se complètent, s'adressent à des auditoires variés, mais de la simple 
reproduction, plus ou moins fidèle, des mêmes arguments. Cette insistance a pour but de donner 
aux  arguments  la  présence.  Nous  retrouvons  ici  certaines  figures  telles  la  répétition  et 
l'amplification (3). » 
\bigskip
(1) Fun Fare, Reader's Digest, 1949, p. 64. 
(2) Winston Churchill, Mémoires sur la deuxième guerre mondiale, t. I, vol. I, p. 112. 
3) Cf. § 42 : Les figures du choix, de la présence et de la communion. 
\bigskip
P  633-634 :  « Outre  que  la  superfétation  est  pleinement  justifiée  dans  l'argumentation,  il  faut 
souligner  qu'elle  n'est  généralement  perçue  que  si  on  analyse  les  arguments  d'une  certaine 
manière, car la distinction entre arguments n'est pas un donné. Elle peut, dans certains cas, être 
fixée par la tradition, telle la défense en fait et en droit. Elle peut aussi résulter de la présentation 
du  discours.  Ainsi,  la  division,  au  sens  d'annonce  des  arguments,  en  accentuant  les  plans 
d'argumentation, soulignera qu'il y a cumul d'arguments. Par contre, ce cumul est beaucoup moins 
marqué lorsque l'orateur n'isole point ses raisons  les unes  des autres. Elles tendront parfois à se 
confondre en une argumentation unique, a fortiori, comme dans le passage suivant : 
\bigskip
\bigskip
\bigskip
\bigskip
341 
\bigskip
Vous jugerez si, en pareil moment, il convient, comme hommes d'Etat, que nous annoncions notre 
propre  faiblesse  et  incapacité  a  continuer  la  lutte,  et  de  déclarer  que  nous  sommes  prêts  à 
négocier immédiatement, sans même savoir qui doit recevoir la déclaration (1). » 
\bigskip
(1) W. Pitt, Orations on the French war, p. 107 (29-10-1795). 
\bigskip
P  634 :  « Tous  les  facteurs  qui  poussent  à  allonger  indéfiniment  une  argumentation,  sont 
cependant tenus en échec, pour la parole comme pour l'écrit, par certaines considérations sociales 
et  psychologiques  :  des  limites,  temporelles  ou  spatiales,  sont  imposées  par  des  règles  de 
procédure parfois fort strictes (2), ou de bienséance, et surtout par l'attention que l'auditoire peut 
et veut accorder à l'orateur. 
\bigskip
L'ampleur  du  discours  dépend  aussi  du  nombre  d'orateurs  qui  prennent  part  au  débat,  d'une 
éventuelle répartition de leurs rôles, de l'occasion que chacun aura de reprendre l'argumentation, 
soit  pour  fournir  de  nouveaux  arguments,  soit  pour  répéter  ou  développer  les  arguments  déjà 
énoncés. 
\bigskip
Elle  dépendra  enfin  du  genre  de  discours,  et  des  fonctions  que  l'on  assigne  à  l'auditoire  :  le 
considère-t-on  comme  partie  d'un  auditoire  universel  et  renonce-t-on  à  tous  les  arguments  qui 
seraient sans action sur lui ? Veut-on persuader les membres d'un auditoire particulier et prend-on 
appui sur ses particularités ? Va-t-on présenter tous les arguments que l'on juge relevants dans le 
débat, parce qu'ils peuvent avoir quelque influence sur son issue, ou se propose-t-on uniquement 
de  développer  ce  qui  serait  favorable  à  un  point  de  vue  donné,  en  critiquant  ce  qui  pourrait  s'y 
opposer ? » 
\bigskip
(2)  Cf.  Chanoine  Rome,  La  vitesse  de  parole  des  orateurs  attiques,  Bulletin  de  la  Classe  des 
Lettres de l'Académie Royale de Belgique, 5e série, 1952, 12. 
\bigskip
P  635 :  « Selon  qu'il  s'agit  de  délibération  intime  ou  publique,  de  plaidoyer  ou  de  discours 
épidictique, l'ampleur de l'argumentation se situera dans un cadre différent. 
\bigskip
On  comprend,  devant  la  multitude  des  facteurs  à  prendre  en  considération,  que  Prodicus, 
renvoyant dos à dos les partisans du discours long et -du discours bref, ait proclamé que la seule 
règle  valable  est  que  le  discours  soit  conforme  à  une  juste  mesure  (1).  La  juste  mesure  est  à 
recommander pour ne pas excéder l'auditoire, mais aussi parce que l'usage de certains arguments, 
isolés ou en conjonction avec d'autres, n'est pas sans inconvénients. Il se peut même que la juste 
mesure consiste à se taire. » 
\bigskip
(1) Platon, Phèdre, 267 b. 
\bigskip
§ 101. LES DANGERS DE L'AMPLEUR 
\bigskip
Pour  bien  estimer  les  dangers  de  l'ampleur,  il  y  aurait  lieu  de  considérer,  d'une  part, 
l'argumentation qui fournit les raisons de croire à ce que l'on admet déjà et d'autre part, celle qui 
tend à solliciter notre adhésion; en d'autres termes, de distinguer celle qui concerne les prémisses 
et les schèmes argumentatifs, de celle qui concerne une thèse faisant office de conclusion. » 
\bigskip
P 635-636 : « Au sujet du premier cas, rappelons que toute argumentation est l'indice d'un doute, 
car elle suppose qu'il y a lieu de préciser ou de renforcer l'accord sur une opinion déterminée, qui 
ne serait pas suffisamment claire ou ne s'imposerait pas avec une force suffisante. Le doute soulevé 
par  le  seul  fait  d'argumenter  en  faveur  d'une  thèse  sera  d'autant  plus  grand  que  les  arguments 
utilisés paraîtront plus faibles, car la thèse semblera dépendre de ces arguments. Le danger réside 
donc à la fois dans la simple adjonction de preuves et dans la qualité de celles-ci. Ainsi, sauf quand 
il  s'agit  d'une  technique  scientifique  ou  professionnelle  reconnue,  l'indication  de  la  source  d'une 
\bigskip
\bigskip
\bigskip
342 
\bigskip
information  laisse  planer  sur  cette  dernière  un  certain  doute,  soit  parce  que  cela  implique  que 
l'orateur ne la prend Pas a son propre compte, soit simplement parce que cela met l'esprit critique 
en éveil. Par contre, en présentant une nouvelle comme un fait, sans plus, on laisse croire qu'il ne 
peut y avoir le moindre doute à son sujet, qu'on ne pense même pas devoir le justifier; par ailleurs 
l'indication de la source sera d'autant plus dangereuse que celle-ci jouira de moins de prestige. De 
même, celui qui croit disposer d'une autorité indiscutable répugne à motiver ses décrets. Et c'est 
avoir une plus grande confiance en la perfection divine de prétendre que tout ce que Dieu fait est 
bien que de fournir des preuves de sa bonté. Tout comme une preuve contraignante rend superflue 
toute preuve ultérieure, une vérité évidente rend superflue toute preuve en général. » 
\bigskip
P 636 : « Napoléon craignait que les longs préambules aux lois ne nuisent à leur autorité. Bentham 
avait déjà remarqué, que l'invocation d'un motif peut aboutir au rejet d'une proposition et que les 
adversaires  d'une  motion  ont  loisir  de  la  torpiller  en  ajoutant  à  son  libellé  une  motivation 
inacceptable.  Il  décrit  longuement  comment  un  projet  allégeant  l'impôt  sur  le  savon  fut  écarté  à 
cause  de  considérants  destinés  à  déplaire,  et  montre  l'effet  qu'aurait  certaine  motivation  sur  un 
projet abolissant les lois pénales relatives au blasphème : 
\bigskip
« Considérant qu'il n'y a point de Dieu, toutes les lois pénales relatives à la nature de la divinité 
sont abolies». Lors même que tous les membres de l'assemblée seraient unanimes pour l'abolition 
de  ces  lois  pénales,  il  ne  s'en  trouverait  peut-être  pas  un  seul  qui  ne  fÛt  révolté  par  cette 
déclaration d'athéisme, et ils aimeraient mieux rejeter la mesure en totalité, que de l'obtenir à ce 
prix (1). » 
\bigskip
(1) Bentham, Œuvres, t. I : Tactique des assemblées politiques délibérantes, pp. 378-79. 
\bigskip
P 636-637 : « Lorsque l'argumentation s'avère indispensable parce que, la question étant sujette à 
controverse, aucune des thèses en présence ne jouit d'un accord suffisant, on pourrait croire que 
tout argument dont la valeur n'est pas nulle serait avantageusement lancé dans le débat. Il n'en est 
rien  pourtant  :  il  y  a,  personne  ne  l'ignore,  danger  à  user  de  certains  arguments,  et  ceci, 
essentiellement, à cause de l'interaction entre tous les arguments en cause. » 
\bigskip
P 637 : « Ceux qui sont utilisés contribuent à l'idée que l'on se fait de l'orateur et, par l'entremise 
de cette idée, peuvent affecter l'ensemble du discours. Un argument faible, aisément réfuté, nuit si 
on l'avance, au prestige de qui s'engage par là à le défendre contre les objections éventuelles. 
\bigskip
D'autre  part,  tout  argument,  par  sa  présence,  attire  l'attention  de  l'auditoire  sur  certains  faits, 
l'introduit de force dans certains domaines auxquels il n'avait peut-être pas songé auparavant, et, 
par ce biais, suscite des objections contre ce qui était peut-être déjà acquis par l'orateur : 
\bigskip
Parler à un individu d'une chose, soit pour en dire du bien soit pour en dire du mal, peut disposer 
cet individu, s'il ne l'est pas encore, a s'occuper de cette chose, ou accroître cette disposition, s'il 
l'a déjà (1). 
\bigskip
(1) V. Pareto, Traité de sociologie générale, II, § 1749, pp. 1089-90. 
\bigskip
P  637-638 :  « Le  même  mécanisme  de  la  présence  intervient  quand  on  reprend,  pour  la  réfuter, 
une  affirmation  de  l'adversaire  :  la  plupart  des  orateurs  préfèrent,  pour  cette  raison,  passer  sous 
silence  une  objection  à  laquelle  ils  ne  pourraient  opposer  qu'une  faible  réfutation.  Au  cours 
d'expériences  sur  le  changement  d'attitudes  provoqué  par  le  discours,  oral  on  écrit,  on  constata 
que l'opinion des auditeurs était parfois bel et bien modifiée par celui-ci mais dans un sens opposé 
à  celui  souhaité  (2).  Sans  doute  est-ce  parce  que  le  discours  avait  fait  entrer  dans  le  champ  de 
l'attention  des  éléments  dont  les  auditeurs  s'étaient  jusqu'alors  complètement  désintéressés.  De 
plus,  chaque  argument  invite  à  une  appréciation  de  sa  force,  et  ces  essais  répétés  de  réfutation 
\bigskip
\bigskip
\bigskip
343 
\bigskip
mentale,  pour  peu  qu'ils  réussissent  quelquefois,  peuvent  dégénérer  en  un  négativisme 
systématique dont il ne faut pas méconnaître l'influence, même dans la délibération intime. » 
\bigskip
(2)  Cf. Ch.  Byrd,  Social  Psychology  pp.  215  et  suiv.  (expériences  de  P.  H.  Knower,  Experimental 
studies  of  changes  in  attitudes,  J.  of  soc.  Psych.,  193.5,  6,  pp.  315347  ;  J.  of  abnormal  and  soc. 
Psych., 1936, 30, pp. 522-532 ; J. applied Psych., 1936, 20, pp. 114-127). Cf. aussi C. I. Hovland, A. 
A. Lumsdaine, et F. D. Sheffield, Experiments on Mass Communication, pp. 46-50, 215-216. 
\bigskip
P 638 : « Enfin les nouveaux arguments introduits dans un débat peuvent paraître incompatibles, 
soit avec les assertions de l'orateur, ce qui le rend ridicule ou fait douter de sa sincérité, soit avec 
des thèses déjà admises par l'auditoire, ce qui met celui-ci dans la pénible situation de devoir, s'il 
accorde quelque crédit à ces  nouveaux arguments, chercher lui-même des aménagements à cette 
incompatibilité. 
\bigskip
Cet inconvénient s'attache aussi au cas où les arguments ne sont émis qu'à titre d'hypothèse. Ainsi, 
à  première  vue,  rien  ne  s'oppose  à  ce  que  l'on  présente  une  multiplicité  d'hypothèses  pour 
expliquer  un  événement,  même  si  elles  sont  incompatibles,  car  il  semble  que  leur  accumulation 
n'ait  d'autre  effet  que  de  rendre  l'événement  plus  vraisemblable.  Et  cependant,  on  disqualifiera 
souvent l'adversaire en suggérant que les nouvelles hypothèses sont preuve de ce qu'il n'attachait 
pas grande confiance à ses arguments antérieurs. L'étudiant Huber, défenseur des « iconolithes » 
de Würzburg se moque d'un de ses censeurs qui : 
\bigskip
passe  de  l'hypothèse  d'un  caprice  de  la  Nature  à  celle  de  vestiges  païens  et  de  cette  dernière  à 
l'idée d'une imposture (1)... 
\bigskip
Le  danger  sera  d'autant  plus  grand  que  les  points  sur  lesquels  porte  l'incompatibilité,  paraîtront 
plus essentiels. Des hypothèses voisines, si on se place dans une perspective très générale, peuvent 
se confondre en un argument unique. D'où l'occasion d'une nouvelle argumentation portant sur le 
fait de savoir si, oui ou non, des arguments sont à envisager comme un seul argument, diversement 
nuancé. » 
\bigskip
(1) Mémoire de Huber dans Vayson De Pradenne, Les fraudes en archéologie préhistorique, p. 41. 
\bigskip
P 638-639 : « L'incompatibilité sera surtout manifeste s'il s'agit d'affirmations de fait. Le ridicule 
qui  atteint  alors  l'orateur  éclate  dans  maintes  histoires  amusantes,  relevant  du  comique  de 
l'argumentation, telle la défense de cette ménagère, accusée de ne pas vouloir rendre un pot : « Ce 
pot, d'abord je ne l'ai jamais vu; et puis je ne l'ai pas emprunté ; par ailleurs, je l'ai déjà rendu, et, 
au  surplus,  il  était  fêlé.  »  On  voit  ici,  clairement,  les  méfaits  de  l'accumulation,  car,  pris  deux  à 
deux, plusieurs de ces arguments ne seraient pas inconciliables. » 
\bigskip
P  639 :  « L'auditeur  envisagera-t-il  les  incompatibilités  dans  le  cadre  strict  d'un  point  particulier 
en discussion ? Les envisagera-t-on dans un cadre plus étendu ? Au juge, par exemple, de décider 
si telle prise de position adoptée par un juriste, dans un ouvrage doctrinal, pourra lui être imputée 
lorsque, agissant comme avocat, il développe des arguments incompatibles avec elle. 
\bigskip
Le  danger  s'étend-il  à  l'usage  d'une  notion,  dans  un  même  ouvrage,  sous  des  aspects  multiples  ? 
Lefebvre utilisera la référence aux « primitifs » tantôt pour disqualifier l'idéalisme dans lequel se 
retrouvent  des  traces  de  pensée  primitive;  tantôt  pour  disqualifier  les  théories  de  Comte  en 
montrant que la pensée primitive renferme des éléments supérieurs à ceux de la pensée ultérieure 
(1). Chaque argument a ses titres. Au lecteur de décider s'il les acceptera, chacun en son  lieu, s'il 
aménagera lui-même l'incompatibilité, ou s'il en fera grief à l'auteur. » 
\bigskip
\bigskip
\bigskip
\bigskip
344 
\bigskip
(1)  H.  Lefebvre,  A  la  lumière  du  matérialisme  dialectique,  I  :  Logique  formelle,  logique 
dialectique, pp. 20, 40. 
\bigskip
P  639-640 :  « Le  danger  des  arguments  superfétatoires,  et  surtout  incompatibles,  détermine 
souvent le renoncement à certains arguments. On renonce à cause d'autres éléments du discours, 
mais aussi à cause des opinions professées par l'auditoire, soit par l'auditoire particulier auquel on 
s'adresse,  soit  par  un  auditoire  dont  on  fait  soimême  partie.  Les  défenseurs  de  Rutilius  auraient 
renoncé  -  pour  sa  perte  -  à  l'usage  d'arguments  qui  ne  convenaient  pas  à  des  stoïciens  (1). 
L'attitude  de  Socrate,  refusant  d'implorer  l'indulgence  de  ses  juges,  est  l'un  des  plus  célèbres 
exemples de renoncement. » 
\bigskip
(1) Cicéron, De Oratore, liv. 1, § 230. 
\bigskip
P  640 :  « En  général,  celui  qui  a  souci  de  l'adhésion  de  l'auditoire  universel  renoncera,  même 
devant  un  auditoire  particulier,  à  des  arguments  inadmissibles  pour  cet  auditoire  universel,  tel 
qu'il  se  le  représente  ;  il  jugera  presque  immoral  de  recourir  à  une  argumentation  qui,  à  ses 
propres yeux, ne serait pas rationnelle. 
\bigskip
D'autre  part,  c'est  souvent  maladresse  que  de  heurter  de  front,  ou  simplement  choquer,  un 
auditoire  particulier.  Mieux  vaut  renoncer,  par  exemple,  à  citer  d'affilée,  devant  un  auditoire 
chrétien,  ainsi  qu'on  l'a  déjà  fait,  les  Prophètes,  jésus,  Spinoza  et  Marx,  comme  illustrant  la 
tendance  universaliste  du  peuple  juif.  Cicéron  donnait  une  série  de  conseils  relatifs  à  ce  qu'il 
convenait  d'éviter  :  faire  des  éloges  outrés  suscitant  l'envie,  s'emporter  contre  une  personne 
estimée des juges, reprocher à quelqu'un des défauts qui se retrouvent chez les juges, avoir l'air, en 
plaidant pour autrui, de plaider pour soi-même (2). 
\bigskip
Indépendamment  de  toute  incompatibilité  entre  eux  ou  avec  les  opinions  de  l'auditoire,  une 
cohorte  d'arguments  fait  croire  que  l'on  n'a  une  confiance  suffisante  en  aucun  d'eux.  Dans  la 
réfutation,  le  déploiement  d'arguments  est  souvent  plus  dangereux  encore  :  il  autorise  à  croire 
qu'une simple remarque de l'adversaire était amplement justifiée, puisque rien ne peut être négligé 
pour la combattre (3). 
\bigskip
(2) Cicéron, De Oratore, liv. II, §§ 304-305. 
(3) Cf. A. Craig Baird, Argumentation, Discussion and Debate. pp. 330-31. 
\bigskip
P 640-641 : « Par ailleurs, étant donné que l'orateur est censé connaître les dangers des arguments 
faibles, capables de nuire à son prestige, leur emploi introduit une présomption grave : c'est qu'il 
n'en a pas de meilleurs à sa disposition, voire qu'il n'en existe pas d'autres (1). Ainsi, en énonçant 
des arguments faibles, l'orateur peut anéantir, sans même y penser, d'autres arguments plus forts, 
qui seraient venus spontanément à l'esprit de l'auditeur. Le silence pourra jouer le même rôle que 
l'argument faible, laisser croire qu'il n'y, a pas d'arguments utiles. L'argument malencontreux et le 
silence peuvent être ainsi causes semblables d'une même défaite. Notons à ce propos que, en tout 
ceci,  l'auditeur  suppose  que  l'orateur  connaît  les  techniques  argumentatives  et  les  utilise  à  bon 
escient.  En  fait  il  est  normal,  même  à  l'égard  d'une  personne  qui  ne  passe  pas  pour 
particulièrement  habile,  d'argumenter  en  faisant  état  de  ces  connaissances  :  on  entendra,  par 
exemple,  B  prétendre  que  A  n'avait  pas  dès  le  début  d'une  controverse  telle  interprétation  d'un 
texte qui eût été décisive en sa faveur, car sinon A n'eût pas produit les arguments faibles qu'il a 
mis en avant en faveur de sa thèse. » 
\bigskip
(1) Cf. Démosthène, Harangues et plaidoyers politiques, t. III: Sur l'ambassade, § 213. 
\bigskip
P 641 : « Dangereux aussi sera tout argument qui permet une réplique facile : il tournera en fin de 
compte à l'avantage de celui qui ne l'a pas introduit dans le débat (2). Ou encore, plus simplement, 
\bigskip
\bigskip
\bigskip
345 
\bigskip
l'argument  qui  peut  donner  lieu  chez  l'auditeur  à  une  interprétation  défavorable.  Tel  tract  en 
faveur  d'un  nouveau  vaccin,  s'il  insiste  sur  la  difficulté  de  la  découverte,  sur  les  illusions  et  les 
échecs antérieurs, pourra suggérer l'idée que, cette fois encore, la confiance serait mal placée. 
\bigskip
D'autres arguments sont, de toute évidence, utilisables par toutes les parties, telle l'affirmation : 
\bigskip
C'est parce que ma cause est bonne, juges, que J'ai été bref (3).  
\bigskip
Le  danger  de  pareil  argument  n'est  pas  tant  que  l'adversaire  le  reprenne  à  son  compte  -  il  n'en 
aurait garde généralement c'est d'être qualifié de procédé (4). » 
\bigskip
(2) Cf. § 98 : L'appréciation de la force des arguments facteur d'argumentation. 
(3) Cicéron, De Inventione, liv. I, XLVIII, § 90. 
(4) Cf. § 96 : La rhétorique comme procédé. 
\bigskip
P  642 :  « Parmi  les  problèmes  liés  aux  dangers  de  l'ampleur,  il  faut  faire  une  place  spéciale  à  la 
diversion, déplacement de la discussion sur un autre objet jugé irrelevant (1). Ce serait un procédé 
dangereux et qui partage tous les inconvénients de l'argument faible, s'il y avait toujours accord sur 
cet  aspect  irrelevant.  Mais  il  n'en  est  presque  jamais  ainsi.  L'accusation  de  diversion  et  celle  de 
sophisme  se  ressemblent  en  ce  qu'elles  supposent  une  argumentation  sciemment  irrelevante  ou 
fallacieuse.  Or  l'accusation  ne  se  soutient  que  dans  la  mesure  où  l'argumentation  s'écarte  d'une 
façon appréciable de ce qui est d'usage. Il serait théoriquement possible, en effet, de dénier toute 
réelle valeur argumentative à certaines parties du discours, telles l'exorde ou la péroraison, et de 
les traiter de diversion. C'est la règle de justice qui permet de se faire une opinion en cette matière. 
\bigskip
Ce que l'on qualifie de diversion consiste souvent à porter la discussion sur des points secondaires, 
où  la  défense  est  aisée,  le  triomphe  facile.  Un  cas  plus  caractérisé  consiste  à  introduire  dans  le 
débat  des  éléments  et  des  distinctions  qui  resteront  par  la  suite  inutilisés  :  introduites  peut-être 
par  précaution,  ces  distinctions  sont  dangereuses  parce  que  l'on  tirera  aisément  de  leur  non-
emploi un aveu implicite de leur caractère irrelevant. 
\bigskip
Lorsque le temps dont on dispose est limité, la diversion peut n'avoir d'autre but que d'esquiver les 
points  délicats.  L'étudiant  ignorant  et  habile  pratique  volontiers  cette  technique  aux  examens. 
Dans la controverse, pareille diversion vise à empêcher la discussion, elle n'est plus qu'un sabotage 
des  conditions  préalables  à  celle-ci.  A  la  limite,  nous  aurons  le  filibuster,  où  ne  subsiste  aucun 
désir de faire illusion sur la portée des interventions. Le  danger de maintes argumentations trop 
développées serait de faire songer au filibuster. » 
\bigskip
(1) Cf. Aristote, Topiques, liv. II, chap. 5,112 a; Schopenhauer, éd. Brockhaus, vol. 6: Parerga und 
Paralipomena,  Zweiter  Band.  Zur  Logik  undDialektik,  ~  26,  p.  31  (Neuntes  Stratagem)  ;  éd. 
Piper, vol. 6: Eristische Dialektik, p. 416 (Kunstgriff 18); p. 419 (Kunstgriff 29). 
\bigskip
P 643 : « La diversion se prête à la caricature. Bon nombre d'anecdotes relèvent du comique de la 
diversion : 
\bigskip
Le mari qui entre aux petites heures, est accueilli par sa femme qui brandit une canne de golf. « 
Comment, lui demande-t-il, tu vas jouer au golf à cette heure-ci ? » 
\bigskip
La remarque n'est point sans rapport avec la situation, mais elle résulte d'une réinterprétation de 
celle-ci : elle transpose la discussion sur un nouveau terrain, donne à celui qui l'utilise un certain 
prestige,  fait  gagner  du  temps,  bref,  remplit,  comiquement,  les  services  que  l'on  demande 
habituellement à la diversion. 
\bigskip
\bigskip
\bigskip
346 
\bigskip
§ 102. LES PALLIATIFS AUX DANGERS DE L'AMPLEUR 
\bigskip
Pour pallier les dangers que nous venons de signaler, on peut utiliser tous les procédés destinés à 
éviter ou rendre plus difficile la réfutation. 
\bigskip
S'agit-il  de  protéger  la  personne  de  l'orateur  contre  le  mauvais  effet  que  produiront  certains 
arguments  ?  On  déclarera  qu'ils  ont  été  suggérés,  voire  imposés  à  l'orateur.  C'est  ainsi  que 
Démosthène  montre  que  les  circonstances  ou  l'attitude  de  son  adversaire  l'obligent  à  faire  son 
propre éloge (1), à s'écarter du sujet (2), ou à utiliser un genre d'arguments qui lui déplaît (3). » 
\bigskip
(1) Démosthène, Harangues et plaidoyers politiques, t. IV; Sur la couronne, 3. 
(2) Ibid., 9. 
(3) Ibid., 123. 
\bigskip
P 643-644 : « Lorsqu'un orateur se rend compte de  ce que l'ensemble des  arguments concernant 
une thèse recèle des incompatibilités telles, que celui qui exposerait successivement ces arguments 
serait  taxé  d'incohérence,  il  opérera  normalement  un  choix  parmi  eux.  Mais  s'il  ne  veut  pas  s'y 
résoudre, il usera de divers moyens pour assurer la cohérence : il fera présenter les arguments par 
plusieurs personnages  - comme dans le dialogue, la pièce de théâtre, le roman. Ou bien encore il 
présentera les diverses opinions, en les attribuant à des auteurs différents : les pseudonymes sous 
lesquels écrivit Kierkegaard représentent la plus grande dissociation à laquelle puisse conduire le 
désir de faire place à toutes les alternatives, de ne rien sacrifier des arguments incompatibles (1). » 
\bigskip
(1) Cf. P. L. Holmer, Kierkegaard and ethical theory, Ethics, avril 1953, p. 159.  
\bigskip
P  644  :  “D'une  manière  générale,  pour  éviter  les  mauvais  effets  d'arguments  incompatibles, 
l'orateur  devra  introduire  une  argumentation  complémentaire,  qui  soulignera  l'incompatibilité 
apparente  entre  les  divers  arguments  énoncés  ou  entre  ceux-ci  et  les  croyances  de  l'auditoire,  et 
qui  s'efforcera  d'en  prévenir  les  inconvénients.  Il  expliquera  les  changements  de  point  de  vue, 
présentera les hypothèses comme successives, délimitera le champ d'application des normes afin 
qu'elles ne s'excluent pas. 
\bigskip
Pour  parer  au  danger  d'un  argument  faible,  on  dira  que  celuici  n'a  été  introduit  qu'à  titre 
subsidiaire. Si l'on craint que l'interlocuteur ne sous-estime un argument qui n'est pas dépourvu de 
force,  on  pourra  le  transformer  en  enjeu  du  débat,  obliger  ainsi  les  adversaires  à  le  prendre  au 
sérieux  ;  nous  rejoignons  ici  la  technique  consistant  à  marquer  d'avance  son  accord  au  sujet  de 
certaines preuves (2). 
\bigskip
Enfin  pour  pallier  les  inconvénients  de  la  diversion,  se  prémunir  contre  l'accusation  de  la 
pratiquer, on insistera sur le caractère relevant de tout ce que l'on avance. 
\bigskip
Pour  éviter  de  devoir  employer  certains  arguments,  on  fera  en  sorte  que  la  partie  adverse  n'y 
donne pas lieu; si l'une des parties renonce à l'audition de certains témoins, elle peut espérer que 
l'autre partie en fera autant. » 
\bigskip
(2) Cf. § 27 : Accords propres à chaque discussion. 
\bigskip
P  644-645 :  « Cependant  le  palliatif  spécifique,  si  l'on  craint  l'utilisation  de  certains  arguments, 
sera de les laisser sous-entendus (3). Il existe des arguments dont il est malséant, dangereux, voire 
interdit de faire un usage trop explicite. On ne peut s'y référer que par insinuation, allusion, ou en 
menaçant  de  les  utiliser.  Ia  menace  peut  d'ailleurs  faire  elle-même  partie  de  ces  arguments 
interdits. » 
\bigskip
(3) Quintilien, Vol. III, liv. IX, chap. 11, § 73. 
\bigskip
\bigskip
\bigskip
347 
\bigskip
 
P 645 : « Ce demi-renoncement à certains arguments donne lieu à des figures de renoncement qui 
n'expriment  pas  uniquement  la  modération  de  l'orateur  (1).  La  réticence,  permet  d'évoquer  une 
idée  tout  en  laissant  le  développement  à  l'auditeur  :  ce  développement  pourra  être  suggéré  par 
certaines  formes  d'expression,  telles  le  rythme,  l'allitération.  La  prétérition  est  le  sacrifice 
imaginaire d'un argument. On ébauche ce dernier tout en annonçant que l'on y renonce. En voici 
un exemple banal pris à la Rhétorique à Herennius : 
\bigskip
Ton enfance, que tu as prostituée à tous, j'en parlerais, si je crovais le moment venu. Mais je me 
tais à dessein (2). 
\bigskip
Le  sacrifice  satisfait  aux  convenances  ;  il  laisse  croire  en  outre  que  les  autres  arguments  sont 
suffisamment forts pour que l'on puisse se passer de celui-ci. On peut rattacher à la prétérition ce 
passage oÙ Démosthène semble répugner à faire son propre éloge : 
\bigskip
Philippe nous éprouvait tous. Comment ? En envoyant quelque chose à chacun personnellement, 
en nous offrant de l'argent et, Athéniens, en quantité. Comme il n'avait pas de succès auprès de 
quelqu'un  (ce  n'est  pas  à  moi  de  nie  nommer  moi-même,  mais  les  faits  et  les  actes  mêmes 
désigneront  l'homme)  ,  il  pensait  que  tous  accepteraient  tout  bonnement  l'argent  offert  en 
commun  ;  ainsi  ceux  qui  s'étaient  vendus  personnellement,  seraient  à  l'abri,  si  tous  nous 
participions, même pour une faible part, au profit commun (3). 
\bigskip
S'agit-il d'une figure argumentative ? Certains hésiteraient à le dire, tant  l'insertion dans le texte 
paraît normale ; nous sommes en tous cas en présence d'un palliatif. » 
\bigskip
(1) Cf. §  98 : L'appréciation de la force des arguments facteur d'argumentation. 
(2) Rhétorique à Herennius liv. IV, § 37. 
(3) Démosthène, Harangues et plaidoyers politiques, t. III : Sur l'ambassade, § 167. 
\bigskip
P  645-646 :  « Parfois  le  demi-renoncement  s'exprime  d'une  manière  plus  indirecte  encore  :  un 
silence significatif, voire l'emploi ostentatoire d'arguments faibles, irrelevants, pour marquer qu'il 
en existe d'autres. » 
\bigskip
P  646 :  « Tous  les  renoncements,  semi-renoncements,  peuvent  être  considérés  comme  des 
concessions.  Toutefois,  le  rôle  principal  de  celle-ci  ne  concerne  pas  tant  l'ampleur  de 
l'argumentation que l'étendue des prétentions et le dynamisme des accords. 
\bigskip
La concession s'oppose surtout aux dangers de la démesure; elle exprime le fait que l'on réserve un 
accueil  favorable  à  certains  arguments  réels  ou  présumés  de  l'adversaire.  En  restreignant  les 
prétentions,  en  abandonnant  certaines  thèses,  en  renonçant  à  certains  arguments,  l'orateur  peut 
rendre sa position plus forte, plus aisée à défendre, et témoigner en même temps dans le débat de 
fair-play et d'objectivité. Vus sous cet angle, les effets de la concession sont à rapprocher de ceux 
que  l'on  obtient  en  n'éliminant  pas  systématiquement  d'un  exposé  toutes  les  circonstances 
défavorables  (1).  La  concession  sera  cependant  désastreuse  si  elle  introduit  une  brèche  dans  un 
ensemble dont tous les éléments sont censés solidaires. Par contre, elle n'aura que de bons effets si 
elle porte sur des éléments secondaires. Aussi, loin de s'irriter des questions irrelevantes, doit-on y 
voir une occasion favorable de faire des concessions : 
\bigskip
... la plupart du temps, la confusion  de ceux qui posent les questions se trouve accrue, si, après 
que  toutes  les  propositions  de  ce  genre  leur  ont  été  concédées,  ils  n'arrivent  pas  à  tirer  de 
conclusion (2). 
\bigskip
La concession donne parfois lieu à une figure, l'épitrope, par laquelle, dit Vico, 
\bigskip
\bigskip
\bigskip
348 
\bigskip
 
nous  concédons  à  l'adversaire  des  choses  même  iniques,  même  fausses,  même  ineptes  ou 
douteuses, par surabondance de droit (3)... » 
\bigskip
(1) Cf. Bentham, Œuvres t. I : Traité des sophismes politiques, p. 473.  
(2) Aristote, Topiques, liv. II, chap. 5, 112 a. 
(3) Vico, Instituzioni oratorie, p. 148. 
\bigskip
P 647 : « Bien souvent, l'un des interlocuteurs semble sommer l'autre, soit de reconnaître le bien-
fondé  d'une  position  (avouez  que  j'ai  raison  sur  ce  point),  soit  de  reconnaître  qu'il  a  certaines 
inclinations qui expliquent son attitude (avouez que vous aimez le paradoxe), soit de reconnaître 
qu'il  est  partisan  de  certaines  idées («  avouez que vous  êtes  réactionnaire »,  dira  le  libéral  à  son 
interlocuteur  conservateur).  Dans  le  premier  cas  seul,  on  cherche  à  saisir  un  accord  qui  pourra 
servir  directement  au  bien-aller  de  la  discussion.  Dans  le  second,  on  cherche  à  minimiser  les 
arguments de l'adversaire, en en diminuant le sérieux et la portée. Dans le dernier, on insinue que 
s'il  reconnaît  être  réactionnaire,  la  discussion  en  sera  facilitée  :  sur  cette  position  plus  conforme 
aux faits, la discussion pourra prendre un tour nouveau. Dans tous les cas, l'aveu que l'on sollicite 
est considéré comme une concession, qui doit bénéficier  dans une certaine mesure à l'adversaire, 
sans quoi on n'oserait l'exiger (1). 
\bigskip
Chaque  fois  que  l'on  suit  l'interlocuteur  sur  son  propre  terrain  on  lui  fait  une  concession,  mais 
celle-ci peut être pleine d'embûches. 
\bigskip
L'une  d'elles  consiste  à  reconnaître  que  la  position  de  l'adversaire  ne  peut  être  infirmée  ;  on 
renonce à la combattre sur un certain plan; mais on montrera en même temps le peu d'importance 
de celui-ci. C'est l'attitude bien connue des néo-positivistes à l'égard des énoncés métaphysiques. 
\bigskip
Une autre forme est de concéder pour renchérir aussitôt on reconnaît que l'opinion que l'on vous 
attribue  serait  erronée,  on  nie  même  l'avoir  exprimée,  mais  c'est  pour  en  formuler  une  plus 
désagréable. 
\bigskip
Moi, te reprocher l'hospitalité d'Alexandre ?... je ne suis pas assez fou pour cela ; à moins qu'on 
ne doive appeler les moissonneurs ou ceux qui font quelque autre chose moyennant salaire, amis 
et hôtes de ceux qui les paient (2). » 
\bigskip
(1) Cf. J. Paulhan, Entretien sur des faits divers, pp. 135-138. 
(2) Démosthène, Harangues et plaidoyers politiques, t. IV : Sur la couronne, § 51. 
\bigskip
P 648 : « L'anecdote suivante, prise à Quintilien, relève du comique de la dénégation : 
\bigskip
Domitia  se  plaignait  de  ce  que  Junius  Bassus,  pour  lui  reprocher  son  avarice,  avait  dit  qu'elle 
avait  l'habitude  de  vendre  ses  vieilles  chaussures  :  "  Pas  du  tout,  répondit-il,  je  n'ai  jamais  dit 
cela; j'ai dit que tu avais coutume d'en acheter de vieilles. » (1). 
\bigskip
La  négation,  en  général,  joue  un  rôle  proche  de  celui  de  la  concession  :  on  renonce  à  une 
affirmation que l'on aurait pu soutenir, ou que des tiers soutiennent, mais en gardant une trace de 
celle-ci comme témoin de la richesse d'information et de la clairvoyance de qui a reconnu la non-
valeur d'une proposition. 
\bigskip
Le comique s'attaque à cet aspect de la négation dans ce passage de Tristram Shandy : 
\bigskip
Bref,  mon  ouvrage  digresse,  mais  progresse  aussi,  et  en  même  temps.  Ceci,  monsieur,  ne 
ressemble en rien au double mouvement de la terre (2)... 

De cette signification argumentative de la négation il résulte que la double négation est rarement, 
dans  l'argumentation, l'équivalent de l'affirmation : une formule telle « je suis content de  ne pas 
avoir ne pas y avoir été » indique des rapports complexes, suggère les raisons de nos raisons. La 
double négation peut ramener en une forme condensée toute une argumentation sous-jacente. » 
\bigskip
(1) Quintilien, Vol. II, liv. Vl, chap. III, § 74. 
(2) Laurence Sterne, Vie et opinions de Tristram Shandy, liv. I, chap. XXII, p. 64. 
\bigskip
P  648-649 :  « Les  problèmes  que  pose  l'ampleur,  ses  dangers,  leurs  palliatifs  dépendent  des 
prétentions  de  ceux  qui  argumentent  :  se  contente-t-on  d'une  attitude  passive,  de  marquer 
nettement son désaccord, en refusant, raisons à l'appui, d'adhérer aux vues de l'interlocuteur, on 
au contraire vent-on modifier le point de vue de ce dernier, agir sur ses croyances ? La première 
attitude permet d'utiliser certains arguments qui dans la seconde, seraient dangereux ; elle permet 
de  marquer  tous  les  points  ;  la  seconde  exige  des  ménagements,  procède  par  approches 
successives,  prudentes.  Aussi  les  problèmes  de  l'ampleur  devraient-ils  être repris  et  examinés  en 
fonction de chaque situation argumentative. » 
\bigskip
§ 103. ORDRE ET PERSUASION 
\bigskip
P  649 :  « L'ordre  des  questions  à  traiter,  l'ordre  des  arguments  à  développer,  a,  depuis  toujours, 
sous le couvert des notions d'exposition, de disposition ou de méthode, préoccupé les théoriciens 
de la dialectique et surtout de la rhétorique. 
\bigskip
On ne doit pas s'en étonner, car c'est dans l'argumentation et non dans la démonstration que ces 
problèmes se posent essentiellement. 
\bigskip
Dans une démonstration formelle, on part des axiomes pour aboutir aux théorèmes. If existe donc 
un  ordre.  Mais  son  importance  est  limitée,  parce  que  les  variantes  en  sont  strictement 
équivalentes. Peu importe, en effet, l'ordre dans lequel sont présentés les axiomes, peu importe la 
succession  des  étapes,  à  condition  que  chacune  d'elles  puisse  être  parcourue  en  application  des 
règles d'inférence adoptées. » 
\bigskip
P 649-650 : « Ce n'est que si l'on tient compte de l'adhésion des esprits, si l'on passe d'un point de 
vue  formel  à  un  point  de  vue  psychologique,  argumentatif,  que  l'ordre,  dans  la  démonstration, 
prendra  de  l'importance  :  lorsque,  au  lieu  de  considérer  les  axiomes  comme  arbitraires,  on  se 
préoccupe  de  leur  caractère  évident  ou  acceptable  ;  lorsque,  dans  le  choix  des  étapes,  on  se 
préoccupe  de  la  plus  ou  moins  grande  intelligibilité  de  tel  ordre  démonstratif.  C'est  ainsi  que 
Wertheimer  a,  par  d'intéressantes  expériences,  montré  que  la  compréhension  de  certaines 
démonstrations mathématiques diffère selon la manière dont se présente la figure qui les illustre. 
Les  variantes  ne  sont  plus  alors  équivalentes,  dans  ce  cas,  parce  que  l'on  s'est  éloigné  des 
conditions  purement  formelles  de  la  démonstration  pour  examiner  la  force  persuasive  -  des 
preuves (1). » 
\bigskip
(1) M. Wertheimer, Productive Thinking, chap. I ; “ The area of the Parallelogram”. 
\bigskip
P 650 : « Dans une argumentation, en tout cas, l'ordre ne peut être indifférent : l'adhésion dépend 
en effet de l'auditoire. Or, au fur et à mesure que se déroule l'argumentation, la situation de celui-ci 
se  modifie,  de  par  le  fait  même  de  cette  argumentation,  et  cela  quel  que  soit  l'accueil  fait  aux 
arguments.  Nous  savons  que  le  conditionnement  de  l'auditoire  peut  se  réaliser  par  tous  moyens 
auxiliaires : parfums, musique, rassemblement de masses ; mais il se réalise aussi d'une manière 
discursive. Le discours lie laisse point l'auditeur tel qu'il était au début; il ne modifie pas non plus 
ses croyances d'une manière inéluctable, comme le font les chaînons d'une démonstration. Car, s'il 
\bigskip
\bigskip
\bigskip
350 
\bigskip
en était ainsi, l'ordre n'aurait point tout son poids. C'est précisément parce que les modifications 
de l'auditoire sont à la fois, effectives et contingentes, que l'ordre adopté importe tant. 
\bigskip
Ceci vaut aussi bien pour les diverses incarnations de l'auditoire universel que pour les auditoires 
particuliers.  A  première  vue,  l'ordre  n'importe  pas  pour  l'auditoire  universel.  Mais  l'auditoire 
universel  est,  comme  les  autres,  un  auditoire  concret,  qui  se  modifie  dans  le  temps,  avec  les 
conceptions que s'en fait l'orateur. 
\bigskip
Dans  la  délibération  intime,  l'ordre  semble  également  perdre  toute  importance.  Il  n'en  est  rien 
sans doute. Tout au plus peut-on admettre que la reprise, dans un ordre nouveau, y est plus aisée. 
Elle  peut  même  constituer  la  nouvelle  argumentation  que  l'on  opposera  à  la  première  pour  les 
confronter. » 
\bigskip
P  650-651 :  « Si  l'argumentation  est  essentiellement  adaptation  à  l'auditoire  (2),  l'ordre  des 
arguments  d'un  discours  persuasif  devrait  tenir  compte  de  tous  les  facteurs  susceptibles  de 
favoriser leur accueil auprès des auditeurs. Trois points de vue au moins peuvent être adoptés dans 
le  choix  de  l'ordre  persuasif  :  celui  de  la  situation  argumentative,  c'est-à-dire  de  l'influence 
qu'auront, sur les possibilités argumentatives d'un orateur, les étapes antérieures de la discussion, 
celui du conditionnement de l'auditoire, c'est-à-dire des modifications d'attitude engendrées par le 
discours  ;  celui  enfin  des  réactions  que  suscite,  dans  l'auditoire,  la  saisie  d'un  ordre  dans  le 
discours. » 
\bigskip
(2) Cf. 5, 4 : L'auditoire comme construction de l'orateur. 
\bigskip
P 651 : « Il s'agit, dans les trois cas, des effets sur l'auditoire; ce qui distingue ces trois perspectives, 
c'est  que,  dans  la  première,  on  songe  surtout  aux  prémisses  que  l'auditoire  est  amené 
progressivement à admettre ; dans la seconde  - qui fera l'objet de notre prochain paragraphe - on 
songe surtout aux effets successifs subis par l'auditeur; dans la troisième enfin - qui fera l'objet de 
notre dernier paragraphe - on considère l'ordre du discours comme matière à réflexion. 
\bigskip
Dans une démonstration, tout est donné, soit qu'il s'agisse d'un système hypothético-déductif, soit 
que  les  axiomes  soient  fournis  par  l'intuition  rationnelle  ou  sensible.  Dans  l'argumentation,  par 
contre, les prémisses sont labiles. Au fur et à mesure de l'argumentation, elles peuvent s'enrichir ; 
mais  elles  sont  par  ailleurs  toujours  précaires,  l'intensité  avec  laquelle  on  y  adhère  se  modifie. 
L'ordre  des  arguments  sera  donc  dicté  en  grande  partie  par  le  désir  de  dégager  de  nouvelles 
prémisses, de donner la présence à certains éléments, et d'obtenir certains engagements de la part 
de l'interlocuteur (1). » 
\bigskip
1) Cf. § 27 : Accords propres à chaque discussion. 
\bigskip
P 651-652 : « L'ordre dans lequel ces engagements s'obtiennent n'est pas indifférent. On sait que, 
au  cours  des  interminables  séances  d'après-guerre  entre  représentants  des  États-Unis,  de  la 
France,  de  la  Grande-Bretagne  et  de  l'U.  R.  S.  S.,  ceux-ci  discutèrent  -à  perte  de  vue  sur 
l'établissement  de  l'ordre  du  jour  de  leurs  négociations.  Normalement,  «  négocier  »,  ce  n'est  pas 
discuter et persuader ; c'est faire des concessions mutuelles, et l'ordre des problèmes n'aurait pas 
dû  exercer  tant  d'influence  si,  envisageant  ceux-ci  comme  liés,  on  avait  négocié  avec  le  désir 
d'aboutir.  Mais  le  manque  d'esprit  d'entente  faisait  que  les  conditions  étaient  plutôt  celles  d'une 
discussion que d'une négociation. D'où la prééminence donnée à l'ordre, car toute prise de position 
constituait un engagement sans contrepartie. » 
\bigskip
P  652 :  « C'est  dans  la  discussion  avec  questions  et  réponses  que  ce  rôle  des  engagements  se 
manifeste  le  plus  clairement.  En  effet,  la  modification  de  la  situation,  en  cours  d'argumentation, 
est toujours profonde, qu'il s'agisse de discours continu ou de controverse. Mais dans le premier 
\bigskip
\bigskip
\bigskip
351 
\bigskip
cas, seul l'orateur énonce ses prises de position; dans le second, on s'achemine vers une conclusion 
en se basant sur des points d'appui successifs résultant d'engagements explicites de l'auditeur. 
\bigskip
Le fait de pouvoir questionner, de choisir comme il plaît les questions et l'ordre dans lequel elles 
sont  posées,  est  un  avantage  incontestable  pour  qui  argumente.  C'est  de  l'usage  habile  de  ce 
privilège que dépend l'efficacité de la méthode socratique. » 
\bigskip
P 652-653 : « Les questions tendent souvent à faire faire un choix entre plusieurs possibilités; elles 
ne donnent alors autre chose qu'une information au sujet des opinions de l'interlocuteur, mais la 
réponse est aussi un engagement et souvent une adhésion à ce que prétend l'orateur. Or il ne faut 
pas se dissimuler que, en fin de compte, la discussion vise généralement à modifier une opinion, 
qu'elle  suppose  donc  un  désaccord  fondamental  entre  les  parties.  Aussi,  l'ordre  adopté  pour  les 
questions aura-t-il souvent pour but de masquer aussi longtemps que possible le rapport entre ces 
accords  partiels  et  le  désaccord  fondamental  :  questions  présentées  sans  ordre  apparent  (1), 
questions dont on ne saisit pas l'importance respective (2), voire questions inutiles (3). » 
\bigskip
(1)  Cf.  Aristote,  Topiques,  liv.  VIII,  chap.  I,  156  a  ;  Réfutations  sophistiques,  chap.  15,  174  a  ; 
Schopenhauer, éd. Piper, vol. 6 : Eristische Dialektik, p. 413 (Kunstgriff 9). 
(2) Ibid., p. 412 (Kunstgriff 7). 
(3) Aristote, Topiques, liv. VIII, chap. 1, 157 a ; cf. § 101 : Les dangers de l'ampleur. 
\bigskip
P 653 : « Au cours de la plupart des discussions, les deux interlocuteurs jouissent du privilège de 
questionner et de choisir en partie l'ordre de leurs arguments. Ils jouissent aussi de la possibilité 
de donner à certains éléments la présence. 
\bigskip
A cet égard, une étude récente (4) a montré que si la discussion dirigée, permet, dans un problème 
concret  -  il  s'agissait  d'organisation  du  travail  -,  d'aboutir  à  une  solution  que  le  groupe  juge 
satisfaisante, mais à laquelle, sans cette direction, il n'était pas arrivé, c'est surtout grâce à l'ordre 
mis dans le débat par quelqu'un qui sait vers quelle conclusion il serait souhaitable de s'acheminer. 
Car l'ordre, c'est aussi une des  conditions qui déterminent l'ampleur ; c'est la sélection de ce qui 
sera  pris  en  considération  par  les  participants.  On  ne  veille  pas  seulement  à  ce  que  la  réflexion 
individuelle  ne  s'égare  pas  sur  de  mauvaises  pistes,  mais  aussi  -  et  ce  dernier  point  est  le  plus 
intéressant - à ce que des sentiers utiles ne soient pas prématurément abandonnés, c'est-à-dire que 
l'on donne à certaines prémisses une présence suffisante pour qu'elles servent de point de départ à 
la réflexion. 
\bigskip
Ce souci d'aiguiller la pensée dans des directions propices avant de pousser plus loin, est la base 
sans doute de certaines figures, telles la subjection (sustentatio) (5). On interroge, l'on répond soi-
même aussitôt, mais cette réponse n'est qu'une hypothèse qui, le plus souvent, sera rejetée par son 
auteur. » 
\bigskip
(4)  Norman  R.  F.  Maier,  The  qualites  of  group  decisions  as  lnfluenced  by  the  discussion  leader, 
Human relations, vol. 111, 2, 1950, pp. 162-163. 
(5) Quintilien, Vol. III, liv. IX, chap. II, § 22. 
\bigskip
P  654 :  «   Certains  arguments  ne  peuvent  être  compris,  admis,  que  si  l'on  en  connaît  d'autres. 
L'ordre est alors imposé. Parfois même on peut dire que l'argument est constitué par cet ordre, tels 
l'argument de la direction, la gradation, l'amplification. N'en est-il pas de même de tout argument 
?  Il  n'y  a  presque  jamais  de  modification  d'ordre  qui  ne  soit,  en  même  temps,  modification  de 
l'argument ou mieux création d'un argument nouveau. Même dans une démonstration formelle, où 
les variantes d'ordre sont équivalentes, les étapes ne sont pas seulement modifiées dans leur suite 
mais,  en  même  temps,  les  opérations  effectuées  peuvent  être  autres.  Dans  l'argumentation,  la 
modification d'ordre n'est presque jamais une simple permutation. Ceci s'applique même à ce que 
\bigskip
\bigskip
\bigskip
352 
\bigskip
l'on pourrait considérer comme éléments de l'argumentation. Puisqu'il s'agit, en général, dans une 
argumentation, de s'assurer de fermes prémisses, on peut admettre que l'exposé  de faits, c'est-à-
dire  de  ce  qui  jouit  du  plus  large  accord,  sera  placé  favorablement  au  début  d'un  discours.  La 
plupart des mémoires scientifiques, des exposés politiques, juridiques, procèdent de cette manière. 
Mais il ne faut pas oublier que les faits, s'ils jouent un rôle important comme élément d'accord, ne 
sont souvent admis que parce que leur interprétation reste ouverte. Comme le remarque fort bien 
Quintilien : 
\bigskip
« Le meurtre a été commis par toi, car tes vêtements étaient ensanglantés. » Il y a moins de force 
dans l'argument si l'accusé avoue, que si on le convainc de mensonge. S'il avoue, bien des raisons 
peuvent expliquer le sang sur les vêtements (1)...  
\bigskip
Les  mêmes  faits,  précédés  de  leur  interprétation,  ne  jouiraient  sans  doute  plus  de  l'adhésion 
unanime et susciteraient des dissociations entre apparence et réalité. C'est dire que la place donnée 
aux éléments modifie leur signification. » 
\bigskip
(1) Quintilien, Vol. II, liv. V, chap. XII, § 3. 
\bigskip
P 654-655 : « Les arguments peuvent cependant être envisagés, dans une certaine mesure, comme 
des énoncés distincts qui interagissent certes, mais que l'on peut disposer avec une grande marge 
de liberté. Ainsi, l'éloge d'un personnage peut, selon les cas, précéder ou suivre l'affirmation qu'on 
le  propose  comme  modèle.  De  même  les  arguments  convergents  peuvent  être  groupés  ou 
éparpillés sans que cet éparpillement exclue l'effet de convergence. La solution choisie ne laissera 
pourtant  jamais  de  retentir  sur  l'argumentation.  Le  groupement  des  arguments,  notamment, 
accentuera  l'effet  de  convergence,  tandis  que  l'éparpillement  l'atténuera.  Aussi  dans  une 
accusation, la personne de l'accusé pourra utilement constituer un centre vers lequel convergent de 
façon  serrée  toutes  les  flèches,  alors  que,  dans  la  défense,  on  fera  en  sorte  que  l'ensemble  des 
arguments  que  l'on  réfute  apparaisse  comme  une  mosaïque  de  pièces  détachées  dont  le  lien  est 
aussi  ténu  que  possible  (1).  De  même,  groupera-t-on  utilement  les  exemples  pour  favoriser  la 
généralisation  ;  de  même  encore  groupera-t-on  utilement  les  arguments  de  l'adversaire  entre 
lesquels on veut montrer une incompatibilité. En général, dans une discussion, les raisons qui font 
adopter  un  ordre  par  une  des  parties  devraient  tendre  à  faire  adopter  un  ordre  différent  par  la 
partie adverse. Mais d'autres considérations détourneront souvent de pareil bouleversement (2). » 
\bigskip
(1) Cf. Quintilien, Vol. II, liv. V, chap. XIII,  
(2) Cf. § 105 : Ordre et méthode. 
\bigskip
§ 104. ORDRE DU DISCOURS ET CONDITIONNEMENT DE L'AUDITOIRE 
\bigskip
P  655-656 :  « On  peut  réduire  un  exposé  démonstratif,  tels  ceux  des  traités  de  géométrie,  à 
l'énoncé  de  la  thèse  suivi  de  sa  démonstration.  Un  discours  argumentatif  sera  presque  toujours 
plus complexe. On l'a reconnu de tout temps, et Platon énumère, avec une complaisance destinée à 
les  ridiculiser,  les  parties  du  discours  prônées  par  les  sophistes  (1).  ;  Aristote  n'est  guère  moins 
sévère  (2)  ;  la  plupart  des  auteurs  anciens  admettent  néanmoins  que  le  discours  judiciaire 
comporte  normalement,  au  bas  mot, exorde,  narration,  preuve,  réfutation, conclusion,  épilogue  ; 
dans le discours délibératif, les deux premières parties seraient moins utiles (3). » 
\bigskip
(1) Platon, Phèdre, 266 d-267 a. 
(2) Aristote, Rhétorique, liv. 111, chap. XIII, §§ 3-5, 1414 a-1414 b. 
(3) R. Volkmann, Rhetorik der Griechen und Römer, p. 33. 
\bigskip
P 656 : « Il est remarquable que, parmi les parties du discours, celle, qui à première vue, paraîtra la 
moins  utile,  l'exorde,  a  cependant  retenu  l'attention  de  tous.  Aristote,  Cicéron,  Quintilien,  en 
traitent  longuement  (4)  ;  l'auteur  de  la  Rhétorique  à  Herennius  se  vante  d'en  avoir  reconnu  le 
\bigskip
\bigskip
\bigskip
353 
\bigskip
premier  certaines  modalités  (5).  Or  l'exorde  est  la  partie  du  discours  qui  vise  le  plus 
spécifiquement à agir sur les dispositions de l'auditoire ; c'est pourquoi il nous retiendra ici. 
\bigskip
Son  but  sera  de  se  concilier  l'auditoire,  de  capter  la  bienveillance,  l'attention,  l'intérêt  (6).  Il 
fournira  aussi  certains  éléments  d'où  naîtront  des  arguments  spontanés  ayant  le  discours  et 
l'orateur pour objet. » 
\bigskip
(4) Aristote, Rhétorique, liv. III, chap. XIV, 1414 b-1416 a ; Cicéron, De Inventione, liv. 1, ~~ 20 et 
suiv. ; Quintilien, Vol. II, liv. IV, chap. 1. 
(5) Rhétorique à Herennius, liv. I, 16. 
(6) Cicéron, De Oratore, liv. II, § 323 ; Quintilien, Vol. II, liv. IV, chap. I, §5. 
\bigskip
P  656-657 :  « Aristote  le  compare  au  prologue  et  au  prélude  (7),  ce  qui  semble  en  faire  un  hors-
d'œuvre,  dont  la  signification  serait  surtout  esthétique  (8).  Mais  dans  bien  des  cas  il  est 
indispensable à l'effet persuasif du discours. Il assure les conditions préalables à l'argumentation 
(9).  En  effet,  tandis  qu'il  peut  être  écourté,  et  même  supprime  quand  ces  conditions  préalables 
sont complètement assurées, il devient indispensable s'il faut compléter ces conditions sur tel ou 
tel point, et spécialement en ce qui concerne la qualité de l'orateur, ses rapports avec l'auditoire, 
l'objet ou l'opportunité du discours. » 
\bigskip
(7) Aristote, Rhétorique, liv. III, chap. XIV, § I, 1414 b. 
(8) A. Ed. Chaignet, La rhétorique ei son histoire, pp. 359-360. 
(9) Cf. I- Partie : Les cadres de l'argumentation. 
\bigskip
P  657 :  « L'orateur  tentera,  dans  son  exorde,  de  faire  connaître  sa  compétence,  son  impartialité, 
son honnêteté  
\bigskip
car c'est surtout aux honnêtes gens qu'on accorde son attention (1). 
\bigskip
Si le discours prétend à convaincre l'auditoire universel, l'exorde lie sera point, pour autant, exclu : 
l'orateur montrera surtout son respect des faits, son objectivité. 
\bigskip
De  simples  remarques  préliminaires  donneront  parfois  à  l'orateur  un  prestige  utile.  Si  Robert 
Browning fait commencer l'apologie de l'évêque Blougram par quelques allusions méprisantes à la 
décoration architecturale de son église, sans doute est-ce manière, pour le poète, de situer discours 
et  personnages  dans  l'esprit  de  son  lecteur,  mais  c'est  aussi  -  et  de  ce  double  rôle  vient  l'entière 
justification  de  ces  quelques  vers  -  une  sorte  d'exorde  conférant  à  l'évêque,  aux  yeux  de  son 
interlocuteur, les avantages d'un homme raffiné, au goût délicat (2). 
\bigskip
L'orateur  s'efforcera  surtout  de  mettre  en  valeur  les  qualités  dont  on  pourrait  douter  et  dont 
l'absence  nuirait  à  son  crédit  ;  celui  qui  est  accusé  communément  d'une  trop  grande  habileté, 
tentera  de  gagner  la  confiance  du  publie  ;  celui  qui  par  sa  situation  sociale,  ses  intérêts,  ses 
antécédents  est  supposé  hautain,  étranger  ou  hostile  à  son  auditoire,  commencera  par  démentir 
pareil soupçon en insistant sur sa communion avec l'auditoire (3). L'allusion  à l'amitié entre deux 
peuples,  l'allusion  à  un  fait  de  culture  commun,  une  citation  bien  choisie  suffiront  à  susciter  la 
confiance, en montrant qu'il y a entre orateur et auditoire une communauté de valeurs. » 
\bigskip
(1) Aristote, Rhétorique, liv. III, chap. XIV, § 7. 
(2) R. Browning, Poems, Bishop Blougram's Apology, pp. 136-137. 
(3) Cf. Dale Carnegie, L'art de parler en publie et de persuader dans les affaires. pp. 228, 230. 
\bigskip
P 658 : « L'exorde sera toujours adapté aux circonstances du discours, à l'orateur et à l'auditoire, à 
l'affaire traitée, aux adversaires éventuels. 
\bigskip
\bigskip
\bigskip
354 
\bigskip
 
Rien n'illustre mieux cette exigence que la déclaration préliminaire « je ne suis pas un orateur », et 
autres  précautions  plus  ou  moins  équivalentes.  Recommandable  souvent,  elle  est  sévèrement 
critiquée  par  Dale  Carnegie  (1).  C'est  que,  si  elle  permet  d'éviter  l'accusation  de  procédé  (2), 
d'éviter qu'une partie de la force d'un argument ne soit reportée sur le talent de l'orateur (3), elle 
ne peut convenir à qui, sans grande réputation et sans  y être obligé, prend la parole ou la plume. 
Dans ce cas, la difficulté sera d'avoir une audience suffisante, et l'exorde qui insiste sur ce que l'on 
est inhabile ou incompétent n'y contribue guère. Caton se moquait de cet auteur qui commençait 
par s'excuser d'écrire en grec alors que rien ne l'y contraignait (4). 
\bigskip
Dans  certaines  circonstances,  loin  de  minimiser  son  habileté  oratoire,  l'orateur  peut  même,  s'il 
jouit d'une réputation suffisante, s'en prévaloir. Tel Isocrate, au début du  Panégyrique d'Athènes, 
et cela pour mieux, dans la péroraison, montrer que, quel que soit son talent, sa tâche le dépasse 
(5). 
\bigskip
L'exorde qui se réfère à l'auditoire visera à stimuler l'amourpropre de celui-ci, en faisant état de ses 
capacités, de son bon sens, de sa bonne volonté. Le prédicateur qui s'adresse publiquement à Dieu 
pour lui demander l'ouverture des coeurs (6), reconnaissant d'ailleurs ainsi que son argumentation 
ne  sera  point  une  démonstration  contraignante,  dispose  favorablement  l'auditoire  par  cette 
invocation même. » 
\bigskip
(1) Dale Carnegie, ibid., p. 207. 
(2) Cf. § 96 La rhétorique comme procédé. 
(3) Cf. § 98 L'appréciation de la force des arguments facteur d'argumentation. 
(4) Cité par Aubrey Gwynn, Roman education front Cicero Io Quintilian, p. 45, d'après POLYBE, 
XXXIX, 1 (éd. Büttner-Wobst). 
(5) Isocrate, Discours, t. II : Panégyrique d'Athènes, §§ 13, 187. 
(6) Cf. Bossuet, Sermons, vol. II : Sur la prédication évangélique, p. 50. 
\bigskip
P 658-659 : « L'exorde qui se réfère au sujet attirera l'attention sur l'intérêt que ce dernier présente 
par  son  importance,  par  son  caractère  extraordinaire,  paradoxal,  par  le  fait  qu'il  est  négligé, 
incompris  ou  déformé  (1).  On  traitera  aussi  de  l'opportunité  du  discours,  en  montrant  pourquoi 
c'est le moment de parler, en quoi les circonstances imposent de prendre position. L'exorde variera 
suivant que l'affaire est « noble, confuse, paradoxale ou honteuse » (2). » 
\bigskip
(1) Cf. Richard D. D. Whately, Elements of Rhetoric, Part 1, chap. IV « Introductions », pp. 109 et 
suiv. 
(2) Principia Rhetorices, dans Appendice à saint Augustin, Patrologie latine, t. XXXII, col. 1447 et 
1448 ; cf. Cicéron, De Inventione, liv. I, § 20 ; Rhétorique à Herennius, liv. I, § 5. 
\bigskip
P  659 :  « L'exorde  est  parfois  inutile,  ou  d'autres  techniques  y  suppléent  :  la  présentation  de 
l'orateur par un président de séance n'a d'autre but que de dispenser l'orateur de faire son propre 
éloge (3). Par ailleurs, l'exorde, lorsqu'il est perçu comme procédé destiné à pallier l'insuffisance de 
certaines conditions préalables, peut  attirer sur ce défaut l'attention de l'auditoire. C'est pourquoi 
maints orateurs, faisant état de cette corrélation entre l'ampleur de l'exorde et les lacunes qu'il doit 
combler,  centreront  tout  leur  exorde,  sur  l'inutilité  de  ce  dernier.  Cela  suppose  évidemment  que 
l'auditoire  est  conscient  des  raisons  qui  justifient  ordinairement  l'exorde.  Une  fois  de  plus  nous 
voyons  ici  que  l'argumentation  suppose  souvent  chez  l'auditoire  la  connaissance,  intuitive  au 
moins, de ses règles mêmes. 
\bigskip
Ajoutons que, lors d'une articulation importante du discours, l'orateur introduit parfois un nouvel 
exorde,  spécialement  approprié.  V.  Goldschmidt  a  noté  avec  raison  que,  dans  les  dialogues 
platoniciens : 
\bigskip
\bigskip
\bigskip
355 
\bigskip
 
Ces  invocations  solennelles  des  dieux  sont  tout  autre  chose  que  des  fioritures  littéraires  ou 
dramatiques et, dans bien des cas, soulignent l'importance philosophique des passages (4). » 
\bigskip
(3) Cf. § 72 : Le discours comme acte de l'orateur. 
(4) V, Goldschmidt, Le paradigme dans la dialectique platonicienne, note, p. 16. 
\bigskip
P  659-660 :  « L’auditoire  ayant  été  préparé  à  écouter  ce  qui  constitue  la  matière  propre  du 
discours,  faut-il  commencer  par  indiquer  la  thèse  que  l'on  défendra  ou,  au  contraire,  faut-il 
conclure  après  avoir  développé  ses  raisons  ?  Dans  les  Partitions  oratoires,  Cicéron  conseille  de 
procéder différemment selon le genre d'argumentation : 
\bigskip
L'argumentation  a deux modes, dont l'un tend directement à convaincre,  tandis que l'autre fait 
un  détour  et  s'adresse  à  l'émotion.  Directement,  lorsqu'elle  a  exposé  un  point  à  faire  admettre, 
qu'elle  a  avancé  les  raisons  sur  lesquelles  elle  s'a  je  et,  celles-ci  établies,  est  revenue  au  point 
exposé et a conclu. L'autre argumentation, la seconde, suit une marche pour ainsi dire inverse et 
contraire ; elle  avance d'abord les raisons choisies et les établit solidement, puis à quand elle a 
vivement ému les âmes, elle lance enfin ce qu'elle aurait û exposer en commençant (1). » 
\bigskip
(1) Cicéron, Partitiones Oratoriae, § 46. 
\bigskip
P 660 : « La thèse qui ne demande pas de préparation particulière de l'auditoire, parce qu'elle ne 
contient  rien  d'extraordinaire  ou  de  choquant,  devrait  être  énoncée  dès  l'abord  (2).  La  thèse 
oriente  le  discours,  mais  elle  est  aussi  une  prise  de  position,  un  engagement  de  l'orateur.  Son 
énoncé immédiat a l'avantage d'éclairer les auditeurs ; c'est une occupation du terrain. Mais une 
autre tactique consiste pour l'orateur à retarder l'engagement et à formuler sa thèse en fonction du 
déroulement de la discussion, de façon à tenir compte des objections et à se présenter alors avec 
une proposition qui aura toutes chances d'être admise (3). L'avantage de parler en premier ou en 
dernier  lieu  doit  être  examiné  en  fonction  des  considérations  qui  précèdent.  Ce  sont  elles,  en 
partie, qui détermineront l'ordre des arguments dans le discours. » 
\bigskip
(2) Richard D. D. Whately, Elements of Rhetoric, Part I, chap. III, § 4, p. 87.  
(3) Cf. T. Kotarbinski, Traktat a dobrej robocie (Traité du bon travail), chap. XIII. 
\bigskip
P 660-661 : « Dans la mesure où cet ordre peut être librement fixé par l'orateur, l'un des facteurs 
dont  on  tiendra  compte  est  la  force  respective  des  arguments.  Quand  celle-ci  est  irrésistible,  on 
peut  serrer  l'argumentation,  se  contenter  de  l'argument  dont  on  est  sûr  qu'il  emportera  la 
conviction. Mais ce n'est que très rarement le cas (1). Quand on dispose, pour étayer sa thèse, d'un 
certain nombre d'arguments, comment faut-il les disposer ? » 
\bigskip
(1) Cf. § 100 : L'ampleur de l'argumentation. 
\bigskip
P 661 : « Trois ordres ont été envisagés : l'ordre de force décroissante, l'ordre de force croissante et 
enfin,  le  plus  recommandé,  l'ordre  homérique,  ou  nestorien,  appelé  ainsi  parce  que  Nestor  avait 
placé  au milieu ses troupes les moins sûres  (2), et selon lequel il faut commencer et finir par les 
arguments les plus forts (3). 
\bigskip
L'inconvénient  de  l'ordre  croissant,  c'est  que  la  présentation,  pour  débuter,  d'arguments 
médiocres, peut indisposer l'auditeur et le rendre rétif. L'inconvénient de l'ordre décroissant est de 
laisser les auditeurs sur une dernière impression, souvent la seule restée présente à leur esprit, qui 
soit défavorable. C'est pour éviter ces deux écueils que l'on préconise l'ordre nestorien, destiné à 
mettre en valeur, en les offrants d'emblée ou en dernier lieu, les arguments les plus solides, tous les 
autres étant groupés au milieu de l'argumentation. 
\bigskip
\bigskip
\bigskip
356 
\bigskip
 
Remarquons  que  ces  considérations  supposent  que  la  force  des  arguments  reste  la  même  quelle 
que soit leur place dans le discours. Or bien souvent ce n'est que grâce à la préparation au moyen 
d'arguments  préalables  qu'un  argument  paraÎtra  fort.  C'est  ainsi  que,  dans  le  jules  César  de 
Shakespeare, Antoine ne révèle qu'à la fin du discours l'argument massue, le testament de César en 
faveur  du  peuple,  et  cela  après  avoir  créé  tout  le  contexte  qui  ferait  attribuer  à  ce  testament 
l'interprétation souhaitée (4). » 
\bigskip
(2) Cf. Homère, Made, chap. IV, v. 297 et suiv. 
(3) Cicéron, De Oratore, liv. II, § 313 ; Rhétorique à Herennius, liv. 111, § 18 ; Quintilien, Vol. II, 
liv. V, chap. XII, § 14; vol. 111, liv. VII, chap. 1, § 10 ; cf. R. Volkmann, Hermagoras oder Elemente 
der  Rhetorik,  p.  197  ;  cf.  Argument  de  Libanios,  §  6,  et  Argument  anonyme,  §  5,  au  sujet  du 
plaidoyer sur la couronne, Démosthène, Harangues et plaidoyers politiques, t. lV. 
(4) Shakespeare, Julius Caesar, acte III, sc. II 
\bigskip
P 661-662 : « L'ordre des arguments devra donc être tel qu'il leur donnera la plus grande force : on 
débutera généralement par celui dont la force est indépendante de celle des autres. Dans la double 
défense,  portant  à  la  fois  sur  le  f  ait  et  le  droit,  l'ordre  n'est  pas  indifférent  :  on  commencera 
toujours  par  la  défense  la  plus  forte,  en  espérant  que  la  conviction  établie  par  le  premier  point 
contribuera à faire accepter le second (1). En général, il faut présenter les arguments dans un ordre 
tel qu'ils paraissent plausibles étant donné ce que l'on sait déjà des éléments du débat. Dans son 
apologie  du  christianisme,  Pascal  recommande  un  ordre  où  la  preuve  de  la  vérité  ne  vient  que 
lorsque a été créé le cadre qui la fera admettre le plus facilement : 
\bigskip
... il faut commencer par montrer que la religion n'est point contraire à la raison, vénérable, en 
donner  respect  ;  la  rendre  ensuite  aimable,  faire  souhaiter  aux  bons  qu'elle  fût  vraie;  et  puis 
montrer qu'elle est vraie (2). » 
\bigskip
(1) Cf. § 100 : L'ampleur de l'argumentation. 
(2) Pascal, Œuvre, Bibl. de la Pléiade, Pensées, 1 (27), p. 823 (187 éd. Brunschvicg). 
\bigskip
P  662 : « Quand  une  objection  grave peut  peser  sur  tout  le  déroulement  du  discours, il  ne  sert  à 
rien  d'avancer  des  arguments  qui  seraient  tous  interprétés  en  fonction  de  cette  objection.  Il  faut 
tout  d'abord  réfuter  celle-ci  pour  laisser  le  champ  libre  à  des  interprétations  plus  favorables  (3). 
Quintilien avait conseillé,  pour les mêmes raisons, de commencer par réfuter une accusation qui 
fait  planer  un  doute  sur  l'intégrité  morale  de  l'accusé,  à  moins  que  certains  griefs  moins  graves 
soient  notoirement  faux;  dans  ce  cas,  il  faut  commencer  par  les  réfuter,  pour  enlever  tout  crédit 
aux accusateurs (4) » 
\bigskip
(3) Cf. Richard D. D. Whately, Elements of Rhetoric, Part 1, chap. 111, § 6, P. 90. 
(4) Quintilien, Vol. III, liv. VII Chap. I, § 11. 
\bigskip
P 662-663 : “Dans certains cas, on n'attendra pas que l'accusation soit formulée ; on la réfutera par 
anticipation.  Cette  procédure  n'est  pas  sans  inconvénient.  Elle  oblige  à  énoncer  l'accusation, 
attribuant ainsi à l'adversaire des idées qu'il n'eût pas toujours eues ou qu'il n'eût pas toujours osé 
exprimer.  La  réfutation  anticipée  implique  que  l'accusation  est  normale,  qu'il  faut  donc  en  tenir 
compte.  Elle  peut  donner  lien  de  ce  chef  à  des  effets  comiques,  comme  le montre  cette  anecdote 
citée par Quintilien : 
\bigskip
Fulvius  Propinquus,  a  qui  le  légat  de  l'empereur  demandait  si  les  documents  qu  produisait 
étaient signés, répondit : « Oui, Monsieur, et la signature n'est pas fausse (1). » 
\bigskip
(1) Quintilien, Vol. II, liv. Vl, chap. III, § 100. 
\bigskip
\bigskip
\bigskip
357 
\bigskip
 
P  663 :  « Quand  elle  prend  la  forme  d'une  objection  que  l'on  se  fait  à  soi-même,  la  réfutation 
anticipée peut donner lieu à une figure, la prolepse, argumentative au premier chef (2). 
\bigskip
Cette  réfutation  anticipée  peut  prendre  aussi  la  forme  d'une  concession.  Nous  avons  déjà  vu  les 
avantages de celle-ci (3). Postérieure à une remarque de l'adversaire, elle constitue un compromis. 
Mais antérieure, placée notamment au début d'un discours, elle consiste à se défendre, par avance, 
d'avoir négligé une valeur ou un fait d'importance. Elle partage les avantages et les inconvénients 
de  la  réfutation  anticipative.  Elle  peut  aussi  suivre  l'énoncé  de  certains  arguments  faibles, 
témoigner de la bonne foi de l'orateur. C'est, selon Quintilien, une des raisons qui peut engager à 
débuter par des arguments faibles, que l'on abandonnera aussitôt (4). On voit ici un lien particulier 
très  étroit,  entre  la  place  donnée  aux  arguments  et  le  rôle  qu'on  leur  assigne  dans  le 
conditionnement de l'auditoire. » 
\bigskip
(2) Cf. § 41 : Figures de rhétorique et argumentation. 
(3) Cf. 9 102 : Les palliatifs aux dangers de l'ampleur. 
(4) Quintilien, Vol. III, liv. VII chap. I, § 16. 
\bigskip
P  663-664 :  « La  disqualification  de  l'adversaire,  s'il  y  a  lieu  d'y  procéder,  se  placera  à  la  fin  du 
discours  dans  l'accusation,  en  tête  dans  la  réplique  (5).  Les  anciens  orateurs  avaient  l'habitude, 
dans  les  débats  judiciaires,  de  terminer  leurs  discours  par  une  attaque  contre  celui  qu'ils 
accusaient, de façon à enlever, par avance, toute valeur à sa plaidoirie ; celui qui se défendait, par 
contre,  devait  regagner,  dans  l'exorde,  la  bienveillance  de  ses  auditeurs  et  juges,  s'efforçant  de 
modifier l'état d'esprit défavorable créé par la péroraison  de son  adversaire. Dans ce cas, comme 
dans  la  plupart  des  autres,  l'ordre  des  discours  étant  adaptation  à  l'auditoire  et  à  la  situation 
argumentative,  toutes  les  règles  que  l'on  pourrait  formuler  à  cet  égard  sont  fonctionnelles.  Des 
préceptes  plus  précis  ne  sont  que  la  codification  de  ce  qui  réussit  normalement  mais  ce  normal 
dont ils s'inspirent n'a lui-même aucune fixité. » 
\bigskip
(5) Cf. Aristote, Rhétorique, liv. III, chap. XIV, § 7, 1414 b; Cf. Whately, Elements of Rhetoric, Part 
11, chap. III, § 5, p. 169. 
\bigskip
P 664 : « La tactique utilisée variera d'ailleurs selon les caractères de l'auditoire. Aristote avait noté 
que certains auditeurs témoignent de plus de sens critique à la fin qu'au début d'un débat (1) ; pour 
d'autres  ce  sera  l'inverse.  Les  réactions  que  l'on  escompte  pourront  être  d'ordre  émotif.  On  ira 
jusqu'à  inciter  graduellement  l'auditeur  à  la  colère  (2).  A  mesure  que  la  tactique  utilisée  spécule 
sur  des  faiblesses  de  l'interlocuteur  qui  ne  paraissent  pas  devoir  être  partagées  par  tous,  et 
notamment  par  l'auditoire  universel,  le  succès  que  l'on  obtient  perdra  de  sa  valeur  aux  yeux  des 
tiers.  Il  n'existe  toutefois  pas  de  limite  tranchée  entre  les  techniques  de  l'ordre  destinées  à 
l'auditoire universel et celles qui ne valent que pour un auditeur déterminé, carl'auditoire universel 
coïncidera toujours, par certains traits, avec l'homme réel, concret; il ne s'éloignera d'un auditoire 
particulier qu'autant que la conception que l'on s'en fait transcende certains auditoires particuliers 
déterminés. 
\bigskip
D'autre part, les réactions d'un auditoire donné, même si elles peuvent être interprétées en termes 
psychologiques,  voire  politiques,  n'en  sont  pas  moins  très  souvent  explicables  et  justifiables  par 
des raisons qui pourraient être admises par l'auditoire universel et qui rendent ces réactions dans 
une certaine mesure rationnelles. » 
\bigskip
(1) Cf. Aristote, Topiques, liv. VIII, chap. 1, 156 b. 
(2)  Cf.  Schopenhauer,  éd.  Piper,  vol.  6  :  Eristische  Dialektik,  p.  413  (Kunstgriff  8).  Aristote, 
Réfutations sophistiques, chap. 15, 174 a. 
\bigskip
\bigskip
\bigskip
358 
\bigskip
§ 105. ORDRE ET METHODE 
\bigskip
P 665 : « Sans doute, l'ordre fait l'objet d'un choix qui n'a d'autre règle que la meilleure adaptation 
possible  aux  états  successifs  de  l'auditoire,  tels  que  lès  imagine  l'orateur,  et  ce  dernier  pourra 
même revendiquer comme un droit de 
\bigskip
laisser chaque plaideur employer le plan et la défense qu'il a choisis et adoptés (1). 
\bigskip
Pourtant Démosthène, après avoir exigé ainsi toute liberté, déclare, dans le même plaidoyer : 
\bigskip
...  j'adopterai  le  même  ordre  que  celui  de  la  plainte  pour  vous  parler  de  chaque  point 
successivement, et je n'omettrai rien consciemment (2). 
\bigskip
Simple courtoisie envers l'auditeur pour lui faciliter la tâche ? Coutume à suivre ? 
\bigskip
Ceci nous amène à un point très important : c'est que l'ordre adopté peut être, lui-même, matière à 
réflexions  chez  l'auditeur,  et,  par  ce  biais,  influer  directement  sur  le  résultat  de  l'argumentation. 
Nous avons maintes fois souligné les arguments spontanés ayant le discours pour objet et dont les 
effets se superposent à ceux des arguments énoncés (3). L'ordre des arguments en fournit un cas 
éminent. » 
\bigskip
(1) Démosthène, Harangues et plaidoyers politiques, t. IV : Sur la couronne, § 2. 
(2) Ibid., § 56. 
(3) Cf. § 72 : Le discours comme acte de l'orateur; § 96 ; La rhétorique comme procédé. 
\bigskip
P 665-666 : « Pour être l'objet de réflexions, il faut qu'un ordre puisse être saisi comme tel. Il en 
sera ainsi chaque fois que l'ordre des énoncés est lié à un ordre extérieur au discours, connu des 
auditeurs, ou qui tout au moins peut être immédiatement compris par eux. L'ordre chronologique, 
lorsqu'il est adopté pour l'exposé des faits, serait l'exemple le plus caractéristique d'ordre extérieur 
au  discours.  Il  représente  semble-t-il  la  forme  la  plus  simple  de  cet  «  ordre  naturel  »  qui  a  tant 
préoccupé les théoriciens (1). » 
\bigskip
(1) Cf. Rodolphus Agricola, De inventione dialectica, liv. III, pp. 167 et Suiv.; Juan L. Vives, Obras 
completas, t. II : Arte de hablar, liv. III, chap. III, p. 783. 
\bigskip
P 666 : « Mais cet ordre chronologique est loin d'être le seul qui puisse, pour l'auditeur, servir de 
schème  de  référence.  Ainsi,  la  coutume  oratoire,  elle  aussi,  fournit  des  schèmes  qui,  à  titre  de 
Patrons semblent extérieurs au discours particulier ; et nous voyons qu'il est malaisé de discerner, 
surtout  à  l'audition  du  discours  continu,  la  part  qui  revient  à  l'habitude,  à  la  tradition,  dans  la 
perception du discours comme répondant à un ordre normal. 
\bigskip
L'ordre adopté par l'adversaire n'est pas moins apte à servir de schème de référence. Et aussi telle 
partie  du  discours  déjà  prononcée  par  l'orateur  et  qui  servira  de  schème  argumentatif,  au même 
orateur,  dans  une  seconde  partie  de  son  exposé.  Bien  plus,  il  est  vraisemblable  que  certains 
arguments sont appréhendés en fonction du rythme qu'ils ont suggéré : on peut se demander si le 
sorite chinois ne tire pas une part de son efficacité du schème qu'il amorce : les premiers chaînons 
feraient  saisir  les  suivants  comme  éléments  successifs  d'un  même  cheminement;  il  en  irait  de 
même de certaines analogies, de certaines doubles hiérarchies. 
\bigskip
L'ordre extérieur, tel l'ordre chronologique, coutumier, et aussi bien l'ordre né de l'argumentation, 
constitueraient  de  bonnes  formes  se  déroulant  dans  le  temps,  avec  tous  les  caractères  que  la 
psychologie  «  gestaltiste  »  a  donnés  à  ce  terme,  c'est-à-dire  aisément  saisissables,  satisfaisantes 
pour  l'esprit,  qui  plus  est,  susceptibles  de  ramener  à  elles  les  perceptions  qui  s'en  écarteraient 
légèrement, et aussi de permettre à certains éléments de trouver leur place dans une série. Ainsi, il 
\bigskip
\bigskip
\bigskip
359 
\bigskip
est possible que certains arguments sous-entendus soient compris grâce à la place qu'ils occupent 
dans pareille succession ordonnée. » 
\bigskip
P  667 :  « La  bonne  forme,  par  le  fait  même  qu'elle  se  développe  dans  le  temps,  se  caractérise 
souvent par une intensité croissante, une sommation. C'est le cas par exemple dans la  gradation 
(climax) qui est une figure de l'ordre. La liaison verbale entre clausules, la répétition de certain  s 
termes suggère un accroissement d'intensité. La répétition ne donne pas seulement la présence (1), 
elle  fait  plus.  Comme  le  dit  Quintilien  «  avant  de  monter  la  marche  suivante,  on  s'arrête  sur  les 
précédentes (2). Le passage ci-après, pris à Démosthène, en est souvent donné comme exemple : 
\bigskip
Et il n'est pas vrai que j'aie parlé ainsi sans rédiger de projet de décret ; que j'aie rédigé un projet 
sans aller en ambassade; que je sois allé en ambassade sans persuader les Thébains (3). 
\bigskip
S'agit-il d'actions demandant une détermination croissante ? Ne s'agit-il pas aussi bien de lacune 
de plus en plus faible dans l'action ? Le point de vue diffère sans doute selon les auditeurs. Si l'on 
ne parle guère d'une figure qui serait l'anticlimax, c'est que la perception d'un ordre peut, presque 
toujours, être conçue comme une progression. 
\bigskip
Les  rapports  entre  discours  et  série  extérieure  à  lui  se  préciseront  parfois,  lorsqu'une  liaison  du 
réel  les  unit,  en  un  argument  de  double  hiérarchie  caractérisé  (4).  C'est  ainsi  que  l'ordre  des 
arguments  selon  leur  force  croissante  sera,  par  certains  auteurs,  conseillé  comme  étant  le  plus 
naturel pour la raison que : 
\bigskip
Il semble qu'on est entraîné à cet ordre par une Ici de la nature qui échauffe, exalte et transporte 
l'imagination et le raisonnement comme la voix de l'orateur à mesure qu'il parle (5). » 
\bigskip
(1) Cf. § 42 : Les figures du choix, de la présence et de la communion. 
(2 Quintilien, Vol. III, liv. IX, chap. III, § 55. 
(3) Démosthène, Harangues et plaidoyers politiques. t. IV : Sur la couronne, § 179. 
(4) Cf. 9 76 : L'argument de double hiérarchie. 
(5) A. Ed. Chaignet, La rhétorique et son histoire, p. 401. 
\bigskip
P 667-668 : « Conseil naïf s'il s'agit de préconiser l'ordonnance des arguments selon leur force - car 
nous  avons  vu  que  cette  force  même  dépend  en  grande  partie  de  la  place  des  arguments  -  mais 
remarque  intéressante,  s'il  s'agit  de  montrer  le  rôle  que  jouent  les  doubles  hiérarchies  dans  les 
réflexions  sur  l'ordre.  L'ordre  des  arguments,  du  moment  que  leurs  caractères  permettent  de  les 
percevoir  assez  aisément  comme  étant  insérés  dans  pareille  double  hiérarchie,  sera  par  là  même 
justifié  :  leur  agencement  ne  paraîtra  pas  comme  procédé,  puisqu'il  devient  la  conséquence  d'un 
fait (1). » 
\bigskip
(1) Cf. § 96 : La rhétorique comme procédé. 
\bigskip
P 668 : « Toute indication relative à l'ordre facilitera son appréhension comme tel : cela pourra se 
faire par simple allusion, par exemple allusion à l'ordre coutumier, ou bien encore par la technique 
bien connue de la division, c'est-à-dire l'annonce des parties à traiter : soit parties du discours, soit 
points à débattre, soit preuves qui seront apportées. On a souligné, dans ce dernier cas surtout, les 
inconvénients de la division : pour Quintilien, elle enlève au discours le charme de la spontanéité, 
fait voir de loin certains arguments difficiles à admettre, prive de l'avantage d'une sortie en masse 
(2). Cependant la division a l'avantage de créer, à partir du moment où elle est proposée, et même 
si elle ne correspond à aucun ordre extérieur au discours, un schème de référence. La preuve en est 
que toute infraction à la division semblera une infraction à un ordre admis et devra être justifiée. 
\bigskip
\bigskip
\bigskip
\bigskip
360 
\bigskip
Cette justification est celle que réclame tout changement (3). En effet l'auditeur risque d'attribuer à 
une  rupture  de  l'ordre  attendu,  quel qu'il  soit,  valeur  d'indice  ou  de  signe :  désir  de  brouiller  les 
idées  de  l'auditoire,  volonté  de  mettre  en  évidence  un  argument  considéré  comme  fort,  désir  de 
passer sous silence certaines questions. » 
\bigskip
(2)  Quintilien,  vol.  II,  liv.  IV,  Chap.  V,  §§  4-8,  cf.  Fénelon,  éd.  Lebel,  t.  XXI:  Dialogues  sur 
l'éloquence, pp. 68-71. 
(3) Cf. § 27 : Accords propres à chaque discussion. 
\bigskip
P 668-669 : « L'ordonnance à laquelle on s'attend importe tant, que souvent elle sera adoptée au 
détriment d'une autre tout aussi favorable par ailleurs. C'est à l'attente déjouée qu'il faut rattacher 
le danger des arguments retardés, lesquels perdent de leur force pour n'avoir pas été énoncés en 
temps voulu (1). Sans doute pourra-t-on délibérément rompre avec tout ordre prévu afin de piquer 
la  curiosité,  de  paraître  original',  mais  la  rupture,  loin  de  donner  l'impression  de  naturel  et  de 
sincérité, risque de favoriser la dissociation procédé. » 
\bigskip
        réalité  
\bigskip
\bigskip
(1) Cf. Quintilien, Vol. Ii, liv. IV, chap. V, § 18 ; liv. V, chap. XIII, § 51 vol. III, liv. VII, chap. I, § 11. 
\bigskip
P 669-670 : « Dès que le discours suit un schème perçu comme extérieur à lui, l'ordre ainsi adopté 
apparaîtra,  nous  l'avons  vu,  comme  un  ordre  naturel,  que  ce  soit  l'ordre  chronologique,  ou  celui 
qui correspond à l'exaltation croissante de l'orateur. Mais la réflexion sur l'ordre considéré comme 
naturel  a  été  poussée  beaucoup  plus  avant.  Lorsque  Agricola  (2),  Ramus  (3),  tentent  de  séparer 
nettement  dialectique  et  rhétorique,  réduisant  cette  dernière  à  l'étude  des  moyens  d'expression 
ornés  et  agréables,  ils  transfèrent  dans  la  dialectique  les  problèmes  d'ordre,  d'exposition,  de 
méthode qui étaient traditionnellement traités dans les ouvrages de rhétorique. Sans doute, par là, 
la  rhétorique  pénètre-t-elle  la  dialectique  malgré  l'effort  pour  les  séparer  (4).  Toutefois  les 
problèmes se transforment. En effet on se demande avant tout, et de plus en plus, s'il n'existe pas 
un ordre unique, qui s'impose, celui de la nature des choses, auquel le discours rationnel devrait se 
conformer.  A  la  méthode  de  prudence,  qui  est  relative  à  l'opinion,  on  opposera  la  méthode  de 
doctrine ou de nature «où ce qui est naturellement plus évident doit précéder » (5). La méthode de 
nature,  pour  les  penseurs  classiques,  ce  sera  l'enchaînement  des  raisons  approprié  à  un  ordre 
naturel,  objectif,  inhérent  au  monde  ou  aussi  bien  à  la  pensée,  car  la  méthode  est  censée 
représenter les opérations d'un esprit qui s'adapte au réel. Le modèle de cette méthode universelle 
est généralement emprunté aux sciences ; tout l'effort de Descartes consistera à donner à cet ordre 
naturel l'aspect constructif des mathématiques. » 
\bigskip
(2) Rodolphus Agricola, De inventione dialectica libri tres, liv. II. pp. 132 et suiv. 
(3) P. Ramus, Dialecticae libri duo, Paris, 1560, liv. I, note p. 10; édition de 1566, liv. 1, note p. 156 
(plus développée). 
(4) Cf. G. Morpurgo, La retorica aristotelica e il barocco dans Retorica e Barocco, p. 124. 
(5) P. Ramus, Dialecticae libri duo, Paris, 1560, liv. 11, p. 208. 
\bigskip
P  670 :  « L'unicité  de  cet  ordre  rationnel  le  distingue  fortement,  et  d'un  ordre  argumentatif,  et 
aussi  d'un  ordre  purement  formel,  au  sens  où  l'entendrait  la  logique  moderne.  Démonstration 
formelle et méthode rationnelle ont en commun la rigueur ; mais la seconde prétend à l'objectivité 
; elle est liée à des notions telles clarté, simplicité, et aussi évidence, qui garantissent les prémisses, 
les raisonnements et les conclusions. 
\bigskip
Cet ordre unique, jouissant d'un privilège aussi éminent, nous le retrouverons chez la plupart des 
théoriciens, qui bien que s'éloignant de la pensée classique, conservent ses aspirations. Whately se 
borne  à  affirmer  que  l'ordre  naturel  est  celui  où  ce  qui  est  plus  évident  (obvious)  précède  ce  qui 
suit (1). Pour Tarde, il existe un ordre rationnel des erreurs aussi bien que des vérités : 
\bigskip
\bigskip
\bigskip
361 
\bigskip
 
...  n'y  a-t-il  pas...  parmi  toutes  les  manières  d'exposer  les  dogmes  de  la  religion  la  plus 
extravagante, les mythes de la mythologie la plus fantaisiste, une combinaison plus propre que 
nulle autre à faire sentir la raison d'être de chacun d'eux (2) ? 
\bigskip
Cette  manière  d'exposer  ne  reproduirait  pas  un  ordre  d'apparition,  mais  la  relation  interne 
naturelle, qui unit effectivement les éléments de cette construction. » 
\bigskip
(1) Richard D. D. Whately, Elements of Rhetoric, Part I, chap. III, § 7, p. 107.  
(2) G. Tarde, La logique sociale, p. 180. 
\bigskip
P 670-671 : « L'ordre naturel ou rationnel n'est pas indépendant de tout auditoire, mais adapté à 
l'auditoire universel et à la rationalité qu'on lui attribue. Si l'on considère l'ordre rationnel comme 
unique, c'est qu'on se représente cet auditoire comme une entité abstraite, hors du temps, et non 
comme  un  auditoire  concret,  c'est-à-dire  variable,  en  fonction  de  l'image  que  l'on  s'en  forme (1). 
On oublie que les notions qui servent de base à l'ordre rationnel, telles que clarté, simplicité, ont 
été  psychologiquement  élaborées  et  rendues,  par  la  suite,  absolues.  L'argumentation  rationnelle 
n'est en réalité qu'un cas particulier d'argumentation  ad hominem,  celui que nous avons qualifié 
d'argumentation  ad  humanitatem.  Cependant,  l'idée  d'un  ordre  naturel,  qui  serait  objectif, 
entraînait  la  conséquence  que  le  discours,  dans  toute  la  mesure  où  il  était  autre  chose  que 
l'application d'une méthode conforme à cet ordre, se réduisait à une activité de rechange, un pis-
aller. Ce n'est que si l'entrée du « vray chemin » lui est fermée que le dialecticien : 
\bigskip
se  fera  autre  voye  par  force  d'esprit  et  prudence,  et  cherchera  de  toutes  parts  toutes  aydes  de 
coustume et d'usage (2)... » 
\bigskip
(1) Cf. § 7 : L'auditoire universel.  
(2) Ch. Waddington, Ramus, sa vie, ses écrits et ses opinions, p. 372. 
\bigskip
P 671 : « Cette activité intéresse encore Ramus, car elle est selon lui, partiellement au moins, celle 
du philosophe, aussi bien que du poète, de l'orateur. Elle n'intéressera plus Descartes. 
\bigskip
La recherche d'une méthode naturelle, objective, unique, s'avère presque toujours corrélative d'une 
conception  selon  laquelle  la  rhétorique  est  pure  technique  d'ornement.  En  effet,  cette  méthode 
laisse indéterminée la forme du discours ; tous les éléments variables de celle-ci, tout ce qui n'est 
pas imposé par l'ordre naturel paraît extérieur ; on renonce, sur ce point, à justifier la forme par le 
fond. » 
\bigskip
P  671-672 :  « Le  discours  chez  les  tenants  d'une  méthode  dialectique  naturelle,  universelle, 
conforme à la nature des choses, peut être considéré en lui-même comme une œuvre d'art, comme 
une entité. A cet égard, l'analogie entre un discours et un organisme, par exemple un « être animé 
ayant corps, tête et pieds » (1), est une manière de séparer la forme du discours de son contenu, 
mais en donnant à la forme un ordre structuré, qui lui est propre. L'analogie se borne à affirmer 
une  relation  entre  parties,  sans  fournir  aucune  détermination  quant  à  la  nature  de  ces  relations. 
Elle  envisage  le  discours  comme  quelque  chose  d'isolé,  qui  se  suffit  à  soi-même.  Il  en  va 
pareillement  des  analogies  entre  discours  et  œuvres  d'art  appartenant  à  d'autres  domaines  :  de 
même  que  G.  Dorfles  rapproche  théâtre  et  musique,  par  l'instrumentation  limitée,  la  succession 
des mouvements rythmiques, l'entrée de personnages ou d'instruments, les actions rétrogrades, les 
reprises et transpositions (2), de même peut-on rapprocher de l'un comme de l'autre, le discours, 
voir notamment dans le discours indirect une perspective dans la perspective. Ce sont là analogies 
qui  viendront  à  l'esprit  de  certains  auditeurs  et  pourront  accroître  leur  bonne  volonté,  par 
l'entremise d'une joie esthétique. Elles n'éclairent pas le théoricien de l'argumentation. » 
\bigskip
\bigskip
\bigskip
\bigskip
362 
\bigskip
(1) Cf. Platon, Phèdre, 264 c. 
(2) Gillo Dorfles, Discorso tecnico delle arti, pp. 180-81. 
\bigskip
P 672-673 : « Quant à nous, nous pensons qu'une théorie de l'argumentation ne doit ni rechercher 
une  méthode  conforme  à  la  nature  des  choses,  ni  envisager  le  discours  comme  une  œuvre  qui 
trouve  en  elle-même  sa  structure.  L'une  comme  l'autre  de  ces  conceptions  complémentaires 
séparent fond et forme, oublient que l'argumentation est un tout, destiné à un auditoire déterminé. 
On  passe  ainsi  d'un  problème  de  communication  à  une  ontologie  et  à  une  esthétique,  alors  que 
l'ordre ontologique et l'ordre organique ne sont que deux déviations d'un ordre adaptatif. Ce sont 
les exigences de l'adaptation  à l'auditoire qui doivent guider dans l'étude de l'ordre du discours : 
cette adaptation agira, soit directement, soit par l'entremise des réflexions de l'auditeur au sujet de 
l'ordre: ce qu'il envisage comme ordre naturel, les analogies qu'il perçoit avec un organisme ou une 
œuvre d'art, ne sont qu'arguments parmi d'autres ; l'orateur devra en tenir compte au même titre 
que  de  tous  les  facteurs  susceptibles  de  conditionner  l'auditoire.  Méthode  et  forme  pourront 
prendre,  respectivement,  plus  ou  moins  d'importance  selon  qu'il  s'agit  d'auditoire  particulier, 
technique,  ou  universel.  Mais  une  théorie  de  l'argumentation  qui  ne  ferait  pas  place  à  tous  ces 
éléments conjointement s'éloignera toujours de son objet. La dissociation entre forme et fond, qui 
a  conduit  à  déshumaniser  la  notion  même  de  méthode,  a  conduit  aussi  à  accentuer  l'aspect 
irrationnel  de  la  rhétorique.  Le  point  de  vue  argumentatif  introduira  sans  doute,  dans  des 
questions considérées généralement comme relevant uniquement de l'expression, des vues qui en 
montrent la secrète rationalité. » 
\bigskip
CONCLUSION 
\bigskip
P  675 :  « Ce  n'est  pas  sans  difficulté  que  nous  avons  réduit  aux  dimensions  du  présent  ouvrage 
notre traité de l'argumentation. Loin d'en avoir épuisé la matière, nous en avons à peine entamé, et 
parfois seulement signalé, la richesse. Des schèmes oubliés depuis longtemps, d'autres dont l'étude 
est  toute  récente,  ont  été  éclairés  les uns  par  les  autres  et  intégrés  dans  une  discipline  ancienne, 
mais déformée depuis des siècles et  négligée actuellement. Des problèmes abordés généralement 
d'un point de vue purement littéraire, d'autres dont se préoccupe la spéculation la plus abstraite  - 
qu'elle  relève  de  courants  existentialistes  ou  de  la  philosophie  analytique  anglaise  -  se  trouvent 
situés  dans  un  contexte  dynamique,  qui  souligne  leur  intérêt  et  permet  de  saisir  sur  le  vif  les 
rapports dialectiques de la pensée et de l'action. 
\bigskip
Chacun  des  points,  dont  l'examen  a  été  à  peine  esquissé,  mériterait  une  étude  approfondie.  Les 
diverses espèces de discours, leur variation en fonction des disciplines et des auditoires, la manière 
dont les notions se modifient et s'organisent, l'histoire de ces transformations, les méthodes et les 
systèmes auxquels a pu donner naissance l'adaptation d'ensembles notionnels à des problèmes de 
connaissance, tant d'autres questions qui n'ont été soulevées qu'en passant, présentent à l'étude de 
l'argumentation un terrain de recherches d'une richesse incomparable. » 
\bigskip
P 675-676 : « Toutes ces questions ont été jusqu'à présent, soit entièrement négligées, soit étudiées 
avec une méthode et dans un esprit étrangers au point de vue rhétorique. En effet, la limitation de 
la logique à l'examen des preuves qu'Aristote qualifiait d'analytiques -et la réduction à celles-ci des 
preuves dialectiques, - quand on attachait quelque intérêt à leur analyse, - a éliminé de l'étude du 
raisonnement  toute  référence  à  l'argumentation.  Nous  espérons  que  notre  traité  provoquera  une 
salutaire réaction ; et que sa seule présence empêchera à l'avenir de réduire toutes les techniques 
de la preuve à la logique formelle et de ne voir dans la raison qu'une faculté calculatrice. » 
\bigskip
P 676 : « Si une conception étroite de la preuve et de la logique a entraîné une conception étriquée 
de la raison, l'élargissement de la notion de preuve et l'enrichissement de la logique qui en résulte 
ne peuvent que réagir, à leur tour, sur la manière dont est conçue notre faculté de raisonner. C'est 
pourquoi  nous voudrions conclure par des considérations qui dépassent,  par leur généralité, une 
théorie  de  l'argumentation,  mais  qui  lui  fournissent  un  cadre  mettant  en  relief  son  intérêt 
\bigskip
\bigskip
\bigskip
363 
\bigskip
philosophique.  De  même  que  le  Discours  de  la  Méthode,  tout  en  n'étant  pas  un  ouvrage  de 
mathématiques,  assure  à  la  méthode  «  géométrique  »  son  champ  d'application  le  plus  vaste  - 
quoique  rien  n'empêche  d'être  géomètre  sans  être  cartésien  -  de  même  les  vues  que  nous 
proposons,  quoique  la  pratique  et  la  théorie  de  l'argumentation  n'en  soient  pas  soli  daires, 
accordent  à  l'argumentation  une  place  et  une  importance  qu'elles  ne  possèdent  nullement  dans 
une vision plus dogmatique de l'univers. 
\bigskip
Nous combattons les oppositions philosophiques, tranchées et irréductibles, que nous présentent 
les  absolutismes  de  toute  espèce:  dualisme  de  la  raison  et  de  l'imagination,  de  la  science  et  de 
l'opinion,  de  l'évidence  irréfragable  et  de  la  volonté  trompeuse,  de  l'objectivité  universellement 
admise  et  de  la  subjectivité  incommunicable,  de  la  réalité  qui  s'impose  à  tous  et  des  valeurs 
purement individuelles. » 
\bigskip
P  676-677 :  « Nous  ne  croyons  pas  à  des  révélations  définitives  et  immuables,  quelle  qu'en  soit 
d'ailleurs la nature ou l'origine; les données immédiates et absolues, qu'on les appelle sensations, 
évidences rationnelles ou intuitions mystiques, seront écartées de notre arsenal philosophique. Ce 
rejet n'implique pas, cela va sans dire, que nous écartons l'effet, sur nos opinions, de l'expérience 
ou du raisonnement, mais nous 'ne ferons pas nôtre la prétention exorbitante d'ériger en données 
définitivement claires, inébranlables, certains éléments de connaissance, identiques dans tous les 
esprits normalement constitués, indépendants des contingences sociales et historiques, fondement 
des vérités nécessaires et éternelles. » 
\bigskip
P  677 :  « Cette façon  de  dissocier  certains  éléments  irréfragables,  de  l'ensemble  de  nos  opinions, 
dont  personne  cependant  n'a  contesté  le  caractère  imparfait  et  perfectible,  de  les  rendre 
indépendants  des  conditions  de  perception  et  d'expression  linguistique,  a  pour  but  de  les 
soustraire à toute discussion et à toute argumentation. Concevoir tout progrès de la connaissance 
uniquement  comme  une  extension  du  champ  couvert  par  ces  éléments  clairs  et  distincts,  aller 
même  jusqu'à  s'imaginer  que,  à  la  limite,  dans  une  pensée  parfaite,  imitant  la  pensée  divine,  on 
pourrait éliminer de la connaissance tout ce qui ne se conformerait pas à cet idéal de clarté et de 
distinction, c'est vouloir réduire progressivement le recours à l'argumentation jusqu'au moment où 
son  usage  deviendrait  complètement  superflu.  Provisoirement,  son  utilisation  stigmatiserait  les 
branches du savoir qui s'en servent, comme des domaines imparfaitement constitués, encore à la 
recherche  de  leur  méthode,  et  ne  méritant  pas  le  nom  de  science.  Rien  d'étonnant  que  cet  état 
d'esprit ait détourné les logiciens et les philosophes de l'étude de l'argumentation, censée indigne 
de  leurs  préoccupations,  la  laissant  pour  compte  aux  spécialistes  de  la  publicité  et  de  la 
propagande, que caractériseraient leur manque de scrupules et leur opposition constante à toute 
recherche sincère de la vérité. » 
\bigskip
P  677-678 :  « Notre  position  sera  bien  différente.  Au  lieu  de  fonder  notre  philosophie  sur  des 
vérités  définitives  et  indiscutables,  nous  partirons  du  fait  que  des  hommes  et  des  groupes 
d'hommes  adhèrent  à  toute  espèce  d'opinions  avec  une  intensité  variable,  que  seule  la  mise  à 
l'épreuve permet de connaître. Les croyances  dont il s'agit ne sont pas toujours évidentes, et leur 
objet consiste rarement en idées claires et distinctes. Les croyances les plus généralement admises 
restent  longtemps  implicites  et  non-formulées,  car,  le  plus  souvent,  ce  n'est  qu'à  l'occasion  d'un 
désaccord quant aux conséquences qui en résultent que se pose le problème de leur formulation ou 
de leur détermination plus précise. » 
\bigskip
P 678 : « Le sens commun oppose régulièrement les faits aux théories, les vérités aux opinions, ce 
qui est objectif à ce qui ne l'est pas, en signalant par là quelles opinions il faut préférer à d'autres, 
que  cette  préférence  soit  fondée  ou  non  sur  des  critères  généralement  acceptés.  J.  St.  Mill  ou  A. 
Lalande,  en  demandant  que  nous  confrontions  nos  croyances  aux  faits  ou  aux  énoncés  vrais, 
n'innovent  guère,  et  s'il  est  facile  de  suivre  leur  avis  quand  les  faits  et  vérités  ne  sont  l'objet 
d'aucune contestation, ceci n'est malheureusement pas toujours le cas. Tout le monde est disposé à 
\bigskip
\bigskip
\bigskip
364 
\bigskip
la  délibération,  à 
\bigskip
reconnaître  aux  faits  et  aux  vérités  un  rôle  normatif  par  rapport  aux  opinions,  mais  celui  qui 
conteste un fait ou doute d'une vérité, hésitera à lui reconnaître ce statut avantageux, et qualifiera 
tout  autrement  l'affirmation  qu'il  refuse  d'accepter;  de  même,  la  plupart  des  hommes  sont 
normalement  disposés  à  agir  conformément  à  ce  qui  leur  paraît  logique  on  raisonnable,  mais 
refusent cet adjectif aux solutions dont ils ne reconnaissent pas le bien-fondé. 
\bigskip
Ceux pour qui les faits et les vérités fournissent les seules normes qui doivent régir les opinions, 
chercheront  à  rattacher  leurs  convictions  à  l'une  ou  l'autre  forme  d'évidence  indubitable  et 
indiscutable. Dans cette perspective, il n'est pas question de fonder ces évidences, à leur tour, car 
sans  elles  la  notion  même  de  fondement  paraîtrait  incompréhensible.  A  partir  d'elles,  la  preuve 
prendra la forme d'un calcul ou d'un recours à l'expérience. » 
\bigskip
P  678-679 :  « La  confiance  accrue,  de  cette  façon,  dans  les  procédés  et  les  résultats  des  sciences 
mathématiques et naturelles allait de pair avec la mise au rancart de tous autres moyens de preuve, 
considérés  comme  dénués  de  valeur  scientifique.  Et  cette  attitude  était  d'ailleurs justifiable  aussi 
longtemps  que  l'on  pouvait  espérer  trouver  à  tous  les  problèmes  humains  réels  une  solution 
scientifiquement  défendable,  grâce  à  l'application  de  plus  en  plus  étendue  du  calcul  des 
probabilités. Par contre, si des problèmes essentiels, qu'il s'agisse de questions morales, sociales ou 
politiques,  philosophiques  ou  religieuses,  échappent,  par  leur  nature  même,  aux  méthodes  des 
sciences mathématiques et naturelles, il ne semble pas raisonnable d'écarter avec mépris toutes les 
techniques  de  raisonnement  propres  à 
la  discussion,  en  un  mot  à 
l'argumentation. Il est trop facile de  disqualifier comme « sophistiques » tous les raisonnements 
non conformes aux exigences de la preuve que Pareto appelle logico-expérimentale. Si l'on devait 
considérer comme raisonnement trompeur toute argumentation de cette espèce, l'insuffisance des 
preuves « logico-expérimentales » laisserait dans tous les domaines essentiels de la vie humaine, le 
champ  entièrement  libre  à  la  suggestion  et  à  la  violence.  En  prétendant  que  ce  qui  n'est  pas 
objectivement  et  indiscutablement  valable  relève  du  subjectif  et  de  l'arbitraire,  on  creuserait  un 
fossé  infranchissable  entre  la  connaissance  théorique,  seule  rationnelle,  et  l'action  dont  les 
motivations seraient entièrement irrationnelles. Dans une pareille perspective, la pratique ne peut 
plus être raisonnable, car l'argumentation critique y devient entièrement incompréhensible et on 
ne  peut  plus  y  prendre  au  sérieux  la  réflexion  philosophique  elle-même.  En  effet,  seuls  les 
domaines  d'où  toute  controverse  a  été  éliminée  peuvent  dès  lors  prétendre  à  une  certaine 
rationalité.  Dès  qu'il  y  a  controverse,  et  que  les  méthodes  «  logico-expérimentales  »  ne  peuvent 
rétablir l'accord des esprits, l'on se trouverait dans le champ de l'irrationnel, qui serait celui de la 
délibération, de la discussion, de l'argumentation. » 
\bigskip
P  680 :  « La  distinction,  si  fréquente  dans  la  philosophie  du  xxe  siècle,  entre  les  jugements  de 
réalité et les jugements de valeur, caractérise une tentative  - que nous croyons, sous cette forme, 
désespérée  de  ceux  qui,  tout  en  reconnaissant  un  statut  particulier  et  éminent  à  l'investigation 
scientifique, voulaient sauver pourtant de l'arbitraire et de l'irrationnel les normes de notre action. 
Mais cette distinction, conséquence d'une épistémologie absolutiste qui tendait à isoler nettement 
deux  faces  de  l'activité  humaine,  n'a  pas  donné  les  résultats  que  l'on  espérait,  et  ceci  pour  deux 
raisons : l'échec dans l'élaboration d'une logique des jugements de valeur et la difficulté de définir 
d'une façon satisfaisante jugements de valeur et jugements de réalité. 
\bigskip
S'il  est  possible,  comme  nous  l'avons  fait,  de  discerner,  dans  la  pratique  argumentative,  des 
énoncés portant sur des faits et d'autres portant sur des valeurs, la  distinction entre ces énoncés 
n'est jamais assurée : elle résulte d'accords précaires, d'intensité variable, souvent implicites. Pour 
pouvoir distinguer nettement deux espèces de jugements, il faudrait pouvoir proposer des critères 
permettant  de  les  identifier,  critères  qui  devraient  échapper  eux-mêmes  à  toute  controverse  et, 
plus  particulièrement,  il  faudrait  un  accord  concernant  les  éléments  linguistiques  sans  lesquels 
aucun jugement n'est formulable. 
\bigskip
\bigskip
\bigskip
\bigskip
365 
\bigskip
Pour  que  des  jugements  de  réalité  fournissent  un  objet  indiscutable  d'un  savoir  commun,  il 
faudrait que les termes qu'ils contiennent soient dépourvus de toute ambiguïté, soit parce qu'il y a 
moyen  de  connaître  leur  vrai  sens,  soit  parce  qu'une  convention  unanimement  admise  supprime 
toute controverse à ce sujet. Ces deux éventualités, qui sont celles du réalisme et du nominalisme 
en matière linguistique, sont toutes deux insoutenables, car elles considèrent le langage comme un 
reflet  du  réel  ou  une  création  arbitraire  d'un  individu,  et  oublient  un  élément  essentiel,  l'aspect 
social du langage, instrument de communication et d'action sur autrui. » 
\bigskip
P 681 : « Tout langage est celui d'une communauté, qu'il s'agisse d'une communauté unie par des 
liens  biologiques  ou  par  la  pratique  d'une  discipline  ou  d'une  technique  commune.  Les  termes 
utilisés, leur sens, leur définition, ne se comprennent que dans le contexte fourni par les habitudes, 
les  façons  de  penser,  les  méthodes,  les  circonstances  extérieures  et  les  traditions  connues  des 
usagers.  Une  déviation  de  l'usage  doit  être  justifiée,  et  le  réalisme  comme  le  nominalisme  ne 
constituent  à  cet  égard  que  deux  tentatives,  diamétralement  opposées  d'ailleurs,  de  justification, 
toutes deux liées à des philosophies du langage pareillement insuffisantes. 
\bigskip
L'adhésion  à  certains  usages  linguistiques  est  normalement  l'expression  de  prises  de  position, 
explicites  ou  implicites,  qui  ne  sont  ni  le  reflet  d'une  réalité  objective  ni  la  manifestation  d'un 
arbitraire individuel. Le langage fait partie des traditions d'une communauté et, comme elles, il ne 
se modifie d'une façon révolutionnaire qu'en cas d'inadaptation radicale à une situation nouvelle; 
autrement  sa  transformation  est  lente  et  insensible.  Mais  un  accord  sur  l'usage  des  termes,  tout 
comme celui concernant la conception du réel et la vision du monde, même s'il est indiscuté, n'est 
pas  indiscutable  :  il  est  lié  à  une  situation  sociale  et  historique,  laquelle  conditionne 
fondamentalement  toute  distinction  que  l'on  voudrait  établir  entre  jugements  de  réalité  et 
jugements de valeur. » 
\bigskip
P  681-682 :  « Vouloir  transcender  ces  conditions  sociales  et  historiques  de  la  connaissance,  en 
transformant certains accords de fait en accords de  droit, n'est possible que grâce à une prise de 
position  philosophique  qui  ne  se  conçoit,  si  elle  est  rationnelle,  que  comme  conséquence  d'une 
argumentation  préalable  (1)  :  la  pratique  et  la  théorie  de  l'argumentation  sont,  à  nos  yeux, 
corrélatives d'un rationalisme critique, qui transcende la dualité jugements de réalité jugements de 
valeur, et rend les uns comme les autres solidaires de la personnalité du savant ou du philosophe, 
responsable  de  ses  décisions  dans  le  domaine  de  la  connaissance  comme  dans  celui  de  l'action 
(1). » 
\bigskip
(1)  Cf.  Ch.  Perelman,  Philosophies  premières  et  philosophie  régressive,  dans  Rhétorique  et 
philosophie, pp. 99-100 et 105, et Réflexions sur la justice, Revue de l'Institut de Sociologie, 1951, 
pp. 280-81. 
(1) Cf. Ch. Perelman, La quête du rationnel, dans Rhétorique et philosophie, pp. 110 à 120, et Le 
rôle de la décision *dans la théorie de la connaissance, dans les Actes du IIe Congrès international 
de l'Union internationale de Philosophie des Sciences, vol. 1, pp. 150-159. 
\bigskip
P 682 : « Seule l'existence d'une argumentation, qui ne soit ni contraignante ni arbitraire, accorde 
un  sens  à  la  liberté  humaine,  condition  d'exercice  d'un  choix  raisonnable.  Si  la  liberté  n'était 
qu'adhésion nécessaire à un ordre naturel préalablement donné, elle exclurait toute possibilité de 
choix ; si l'exercice de la liberté n'était pas fondé sur des raisons, tout choix serait irrationnel et se 
réduirait à une décision arbitraire agissant dans un vide intellectuel (2). C'est grâce à la possibilité 
d'une  argumentation,  qui  fournit  des  raisons,  mais  des  raisons  non-contraignantes,  qu'il  est 
possible d'échapper au dilemme : adhésion à une vérité objectivement et universellement valable, 
ou recours à la suggestion et à la violence pour faire admettre ses opinions et décisions. Ce qu'une 
logique  des  jugements  de  valeur  a  en  vain  essayé  de  fournir,  à  savoir  la  justification  de  la 
possibilité  d'une  communauté  humaine  dans  le  domaine  de  l'action,  quand  cette  justification  ne 
peut être fondée sur une réalité ou une vérité objective, la théorie de l'argumentation contribuera à 
\bigskip
\bigskip
\bigskip
366 
\bigskip
l'élaborer,  et  cela  à  partir  d'une  analyse  de  ces  formes  de  raisonnement  qui,  quoique 
indispensables dans  la pratique, ont  été négligées, à la suite de Descartes, par les logiciens et les 
théoriciens de la connaissance. » 
\bigskip
(2)  Cf.  Ch.  Perelman,  Liberté  et  raisonnement,  dans  Rhétorique  et  philosophie,  pp.  44  à  .18,  Le 
problème du bon choix, ibid., p. 160. 
\bigskip
\bigskip
\bigskip
367 
\bigskip
 
\bigskip
TRAITE DE L’ARGUMENTATION 
\bigskip
INTRODUCTION 
\bigskip
PREMIERE PARTIE : LES CADRES DE L’ARGUMENTATION 
\bigskip
§ I. Démonstration et argumentation 
§ 2. Le contact des esprits 
§ 3. L'orateur et son auditoire 
§ 4. L'auditoire comme construction de l'orateur 
§ 5. Adaptation de l'orateur à l'auditoire 
§ 6. Persuader et convaincre 
§ 7. L'auditoire universel 
§ 8. L'argumentation devant un seul auditeur 
§ 9. La délibération avec soi-même 
§ 10. Les effets de l'argumentation 
§ 11. Le genre épidictique 
§ 12. Éducation et propagande 
§ 13. Argumentation et violence 
§ 14. Argumentation et engagement 
\bigskip
DEUXIEME PARTIE :  LE POINT DE DEPART DE L’ARGUMENTATION 
\bigskip
CHAPITRE PREMIER : L'ACCORD 
§ 15. Les prémisses de l'argumentation 
\bigskip
a) LES TYPES D'OBJET D'ACCORD 
\bigskip
§ 16. Les faits et les vérités 
§ 17. Les présomptions 
§ 18. Les valeurs 
§ 19. Valeurs abstraites et valeurs concrètes 
§ 20. Les hiérarchies 
§ 21. Les lieux 
22. Lieux de la quantité 
§ 23. Lieux de la qualité 
§ 24. Autres lieux 
§ 25. Utilisation et réduction des lieux esprit classique et esprit romantique 
b) LES ACCORDS PROPRES A CERTAINES ARGUMENTATIONS 
\bigskip
§ 26. Accords de certains auditoires particuliers 
§ 27. Accords propres à chaque discussion 
§ 28. L'argumentation « ad hominem » et la pétition de principe 
\bigskip
CHAPITRE II LE CHOIX DES DONNÉES ET LEUR ADAPTATION EN VUE DE L'ARGUMENTATION 
\bigskip
§ 29. La sélection des données et la présence 
§ 30. L'interprétation des données 
§ 31. L'interprétation du discours et ses problèmes 
§ 32. Le choix des qualifications 
§ 33. De l'usage des notions 
§ 34. Clarification et obscurcissement des notions 
§ 35. Usages argumentatifs et plasticité des notions 
\bigskip
CHAPITRE III PRÉSENTATION DES DONNÉES ET FORME DU DISCOURS 
\bigskip
§ 36. Matière et forme du discours 
§ 37. Problèmes techniques de présentation des données 
§ 38. Formes verbales et argumentation 
§ 39. Les modalités dans l'expression de la pensée 
§ 40. Forme du discours et communion avec l'auditoire 
§ 41. Figures de rhétorique et argumentation 
§ 42. Les figures du choix, de la présence et de la communion 
§ 43. Le statut des éléments d'argumentation et leur présentation 
\bigskip
TROISIEME PARTIE : LES TECHNIQUES ARGUMENTATIVES 
\bigskip
\bigskip
\bigskip
1 
\bigskip
1 
\bigskip
7 
7 
8 
10 
11 
14 
16 
19 
22 
26 
29 
31 
34 
36 
39 
\bigskip
42 
\bigskip
42 
42 
43 
43 
45 
48 
49 
52 
54 
56 
58 
61 
63 
65 
65 
69 
74 
\bigskip
77 
77 
80 
82 
84 
87 
89 
93 
\bigskip
95 
95 
96 
100 
104 
111 
113 
117 
122 
\bigskip
126 
\bigskip
368 
\bigskip
§ 44. Généralités 
\bigskip
CHAPITRE, PREMIER : LES ARGUMENTS QUASI LOGIQUES 
\bigskip
§ 45 Caractéristiques de l'argumentation quasi logique 
§ 46. Contradiction et incompatibilité 
§ 47. Procédés permettant d'éviter une incompatibilité 
§ 48. Techniques visant à présenter des thèses comme compatibles ou incompatibles 
§ 49. Le ridicule, et son rôle dans l'argumentation 
§ 50. Identité et définition dans l'argumentation 
§ 51. Analycité, analyse et tautologie 
§ 32. La règle de justice 
§ 53. Arguments de réciprocité 
§ 54. Arguments de transitivité 
§ 55. L'inclusion de la partie dans le tout 
§ 56. La division du tout en ses parties 
§ 57. Les arguments de comparaison 
§ 58. L'argumentation par le sacrifice 
§ 59. Probabilités 
\bigskip
CHAPITRE II LES ARGUMENTS BASÉS SUR LA STRUCTURE DU RÉEL 
\bigskip
§ 60. Généralités. 
\bigskip
a) LES LIAISONS DE SUCCESSION 
§ 61. Le lien causal et l'argumentation 
§ 62. L'argument pragmatique 
§ 63. Le lien causal comme rapport d'un fait à sa conséquence ou d'un moyen à une fin 
§ 64. Les fins et les moyens 
§ 65. L'argument du gaspillage 
§ 66. L'argument de la direction 
§ 67. Le dépassement 
\bigskip
b) LES LIAISONS DE COEXISTENCE 
\bigskip
§ 68. La personne et ses actes 
§ 69. Interaction de l'acte et de la personne 
§ 70. L'argument d'autorité 
§ 71. Les techniques de rupture et de freinage opposées à l'interaction acte-personne 
§ 72. Le discours comme acte de l'orateur 
§ 73. Le groupe et ses membres 
§ 74. Autres liaisons de coexistence, l'acte et l'essence 
§ 75. La liaison symbolique 
§ 76. L'argument de double hiérarchie appliqué aux liaisons de succession et de coexistence 
§ 77. Arguments concernant les différences de degré et d'ordre 
\bigskip
CHAPITRE III LES RAISONS QUI FONDENT LA STRUCTURE DU RÉEL 
\bigskip
a) LE FONDEMENT PAR LE CAS PARTICULIER 
\bigskip
§ 78. L'argumentation par l'exemple 
§ 79. L'illustration 
§ 80. Le modèle et l'antimodèle 
§ 81. L'Être parfait comme modèle 
\bigskip
b) LE RAISONNEMENT PAR ANALOGIE 
\bigskip
§ 82. Qu'est-ce que l'analogie 
§ 83. Relations entre les termes d'une analogie 
§ 84. Effets de l'analogie 
§ 85. Comment on utilise l'analogie 
\bigskip
§ 86. Le statut de l'analogie 
§ 87. La métaphore 
§ 88. Les expressions à sens métaphorique ou métaphores endormies 
\bigskip
CHAPITRE IV LA DISSOCIATION DES NOTIONS 
\bigskip
§ 89. Rupture de liaison et dissociation 
§ 90. Le couple « apparence-réalité » 
§ 91. Les couples philosophiques et leur justification 

§ 92. Le rôle des couples Philosophiques et leurs transformations 
§ 93. L'expression des dissociations 
§ 94. Énoncés incitant à la dissociation 
§ 95. Les définitions dissociatives 
§ 96. La rhétorique comme procédé 
\bigskip
CHAPITRE V L'INTERACTION DES ARGUMENTS 
\bigskip
§ 97. Interaction et force des arguments 
§ 98. L'appréciation de la force des arguments facteur d'argumentation 
§ 99. L'interaction par convergence 
§ 100. L'ampleur de l'argumentation 
§ 101. Les dangers de l'ampleur 
§ 102. Les palliatifs aux dangers de l'ampleur 
§ 103. Ordre et persuasion 
§ 104. Ordre du discours et conditionnement de l'auditoire 
§ 105. Ordre et méthode 
\bigskip
CONCLUSION 

\bigskip
\ifdefined\COMPLETE
\else
    \end{document}
\fi
\bigskip
