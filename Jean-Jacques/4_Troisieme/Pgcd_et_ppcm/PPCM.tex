\input{./preambule.ltx}\usepackage{amsmath,bm}
%\usepackage{graphicx}
%\usepackage{animate}
\usepackage[tikz]{bclogo}



\begin{document}

\newcommand{\titre}[1]{\begin{center}{\Large\textcolor{Black}{#1}}\end{center}}

\newcommand{\paragraphe}[1]{\large\textcolor{blue}{#1}}

\newcommand{\NBVert}[1]{\large\textcolor{DarkGreen}{#1}}

\newcommand{\Attention}[3]{
\begin{bclogo}[%
%epBarre = 0,
%couleurBarre = white,
barre = none,
couleurBord=white,%
logo=\bcattention,% 
margeG = -1,% 
margeD = 1,%
marge = 15%
]{\textcolor{#1}{$\quad$ #2}}
#3
\end {bclogo}
}

\titre{\bf Plus petit commun multiple (\textsc{ppcm})}
 

\textbf{Définitions. — On appelle multiple commun à deux ou plusieurs nombres entiers tout nombre multiple de chacun d'eux.} 

Ainsi 60 est un multiple commun à 6, 10 et 15. \\
Il y a toujours une infinité de multiples communs à plusieurs nombres (en particulier le produit de ces nombres et ses multiples).

\textbf{ Le plus petit des multiples communs à plusieurs nombres s'appelle leur plus petit commun multiple.}\\ 
En abrégé \textsc{ppcm}.\\
 Il est facile de vérifier que le  \textsc{ppcm} de 6, 10 et 15 est égal à 30. 
 
\textbf{ Multiples communs à deux nombres factorisés.} \label{Multiples_communs}

 Il résulte immédiatement de la condition de divisibilité (\label{Quotient_exact}) que : 
 
\textbf{  Pour qu'un nombre entier soit un multiple commun à deux nombres entiers A et B, il faut et il suffit qu'il contienne tous les facteurs premiers contenus dans A et dans B, chacun d'eux étant affecté d'un exposant au moins égal à son plus grand exposant dans A et B. } 

Ainsi les nombres : $360 =2^3 \times 3^2 \times  5$  et $500 = 2^2 \times  5^3$ admettent pour multiples communs : \\

 $2^3 \times 3^2 \times 5^3  = 27\,000, \quad 2^4 \times 3^2 \times 5^3 \times 7 = 126\, 000$.
 
  
Le plus petit des multiples communs s'obtient, donc en prenant seulement les facteurs contenus dans les deux nombres et en affectant chacun d'eux de l'exposant le plus petit possible. D'où : 


\textbf{Règle. — Le \textsc{ppcm} de deux nombres entiers décomposés en facteurs premiers s'obtient en faisant le produit de tous les facteurs contenus dans les deux nombres, chacun d'eux étant affecté de son plus grand exposant.}

 Ainsi le \textsc{ppcm}  des nombres 360 et 500 est égal à : 
 
 \centerline{$2^3 \times 3^2 \times 5^3 = 9\,000$}
 
Nous voyons d'autre part que les multiples communs aux deux nombres contiennent tous les facteurs premiers
de leur \textsc{ppcm} avec de exposants  au moins égaux à ceux de ce \textsc{ppcm}. D'où : 

\textbf{Théorème. — Les multiples communs d deux nombres entiers sont les multiples de leur \textsc{ppcm}.}\label{Multiples_du_ppcm}

  Ainsi les multiples communs à 360 et à 500 sont les multiples de leur  \textsc{ppcm} : 9~000.  Leur liste commence donc par : 
  
  \centerline{9~000, 18~000, 27~000, 36~000, etc. }

\textbf{\textsc{pgcd} et \textsc{ppcm} de plusieurs nombres.} — Les conditions nécessaire et suffisantes des \ref{Diviseurs_communs} et \ref{Multiples_communs} s'étendent à plusieurs nombres. Il en est par suite de même des règles du \textsc{pgcd} et \textsc{ppcm}  ainsi que des théorèmes \ref{Diviseur_du_pgcd}  et \ref{Multiples_du_ppcm}


\textsc{Exemple} — \textit {Calculer le \textsc{pgcd} et \textsc{ppcm}  des nombres 300, 360 et 480. Établir la liste de leurs diviseurs communs et celle de leurs multiples communs.}

 Décomposons ces nombres en facteurs premiers : 
 
 $300 = 2^2  \times  3 \times  5^2 ; \qquad 360 = 2^3  \times 3^2  \times  5; \qquad 480= 2^5 \times 3 \time 5 $\\
  Leur\textsc{pgcd} est égal à : \\
  \centerline {$ 2^2 \times 3 \times 5 = 60 $}.\\
  
 Leur \textsc{ppcm}  est égal à : \\
 
  \centerline {$ 2^5 \times 3^2 \times 5^2 = 7\,200 $}.\\


  La liste de leurs diviseurs communs est la liste des diviseurs de 60 :\\
    \centerline { 1, 2, 3, 4, 5, 6, 10, 12, 15, 20, 30, 60. } 
    
    La liste de leurs multiples communs est la liste des multiples de 7~200 :   \\
   
       \centerline { 7~200, 14~400, 21~600, 28~800, etc.}  


\end{document}


