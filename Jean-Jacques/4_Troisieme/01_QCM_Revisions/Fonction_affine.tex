\ifdefined\COMPLETE
\else
    \input{./preambule-2010-utf8.ltx}
    \begin{document}
\fi

\ifdefined\CORRECTION
    \begin{alterqcm}[lq=10cm,correction]
\else
    \begin{alterqcm}[lq=10cm]
\fi


 \AQquestion[br=2]{$f(x)$ est l'image d'un nombre $x$ par une fonction affine, lorsque\ldots}{%
 {$f(x)=ax^2 + b$,},
 {$f(x)=ax + b$,},
 {$f(x)=a\dfrac{a}{x} + b$.}
 }
 
 \AQquestion[br=3]{$f$ est la fonction affine $\quad x\longmapsto -2x +1$. \\
 Alors, l'image de $\dfrac{1}{2}$ est\ldots}{%
 {$\dfrac{1}{4}$,},
 {5,},
 {0.}
 }
  

 \AQquestion[br=2]{Dans un repère, la droite représentant la fonction affine $\quad x\longmapsto -2x +1$\ldots}
  {%
   {\begin{minipage}[t]{7cm} passe par l'origine du repère et le point $A(1;-2)$, \end{minipage}},
   {\begin{minipage}[t]{7cm}  passe par les deux points $B(0;1)$ et $C(-1;3)$,
    \end{minipage}},
   {\begin{minipage}[t]{7cm}  passe par les deux points $A(1;-2)$ et $B(0;1)$.
    \end{minipage}}
 }


 \AQquestion[br=3]{~}
  {%
   {\begin{minipage}[t]{7.5cm}  Aucune fonction linéaire n'est une fonction affine. \end{minipage}},
   {\begin{minipage}[t]{7cm}  Certaines fonctions linéaires sont des fonctions affines,
    \end{minipage}},
   {\begin{minipage}[t]{7cm}  Toute fonction linéaire est une fonction affine.
    \end{minipage}}
 }
 
 \AQquestion[br=1]{Dans un repère, la droite de coefficient directeur -1 et d'ordonnée à l'origine 5 a pour équation\ldots}{%
 { $ y=-x+5$,},
 {$y=5x-1$,},
 {$ y=x-5$.}
 }
 
 \AQquestion[br=3]{Une droite de coefficient -1 est tracée sur la figure\ldots}{%
{\raisebox{-1.5cm}{
\begin{tikzpicture}[line cap=round,line join=round,>=triangle 45,x=5mm,y=5mm,scale=.7]
\draw [color={gray!40},dash pattern=on 1pt off 1pt, xstep=5mm,ystep=5mm] (-1.1,-1.1) grid (2.1,4.1);
\draw[->,color=black] (-1.1,0) -- (2.1,0);
\foreach \x in {-1,1}
     \draw[shift={(\x,0)},color=black] (0pt,2pt) -- (0pt,-2pt);
\draw[->,color=black] (0,-1.1) -- (0,4.1);
\foreach \y in {1}
     \draw[shift={(0,\y)},color=black] (2pt,0pt) -- (-2pt,0pt) ;
\draw[color=black] (2pt,-7pt) node[left] {\footnotesize $0$};
\draw[color=black] (0,1) node[left] {\footnotesize $1$};
\draw[color=black] (1,0) node[below] {\footnotesize $1$};
\draw [red] (-1,1) -- (2,4) ; 
\end{tikzpicture}}
}, 
{\raisebox{-1.5cm}{\begin{tikzpicture}[line cap=round,line join=round,>=triangle 45,x=5mm,y=5mm,scale=.7]
\draw [color={gray!40},dash pattern=on 1pt off 1pt, xstep=5mm,ystep=5mm] (-1.1,-1.1) grid (3.1,4.1);
\draw[->,color=black] (-1.1,0) -- (3.1,0);
\foreach \x in {-1,1}
     \draw[shift={(\x,0)},color=black] (0pt,2pt) -- (0pt,-2pt);
\draw[->,color=black] (0,-1.1) -- (0,4.1);
\foreach \y in {1}
     \draw[shift={(0,\y)},color=black] (2pt,0pt) -- (-2pt,0pt) ;
\draw[color=black] (2pt,-7pt) node[left] {\footnotesize $0$};
\draw[color=black] (0,1) node[left] {\footnotesize $1$};
\draw[color=black] (1,0) node[below] {\footnotesize $1$};    
\draw [red] (-1,3) -- (1,-1) ;  
%
\end{tikzpicture} }   }, 
 {\raisebox{-1.5cm}{\begin{tikzpicture}[line cap=round,line join=round,>=triangle 45,x=5mm,y=5mm,scale=.7]
\draw [color={gray!40},dash pattern=on 1pt off 1pt, xstep=5mm,ystep=5mm] (-1.1,-1.1) grid (3.1,4.1);
\draw[->,color=black] (-1.1,0) -- (3.1,0);
\foreach \x in {-1,1}
     \draw[shift={(\x,0)},color=black] (0pt,2pt) -- (0pt,-2pt);
\draw[->,color=black] (0,-1.1) -- (0,4.1);
\foreach \y in {1}
     \draw[shift={(0,\y)},color=black] (2pt,0pt) -- (-2pt,0pt) ;
\draw[color=black] (2pt,-7pt) node[left] {\footnotesize $0$};
\draw[color=black] (0,1) node[left] {\footnotesize $1$};
\draw[color=black] (1,0) node[below] {\footnotesize $1$};
\draw [red] (-1,3) -- (3,-1) ; 
\end{tikzpicture}}
} 
}  

 
 \AQquestion[br=2]{$f$ est la fonction affine $x\longmapsto 3x -1$.\\
 Lorsque $x$ augmente de 2, $f(x)$\ldots}{%
 {augmente de 2,},
 {augmente de 6,},
 {diminue de 6.}
 }
 
 
 
 \AQquestion[br=2]{$f$ est la fonction affine $x\longmapsto 2x + 5$.\\
 Lorsque $x$ augmente de 3, $f(x)$\ldots}{%
 {augmente de 2,},
 {augmente de 6,},
 {diminue de 6.}
 }
 
 
 \AQquestion[br=1]{$f$ est une fonction affine.\\
 On sait que $f(3) = 4$ et $f(1)=10$\\
 Le coefficient directeur de la droite représentant $f$ dans le repère est\ldots}{%
 {$-3$,},
 {$3$,},
 {$ -\dfrac{1}{3}$.}
 }  
\end{alterqcm}

\ifdefined\COMPLETE
\else
    \end{document}
\fi