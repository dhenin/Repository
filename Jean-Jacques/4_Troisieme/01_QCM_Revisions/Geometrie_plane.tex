\ifdefined\COMPLETE
\else
    \input{./preambule-2010-utf8.ltx}   
    \begin{document}
\fi

% \newcommand{\CORRECTION}

\ifdefined\CORRECTION
    \begin{alterqcm}[lq=11cm,correction]
\else
    \begin{alterqcm}[lq=11cm]
\fi


\AQquestion[br=3]{$ABC$ est le triangle équilatéral ci-dessous.\\
\hspace*{3cm}
\begin{tikzpicture}
\tkzDefPoint(0,0){A}
\tkzDefPoint(2,0){B}
\tkzDefTriangle[equilateral](A,B)
\tkzGetPoint{C}
\tkzDrawPolygon(A,B,C)
\tkzLabelPoints[below left,font=\scriptsize](A)
\tkzLabelPoints[below right,font=\scriptsize](B)
\tkzLabelPoints[above,font=\scriptsize](C)
\end{tikzpicture}}
{%
{\begin{minipage}[t]{6cm} $A$ ne peut pas avoir pour image C par une rotation de centre $B$ \end{minipage}},
{\begin{minipage}[t]{6cm} $A$ a pour image C par une rotation de centre $B$ et d'angle $60$\textdegree dans le sens contraire à celui des aiguilles d'une montre \end{minipage}},
{\begin{minipage}[t]{6cm} $A$ a pour image C par une rotation de centre $B$ et d'angle $60$\textdegree dans le sens  des aiguilles d'une montre \end{minipage}}
}
 
  \AQquestion[br=3]{Soit un point $P$ et $ABC$  un triangle tel que : $AB = 5 $cm, $AC=3$cm et $\widehat{BAC}=30$\textdegree. $A'$, $B'$, $C'$ sont les images respectives de $A$, $B$, $C$ par une rotation d'angle de $80$\textdegree de centre $P$. Alors\ldots}{%
 {$A'B'=5$cm. $A'C'=3$cm et $\widehat{B'A'C'}=110$\textdegree ,},
 {$A'B'=10$cm. $A'C'=6$cm et $\widehat{B'A'C'}=50$\textdegree ,},
 {$A'B'=5$cm. $A'C'=3$cm et $\widehat{B'A'C'}=30$\textdegree .}
 }

 
  \AQquestion[br=2]{Pour construire l'image d'un segment $[AB]$ par une rotation, il suffit de construire\ldots}{%
 {les images de deux points quelconque de $AB$,},
 {le images de $A$ et $B$,},
 {l'image du milieu de $[AB]$.}
 }


\AQquestion[br=3]{$A,B,M,N,C$ sont les points du cercle ci-dessous de centre O. Alors un autre angle que l'angle $\widehat{BAC}$ de mesure 30\textdegree~ est\ldots\\
\hspace*{3cm}
\begin{tikzpicture}[scale=.5]
\tkzInit[ymin=-2.25,ymax=2.25,xmin=-2.25,xmax=2.25]
\tkzDefPoint(0,0){O}
\tkzDefPoint(2,0){X} 
\tkzDefPointBy[rotation=center O angle   0](X)
\tkzDefPointBy[rotation=center O angle 110](X)   \tkzGetPoint{A}
\tkzDefPointBy[rotation=center O angle 190](X)   \tkzGetPoint{B} 
\tkzDefPointBy[rotation=center O angle -90](X)   \tkzGetPoint{C} 
\tkzDefPointBy[rotation=center O angle  30](X)   \tkzGetPoint{M}          
\tkzDefPointBy[rotation=center O angle -25](X)   \tkzGetPoint{N}
\tkzGetPoint{P}      \tkzDefPointBy[rotation=center O angle 125](X)      
%\tkzGetPoint{P'}     \tkzLabelCircle[above=4pt,font=\scriptsize](O,N)(120){$\mathcal{C}$}
\tkzDrawCircle(O,X)
%\tkzDrawSegment[dashed](O,P)
%\tkzDrawPoints(A,O,N, M,P)\tkzLabelPoints[right,font=\scriptsize](A,N,M,P)
\draw (N) -- (M) -- (B) -- (A) -- (C) -- (M) ;
\draw (C) -- (O) -- (B) ; 
%\tkzDrawPoints(O,A,B,C,M,N)
\tkzDrawPoints[shape=cross out](O)
\tkzLabelPoints[above,font=\scriptsize](A) \tkzLabelPoints[below left,font=\scriptsize](B)
\tkzLabelPoints[below,font=\scriptsize](C) \tkzLabelPoints[above right,font=\scriptsize](M)
\tkzLabelPoints[below right,font=\scriptsize](N)
\tkzLabelPoints[right,font=\tiny](O)
\tkzMarkAngle[color=red, size=0.7](B,A,C)
\tkzLabelAngle[pos=1,color=red,font=\scriptsize](B,A,C){30\textdegree}
\end{tikzpicture}
}
{%
{$\widehat{CMN}$,},
{$\widehat{BOC}$,},
{$\widehat{BMC}$.}
}
 
  \AQquestion[br=3]{Sur la figure ci-dessus, l'angle  $\widehat{BOC}$ \ldots}{%
 {15\textdegree,},
 {30\textdegree,},
 {60\textdegree.}
 }

 
\AQquestion[br=1]{Un décagone régulier est inscrit dans un cercle de centre $O$.\\
On passe d'un sommet au sommet suivant par une rotation de centre $O$ est d'angle\ldots}{%
{36\textdegree,},
{72\textdegree,},
{108\textdegree.}
}
 
\AQquestion[br=3]{$A,B,C$ sont trois sommets consécutifs d'un pentagone régulier. La mesure de l'angle $\widehat{ABC}$ est\ldots}{%
{54\textdegree,},
{72\textdegree,},
{108\textdegree.}
}
    
 
\AQquestion[br=1]{Un hexagone régulier est inscrit dans un cercle de rayon 5 cm.\\
Son côté mesure en cm\ldots}{%
{5,},
{$\cos$ 30\textdegree,},
{$5 \sin 60$\textdegree.}
}

\end{alterqcm}


\ifdefined\COMPLETE
\else
    \end{document}
\fi
