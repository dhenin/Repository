\ifdefined\COMPLETE
\else
    \input{./preambule-2010-utf8.ltx}   
    \begin{document}
\fi

\ifdefined\CORRECTION
    \begin{alterqcm}[lq=11cm,correction]
\else
    \begin{alterqcm}[lq=11cm]
\fi


 \AQquestion[br=2]{Le vecteur représenté ci-dessous a pour coordonnées\ldots\\
 \hspace*{3cm}\begin{tikzpicture}[scale=.3]
\tkzInit[xmin=-2, xmax=4,ymin=-4,ymax=5]
\tkzGrid[color={gray!40}]
\tkzRep[ynorm=1,xlabel={\tiny $\vec{\imath}$},ylabel={\tiny $\vec{\jmath}$}]
\tkzDrawX[noticks,right space=-.1] \tkzDrawY[noticks, up space=-.15] 
\tkzDefPoints{0/0/O, 3/4/A, -1/-3/B} \tkzDrawPoints[shape=cross](O,A,B)
\tkzLabelPoints[below left,font=\tiny](O) 
\tkzLabelPoints[above left,font=\tiny](A) \tkzLabelPoints[above left,font=\tiny](B)
\draw [thick, red,->] (A) -- (B) ; 
\end{tikzpicture}}{%
{$(4 ; 7)$},
{$(-4 ; -7)$},
{$(4 ; -7)$}
 }
 
  \AQquestion[br=3]{Dans un repère, les vecteurs $\overrightarrow{u}$ et $\overrightarrow{v}$ ont pour coordonnées respectives $\left( \begin{matrix}
a\\b
\end{matrix}\right)$ et $\left( \begin{matrix}
c\\d
\end{matrix}\right)$.\\
 Si $\overrightarrow{u} = \overrightarrow{v}$, alors\ldots}{%
 {$a=b$ et $c=d$},
 {$a=d$ et $b=c$},
 {$a=c$ et $b=d$}
 }


 \AQquestion[br=2]{Dans un repère, on considère les points $A(2;-3)$ et $B(-1;4)$. \\
 Les coordonnées $(x ; y)$ du vecteur  $\overrightarrow{AB}$ sont alors\ldots}{%
 {$x =2-(-1) $ et $y  = -3-4 $ },
 {$x = -1- 2  $ et $y  = 4 -(-3) $ },
 {$x =4-2  $ et $y  = -1 -(-3) $ }
 }


 \AQquestion[br=1]{Dans un repère, on considère les points $A(2;-3)$ et $B(-1;4)$. \\
 Les coordonnées $(x_i ; y_i)$ du milieu $I$ de $[AB]$ sont alors\ldots}{%
 {$x_i =\dfrac{2+(-1)}{2}$ et $y_i =\dfrac{-3+4}{2}$ },
 {$x_i =\dfrac{2-(-1)}{2}$ et $y_i =\dfrac{-3-4}{2}$ },
 {$x_i = 2+(-1)$ et $y_i =-3+4 $ }
 }


 \AQquestion[br=3]{Dans un repère, on considère les points $A(2;-3)$ et $B(-1;4)$. \\
 La distance $AB$ est  alors\ldots}{%
 {$ (2 - (-1))^2 + (-3-4)^2$},
 {$ (2 - (-1)) + (-3-4)$},
 {$ \sqrt{(2 - (-1))^2 + (-3-4)^2}$}
 }


\end{alterqcm}


\ifdefined\COMPLETE
\else
    \end{document}
\fi