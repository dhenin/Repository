\ifdefined\COMPLETE
\else
    \input{./preambule-2010-utf8.ltx}
    \begin{document}
\fi

\ifdefined\CORRECTION
    \begin{alterqcm}[lq=11cm,correction]
\else
    \begin{alterqcm}[lq=11cm]
\fi

 \AQquestion[br=2]{Une expression développée est\ldots}{%
 {$(x-1)(x+2)$},
 {$x^2+x-2$},
 {$x(x-1)+2x-2$}
 }



\AQquestion[br=3]{Le développement de $(a+b)^2$ est\ldots}{%
 {$a^2+b^2$},
 {$a^2+ab+b^2$},
 {$a^2+2ab+b^2$}
 }

\AQquestion[br=3]{Le développement de $(a-b)^2$ est\ldots}{%
 {$a^2-b^2$},
 {$a^2+2ab-b^2$},
 {$a^2-2ab+b^2$}
}

\AQquestion[br=1]{Le développement de $(a-b)(a+b)$ est\ldots}{%
 {$a^2-b^2$},
 {$a^2+b^2$},
 {$a^2-2ab+b^2$}
}

\AQquestion[br=2]{Une expression factorisée est\ldots}{%
{$(x-2)^2+(x-2)(x-1)$},
{$(x-2)(2x-3)$},
{$2x^2-7x+6$}
}


\AQquestion[br=1]{$A = (x-1)(x+2)-5(x-1)$}{
{\begin{minipage}[t]{6cm}
   L'expression $A$ peut être factorisée \\
   avec un facteur commun évident,
\end{minipage}},
{\begin{minipage}[t]{6cm}
   L'expression $A$ peut être factorisée \\
   avec une identité remarquable, 
\end{minipage}},
{\begin{minipage}[t]{6cm}
   L'expression $A$ \\ 
   ne peut pas  être factorisée.
\end{minipage}}
}


\AQquestion[br=2]{$A = 9x^2 -2x + \dfrac{1}{9}$}{%
{\begin{minipage}[t]{6cm}
   L'expression $A$ peut être factorisée \\
   avec un facteur commun évident,
\end{minipage}},
{\begin{minipage}[t]{6cm}
   L'expression $A$ peut être factorisée \\
   avec une identité remarquable, 
\end{minipage}},
{\begin{minipage}[t]{6cm}
   L'expression $A$ \\ 
   ne peut pas  être factorisée.
\end{minipage}}
}

\end{alterqcm}

\ifdefined\COMPLETE
\else
    \end{document}
\fi