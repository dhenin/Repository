\ifdefined\COMPLETE
\else
    \input{./preambule-2010-utf8.ltx}
    

    \begin{document}
\fi

% \newcommand{\CORRECTION}

\ifdefined\CORRECTION
    \begin{alterqcm}[lq=11cm,correction]
\else
    \begin{alterqcm}[lq=11cm]
\fi


 \AQquestion[br=3]{Les vecteurs $\overrightarrow{AB}$ et $\overrightarrow{CD}$ sont égaux sur la figure\ldots}{%
{
\begin{tikzpicture}[scale=.5,rotate=-95]
\begin{tiny}
\tkzDefPoint(0,0){A}
\tkzDefShiftPoint[A](-60:-4){B}
\tkzDefShiftPoint[B](90:2){C}
\tkzDefShiftPoint[A](90:2){D}
\draw [->] (A) -- (B) ; \draw [->] (C) -- (D) ; 
\tkzLabelPoints[left,font=\fontsize{8}{10}\selectfont](A) \tkzLabelPoints[right,font=\fontsize{8}{10}\selectfont](B) 
\tkzLabelPoints[right,font=\fontsize{8}{10}\selectfont](C) \tkzLabelPoints[left,font=\fontsize{8}{10}\selectfont](D) 
\end{tiny}
\end{tikzpicture}
},
 {\begin{tikzpicture}[scale=.5,rotate=-95]
\begin{tiny}
\tkzDefPoint(0,0){A}
\tkzDefShiftPoint[A](60:-4){B}
\tkzDefShiftPoint[B](95:8){D}
\tkzDefShiftPoint[A](85:1){C}
\draw [->] (A) -- (B) ; \draw [->] (C) -- (D) ; 
\tkzLabelPoints[below,font=\fontsize{8}{10}\selectfont](A) \tkzLabelPoints[above left,font=\fontsize{8}{10}\selectfont](B) 
\tkzLabelPoints[below,font=\fontsize{8}{10}\selectfont](C) \tkzLabelPoints[above right,font=\fontsize{8}{10}\selectfont](D) 
\end{tiny}
\end{tikzpicture}},
 {\begin{tikzpicture}[scale=.5,rotate=40]
\begin{tiny}
\tkzDefPoint(0,0){A}
\tkzDefShiftPoint[A](-60:-4){B}
\tkzDefShiftPoint[B](90:-2){C}
\tkzDefShiftPoint[A](90:-2){D}
\draw [->] (B) -- (A) ; \draw [->] (C) -- (D) ; 
\tkzLabelPoints[right,font=\fontsize{8}{10}\selectfont](A) \tkzLabelPoints[left,font=\fontsize{8}{10}\selectfont](B) 
\tkzLabelPoints[left,font=\fontsize{8}{10}\selectfont](C) \tkzLabelPoints[right,font=\fontsize{8}{10}\selectfont](D) 
\end{tiny}
\end{tikzpicture}}
 }
 
 \AQquestion[br=2]{Si $\overrightarrow{AM} = \overrightarrow{EK}$, alors\ldots}{%
 {$AMEK$ est parallèlogramme,},
 {$AMKE$ est parallèlogramme,},
 {$KEMA$ est parallèlogramme.}
 }

 \AQquestion[br=3]{Si $MNPQ$ est un parallélogramme, alors\ldots}{%
 {$\overrightarrow{MP} = \overrightarrow{NQ}$,},
 {$\overrightarrow{MN} = \overrightarrow{PQ}$,},
 {$\overrightarrow{MQ} = \overrightarrow{NP}$.}
 }
 
 
 \AQquestion[br=2]{Si $N$ est le symétrique de $M$ par rapport à $A$, alors\ldots}{%
 {$\overrightarrow{AM} = \overrightarrow{AN}$,},
 {$\overrightarrow{MA} = \overrightarrow{AN}$,},
 {$\overrightarrow{MA} = \overrightarrow{NA}$.}
 }
   

 \AQquestion[br=3]{Pour démontrer que $I$ st le milieu de $[AB]$, il suffit de démontrer  que\ldots}{%
 {$AI = IB$},
 {$AB = \dfrac{1}{2} AB$},
 {$\overrightarrow{AI} = \dfrac{1}{2}\overrightarrow{AB}$}
 }
   
   
 \AQquestion[br=3]{D'après la relation de {\sc Chasles}\ldots}{%
 {$\overrightarrow{MN} - \overrightarrow{NA} = \overrightarrow{MA}$,},
 {$\overrightarrow{AM} + \overrightarrow{AN} = \overrightarrow{MN}$,},
 {$\overrightarrow{MN} - \overrightarrow{AN} = \overrightarrow{MA}$.}
 }
      
   
 \AQquestion[br=3]{$ABCD$ est un parallélogramme. Donc\ldots}{%
 {$\overrightarrow{BA} - \overrightarrow{BD} = \overrightarrow{BC}$,},
 {$\overrightarrow{AB} + \overrightarrow{AD} = \overrightarrow{BD}$,},
 {$\overrightarrow{AB} - \overrightarrow{AD} = \overrightarrow{AC}$.}
 }
    
 \AQquestion[br=2]{Sur la figure ci-dessus, $ABCD$ est un carré de centre $I$.\\
 $\overrightarrow{CD} - \overrightarrow{IA}$ n'est pas égal à\ldots\\
\hspace*{3cm}\begin{tikzpicture}[scale=.5]
\tkzDefPoint(0,0){A}
\tkzDefPoint(4,0){B}
\tkzDefMidPoint(A,C)\tkzGetPoint{I}
\tkzDefSquare(A,B)\tkzGetPoints{C}{D}
\tkzDrawSquare(A,B) \tkzDrawSegments(A,C B,D)
\tkzLabelPoints[left,font=\fontsize{8}{10}\selectfont](A) \tkzLabelPoints[right,font=\fontsize{8}{10}\selectfont](B)
\tkzLabelPoints[right,font=\fontsize{8}{10}\selectfont](C) \tkzLabelPoints[left,font=\fontsize{8}{10}\selectfont](D)
 \tkzLabelPoints[above,font=\fontsize{8}{10}\selectfont](I)
\end{tikzpicture}
}{%
 {$\overrightarrow{BI}$,},
 {$\overrightarrow{IC}$,},
 {$\overrightarrow{ID}$.}
 }

\iffalse % semble trop compliqué 
 \AQquestion[br=3]{La composée de la symétrie de centre $E$ suivie de la symétrie de centre $F$ est\ldots}{%
 {\begin{minipage}[t]{6cm}
   La symétrie ayant pour centre le milieu du segment $[EF]$,
\end{minipage}},
 {La symétrie axiale d'axe $(EF)$,},
 {La translation de vecteur $2\overrightarrow{EF}$.}
 }
\fi    
\end{alterqcm}


\ifdefined\COMPLETE
\else
    \end{document}
\fi