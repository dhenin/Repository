\ifdefined\COMPLETE
\else
    \input{./preambule-2010-utf8.ltx}
    \begin{document}
\fi

\vspace*{-1mm}
\begin{alterqcm}[lq=10cm]

 \AQquestion[br=2]{$M$ est un point de la sphère de centre $A$ et de rayon $5cm$. Donc\ldots}{%
 {$AM < 5 $},
 {$AM = 5 $},
 {$AM > 5$}
 }
 
 \AQquestion[br=2]{$P$ est un point de la boule de centre $A$ et de rayon $5cm$. Donc\ldots}{%
 {$AP \leqslant 5 $},
 {$AP >5 $},
 {$AP \leqslant 10$}
 }
 
   
 \AQquestion[br=2]{L'aire d'une sphère de rayon $R$ est\ldots}{%
 {$\dfrac{4}{3}\pi R^3$},
 {$\dfrac{4}{3}\pi R^2$},
 {$4 \pi R^2$}
 }

 \AQquestion[br=2]{Le volume de la boule de rayon $R$ est\ldots}{%
 {$\dfrac{4}{3}\pi R^3$},
 {$\dfrac{1}{3}\pi R^3$},
 {$\dfrac{4}{3}\pi R^2$}
 }
   
 \AQquestion[br=2]{Un plan coupe une sphère selon\ldots}{%
 {un carré},
 {un cercle ou un ovale},
 {un cercle ou un point}
 }
 
     
 \AQquestion[br=2]{Un plan coupe une sphère de centre $O$ selon un cercle $\mathcal{C}$ de centre $I$ et de $M$ est un point de  $\mathcal{C}$. Alors, le triangle $OIM$ est\ldots}{%
 {isocèle en $O$},
 {rectangle en $O$},
 {rectangle en $I$}
 }
 
  
     
 \AQquestion[br=2]{La section d'un parallélépipède  rectangle par un plan parallèle à une face est\ldots}{%
 {un triangle},
 {un rectangle},
 {un hexagone}
 }
    
     
 \AQquestion[br=2]{La section d'un parallélépipède  rectangle par un plan parallèle à une arête est\ldots}{%
 {un triangle},
 {un rectangle},
 {un hexagone}
 }
    
 \AQquestion[br=2]{La section d'un cylindre de révolution  par un plan parallèle à axe est\ldots}{%

{un rectangle},
 {un ovale},
 {un  cercle}
 }
 
      
 \AQquestion[br=2]{La section d'une pyramide à base triangulaire   par un plan parallèle à sa base  est\ldots}{%
 {un rectangle},
  {un triangle},
 {un hexagone}
 }
     
 \AQquestion[br=2]{Un cône de révolution est coupé par un plan parallèle à la base. Ce plan partage le cône en\ldots}{%
 {deux cônes},
  {un cône réduit et un tronc de cône},
 {un cône et une pyramide}
 }
 
     
 \AQquestion[br=2]{Une pyramide à base carrée de hauteur 8 cm est coupée par un plan situé à 3 cm du sommet de la pyramide et parallèle à sa base.\\
 La section a un côté de 6cm.\\
 Le côté de la base de la pyramide est de\ldots}{%
 {2,25 cm},
  {11 cm},
 {16 cm}
 }
 

\end{alterqcm}


\ifdefined\COMPLETE
\else
    \end{document}
\fi