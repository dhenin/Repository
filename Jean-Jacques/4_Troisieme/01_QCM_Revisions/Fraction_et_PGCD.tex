\ifdefined\COMPLETE
\else
    \input{./preambule-2010-utf8.ltx}
    \begin{document}
\fi

% \newcommand{\CORRECTION}

\ifdefined\CORRECTION
    \begin{alterqcm}[lq=8cm,correction]
\else
    \begin{alterqcm}[lq=8cm]
\fi

 \AQquestion[br=1]{$\left(\sqrt{2}\right)^2$ est un nombre \ldots}{%
 {entier,},
 {décimal non entier,},
 {irrationnel.}
 }
 

 \AQquestion[br=2]{$\dfrac{3}{4}$ est un nombre \ldots}{%
 {entier,},
 {décimal non entier,},
 {irrationnel.}
 }

 \AQquestion[br=2]{$\dfrac{7}{3}$ est un nombre \ldots}{%
 {entier,},
 {décimal non entier,},
 {irrationnel.}
 } 
 

 \AQquestion[br=3]{$\left(1+\sqrt{2}\right)^2$, est un nombre \ldots}{%
 {entier,},
 {décimal non entier,},
 {irrationnel.}
 }


 \AQquestion[br=2]{36 admet\ldots}{%
 {six diviseurs,},
 {neuf diviseurs,},
 {douze diviseurs.}
 }
  

 \AQquestion[br=2]{Le {\sc pgcd} de 24 et 36 est\ldots}{%
 {4,},
 {12,},
 {24.}
 }  

 \AQquestion[br=3]{L'algorithme d'{\sc Euclide} permet de calculer\ldots}{%
   {\begin{minipage}[t]{8cm}le plus petit diviseur commun à deux nombres,\end{minipage}},
   {\begin{minipage}[t]{8cm} le reste de la division euclidienne de deux nombres,
    \end{minipage}},
   {\begin{minipage}[t]{8cm} le {\sc pgcd} de deux nombres.
    \end{minipage}}
 }
 
\AQquestion[br=2]{Deux nombres premiers entre eux\ldots}{%
   {\begin{minipage}[t]{8cm}n'ont aucun diviseur commun,\end{minipage}},
   {\begin{minipage}[t]{8cm} admettent 1 pour seul diviseur commun,
    \end{minipage}},
   {\begin{minipage}[t]{8.5cm} sont deux nombres dont l'un est multiple de l'autre.
    \end{minipage}}
 } 
 

\AQquestion[br=3]{Une fraction irréductible est\ldots}{%
 {$\dfrac{45}{21}$,},
 {$\dfrac{4220}{542}$,},
 {$\dfrac{17}{14}$.}
 }   

 
\AQquestion[br=2]{Pour rendre irréductible une fraction\ldots}{%
   {\begin{minipage}[t]{8.5cm} on soustrait le dénominateur au numérateur,\end{minipage}},
   {\begin{minipage}[t]{8.5cm} on divise numérateur et dénominateur par leur {\sc pgcd},
    \end{minipage}},
   {\begin{minipage}[t]{8.5cm} on divise numérateur et dénominateur \\
   par un nombre quelconque, autre que 0.
    \end{minipage}}
 } 
\end{alterqcm}

\ifdefined\COMPLETE
\else
    \end{document}
\fi