\ifdefined\COMPLETE
\else
    \input{./preambule-2010-utf8.ltx}
    \begin{document}
\fi

\ifdefined\CORRECTION
    \begin{alterqcm}[lq=10cm,correction]
\else
    \begin{alterqcm}[lq=10cm]
\fi

 \AQquestion[br=2]{On ne peut calculer la racine carrée d'un nombre $a$ que si\ldots}
  {%
   {$a$ est un nombre entier},
   {$a \geqslant 0$},
   {$a \leqslant 0$}
 }

 \AQquestion[br=1]{Si $a$ désigne un nombre positif, alors $\sqrt{a}$ désigne\ldots}{%
   {le nombre positif dont le carré est $a$},
   {la moitié de $a$},
   {le carré de $a$}
 }
 

 \AQquestion[br=2]{Si $a$ désigne un nombre positif, alors $\sqrt{a^2}$ est égal à \ldots}{%
   {$\dfrac{1}{2} a$},
   {$a$},
   {$\dfrac{1}{2} a^2$}
 } 

 \AQquestion[br=3]{L'équation $x^2 = 5$\ldots}{%
   {n'a pas de solution,},
   {a une seule solution},
   {a deux solutions opposées}
 } 


 \AQquestion[br=1]{L'équation $2x^2 +3 = 0$\ldots}{%
   {n'a pas de solution,},
   {a une seule solution},
   {a deux solutions opposées}
 } 


 \AQquestion[br=2]{$\sqrt{2} \times \sqrt{3}$ est égal à \ldots}{%
   {$\sqrt{5}$},
   {$\sqrt{6}$},
   {6}
 } 


 \AQquestion[br=1]{$\sqrt{\dfrac{5}{3}}$ est égal à \ldots}{%
   {$\dfrac{\sqrt{5}}{\sqrt{3}}$},
   {$\dfrac{\sqrt{5}}{3}$},
   {$\sqrt{2}$}
 } 

 \AQquestion[br=1]{$\dfrac{2}{\sqrt{2}}$ est égal à \ldots}{%
   {$\sqrt{2}$},
   {$2\sqrt{2}$},
   {$\dfrac{\sqrt{2}}{2}$}
 }  
 

 \AQquestion[br=3]{$\sqrt{72}$ peut s'écrire \ldots}{%
   {$36\sqrt{2}$},
   {$2\sqrt{6}$},
   {$6\sqrt{2}$}
 }  
 
 \AQquestion[br=2]{$3\sqrt{20} - \sqrt{45}  \sqrt{5}$ peut s'écrire sous la forme\ldots}{%
   {$a\sqrt{3}$ avec $a$ nombre entier,},
    {$a\sqrt{5}$ avec $a$ nombre entier,},
   {$a\sqrt{2} + b\sqrt{5}$ avec $a$ et $b$ nombres entiers}
 }  
 
  
\end{alterqcm}

\ifdefined\COMPLETE
\else
    \end{document}
\fi