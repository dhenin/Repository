\ifdefined\COMPLETE
\else
    \input{./preambule-sacha-utf8.ltx}
    \usepackage{alterqcm}
    \begin{document}
\fi

\begin{alterqcm}[lq=10cm]

 \AQquestion[br=2]{Dans une classe, 12 élèves étudient l'anglais et 13 n'étudient pas l'anglais. La fréquence des élèves de cette classe qui étudient l'anglais est,\ldots}{%
 {$\dfrac{12}{13}$},
 {$\dfrac{25}{12}$},
 {$\dfrac{12}{25}$}
 }
 

 \AQquestion[br=2]{Voici les résultats à un devoir de mathématiques dans une classe de $3^{e}$.\\ \medskip 
 \centerline{\begin{tabular}{|c|c|c|c|c|}
\hline 
 Note     &  8  & 10  & 12 & 15 \\
\hline  
Effectif  &  5  &  6  & 12 &  2 \\ 
\hline  
 \end{tabular}} \medskip 
Pour calculer la moyenne de ces notes, on effectue\ldots}{%
 {$\dfrac{5 \times 8 + 6 \times 10 + 12 \times 12 + 2 \times 15}{5 + 6 + 12 + 2}$},
 {$\dfrac{5 \times 8 + 6 \times 10 + 12 \times 12 + 2 \times 15}{8 + 10 + 12 + 15 }$},
 {$\dfrac{5 + 6 + 12 + 2}{5 \times 8 + 6 \times 10 + 12 \times 12 + 2 \times 15}$}
 }
 
 \AQquestion[br=2]{La médiane de la série \\
 $2 - 4 - 5 - 5 - 7 - 11$ est \ldots}{%
 {5},
 {7},
 {8}
 }

 
 \AQquestion[br=2]{La médiane de la série \\
 $2 - 4 - 5 - 7 - 8 - 10 - 18$ est \ldots}{%
 {4},
 {5},
 {6}
 }
   
 \AQquestion[br=2]{La médiane de la série \\
 $2 - 4 - 5 - 6 - 7 - 8 $ est \ldots}{%
 {4},
 {6},
 {5,5}
 }
  
 \AQquestion[br=2]{Voici la répartition des âges des élèves d'une classe de $3^{e}$.\\ \medskip 
 \centerline{\begin{tabular}{|c|c|c|c|}
\hline 
 Âge $a$    &  $a < 14$   & $14 \leqslant a < 16 $ & $ a \geqslant 16 $ \\
\hline  
Effectif  &  2  &  15   & 8  \\ 
\hline  
 \end{tabular}} }{%
 {\begin{minipage}[t]{7cm} On peut connaître l'âge médian et l'âge moyen \end{minipage}},
   {\begin{minipage}[t]{7cm} On peut connaître l'âge médian mais pas  l'âge moyen
    \end{minipage}},
   {\begin{minipage}[t]{7cm} On ne peut connaître  ni l'âge médian n l'âge moyen
    \end{minipage}}
 }
 
 \AQquestion[br=2]{L'étendue d'une série statistique est\ldots}
  {%
   {\begin{minipage}[t]{6cm} la plus grande valeur du caractère \end{minipage}},
   {\begin{minipage}[t]{6cm}la plus petite valeur du caractère 
    \end{minipage}},
   {\begin{minipage}[t]{6cm} la différence entre la plus grande et la plus petite des valeurs du caractère.
    \end{minipage}}
 }

\end{alterqcm}



\ifdefined\COMPLETE
\else
    \end{document}
\fi