\ifdefined\COMPLETE
\else
    \input{./preambule-2010-utf8.ltx}
    \usepackage{frcursive}
    \usepackage{calligra}
    \begin{document}
\fi

%    \newcommand{\CORRECTION}

\ifdefined\CORRECTION
    \begin{alterqcm}[lq=11cm,correction]
\else
    \begin{alterqcm}[lq=11cm]
\fi


\AQquestion[br=3]{\setbox2=\vtop{\hsize=45mm
                            Dans la situation ci-contre, \\
                            pour calculer $AC$, on utilise\ldots
}                           
%
\setbox3=\vtop{\hsize=50mm
\begin{tikzpicture}[scale=.5,rotate=-15]
\tkzDefPoint(0,0){A}
\tkzDefShiftPoint[A](0:2.2){M}
\tkzDefShiftPoint[A](30:3){N}
\tkzDefShiftPoint[A](0:6){B}
\tkzDefShiftPoint[A](30:8){C}
\tkzDrawSegments(A,B B,C C,A M,N)
\tkzDrawPoints(A,B,C,M,N)
\tkzLabelPoints[left](A) \tkzLabelPoints[below](B) \tkzLabelPoints[right](C)
\tkzLabelPoints[below](M) \tkzLabelPoints[above](N)
\tkzLabelSegment[below,pos=.5,red](A,M){2}
\tkzLabelSegment[below,pos=.5,red](M,B){4}
\tkzLabelSegment[above,pos=.5,red](A,N){5}
\end{tikzpicture}
}   
{\box2} {\raisebox{-18mm}{\box3}}
}{%
 {$\cos(\widehat{B})$},
 {$\sin(\widehat{B})$}, 
 {$\tan(\widehat{B})$}
} 

\AQquestion[br=3]{\setbox2=\vtop{\hsize=50mm
                            Dans la situation ci-contre, \\
                            le théorème de {\sc Thalès} permet d'écrire\ldots}                           %                           
\setbox3=\vtop{\hsize=40mm
\begin{tikzpicture}[scale=.3,rotate=-40]
\tkzDefPoint(0,0){O}
\tkzDefShiftPoint[O](0:-2.2){A}
\tkzDefShiftPoint[O](30:-3){B}
\tkzDefShiftPoint[O](0:6){D}
\tkzDefShiftPoint[O](30:8){C}
\tkzDrawSegments(A,B B,C A,D D,C)
\tkzDrawPoints(A,B,C,D,O)
\tkzLabelPoints[above right](O) \tkzLabelPoints[left](B) \tkzLabelPoints[right](C)
\tkzLabelPoints[below](D) \tkzLabelPoints[above](A)
\end{tikzpicture}
%
}   
 {\box2} $\quad$ {\raisebox{-20mm}{\box3} }
}{%
 {$\dfrac{OA}{OB} = \dfrac{OD}{OC} = \dfrac{AB}{CD}$},
 {$\dfrac{OA}{OC} = \dfrac{OB}{OD} = \dfrac{AB}{CD}$}, 
 {$\dfrac{OA}{OD} = \dfrac{OB}{OC} = \dfrac{AB}{CD}$}
} 

\AQquestion[br=2]{\setbox2=\vtop{\hsize=60mm
                           Avec les données de la figure à main levée ci-contre, 
                           on peut affirmer que la longueur de $FM$ est égale à\ldots
}                           
                   \setbox3=\vtop{\hsize=35mm
\font\pc=cmitt10 \rm
\begin{tikzpicture}[scale=.4,rotate=7,decoration={random   steps, amplitude=1pt,segment  length=3pt}]
\begin{tiny}
\tkzDefPoint(0,0){A}
\tkzDefShiftPoint[A](0:-2.2){N}
\tkzDefShiftPoint[A](40:-3){E}
\tkzDefShiftPoint[A](40:5){F}
\tkzDefShiftPoint[A](0:5){M}
\draw[decorate,decoration=random steps,segment length=4.5mm] (E) -- (F) -- (M) -- (N) -- cycle ; 
\tkzDrawPoints(A,E,F, M,N)
\draw (A) [above left] node  {\pc A} ;
\draw (E) [below right ] node  {\pc E} ;
\draw (F) [above right] node  {\pc F} ;
\draw (M) [right] node  {\pc M} ;
\draw (N) [above] node  {\pc N} ;
\tkzLabelSegment[below right,pos=.5,red](A,E){\pc 3}
\tkzLabelSegment[left,pos=.5,red](E,N){\pc  2}
\tkzLabelSegment[above left ,pos=.5,red](A,F){\pc  5}
\draw (2,-1.5) node []{ { {\pc (FM)//(EN)}}};
\end{tiny}
\end{tikzpicture}
}   
 {\box2} $\quad${\raisebox{-25mm}{\box3}}
}{%
 {3,3},
 {$\dfrac{10}{3}$}, 
 {4}
} 


 \AQquestion[br=1]{$ABC$ est un triangle tel que $AB=3$cm et $AC=4$cm.\\
 $M$ est un point de $(AB)$ et $N$, un point de $(AC)$, tel que : 
 $AM=1,5$cm et $AN=2$cm\\
 Pour pouvoir en déduire que : $(MN)\sslash (BC)$, il faut savoir aussi que\ldots}
  {%
   {$A,M,B \; \mathrm{ et } \;  A, N, C$  sont dans cet ordre,},
   {$A,M,B \; \mathrm{ et } \; A, C, N$   sont dans cet ordre,},
   {$A,N,B \; \mathrm{ et } \; A, M, C$   sont dans cet ordre.}
 }
 
 \AQquestion[br=1]{Avec les données ci-contre\ldots $\qquad \qquad$
\raisebox{-25mm}{ 
\begin{tikzpicture}[scale=.5]
\tkzDefPoint(0,0){A}
\tkzDefShiftPoint[A](0:4){B}
\tkzDefShiftPoint[A](90:-3){C}
\tkzDefShiftPoint[A](0:6){M}
\tkzDefShiftPoint[A](90:-5){N}
\tkzDrawSegments(A,N A,M  C,B M,N)
%\tkzDrawPoints(A,B,C,M,N)
\tkzLabelPoints[above left](A) \tkzLabelPoints[above](B) \tkzLabelPoints[left ](C)
\tkzLabelPoints[above](M) \tkzLabelPoints[left](N)
\tkzLabelSegment[left,pos=.5,red](A,C){3}
\tkzLabelSegment[above,pos=.5,red](A,B){4}
\tkzLabelSegment[above,pos=.5,red](B,M){1}
\end{tikzpicture}}
 }
  {%
   {on calcule $BC$ avec le théorème de {\sc Pythagore}, },
   {on calcule $BC$ avec le théorème de {\sc Thalès},},
   {on ne peut pas calculer $BC$}
 }

\end{alterqcm}


\ifdefined\COMPLETE
\else
    \end{document}
\fi