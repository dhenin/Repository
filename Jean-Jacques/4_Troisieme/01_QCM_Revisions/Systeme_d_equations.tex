\ifdefined\COMPLETE
\else
    \input{./preambule-sacha-utf8.ltx}
    \usepackage{alterqcm}
    \begin{document}
\fi

\ifdefined\CORRECTION
    \begin{alterqcm}[lq=8.5cm,correction]
\else
    \begin{alterqcm}[lq=8.5cm]
\fi



\AQquestion[br=3]{Le système     $ \; \left\lbrace \begin{array}{c}
x + y = 5\\
2x + 3y = 16\\
\end{array} 
  \right. $  \\                                      
                   admet pour solution le couple\ldots}{%
 {$(6 ; -1)$},
 {$(1 ; -6)$},
 {$(-1 ; 6)$}
 }

\AQquestion[br=2]{$(E)$ désigne le système    
$\; \left\lbrace \begin{array}{c}
        2x + y = 1\\
      x - y = 2\\
\end{array} 
  \right. $   \\ 
Pour résoudre $(E)$, on exprime $y$ en fonction de $x$ à l'aide de la 1\textsuperscript{re} équation et on remplace $y$  par cette expression dans la 2\textsuperscript{e} équation. On obtient les deux équations\ldots
}{%
 {$  \left\lbrace\begin{array}{l}
   y = 1 -2x\\
      x - 1 -2x  = 2\\
\end{array}  \right. $},
 {$ \left\lbrace\begin{array}{l}
   y = 1 -2x\\
      x - (1 -2x)  = 2\\
\end{array}  \right. $},
 {$ \left\lbrace \begin{array}{l}
y = 2x -1\\
      x - (2x -1) = 2\\
\end{array}  \right. $}
 }

\AQquestion[br=1]{Pour lire graphiquement la solution du système \centerline{$ \left\lbrace \begin{array}{c}
-x + y = 5\\
x + 2y = 1\\
\end{array} 
  \right. $ } \\on trace dans un repère les droites d'équations\ldots}{%
 {$y=x+5\qquad $ et $\quad y=\dfrac{1}{2} - \dfrac{1}{2} x$},
  {$y=-x-5\quad $ et $\quad y=\dfrac{x}{2} - \dfrac{1}{2}$},
 {$y=x+5\qquad $ et $\quad y=-2x+2$}
 }

\AQquestion[br=3]{Pour lire graphiquement la solution du système \centerline{$ \left\lbrace \begin{array}{c}
-x + y = 5\\
x + 2y = 1\\
\end{array} 
  \right. $ } \\on trace deux droites $(d)$ et $(d')$ dans un repère.\\
  La solution est donnée par\ldots}{%
   {\begin{minipage}[t]{6cm} les abscisses des points d'intersection \\ de $(d)$ et $(d')$ avec l'axe des abscisses,\end{minipage}},
   {\begin{minipage}[t]{6cm} les ordonnées des points d'intersection\\ de $(d)$ et $(d')$ avec l'axe des ordonnées,\end{minipage}},
   {\begin{minipage}[t]{6cm} les coordonnées du point d'intersection\\
    de $(d)$ et $(d')$.
    \end{minipage}}
 }

\AQquestion[br=2]{Une personne dispose de 6€ ; elle peut dépenser cette somme soit en achetant 10 croissants et un cake soit en achetant 4 croissants et 2 cakes.\\
Pour calculer le prix d'un croissant et celui d'un cake, on peut résoudre le système\ldots
}{%
 {$\left\lbrace \begin{array}{l}
   10x +4y = 6\\
      x +2y  = 6\\
\end{array} \right.$},
 {$\left\lbrace  \begin{array}{l}
   10x + y = 6\\
      4x +2y  = 6\\
\end{array} \right. $},
 {$\left\lbrace  \begin{array}{l}
   10x +2y = 6\\
      x +4y  = 6\\
\end{array} \right. $}
 }

\end{alterqcm}


\ifdefined\COMPLETE
\else
    \end{document}
\fi