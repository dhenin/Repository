\ifdefined\COMPLETE
\else
    \input{./preambule_Analyse_binaire.ltx}
    \begin{document}
\fi

\chapter*{Notations pour les ensembles }

Il est incontestable que les mathématiques ont une base expérimentale,
et malgré l'abstraction que leur confère la pensée, leur valeur ne
peut être complète que lorsqu'elles reviennent aux sources et aboutissent
à des applications pratiques.

Des considérations générales montrent que la pensée logique se caractérise,
fondamentalement, par la possibilité élémentaire d'identification
qui gouverne le jugement et détermine les actions à entreprendre en
fonction de connaissances acquises. L'identification peut revêtir
deux aspects différents suivant les conditions qui la déterminent.

\begin{itemize}
\item C'est d'une part la synthèse, qui n'est en somme que l'identification
globale d'un tout particulier, 
\item et d'autre part, l'analyse, qui consiste en une identification par
décomposition d'un tout en ses éléments constitutifs déjà identifiés
eux-mêmes.
\end{itemize}

L'analyse et la synthèse expriment, en fait, un même cheminement parcouru
dans deux sens différents.

La pensée peut ainsi évoluer, s'élever, se perfectionner, puisqu'il
lui est possible, à tout instant, de modifier sa progression et de
revenir sur ses pas lorsque la voie empruntée s'avère sans issue.
Analyse et synthèse donnent à la pensée une liberté essentielle qui
l'autorise à une remise en cause permanente de l'acquis en fonction
des éléments nouveaux retenus au cours de son évolution.

Exprimés sous forme mathématique, nous retrouvons, à la base de la
théorie des ensembles, ces procédés élémentaires d'analyse et de synthèse
qui traduisent les deux courants inverses et complémentaires suivant
lesquels s'élabore le processus d'identification dans la pensée logique.

L'identification d'un ensemble ou d'une collection quelconque d'objets
n'est rendue possible que par la définition qui en est donnée.

\begin{itemize}
\item Il est possible d'établir une liste nominative de tous les objets
ou éléments identifiés qui le constituent. Dans ce cas, l'ensemble
est défini en \emph{extension}.
\item Il est possible, également, de préciser une ou plusieurs propriétés
nécessaires et suffisantes auxquelles satisfont tous les éléments
de l'ensemble. L'ensemble est défini en \emph{compréhension}.
\end{itemize}

L'ensemble « $E_{c}$ » des chiffres $1,\:3\:,5\:,7\:,9$ par
exemple, est défini en extension, alors que l'ensemble « $E_{i}$ » des
nombres impairs, se trouve par contre défini en compréhension, puisque
tout élément qui appartient à « $E{i}$ » possède nécessairement
la propriété d'être un nombre ayant la qualité d'être impair. Si « $n_{i}$ » est
un nombre impair, nous disons que « $n_{i}$ » appartient à l'ensemble
« $E_{i}$ » et nous écrivons :
\medskip
\begin{center}
\fbox{\parbox[c]{0.08\columnwidth}{%
$n_{i}\in E_{i}$%
}}
\end{center}

\medskip


Si « $n_{p}$ » est un nombre pair, nous disons que « $n_{p}$ » 
n'appartient pas à « $E_{i}$ » , et nous écrivons :

\medskip


\begin{center}
\fbox{\begin{minipage}[c]{0.2\columnwidth}%
\begin{center}
$n_{p}\notin E_{i}$
\end{center}%
\end{minipage}}
\end{center}

\medskip


L'ensemble « $E_{c}$ » est constitué des chiffres impairs $1,3,5,7,9,$
chacun de ces éléments appartient à « $E_{i}$ » sans constituer
la totalité des éléments de ce dernier. Nous disons alors que l'ensemble
« $E_{c}$ » est inclus dans l'ensemble « $E_{i}$ » et
nous écrivons~:

\medskip


\begin{center}
\fbox{\begin{minipage}[c]{0.2\columnwidth}%
\begin{center}
$E_{c}\subseteq E_{i}$
\end{center}%
\end{minipage}}
\end{center}


\medskip


L'inclusion de « $E_{c}$ » dans « $E_{i}$ » implique que
tout élément «  $x$  » appartenant à « $E_{c}$ »  ap\-par\-tient
également à « $E_{i}$ » ~. L'implication s'écrit à l'aide d'un
signe égal complété par une flèche ($\Longrightarrow$) dirigé dans
le sens de l'implication. Dans le cas considéré, nous écrivons :
\medskip
\begin{center}
\fbox{\begin{minipage}[c]{0.3\columnwidth}%
\begin{center}
si $E_{c}\subseteq E_{i}$, $(x\in E_{c})\Longrightarrow(x\in E_{i})$
\end{center}%
\end{minipage}}
\end{center}
\medskip
et nous lisons : si « $E_{c}$ » est inclus dans « $E_{i}$ » ~,
« $x$ » appartenant à « $E_{c}$ » implique que «  $x$
 » appartient à « $E_{i}$ » ~.

Si deux ensembles sont égaux ($E_{i}=E_{j}$), l'implication est valable
dans les deux sens. Tout élément de « $E_{i}$ » appartient à
« $E_{j}$ » et réciproquement. Nous écrivons alors~:
\medskip
\begin{center}
\fbox{\begin{minipage}[c]{0.3\columnwidth}%
\begin{center}
si $E_{i}=E_{j}$, $(x\in E_{i})\Longleftrightarrow(x\in E_{j})$
\end{center}%
\end{minipage}}
\end{center}
\medskip
Définissons l'ensemble « $E_{9}$ » des neufs premiers nombres
entiers $1,2,3,4,5,6,7,8,9$. Cet ensemble contient les éléments $2,4,6,8$,
qui n'appartiennent pas à « $E_{i}$ » ~. « $E_{9}$ » n'est
donc pas inclus dans « $E_{i}$ » et nous écrivons~:
\medskip
\begin{center}
\fbox{\begin{minipage}[c]{0.2\columnwidth}%
\begin{center}
$E_{9}\not\subseteq E_{i}$
\end{center}%
\end{minipage}}
\end{center}
\medskip
Deux ensembles sont égaux s'ils contiennent exactement les mêmes éléments.
Pour vérifier que deux ensembles sont égaux il faut donc s'assurer
que tout élément du premier appartient au second et que tout élément
du second appartient au premier.

Il y a donc lieu de noter également l'utilisation courante des signes
équivalents à des phrases telles que : « quel que soit\ldots{} » ($\forall$)
ou « il existe\ldots » ($\exists$).

\medskip

\begin{center}
\textit{$\forall a$, signifie : quel que soit « $a$ ».}

\medskip

\textit{$\exists x$, signifie : il existe « $x$ ».}  
\end{center}

\newpage 

En utilisant tous les éléments communs à deux ensembles « $A$ » et
« $B$ » ~, il est possible de constituer un ensemble « $I$ » ~.
Nous disons que « $I$ » représente l'\textit{intersection} des
ensembles « $A$ » et « $B$ » et nous écrivons~:
\medskip
\begin{center}
\fbox{\begin{minipage}[c]{0.2\columnwidth}%
\begin{center}
$I=A\cap B$
\end{center}%
\end{minipage}}
\end{center}
\medskip
\begin{center}
\fbox{\begin{minipage}[c]{0.5\columnwidth}%
\begin{center}
$(x\in A$ et $x\in B)\Longleftrightarrow(x\in I)$
\end{center}%
\end{minipage}}
\end{center}
\medskip
Nous pouvons également rassembler ou réunir la totalité des éléments
appartenant aux ensembles « $A$ » et « $B$ » ~. Nous définissons
alors {\samepage un nouvel ensemble « $R$ » qui est la \textit{réunion
des ensembles} « $A$ » et « $B$ » et nous écrivons~:

\nopagebreak\medskip

\begin{center}
\fbox{\begin{minipage}[c]{0.2\columnwidth}%
\begin{center}
$R=A\cup B$
\end{center}%
\end{minipage}}
\end{center}
\medskip
\nopagebreak

Nous en déduisons les implications :
\medskip
\begin{center}
$(x\in A)\Longrightarrow(x\in R)$
\end{center}
\medskip
\begin{center}
$(x\in B)\Longrightarrow(x\in R)$ 
\end{center}
\medskip
Si à l'élément générique « $a$ » d'un ensemble « $A$ » nous
faisons correspondre un élément « $b$ » ~, et un seul, de l'ensemble
« $B$ » ~, « $b$ » est appelé image de « $a$ » ~. Si tout
élément de « $B$ » est l'image d'un ou de plusieurs éléments
de « $A$ » ~, nous disons que la correspondance est une \textit{application
de l'ensemble « $A$ » sur l'ensemble « $B$ » ~}. Si, seuls
certains éléments de « $B$ » sont des éléments de « $A$ » ~,
la correspondance est une application de « $A$ » dans « $B$ » ~.
Si l'application est réciproque et biunivoque, elle prend le nom de
\textit{bijection}.

Nous pouvons établir, entre les éléments d'un ensemble, des relations
de composition interne dont nous allons définir les propriétés élémentaires
parmi lesquelles : \textit{la réflexivité}, \textit{la symétrie} et
\textit{la transitivité}.

Soit « ${\cal R}$ » une relation de composition interne :
\begin{itemize}
\item « ${\cal R}$ » est une relation réflexive si nous pouvons écrire
$A\;{\cal R}\;A$,
\item « ${\cal R}$ » possède la propriété de symétrie si $(A\;{\cal R}\;B)\Longleftrightarrow(B\;{\cal R}\;A)$,
\item « ${\cal R}$ » est une relation transitive si $(A\;{\cal R}\;C$
et $C\;{\cal R}\;B)\Longrightarrow(A\;{\cal R}\;B)$.
\end{itemize}
Lorsqu'une relation « ${\cal R}$ » ~possède simultanément les trois
propriétés, nous disons que c'est une relation d'\emph{équivalence}.

Nous pouvons aussi définir dans l'ensemble « $E$ » ~des \emph{lois
de composition internes} si à tout couple d'éléments $(a,b)\in E$
nous faisons correspondre un élément $d\in E$ tel que $d=a\circ b)$.
\begin{itemize}
\item La loi est commutative si $a\circ b=b\circ a$,
\item elle est associative si $a\circ b\circ c=(a\circ b)\circ c=a\circ(b\circ c)$. 
\end{itemize}
L'élément « $e\in E$ » est appelé \textit{élément neutre} de
la loi interne « $\circ$ » de « $E$ » si $\forall x\in E$,
nous pouvons écrire $e\circ x=x\circ e=x$.

L'élément « $a\in E$ » est appelé \textit{élément absorbant}
de la la loi interne « $\circ$ » ~de « $E$ » ~, si $\forall x\in E$ » ~,
nous pouvons écrire :

\begin{center}
$a\circ x=x\circ a=a$
\end{center}

L'existence d'un certain nombre de lois dont les propriétés sont posées
comme axiomes, définit dans un ensemble, une structure algébrique.

\ifdefined\COMPLETE
\else
    \end{document}
\fi
