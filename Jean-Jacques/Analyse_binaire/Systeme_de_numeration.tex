\ifdefined\COMPLETE
\else
    \input{./preambule_Analyse_binaire.ltx}
    \begin{document}
\fi
\chapter{Systèmes de numération}

\section{Généralités}

Il est aisé, connaissant la base du système de numération utilisé,
d'exprimer un nombre dans un autre système. Il suffit pour cela d'exprimer
les chiffres et les poids dans le système de numération choisi.

Exemples~:
\begin{itemize}
\item Dans le système à base « six » ~$(0,1,2,3,4,5)$ le nombre « 324 » est
représenté en décimal par :

\begin{center}
$3\times(6)^{2}+2\times(6)+4=124$
\end{center}
\item Transposons le nombre $3243312_{5}$ dans le système hexadécimal (base
16), en passant par le système décimal.

\begin{eqnarray*}
3243312_{5} & = & 3\times(5)^{6}+2\times(5)^{5}+4\times(5)^{4}+3\times(5)^{3}+3\times(5)^{2}+1\times(5)+2\\
 & = & 56082_{10}
\end{eqnarray*}

Pour calculer la valeur en base 16, on effectue les divisions successives
par « 16 » ~du nombre décimal :

\begin{center}
\begin{tabular}{lrrr}
56082  & \multicolumn{1}{|l}{16} &  & \tabularnewline
\cline{1-2} \cline{4-4} 
080  & 3505  & \multicolumn{1}{|l}{16} & \tabularnewline
\cline{1-1} \cline{3-3} 
00082  & \multicolumn{1}{l}{030} & 219  & \multicolumn{1}{|l}{16}\tabularnewline
\hline 
000(2)  & 145  & 059  & (13) \tabularnewline
 & 0(1)  & (11)  & (D) \tabularnewline
 &  & (B)  & \tabularnewline
\end{tabular}
\end{center}

D'où

\begin{center}
${\rm 3243312}_{5}={\rm DB12}_{16}$
\end{center}
\end{itemize}

\section{Numération binaire}

La correspondance entre nombres binaires et décimaux s'établit aisément
selon les tableaux suivants~:

\begin{center}
\begin{tabular}{|r|r|c|lcr|r|}
\hline 
\multicolumn{1}{|c|}{\emph{ décimal}} & \multicolumn{1}{|c||}{\emph{ binaire}} &  & \multicolumn{3}{|c|}{\emph{ décimal}} & \multicolumn{1}{|c|}{\emph{ binaire}}\tabularnewline
\hline 
1  & 1  &  &  &  & 2  & 10 \tabularnewline
2  & 10  &  & $2^{2}$  & =  & 4  & 100 \tabularnewline
3  & 11  &  & $2^{3}$  & =  & 8  & 1000 \tabularnewline
4  & 100  &  & $2^{4}$  & =  & 16  & 10000 \tabularnewline
5  & 101  &  & $2^{5}$  & =  & 32  & 100000 \tabularnewline
6  & 110  &  & $2^{6}$  & =  & 64  & 1000000 \tabularnewline
7  & 111  &  & $2^{7}$  & =  & 128  & 10000000 \tabularnewline
8  & 1000  &  & $2^{8}$  & =  & 256  & 100000000 \tabularnewline
9  & 1001  &  & $2^{9}$  & =  & 512  & 1000000000 \tabularnewline
10  & 1010  &  & $2^{10}$  & =  & 1024  & 10000000000 \tabularnewline
11  & 1011  &  & $2^{11}$  & =  & 2048  & 100000000000 \tabularnewline
12  & 1100  &  & $2^{12}$  & =  & 4096  & 1000000000000 \tabularnewline
13  & 1101  &  &  &  &  & \tabularnewline
14  & 1110  &  &  &  &  & \tabularnewline
15  & 1111  &  &  &  &  & \tabularnewline
16  & 10000  &  &  &  &  & \tabularnewline
\hline 
\end{tabular}
\end{center}

Pour passer d'un nombre binaire, au même nombre exprimé dans le système
à base $b=(2)^{n}$, on groupe les chiffres binaires $n$ par $n$
en partant de la droite, puis on remplace chacun des groupes obtenus
à la place qu'il occupe par le chiffre du système à base $(2)^{n}$
qui lui correspond.

\ifdefined\COMPLETE
\else
    \end{document}
\fi
