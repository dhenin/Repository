\section{Représentation analytique d'une droite de l'espace}
\begin{bclogo}{Représentation paramétrique d'une droite}
$\mathcal{D}$, une droite de l'espace passant par $A(a_A;y_A;z_A)$ et de vecteur directeur $\vect{u}(a;b;c)$.

$M(x;y;z)$ : $\vect{AM}$ et $\vect{u}$ sont colinéaires. $\vect{AM}=t\vect{u}$.

\[\left\lbrace\begin{array}{l} x-x_A=a.t\\ y-y_A=b.t\\ z-z_A=c.t \end{array}\right.\;\Longleftrightarrow\;(S)\left\lbrace\begin{array}{l} x=x_A+a.t\\ y=y_A+b.t\\ z=z_A+c.t \end{array}\right.\]

Le système $(S)$ est une représentation paramétrique de la droite $\mathcal{D}$.
\end{bclogo}

\medskip

\begin{bclogo}{Système de deux équations cartésiennes représentant une droite}
Soit $\mathcal{P}$, le plan d'équation $ax+by+cz+d=0$, de vecteur normal $\vect{n} (a;b;c)$.

Soit $\mathcal{Q}$, le plan d'équation $a'x+b'y+c'z+d'=0$, de vecteur normal $\vect{n} (a';b';c')$.

Supposons que $\vect{n'}$ et $\vect{n}$ ne sont pas colinéaires. Donc $\mathcal{P}$ et $\mathcal{Q}$ sont sécants; soit $\mathcal{D}$ la droite d'intersection des plans $\mathcal{P}$ et $\mathcal{Q}$.

La droite $\mathcal{D}$ est représentée par le système $\left\lbrace\begin{array}{l} ax+by+cz+d=0\\ a'z+b'y+c'z+d'=0 \end{array}\right.$

Réciproquement, si les vecteurs $\vect{n}(a;b;c)$ et $\vect{n'}(a';b';c')$ ne sont pas colinéaires, l'ensemble des points $M(x;y;z)$  tels que $\left\lbrace\begin{array}{l} ax+by+cz+d=0\\ a'z+b'y+c'z+d'=0 \end{array}\right.$ est une droite. C'est l'intersection des plans d'équations respectives $ax+by+cz+d=0$ et $a'x+b'y+c'z+d'=0$.
\end{bclogo}