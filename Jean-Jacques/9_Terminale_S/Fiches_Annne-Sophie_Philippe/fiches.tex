\documentclass[11pt,a4paper]{article}
\usepackage[utf8]{inputenc}
\usepackage{amsmath}
\usepackage [utopia]{mathdesign} 
\usepackage{geometry} %marge
\geometry{hmargin=2cm,vmargin=2.5cm}
\usepackage{fancyhdr}
\usepackage{fancybox}
\pagestyle{fancy}
\usepackage{graphicx} 
\usepackage{xcolor}
\usepackage[tikz]{bclogo}
\usepackage{pst-text}
\usepackage{pst-blur}

\usepackage[frenchb]{babel}
\definecolor{bleu}{rgb}{0.2,0.0,1}
\definecolor{vert}{rgb}{0.2,0.8,0.3}
\definecolor{rouge}{RGB}{88,0,0}
\usepackage[colorlinks=true,linkcolor=rouge]{hyperref}
\author{anneso}
\parindent0pt
\renewcommand{\Re}{\,\mathrm{Re}\,}
\renewcommand{\Im}{\,\mathrm{Im}\,}
\newcommand\I{\mathrm{i}}
\newcommand\vect[1]{\overrightarrow{#1}}
\newcommand\boiteo[1]{%
\begin{center}
    \shadowbox%[shadowcolor=gray, shadowsize=3pt]
    {%
        $\displaystyle #1$
    }
\end{center}
}

\definecolor{orange}{RGB}{232,208,157}
%
% jjd
%
%\usepackage{auto-pst-pdf}
%\usepackage{pstricks-add}
\setlength{\headheight}{14pt}
\begin{document}


\begin{titlepage}
\iffalse
\begin{pspicture}(2cm,0)(\paperwidth,24.7)
%\rput(0.5\paperwidth,23{\psscalebox{2.5}{\pscharpath[linecolor=rouge,fillstyle=solid,fillcolor=orange,linewidth=0.7pt,blur=true]{\Huge\textbf{Fiches}}}}
%  \rput(0.5\paperwidth,20.5){\psscalebox{2}{\pscharpath[linecolor=rouge,fillstyle=solid,fillcolor=orange,linewidth=0.7pt,blur=true]{\Huge\textbf{de Mathématiques}}}}
  \rput(0.5\paperwidth,18){\psscalebox{3}{\bfseries Terminal S}}
  \rput[Br](0.95\paperwidth,13.5){\Large Anne-Sophie \textsc{Philippe}}
  \psframe[linecolor=orange, fillstyle=solid,fillcolor=orange](0,-5cm)(\paperwidth,13.2)
  \psframe[linecolor=rouge,fillstyle=solid,fillcolor=rouge](0,13)(\paperwidth,13.4)
  \rput(0.5\paperwidth, 6){\includegraphics{figureJMS.1}}
 \end{pspicture}%
 \fi
\end{titlepage}


\lhead{Table des matières}
\thispagestyle{empty}
\setcounter{page}{0}
\tableofcontents

\newpage

\definecolor{suite}{RGB}{255,250,188}
\presetkeys{bclogo}{arrondi=0.1, couleur =suite, logo=\bccrayon}{} 


\section{Suites}
\subsection{Rappels sur les suites}

\begin{bclogo}{Variations d'une suite}
$\ast$ La suite $(u_n)_{n\in\mathbb{N}}$ est croissante à partir du rang $n_0$ si et seulement si, pour tout $n\geqslant n_0$, $u_{n+1}\geqslant n_n$.

\noindent
$\ast$ La suite $(u_n)_{n\in\mathbb{N}}$ est décroissante à partir du rang $n_0$ si et seulement si, pour tout $n\geqslant n_0$, $U_{n+1}\leqslant U_n$.

\noindent
$\ast$ Une suite $(u_n)_{n\in\mathbb{N}}$ est dite monotone si elle est croissante ou décroissante.
\end{bclogo}

\medskip

\begin{bclogo}{Etude du sens de variation d'une suite}
$\ast$ Etude du signe de $u_{n+1}-u_n$.

\noindent
$\ast$ $u_n=f(n)$, si $f$ est monotone sur $\left[ 0;+\infty \right]$, alors la suite $(u_n)_{n\in\mathbb{N}}$ est monotone, de même variation que $f$ (formule explicite).

\noindent
$\ast$ Si $(u_n)_{n\in\mathbb{N}}$ est strictement positive, on peut comparer $\dfrac{u_{n+1}}{u_n}$ et 1. 

Si $\dfrac{u_{n+1}}{u_n}>1$, $(u_n)_{n\in\mathbb{N}}$ est strictement croissante.

Si $\dfrac{u_{n+1}}{u_n}<1$, $(u_n)_{n\in\mathbb{N}}$ est strictement décroissante.
\end{bclogo}

\medskip

\begin{bclogo}{Suites majorées, minorées, bornées...}
$\ast$ La suite $(u_n)_{n\in\mathbb{N}}$ est majorée s'il existe un réel $M$ tel que pour tout entier $n$, $u_n\leqslant M$.

\noindent 
$\ast$ La suite $(u_n)_{n\in\mathbb{N}}$ est minorée s'il existe un réel $m$ tel que pour tout entier $n$, $u_n\geqslant m$.

\noindent 
$\ast$ La suite $(u_n)_{n\in\mathbb{N}}$ est bornée si elle est à la fois majorée et minorée. 
\end{bclogo}



\subsection{Suites arithmétiques et suites géométriques}

\begin{bclogo}{Suites arithmétiques}
$\ast$ Une suite $(u_n)_{n\in\mathbb{N}}$ est arithmétique s'il existe un réel $r$ (la raison) indépendant de $n$ tel que, pour tout $n\in \mathbb{N}$, \[\boxed{u_{n+1}=u_n+r}\]. 

\noindent
$\ast$ Pour tous entiers $n$ et $p$, $u_n=u_p+(n-p)\times r$.

\noindent
$\ast$ $u_n=u_0+n.r$.

\noindent
$\ast$ $\lim\limits_{n \to +\infty} u_n= \left\lbrace\begin{array}{l}+\infty ,$ si $r>0 \\-\infty, $ si $ r<0 \end{array}\right.$

\noindent
$\ast$ Somme de termes consécutifs :
\[\text{(nombre de termes)} \times \dfrac{1^{\text{er}} \text{terme} \times \text{dernier terme} }{2}\]

Exemple :

\boiteo{1+2+...+n=\dfrac{n\times (n+1)}{2}}
\end{bclogo}


\begin{bclogo}{Suites géométriques}
$\ast$ Une suite $(u_n)_{n\in\mathbb{N}}$ est géométrique s'il existe un réel $q$ (la raison) indépendant de $n$ tel que, pour tout $n\in \mathbb{N}$, \[\boxed{u_{n+1}=u_n+q}\]. 

\noindent
$\ast$ Pour tous entiers $n$ et $p$, $u_n=u_p\times q^{n-q}$.

\noindent
$\ast$ $u_n=u_0\times q^n$.

\noindent
$\ast$ $\lim\limits_{n \to +\infty} q^n= \left\lbrace\begin{array}{l}+\infty ,$ si $q>1 \\0, $ si $ 0<q<1 \end{array}\right.$

\noindent
$\ast$ Somme de termes consécutifs :
\[(1^{\text{er}}\text{termes}) \times \dfrac{1-q^{\text{nombre de termes}}}{1-q}\]

Exemple :

\boiteo{1+q^1+q^2+...+q^n=1\times \dfrac{1-q^{n+1}}{1-q}}

Attention : nombre de termes $=n+1-1^{\text {er}} \text{terme}$
\end{bclogo}



\subsection{Démonstration par récurrence}

\begin{bclogo}{Démonstration par récurrence}
Pour démontrer que pour tout entier $n\geqslant n_0$, $P_n$ (proposition qui dépend de $n$) est vraie, il faut :

\noindent
$\ast$ \textbf{Initialisation} : vérifier que $P_{n_0}$ est vraie pour $n_0\geqslant 0$.

\noindent
$\ast$ \textbf{Hypothèse de récurrence} : considérer que $P_k$ est vraie pour un certain entier $k\geqslant n_0$.

\noindent
$\ast$ \textbf{Propriété d'hérédité} : démontrer que $P_{n+1}$ est vraie.

\noindent
$\ast$ \textbf{Conclusion} : pour tout $n\geqslant n_0$, $P_n$ est vraie.
\end{bclogo}

\subsection{Limite d'une suite}

\begin{bclogo}{Limites d'une suite numérique $(u_n)_{n\in\mathbb{N}}$}
$\ast$ La suite $(u_n)_{n\in\mathbb{N}}$ converge vers un réel $\ell$. Ceci signifie que tout intervalle contenant $\ell$ contient aussi tous les termes de la suite à partir d'un certain rang $p$.

\[\lim\limits_{n \to +\infty} u_n=\ell \]

$(u_n)_{n\in\mathbb{N}}$ est convergente et converge vers $\ell$.

\noindent
$\ast$ La suite $(u_n)_{n\in\mathbb{N}}$ a pour limite $+\infty$. Cela signifie que tout intervalle ouvert $\left] A;+\infty\right[$ contient tous les termes de la suite à partir d'un certain rang $p$. La suite est divergente.

\noindent
$\ast$ La suite $(u_n)_{n\in\mathbb{N}}$ a pour limite $-\infty$. Ceci signifie que tout intervalle ouvert $\left] -\infty ;B\right[ $ contient tous les termes de la suite à partir d'un certain rang $p$. La suite est divergente.

\noindent
$\ast$ La suite $(u_n)_{n\in\mathbb{N}}$ n'admet aucune limite. La suite est divergente.
\end{bclogo}

\newpage

\begin{bclogo}{Suites monotones}
$\ast$ Si une suite $(u_n)_{n\in\mathbb{N}}$ est croissante et non majorée, alors : \[\lim\limits_{n \to +\infty} u_n=+\infty\]

\noindent
$\ast$ Si une suite $(u_n)_{n\in\mathbb{N}}$ est décroissante et non minorée, alors : \[\lim\limits_{n \to +\infty} u_n=-\infty\]

\noindent
$\ast$ Une suite croissante et majorée est convergente.

\noindent
$\ast$ Une suite décroissante et minorée est convergente.
\end{bclogo}
\medskip

\begin{bclogo}{\textbf{ROC 1} : limite d'une suite croissante non majorée}
$\ast$ La suite $(u_n)_{n\in\mathbb{N}}$ n'est pas majorée : quelque soit le réel $A$, on peut trouver un entier $p$ tel que $u_p\geqslant A$.

\noindent
$\ast$ La suite $(u_n)_{n\in\mathbb{N}}$ est croissante. Pour tout $n\geqslant p$: $\left\lbrace\begin{array}{l} u_n\geqslant u_p \\ u_n > A\end{array}\right.$.

\noindent
$\ast$ A partir du rang $p$, tous les termes de la suite sont dans $\left] A;+\infty \right[$.

\noindent
$\ast$ Conclusion : par définition, cela prouve : \[\boxed{\lim\limits_{n \to +\infty} u_n=+\infty}\]
\end{bclogo}

\medskip

\begin{bclogo}{\textbf{ROC 2} : limite d'une suite décroissante non minorée}
$\ast$ La suite $(u_n)_{n\in\mathbb{N}}$ n'est pas minorée : quelque soit le réel $B$, on peut trouver un entier $p$ tel que $u_p\leqslant B$ 

\noindent
$\ast$ La suite $(u_n)_{n\in\mathbb{N}}$ est décroissante. Pour tout $n\geqslant p$ : $\left\lbrace\begin{array}{l} u_n\leqslant u_p \\ u_n < B\end{array}\right.$.

\noindent
$\ast$ A partir du rang $p$, tous les termes de la suite sont dans $\left] -\infty ;B \right[$.

\noindent
$\ast$ Conclusion : par définition, cela prouve : \[\boxed{\lim\limits_{n \to +\infty} u_n=-\infty}\]
\end{bclogo}

\medskip

\begin{bclogo}{\textbf{ROC 3} : limite d'une suite croissante et majorée}
$\ast$ Soit la suite $(u_n)_{n\in\mathbb{N}}$, croissante et majorée par un réel $M$. Notons $A$, le plus petit des majorants.

\noindent
$\ast$ Tout intervalle $\left] A-\alpha ; A+\alpha \right[$  contient au moins un terme $u_p$ de la suite. Sinon, $A-\alpha $ serait un majorant de la suite, ce qui contredit le fait que $A$ soit le plus petit des majorants.

\noindent
$\ast$ La suite $(u_n)_{n\in\mathbb{N}}$ est croissante : pour tout $n\geqslant p$, $u_n \geqslant u_p$.

\noindent
$\ast$ \textbf{Conclusion} : l'intervalle $\left] A-\alpha ; A+\alpha \right[$ contient tous les termes de la suite à partir du rang $p$. Ceci est vrai, quel que soit le réel $\alpha > 0$. 

Par définition, la suite $(u_n)_{n\in\mathbb{N}}$ converge et à pour limite $A$.
\end{bclogo}

\newpage

\begin{bclogo}{\textbf{ROC 4} : limite d'une suite décroissante et minorée}
$\ast$ Soit la suite $(u_n)_{n\in\mathbb{N}}$ décroissante et minorée par un réel $m$. Notons $B$, le plus grand des minorants.

\noindent
$\ast$ Tout intervalle $\left] B-\alpha ; B+\alpha \right[$  contient au moins un terme $u_p$ de la suite. Sinon, $B+ \alpha $ serait un minorant de la suite, ce qui contredit le fait que $B$ soit le plus grand des minorants.


\noindent
$\ast$ La suite $(u_n)_{n\in\mathbb{N}}$ est décroissante : pour tout $n\geqslant p$, $u_n \leqslant u_p$.

\noindent
$\ast$ \textbf{Conclusion} : l'intervalle $\left] B-\alpha ; B+\alpha \right[$ contient tous les termes de la suite à partir du rang $p$. Ceci est vrai, quelque soit le réel $\alpha > 0$. 

Par définition, la suite $(u_n)_{n\in\mathbb{N}}$ converge et à pour limite $B$.
\end{bclogo}

\medskip

\begin{bclogo}{Limite d'une suite géométrique}
$\ast$ Soit $(u_n)_{n\in\mathbb{N}}$, une suite géométrique de raison $q$ non nulle. 

\noindent
Pour tout entier $n$ : \boiteo{u_n=u_0\times q^n}

\noindent
$\ast$ Si $|q|<1$, $\lim\limits_{n \to +\infty} q^n=0$

\noindent
$\ast$ Si $q>1$, $\lim\limits_{n \to +\infty} q^n=+\infty$

\noindent
$\ast$ Si $q=1$, $\lim\limits_{n \to +\infty} q^n=1$

\noindent
$\ast$ Si $q\leqslant -1$, $q^n$ n'a pas de limite.
\end{bclogo}

\medskip

\begin{bclogo}{Théorème d'encadrement (\og{}des gendarmes\fg{})}
Soient trois suites $(u_n)_{n\in\mathbb{N}}$, $(v_n)_{n\in\mathbb{N}}$, $(w_n)_{n\in\mathbb{N}}$ telles que :
 \[\forall n\geqslant n_0,\;\left.\begin{array}{c} v_n\leqslant u_n\leqslant w_n \\ \lim\limits_{n\to +\infty} v_n =\ell \\ \lim\limits_{n\to +\infty} w_n =\ell \end{array}\right\}\lim\limits_{n\to +\infty} u_n =\ell\] 
\end{bclogo}



\subsection{Suites adjacentes}

\begin{bclogo}{Théorème et définition}
Deux suites $(u_n)_{n\in\mathbb{N}}$ et $(v_n)_{n\in\mathbb{N}}$ sont adjacentes si et seulement si :

\noindent
$\ast$ $(u_n)_{n\in\mathbb{N}}$ est croissante.

\noindent
$\ast$ $(u_n)_{n\in\mathbb{N}}$ est décroissante.

\noindent
$\ast$ $\lim\limits_{n \to+\infty} u_n-v_n=0$

\textbf{Théorème} : Si deux suites sont adjacentes alors elles convergent et elles ont la même limite.
\end{bclogo}

\definecolor{fonction}{RGB}{191,227,176}
\presetkeys{bclogo}{arrondi=0.1, couleur =fonction, logo=\bcorne}{} 


\section{Les fonctions}
\subsection{Les limites d'une fonction}

\begin{bclogo}{Définitions}
$\ast$ Limite finie d'une fonction en $+ \text{ ou } -\infty$ : présence d'une assymptote horizontale (d'équation $y=\ell$) à $\mathcal{C}_f$ en $+ \text{ ou } - \infty$.
\[\lim\limits_{x \to +\infty} \dfrac{1}{x^n}=0\]
\[\lim\limits_{x \to +\infty} \dfrac{1}{\sqrt{x}}=0\]


$\ast$ Limite infinie d'une fonction à l'infini. Pas d'assymptote.
\[\lim\limits_{x \to +\infty} x^n=+\infty\]
\[\lim\limits_{x \to +\infty} \sqrt{x}=+\infty\]
\[\lim\limits_{x \to -\infty} x^n=+\infty  \text{ ($n$ pair)}\]
\[\lim\limits_{x \to +\infty} x^n=-\infty  \text{ ($n$ impair)}\]


$\ast$ Cas particulier : \[\lim\limits_{x \to +\infty} f(x)-(ax+b) =0\]
La droite d'équation $y=ax+b$ est assymptote oblique à $\mathcal{C}_f$ en $+ \infty$.

$\ast$ Limite de $f(x)$ quand $x$ tend vers $a$ en $+\infty$ : présence d'une assymptote verticale ($x=a$) à $\mathcal{C}_f$.
\[\lim\limits_{x \to 0^+} \dfrac{1}{x^n}= \lim\limits_{x \to 0^-} \dfrac{1}{x^n}=+\infty \text{ ($n$ pair)}\]
\[\lim\limits_{x \to 0^+} \dfrac{1}{x^n}=+\infty \text{ et}  \lim\limits_{x \to 0^-} \dfrac{1}{x^n}=-\infty \text{ ($n$ impair)}\]

$\ast$ Limite finie de la fonction en un réel $a$. $\lim\limits_{x \to a} f(x)=\ell$

\end{bclogo}

\medskip

\subsection{Opérations sur les limites}
\begin{bclogo}{Formes indéterminées}
\[\;\left.\begin{array}{c}  \lim\limits_{x \to \alpha} f=+\infty \\ \lim\limits_{x\to \alpha } g =-\infty \end{array}\right\}\;\lim\limits_{x\to \alpha} f+g \text{ est indéterminée}\] 

\[\;\left.\begin{array}{c}  \lim\limits_{x \to \alpha} f=\pm \infty \\ \lim\limits_{x\to \alpha } g =0 \end{array}\right\}\;\lim\limits_{x\to \alpha} f\times g \text{ est indéterminée}\] 

\[\;\left.\begin{array}{c}  \lim\limits_{x \to \alpha} f=\pm \infty \\ \lim\limits_{x\to \alpha } g =\pm \infty \end{array}\right\}\;\lim\limits_{x\to \alpha} \frac{f}{g} \text{ est indéterminée}\] 

\[\;\left.\begin{array}{c}  \lim\limits_{x \to \alpha} f=0 \\ \lim\limits_{x\to \alpha } g =0 \end{array}\right\}\;\lim\limits_{x\to \alpha} \frac{f}{g} \text{ est indéterminée}\] 
\end{bclogo}


\medskip 

\begin{bclogo}{Limite d'une fonction polynôme ou d'une fonction rationnelle}
$\ast$ \textbf{Règle 1} : en $\pm \infty$, la limite d'une fonction polynôme est égale à la limite de son terme de plus haut degré.

$\ast$ \textbf{Règle 2} : en $\pm \infty$, la limite d'une fonction rationnelle (quotient de deux polynômes) est égale à la limite du quotient du terme de plus haut degré du numérateur par le terme de plus haut degré du dénominateur.

\end{bclogo}

\medskip
\begin{bclogo}{Composé de deux fonctions}
On note $f$, la composé de $u$ suivie de $v$ : \[f=v\circ u\]

\[\;\left.\begin{array}{c}  \lim\limits_{x \to a} u(x)=b \\ \lim\limits_{x\to b } v(x) =c \end{array}\right\}\lim\limits_{x\to a} v\circ u(x) =c\] 

\textbf{Remarque }: vérifier les domaines de définition. $u$, définie sur l'intervalle $I$ et $v$ définie sur l'intervalle $J$ tel que : $\forall x \in I, u(x) \in J$
\end{bclogo}

\subsection{Propriétés des limites}

\medskip
\begin{bclogo}{Unicité}
Si $f$ admet une limite en $\alpha$, alors, cette limite est unique.
\end{bclogo}

\medskip

\begin{bclogo}{Théorèmes de comparaison}
$\ast$ \textbf{Théorème 1} : au voisinage de $\alpha$, 

Si $f(x)\geqslant u(x)$ et $\lim\limits_{x \to \alpha} u(x)=+ \infty$, alors, $\lim\limits_{x \to \alpha} f(x)=+\infty$ (1)

Si $f(x)\leqslant v(x)$ et $\lim\limits_{x \to \alpha} u(x)=- \infty$, alors, $\lim\limits_{x \to \alpha} f(x)=-\infty$ (2)

\medskip

$\ast$ \textbf{Démonstrations (ROC)}

(1) Soit, $\alpha =+\infty$. Tout intervalle $\left] M;+\infty\right[ $, où $M$ est un réel, contient tous les $u(x)$ pour $x$ assez grand.

Or, au voisinage de $\alpha$, $f(x)\geqslant u(x)$. Donc, pour $x$ assez grand, tous les $f(x)$ sont contenus dans $\left]M;+\infty \right[$.

Par définition,   \[\lim\limits_{x \to +\infty } f(x)=+\infty\]

(2) Idem

\medskip

$\ast$ \textbf{Théorème 2} : au voisinage de $\alpha$,

Si $\lim\limits_{x \to \alpha} |f(x)-\ell |\leqslant u(x)$ et $\lim\limits_{x \to \alpha} u(x) = 0$

Alors, $\lim\limits_{x \to \alpha} f(x)=\ell$.

\medskip
$\ast$ \textbf{Théorème 3 : Théorème des gendarmes} : au voisinage de $\alpha$

Si $u(x)\leqslant f(x) \leqslant v(x)$ et $\lim\limits_{x \to \alpha} u(x)=\lim\limits_{x \to \alpha} v(x)=\ell$,

alors, $\lim\limits_{x \to \alpha} f(x)=\ell$.

\medskip
$\ast$ \textbf{Démonstration (ROC)}

Soit, $\alpha=+\infty $.

Pour $x>A$ : $u(x)\leqslant f(x)\leqslant v(x)$

$\lim\limits_{x \to +\infty} u(x)=\ell $ signifie que pour $x>B$, $u(x)\in I$ avec $I$ intervalle contenant $\ell$.

$\lim\limits_{x \to +\infty} v(x)=\ell$ signifie que pour $x>C$, $v(x)\in I$.

Prenons $M$ le plus grand des nombres $A,B,C$. 
\[\forall x\geqslant M, \text{on a} \left\lbrace\begin{array}{l} u(x)\leqslant f(x) \leqslant v(x) \\ u(x)\in I\\ v(x)\in I \end{array}\right. \]

Donc $f(x)\in I$.

Par définition, $\lim\limits_{x \to +\infty} f(x)=\ell$.

\medskip
$\ast$ \textbf{Comptabilité avec l'ordre}

Au voisinage de $\alpha$ : si $f(x)\leqslant g(x)$ et $\lim\limits_{x \to \alpha} f(x)=\ell$ et $\lim\limits_{x \to \alpha} g(x)=\ell '$ 

Alors, $\ell \leqslant \ell '$
\end{bclogo}


\subsection{Continuité}
\begin{bclogo}{Définitions et théorèmes}
$\ast$ Si $f$ est continue en $a$ : \[\lim\limits_{x\to a^-} f(x)=\lim\limits_{x\to a^+} f(x)=f(a)\].

$\ast$ Si $f$ est dérivable en $a\in I$, alors $f$ est continue en $a$.

$\ast$ Si $f$ est dérivable sur $I$, alors $f$ est continue sur $I$.

\textbf{Remarque} : la réciproque est fausse, une fonction continue n'est pas toujours dérivable.
\end{bclogo}

\medskip

\begin{bclogo}{Démonstration \textbf{(ROC)} toute fonction dérivable est continue}
$f$ est dérivable en $a$ signifie que, \boiteo{\lim\limits_{x\to a}\frac{f(x)-f(a)}{x-a}=f'(a)}

Soit $g$, la fonction définie sur un voisinage de $a$ par : \[g(x)=\frac{f(x)-f(a)}{x-a}\]
avec $x\neq a$
\[f(x)=(x-a)\times g(x) + f(a)\]

\medskip

$\lim\limits_{x\to a} x-a=0$  et $\lim\limits_{x\to a}g(x)=f'(a)$

\medskip
Donc $\lim\limits_{x\to a} f(x)=f(a)$

\medskip
Par définition, $f$ est continue en $a$.
\end{bclogo}

\medskip

\begin{bclogo}{Cas particuliers}
$\ast$ Les fonctions polynômes sont continues sur $\mathbb{R}$.

$\ast$ Les fonctions rationnelles sont continues sur chacun des intervalles du domaine de définition.

$\ast$ Les fonctions sinus et cosinus sont continues sur $\mathbb{R}$

$\ast$ Toute fonction construite par addition, multiplication ou composition de fonctions continues est une fonction continue.

$\ast$ La fonction racine carrée est définie sur $\left[  0;+\infty \right[$ et est dérivable sur $\left] 0;+\infty\right[$. 

Selon le théorème, cette fonction est continue sur $\left] 0;+\infty\right[$. 

Mais, sa limite en $0$ est $0$ donc elle est continue sur $\left[ 0;+\infty\right[$.
\end{bclogo}


\newpage

\begin{bclogo}{Nombre dérivé}
\boiteo{\lim\limits_{h\to 0}\frac{f(a+h)-f(a)}{h}=\ell}
\[f(a+h)=f(a)+\ell h+h\varphi (h) \text{ avec } \lim\limits_{h\to 0} \varphi (h)=0\]

Si ces propositions sont vraies, $f$ est dérivable en $a$ et $\ell $ est le nombre dérivé de $f$ en $a$ noté $f'(a)$.

Si $f$ est dérivable en $a$, la courbe $\mathcal{C}_f$ admet au point $A(a;f(a))$ une tangente $\mathcal{T}$ dont le coefficient directeur est $f'(a)$. L'équation de $\mathcal{T}$ est : \boiteo{y=f'(a)\times (x-a)+f(a)}

Si la limite du taux d'accroissement entre $a$ et $a+h$ de $f$ est $\pm \infty$, alors $f$ n'est pas dérivable. Il y a pas de tangente verticale en $a$. 

Si les limites sont différentes à droite et à gauche, alors $f$ n'est pas dérivable en $a$. Il y a un point anguleux en $a$.

\end{bclogo}


\medskip

\begin{bclogo}{Théorème des valeurs intermédiaires}
Si $f$ est continue sur $\left[ a;b\right]$, alors, pour tout réel $k$ compris entre $f(a)$ et $f(b)$, il existe au moins un réel $c$ appartenant à $\left[a;b\right]$ tel que \[f(c)=k.\]
 L'équation $f(x)=k$ admet au moins une solution dans $\left[ a;b\right]$.
\end{bclogo}

\medskip

\begin{bclogo}{Théorème de bijection ou corollaire du theorème des valeurs intermédiaires}
Si $f$ est continue et strictement croissante sur $\left[ a;b\right]$, $f(\left[ a;b\right])=\left[ f(a);f(b)\right]$.

Alors,
\begin{center} $\forall y\in \left[ f(a);f(b)\right]$, il existe un et un seul réel $c\in \left[ a;b\right]$ tel que $f(c)=y$.
\end{center}

L'équation $f(x)=y$ admet une et une seule solution dans $\left[ a;b\right]$. 

Idem pour une fonction strictement décroissante.     $f(\left[ a;b\right])=\left[ f(b);f(a)\right]$.

Toute fonction continue et strictement monotone sur un intervalle donné réalise une bijection... 
\end{bclogo}


\medskip

\begin{bclogo}{\textbf{Démonstration (ROC)}}
$\ast$ Supposons $f$ continue et strictement croissante sur $\left[a;b\right]$.

$\ast$ Existence :

$f$ est continue sur $\left[a;b\right]$. D'après le théorème des valeurs intermédiaires, $\forall y \in \left[ f(a);f(b)\right]$, l'équation $f(x)=y$ admet au moins une solution.  

$\ast$ Unicité :

Supposons que $f(c_1)=f(c_2)=y$ avec $c_1<c_2$. $f$ est strictement croissante sur $\left[ a;b\right]$, alors pour $c_1<c_2$ on a $f(c_1)<f(c_2)$. 

Cela contredit la supposition $f(c_1)=f(c_2)=y$.

Donc, il existe un seul réel $c$ tel que $f(c)=y$.
\end{bclogo} 

\medskip

\subsection{Dérivation}

\begin{bclogo}{Rappels}
$\ast$ $f$ est constante si et seulement si $f'$ est nulle.

$\ast$ $f$ est croissante si et seulement si $f'$ est positive.

$\ast$ $f$ est décroissante si et seulement si $f'$ est négative.

$\ast$ Si $f(a)$ est un extremum local de $f$ en $a$ alors, $f'(a)=0$. (réciproque fausse)

$\ast$ Si $f'$ s'annule et change de signe en $a$ alors, $f(a)$ est un extremum local.
\end{bclogo}

\medskip

\begin{bclogo}{Dérivée d'une fonction composée}
$g$ dérivable sur $J$ et $u$ dérivable sur $I$ tels que : $\forall x\in I, u(x)\in J$.

Alors, $f=g\circ u$ est dérivable sur $I$ et on a $(g\circ u)'(x)=g'(u(x))\times u'(x)$

\boiteo{(g\circ u)'=(g'\circ u)\times u'}
\end{bclogo}

\medskip


\begin{bclogo}{Exemples importants}
$u$, fonction positive et dérivable sur $I$. 

$\ast$ $f=\sqrt{u}$ est dérivable et donne : $(\sqrt{u})'=\frac{u'}{2\sqrt{u}}$.

$\ast$ $f=u^n$ est dérivable et donne : $(u^n)'=n\times u^{n-1}\times u'$
\end{bclogo}









\definecolor{exp}{RGB}{205,209,255}
% \presetkeys{bclogo}{arrondi=0.1, couleur =exp, logo=\bcying}{}


\section{Fonction exponentielle et équation différentielle}
\begin{bclogo}{Définition}
On dit que $f$, fonction dérivable sur un intervalle $I$, est solution de l'équation différentielle $y'=k.y$, lorsque $\forall x\in I$, $f'(x)=k.f(x)$.
\end{bclogo}



\begin{bclogo}{Fonction exponentielle}
Il existe \emph{une et une seule fonction} dérivable sur $\mathbb{R}$ telle que $y'=y$ et $y(0)=1$ (condition initiale). C'est la fonction exponentielle.

\begin{center}
\begin{tabular}{ll}
$\ast$ $(\exp )'=\exp $&$\ast$ $e^{a+b}=e^a \times e^b$\\

$\ast$ $e^{2a}=\left[ e^a\right] ^2$&$\ast$ $e^{-a}=\frac{1}{e^a}$\\

$\ast$ $e^{a-b}=\frac{e^a}{e^b}$&$\ast$ $\left( e^a\right) ^n=e^{n.a}$\\

$\ast$ $e^0=1$
\end{tabular}
\end{center}
$\ast$ La fonction exponentielle est strictement croissante sur $\mathbb {R}$

$\ast$ $\lim\limits_{x\to +\infty} e^x=+\infty$ et $\lim\limits_{x\to -\infty} e^x=0$

$\ast$ $\lim\limits_{h\to 0} \frac{e^h-1}{h}=1$

$\ast$ $\lim\limits_{x\to +\infty} \frac{e^x}{x}=+\infty$ et $\lim\limits_{x\to -\infty} x.e^x=0$

$\ast$ $\lim\limits_{x\to +\infty}\frac{e^x}{x^n}=+\infty$ et $\lim\limits_{x\to -\infty} x^n.e^x=0$

$\ast$ Fonction composée $e^u$. $\left( e^u\right)'=u'.e^u$ 

\end{bclogo}

\medskip

\begin{bclogo}{Equation $y'=a.y$}
L'ensemble des solutions dans $\mathbb{R}$ de l'équation $y'=ay$ est l'ensemble des fonctions \[x\mapsto c.e^{ax}\] où $c$ est un réel quelconque.

Il existe \emph{une unique solution} vérifiant la condition initiale $y'(x_0)=y_0$.
\end{bclogo}

\medskip

\begin{bclogo}{Equation $y'=ay+b$}
Les solutions de l'équation $(E):y'=a.y+b$ sont les fonctions définies sur $\mathbb{R}$, de la forme $f-\frac{b}{a}$ où $f$ est solution de $y'=ay$. C'est-à-dire
\[x\mapsto Ce^{ax}-\frac{b}{a}\]
où $C\in\mathbb{R}$. Si $y(x_0)=y_0$, $(E)$ admet \emph{une unique} solution.
\end{bclogo}


\definecolor{ln}{RGB}{203,212,212}
\presetkeys{bclogo}{arrondi=0.1, couleur =ln, logo=\bcrosevents}{}

\section{Fonction logarithme népérien}

\begin{bclogo}{Propriétés}
\begin{center}
\begin{tabular}{ll}

$\ast$ $\ln (1)=0$&$\ast$ $\ln (e)=1$\\

$\ast$ $e^{\ln (a)}=a$&$\ast$ $\ln (e^a)=a$\\

$\ast$ $\ln (a\times b)=\ln (a)+\ln (b)$&$\ast$ $\ln \left( \frac{a}{b}\right) =\ln (a)-\ln (b)$\\

$\ast$ $\ln \left( \frac{1}{a}\right) =-\ln (a)$&$\ast$ $\ln \sqrt{a}=\frac{1}{2}\ln (a)$\\

$\ast$ $\ln (a^n)=n\times \ln(a)$
\end{tabular}
\end{center}
\end{bclogo}
\medskip


\begin{bclogo}{Etude de la fonction}
$\ast$ La fonction $\ln$ est définie et continue sur $\left] 0;+\infty\right[$.

$\ast$ $\forall x\in \left] 0;+\infty\right[,\;\ln '(x)=\frac{1}{x}$.

$\ast$ La fonction $\ln$ est croissante sur $\left]0;+\infty\right[$.

$\ast$ $\lim\limits_{x\to +\infty} \ln (x)=+\infty$ et $\lim\limits_{x\to 0} \ln (x)=-\infty$

$\ast$ $\lim\limits_{x\to +\infty} \frac{\ln (x)}{x}=0$ et $\lim\limits_{x\to 0} x.\ln (x)=0$

$\ast$ $\lim\limits_{x\to 1} \frac{\ln (x)}{x-1}=1$ et $\lim\limits_{x\to 0} \frac{\ln (x+1)}{x}=1$

\end{bclogo}

\medskip

\begin{bclogo}{\textbf{Démonstration (ROC)}}  
Soit $a>0$, démontrons que $\lim\limits_{h\to 0}\frac{\ln (a+h)-\ln (a)}{h}=\frac{1}{a}$ ou $\lim\limits_{x\to 0}\frac{\ln (a)-\ln (x)}{a-x}=\frac{1}{a}$.

\vspace{0.3cm}

Posons $A=\ln (a)$, $a=e^A$ et $X=\ln (x)$, $x=e^X$.

\vspace{0.3cm}

$\frac{\ln (a)-\ln (x)}{a-x}=\frac{A-X}{e^A-e^X}=\frac{1}{\frac{e^A-e^X}{A-X}}$.

\vspace{0.3cm}

Comme $\ln$ est continue sur $\left] 0;+\infty\right[$, $\lim\limits_{x\to a} \ln (x)=\ln (a)$.

\vspace{0.3cm}

$\lim\limits_{X\to \ln (a)} \frac{e^A-e^X}{A-X}= \lim\limits_{X\to A} \frac{e^A-e^X}{A-X}=\exp '(A)=\exp (A)=\exp (\ln (a))=a$.

\vspace{0.3cm}

$\lim\limits_{x\to a} \frac{\ln (a)-\ln (x)}{a-x}=\lim\limits_{X\to A} \frac{1}{\frac{e^A-e^X}{A-X}}=\frac{1}{a}$.

\vspace{0.3cm}

D'où $\ln$ est dérivable en $a>0$ et $\ln '(a)=\frac{1}{a}$.
\end{bclogo}

\medskip

\begin{bclogo}{Fonction $\ln \circ u$}
\[(\ln \circ u)'(x)=\frac{u'(x)}{u(x)}\]
\end{bclogo}

\medskip

\begin{bclogo}{Fonction logarithme décimale}
$\log(x)=\frac{\ln (x)}{\ln (10)}$

Cette fonction a les mêmes propriétés algébriques que $\ln$.
\end{bclogo}







\definecolor{puiss}{RGB}{222,210,238}
\presetkeys{bclogo}{arrondi=0.1, couleur =puiss, logo=\bctakecare}{}


\section{Fonctions puissances et croissances comparées}
\subsection{Fonctions puissances $x^n$ et $\frac{1}{x^n}$}
\begin{bclogo}{La fonction $x^n$}
$\ast$ Si $n$ est pair, pour tout réel $x$, $f_n(-x)=f_n(x)$ donc $f_n$ est paire.

$\ast$ Si $n$ est impaire, $f_n$ est impaire.

$\ast$ $f'(x)=n.x^{n-1}$

$\ast$ $f_n$ est croissante sur $\mathbb{R}$.
\end{bclogo}

\newpage

\begin{bclogo}{La fonction $\frac{1}{x^n}$}
$\ast$ $g_n$ est définie sur $\mathbb{R}^*$.

$\ast$ Si $n$ est pair, $g_n(-x)=g_n(x)$ donc $g_n$ est impaire.

$\ast$ $g'_n(x)=\frac{-n}{n+1}$

$\ast$ $g_n$ est décroissante sur $]0;+\infty [$
\end{bclogo}

\subsection{Fonctions racine $n^{\text{ième}}$}
\begin{bclogo}{Définitions}
La fonction racine $n^{\text{ième}}$ est définie sur $[0;+\infty[$ par \[x\mapsto x^{\frac{1}{n}}=\sqrt[n]{x}\].

La fonction racine $n^{\text{ième}}$ est dérivable sur $]0;+\infty[$ et sa dérivée est : \[x\mapsto \frac{1}{n}x^{\frac{1}{n}-1}\]

La fonction racine $n^{\text{ième}}$ est continue sur $]0;+\infty[$. 
\[\lim\limits_{x \to 0} x^{\frac{1}{n}}= \lim\limits_{x \to 0} e^{\frac{1}{n}\ln x}\]
Or 
\[\lim\limits_{x \to 0^+} \frac{1}{n} \ln x=-\infty \text{ et } \lim\limits_{x \to -\infty} e^X=0\]

Donc
\[\lim\limits_{x \to 0} x^{\frac{1}{n}}=0\]

La fonction racine $n^{\text{ième}}$ est croissante et continue sur $[0;+\infty[$.
\end{bclogo}

\subsection{Croissances comparées}
\begin{bclogo}{}
\[\forall n\geqslant 1, \lim\limits_{x \to +\infty} \frac{\ln x}{x^n}=0 \text{ et } \lim\limits_{x \to +\infty} \frac{e^x}{x^n}=+\infty\]
\end{bclogo}

\medskip

\begin{bclogo}{}
\[\forall n\geqslant 1, \lim\limits_{x \to 0} x^n \ln x=0 \text{ et } \lim\limits_{x \to -\infty} x^n e^x=0\]
\end{bclogo}

\subsection{Fonctions exponentielles de base}
\begin{bclogo}{Définition}
\[f_a(x)=a^x=e^{x\ln a}\]
Si $a=1$, $f(a)$ est la fonction constante dégale à$1$.

Si $a=e$, $f_e$ est la fonction $\exp$
\end{bclogo}

\medskip

\begin{bclogo}{Dérivabilité}
$f_a$ est dérivable sur $\mathbb{R}$ : \[f'_a(x)=\ln a\times e^{x.\ln a}=\ln a \times a^x\]

Si $0< a< 1$ alors, $\ln a< 0$ donc $f'_a <0$. $f_a$ est décroissante sur $\mathbb{R}$.

Si $a< 1$ alors, $\ln a> 0$ donc $f'_a> 0$. $f_a$ est croissante sur $\mathbb{R}$.
\end{bclogo}

\medskip

\begin{bclogo}{Limites}
$\ast$ Si $0<a<1$

$\Rightarrow$ \[\lim\limits_{x \to +\infty} x \ln a=-\infty\] 
\[\lim\limits_{X \to -\infty} e^X=0 \text{ donc } \lim\limits_{x \to +\infty} a^x=0\]

$\Rightarrow$ \[\lim\limits_{x \to -\infty} x \ln a=+\infty\]
\[\lim\limits_{X \to +\infty} e^X=+\infty \text{ donc } \lim\limits_{x \to -\infty} a^x=+\infty\]

$\ast$ Si $a>1$

$\Rightarrow$ \[\lim\limits_{x \to +\infty} x \ln a=+\infty\]
\[\lim\limits_{X \to +\infty} e^X=+\infty \text{ donc } \lim\limits_{x \to +\infty} a^x=+\infty\]

$\Rightarrow$ \[\lim\limits_{x \to -\infty} x \ln a=-\infty\]
\[\lim\limits_{X \to -\infty} e^X=0 \text{ donc } \lim\limits_{x \to -\infty} a^x=0\]
\end{bclogo}
\definecolor{prod}{RGB}{204,197,220}
\presetkeys{bclogo}{arrondi=0.1, couleur =prod, logo=\bcdodecaedre}{}


\section{Les produits scalaires}
\subsection{Produits scalaires dans le plan}
\begin{bclogo}{Propriétés}
$\ast$ Si $\vect{u}$ et $\vect{v}$ sont non nuls, alors, \boiteo{\vect{u}\cdot \vect{v}=||\vect{u}||\times ||\vect{v}||\times \cos(\vect{u}\cdot\vect{v})}

$\ast$ $\vect{u}\cdot \vect{v}=xx'+yy'$

$\ast$ $\vect{u}\cdot \vect{v}=\frac{1}{2}\left( ||\vect{u}+\vect{v}||^2-||\vect{u}||^2-||\vect{v}||^2\right) $

$\ast$ $\left( k\vect{u}\right) \cdot \vect{v}=k\left( \vect{u}\cdot \vect{v}\right) =\vect{u}\left( k\vect{v}\right)$ 

$\ast$ $\vect{u}\cdot \left( \vect{v}+\vect{w}\right)=\vect{u}\cdot \vect{v}+\vect{u}\cdot \vect{w}$ 

$\ast$ Dire que deux vecteurs sont orthogonaux signifie que l'un des deux est nul ou que les segments sont perpendiculaires. $\vect{u}\cdot \vect{v}=0$
\end{bclogo}

\medskip

\begin{bclogo}{Théorème}
$\ast$ \textbf{Théorème d'Al Kashi} : dans un triangle ABC, (longueurs a,b,c), \boiteo{a^2=b^2+c^2-2ab\times \cos(\widehat{A})}

$\ast$ \textbf{Théorème de la médiane} : $I$ milieu de $\left[AB\right]$, \boiteo{MA^2+MB^2=\vect{MA}^2+\vect{MB}^2 =2MI^2+\frac{1}{2}AB^2}
\end{bclogo} 

\medskip

\begin{bclogo}{Définitions}
$\ast$ Une droite de vecteur normal $\vect{n}(a;b)$ a une équation de la forme $ax+by+c=0$ où $c$, est un réel. Et réciproquement, l'ensemble des points du plan dont les coordonnées vérifient l'équation $ax+by+c=0$ est une droite de vecteur normal $\vect{n}(a;b)$.

$\ast$ La distance du point $A$ à la droite $d$ est égale à \boiteo{AH=\frac{|ax_A+by_A+c|}{\sqrt{a^2+b^2}}}
 où $H$ est le projeté orthogonal de $A$ sur $d$.

$\ast$ $\mathcal{C}$ est le cercle de centre $I(\alpha ;\beta )$ et de rayon $\mathbb{R}$. Une équation de $\mathcal{C}$ est \boiteo{(x-\alpha)^2+(y-\beta)^2=R^2} 
avec $R^2=IM^2$.

$\ast$ Le cercle $\mathcal{C}$ de diamètre $\left[ AB\right]$  est l'ensemble des points $M$ tels que : $\vect{MA}\cdot \vect{MB}=0$.
\end{bclogo} 

\medskip


\subsection{Produits scalaires dans l'espace}
\begin{bclogo}{Définitions}
$P$, plan; $(O;\vec{\imath};\vec{\jmath})$, un repère orthonormal de ce plan; $\vec{k}$, vecteur normal à $P$. $||\vec{\imath}||=||\vec{\jmath}||=||\vec{k}||=1$ et $\vec{\imath}\vec{\jmath}=\vec{\imath}\vec{k}=\vec{\jmath}\vec{k}=0$. $(\vec{\imath},\vec{\jmath},\vec{k})$, base orthonormale de l'espace.
\end{bclogo}

\medskip

\begin{bclogo}{Produit scalaire dans une base orthonormale}
$\ast$ $\vect{u} (x;y;z)$ et $\vect{v}(x';y';z')$

$\ast$ $\vect{u}\cdot \vect{v}=xx'+yy'+zz'$

$\ast$ $||\vect{u}||=\sqrt{x^2+y^2+z^2}$

$\ast$ $A(x_A;y_A;z_A)$ et $B(x_B;y_B;z_B)$

\boiteo{AB=\sqrt{(x_B-x_A)^2+(y_B-y_A)^2+(z_B-z_A)^2}}
\end{bclogo}


\medskip

\begin{bclogo}{Equation cartésienne d'un plan dans un repère orthonormal}
Dans un repère orthonormal :

$\ast$ Un plan de vecteur normal $\vect{n}(a;b;c)$ a une équation de la forme \boiteo{ax+by+cz+d=0}

$\ast$ L'ensemble des point de l'espace  dont les coordonnées vérifient l'équation $ax+by+cz+d=0$ est un plan de vecteur normal $\vect{n}(a;b;c)$ 
\end{bclogo}

\medskip

\begin{bclogo}{Distance d'un point à un plan}
$P$, plan d'équation $ax+by+cz+d=0$ et $A(x_A;y_A;z_A)$, point de l'espace. Distance de $A$ à $P$ : 

\boiteo{AH=\frac{|ax_A+by_A+cy_A+d|}{\sqrt{a^2+b^2+c^2}}}
 où $H$ est le projeté orthogonal de $A$ sur $P$.
\end{bclogo}

\medskip

\begin{bclogo}{Demi-espace}
$\ast$ L'ensemble des points $M(x;y;z)$ de l'espace tels que $ax+by+cz+d\geqslant 0$ (respectivement $>0$) est un demi-espace fermé (respectivement ouvert) de frontière le plan $P$.

$\ast$ L'ensemble des points $M(x;y;z)$ de l'espace tels que $ax+by+cz+d\leqslant 0$ (respectivement $<0$) est un demi-espace fermé (respectivement ouvert) de frontière le plan $P$.
\end{bclogo}


\medskip

\begin{bclogo}{Plan médiateur d'un segment}
L'ensemble des points de l'espace équidistants de $A$ et de $B$ est un plan passant par le milieu de $\left[ AB\right] $ et perpendiculaire à la droite $(AB)$ : plan médiateur de $\left[ AB\right] $.
\end{bclogo}

\newpage

\begin{bclogo}{Plans parallèles et perpendiculaires}
$P: ax+by+cz+d=0$ et $Q: a'x+b'y+c'z+d'=0$

Les plans $P$ et $Q$ sont parallèles ssi les triplets $(a;b;c)$ et $(a';b';c')$ sont proportionnels (vecteurs normaux colinéaires).

Les plans $P$ et $Q$ sont perpendiculaires ssi $aa'+bb'+cc'=0$ (vecteurs normaux orthogonaux).
\end{bclogo}

\medskip

\begin{bclogo}{Sphère et produit scalaire}
La sphère de centre $I(\alpha ; \beta ; \gamma)$ et de rayon $R$ a pour équation : \boiteo{(x-\alpha )^2+(y-\beta )^2+(z-\gamma )^2=R^2}

La sphère de diamètre $\left[ AB\right] $ est l'ensemble des points $M$ de l'espace tels que $\vect{MA}\cdot \vect{MB}=0$.
\end{bclogo}






\presetkeys{bclogo}{arrondi=0.1, couleur =prod, logo=\bccube}{}


\section{Représentation analytique d'une droite de l'espace}
\begin{bclogo}{Représentation paramétrique d'une droite}
$\mathcal{D}$, une droite de l'espace passant par $A(a_A;y_A;z_A)$ et de vecteur directeur $\vect{u}(a;b;c)$.

$M(x;y;z)$ : $\vect{AM}$ et $\vect{u}$ sont colinéaires. $\vect{AM}=t\vect{u}$.

\[\left\lbrace\begin{array}{l} x-x_A=a.t\\ y-y_A=b.t\\ z-z_A=c.t \end{array}\right.\;\Longleftrightarrow\;(S)\left\lbrace\begin{array}{l} x=x_A+a.t\\ y=y_A+b.t\\ z=z_A+c.t \end{array}\right.\]

Le système $(S)$ est une représentation paramétrique de la droite $\mathcal{D}$.
\end{bclogo}

\medskip

\begin{bclogo}{Système de deux équations cartésiennes représentant une droite}
Soit $\mathcal{P}$, le plan d'équation $ax+by+cz+d=0$, de vecteur normal $\vect{n} (a;b;c)$.

Soit $\mathcal{Q}$, le plan d'équation $a'x+b'y+c'z+d'=0$, de vecteur normal $\vect{n} (a';b';c')$.

Supposons que $\vect{n'}$ et $\vect{n}$ ne sont pas colinéaires. Donc $\mathcal{P}$ et $\mathcal{Q}$ sont sécants; soit $\mathcal{D}$ la droite d'intersection des plans $\mathcal{P}$ et $\mathcal{Q}$.

La droite $\mathcal{D}$ est représentée par le système $\left\lbrace\begin{array}{l} ax+by+cz+d=0\\ a'z+b'y+c'z+d'=0 \end{array}\right.$

Réciproquement, si les vecteurs $\vect{n}(a;b;c)$ et $\vect{n'}(a';b';c')$ ne sont pas colinéaires, l'ensemble des points $M(x;y;z)$  tels que $\left\lbrace\begin{array}{l} ax+by+cz+d=0\\ a'z+b'y+c'z+d'=0 \end{array}\right.$ est une droite. C'est l'intersection des plans d'équations respectives $ax+by+cz+d=0$ et $a'x+b'y+c'z+d'=0$.
\end{bclogo}
\definecolor{comp}{RGB}{234,185,173}
\presetkeys{bclogo}{arrondi=0.1, couleur =comp, logo=\bctrefle}{}

\section{Les nombres complexes}
\subsection{Introduction aux nombres complexes}
\begin{bclogo}{Définitions}


Un nombre complexe est un nombre de la forme $x+\I y$, où $x$ et $y$ désignent des réels et $\I $ un nombre imaginaire vérifiant $\I ^2=-1$. L'ensemble des complexes est noté $\mathbb{C}$.

Soit, un point $M$ de coordonées $(x;y)$, le nombre complexe $x+\I .y$ est l'affixe du point $M$ ou du vecteur $\vect{OM}$. \[z_M=x+\I .y \text{ ou } z_{OM}=x+\I .y\]

Le point $M(x;y)$ est l'image du nombre $x+\I .y$.

Le plan de repère orthonormal direct $(O,\vect{u},\vect{v})$ est appelé plan complexe.
\end{bclogo}

\medskip

\begin{bclogo}{Forme algébrique}
$\ast$ Tout nombre $z$ admet une unique écriture de la forme $x+\I .y$ (forme algébrique) avec :

$x$, partie réelle de $z$ notée $\Re (z)$.

$y$, partie imaginaire de $z$ notée $\Im (z)$.

$\ast$ Si $z\in \mathbb{R}$ alors, $\Im (z)=0$.

$\ast$ Si $z$ est un imaginaire pur, $\Re (z)=0$.

$\ast$ Si $z=z'$, $\Re (z)=\Re (z')$, $\Im (z)=\Im (z')$.

$\ast$ Si $z=0$, $\Re (z)=\Im (z)=0$.
 
\end{bclogo}

\medskip

\begin{bclogo}{Conjugué}
Le conjugué de $z$ est le nombre $\overline{z}=x-\I .y$.

$\ast$ $\overline{z}=z$

$\ast$ $z+\overline{z}=2\times \Re (z)$

$\ast$ $z-\overline{z}=\I \times 2\Im (z)$

$\ast$ $z.\overline{z}=x^2+y^2$

$\ast$ Si, $z\in \mathbb{R}$ alors $z=\overline{z}$

$\ast$ Si, $z$ est imaginaire pur, alors, $\overline{z}=-z$
\end{bclogo}

\subsection{Calculs avec les nombres complexes}

\begin{bclogo}{Sommes et produits}
\[z+z'=(x+x')+\I (y+y')\]

\[kz=kz+\I ky\]

\[zz'=xx'-yy'+\I (xy'+x'y)\]

\[-1z=-x-\I y=-z\]

\[z-z'=z+(-z')\]
\end{bclogo}

\medskip

\begin{bclogo}{Inverses et quotients}
\[\frac{1}{z}=\frac{\overline{z}}{z\overline{z}}\]

\[\frac{z}{z'}=\frac{z\overline{z'}}{z'\overline{z'}}\]
Avec $z.\overline{z}=x^2+y^2$...
\end{bclogo}

\medskip

\begin{bclogo}{Opérations sur les conjugués}
$\ast$ Le conjugué d'une somme est égal à la somme des conjugués. \[\overline{z+z'}=\overline{z}+\overline{z'}\]

$\ast$ Le conjugué d'un produit est égal au produit des conjugués. \[zz'=\overline{z}.\overline{z'}\]

$\ast$ Le conjugué d'un quotient est égal au quotient des conjugués. \[\overline{\frac{z}{z'}}=\frac{\overline{z}}{\overline{z'}}\]

$\ast$ \[\overline{z^n}=\overline{z}^n\]
\end{bclogo}

\subsection{Equation du second degré à coefficients réels}

\begin{bclogo}{Théorème}
\[\Delta=b^2-4ac\]

Dans $\mathbb{C}$, l'équation $az^2+bz+c=0$ a toujours des solutions. (si $\Delta=0$, $z_1$ et $z_2$ sont confondus).

\[z_1=\frac{-b-\delta}{2a} \text{ et }  z_2=\overline{z_1}=\frac{-b+\delta}{2a}\]

avec $\delta ^2=\Delta$
\end{bclogo}

\subsection{Module et argument d'un nombre complexe}

\begin{bclogo}{Coordonnées polaires}
Les nombres polaires sont notés $(r,\alpha)$. 

$\ast$ Pour $r>0$, $r=OM$.

$\ast$ $\alpha$ est une mesure en radian de $(\vect{u},\vect{OM})$.

$\ast$ Si $(r,\alpha)$ est un couple de coordonées de $M$, alors les coordonnées cartésiennes $(x,y)$ sont : \boiteo{x=r.\cos \alpha \text{ et } y=r.\sin \alpha}

$\ast$ Réciproquement, si $M$ a pour coordonnées cartésiennes $(x,y)$ alors les coordonnées polaires $(r,\alpha)$ sont définies par : \boiteo{r=\sqrt{x^2+y^2}}

\[\cos \alpha=\frac{x}{r} \text{ et } \sin \alpha=\frac{y}{r}\]
\end{bclogo}

\medskip

\begin{bclogo}{Module d'un nombre complexe}
Le module $z$ est le nombre réel positif noté $|z|$, défini par $|z|=\sqrt{x^2+y^2}$.

Dans le plan complexe, $|z|=OM$

$\ast$ Si $z$ est un nombre réel $x$, alors $|z|$ est la valeur absolue de $x$. $z=\sqrt{x^2}$.

$\ast$ Si $|z|=0$, alors $z=0$.

$\ast$ $z.\overline{z}=x^2+y^2$ avec $z=x+\I y$, alors $z.\overline{z}=|z|^2$.
\end{bclogo}

\medskip

\begin{bclogo}{Arguments d'un nombre complexe non nul}
Dans le plan complexe $z$ a pour image un point $M$. L'argument de $z$ est noté $\arg z$ et correspond à toute mesure en radians de l'angle $(\vect{u},\vect{OM})$.

Un nombre complexe a une infinité d'arguments. Si $\theta$ est l'un d'entre eux, les réels $\theta +k2\pi$ sont des arguments de $z$. On note : $\arg (z)=\theta$ ($\mod 2\pi$ ou $[2\pi]$) ou $\arg (z)=\theta$.
\end{bclogo}

\newpage

\begin{bclogo}{Forme trigonométrique d'un nombre complexe}
\boiteo{z=r(\cos \theta +\I\sin \theta)}

avec $r>0$, $r=|z|$ et $\theta=\arg (z)$.

$\ast$ Deux nombres complexes sont égaux ssi ils ont le même module et le même argument à $2\pi$ près.

$\ast$ Si $z=\ell (\cos \theta+\I \sin \theta)$ (avec $\ell >0$), alors $|z|=\ell$ et $\arg (z)=\theta$ $(\mod 2\pi)$.
\end{bclogo}

\subsection{Propriétés du module et des arguments}
\begin{bclogo}{Propriétés}
$\ast$ $|\overline{z}|=|z|$ $\arg (\overline{z})=\arg (z)$ $[2\pi]$.

$\ast$ $|-z|=|z|$ $\arg (-z)=\arg (z) +\pi$ $[2\pi]$.

$\ast$ $\forall z\in \mathbb{R}$, $z=0$ ou $\arg (z) =0$ ou $\arg (z) =\pi$ $[2\pi]$.

$\ast$ Pour $z$ imaginaire pur, $\arg (z)=\frac{\pi}{2}$ ou $\arg (z)=\frac{-\pi}{2}$ $[2\pi]$.
\end{bclogo}

\medskip

\begin{bclogo}{Opérations}
$\ast$ \textbf{Théorème :}

Soit $z=r(\cos \alpha +\I .\sin \alpha)$ et $z'=r'(\cos \beta+\I .\sin \beta)$ avec $r$ et $r'$ supérieurs à $0$.

\begin{equation}\label{eq} zz'=rr'\left( \cos (\alpha +\beta ) +\I .\sin (\alpha +\beta )\right) \end{equation}
\begin{equation}\label{equa}\frac{z}{z'}=\frac{r}{r'}\left( \cos (\alpha -\beta) +\I .\sin (\alpha -\beta)\right) \end{equation}

$\ast$ \textbf{Démonstration ROC :}

(\ref{eq})
\[zz'= \left[ r(\cos \alpha +\I .\sin \alpha)\right] \times \left[ r'(\cos \beta+\I .\sin \beta)\right] \]
\[zz'= rr'\left( \cos \alpha +\I .\sin \alpha\right) \times \left( \cos \beta+\I .\sin \beta\right) \]
\[zz'=rr'\times \left[ \left( \cos \alpha .\cos \beta - \sin \alpha \sin \beta \right) +\I  \left( \cos \alpha .\sin \beta + \sin \alpha \cos \beta\right) \right] \]
\[zz'=rr'\left( \cos (\alpha +\beta ) +\I .\sin (\alpha +\beta )\right) \]

(\ref{equa})

\[\frac{z}{z'}=Z\]
avec $Z=\ell (\cos \theta + \sin \theta)$

\[z=z'Z\]
\[r(\cos \alpha +\I \sin \alpha) =r' \ell \left[ \cos (\beta + \theta)+\I \sin (\beta +\theta)\right] \]
\[\Longleftrightarrow\;\left\lbrace\begin{array}{l} r=r'\ell \\ \alpha =\beta +\theta [2\Pi] \end{array}\right. \Longleftrightarrow\;\left\lbrace\begin{array}{l} \ell =\frac{r}{r'} \\ \theta=\alpha -\beta [2\pi] \end{array}\right.\]
\[Z=\frac{z}{z'}=\frac{r}{r'}\left[ \cos (\alpha -\beta)+\I .\sin (\alpha -\beta)\right] \]

\end{bclogo}

\newpage

\begin{bclogo}{Conséquences}
\begin{center}
\begin{tabular}{|c|c|c|}
\hline
Produit&$|zz'|=|z|\times |z'|$&$\arg (zz')=\arg (z)+\arg (z')$ $[2\pi]$\\
\hline
Puissance&$|z^n|=|z|^n$ avec $n\in \mathbb{N}$&$\arg (z^n)=n.\arg (z)$ $[2\pi]$\\
\hline
Inverse&$|\dfrac{1}{z}|=\dfrac{1}{|z|}$&$\arg \left( \dfrac{1}{z}\right) =-\arg (z)$ $[2\pi]$\\
\hline
Quotient&$|\dfrac{z}{z'}|=\dfrac{|z|}{|z'|}$&$\arg \left( \dfrac{z}{z'}\right) =\arg (z)-\arg (z')$ $[2\pi]$\\
\hline
\end{tabular}
\end{center}
\end{bclogo}

\medskip

\begin{bclogo}{Inégalité triangulaire}
\boiteo{|z+z'|\leqslant |z|+|z'|}
\end{bclogo}

\subsection{Lien avec le plan complexe}

\begin{bclogo}{Propriétés des affixes}
$I$, milieu de $[AB]$ signifie que : \boiteo{z_I=\frac{z_A+z_B}{2}}

$G$, barycentre de $\left\lbrace (A,\alpha); (B,\beta); (C,\gamma )\right\rbrace$ signifie que :
\boiteo{z_G=\frac{\alpha z_A +\beta z_B+\gamma z_C}{\alpha +\beta +\gamma}}
\end{bclogo} 

\medskip

\begin{bclogo}{Propriétés des modules}
$A$ et $B$, deux points d'affixes $z_A$ et $z_B$.

$\ast$ $AB=|z_A-z_B|$

$\ast$ Si $A\neq B$, alors $(\vect{u},\vect{AB})=\arg (z_B-z_A) [2\pi]$
\end{bclogo}

\medskip

\begin{bclogo}{Conséquences }
$A,B,C,D$ quatre points distincts deux à deux d'affixes respectives $z_A,z_B,z_C,z_D$.
\boiteo{(\vect{AB},\vect{CD})=\arg \left( \frac{z_D -z_C}{z_B-z_A}\right) [2\pi]}

\end{bclogo}

\subsection{Notation exponentielle}

\begin{bclogo}{Définitions et propriétés}
\boiteo{e^{\I \theta} =\cos \theta +\I \sin \theta}
$\forall z \in \mathbb{C} \left\lbrace 0\right\rbrace $, de module $r$ et d'argument $\theta$, la forme exponentielle de $r$ s'écrit :\boiteo{z=r.e^{\I \theta}}

\textbf{Propriétés :}

$\ast$ $|e^\I \theta|=1$ et $\arg (e^{\I \theta} )=\theta$
\medskip

$\ast$ $e^{\I \theta}\times e^{\I \theta '}=e^{\I (\theta +\theta ')}$
\medskip

$\ast$ $\frac{e^{\I \theta}}{e^{\I \theta '}}=e^{\I (\theta-\theta ')}$
\medskip

$\ast$ $\overline{e^{\I \theta}}=e^{-\I \theta}$
\medskip

$\ast$ $(e^{\I \theta})^n=e^{\I .n.\theta}$

\end{bclogo} 

\medskip

\begin{bclogo}{Formules de Moine et d'Euler}
$\ast$ \textbf{Formule de Moine :} \boiteo{(\cos \alpha +\I .\sin \alpha )^n=\cos (n.\alpha ) +\I .\sin (n.\alpha )}
\boiteo{ (\cos \alpha +\I .\sin \alpha )^n=\cos (n.\alpha ) -\I .\sin (n.\alpha )}

$\ast$ \textbf{Formule d'Euler :} \boiteo{\cos \alpha = \frac{e^{\I .\alpha} + e^{-\I .\alpha}}{2}}
\boiteo{\sin \alpha = \frac{e^{\I .\alpha} - e^{-\I .\alpha}}{2.\I }}
\end{bclogo}

\medskip

\begin{bclogo}{Equation paramétrique d'un cercle du plan complexe}
$\mathcal{C}$, cercle de centre $\Omega$, d'affixe $\omega$ et de rayon $R$. $M$, d'affixe $z$.

$M\in \mathcal{C} \Longleftrightarrow$ il existe $\theta \in \mathbb{R}$, tel que : \boiteo{z=R.e^{\I .\theta}+\omega}

C'est l'équation paramétrique du cercle $\mathcal{C}$.
\end{bclogo}

\subsection{Nombres complexes et transformations}
\begin{bclogo}{Translation}
Soit $\vect{w}$, le vecteur d'affixe $b$. L'écriture complexe de la translation de vecteur $\vect{w}$ s'écrit : \boiteo{z'=z+b}
\end{bclogo}

\medskip

\begin{bclogo}{Homothétie}
Soit $\Omega$, d'affixe $\omega$ et $k$, réel non nul. L'écriture complexe de l'homothétie de centre $\Omega$ et de rapport $k$ est  : \boiteo{z'=k.(z-\omega)+\omega}
\end{bclogo}

\medskip

\begin{bclogo}{Rotation}
Soit $\Omega$, le point d'affixe $\omega$ et $\theta$, un réel. L'écriture complexe de la rotation de centre $\Omega$ et d'angle $\theta$ est : \boiteo{z'=e^{\I .\theta}.(z-\omega)+\omega}

\textbf{Démonstration ROC :}

Soit, $r$, la rotation de centre $\Omega$ et d'angle $\theta$.

$\ast$ $M\neq \Omega$, $M'=r.M$ \[\Longleftrightarrow \Omega .M=\Omega .M' \text{ et, } (\vect{\Omega M},\vect{\Omega M'})=\theta\]
\[\Longleftrightarrow |z-\omega|=|z'-\omega| \text{ et } \arg \left( \frac{z'-\omega}{z-\omega}\right) =\theta\]
\[\Longleftrightarrow |\frac{z'-\omega}{z-\omega}|=1 \text{ et } \arg \left( \frac{z'-\omega}{z-\omega}\right) =\theta\]

$\frac{z'-\omega}{z-\omega}$ est le nombre complexe de module $1$ et d'argument $\theta$.

Donc, \[\frac{z'-\omega}{z-\omega}=e^{\I .\theta}\]
\[z'-\omega = e^{\I .\theta}.(z-\omega )\]
\[z'=e^{\I .\theta}.(z-\omega )+\omega\]

$\ast$ $M=\Omega$ $\Longleftrightarrow$ $z=\omega$ donc $z'=\omega$.
\end{bclogo}

\definecolor{int}{RGB}{244,222,180}
\presetkeys{bclogo}{arrondi=0.1, couleur =int, logo=\bcplume}{}

\section{Intégration}
\subsection{Intégration des fonctions}
\begin{bclogo}{Intégration d'une fonction positive}
$\ast$ Soit une fonction $f$ continue et positive sur $[a;b]$. Le réel noté \[\int_{a}^{b} f(x)\,\mathrm{d}x\] est l'aire, en unité d'aire, du domaine $\mathbb{D}$ délimité par $\mathcal{C}_f$, l'axe des abscisses, les droites d'équations $x=a$ et $x=b$. $a$ et $b$ sont les bornes de l'intégrale.

$\ast$ \textbf{Valeur moyenne : }la valeur moyenne de $f$ sur $[a;b]$ est : \boiteo{\mu =\frac{1}{b-a}\times \int_{a}^{b} f(x)\,\mathrm{d}x}

\end{bclogo}

\medskip

\begin{bclogo}{Intégration d'une fonction de signe quelconque}
$\ast$ Soit une fonction $f$ continue et négative sur $[a;b]$. \[\int_{a}^{b} f(x)\,\mathrm{d}x=-\text{aire} \mathcal{D}\]
Où $\mathcal{D}$ est le domaine délimité par $\mathcal{C}_f$, l'axe des abscisses et les droites d'quation $x=a$ et $x=b$.


$\ast$ La valeur moyenne se calcul de la même façon.
\end{bclogo}

\subsection{Propriétés de l'intégrale}

\begin{bclogo}{Relation de Chasles}
Soit $f$, fonction continue sur $[a;b]$ et $c\in [a;b]$. \[\int_{a}^{c}f(x)\,\mathrm{d}x+\int_{c}^{b}f(x)\,\mathrm{d}x=\int_{a}^{b}f(x)\,\mathrm{d}x\]

Cas particulier : \[\int_{a}^{c}f(x)\,\mathrm{d}x+\int_{c}^{a}f(x)\,\mathrm{d}x=\int_{a}^{a}f(x)\,\mathrm{d}x=0\]
\[\int_{a}^{c}f(x)\,\mathrm{d}x=-\int_{c}^{a}f(x)\,\mathrm{d}x\]
\end{bclogo}


\begin{bclogo}{Linéarité}
\[\int_{a}^{b}f(x)\,\mathrm{d}x+\int_{a}^{b}g(x)\,\mathrm{d}x=\int_{a}^{b}f(x)+g(x)\,\mathrm{d}x\]
\[\int_{a}^{b}\alpha f(x)\,\mathrm{d}x=\alpha \int_{c}^{b}f(x)\,\mathrm{d}x\]
\end{bclogo}

\medskip

\begin{bclogo}{Positivité}
Soit $f$, fonction continue sur $[a;b]$ avec $a\leqslant b$. Si $f$ est positive sur $[a;b]$ alors, $\int_{a}^{b}f(x)\,\mathrm{d}x\geqslant 0$

\begin{center}
\begin{tabular}{|c|c|c|}
\hline
Si $f$ est positive sur $I$&$a\leqslant b$, $\int_{a}^{b}f(x)\,\mathrm{d}x\geqslant 0$&$a\geqslant b$, $\int_{a}^{b}f(x)\,\mathrm{d}x\leqslant 0$\\
\hline
Si $f$ est négative sur $I$&$a\leqslant b$, $\int_{a}^{b}f(x)\,\mathrm{d}x\leqslant 0$&$a\geqslant b$, $\int_{a}^{b}f(x)\,\mathrm{d}x\geqslant 0$\\
\hline
\end{tabular}
\end{center}
\end{bclogo}

\medskip

\begin{bclogo}{Conservation de l'ordre}
Si $f\leqslant g$ sur $[a;b]$, alors $\int_{a}^{b}f(x)\,\mathrm{d}x\leqslant \int_{a}^{b} g(x)\,\mathrm{d}x$
\end{bclogo}

\medskip

\begin{bclogo}{Inégalité de la moyenne}
S'il existe deux réels $m$ et $M$ tels que $m\leqslant f\leqslant M$ sur $I$ et si $a\leqslant b$, alors : \[m(a-b)\leqslant \int_{a}^{b} f(x)\,\mathrm{d}x\leqslant M(b-a)\]

S'il existe un réel $M$ tel que $|f|\leqslant M$ alors : \[\left| \int_{a}^{b}f(x)\,\mathrm{d}x\right| \leqslant M|a-b|\]
\end{bclogo}

\subsection{Primitive}

\begin{bclogo}{Définition}
Soit $f$ une fonction définie sur un intervalle $I$ de $\mathbb{R}$.
On appelle primitive de $f$ sur $I$ toute fonction $F$ dérivable sur $I$ telle que, pour tout $x$ de $I$, $F'(x) = f(x)$.
\end{bclogo}
\medskip

\begin{bclogo}{Quelques primitives importantes}

\begin{center}
\begin{tabular}{|c|c|}
\hline
Fonction&Une primitive\\
\hline
$f(x)=a$&$F(x)=a.x$\\
\hline
$f(x)=x^n$&$F(x)=\dfrac{1}{n+1}.x^{n+1}$\\
\hline
$f(x)=\dfrac{1}{\sqrt{x}}$&$F(x)=2\sqrt{x}$\\
\hline
$f(x)=\cos x$&$F(x)=\sin x$\\
\hline
$f(x)=\sin x$&$F(x)=-\cos x$\\
\hline
$f(x)=1+\tan ^2 x=\dfrac{1}{\cos ^2x}$&$F(x)=\tan x$\\
\hline
$f(x)=\dfrac{1}{x}$&$F(x)=\ln x$\\
\hline
$f(x)=\exp (x)$&$F(x)=\exp (x)$\\
\hline
\end{tabular}
\end{center}
\end{bclogo}

\medskip

\begin{bclogo}{Primitives de fonctions "composées"}
Soient $u$ et $v$, deux fonctions admettant pour primitives respectives $U$ et $V$ sur un intervalle $I$ et $g$, une fonction admettant une primitive $G$ sur l'intervalle $J$ contenant $u(I)$.

\begin{center}
\begin{tabular}{|c|c|}
\hline
$f=au+bv$&$F=aU+bV$\\
\hline
$f=u'\times g\circ u$&$F=G\circ u$\\
\hline
$f=u'.u^n$&$F=\dfrac{1}{n+1}.u^{n+1}$\\
\hline
$f=\dfrac{u'}{\sqrt{u}}$&$F=2\sqrt{u}$\\
\hline
$f=u'.\cos (u)$&$F=\sin u$\\
\hline
$f=u' \sin (u)$&$F=-\cos u$\\
\hline
$f=\dfrac{u'}{u}$&$F=\ln u$\\
\hline
$f=u'.e^{u}$&$F=e^{u}$\\
\hline
\end{tabular} 
\end{center}
\end{bclogo}

\medskip

\begin{bclogo}{Existence des primitives}
\textbf{Théorème :} Si $f$ est une fonction continue sur un intervalle $I$ alors $f$ admet des primitives sur $I$.

Si $F$ est une primitive de $f$ sur $I$, alors les primitives de $f$ sur $I$ sont les fonctions de la forme $F(x)+k$. Pour tout couple $(x_0,y_0)$, il existe une unique primitive $F_0$ de $f$ sur $I$ telle que $F_0(x_0)=y_0$.
\end{bclogo}

\subsection{Intégrale et primitive}

\begin{bclogo}{Théorème}
Soit $f$, une fonction continue sur un intervalle $I$ et $a$, un réel quelconque de $I$. La fonction $\phi$ définie sur $I$ par $\phi (x)=\int_{a}^{x} f(t)d(t)$ est l'unique primitive de $f$ qui s'annule en $a$.
\end{bclogo}

\medskip

\begin{bclogo}{Remarques}
$\phi$ est dérivable sur $I$de dérivée $f$.

Les solutions de l'équation différentielle $y'=f(t)$ sont les fonctions : $\phi (x)=\int_{x}^{a} f(t)\mathrm{d}t+k$.
\end{bclogo}

\medskip

\begin{bclogo}{Calcul d'une intégrale à l'aide des primitives}
\[\int_{a}^{b} f(x)\,\mathrm{d}x=F(b)-F(a)\]
\end{bclogo}

\medskip

\subsection{Intégration par parties}
\begin{bclogo}{Théorème}
\boiteo{\int_{a}^{b} u(x).v'(x)\,\mathrm{d}x=\left[ u(x).v(x)\right]_{a}^{b} - \int_{a}^{b} u'(x).v(x)\,\mathrm{d}x}
\end{bclogo}

\medskip

\begin{bclogo}{Démonstration ROC}
$(u.v)'=u'v+uv'$ donc, $uv'=(uv)'-u'v$. 

$u$, $v$, $u'$, $v'$ sont continues, donc $uv$, $u'v$ et $uv'$ sont continues aussi.

Par linéarité de l'intégrale :
\[\int_{a}^{b} u(x).v'(x)\,\mathrm{d}x= \int_{a}^{b}(uv)'(x)\,\mathrm{d}x-\int_{a}^{b}u'(x)v(x)\,\mathrm{d}x\]

\[\left[ u(x) v(x)\right]_{a}^{b}-\int_{a}^{b} u'(x)v(x)\,\mathrm{d}x\]
\end{bclogo} 

\definecolor{proba}{RGB}{227,244,180}
\presetkeys{bclogo}{arrondi=0.1, couleur =proba, logo=\bcquestion}{}

\section{Les probabilités}
\subsection{Introduction aux probabilités}
\begin{bclogo}{Définitions}
$\ast$ $\Omega = \left\lbrace \omega _{1}, \omega _{2},..., \omega _{n}\right\rbrace $ est l'ensemble des résultats d'une expérience aléatoire. On l'appelle univers. Un événement est une partie de $\Omega$. Lorsque $\omega$ appartient à l'événement $A$, on dit qu'il réalise $A$. $\oslash$ est un évenement impossible. $\Omega$ est l'évènement certain. Un événement élémentaire est constitué d'un seul résultat.
\end{bclogo}

\medskip

\begin{bclogo}{Probabilité d'un événement}
La probabilité de l'événement $A$ est notée $p(A)$. 

$0\leqslant p(A)\leqslant1$

$p(\Omega )=1$

$p(\oslash )=0$.
\end{bclogo}

\medskip

\begin{bclogo}{Définitions}
$\ast$ L'espérance de la loi de probabilité est : \boiteo{\mu =\sum _{i=1}^{n} p_i.\omega _i}

où, pour tout $i\in\{1,2\ldots,n\}$, $p_i$ est la probabilité de l'évènement $\omega_i$.

$\ast$ La variance de la loi de probabilité est : \boiteo{V= \sum _{i=1}^n p_i (\omega _i -\mu )^2 =\left( \sum _{i=1}^n p_i \omega _i ^2\right) -\mu ^2}

$\ast$ L'écart type de la loi de probabilité est : \boiteo{\sigma =\sqrt{V}}
\end{bclogo}

\medskip



\medskip

\begin{bclogo}{Cas de l'équiprobabilité}
Lorsque la loi de probabilité associe à tous les résultats d'une expérience la même probabilité, on parle de loi équirépartie et la situation est dite d'équiprobabilité.
\boiteo{p(A) =\frac{\text{nombre de résultats de A}}{\text{nombre de résultats de } \omega}}
\end{bclogo}

\subsection{Calculs de probabilités}
\begin{bclogo}{Probabilité de la réunion, de l'intersection d'événements}
$A\cap B$ est l'événement formé des résultats qui réalisent à la fois $A$ et $B$.

$A\cup B$ est l'événement formé des résultats qui réalisent au moins un des événements $A$ ou $B$.

\boiteo{p(A\cup B)=p(A)+p(B)-p(A\cap B)}
\end{bclogo}

\medskip

\begin{bclogo}{Probabilité de l'événement contraire}
L'événement contraire de $A$ est l'événement formé des résultats qui ne réalisent pas $A$. On le note $\overline{A}$. 

$p(\overline{A})=1-p(A)$
\end{bclogo}

\subsection{Variable aléatoire}
\begin{bclogo}{Loi de probabilité d'une variable aléatoire}
Une loi de probabilité est définie sur $\Omega$. 

$\Omega '= {x_1 ,x_2,...x_n}$ est l'ensemble des valeurs prises par une variable aléatoire $X$.

Loi de probabilité de la variable aléatoire $X$ sur $\Omega '$ associe à chaque valeur $x_i$ la probabilité de l'événement ($X=x_i$).
\end{bclogo}

\medskip

\begin{bclogo}{Espérence, variance, écart-type d'une variable aléatoire}
Espérence : \boiteo{E(X)=\sum _{i=1}^m x_i p_i}

Variance : \boiteo{V(X)=\sum _{i=1}^m p_i \left[ x_i -E(X)\right] ^2 =\sum _{i=1}^m p_i x_i ^2 -\left[ E(X)\right] ^2}

Ecart-type : \boiteo{\sigma (X)=\sqrt{V(X)}}
\end{bclogo}

\section{Dénombrement et lois de probabilité}
\subsection{Dénombrement}

\begin{bclogo}{Tirages successifs}
$\ast$ Avec remise :

On tire un jeton d'une urne, on note son numéro puis on le remet dans l'urne. On effectue $p$ tirages ($p\geqslant1$) dits successifs avec remise. Le nombre de $n$ listes ordonnées de $p$ éléments de l'urne est \[n^p\]

$\ast$ Sans remise : 

On tire un jeton de l'urne contenant $n$ jetons, on note le numéro mais on ne le remet pas dans l'urne. On effectue $p$ tirage. Le nombre d'arrangements de $p$ éléments de l'urne est : \[n\times (n-1)\times ...\times (n-p+1)\]

$\ast$ Cas particulier : les permutations. 

Lorsque $p=n$, tous les jetons de l'urne ont été tirés. Le nombre d'arrangements de l'urne est : \[n\times (n-1)\times ...\times 1=n!\]
\end{bclogo}

\medskip

\begin{bclogo}{Tirages simultannés}
On tire simultanément $p$ jetons de l'urne. On obtient un ensemble de $p$ éléments purs parmi $n$ que l'on appelle combinaison. Le nombre de combinaisons de $p$ éléments parmi $n$ est noté $\binom{n}{p}$, on le lit $p$ parmi $n$ et il est égal à : \[\binom{n}{p} =\frac{n\times (n-1)\times (n-2)\times \ldots\times (n-p+1)}{p!}=\frac{n!}{p!(n-p)!}\]

\end{bclogo}

\medskip

\begin{bclogo}{Coefficients binômiaux}
$\ast$ Pour tout entier $n$ non nul, $\binom{n}{0}=1$, $\binom{n}{1}=n$, $\binom{n}{n}=1$.

$\ast$ Pour tout entier $p$ avec $0\leqslant p\leqslant n$ on a \[\binom{n}{p}=\binom{n}{n-p}\]

$\ast$ Pour $1\leqslant p\leqslant n-1$, la \textbf{Relation de Pascal} \[\binom{n}{p}=\binom{n-1}{p-1}+\binom{n-1}{p}\]
\end{bclogo}

\medskip

\begin{bclogo}{Triangle de Pascal}
\begin{center}
\begin{tabular}{|c|c|c|c|c|c|c|c|c|}
\hline
&0&1&2&3&4&5&6&7\\
\hline
0&1&&&&&&&\\
\hline
1&1&1&&&&&&\\
\hline
2&1&2&1&&&&&\\
\hline
3&1&3&3&1&&&&\\
\hline
4&1&4&6&4&1&&&\\
\hline
5&1&5&10&10&5&1&&\\
\hline
6&1&6&15&20&15&6&1&\\
\hline
7&1&7&21&35&35&21&7&1\\
\hline
\end{tabular}
\end{center}

\begin{center}
\begin{tabular}{|c|c|}
\hline
$\binom{n-1}{p-1}$ $+$&$\binom{n-1}{p}$\\
\hline
&$=$ $\binom{n}{p}$\\
\hline
\end{tabular}
\end{center}
\end{bclogo}

\medskip

\begin{bclogo}{Formule du binôme de Newton}
\[(a+b)^n=\sum _{p=0} ^{n} \binom{n}{p} a^p b^{n-p}\]

\[(a+b)^n =\binom{n}{0} a^0 b^n +\binom{n}{1} a^1 b^{n-1} + \binom{n}{2} a^2 b^{n-2}+\cdots+\binom{n}{n-1} a^{n-1} b^{1}+ \binom{n}{n} a^n b^{0}\]

\[(a+b)^n=b^n+n.a.b^{n-1}+\binom{n}{2} a^2 b^{n-2}+\cdots+n.a^{n-1}b+a^n\]
\end{bclogo}

\subsection{Exemples de lois discrètes}
\begin{bclogo}{Loi de Bernouilli}
$\ast$ L'épreuve de Bernouilli :

C'est une expérience aléatoire qui ne comporte que 2 issus $S$ et $\overline{S}$.

$S$ correspond au succès : $p=p(S)$.

$\overline{S} =E$ correspond à l'échec : $q=1-p=p(\overline{S})$.

$\ast$ Loi de Bernouilli : 

Soit une épreuve de Bernouilli d'issues $S$ (de probabilité $p$) et $E$ (de probabilité $q=1-p$) et $X$, la variable aléatoire qui prend la valeur $1$ quand $S$ est réalisée et $0$ sinon. Par définition, cette variable aléatoire suit la loi de Bernouilli de paramètre $p$. 

On a : $E(X)=p$ et $V(X)=pq$
\end{bclogo}

\medskip

\begin{bclogo}{La loi binomiale}
$\ast$ Un schéma de Bernouilli est la répétition de $n$ épreuves de Bernouilli identiques et indépendantes.

La variable aléatoire $X$ à valeurs dans $\{0;1;2;\ldots;n\}$ associe à chaque liste le nombre de succès.

Par définition, $X$ suit la loi binomiale de paramètres $n$ et $p$.

On note : $\mathcal{B} (n;p)$ et $p=p(S)$.

$\ast$ Caractéristiques : 

Soit $X$, une variable aléatoire qui suit la loi binomiale $\mathcal{B}(n,p)$.

Pour $k\in \{0;1;\ldots;n\}$, \boiteo{p(X=k)=\binom{n}{k} p^k\times (1-p)^{n-k}}

$E(X)=np$ et $V(X)=np(1-p)$
\end{bclogo}



\subsection{Lois de probabilité continue}
\begin{bclogo}{Quand l'intervalle est un univers}
Une expérience prend ses valeurs dans un intervalle et peut atteindre n'importe quel nombre de cet intervalle.
\end{bclogo}

\medskip

\begin{bclogo}{Densité}
On appelle densité de probabilité sur l'intervalle $I$, toute fonction $f$ définie sur $I$ et vérifiant :

$f$ est continue et positive sur $I$

On définit la loi de probabilité $P$ de densité $f$ sur $I$ associant à tout intervalle $[a;b]$ de $I$ :
\[P([a;b])=\int _a ^b f(x)dx\]
\end{bclogo}

\medskip

\begin{bclogo}{La loi uniforme}
On appelle loi uniforme sur $I=[a;b]$, la loi de probabilité continue sur $I$ dont la fonction $f$ de densité est constante égale à $\frac{1}{b-a}$
\end{bclogo}

\medskip

\begin{bclogo}{La loi exponentielle}
On appelle loi exponentielle de paramètre $\lambda$, la loi continue admettant pour densité la fonction définie sur $ \mathbb{R}^+$ par : \[f(x)=\lambda e^{-\lambda x}\]
avec $\lambda >0$.

Pour tout intervalle $[\alpha ;\beta]$ de $\mathbb{R}^+$ : \[p([\alpha ,\beta])=\int _{\alpha} ^{\beta} \lambda e^{-\lambda x}dx\]
\[p([\alpha ,\beta ])=\left[ -e^{-\lambda x}\right] _{\alpha } ^{\beta }\]
\boiteo{p([\alpha ,\beta ])=-e^{-\lambda \beta }+e^{-\lambda \alpha }}
\end{bclogo}

\section{Probabilités conditionnelles}

\subsection{Les probabilités conditionnelles}
\begin{bclogo}{Définition}
$A$ et $B$ sont deux événements d'une même expérience aléatoire, avec $p(A)\neq 0$. La probabilité que l'événement $B$ se réalise est : \[p_A(B)=\frac{p(A\cap B)}{p(A)}\]
\end{bclogo}

\medskip

\begin{bclogo}{Probabilité d'une intersection}
\[p(A\cap B)=p_A(B)\times p(A)\]
ou
\[p(A\cap B)=p_B(A)\times p(B)\]
\end{bclogo}

\medskip

\begin{bclogo}{Formule des probabilités totales}
Les événements $\Omega_1, \Omega_2,\ldots\Omega_n$ forment une partition de l'univers $\Omega$ quand :

les $\Omega_i$ sont deux à deux disjoints

la réunion des $\Omega_i$ est l'univers $\Omega$

Formule des probabilités totales :

\[p(A)=p(A\cap \Omega_1)+p(A\cap \Omega_2)+\ldots+p(A\cap \Omega_n)\]
\[p_{\Omega_1}(A)\times p(\Omega_1)+p_{\Omega_2}(A)\times p(\Omega_2)+\ldots+p_{\Omega_n}(A)\times p(\Omega_n)\]
\end{bclogo} 

\subsection{Indépendance}

\begin{bclogo}{Définition}
Si deux événements $A$ et $B$ sont indépendants, alors :\[p(A\cap B)=p(A)\times p(B)\]

Si $p(A)\neq 0$, alors $p_A(B)=p(B)$

Deux événements de probabilités non nulles, incompatibles ne sont pas indépendants :

\[\forall A\cap B =\oslash, \left. \begin{array}{l} p(A\cap B)=0\\ p(A)\times p(B)\neq 0\end{array}\right\} \text{ donc } p(A\cap B) \neq p(A)\times p(B)\]
\end{bclogo}

\medskip

\begin{bclogo}{Expériences aléatoires indépendantes}
Soit une expériencen succession de $n$ expériences aléatoires indépendantes $E_1,E_2,E_3\ldots,E_n$. Une issue de l'expérience est une liste $(e_1,e_2,\ldots,e_n)$. Soit $\Omega_i$ et $p_i$ l'univers et la loi de probabilité de l'expérience $E_i$.
\[p(e_1,e_2,\ldots,e_n)=p_1(E_1)\times p_2(E_2)\times \cdots \times p_n(E_n)\]
\end{bclogo}

\medskip

\begin{bclogo}{Variables aléatoires indépendantes}
Soient $X$ et $Y$, deux variables aléatoires discrètes sur un univers $\Omega$. 

$X$ prend pour valeurs : $x_1,x_2,\ldots,x_p$

$Y$ prend pour valeurs : $y_1,y_2,\ldots,y_k$

Dire que $X$ et $Y$ sont indépendantes signifie que : pour tout $i\in {1;2;\ldots;p}$ et pour tout $j\in {1;2;\ldots;k}$, les événements $(X=x_i)$ et $(Y=y_i)$ sont indépendants. \[p\left[ (X=x_i)\cap (Y=y_j)\right] =p(X=x_i)\times p(Y=y_j)\]
\end{bclogo}

\end{document}