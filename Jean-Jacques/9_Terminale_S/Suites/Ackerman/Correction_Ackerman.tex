
\ifdefined\COMPLETE
\else
    \input{./preambule.ltx}
\usepackage{diagbox}
\frenchbsetup{StandardItemLabels=true, CompactItemize=false, ReduceListSpacing=true}
    \begin{document}
\fi


\centerline{\huge \bf Fonction Ackerman}

\bigskip

\begin{enumerate}


\item On a $A(0,n) = n+1$ pour tout $n \in \mathbb{N}$.

Pour $n=0$, on a donc \fbox{$A(0,0) = 0 + 1 = 1$}

\begin{itemize}
\item De même, pour  $n=1$, on  \fbox{$A(0,1) = 1 + 1 = 2$}

\item On a $A(n+1,0) = A(m,1)$  pour tout  $m \in \mathbb{N}$.

Pour  $m=0$, on a $A(0+1,0) = A(1, 0) = A(0,1)$

Ainsi, d'après le point précèdent, \fbox{$A(1,0)=2$}
\end{itemize} 

\bigskip 

\item ~ \\
% \renewcommand{\arraystretch}{1.8}
\begin{tabular}{l|l}
\begin{tabular}{|c|c|c|c|}
\hline
        \diagbox%[width=10em]
        {n}{m}
             &  0 & 1 &  2 \\
\hline             
       0    &  1 & 2 &  3 \\    
\hline       
       1    &  2 & 3 &  5 \\
\hline       
       2    &  3 & 4 &  7 \\
\hline       
       3    &  4 & 5 &  9 \\
\hline       
       4   &   5 & 6 & 11 \\
\hline
\multicolumn{4}{c}{}
\end{tabular} % \medskip 
 &
                          \begin{minipage}{8cm}
                          \begin{itemize}
                          \item $A(0,n) = n +1$ pour tout $n \in \mathbb{N}$.\\
                                   Donc $A(0,0) = 1 $ d'après 1. \\
                                            $A(0,1) = 2 $ d'après 1. \\
                                            $A(0,2) = 3 $ ,  $A(0,3) = 4 $ ,   $A(0,4) = 5 $                                               
                         \item $A(m+1, 0) = A(m,1)$ pour tout $m \in \mathbb{N}$.\\
                                 Donc, $A(1,0 = A(0,1) = 2$
                          \end{itemize}          
                          \end{minipage} \\
 \hline
 & \\
 Pour $m=n=0 $, $ \quad A(1,1)$  &
                             Pour $m=1, n=0$, $ \quad  A(2,1) $ \\
$  \qquad \qquad = A\left(0,A(1,0)\right) = A(0,2) = 3$ &
                             $\qquad   = A\left(1,A(2,0)\right)= A(1,3) = 5$ \\
 & \\                             
 \hline 
 & \\
 Pour $m=0, n=1 $, $ \quad A(1,2)$  &
                             Pour $m=1, n=1$, $ \quad  A(2,2) = A\left(1,A(2,1)\right)  $ \\
$  \qquad = A\left(0,A(1,1)\right) = A(0,3) = 4$ &
                             $\qquad   = A(1,5) = A\left(0,A(1,4)\right) = A(0,6) = 7$ \\
 & \\                             
 \hline
 & \\
 Pour $m=0, n=2 $, $ \quad A(1,3)$  &
                             Pour $m=1, n=2$, $ \quad  A(2,3) = A\left(1,A(2,2)\right)  $ \\
$  \qquad = A\left(0,A(1,2)\right) = A(0,4) = 5$ &
                             $\qquad   = A(1,7) = A\left(0,A(1,6)\right) = \ldots = 9$ \\
 & \\                             
 \hline
 & \\
 Pour $m=0, n=3 $, $ \quad A(1,4)$  &
                             Pour $m=1, n=3$, $ \quad  A(2,4)= \ldots = 11$ \\
$  \qquad = A\left(0,A(1,3)\right) = A(0,5) = 6$ & \\
 & \\                             
 \hline                                                
\end{tabular}


\bigskip 


\item ~
\begin{itemize}
\item Montrons que $A(1,n) = n+2 $ pour tout $n \in \mathbb{N}$ \\

\underline{Initialisation} : pour $n=0$, on a  $ \left\lbrace 
                                                                       \begin{array}{rl}
                                                                       A(1,0) & = 2 \\
                                                                       n+2    & =2  \\
                                                                       \end{array}
                                                                    \right. $ \\
 D'où $A(1,0) = 0 + 2 = 2$, ainsi la propriété est initialisée.
 
 \bigskip 
 
 \underline{Hérédité} :  \\
 $  \begin{array}{rll}
                                         A (1, n+1) & = A\left(0, A(1,n)\right)  & \qquad \mathrm{Par\; hypothèse\; de\; récurrence} \\
                                                          & = A (0, n+2)  & \qquad \mathrm{Par\;définition} \\
                                                          & = n + 3 \\
                                        \end{array} $ \\
Donc, la propriété est héréditaire . \\

Donc $\forall n \in  \mathbb{N}, A(1,n) = n+ 2 $            

\newpage                

\item              Montrons que $A(2,n) = 2n+3$   

\underline{Initialisation} :  pour $n=0$, on a  $ \left\lbrace 
                                                                       \begin{array}{rl}
                                                                       A(2,0) & = 3 \\
                                                                       2n + 3    & =2  \\
                                                                       \end{array}
                                                                    \right. $ \\
 D'où $A(2,0) = 0 + 3 = 2$, ainsi la propriété est initialisée.

 \underline{Hérédité} :  \\
  $  \begin{array}{rll}
                                         A (2, n+1) & = A\left(1, A(2,n)\right)  & \qquad \mathrm{Par\; hypothèse\; de\; récurrence} \\
                                                          & = A (1, 2n+3)  & \qquad \mathrm{d'après \;la \;première \;partie \;de\;la\;question} \\
                                                          & = 2n + 5 & \qquad \mathrm{en \; simplifiant \; pour} \\
                                                          & = 2(n+1) +3   & \qquad \mathrm{montrer \; que \; l'on \; a \; fini} \\
                                        \end{array} $ \\
Donc, la propriété est héréditaire . \\   

Donc $\forall n \in  \mathbb{N}, A(2,n) = 2n+ 3 $           
                                   
\end{itemize}

\item  ~ 

\begin{tabular}{l|l}
\begin{tabular}{|c|c|}
\hline
        \diagbox{n}{m} & 3 \\
\hline        
        0 & 5 \\
\hline
        1  & 13 \\
\hline
         2 &  29 \\
\hline
         3 & 61 \\
\hline
         4 & 125 \\
%          4   &   5 & 6 & 11 \\
\hline
\multicolumn{2}{c}{}         
\end{tabular}  &  \begin{minipage}{8cm}
                           $\begin{array}{lll}
                           A(3,0)  & = A(2,1)                       &  = 5 \\
                           A(3,1)  & = A\left(2, A(3,0)\right) & = A(2,5) \\
                                      &                                      & = 2\times 5 + 3 = 13 \\
                            A(3,2)  & = A\left(2, A(3,1)\right) & = A(2,13) \\
                                      &                                      & = 2\times 13 + 3 = 29 \\                          
                            A(3,3)  & = A\left(2, A(3,2)\right) & = A(2,29) \\
                                      &                                      & = 2\times 29 + 3 = 61 \\           
                            A(3,4)  & = A\left(2, A(3,3)\right) & = A(2,61) \\
                                      &                                      & = 2\times 61 + 3 = 125 \\          
                           \end{array}
                           $        
                          \end{minipage} \\                       
\end{tabular}
  
\bigskip 

\item Montrons que   $A(3,n) = 2^{n+3} -3 $ 

\underline{Initialisation} :  pour $n=0$, on a  $ \left\lbrace 
                                                                       \begin{array}{rl}
                                                                       A(3,0) & = 5 \\
                                                                       2^{0+3} - 3    & = 8 -3 = 5  \\
                                                                       \end{array}
                                                                    \right. $ \\
 D'où $A(3,0) = 2^{0+3} - 3 $,  la propriété est initialisée.

 \underline{Hérédité} :  \\
  $  \begin{array}{rll}
                                         A (3, n+1) & = A\left(2, A(3,n)\right)  & \qquad \mathrm{Par\; hypothèse\; de\; récurrence} \\
                                                          & = A (2, 2^{n+3}-3)  & \qquad \mathrm{d'après \;la \;seconde \;partie \;de\;la\;question 3 } \\
                                                          & = 2(2,2^{n+3}-3)+3   &  \\                                                          
                                                          & = 2 \times 2^{n+3} -6 + 3 & \qquad \mathrm{en \; développant} \\
                                                          & =   2^{(n+1)+3 } -3  & \qquad \mathrm{en \; simplifiant} \\
                                        \end{array} $ 
                                        
 \bigskip 
                                       
Donc, la propriété est héréditaire.

 \bigskip 

Donc $\forall n \in  \mathbb{N}, A(3,n) = 2^{n+3} - 3 $         


\end{enumerate}

\ifdefined\COMPLETE
\else
    \end{document}
\fi
