\input{./preambule.ltx}\usepackage{amsmath,bm}
%\usepackage{graphicx}
%\usepackage{animate}
\usepackage[tikz]{bclogo}



\begin{document}

\newcommand{\titre}[1]{\begin{center}{\Large\textcolor{red}{#1}}\end{center}}

\newcommand{\paragraphe}[1]{\large\textcolor{red}{#1}}

\newcommand{\NBVert}[1]{\large\textcolor{DarkGreen}{#1}}

\newcommand{\Attention}[3]{
\begin{bclogo}[%
%epBarre = 0,
%couleurBarre = white,
barre = none,
couleurBord=white,%
logo=\bcattention,% 
margeG = -1,% 
margeD = 1,%
marge = 15%
]{\textcolor{#1}{$\quad$ #2}}
#3
\end {bclogo}
}

\titre{Fonctions composées (associées)\\
\bigskip
Exemples rédigés}

\paragraphe{Méthode} : \begin{itemize}
\item On décompose la fonction en « blocs » simples, 
\item on applique les règles de composition  \underline{de droite à gauche}.
\end{itemize}

\bigskip 

\paragraphe{Exemple n°1} : Soit $f : x \mapsto \sqrt{x-2}$,   $\quad$   $f$ définie sur $[ 2 ; +\infty [$ . 

\NBVert{Au brouillon : $f$, c'est $\sqrt{\textcolor{white}{X}}\quad  \text{\fbox{de}} \quad x -2$} .

On pose $g(X) = \sqrt{X}$ et $h(x)= x - 2$ 

On a donc $f = g \circ h $ 

On sait que $h$ est croissante sur  $[2; +\infty[$ .\\
On sait que composer par $ X \mapsto  \sqrt{X} $ ne change pas le sens de variation.

donc $ g \circ h $  est croissante sur  $[ 2 ; +\infty [$ 


\Attention{Red}{à l'ordre quand on décompose en blocs !}

\paragraphe{Exemple n°2} :

Soit $f:x\mapsto  - \dfrac{2}{3x}$,  $\quad$   $f$ définie sur $]-\infty ; 0 [ \cup ]0 ; +\infty[$.

\Attention{Blue}{On cherche toujours les variations sur un intervalle, donc un ensemble « sans trou »}

\underline{Sur $]0;+\infty[$} :

\NBVert {Au brouillon : $f$, c'est $-\dfrac{2}{3}$ \fbox{$\;\times\:$} $\dfrac{2}{x}$} .

On pose $g(x) = -\dfrac{2}{3}$ et $h(x) = \dfrac{1}{x}$ . 

On a $f=g$ \fcolorbox{blue}{white}{$\;\times\;$}$h$ .\\
On sait que $h$ est décroissante sur $]0;+\infty[$.

Or, multiplier par une constante négative change le sens de variation. 

Donc $g\times h$ est croissante.

On en conclut que $f$ est croissante sur $]0; +\infty[$. 

\underline{Remarque } : Le résultat est le même sur $]-\infty ; 0[$ . 

\newpage

\paragraphe{Exemple n°3 : un peu plus long\ldots}

Soit $f:x\mapsto  - \dfrac{\sqrt{3}}{\sqrt{x^2+1}+8}$,  $\quad$   définie sur $\mathbb{R}$.

Dans un premier temps, on s'intéresse aux variations de $f$ sur $[0, +\infty[$

\NBVert{au brouillon, on a $f$, c'est $-\sqrt{3} \times \dfrac{1}{\sqrt{x^2+1}+8}$,\\
donc $\-\sqrt{3} \times \left(\dfrac{1}{\ldots} \text{ et } \sqrt{x^2+1}+8\right)$ \\
donc $\-\sqrt{3} \times \left(\dfrac{1}{\ldots} \text{ et } \left(\sqrt{x^2+1}  \text{ et }  8\right)\right)$ \\             
donc $\-\sqrt{3} \times \left(\dfrac{1}{\ldots} \text{ et } 
         \left(\left(\sqrt{\textcolor{white}{X}}   \text{ et }        {x^2+1}\right)  \text{ et }  8\right)\right)$ \\  
}                    

On pose $\left\lbrace \begin{array}{l@{\,:\,}l}
                        g & x \mapsto \sqrt{3} \\
                        h & x \mapsto \dfrac{1}{x} \\
                        i & x \mapsto \sqrt{x} \\
                    \end{array}    \right. $  $\qquad$ et $ \left\lbrace \begin{array}{l@{\,:\,}l}
                                                                        j & x \mapsto x^2 + 1 \\
                                                                        k & x \mapsto 8 \\
                                                                          \end{array}    \right. $ 
                                                                          
On a $f = g \times \left( h \circ \left(\left(i \circ j\right)+k\right)\right)$

\Attention{Blue}{Quand il y a des parenthèses, elles sont prioritaires !}
{\textcolor{Blue}{\bf Mais on va toujours de droite à gauche.}}

On sait que $j : x \mapsto x^2 +1$.

Or, la composition par $x \mapsto \sqrt{x}$ ne modifie pas le sens de variation. Donc $i \circ j$ est croissante.

Or, ajouter une constante à une fonction ne change pas son sens de variation, donc $(i \circ j) +k$ est croissante.

Or, la composition par $x \mapsto \dfrac{1}{x}$ inverse le sens de variation, donc $h\circ \left(\left(i\circ j\right)+k\right)$ est décroissante. 

Or, la multiplication par une constante négative change le sens de variation.

Donc, $g \times \left( h\circ \left(\left(i\circ j\right)+k\right) \right)$ est croissante. 

On en conclut que $f$ est croissante. 


\end{document}


