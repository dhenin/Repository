\input{./preambule.ltx}\usepackage{amsmath,bm}
%\usepackage{graphicx}
%\usepackage{animate}
\usepackage[tikz]{bclogo}



\begin{document}

\newcommand{\titre}[1]{\begin{center}{\Large\textcolor{red}{#1}}\end{center}}

\newcommand{\paragraphe}[1]{\large\textcolor{red}{#1}}

\newcommand{\NBVert}[1]{\large\textcolor{DarkGreen}{#1}}

\newcommand{\Attention}[3]{
\begin{bclogo}[%
%epBarre = 0,
%couleurBarre = white,
barre = none,
couleurBord=white,%
logo=\bcattention,% 
margeG = -1,% 
margeD = 1,%
marge = 15%
]{\textcolor{#1}{$\quad$ #2}}
#3
\end {bclogo}
}

\titre{Sens de variations de fonctions\\
\bigskip
 (sans les dérivées)}
 

Étudier chaque fonction suivante, en donnant ses variations sur l'ensemble demandé (si un ensemble contient plusieurs intervalles, on étudiera la fonction sur chaque intervalle)


$f_1(x) = \sqrt{x} -3$ sur $D_{f_1}$ \\

 $f_2(x) = -2 \sqrt{x} +5 $ sur $D_{f_2}$ \\

$f_3(x) = 1 - \dfrac{2}{x}$ sur $D_{f_3}$ \\

 $f_4(x) = \dfrac{2}{3x} $ sur $D_{f_4}$ \\

$f_5(x) = \sqrt{x -3}$ sur $D_{f_5}$ \\

$f_6(x) = -2 \sqrt{1-2x} $ sur $D_{f_6}$ \\

$f_7(x) = \sqrt{x^2 -1}$ sur $D_{f_7}$ \\

 $f_8(x) =  \dfrac{1}{x -2}  $ sur $D_{f_8}$ \\

$f_9(x) = \dfrac{1}{1-2x}$ sur $D_{f_9}$ \\

 $f_{10}(x) = \dfrac{1}{\sqrt{x}} $ sur $D_{f_{10}}$.




\end{document}


