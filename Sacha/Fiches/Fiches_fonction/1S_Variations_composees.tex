\input{./preambule.ltx}\usepackage{amsmath,bm}
%\usepackage{graphicx}
%\usepackage{animate}
\usepackage[tikz]{bclogo}



\begin{document}

\newcommand{\titre}[1]{\begin{center}{\Large\textcolor{red}{#1}}\end{center}}

\newcommand{\paragraphe}[1]{\large\textcolor{red}{#1}}

\newcommand{\NBVert}[1]{\large\textcolor{DarkGreen}{#1}}

\newcommand{\Attention}[3]{
\begin{bclogo}[%
%epBarre = 0,
%couleurBarre = white,
barre = none,
couleurBord=white,%
logo=\bcattention,% 
margeG = -1,% 
margeD = 1,%
marge = 15%
]{\textcolor{#1}{$\quad$ #2}}
#3
\end {bclogo}
}

\titre{Variations de fonctions\\
\bigskip
 composées (ou associées)}
 
 Soient $f$ et $g$ deux fonctions. Soient $\lambda$ et $k$ deux réels, \underline{avec $\lambda \neq 0$}
 
 
\paragraphe{1) Somme (affinité)} : 

Si $g(x) = f(x) + k$, alors $f$ et $g$ ont le même sens de variations.

\paragraphe{2) Produit} : 

\begin{tabular}{c|c}
\multicolumn{2}{l}{Si $g(x) = \lambda f(x)$, alors\ldots} \\
\\
\underline{Si $\lambda > 0$} & \underline{Si $\lambda < 0$} \\
$f$ et $g$ ont le même sens & $f$ et $g$ont des sens de \\
de variations & variations contraires \\
\end{tabular}

\paragraphe{3) Composition par $\sqrt{\textcolor{white}{X}}$} :
\begin{itemize}
\item Si $g(x) = \sqrt{f(x)}$, alors $f$ et $g$ ont le même sens de variations.

\item Si $g(x) = \dfrac{1}{f(x)}$, alors $f$ et $g$ ont des  sens de variations contraires.

\end{itemize} 

\paragraphe{À retenir } : La multiplication par un nombre négatif et le passage à l'inverse changent les variations.

\end{document}


