\input{./preambule.ltx}
\usepackage{amsmath,bm}
%\usepackage{graphicx}
%\usepackage{animate}
\usepackage[tikz]{bclogo}
\usepackage{texgraph}



\begin{document}

\newcommand{\titre}[1]{\begin{center}{\Large\textcolor{red}{#1}}\end{center}}

\newcommand{\paragraphe}[1]{\large\textcolor{red}{#1}}

\newcommand{\NBVert}[1]{\large\textcolor{DarkGreen}{#1}}

\newcommand{\Attention}[3]{
\begin{bclogo}[%
%epBarre = 0,
%couleurBarre = white,
barre = none,
couleurBord=white,%
logo=\bcattention,% 
margeG = -1,% 
margeD = 1,%
marge = 15%
]{\textcolor{#1}{$\quad$ #2}}
#3
\end {bclogo}
}

\titre{Représentation graphique\\
\bigskip
des suites\ldots \hspace*{3cm} Récurrentes $(u_{n+1} = au_n +b)$}
 


\setbox5=\hbox{\parbox[b][4em][t]{0.2\textwidth}{
\begin{tikzpicture}[line cap=round,line join=round,>=triangle 45,x=1cm,y=1cm,scale=.5]
\draw[->] (-0.7,0) -- (6,0);
\foreach \x in {,1,2,3,4,5}
\draw[shift={(\x,0)}] (0pt,2pt) -- (0pt,-2pt) node[below] {\footnotesize $\x$};
\draw[->] (0,-0.7) -- (0,5);
\draw[color=black] (0pt,-8pt) node[left] {\footnotesize $0$};
\clip(-1,-0.7) rectangle (6,5);
\draw [color=red] (0,1)-- ++(-1.0pt,-1.0pt) -- ++(2.0pt,2.0pt) ++(-2.0pt,0) -- ++(2.0pt,-2.0pt);
\draw [color=red] (0.98,1.65)-- ++(-1.0pt,-1.0pt) -- ++(2.0pt,2.0pt) ++(-2.0pt,0) -- ++(2.0pt,-2.0pt);
\draw [color=red] (2,2.33)-- ++(-1.0pt,-1.0pt) -- ++(2.0pt,2.0pt) ++(-2.0pt,0) -- ++(2.0pt,-2.0pt);
\draw [color=red] (3,3)-- ++(-1.0pt,-1.0pt) -- ++(2.0pt,2.0pt) ++(-2.0pt,0) -- ++(2.0pt,-2.0pt);
\draw [color=red] (4.02,2.68)-- ++(-1.0pt,-1.0pt) -- ++(2.0pt,2.0pt) ++(-2.0pt,0) -- ++(2.0pt,-2.0pt);
\draw [color=red] (5.02,4.35)-- ++(-1.0pt,-1.0pt) -- ++(2.0pt,2.0pt) ++(-2.0pt,0) -- ++(2.0pt,-2.0pt);
\begin{pgfonlayer}{background}   
\draw[step=1mm,ultra thin,AntiqueWhite!10] (-0.7,-0.7) grid (6,5);
\draw[step=5mm,very thin,AntiqueWhite!30]  (-0.7,-0.7) grid (6,5);
\draw[step=1cm,very thin,AntiqueWhite!50](-0.7,-0.7) grid (6,5);
\draw[step=5cm,thin,AntiqueWhite]         (-0.7,-0.7) grid (6,5);
\end{pgfonlayer}
\end{tikzpicture}}}

\setbox1=\hbox{\parbox[t][4em][t]{0.25\textwidth}{\centerline{\underline{Explicites}}
On trace la représentation graphique de la fonction associée, en plaçant les points dont l'abscisse est entière, et \underline{sans relier les points}.

\box5
}}

\setbox2=\hbox{\parbox[b][4em][t]{0.20\textwidth}{$u_{n+1} = b$,\\
                                                   on est dans le \\
                                                   cas explicite \\      
                                              }}
                                                   
\setbox3=\hbox{
\begin{texgraph}[file,name=suite_02,export=tkz]
Include "papiers.mac";
Graph image = [
Fenetre(-.9+12*i, 12-.9*i, 1+i),Marges(0,0,0,0),
papier(milli,-1-i,12+12*i,
[subsubgridcolor :=beige,
subgridcolor:=antiquewhite,
gridcolor :=bisque]
),
Arrows:=1,
Axes(0,1 +i),
Arrows:=0,
u0:=11,nb:=15, Width:=6,
Color:=darkseagreen, Droite(1,-1,0), 
LabelAngle:=0, Label (2+i, "$y=x$") ,
Color:=red,  tMin:=-1, tMax:=12, Width:=8, 
              Cartesienne((3/4)*x+1),
LabelAngle:=0, Label (6+4.3*i, "$y=\frac{3x}{4}+1$"),              
Width:=6, Color:=gray,
suite((3/4)*x+1, u0,nb),
Color:=black,LabelAngle:=0, 
Label(8.5+9.25*i, "$U_1$", 
      7.5+7.93*i, "$U_2$",  
      6.6+6.95*i, "$U_3$",   
      5.7+6.21*i, "$U_4$",   
      5.3+5.66*i, "$U_5$")   
];
\end{texgraph}}


\setbox4=\hbox{\begin{texgraph}[file,name=Suite03,export=pgf]
Include "papiers.mac";
Graph image = [
Fenetre(-3+12*i, 12-.9*i, 1+i),Marges(0,0,0,0),
papier(milli,-3-i,12+12*i,
          [subsubgridcolor :=beige,
              subgridcolor :=antiquewhite,
                 gridcolor :=bisque]
      ),
  Arrows:=1,
  Axes(0,1 +i),
  Arrows:=0,
  u0:=-2,nb:=15, Width:=6,
Color:=darkseagreen, Droite(1,-1,0), 
 Label (2+i, "$y=x$"),
Color:=red,  tMin:=-3, tMax:=11, Width:=8, 
              Cartesienne(-1*(3/4)*x+7),
 Label (-2+9.6*i, "$y=\frac{-3x}{4}+7$"),              
 Width:=6, Color:=gray,
             suite(-1*(3/4)*x+7, u0,nb),
Color:=black,LabelAngle:=0, 
Label(-5/2+8.5*i, "$U_1$", 
        9+.7*i,   "$U_2$",  
        .8+6.8*i, "$U_3$",   
         7+2.2*i,   "$U_4$",   
       2.5+5.66*i, "$U_5$"
)   
];
\end{texgraph}}

\begin{tabular}{c|c|c|c}
 &  \\
\multirow{8}{0.25\textwidth}{\box1}   & $a = 0 $ & $a > 0$ (escalier) & $a < 0 $ (escargot) \\
 \cline {2-4}
 & \multirow{6}{0.20\textwidth} {\box2} & \multicolumn{2}{l}{$\bullet$ On trace la courbe ($\mathcal{C}$) de la fonction associée $y=ax+b$ } \\
 & & \multicolumn{2}{l}{$\bullet$ On place $u_0$ sur l'axe de abscisses. } \\
 & & \multicolumn{2}{l}{$\bullet$ On trace la perpendiculaire jusqu'à ($\mathcal{C}$).} \\
 & & \multicolumn{2}{l}{$\bullet$ On « bascule » sur $y=x$. } \\
 & & \multicolumn{2}{l}{$\bullet$ On recommence les points 3 et 4 autant qu'il le faut } \\    
 & &  \scalebox{.3}{\box3} & \scalebox{.3}{\box4} \\
\end{tabular}



\end{document}


