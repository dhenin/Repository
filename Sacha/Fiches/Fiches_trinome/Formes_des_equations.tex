\ifdefined\COMPLETE
\else
    \input{./preambule-sacha-utf8.ltx}
    
\frenchbsetup{StandardLists=true} % à inclure si on utilise \usepackage[french]{babel}
\usepackage{enumitem}   

    \begin{document}
\fi

\centerline{\huge \bf « Formes » des trinômes du 2\up{nd} degré} 

\bigskip 

\begin{enumerate}


\item \textcolor{red}{Formes :}

Il existe 3 formes :


\begin{itemize} [label=$\bullet$]
\item la \textcolor{red}{forme développée :} $\; ax^2 + bx + c$ 
\item   la \textcolor{red}{forme factorisée :} $\quad a(x-x_1) x-x_2)$ ou $a(x-x_0)^2$ 
\item  la \textcolor{red}{forme canonique:} $\;\; a(x-\alpha)^2 + \beta$ 
     \end{itemize}

\item \textcolor{red}{Passer d'une forma à l'autre :}

Pour passer de \textcolor{darkgreen}{\sc FD} à  \textcolor{darkgreen}{\sc FF :}

\color{blue}{
\begin{itemize} [label=]
\item \underline{Étape 1} : On résout l'équation $ax^2 + bx +c = 0$.
\item   \underline{Étape 2} : Selon le signe de $\Delta$ : 

\begin{tabular}{M{3cm}|M{5cm}|M{5cm}}
\multicolumn{1}{c|}{Si $ \Delta < 0$} & \multicolumn{1}{c|}{Si $ \Delta = 0$} &  \multicolumn{1}{c}{Si $ \Delta > 0$} \\
$\bullet$ Il n'y a pas de forme factorisée 
              & $\bullet$  Après avoir identifié \\la solution $x_0$, on a : \\
                                 $ax^2 + bx +c = a (x -x_0)^2 $  
                          & $\bullet$  Après avoir identifié \\les solutions $x_1$ et $x_2$, on a :
                                       $ax^2 + bx +c = a (x -x_1) (x-x_2) $  \\
\end{tabular}
 \end{itemize} }
 
\bigskip 

 Pour passer de \textcolor{darkgreen}{\sc FD} à  \textcolor{darkgreen}{\sc FC :}

\color{blue}{
\begin{itemize} [label=]
\item \underline{Étape 1} : On calcule \textcolor{darkgreen}{$\alpha = \dfrac{-b}{2a}$} et  \textcolor{darkgreen}{$\beta = f(\alpha)$}
\item   \underline{Étape 2} : On a $ax^2 +bx +c = a(x-\alpha)^2 +\beta$
 \end{itemize} }
  
\bigskip 

  Pour passer de \textcolor{darkgreen}{\sc FF} à  \textcolor{darkgreen}{\sc FD :}
  
\color{blue}{
On applique les règles de distributivité (ordre de calcul : puissance, multiplication, addition)
 }
 
\bigskip 
 
  Pour passer de \textcolor{darkgreen}{\sc FF} à  \textcolor{darkgreen}{\sc FC :}

\color{blue}{
\begin{itemize} [label=]
\item \underline{Étape 1} :  On passe de {\sc FF} à {\sc FD}
\item   \underline{Étape 2} : On passe de {\sc FD} à {\sc FC}
 \end{itemize} }
 
\bigskip 
     
  Pour passer de \textcolor{darkgreen}{\sc FC} à  \textcolor{darkgreen}{\sc FD :}
  
\color{blue}{
On applique les règles de distributivité (ordre de calcul : puissance, multiplication, addition)
 }
  
\bigskip 

  Pour passer de \textcolor{darkgreen}{\sc FC} à  \textcolor{darkgreen}{\sc FF :}

\color{blue}{
\begin{itemize} [label=]
\item \underline{Étape 1} :  On passe de {\sc FC} à {\sc FD}
\item   \underline{Étape 2} : On passe de {\sc FD} à {\sc FF}
 \end{itemize} }
     
           
\end{enumerate}



\ifdefined\COMPLETE
\else
    \end{document}
\fi

\item [] \textcolor{blue}{Étape préliminaire : On vérifie que l'on n'a pas d'identité remarquable}

\underline {Ex} : $x^2 -2 x + 1 = 0 $ 
\end{itemize}