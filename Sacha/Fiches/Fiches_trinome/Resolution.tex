\ifdefined\COMPLETE
\else
    \input{./preambule-sacha-utf8.ltx}
    
\frenchbsetup{StandardLists=true} % à inclure si on utilise \usepackage[french]{babel}
\usepackage{enumitem}   

    \begin{document}
\fi

\centerline{\color{red} \huge \bf « Résolution » des équations du 2\up{nd} degré} 

\medskip 

\centerline{\color{red} \huge \bf $ax^2 + bx +c = 0$} 

\bigskip 

\begin{itemize}[label=]

\item  \textcolor{blue}{Étape préliminaire : On vérifie que l'on n'a pas d'identité remarquable} 

% by placing rules in the first line the \parbox[t] 
% code aligns on the rule (so effectively at the top of the first real line 
% rather than its baseline)

\parbox[t]{1cm}{\hrule height 0pt width 0pt%
\underline {Ex} :} 
%
\parbox[t]{10cm}{\hrule height 0pt width 0pt%
$\begin{array}{l}
\star  x^2 -2 x + 1 = 0 
        \xLeftrightarrow{\text{id. rem.}} (x-1)^2 = 0
        \xLeftrightarrow{\text{eq. prod nul.x}} x-1 = 0 
        \Longleftrightarrow x=1 \\
\star        x^2 -2x -3 = 0 \longrightarrow \text{Pas d'identité remarquable} \\                                  
\end{array} $ } 

\bigskip 

\item  \textcolor{blue}{Étape 1 : On calcule $\Delta = b^2 -4ac$}

\parbox[t]{1cm}{\hrule height 0pt width 0pt \underline {Ex} :} 
%
\parbox[t]{10cm}{\hrule height 0pt width 0pt $\begin{array}{l}
\star \text{Pour } x^2 -2x -3 = 0 \text{, on a } \Delta = (-2)^2 -4 \times 1 \times (-3) = 4 + 12 = 16\\
\star \text{Pour } 7x^2 +x +9 = 0\text{, on a } \Delta = (1)^2 -4 \times 7 \times (9) = 1-252 = -251\\
\star \text{Pour } x^2 -2x +1 = 0 \text{, on a } \Delta = (-2)^2 -4 \times 1 \times (1) = 0\\       
\end{array} $}


\bigskip 

\item  \textcolor{blue}{Étape 2 : On calcule le signe de $\Delta$}

\parbox[t]{1cm}{\hrule height 0pt width 0pt \underline {Ex} :} 
%
\parbox[t]{10cm}{\hrule height 0pt width 0pt $\begin{array}{l}
\star \text{Pour } x^2 -2x -3 = 0 \text{, on a } \Delta > 0\\
\star \text{Pour } 7x^2 +x +9 = 0\text{, on a } \Delta < 0\\
\star \text{Pour } x^2 -2x +1 = 0 \text{, on a } \Delta = 0\\       
\end{array} $}


\bigskip 

\item  \textcolor{blue}{Étape 3 : Selon le signe de $\Delta$, on conclut : }

\color{blue}{
\begin{tabular}{M{3cm}|M{4cm}|M{4cm}}
\multicolumn{1}{c|}{Si $ \Delta < 0$} & \multicolumn{1}{c|}{Si $ \Delta = 0$} &  \multicolumn{1}{c}{Si $ \Delta > 0$} \\
 Alors n'y a pas de forme factorisée 
              & Alors l'équation admet une solution unique :
                   \hspace*{.75cm}$ x_0 = \dfrac{-b}{2a}$
                          & Alors l'équation admet deux solutions :
                             \hspace*{.75cm} $x_1 = \dfrac{-b-\sqrt{\Delta}}{2a}$ 
                              \hspace*{.75cm} $x_2 = \dfrac{-b+\sqrt{\Delta}}{2a} $\\
\end{tabular}}

\item  \textcolor{blue}{Étape 4  : On note $\mathcal{S}=\lbrace ...  \rbrace $} 
\end{itemize}






\ifdefined\COMPLETE
\else
    \end{document}
\fi
