\chapter{Continuité de fonctions à variable réelle}

\section*{Exercice \no 1}

On donne le tableau de variations suivant : \\

\variations
% \begin{variations}
x & \mI  & &3 & & 7 & & \pI \\
% \filet
f(x) & \b \mI & \c & \h{4} & \d & \b {-10} & \c & \h{-1} \\
% \end{variations}
\fin 

\vspace*{.3cm}

Déterminer, par lecture graphique, le nombre de solutions de l'équation $f\left(x\right) = 0$ sur $\R$. Placer alors ces solutions sur le tableau de variations (approximativement).

\section*{Exercice \no 2}

Soit la fonction $g$ définie sur $\R$ par : $g\left(x\right) = x^2$. \\ Soit la fonction $h$ définie sur $\left[-2 \; ; \; +\infty\right[$ par $h\left(x\right) = \sqrt{x + 2}$. \\

On donne à la fin des énoncés des exercices les représentations graphiques de ces fonctions.

\begin{itemize}
\item[1.] \textbf{Sans lecture graphique}, montrer que l'équation $f\left(x\right) = 0$ admet au moins une solutions sur $\left[-2 \; ; \; 0\right]$. \\
\item[2.] \textbf{Par lecture graphique}, montrer que $f\left(x\right) = 0$ admet une unique solutions sur $\left[-2 \; ; \; 0\right]$. Conjecturer graphiquement la valeur de cette solution. \\
\item[3.] Valider la conjecture de la question 2. par le calcul. 
\end{itemize}

\section*{Exercice \no 3}

Soit la fonction $f$ définie sur $\R$ par : $f(x) = x^3 + x^2 - 3$. \\

\begin{itemize}
\item[1.] Déterminer les variations de la fonction $f$. (on pourra dresser son tableau de variation) \\
\item[2.] Montrer que la fonction f est strictement négative pour tout réel $x\leqslant 0$. \\ 
\item[3.] Montrer que l'équation $f(x) = 0$ admet une unique solution $\alpha$, avec $\alpha \in \left[1 \; ; \; 2\right]$. \\
\item[4.] Grâce à la calculatrice, montrer que $\alpha \in \left[1,17 \; ; \; 1,18\right]$. 
\end{itemize}

\section*{Exercice \no 4} 

Soit $f$ une fonction. \\

On sait que : \\

\begin{itemize}
\item[•] $f$ est définie sur $\R$.
\item[•] $f$ est continue sur $\R$.
\item[•] $f$ est strictement croissante sur $\R$.
\item[•] $f(0) = 0$.
\item[•] $f(3) = 3$.
\end{itemize}

Combien de solutions réelles l'équation $\left(E\right) : f(x) = 1$ a-t-elle avoir sur $\left[0 \; ; \; 3\right]$ ?

\section*{Exercice \no 5}

Montrer que le polynôme $x^3 + 3x^2 - 5$ n'admet qu'une seule racine réelle. Donner un encadrement de la solution. 

\section*{Exercice \no 6}

Soit la fonction $f$ définie sur $\left[0 \; ; \; +\infty\right[$ par : $f\left(x\right) = \dfrac{3x+7}{x+1}$. \vspace*{.3cm}

\begin{itemize}
\item[1.] Déterminer les variations de $f$ sur $\R$. \\
\item[2.] Donner l'équation de la tangente $\left(T\right)$ à la courbe représentative de $f$ au point d'abscisse $1$. 
\end{itemize}

\newpage

\section*{Exercice \no 7 : Exercice Type Bac}

On considère une fonction $f$, dont on sait que :

\begin{itemize}
\item[•] elle est définie, continue et dérivable sur $\left[-1 \; ; \; +\infty\right[$. 
\item[•] elle est strictement croissante sur l'intervalle $\left[0 \; ; \; 2\right]$.
\item[•] elle est strictement décroissante sur les intervalles $\left[-1 \; ; \; 0\right]$ et $\left[2 \; ; \; +\infty\right[$.
\end{itemize}

\vspace*{.3cm}

On note $f'$ la fonction dérivée de $f$ sur $\left[-1 \; ; \; +\infty\right[$. \\

La courbe $C_f$ est la représentation graphique de la fonction $f$. Elle est tracé dans un repère orthogonal à la fin des énoncés des exercices. \\

La courbe $C_f$ passe par les points : $A\left(-1\; ; \; 6\right)$, $B\left(0 \; ; \; -2\right)$ et $E\left(2 \; ; \; 6\right)$. Elle admet au point $D$ une tangente passant par $G\left(0 \; ; \; -4\right)$. Enfin, la courbe $C_f$ admet deux tangentes horizontales, aux points $B$ et $E$. \\

\begin{itemize}
\item[1.] Déterminer $f'(1)$ et $f'(2)$, en justifiant les réponses. \\
\item[2.] Déterminer une équation de la tangente $\left(T\right)$ à la courbe $C_f$ au point $D$. \\
\item[3.] Montrer que sur l'intervalle $\left[-1 \; ; \; 0\right]$, l'équation $f(x) = 0$ n'admet qu'une unique solution, que l'on notera $x_1$ (par le calcul). \\
\item[4.] On admet que l'équation $f(x) = 0$ admet deux autres solutions sur l'intervalle $\left[-1 \; ; \; +\infty\right[$, que l'on notera $x_2$ et $x_3$, avec $x_2 < x_3$. \\ Dresser alors le tableau de signes de la fonction $f$. \\
\item[5.] Parmi les trois courbes tracées en annexe, préciser, en justifiant la réponse, laquelle représente $f'$. \\
\end{itemize}

\newpage

\chapter*{Données graphiques}

\section*{Exercice \no 2}

La fonction $g$ est représentée en rouge, et la fonction $h$ en bleu. \\

\definecolor{qqqqcc}{rgb}{0,0,0.8}
\definecolor{ffqqqq}{rgb}{1,0,0}
\definecolor{cqcqcq}{rgb}{0.75,0.75,0.75}
\begin{tikzpicture}[line cap=round,line join=round,>=triangle 45,x=1.0cm,y=1.0cm]
\draw [color=cqcqcq,dash pattern=on 1pt off 1pt, xstep=1.0cm,ystep=1.0cm] (-3.73,-1.01) grid (5.04,4.35);
\draw[->,color=black] (-3.73,0) -- (5.04,0);
\foreach \x in {-3,-2,-1,1,2,3,4,5}
\draw[shift={(\x,0)},color=black] (0pt,2pt) -- (0pt,-2pt) node[below] {\footnotesize $\x$};
\draw[->,color=black] (0,-1.01) -- (0,4.35);
\foreach \y in {-1,1,2,3,4}
\draw[shift={(0,\y)},color=black] (2pt,0pt) -- (-2pt,0pt) node[left] {\footnotesize $\y$};
\draw[color=black] (0pt,-10pt) node[right] {\footnotesize $0$};
\clip(-3.73,-1.01) rectangle (5.04,4.35);
\draw [samples=50,rotate around={0:(0,0)},xshift=0cm,yshift=0cm,color=ffqqqq,domain=-4.0:4.0)] plot (\x,{(\x)^2/2/0.5});
\draw[color=qqqqcc,smooth,samples=100,domain=-1.999994741782436:5.035548466391935] plot(\x,{sqrt((\x)+2)});
\end{tikzpicture}

\newpage

\section*{Exercice \no 7}

Représentation graphique de $f$ : \\

\definecolor{uuuuuu}{rgb}{0.27,0.27,0.27}
\definecolor{cqcqcq}{rgb}{0.75,0.75,0.75}
\begin{tikzpicture}[line cap=round,line join=round,>=triangle 45,x=2.5cm,y=1.0cm]
\draw [color=cqcqcq,dash pattern=on 2pt off 2pt, xstep=0.5cm,ystep=1.0cm] (-1.63,-5.34) grid (3.89,7.79);
\draw[->,color=black] (-1.63,0) -- (3.89,0);
\foreach \x in {-1.5,-1,-0.5,0.5,1,1.5,2,2.5,3,3.5}
\draw[shift={(\x,0)},color=black] (0pt,2pt) -- (0pt,-2pt) node[below] {\footnotesize $\x$};
\draw[->,color=black] (0,-5.34) -- (0,7.79);
\foreach \y in {-5,-4,-3,-2,-1,1,2,3,4,5,6,7}
\draw[shift={(0,\y)},color=black] (2pt,0pt) -- (-2pt,0pt) node[left] {\footnotesize $\y$};
\draw[color=black] (0pt,-10pt) node[right] {\footnotesize $0$};
\clip(-1.63,-5.34) rectangle (3.89,7.79);
\draw plot[raw gnuplot, id=func0] function{set samples 100; set xrange [-1.52:3.78]; plot -2*x**(3)+6*x**(2)-2};
\draw [samples=100,domain=-1.63:3.89] plot (\x,{-2*(\x)^3 + 6*(\x)^2 - 2});

\draw (2.52,4.69) node[anchor=north west] {\parbox{1.09 cm}{$(C_f \\ $}};
\draw (1.4,6.89) node[anchor=north west] {$(T)$};
\begin{scriptsize}
\draw[color=black] (-1.03,7.66) node {$f$};
\draw [fill=uuuuuu] (1,2) circle (1.5pt);
\draw[color=uuuuuu] (1.04,1.9) node {$D$};
\draw [fill=uuuuuu] (2,6) circle (1.5pt);
\draw[color=uuuuuu] (2.03,6.25) node {$E$};
\draw [fill=uuuuuu] (0,-2) circle (1.5pt);
\draw[color=uuuuuu] (0.04,-1.76) node {$B$};
\draw [fill=uuuuuu] (-1,6) circle (1.5pt);
\draw[color=uuuuuu] (-0.97,6.25) node {$A$};
\draw [fill=uuuuuu] (0,-4) circle (1.5pt);
\draw[color=uuuuuu] (0.06,-3.98) node {$G$};
\end{scriptsize}
\end{tikzpicture}

Proposition de réponses pour la question 5. : 

\begin{tabular}{lll}
\definecolor{gris_jj}{rgb}{0.85,0.85,0.85}
\begin{tikzpicture}[line cap=round,line join=round,>=triangle 45,x=1cm,y=0.5cm]
% \begin{pgfonlayer}{background}  
% Attention l'ordre de ces lignes est important 
% Ne pas le modifier   
\draw[step=1mm,ultra thin,AntiqueWhite!10](-1.6,-3.5) grid (3.5,8);
\draw[step=5mm,very thin,AntiqueWhite!30] (-1.6,-3.5) grid (3.5,8);
\draw[step=1cm,very thin,AntiqueWhite!50] (-1.6,-3.5) grid (3.5,8);
\draw[step=5cm,thin,AntiqueWhite]         (-1.6,-3.5) grid (3.5,8);
% \end{pgfonlayer} 
\draw [color=gris_jj,dash pattern=on 5pt off 5pt, xstep=1.0cm,ystep=1.0cm] (-1.6,-3.5) grid (3.5,8);
\draw[->] (-1.6,0) -- (3.54,0);
\foreach \x in {-1,1,2,3}
\draw[shift={(\x,0)},color=black] (0pt,2pt) -- (0pt,-2pt) node[below] {\footnotesize $\x$};
\draw[->] (0,-3.5) -- (0,8);
\foreach \y in {-3,-2,-1,1,2,3,4,5,6,7}
\draw[shift={(0,\y)},color=black] (2pt,0pt) -- (-2pt,0pt) node[left] {\footnotesize $\y$};
\draw (0pt,-10pt) node[right] {\footnotesize $0$};
\clip(-1.6,-3.5) rectangle (3.5,8);
\draw [samples=50,domain=-2.0:3.0)] plot (\x,{\x * (-6*\x +12) });
\draw (2.5,3) node  {\footnotesize $\mathcal{C}_1$};
\end{tikzpicture}

&
\definecolor{gris_jj}{rgb}{0.85,0.85,0.85}
\begin{tikzpicture}[line cap=round,line join=round,>=triangle 45,x=1cm,y=0.5cm]
% \begin{pgfonlayer}{background}  
% Attention l'ordre de ces lignes est important 
% Ne pas le modifier   
\draw[step=1mm,ultra thin,AntiqueWhite!10](-1.6,-3.5) grid (4.5,8);
\draw[step=5mm,very thin,AntiqueWhite!30] (-1.6,-3.5) grid (4.5,8);
\draw[step=1cm,very thin,AntiqueWhite!50] (-1.6,-3.5) grid (4.5,8);
\draw[step=5cm,thin,AntiqueWhite]         (-1.6,-3.5) grid (4.5,8);
% \end{pgfonlayer} 
\draw [color=gris_jj,dash pattern=on 5pt off 5pt, xstep=1.0cm,ystep=1.0cm] (-1.6,-3.5) grid (3.5,8);
\draw[->] (-1.6,0) -- (4.5,0);
\foreach \x in {-1,1,2,3}
\draw[shift={(\x,0)},color=black] (0pt,2pt) -- (0pt,-2pt) node[below] {\footnotesize $\x$};
\draw[->] (0,-3.5) -- (0,8);
\foreach \y in {-3,-2,-1,1,2,3,4,5,6,7}
\draw[shift={(0,\y)},color=black] (2pt,0pt) -- (-2pt,0pt) node[left] {\footnotesize $\y$};
\draw (0pt,-10pt) node[right] {\footnotesize $0$};
\clip(-1.6,-3.5) rectangle (4,8);
\draw [samples=50,domain=-2.0:4.0)] plot (\x,{-.45*\x*(\x+1)*(\x-1.2)*(\x-3.75)});
\draw (2.5,3) node  {\footnotesize $\mathcal{C}_2$};
\end{tikzpicture}

&

\definecolor{gris_jj}{rgb}{0.85,0.85,0.85}
\begin{tikzpicture}[line cap=round,line join=round,>=triangle 45,x=1cm,y=0.5cm]
% \begin{pgfonlayer}{background}  
% Attention l'ordre de ces lignes est important 
% Ne pas le modifier   
\draw[step=1mm,ultra thin,AntiqueWhite!10](-1.26,-6.92) grid (3.54,5.12);
\draw[step=5mm,very thin,AntiqueWhite!30] (-1.26,-6.92) grid (3.54,5.12);
\draw[step=1cm,very thin,AntiqueWhite!50] (-1.26,-6.92) grid (3.54,5.12);
\draw[step=5cm,thin,AntiqueWhite]         (-1.26,-6.92) grid (3.54,5.12);
% \end{pgfonlayer} 
\draw [color=gris_jj,dash pattern=on 5pt off 5pt, xstep=1.0cm,ystep=1.0cm] (-1.6,-6.92) grid (3.54,5.12);
\draw[->] (-1.6,0) -- (3.54,0);
\foreach \x in {-1,1,2,3}
\draw[shift={(\x,0)},color=black] (0pt,2pt) -- (0pt,-2pt) node[below] {\footnotesize $\x$};
\draw[->] (0,-6.92) -- (0,5.12);
\foreach \y in {-6,-5,-4,-3,-2,-1,1,2,3,4}
\draw[shift={(0,\y)},color=black] (2pt,0pt) -- (-2pt,0pt) node[left] {\footnotesize $\y$};
\draw (0pt,-10pt) node[right] {\footnotesize $0$};
\clip(-1.26,-6.92) rectangle (3.54,5.12);
\draw [samples=50,domain=-2.0:3.0)] plot (\x,{\x * (6*\x -12) });
\draw (2.5,3) node  {\footnotesize $\mathcal{C}_3$};
\end{tikzpicture} 

\end{tabular}

\vspace*{-50cm}