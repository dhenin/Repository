\chapter{Limites}

\section{Limites simples}

Dans chaque cas, déterminer la limites de la suite $\left(u_n\right)$ \\ (\textbf{Remarque :} sous-entendu, en $+\infty$) : \\

a)\ \ $u_n=n^3+\dfrac{1}{n}$ 
\quad
b)\ \ $u_n=(3n+1)(-7n+5)$
\quad
c)\ \ $u_n=\dfrac{3-\dfrac{4}{n}}{\dfrac{2}{n^2}}$
\quad
d)\ \ $u_n=n^3-n^2+3n-1$

e)\ \ $u_n=\dfrac{2n^2+1}{-n^2+6}$
\quad
f)\ \ $u_n=\dfrac{n^2+3n-5}{n^3-6n^2+1}$
\quad
g)\ \ $u_n=n\sqrt{n}-n$
\\
h)\ \ $u_n=(-2n+3)\dfrac{n+3}{-n^2+n+6}$
\quad
i)\ \ $u_n=\dfrac{n}{n+\sqrt{n}}$ 
\quad
j)\ \ $u_n=\dfrac{9-n^2}{(n+1)(2n+1)}$
\\\
k)\ \ $u_n=\dfrac13-\dfrac{n}{(2n+1)^2}$
\quad
l)\ \ $u_n=\dfrac{2}{3n}-\dfrac{2n^2+3}{3n^2+n+1}$

\section{Application du théorème des gendarme - exercice type Bac}

$(u_n)$ est une suite définie par $u_0=1$ et,
pour tout entier naturel $n$, $u_{n+1}=u_n+2n+3$. 

\begin{enumerate}
\item Etudier le sens de variation de $(u_n)$. 
\item Démontrer par récurrence que, 
  pour tout entier $n$, $u_n=(n+1)^2$. 
\item En déduire que, pour tout entier $n$, 
  $u_n\geqslant n^2$.
\item La suite $(u_n)$ est-elle minorée ? majorée ? 
  Justifier. 
\item Donner la limite de $(u_n)$. 
\end{enumerate}