\input{./preambule-sacha-utf8_new.ltx}

\usepackage{parcolumns}

        \title {Exercices pour se préparer à la terminale S}

        \author{Sacha Dhénin}
        \date{Vacances d'été 2016}
        
\usepackage{amsthm}

\begin{document}

%\iffalse

\pgfmathdeclarefunction{gauss}{2}{%
  \pgfmathparse{1/(#2*sqrt(2*pi))*exp(-((\x-#1)^2)/(2*#2^2))}%
}

\maketitle
\newpage
\thispagestyle{empty}

%\fi

% POUR LES FIGURES : 
\def\myscale{.75} % par défaut 
\newcommand{\myfigure}[2]{  % entrée : echelle, fichier figure
\def\myscale{#1}\begin{center}\footnotesize{#2}\end{center}}
\newcommand{\E}{(-4,-1) rectangle (4,4)}
\newcommand{\A}{(0,0) ++(135:2) circle (1.9)}
\newcommand{\B}{(0,0) ++(45:2) circle (1.9)}
\definecolor{myred}{rgb}{0,0,0}
\definecolor{myorange}{rgb}{0,0,0}

\setcounter{tocdepth}{4}
\tableofcontents

\newpage

\chapter{Raisonnement par récurrence}

\vspace*{-1cm}

\section{Mot clef}

Dans un énoncé de bac, il est fréquent que l'on note « démontrer par récurrence » lorsque l'on attend une démonstration par récurrence. Disons que, très souvent, lorsque l'on dit « démontrer que, \underline{pour tout entier naturel} », on devra utiliser un raisonnement par récurrence.

\vspace*{-.3cm}

\section{Exercices pour débuter}

Montrer par récurrence sur $n$ les assertions suivantes : \\

\begin{itemize}
\item[•] Pour tout $n \geqslant 1$, $3^n \geqslant 1 + 2n$. \\

\item[•] $4^n + 6n - 1$ est divisible par $9$. \\

\item[•] Soit la suite $\left(u_n\right)$ définie par $\left\{
  \begin{array}{l}
    u_0 = 1,5 \\
    \forall n \in \N, \dfrac{6}{4 - u_n} - 1 \\
  \end{array}
\right.$. \\

Montrer que, pour tout $n \in \N$, $1 \leqslant u_n \leqslant 2$. \\

\item[•] Soit la suite $\left(u_n\right)$ définie par $\left\{
  \begin{array}{l}
    u_0 = 1,5 \\
    \forall n \in \N, \dfrac{4u_n - 2}{u_n + 1 }\\
  \end{array}
\right.$. \\

Montrer que, pour tout $n \in \N$, $1 \leqslant u_n \leqslant 2$. 
\end{itemize}

\section{Hérédité vraie d'une propriété fausse}

On veut calculer plus simplement la somme des entiers naturels de $1$ à $n$. On conjecture la propriété suivante : \\

$1 + 2 + 3 + \ldots + n = \dfrac{1}{2}\left(n + \dfrac{1}{2}\right)^2$. \\

Cette conjecture est fausse. Montrer que l'implication $\left(P_n\right) \Longrightarrow \left(P_{n+1}\right)$ est cependant vérifiée. 

\vspace*{-50cm}

\newpage

\section{Quelques résultats fondamentaux}

Montrer les égalités suivantes par récurrence sur $n$ : \\

\begin{itemize}
\item[•] Pour tout $n \geqslant 1$, $n! \geqslant 2^{n-1}$ (cf remarque). \\

\item[•] Pour tout $n \geqslant 1$, $\somme{k=1}{n}{k} := 1 + 2 + 3 + \ldots + n = \dfrac{n\left(n+1\right)}{2}$. \\

\item[•] Pour tout $n \geqslant 1$, $\somme{k=1}{n}{k^2} = \dfrac{n\left(n+1\right)\left(2n+1\right)}{6}$. \\

\item[•] Pour tout $n \geqslant 1$, $\somme{k=1}{n}{k^3} = \left(\dfrac{n\left(n+1\right)}{2}\right)^2$. \\

\item[•] (Inégalité de Bernoulli) Soit $\alpha \in \R_+^*$. Démontrer que $\left(1+\alpha\right)^n \geqslant 1 + n\alpha$.
\end{itemize}







\textbf{Remarque :} On rappelle que $n! = \produit{k=1}{n}{k} = 1 \times 2 \times \ldots \times n$, et que, par convention, $0! = 1$. 

\section{Récurrence plus difficiles}

\subsection{Exercice \no 1}

On pose $A_n = \somme{k=1}{2n}{\dfrac{\left(-1\right)^{k+1}}{k}} = 1 - \dfrac{1}{2} + \dfrac{1}{3} - \dfrac{1}{4} + \ldots + \dfrac{\left(-1\right)^{2n+1}}{2n}$. \\
On pose également $B_n = \somme{k=n+1}{2n}{\dfrac{1}{k}} = \dfrac{1}{n+1} + \dfrac{1}{n+2} + \ldots + \dfrac{1}{2n}$. \\

Montrer, par récurrence sur $n$, que pour tout $n \geqslant 1$, $A_n = B_n$.

\subsection{Exercice \no 2}

Soit $\left(u_n\right)_{n \in \N}$ la suite factorielle, définie par : $\left\{
  \begin{array}{l}
    u_0 = 1 \\
    \forall n \in \N, u_{n+1} = \left(n+1\right)u_n
  \end{array}
\right.$. \\

\textbf{Remarque :} Il suffit alors de noter $u_n = n!$ pour retrouver la définition de la factorielle, énoncée précédemment. \\

\begin{itemize}
\item[1.] Calculer les premiers termes de la suite $\left(u_n\right)$ pour conjecturer que $u_n = \somme{k=1}{n}{\dfrac{k}{\left(k+1\right)!}}$. \\

Démontrer, par récurrence sur $n$, cette conjecture. \\

\item[2.] On pose, pour tout entier naturel $n$, $S_n = \somme{k=1}{n}{k \times \left(k!\right)}$. \\

Montrer que, pour tout entier naturel $n$, $S_n = \left(n+1\right) - 1$.  
\end{itemize} 
 \newpage
\chapter{Fonction exponentielle}

\section{Équations avec des exponentielles}

Résoudre dans $\R$ les équations et inéquations suivantes : \\

\begin{itemize}
\item[1)] $e^{3x} = 1$. \\
\item[2)] $e^{3x} \geqslant 1$. \\
\item[3)] $e^{-x^2} = e^{2x+1}$. \\
\item[4)] $e^{-x^2 + x} \leqslant 1$. \\
\item[5)] $e^{x-3} \geqslant \dfrac{1}{e^x}$. \\
\item[6)] $e^{x^{-1}} \leqslant e^{x-2}$. \\
\item[7)] $e^{2x} + e^x - 2 = 0$. \\
\item[8)] $e^{2x} - \left(e+1\right)e^x + e = 0$. \\
\item[9)] $e^x + e^{-x} \geqslant 2$. \\
\item[10)] $e^x = -2$. \\
\item[11)] $e^x \geqslant -2$. \\
\item[12)] $\left\{
  \begin{array}{l}
   e^x + e^y = 5 \\
   2e^x - 3e^y = -5 \\
  \end{array}
\right.$.
\end{itemize}

\newpage

\section{Calculs avec des exponentielles}

Écrire les expressions suivantes sous la formes d'une exponentielle de base $e$ ou $2$, en fonction de $x$. On cherchera à se rapprocher de la forme la plus simple (c'est-à-dire dans la plupart des cas, la forme factorisée). \\

\textbf{N.B. :} Ne pas hésiter à mettre du détail ! Surtout quand on ne sait pas, écrire le détail de ce que l'on voit sur le coup permet de trouver des relations que l'on ne découvre qu'au fur et à mesure. \\

\begin{tabular}{ll}
\begin{minipage}{7.3cm}
\begin{itemize}
\item[•] $A = e^5 \times e^{-2} \times e^3$. \vspace*{.3cm} \\
\item[•] $B = \left(e^x\right)^3$. \vspace*{.3cm} \\
\item[•] $C = \dfrac{e^{x+2}}{e^2}$. \vspace*{.3cm} \\
\item[•] $D = e^x \times e$. \vspace*{.3cm} \\
\item[•] $E = \dfrac{e^{x+2}}{e^{-x}}$. \vspace*{.3cm} \\
\item[•] $F = \dfrac{1}{e^{1-x}} \times e^x$. \vspace*{.3cm} \\
\item[•] $G = \left(e^x\right)^3 \times e^{-4x}$. \vspace*{.3cm} \\
\item[•] $H = \dfrac{e^{2x} + e^x}{e^x}$. \vspace*{.3cm} \\
\item[•] $I = \left(e^x - e^{-x}\right)^2 - e^{-x}\left(e^{3x} + e^{-x}\right)$.\vspace*{.3cm} \\ 
\item[•] $J = \left(e^x + 1\right)^2 - \left(e^{-x} - 1\right)^2$. \vspace*{.3cm} \\
\item[•] $K = e^{2x} - e^x$. \vspace*{.3cm} \\
\item[•] $L = e^{2x} - 1$. \vspace*{.3cm} \\
\item[•] $M = 4e^{2x} + 4e^x + 1$. \vspace*{.3cm} \\
\end{itemize}
\end{minipage}
&
\begin{minipage}{8cm}
\begin{itemize}
\item[•] $N = xe^x - e^{3x}$. \vspace*{.3cm} \\
\item[•] $O = \dfrac{2^{x+3} - 8 \times 2^{x-2}}{6}$.\vspace*{.3cm} \\
\item[•] $P = 8 \times 2^x$. \vspace*{.3cm} \\
\item[•] $Q = 32 \times \left(2^x\right)^3$. \vspace*{.3cm} \\
\item[•] $R =  \dfrac{128^x}{64}$. \vspace*{.3cm} \\
\item[•] $S = e^x \times e^{-x}$. \vspace*{.3cm} \\
\item[•] $T = e^x + 3e^x$. \vspace*{.3cm} \\
\item[•] $U = \dfrac{e^{2x+1}}{e^{2-x}}$. \vspace*{.3cm}
\item[•] $V = \sqrt{2e^{3x+1}{e^{2x-1}}}$. \vspace*{.3cm} \\
\item[•] $W = \dfrac{\left(e^{x+1}\right)^2}{e^{2x}} \times \left(e^{2}\right)^{-1}$. \vspace*{.3cm} \\
\item[•] $X = e^{e^x} \times e^{2e^{x}} \times e^{-3e^{x}} - 1$. \vspace*{.3cm} \\
\item[•] $Y = \left(\dfrac{e}{e^{-x}}\right)^4$. \vspace*{.3cm} \\
\item[•] $Z = \left(e^{\pi x} + e^{-\pi x}\right)^2 - e^{2\pi x} - 2e^{\pi x}\times e^{-\pi x} - e^{-2\pi x}$. \vspace*{.3cm} \\
\end{itemize}
\end{minipage}
\end{tabular}

\section{Sens de variations}

Étudier les fonctions suivantes sur $\R$ (sens de variations, dérivées, limites, tableau de variations) : \\

\begin{itemize}
\item[•] $f : x \longmapsto \dfrac{e^x}{x}$ \\

\item[•] $g : x \longmapsto \left(2 - x\right)e^x$ \\

\item[•] $h : x \longmapsto 2e^x - e^{2x} + \dfrac{e^2}{4}$ 
\end{itemize}

\section{Limites}

Déterminer les limites aux bornes de leurs ensembles de définition des fonctions suivantes : \\

\begin{itemize}
\item[•] $f : x \longmapsto e^x - x$ \\
\item[•] $g : x \longmapsto \dfrac{e^x - 3}{e^x + 3}$ \\
\item[•] $h : x \longmapsto \dfrac{e^x - 1}{x^2}$ \\
\item[•] $i : x \longmapsto e^{2x} - e^x$ \\
\item[•] $j : \longmapsto xe^x - 1$ \\
\end{itemize}

 
\chapter{Continuité de fonctions à variable réelle}

\section*{Exercice \no 1}

On donne le tableau de variations suivant : \\

\variations
% \begin{variations}
x & \mI  & &3 & & 7 & & \pI \\
% \filet
f(x) & \b \mI & \c & \h{4} & \d & \b {-10} & \c & \h{-1} \\
% \end{variations}
\fin 

\vspace*{.3cm}

Déterminer, par lecture graphique, le nombre de solutions de l'équation $f\left(x\right) = 0$ sur $\R$. Placer alors ces solutions sur le tableau de variations (approximativement).

\section*{Exercice \no 2}

Soit la fonction $g$ définie sur $\R$ par : $g\left(x\right) = x^2$. \\ Soit la fonction $h$ définie sur $\left[-2 \; ; \; +\infty\right[$ par $h\left(x\right) = \sqrt{x + 2}$. \\

On donne à la fin des énoncés des exercices les représentations graphiques de ces fonctions.

\begin{itemize}
\item[1.] \textbf{Sans lecture graphique}, montrer que l'équation $f\left(x\right) = 0$ admet au moins une solutions sur $\left[-2 \; ; \; 0\right]$. \\
\item[2.] \textbf{Par lecture graphique}, montrer que $f\left(x\right) = 0$ admet une unique solutions sur $\left[-2 \; ; \; 0\right]$. Conjecturer graphiquement la valeur de cette solution. \\
\item[3.] Valider la conjecture de la question 2. par le calcul. 
\end{itemize}

\section*{Exercice \no 3}

Soit la fonction $f$ définie sur $\R$ par : $f(x) = x^3 + x^2 - 3$. \\

\begin{itemize}
\item[1.] Déterminer les variations de la fonction $f$. (on pourra dresser son tableau de variation) \\
\item[2.] Montrer que la fonction f est strictement négative pour tout réel $x\leqslant 0$. \\ 
\item[3.] Montrer que l'équation $f(x) = 0$ admet une unique solution $\alpha$, avec $\alpha \in \left[1 \; ; \; 2\right]$. \\
\item[4.] Grâce à la calculatrice, montrer que $\alpha \in \left[1,17 \; ; \; 1,18\right]$. 
\end{itemize}

\section*{Exercice \no 4} 

Soit $f$ une fonction. \\

On sait que : \\

\begin{itemize}
\item[•] $f$ est définie sur $\R$.
\item[•] $f$ est continue sur $\R$.
\item[•] $f$ est strictement croissante sur $\R$.
\item[•] $f(0) = 0$.
\item[•] $f(3) = 3$.
\end{itemize}

Combien de solutions réelles l'équation $\left(E\right) : f(x) = 1$ a-t-elle avoir sur $\left[0 \; ; \; 3\right]$ ?

\section*{Exercice \no 5}

Montrer que le polynôme $x^3 + 3x^2 - 5$ n'admet qu'une seule racine réelle. Donner un encadrement de la solution. 

\section*{Exercice \no 6}

Soit la fonction $f$ définie sur $\left[0 \; ; \; +\infty\right[$ par : $f\left(x\right) = \dfrac{3x+7}{x+1}$. \vspace*{.3cm}

\begin{itemize}
\item[1.] Déterminer les variations de $f$ sur $\R$. \\
\item[2.] Donner l'équation de la tangente $\left(T\right)$ à la courbe représentative de $f$ au point d'abscisse $1$. 
\end{itemize}

\newpage

\section*{Exercice \no 7 : Exercice Type Bac}

On considère une fonction $f$, dont on sait que :

\begin{itemize}
\item[•] elle est définie, continue et dérivable sur $\left[-1 \; ; \; +\infty\right[$. 
\item[•] elle est strictement croissante sur l'intervalle $\left[0 \; ; \; 2\right]$.
\item[•] elle est strictement décroissante sur les intervalles $\left[-1 \; ; \; 0\right]$ et $\left[2 \; ; \; +\infty\right[$.
\end{itemize}

\vspace*{.3cm}

On note $f'$ la fonction dérivée de $f$ sur $\left[-1 \; ; \; +\infty\right[$. \\

La courbe $C_f$ est la représentation graphique de la fonction $f$. Elle est tracé dans un repère orthogonal à la fin des énoncés des exercices. \\

La courbe $C_f$ passe par les points : $A\left(-1\; ; \; 6\right)$, $B\left(0 \; ; \; -2\right)$ et $E\left(2 \; ; \; 6\right)$. Elle admet au point $D$ une tangente passant par $G\left(0 \; ; \; -4\right)$. Enfin, la courbe $C_f$ admet deux tangentes horizontales, aux points $B$ et $E$. \\

\begin{itemize}
\item[1.] Déterminer $f'(1)$ et $f'(2)$, en justifiant les réponses. \\
\item[2.] Déterminer une équation de la tangente $\left(T\right)$ à la courbe $C_f$ au point $D$. \\
\item[3.] Montrer que sur l'intervalle $\left[-1 \; ; \; 0\right]$, l'équation $f(x) = 0$ n'admet qu'une unique solution, que l'on notera $x_1$ (par le calcul). \\
\item[4.] On admet que l'équation $f(x) = 0$ admet deux autres solutions sur l'intervalle $\left[-1 \; ; \; +\infty\right[$, que l'on notera $x_2$ et $x_3$, avec $x_2 < x_3$. \\ Dresser alors le tableau de signes de la fonction $f$. \\
\item[5.] Parmi les trois courbes tracées en annexe, préciser, en justifiant la réponse, laquelle représente $f'$. \\
\end{itemize}

\newpage

\chapter*{Données graphiques}

\section*{Exercice \no 2}

La fonction $g$ est représentée en rouge, et la fonction $h$ en bleu. \\

\definecolor{qqqqcc}{rgb}{0,0,0.8}
\definecolor{ffqqqq}{rgb}{1,0,0}
\definecolor{cqcqcq}{rgb}{0.75,0.75,0.75}
\begin{tikzpicture}[line cap=round,line join=round,>=triangle 45,x=1.0cm,y=1.0cm]
\draw [color=cqcqcq,dash pattern=on 1pt off 1pt, xstep=1.0cm,ystep=1.0cm] (-3.73,-1.01) grid (5.04,4.35);
\draw[->,color=black] (-3.73,0) -- (5.04,0);
\foreach \x in {-3,-2,-1,1,2,3,4,5}
\draw[shift={(\x,0)},color=black] (0pt,2pt) -- (0pt,-2pt) node[below] {\footnotesize $\x$};
\draw[->,color=black] (0,-1.01) -- (0,4.35);
\foreach \y in {-1,1,2,3,4}
\draw[shift={(0,\y)},color=black] (2pt,0pt) -- (-2pt,0pt) node[left] {\footnotesize $\y$};
\draw[color=black] (0pt,-10pt) node[right] {\footnotesize $0$};
\clip(-3.73,-1.01) rectangle (5.04,4.35);
\draw [samples=50,rotate around={0:(0,0)},xshift=0cm,yshift=0cm,color=ffqqqq,domain=-4.0:4.0)] plot (\x,{(\x)^2/2/0.5});
\draw[color=qqqqcc,smooth,samples=100,domain=-1.999994741782436:5.035548466391935] plot(\x,{sqrt((\x)+2)});
\end{tikzpicture}

\newpage

\section*{Exercice \no 7}

Représentation graphique de $f$ : \\

\definecolor{uuuuuu}{rgb}{0.27,0.27,0.27}
\definecolor{cqcqcq}{rgb}{0.75,0.75,0.75}
\begin{tikzpicture}[line cap=round,line join=round,>=triangle 45,x=2.5cm,y=1.0cm]
\draw [color=cqcqcq,dash pattern=on 2pt off 2pt, xstep=0.5cm,ystep=1.0cm] (-1.63,-5.34) grid (3.89,7.79);
\draw[->,color=black] (-1.63,0) -- (3.89,0);
\foreach \x in {-1.5,-1,-0.5,0.5,1,1.5,2,2.5,3,3.5}
\draw[shift={(\x,0)},color=black] (0pt,2pt) -- (0pt,-2pt) node[below] {\footnotesize $\x$};
\draw[->,color=black] (0,-5.34) -- (0,7.79);
\foreach \y in {-5,-4,-3,-2,-1,1,2,3,4,5,6,7}
\draw[shift={(0,\y)},color=black] (2pt,0pt) -- (-2pt,0pt) node[left] {\footnotesize $\y$};
\draw[color=black] (0pt,-10pt) node[right] {\footnotesize $0$};
\clip(-1.63,-5.34) rectangle (3.89,7.79);
\draw plot[raw gnuplot, id=func0] function{set samples 100; set xrange [-1.52:3.78]; plot -2*x**(3)+6*x**(2)-2};
\draw [samples=100,domain=-1.63:3.89] plot (\x,{-2*(\x)^3 + 6*(\x)^2 - 2});

\draw (2.52,4.69) node[anchor=north west] {\parbox{1.09 cm}{$(C_f \\ $}};
\draw (1.4,6.89) node[anchor=north west] {$(T)$};
\begin{scriptsize}
\draw[color=black] (-1.03,7.66) node {$f$};
\draw [fill=uuuuuu] (1,2) circle (1.5pt);
\draw[color=uuuuuu] (1.04,1.9) node {$D$};
\draw [fill=uuuuuu] (2,6) circle (1.5pt);
\draw[color=uuuuuu] (2.03,6.25) node {$E$};
\draw [fill=uuuuuu] (0,-2) circle (1.5pt);
\draw[color=uuuuuu] (0.04,-1.76) node {$B$};
\draw [fill=uuuuuu] (-1,6) circle (1.5pt);
\draw[color=uuuuuu] (-0.97,6.25) node {$A$};
\draw [fill=uuuuuu] (0,-4) circle (1.5pt);
\draw[color=uuuuuu] (0.06,-3.98) node {$G$};
\end{scriptsize}
\end{tikzpicture}

Proposition de réponses pour la question 5. : 

\begin{tabular}{lll}
\definecolor{gris_jj}{rgb}{0.85,0.85,0.85}
\begin{tikzpicture}[line cap=round,line join=round,>=triangle 45,x=1cm,y=0.5cm]
% \begin{pgfonlayer}{background}  
% Attention l'ordre de ces lignes est important 
% Ne pas le modifier   
\draw[step=1mm,ultra thin,AntiqueWhite!10](-1.6,-3.5) grid (3.5,8);
\draw[step=5mm,very thin,AntiqueWhite!30] (-1.6,-3.5) grid (3.5,8);
\draw[step=1cm,very thin,AntiqueWhite!50] (-1.6,-3.5) grid (3.5,8);
\draw[step=5cm,thin,AntiqueWhite]         (-1.6,-3.5) grid (3.5,8);
% \end{pgfonlayer} 
\draw [color=gris_jj,dash pattern=on 5pt off 5pt, xstep=1.0cm,ystep=1.0cm] (-1.6,-3.5) grid (3.5,8);
\draw[->] (-1.6,0) -- (3.54,0);
\foreach \x in {-1,1,2,3}
\draw[shift={(\x,0)},color=black] (0pt,2pt) -- (0pt,-2pt) node[below] {\footnotesize $\x$};
\draw[->] (0,-3.5) -- (0,8);
\foreach \y in {-3,-2,-1,1,2,3,4,5,6,7}
\draw[shift={(0,\y)},color=black] (2pt,0pt) -- (-2pt,0pt) node[left] {\footnotesize $\y$};
\draw (0pt,-10pt) node[right] {\footnotesize $0$};
\clip(-1.6,-3.5) rectangle (3.5,8);
\draw [samples=50,domain=-2.0:3.0)] plot (\x,{\x * (-6*\x +12) });
\draw (2.5,3) node  {\footnotesize $\mathcal{C}_1$};
\end{tikzpicture}

&
\definecolor{gris_jj}{rgb}{0.85,0.85,0.85}
\begin{tikzpicture}[line cap=round,line join=round,>=triangle 45,x=1cm,y=0.5cm]
% \begin{pgfonlayer}{background}  
% Attention l'ordre de ces lignes est important 
% Ne pas le modifier   
\draw[step=1mm,ultra thin,AntiqueWhite!10](-1.6,-3.5) grid (4.5,8);
\draw[step=5mm,very thin,AntiqueWhite!30] (-1.6,-3.5) grid (4.5,8);
\draw[step=1cm,very thin,AntiqueWhite!50] (-1.6,-3.5) grid (4.5,8);
\draw[step=5cm,thin,AntiqueWhite]         (-1.6,-3.5) grid (4.5,8);
% \end{pgfonlayer} 
\draw [color=gris_jj,dash pattern=on 5pt off 5pt, xstep=1.0cm,ystep=1.0cm] (-1.6,-3.5) grid (3.5,8);
\draw[->] (-1.6,0) -- (4.5,0);
\foreach \x in {-1,1,2,3}
\draw[shift={(\x,0)},color=black] (0pt,2pt) -- (0pt,-2pt) node[below] {\footnotesize $\x$};
\draw[->] (0,-3.5) -- (0,8);
\foreach \y in {-3,-2,-1,1,2,3,4,5,6,7}
\draw[shift={(0,\y)},color=black] (2pt,0pt) -- (-2pt,0pt) node[left] {\footnotesize $\y$};
\draw (0pt,-10pt) node[right] {\footnotesize $0$};
\clip(-1.6,-3.5) rectangle (4,8);
\draw [samples=50,domain=-2.0:4.0)] plot (\x,{-.45*\x*(\x+1)*(\x-1.2)*(\x-3.75)});
\draw (2.5,3) node  {\footnotesize $\mathcal{C}_2$};
\end{tikzpicture}

&

\definecolor{gris_jj}{rgb}{0.85,0.85,0.85}
\begin{tikzpicture}[line cap=round,line join=round,>=triangle 45,x=1cm,y=0.5cm]
% \begin{pgfonlayer}{background}  
% Attention l'ordre de ces lignes est important 
% Ne pas le modifier   
\draw[step=1mm,ultra thin,AntiqueWhite!10](-1.26,-6.92) grid (3.54,5.12);
\draw[step=5mm,very thin,AntiqueWhite!30] (-1.26,-6.92) grid (3.54,5.12);
\draw[step=1cm,very thin,AntiqueWhite!50] (-1.26,-6.92) grid (3.54,5.12);
\draw[step=5cm,thin,AntiqueWhite]         (-1.26,-6.92) grid (3.54,5.12);
% \end{pgfonlayer} 
\draw [color=gris_jj,dash pattern=on 5pt off 5pt, xstep=1.0cm,ystep=1.0cm] (-1.6,-6.92) grid (3.54,5.12);
\draw[->] (-1.6,0) -- (3.54,0);
\foreach \x in {-1,1,2,3}
\draw[shift={(\x,0)},color=black] (0pt,2pt) -- (0pt,-2pt) node[below] {\footnotesize $\x$};
\draw[->] (0,-6.92) -- (0,5.12);
\foreach \y in {-6,-5,-4,-3,-2,-1,1,2,3,4}
\draw[shift={(0,\y)},color=black] (2pt,0pt) -- (-2pt,0pt) node[left] {\footnotesize $\y$};
\draw (0pt,-10pt) node[right] {\footnotesize $0$};
\clip(-1.26,-6.92) rectangle (3.54,5.12);
\draw [samples=50,domain=-2.0:3.0)] plot (\x,{\x * (6*\x -12) });
\draw (2.5,3) node  {\footnotesize $\mathcal{C}_3$};
\end{tikzpicture} 

\end{tabular}

\vspace*{-50cm} 
\chapter{Limites}

\section{Limites simples}

Dans chaque cas, déterminer la limites de la suite $\left(u_n\right)$ \\ (\textbf{Remarque :} sous-entendu, en $+\infty$) : \\

a)\ \ $u_n=n^3+\dfrac{1}{n}$ 
\quad
b)\ \ $u_n=(3n+1)(-7n+5)$
\quad
c)\ \ $u_n=\dfrac{3-\dfrac{4}{n}}{\dfrac{2}{n^2}}$
\quad
d)\ \ $u_n=n^3-n^2+3n-1$

e)\ \ $u_n=\dfrac{2n^2+1}{-n^2+6}$
\quad
f)\ \ $u_n=\dfrac{n^2+3n-5}{n^3-6n^2+1}$
\quad
g)\ \ $u_n=n\sqrt{n}-n$
\\
h)\ \ $u_n=(-2n+3)\dfrac{n+3}{-n^2+n+6}$
\quad
i)\ \ $u_n=\dfrac{n}{n+\sqrt{n}}$ 
\quad
j)\ \ $u_n=\dfrac{9-n^2}{(n+1)(2n+1)}$
\\\
k)\ \ $u_n=\dfrac13-\dfrac{n}{(2n+1)^2}$
\quad
l)\ \ $u_n=\dfrac{2}{3n}-\dfrac{2n^2+3}{3n^2+n+1}$

\section{Application du théorème des gendarme - exercice type Bac}

$(u_n)$ est une suite définie par $u_0=1$ et,
pour tout entier naturel $n$, $u_{n+1}=u_n+2n+3$. 

\begin{enumerate}
\item Etudier le sens de variation de $(u_n)$. 
\item Démontrer par récurrence que, 
  pour tout entier $n$, $u_n=(n+1)^2$. 
\item En déduire que, pour tout entier $n$, 
  $u_n\geqslant n^2$.
\item La suite $(u_n)$ est-elle minorée ? majorée ? 
  Justifier. 
\item Donner la limite de $(u_n)$. 
\end{enumerate} 

%\input{01_Trigo.tex} \newpage

\input{02_Calculs_fractionnaires.tex} \newpage

\input{03_Nombres_relatifs.tex} \newpage

\input{04_Calcul_litteral.tex} \newpage

\input{05_Equations_1er_degre.tex} \newpage

\input{06_Inequations.tex} \newpage

\input{07_Puissances.tex} \newpage

\input{09_Demontrer_en_geometrie.tex} \newpage

\input{08_Pythagore.tex} \newpage

\input{16_Proportionnalite.tex} \newpage

\input{11_Triangle_milieux_paralles_agrandissement.tex} \newpage

\input{10_Thales.tex} \newpage

\input{12_Statistique.tex} \newpage

\input{13_Temps_vitesse_distance.tex} \newpage

\input{14_Distances.tex} \newpage

\input{15_Geometrie_dans_l_espace.tex} \newpage

\end{document}