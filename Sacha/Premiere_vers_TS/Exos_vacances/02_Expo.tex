\chapter{Fonction exponentielle}

\section{Équations avec des exponentielles}

Résoudre dans $\R$ les équations et inéquations suivantes : \\

\begin{itemize}
\item[1)] $e^{3x} = 1$. \\
\item[2)] $e^{3x} \geqslant 1$. \\
\item[3)] $e^{-x^2} = e^{2x+1}$. \\
\item[4)] $e^{-x^2 + x} \leqslant 1$. \\
\item[5)] $e^{x-3} \geqslant \dfrac{1}{e^x}$. \\
\item[6)] $e^{x^{-1}} \leqslant e^{x-2}$. \\
\item[7)] $e^{2x} + e^x - 2 = 0$. \\
\item[8)] $e^{2x} - \left(e+1\right)e^x + e = 0$. \\
\item[9)] $e^x + e^{-x} \geqslant 2$. \\
\item[10)] $e^x = -2$. \\
\item[11)] $e^x \geqslant -2$. \\
\item[12)] $\left\{
  \begin{array}{l}
   e^x + e^y = 5 \\
   2e^x - 3e^y = -5 \\
  \end{array}
\right.$.
\end{itemize}

\newpage

\section{Calculs avec des exponentielles}

Écrire les expressions suivantes sous la formes d'une exponentielle de base $e$ ou $2$, en fonction de $x$. On cherchera à se rapprocher de la forme la plus simple (c'est-à-dire dans la plupart des cas, la forme factorisée). \\

\textbf{N.B. :} Ne pas hésiter à mettre du détail ! Surtout quand on ne sait pas, écrire le détail de ce que l'on voit sur le coup permet de trouver des relations que l'on ne découvre qu'au fur et à mesure. \\

\begin{tabular}{ll}
\begin{minipage}{7.3cm}
\begin{itemize}
\item[•] $A = e^5 \times e^{-2} \times e^3$. \vspace*{.3cm} \\
\item[•] $B = \left(e^x\right)^3$. \vspace*{.3cm} \\
\item[•] $C = \dfrac{e^{x+2}}{e^2}$. \vspace*{.3cm} \\
\item[•] $D = e^x \times e$. \vspace*{.3cm} \\
\item[•] $E = \dfrac{e^{x+2}}{e^{-x}}$. \vspace*{.3cm} \\
\item[•] $F = \dfrac{1}{e^{1-x}} \times e^x$. \vspace*{.3cm} \\
\item[•] $G = \left(e^x\right)^3 \times e^{-4x}$. \vspace*{.3cm} \\
\item[•] $H = \dfrac{e^{2x} + e^x}{e^x}$. \vspace*{.3cm} \\
\item[•] $I = \left(e^x - e^{-x}\right)^2 - e^{-x}\left(e^{3x} + e^{-x}\right)$.\vspace*{.3cm} \\ 
\item[•] $J = \left(e^x + 1\right)^2 - \left(e^{-x} - 1\right)^2$. \vspace*{.3cm} \\
\item[•] $K = e^{2x} - e^x$. \vspace*{.3cm} \\
\item[•] $L = e^{2x} - 1$. \vspace*{.3cm} \\
\item[•] $M = 4e^{2x} + 4e^x + 1$. \vspace*{.3cm} \\
\end{itemize}
\end{minipage}
&
\begin{minipage}{8cm}
\begin{itemize}
\item[•] $N = xe^x - e^{3x}$. \vspace*{.3cm} \\
\item[•] $O = \dfrac{2^{x+3} - 8 \times 2^{x-2}}{6}$.\vspace*{.3cm} \\
\item[•] $P = 8 \times 2^x$. \vspace*{.3cm} \\
\item[•] $Q = 32 \times \left(2^x\right)^3$. \vspace*{.3cm} \\
\item[•] $R =  \dfrac{128^x}{64}$. \vspace*{.3cm} \\
\item[•] $S = e^x \times e^{-x}$. \vspace*{.3cm} \\
\item[•] $T = e^x + 3e^x$. \vspace*{.3cm} \\
\item[•] $U = \dfrac{e^{2x+1}}{e^{2-x}}$. \vspace*{.3cm}
\item[•] $V = \sqrt{2e^{3x+1}{e^{2x-1}}}$. \vspace*{.3cm} \\
\item[•] $W = \dfrac{\left(e^{x+1}\right)^2}{e^{2x}} \times \left(e^{2}\right)^{-1}$. \vspace*{.3cm} \\
\item[•] $X = e^{e^x} \times e^{2e^{x}} \times e^{-3e^{x}} - 1$. \vspace*{.3cm} \\
\item[•] $Y = \left(\dfrac{e}{e^{-x}}\right)^4$. \vspace*{.3cm} \\
\item[•] $Z = \left(e^{\pi x} + e^{-\pi x}\right)^2 - e^{2\pi x} - 2e^{\pi x}\times e^{-\pi x} - e^{-2\pi x}$. \vspace*{.3cm} \\
\end{itemize}
\end{minipage}
\end{tabular}

\section{Sens de variations}

Étudier les fonctions suivantes sur $\R$ (sens de variations, dérivées, limites, tableau de variations) : \\

\begin{itemize}
\item[•] $f : x \longmapsto \dfrac{e^x}{x}$ \\

\item[•] $g : x \longmapsto \left(2 - x\right)e^x$ \\

\item[•] $h : x \longmapsto 2e^x - e^{2x} + \dfrac{e^2}{4}$ 
\end{itemize}

\section{Limites}

Déterminer les limites aux bornes de leurs ensembles de définition des fonctions suivantes : \\

\begin{itemize}
\item[•] $f : x \longmapsto e^x - x$ \\
\item[•] $g : x \longmapsto \dfrac{e^x - 3}{e^x + 3}$ \\
\item[•] $h : x \longmapsto \dfrac{e^x - 1}{x^2}$ \\
\item[•] $i : x \longmapsto e^{2x} - e^x$ \\
\item[•] $j : \longmapsto xe^x - 1$ \\
\end{itemize}

