\ifdefined\COMPLETE
\else
    \input{./preambule-sacha-utf8.ltx}
    \begin{document}
\fi

                 
                 \intitule{Inéquations du premier degré : cours} 
                 
\bigskip     

De manière générale :

\begin{itemize}
\item Une inéquation se résout comme une équation : 
     \begin{itemize}
     \item [$\hookrightarrow$] le but est toujours d'isoler $x$ ; 
     \item [$\hookrightarrow$] les formes sont les mêmes.
     \end{itemize}
\item Le $=$ devient $<$, $>$, $\leqslant$, $\geqslant$.     
\end{itemize}    

\bigskip 

\Asavoir{Une règle supplémentaire}     


\medskip 

Si on \Asavoir{multiplie} par un nombre \TextSoulign{red}{négatif}, on change le signe de l'inéquation (\vocabulaire{le sens}).
   

\medskip 

\begin{description}
\item $<$ devient $>$
\item $>$ devient $<$
\item $\leqslant$ devient $\geqslant$
\item $\geqslant$ devient $\leqslant$
\end{description}             

\bigskip         
%                 \intitule{Inéquations du premier degré} 
                 
                 \centerline{\intitule{Exemples rédigés...}}   
                 
\bigskip   

On rédigera ici 2 exemples, pour montrer le changement de sens lorsque l'on multiplie par un nombre négatif.

Pour la technique de résolution, consulter les exemples rédigés de la fiche sur les équations.

\setbox1=\vtop{ \hsize=5cm \null % null assure l'alignement par le haut 
\begin{tikzpicture}% [every node/.style={anchor=west}]
  \matrix (m) [matrix of math nodes,
row sep=0cm,column sep=0cm,  
]{
2x -5            &\leqslant& |(a)| 0  \\%  
2x               &\leqslant& |(b)|5 \\
x                &\leqslant& |(c)| \dfrac{5}{2} \\
};  
% \node [right = 1cm of a.west] (k) {} ; 
% \node [right = 1cm of i.west] (j) {} ; 
\draw[color=blue,->,>=latex] (a) to[out=0,in=0] node[midway,right]{ On ajoute 5} (b); 
\draw[color=blue,->,>=latex] (b) to[out=0,in=0] node[midway,right]{ On divise par 2} (c);
\end{tikzpicture}

\vocabulaire{$2>0$, donc pas de changement de sens}}

\setbox2=\vtop { \hsize=5cm  \null
\begin{tikzpicture}% [every node/.style={anchor=west}]
  \matrix (m) [matrix of math nodes,
row sep=0cm,column sep=0cm,  
]{
-3x +7           &\leqslant& |(a)| 0  \\%  
-3x               &\leqslant& |(b)| -7\\
x                &\textcolor{red}{\geqslant}& |(c)| \dfrac{-7}{-3} \\
x                &\geqslant& |(d)| \dfrac{7}{3} \\
};  
% \node [right = 1cm of a.west] (k) {} ; 
% \node [right = 1cm of i.west] (j) {} ; 
\draw[color=blue,->,>=latex] (a) to[out=0,in=0] node[midway,right]{ On ajoute -7} (b); 
\draw[color=blue,->,>=latex] (b) to[out=0,in=0] node[midway,right]{ On divise par -3} (c);
\draw[color=blue,->,>=latex] (c) to[out=0,in=0] node[midway,right]{ On simplifie} (d);
\end{tikzpicture}
\vocabulaire{$-3<0$, donc changement de sens}}  


\centerline{ \box1 \quad\vrule\quad \box2 \hfill}



\ifdefined\COMPLETE
\else
    \end{document}
\fi