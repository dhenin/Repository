\ifdefined\COMPLETE
\else
    \input{./preambule-sacha-utf8.ltx}
    \begin{document}
\fi

                 \intitule{Vitesse, distance et temps : cours} 
                 
\Asavoir{
\begin{tabular}{l@{\hspace*{3cm}}l}
\underline{\Large Formules à retenir} & \begin{tabular}{l}
	$ v \longrightarrow$  vitesse \\
	$ d \longrightarrow$ distance \\
	$ t \longrightarrow$ temps \\
\end{tabular} \\
  & \\
 & $v = \dfrac{d}{t}$ \\
\end{tabular} 
}

\begin{labeling}{On en déduit que :} 
\item [On en déduit que :] \Asavoir{$d = v \times t$}
\item [et :] \Asavoir {$t = \dfrac{d}{v}$} 
\end{labeling}    

\medskip

Dans ces formules, le temps est exprimé sous forme \Asavoir{décimale}.

\begin{labeling}[$\bullet$]{Remarques : }
\item [\underline{Remarques} : ] On utilise $v=\dfrac{d}{t}$ si on cherche la vitesse.   
\item [] On utilise $d=vt$ si on cherche la distance.   
\item [] On utilise $t=\dfrac{d}{v}$ si on cherche la temps.   
\end{labeling}

\bigskip
                   
%                 \intitule{Vitesse, distance et temps} 

                 \centerline{\intitule{Exemples rédigés de trois types}}   

\bigskip                     
                 
\Asavoir{\Large 1) On cherche la vitesse}    

\medskip  

Une personne roule à la même vitesse pendant 4 heures et parcourt 250 km.\\
À quelle \underline{vitesse} roule-t-elle ? 

\begin{labeling}{On sait que}
\item [On sait que] $ d = 250$ km et que $t=4$ h,
\item [or] $v=\dfrac{d}{t}$, 
\item [donc] $v=\dfrac{d}{t} = \dfrac{250}{4}=62,5$.
\end{labeling}

La personne roule à $62,5$km/h 

\bigskip

\Asavoir{\Large 2) On cherche une distance}    

\medskip  

Une personne roule à 120 km/h pendant 6 heures. Quelle \underline{distance} aura-t-elle parcourue ?  

\begin{labeling}{On sait que}
\item [On sait que] $ v = 120$ km/h et que $t=6$ h,
\item [or] $d=v\times t$, 
\item [donc] $d=vt=120\times6=720$.
\end{labeling}

La personne aura parcourue $720$ km
   
\bigskip

\Asavoir{\Large 3) On cherche une durée}    

\medskip 

Une personne parcourt 480 km à la vitesse de 120 km/h. Combien de \underline{temps} cette personne a-t-elle roulé  ?  

\begin{labeling}{On sait que}
\item [On sait que] $ d = 480$ km et que $v = 120$ km/h, 
\item [or] $t=\dfrac{d}{v}$, 
\item [donc] $t=\dfrac{d}{v} = \dfrac{480}{120} = 4$.
\end{labeling}

La personne aura roulé 4 heures.


\ifdefined\COMPLETE
\else
    \end{document}
\fi