\ifdefined\COMPLETE
\else
    \input{./preambule-sacha-utf8.ltx}
    \begin{document}
\fi
\intitule{Statistique : cours}

\begin{itemize}
\item Le symbole $\somme$ signifie que l'on doit sommer.
\item On appelle $x_i$ les valeurs que peut prendre la chose que l'on étudie \vocabulaire{(le caractère)}.
\item On appelle $n_i$ le nombre de fois où apparaît une valeur \vocabulaire{(les effectifs)}.
\end{itemize}

On appelle \Asavoir{moyenne} le nombre \Asavoir{$\overline{x}=\dfrac{\somme n_i x_i}{\somme n_i}$}

\bigskip 

\bigskip 

\intitule{Statistique : exemple rédigé}

\begin{labeling}{Question}
\item [\underline{Énoncé :}] 
Les résultats au dernier contrôle de mathématiques de la classe de $4^{e}C$ sont : 

\centerline{\begin{tabular}{rrrrrrrrrr}
6 & 2 & 10 & 10 & 18 & 10 & 14 & 10 & 6 & 14 \\
14 & 14 & 2 & 6 & 14 & 10 & 6 & 14 & 14 & 6 \\
14 & 6 & 10 &  10 & 6 & 10 & 14 & 10 &  \\
\end{tabular}}

\item [\underline{Question :}] Déterminer la moyenne des notes au dernier contrôle de mathématiques. 
\end{labeling}

\begin{labeling}{Etape 1}
\item [\methode{\underline{Étape 1 :}}] \methode{On met les données dans un tableau}

\medskip
\centerline{\begin{tabular}{|l|c|c|c|c|c|}
\hline
Note obtenue & 2 & 6 & 10 & 14 & 18 \\
\hline
\multirow{3}{3cm}{Nombre d'élèves ayant obtenue cette note} &   &  &   &   &   \\
 & 2 & 7 & 9 & 9 & 1 \\
  &   &  &   &   &   \\
\hline
\end{tabular}}

\item [\methode{\underline{Étape 2 :}}] \methode{On identifie « qui sont » les $x_i$ et les $n_i$ ? }\\
On étudie des notes, les $x_i$ sont donc les notes obtenues et les $n_i$ les nombres d'élèves.

\item [\methode{\underline{Étape 3 :}}] \methode{On fait évoluer le tableau en le complétant de la manière suivante}\\(la partie supplémentaire est sur fond  \textcolor{blue}{\framebox{blue}}% \textcolor{blue}{bleu)}

\medskip

\centerline{\setlength\doublerulesep{1pt}
 \doublerulesepcolor{blue}
 \begin{tabular}{|c|c|c|c|c|c||>{\columncolor{blue!20}\color{black}\bfseries}c||}
\hline
$x_i$ & 2 & 6 & 10 & 14 & 18 & $\somme$ \\
\hline
$n_i$ & 2 & 7 & 9 & 9 & 1 & 28\\
 \arrayrulecolor{blue}
\hline
\arrayrulecolor{blue} 
 \rowcolor{blue!20}$x_i n_i$ & 4 & 42 & 90 & 126 & 18 & 280 \\
\hline
 \arrayrulecolor{blue}
\end{tabular}}


\item [\methode{\underline{Étape 4 :}}] \methode{On calcule $\overline{x}$ (la moyenne)}

D'après le cours : $\overline{x} = \dfrac{\somme n_i x_i}{\somme n_i}$.

On lit $\somme n_i x_i$ et $\somme n_i$ dans le tableau.

On obtient  $\overline{x} = \dfrac{\somme n_i x_i}{\somme n_i} = \dfrac {280}{28} = 10$.
Donc la moyenne des notes des élèves est de $10/20$.

\end{labeling}



\ifdefined\COMPLETE
\else
    \end{document}
\fi