\ifdefined\COMPLETE
\else
    \input{./preambule-sacha-utf8.ltx}
    \begin{document}
\fi

             
                 \intitule{Droite des milieux : cours } 

\Asavoir{{\large I)} Théorème de parallélisme}

\setbox1=\vtop { \hsize=5cm  \null
\begin{tikzpicture}[line cap=round,line join=round,>=triangle 45,x=1.0cm,y=1.0cm,scale=.6, rotate=0]
\begin{scriptsize}
% \draw (0.,0.) node [above ] {$A$} -- (2.18,1.) node [right] {$M$} ; 
\draw   (0,0) node [below] {$B$} -- node[midway,rotate=50] {$\vert\vert$}
        (3,2) node [left]  {$M$} -- node[midway,rotate=50] {$\vert\vert$} 
        (6,4) node [above] {$A$} -- node[midway,rotate=-20] {\textcolor{red}{ $-$}}
        (5,2) node [right] {$N$} -- node[midway,rotate=-20] {\textcolor{red}{ $-$}}
        (4,0) node [below] {$C$} -- cycle ; 
\draw (3,2) -- (5,2) ;         
\end{scriptsize}        
\end{tikzpicture} }


\begin{tabular}{M{10cm}M{10cm}}
\parbox{10cm}{
   Dans un triangle, si \TextSoulign {blue} {une droite passe par les milieux de deux côtés}, alors elle est parallèle au troisième.
}
 & \box1 
\end{tabular} 

\methode{On utilise ce théorème pour montrer que \underline{deux droites sont parallèles}}

\bigskip 

\Asavoir{{\large II)} Théorème du milieu}

\setbox1=\vtop { \hsize=5cm  \null
\begin{tikzpicture}[line cap=round,line join=round,>=triangle 45,x=1.0cm,y=1.0cm,scale=.6, rotate=0]
\begin{scriptsize}
% \draw (0.,0.) node [above ] {$A$} -- (2.18,1.) node [right] {$M$} ; 
\draw   (0,0) node [below] {$B$} -- node[midway,rotate=50] {\textcolor{red}{$\vert\vert$}}
        (3,2) node [left]  {$M$} -- node[midway,rotate=50] {\textcolor{red}{$\vert\vert$}}
        (6,4) node [above] {$A$} -- 
        (5,2) node [right] {$N$} -- 
        (4,0) node [below] {$C$} -- cycle ; 
\draw (3,2) -- (5,2) ;         
\end{scriptsize}        
\end{tikzpicture} }


\begin{tabular}{M{10cm}M{10cm}}
\parbox{10cm}{
   Dans un triangle, si \TextSoulign {blue} {une droite passe par le milieu d'un côté}, 
   et est \TextSoulign {blue} {parallèle à un second côté}, 
   alors elle coupe le troisième côté en son milieu.
}
 & \box1 
\end{tabular} 

\methode{On utilise ce théorème pour montrer qu'un point est \underline{le milieu d'un segment}}.


\bigskip 

\Asavoir{{\large II)} Théorème des longueurs}

\setbox1=\vtop { \hsize=5cm  \null
\begin{tikzpicture}[line cap=round,line join=round,>=triangle 45,x=1.0cm,y=1.0cm,scale=.6, rotate=0]
\begin{scriptsize}
% \draw (0.,0.) node [above ] {$A$} -- (2.18,1.) node [right] {$M$} ; 
\draw   (0,0) node [below] {$B$} -- node[midway,rotate=50] {$\vert\vert$}
        (3,2) node [left]  {$M$} -- node[midway,rotate=50]  {$\vert\vert$}
        (6,4) node [above] {$A$} -- node[midway,rotate=-20]  {$-$}
        (5,2) node [right] {$N$} -- node[midway,rotate=-20]  {$-$}
        (4,0) node [below] {$C$} -- node[midway,below]  {$4$ cm} cycle ; 
\draw (3,2) -- (5,2) node[midway,below]  {$2$ cm} ;         
\end{scriptsize}        
\end{tikzpicture} }


\begin{tabular}{M{10cm}M{10cm}}
\parbox{10cm}{
   Dans un triangle, la \TextSoulign {blue} {longueur du segment joignant les milieux} de deux côtés est  
   \TextSoulign {blue} {égale à la moitié de celle du troisième côté}. 
 
}
 & \box1 
\end{tabular} 

\methode{On utilise ce théorème pour \underline{calculer une longueur}}.


\newpage        
 %                 \intitule{Droite des milieux} 
                 
                 \centerline{\intitule{Deux exemples rédigés}} 
\setbox1=\vtop { \hsize=5cm  \null
\begin{tikzpicture}[line cap=round,line join=round,>=triangle 45,x=1.0cm,y=1.0cm,scale=.7, rotate=150]
\begin{scriptsize}
\draw   (0,0) node [below] {$C$} -- node[midway,rotate=50] {$\vert\vert$}
        (1.5,2) node [below ] {$F$} -- node[midway,rotate=50]  {$\vert\vert$}
        (3,4) node [below] {$B$} -- node[midway,rotate=-0]  {$-$}
        (4,2) node [below left] {$E$} -- node[midway,rotate=-0]  {$-$}
        (5,0) node [above] {$A$} --  cycle ; 
\draw (.5,2) -- (5,2)  ;  
\draw   (2,0) node [above right] {$D$} --  (3,4)   ;    
\draw (2.5, 2) node   [above] {$G$}  ;      
\end{scriptsize}        
\end{tikzpicture} }                 
                 

\begin{tabular}{M{9cm}M{9cm}}
\parbox{10cm}{
\underline{Exemple 1 : Parallèle et milieu}\\
\\
Soit $ABC$ un triangle et $D$ un point du segment $[AC]$. \\
Soient $E$ et $F$ les \TextSoulign{blue}{milieux} respectifs de $[AB]$ et $[BC]$. \\
Soit $G$ le point d'intersection de $(EF)$ et $(BD)$.  
}
 & \box1 
\end{tabular} 

\underline{a) Démontrer que $(EF)$ et $(AC)$ sont parallèles}

\medskip 

On se place dans le triangle $ABC$.

\addtokomafont{labelinglabel}{\textnormal}

\begin{labeling}{On sait que } 
\item [On sait que ] $[AB)$ et $[BC)$ sont \TextSoulign{blue}{deux demi-droites} d'origine $B$,\\
           les points $B, E$ et $A$ et $B, F$ et $C$ sont  \TextSoulign{blue}{alignés dans cet ordre}, \\
            et que $E$ et $F$ sont les \TextSoulign{blue}{milieux} de $[AB]$ et $[BC]$. 
\item [Or, ] si, dans un triangle, une droite passe par les milieux  de deux côtés, alors elle est parallèle au troisième. 
\item [Donc] $(EF)$ et $(AC$) sont parallèles.             
\end{labeling}

\bigskip 

\underline{b) Démontrer que $G$ est le milieu de $[BD]$}

\begin{labeling}{On sait que } 
\item [On sait que ]  $(EF)$ est \TextSoulign{blue}{parallèle} à $(AC)$, \\
                      $G$ est un point de $[EF]$ et $D$ appartient à $[AC]$.\\
\item [Ainsi] $(FG)$ est parallèle à $(CD)$.            
\end{labeling}

On se place dans le triangle $BCD$.

\begin{labeling}{On sait que } 
\item [On sait que ] $[BC)$ et $[BD)$ sont \TextSoulign{blue}{deux demi-droites} d'origine $B$,\\ 
           les points $B, G, D$ et $B, F, C$ sont  \TextSoulign{blue}{alignés dans cet ordre},\\
$(FG)$ est \TextSoulign{blue}{parallèle} $(CD)$ et que $F$ est le \TextSoulign{blue}{milieu} de $[BC]$, 
\item [Or, ] si, dans un triangle, une droite passe par le milieu d'un côté, 
   et est parallèle à un second côté, alors elle coupe le troisième côté en son milieu.
\item [Donc] $G$ est le milieu de $[BD]$. 
\end{labeling}

\setbox1=\vtop { \hsize=5cm  \null
\begin{tikzpicture}[line cap=round,line join=round,>=triangle 45,x=1.0cm,y=1.0cm,scale=1, rotate=0]
\begin{scriptsize}
\draw   (0,0)   node [below] {$B$} -- node[midway,rotate=50] {$\vert$}
        (1.5,2) node [left] {$E$} -- node[midway,rotate=50]  {$\vert$}
        (3,4)   node [above] {$C$} -- node[midway,rotate=-0] (M) {$\circ$}
        (4,2)   node [right] {$D$} -- node[midway,rotate=-0]  {$\circ$}
        (5,0)   node [below] {$A$} --  node[midway,below ] {$5,4$cm} cycle ; 
\draw (0,0) --  (4,2)  ;  
\draw (1.5,2)  --  (4,2)   ;    
\draw (1.5,2)--  node[midway,above,rotate=30] {$1,9$cm} (M.center) ;
\draw  (3,4) --   node[midway,rotate=-90] {$\vert\vert$}   (M.center) ;
\draw  (M.center) --   node[midway,rotate=-90] {$\vert\vert$} (4,2)  ;
\end{scriptsize}        
\end{tikzpicture} }                 
                 
\newpage 

\begin{tabular}{M{9cm}M{9cm}}
\parbox{10cm}{
\underline{Exemple 2 : Parallèle et milieu}\\
\\
Soit $ABC$ un triangle et $D$ un point du segment $[AC]$. \\
Soient $E$ et $F$ les \TextSoulign{blue}{milieux} respectifs de $[AB]$ et $[BC]$. \\
Soit $G$ le point d'intersection de $(EF)$ et $(BD)$.  
}
 & \box1 
\end{tabular} 

\underline{a) Calculer la longueur $BD$}

\bigskip 

On se place dans le triangle $ABD$.\\

\begin{labeling}{On sait que } 
\item [On sait que ] $E$ est le  \TextSoulign{blue}{milieu} de $[AB]$ \\
                  et $F$ est le  \TextSoulign{blue}{milieu} de $[AD]$
\item [Or, ] Dans un triangle, la longueur du segment joignant les milieux  de deux côtés est  
    égale à la moitié de celle du troisième côté. 
\item [Donc] $EF=\dfrac{1}{2}\times BD$, donc $BD= 2 \times EF$.
\item [Ainsi] $BD= 2 \times 1,9$.
\end{labeling}

On en conclut que $BD = 3,8$ cm.

\bigskip 

\underline{b) Calculer la longueur $ED$}

\bigskip 
On se place dans le triangle $ABC$.\\

\begin{labeling}{On sait que } 
\item [On sait que ] $E$ est le  \TextSoulign{blue}{milieu} de $[AB]$ \\
                  et $D$ est le  \TextSoulign{blue}{milieu} de $[AC]$
\item [Or, ] Dans un triangle, la longueur du segment joignant les milieux  de deux côtés est  
    égale à la moitié de celle du troisième côté. 
\item [Donc] $ED=\dfrac{1}{2}\times BC$, donc $ED= \dfrac{1}{2} \times 5,4$.
\end{labeling}

On en conclut que $ED = 2,7$ cm.

\newpage        



\ifdefined\COMPLETE
\else
    \end{document}
\fi