\ifdefined\COMPLETE
\else
    \input{./preambule-sacha-utf8.ltx}
    \begin{document}
\fi

                 
                 \intitule{Trigonométrie : cours} 
                 


\bigskip   


\setbox1=\hbox { \hsize=5cm  \null
\begin{tikzpicture}[line cap=round,line join=round,>=triangle 45,x=1.0cm,y=1.0cm, scale=.75]
\begin{scriptsize}
\draw [color=blue, domain=-1.:5.5] plot(\x,{(-0.--3.*\x)/4.});
\draw [color=blue, domain=-1.8:5.5] plot(\x,{(-0.-0.*\x)/4.});
\draw (0,0) node [below] {$O$} -- (4,0) node[below]{$B$} -- (4,3) node[above left]{$A$} -- (0,0) ;
\draw[color=darkgreen,fill={darkgreen!75},fill opacity=0.1] (4.,0.5) -- (3.5,0.5) -- (3.5,0.) -- (4.,0.) -- cycle; 
\draw (0.7,0.3) node[right] {$\alpha$};
\draw [shift={(0.,0.)},color=darkgreen,fill={darkgreen!75},fill opacity=0.1] 
      (0,0) -- (0.:0.6) arc (0.:36.85:0.6) -- cycle;
\end{scriptsize}
\end{tikzpicture} }

\newcolumntype{M}[1]{>{\raggedright}m{#1}}

\begin{tabular}{M{7cm}M{6cm}}
\Asavoir{\Large $ \bullet \quad  0 \leqslant \cos (x)  \leqslant 1$}    \medskip 
\Asavoir{\Large $\bullet \quad  \cos(x) = \tfrac{\text{adjacent}}{\text{hypothénuse}} = \dfrac{OB}{OA}$ }   

 & \box1                    

\end{tabular} 
                    
$\hookrightarrow$ Par définition, \Asavoir{toujours} dans un triangle rectangle.                                    

\bigskip

\bigskip
                   
% \intitule{Trigonométrie} 

                 \centerline{\intitule{Exemples rédigés de quatre types}}   

\bigskip                          
                 
\Asavoir{\Large \underline{Exemple 1 : } On cherche le cosinus}, ici de $\widehat{AOB}$.    

\setbox1=\hbox { \hsize=5cm  \null
\begin{tikzpicture}[line cap=round,line join=round,>=triangle 45,x=1.0cm,y=1.0cm, scale=.75]
\begin{scriptsize}
\draw [color=blue, domain=-1.:5.5] plot(\x,{(-0.--3.*\x)/4.});
\draw [color=blue, domain=-1.8:5.5] plot(\x,{(-0.-0.*\x)/4.});
\draw (0,0) node [below] {$O$} -- node[midway,below]{4} (4,0) node[below]{$B$} -- 
      (4,3) node[above left]{$A$} -- node[midway,above]{5}(0,0) ;
\draw[color=darkgreen,fill={darkgreen!75},fill opacity=0.1] (4.,0.5) -- (3.5,0.5) -- (3.5,0.) -- (4.,0.) -- cycle; 
\draw (0.7,0.3) node[right] {$\alpha$};
\draw [shift={(0.,0.)},color=darkgreen,fill={darkgreen!75},fill opacity=0.1] 
      (0,0) -- (0.:0.6) arc (0.:36.85:0.6) -- cycle;
\end{scriptsize}
\end{tikzpicture} }

\begin{tabular}{M{1.9cm}M{6cm}r}
\methode{On a besoin de 2 côtés} & 
\copy1 & \vocabulaire{On applique le cours} \\
\end{tabular}

\begin{labeling}{On sait que}
\item [On sait que] le triangle $AOB$ est rectangle en $B$,
\item [or] $ \cos(\alpha) = \tfrac{\text{adjacent}}{\text{hypothénuse}}$, 
\item [donc] $\cos(\widehat{AOB}) = \dfrac{OB}{OA} = \dfrac{4}{5}$.
\end{labeling}

\bigskip                     
                 
\Asavoir{\Large \underline{Exemple 2 : } On cherche un angle}, ici l'angle $\widehat{AOB}$    

\begin{tabular}{M{2.5cm}M{6cm}M{6cm}}
\methode{On a besoin \\du cosinus \\de l'angle $\widehat{AOB}$ } & 

\box1
                  & \vocabulaire{$\bullet$ On fait comme l'exemple 1\\
                                 $\bullet$ On utilise la calculatrice} \\
\end{tabular}

\begin{labeling}{On sait que}
\item [On sait que] le triangle $AOB$ est rectangle en $B$,
\item [or] $ \cos(\alpha) = \tfrac{\text{adjacent}}{\text{hypothénuse}}$, 
\item [donc] $\cos(\widehat{AOB}) = \dfrac{OB}{OA} = \dfrac{4}{5}$,
\item []     $\widehat{AOB} = \arccos \left(\dfrac{4}{5}\right)$,
\item[] D'après la calculatrice $\widehat{AOB} \approx 36,9$\degre.
\end{labeling}

\newpage

\Asavoir{\Large \underline{Exemple 3 : } On cherche l'hypoténuse}, ici la mesure du segment $[OA]$  


\setbox1=\hbox { \hsize=5cm  \null
\begin{tikzpicture}[line cap=round,line join=round,>=triangle 45,x=1.0cm,y=1.0cm, scale=.75]
\begin{scriptsize}
\draw [color=blue, domain=-1.:5.5] plot(\x,{(-0.--3.*\x)/4.});
\draw [color=blue, domain=-1.8:5.5] plot(\x,{(-0.-0.*\x)/4.});
\draw (0,0) node [below] {$O$} -- node[midway,below]{4} (4,0) node[below]{$B$} -- 
      (4,3) node[above left]{$A$} -- node[midway,above]{\normalsize x}(0,0) ;
\draw[color=darkgreen,fill={darkgreen!75},fill opacity=0.1] (4.,0.5) -- (3.5,0.5) -- (3.5,0.) -- (4.,0.) -- cycle; 
\draw (0.7,0.3) node[right] {$30$ \degre};
\draw [shift={(0.,0.)},color=darkgreen,fill={darkgreen!75},fill opacity=0.1] 
      (0,0) -- (0.:0.6) arc (0.:36.85:0.6) -- cycle;
\end{scriptsize}
\end{tikzpicture} }  

% \newcolumntype{M}[1]{>{\raggedright}m{#1}}

\begin{tabular}{M{4cm}M{6cm}M{6cm}}
\methode{On a besoin du côté \\adjacent et\\ du cosinus de l'angle } & 
\box1
                  & \vocabulaire{$\bullet$ On applique le cours\\
                                 $\bullet$ On résout l'équation\\
                                 $\bullet$ On utilise la calculatrice} \\
\end{tabular}

\begin{labeling}{D'après la calculatrice : }
\item [On sait que] le triangle $AOB$ est rectangle en $B$,
\item [or] $ \cos(\alpha) = \tfrac{\text{adjacent}}{\text{hypothénuse}}$, 
\item [donc] $\cos(\widehat{AOB}) = \dfrac{OB}{OA}$,
\item [] $\cos(30) = \dfrac{4}{x}$.
\item [On résout l'équation : ] $\cos(30) = \dfrac{4}{x}$    
\item [] $  x \times \cos (30) = 4 $ \hspace*{1cm}\methode {On a multiplié chaque membre par $x$}
\item [] $x = \dfrac{4}{\cos(30)}$    \hspace*{1.5cm}\methode {On a divisé chaque membre par $\cos(30)$}
\item [D'après la calculatrice : ]  $\cos(30)\approx 0,87$
\item [] $x = \dfrac{4}{0,87}\approx4,6$
\end{labeling}

Donc $OA \approx 4,6$ cm    

\bigskip  

\Asavoir{\Large \underline{Exemple 4 : }} On cherche le côté adjacent, ici la mesure du segment $[OB]$    

\setbox1=\hbox { \hsize=5cm  \null
\begin{tikzpicture}[line cap=round,line join=round,>=triangle 45,x=1.0cm,y=1.0cm, scale=.75]
\begin{scriptsize}
\draw [color=blue, domain=-1.:5.5] plot(\x,{(-0.--3.*\x)/4.});
\draw [color=blue, domain=-1.8:5.5] plot(\x,{(-0.-0.*\x)/4.});
\draw (0,0) node [below] {$O$} -- node[midway,below]{\normalsize x} (4,0) node[below]{$B$} -- 
      (4,3) node[above left]{$A$} -- node[midway,above]{5}(0,0) ;
\draw[color=darkgreen,fill={darkgreen!75},fill opacity=0.1] (4.,0.5) -- (3.5,0.5) -- (3.5,0.) -- (4.,0.) -- cycle; 
\draw (0.7,0.3) node[right] {$20$ \degre};
\draw [shift={(0.,0.)},color=darkgreen,fill={darkgreen!75},fill opacity=0.1] 
      (0,0) -- (0.:0.6) arc (0.:36.85:0.6) -- cycle;
\end{scriptsize}
\end{tikzpicture} } 

\begin{tabular}{M{4cm}M{6cm}M{6cm}}
\methode{On a besoin du l'hypoténuse \\ et du cosinus de l'angle } & 
\box1
                  & \vocabulaire{$\bullet$ On applique le cours\\
                                 $\bullet$ On résout l'équation}\\
                                 
\end{tabular}

\begin{labeling}{D'après la calculatrice : }
\item [On sait que] le triangle $AOB$ est rectangle en $B$,
\item [or] $ \cos(\alpha) = \tfrac{\text{adjacent}}{\text{hypothénuse}} $ 
\item [donc] $\cos(\widehat{AOB}) = \dfrac{OB}{OA}$,
\item [] $\cos(20) = \dfrac{x}{5}$,
\item [On résout l'équation : ] $\cos(20) = \dfrac{x}{5}$    
\item [] $  5 \times \cos (20) = x $ \hspace*{1cm}\methode {On a multiplié chaque membre par $5$}
\item [D'après la calculatrice : ]  $\cos(20)\approx 0,94$
\item [] $x = 5 \times 0,94 \approx 4,7$
\end{labeling}

Donc $OB \approx 4,7$ cm                                                                                  


\ifdefined\COMPLETE
\else
    \end{document}
\fi