\ifdefined\COMPLETE
\else
    \input{./preambule-sacha-utf8.ltx}   
    \usepackage{variations}
    \begin{document}
\fi

\section{Application des dérivées : Étude de fonctions}

\subsection{Caractéristiques du fonction}

\begin{tabular}{lllll}
Soit $f :$ & $\R$ & $\longrightarrow$ & $\R$ & une fonction \\
& $x$ & $\longmapsto$ & $f(x)$ & \\
\end{tabular}

\vspace*{.3cm}

Soit $D_f$ le domaine de définition de la fonction $f$. \\
Soit $I$ un intervalle inclus dans $D_f$. \\
On suppose que $f$ est dérivable sur $I$, c'est-à-dire dérivable en $x_0$ pour tout $x_0 \in I$. 

\subsubsection{Sens de variation}

L'étude du signe de $f'(x)$ sur $I$ permet de connaître le sens de variation de $f$ sur $I$. \\

\textbf{Théorème fondamental :} \\

\begin{itemize}
\item[•] Pour tout $x \in I$, $f'(x) > 0 \Longleftrightarrow$ $f$ est strictement croissante sur $I$. 
\item[•] Pour tout $x \in I$, $f'(x) < 0 \Longleftrightarrow$ $f$ est strictement décroissante sur $I$. 
\item[•] Pour tout $x \in I$, $f'(x) = 0 \Longleftrightarrow$ $f$ est strictement constante sur $I$. 
\end{itemize}

\vspace*{.3cm}

Pour étudier le sens de variations de $f$ sur $D_f$, on partage $D_f$ en intervalles sur lesquels $f'(x)$ garde un signe constant. 

\subsubsection{Extremum}

Soit $\left( O \; ; \; \overrightarrow{i} \; ; \; \overrightarrow{j} \right)$ un repère. \\

Soit $C_f$ la représentation graphique de $f$ dans $\left( O \; ; \; \overrightarrow{i} \; ; \; \overrightarrow{j} \right)$. \\

Soit $M_0\left(x_0\; ; \; f(x_0)\right) \in C_f$. \\

$M_0$ est un extremum $\Longleftrightarrow$ $\left\{
  \begin{array}{l}
    f'(x_0) = 0 \\
    f'(x) \mathrm{\; change \; de \; signe \; en \; x_0} \\
  \end{array}
\right.$

\vspace*{.3cm}

\variations
x & \mI  & & x_0 & & \pI  \\
f'(x) & & - & \z & + & \\
{f(x)} & & \dl & \b {f(x_0)} & \cl & \\
\fin


\vspace*{-2.7cm}

\hspace*{9cm}
\variations
x & \mI  & & x_0 & & \pI  \\
f'(x) & & - & \z & + & \\
f(x) & & \cl & \h{f(x_0)} & \dl & \\
\fin

\vspace*{.3cm}

\hspace*{2cm} $m_0\left(x_0 \; ; \; f(x_0)\right)$ est un minimum. \hspace*{3.3cm} $M_0\left(x_0 \; ; \; f(x_0)\right)$ est un maximum. \\

\textbf{Attention !} \\

\begin{tabular}{llll}
Soit la fonction $f :$ & $\R$ & $\longrightarrow$ & $\R$ \\
& $x$ & $\longmapsto$ & $f(x) = x^3$ \\
\end{tabular}

\vspace*{.3cm}

$f'(x) = 3x^2$. \\
$f'(x) = 0 \Longleftrightarrow x = 0$. \\

\vspace*{-3cm}

\hspace*{9cm}
\variations
x & \mI  & & x_0 & & \pI  \\
f'(x) & & + & \z & + & \\
f(x) &  \b \mI & \cb  & {0} & \ch & \h  \pI \\
\fin

\vspace*{.1cm}

\hspace*{8.3cm}
Cependant, $O(0 \; ; \; 0)$ n'est pas un extremum.

\newpage

\subsubsection{Tangente horizontale}

Soit $\left( O \; ; \; \overrightarrow{i} \; ; \; \overrightarrow{j} \right)$ un repère. \\

Soit $C_f$ la représentation graphique de $f$ dans $\left( O \; ; \; \overrightarrow{i} \; ; \; \overrightarrow{j} \right)$. \\

Soit $M_0\left(x_0\; ; \; f(x_0)\right) \in C_f$ tel que $f'(x_0) = 0$ \\

On cherche l'équation de la tangente au point $M_0\left(x_0 \; ; \; f(x_0)\right)$\\

$y = f'(x_0)\left(x-x_0\right) + f(x_0)$ \\
$y = 0\left(x-x_0\right) + f(x_0)$ \\
$y = f(x_0)$. \\

La tangente est au point $M_0$ est donc une droite parallèle à l'axe des abscisses. \\

\begin{tabular}{ll}
\begin{tikzpicture}[line cap=round,line join=round,>=triangle 45,x=1.0cm,y=1.0cm]
\draw[->,color=black] (-1,0) -- (4.26,0);
\foreach \x in {,1,2,3,4}
\draw[shift={(\x,0)}] (0pt,2pt) -- (0pt,-2pt);
\draw[->] (0,-0.92) -- (0,3.42);
\foreach \y in {,1,2,3}
\draw[shift={(0,\y)}] (2pt,0pt) -- (-2pt,0pt);
\draw (0pt,-10pt) node[right] {\footnotesize $0$};
\clip(-1,-0.88) rectangle (4.26,3.46);
\draw [samples=50,rotate around={0:(2,1)},xshift=2cm,yshift=1cm,domain=-3.0:3.0)] plot (\x,{(\x)^2/2/0.5});
\draw [<->, color=black] (1.5,1) -- (2.5,1);
\draw [dashed] (0,1) -- (2,1) -- (2,0) ; 
\node at (0,1) [left] {$f(x_0)$} ; 
\node at (2,0) [below] {$x_0$} ; 
\begin{pgfonlayer}{background}   
\draw[step=1mm,ultra thin,AntiqueWhite!10] (-1,-0.88) grid  (4.26,3.46);
\draw[step=5mm,very thin,AntiqueWhite!30]  (-1,-0.88) grid  (4.26,3.46);
\draw[step=1cm,very thin,AntiqueWhite!50]  (-1,-0.88) grid  (4.26,3.46);
\draw[step=5cm,thin,AntiqueWhite]          (-1,-0.88) grid  (4.26,3.46);
\end{pgfonlayer}
\end{tikzpicture}
&
\begin{tikzpicture}[line cap=round,line join=round,>=triangle 45,x=1.0cm,y=1.0cm]
\draw[->,color=black] (-0.78,0) -- (4.26,0);
\foreach \x in {,1,2,3,4}
\draw[shift={(\x,0)}] (0pt,2pt) -- (0pt,-2pt);
\draw[->] (0,-0.92) -- (0,3.42);
\foreach \y in {,1,2,3}
\draw[shift={(0,\y)}] (2pt,0pt) -- (-2pt,0pt);
\draw (0pt,-10pt) node[right] {\footnotesize $0$};
\clip(-1.5,-0.92) rectangle (4.26,3.42);
\draw [samples=50,rotate around={-180:(2,2)},xshift=2cm,yshift=2cm,domain=-3.0:3.0)] plot (\x,{(\x)^2/2/0.5});
\draw [<->, color=black] (1.5,2) -- (2.5,2);
\draw [dashed] (0,2) -- (2,2) -- (2,0) ; 
\node at (0,2) [left] {$f(x_0)$} ; 
\node at (2,0) [below] {$x_0$} ; 
\begin{pgfonlayer}{background}   
\draw[step=1mm,ultra thin,AntiqueWhite!10] (-0.78,-0.92) grid (4.26,3.42);
\draw[step=5mm,very thin,AntiqueWhite!30]  (-0.78,-0.92) grid (4.26,3.42);
\draw[step=1cm,very thin,AntiqueWhite!50]  (-0.78,-0.92) grid (4.26,3.42);
\draw[step=5cm,thin,AntiqueWhite]          (-0.78,-0.92) grid (4.26,3.42);
\end{pgfonlayer}
\end{tikzpicture}
\\
\end{tabular}

\vspace*{.3cm}

On en conclut que les tangentes à la courbe aux extrema de la fonction sont parallèles à l'axe des abscisses. \\

\begin{tabular}{llll}
Cependant, on a aussi, pour la fonction $f$ : & $\R$ & $\longrightarrow$ & $\R$ \\
& $x$ & $\longmapsto$ & $f(x) = x^3$ \\
\end{tabular}

\vspace*{.3cm}

\begin{tikzpicture}[line cap=round,line join=round,>=triangle 45,x=1.0cm,y=1.0cm]
\draw[->] (-2.26,0) -- (2.78,0);
\foreach \x in {-2,-1,1,2}
\draw[shift={(\x,0)},color=black] (0pt,2pt) -- (0pt,-2pt);
\draw[->,color=black] (0,-1.9) -- (0,2.44);
\foreach \y in {-1,1,2}
\draw[shift={(0,\y)}] (2pt,0pt) -- (-2pt,0pt);
\clip(-2.26,-1.9) rectangle (2.78,2.44);
\draw[smooth,samples=100,domain=-2.26:2.78] plot(\x,{(\x)*(\x)*(\x)});
\draw [<->, color=black] (-0.5,0) -- (0.5,0);
\begin{pgfonlayer}{background}   
\draw[step=1mm,ultra thin,AntiqueWhite!10] (-2.26,-1.9) grid (2.78,2.44);
\draw[step=5mm,very thin,AntiqueWhite!30]  (-2.26,-1.9) grid (2.78,2.44);
\draw[step=1cm,very thin,AntiqueWhite!50]  (-2.26,-1.9) grid (2.78,2.44);
\draw[step=5cm,thin,AntiqueWhite]          (-2.26,-1.9) grid (2.78,2.44); 
\end{pgfonlayer}
\end{tikzpicture}

\newpage

\subsection{Études de fonctions}

\subsubsection{Exemple \no 1}

\begin{tabular}{llll}
Soit la fonction $f :$ & $\R$ & $\longrightarrow$ & $\R$ \\
& $x$ & $\longmapsto$ & $f(x) = x^2 - 2x - 3$ \\
\end{tabular}

\vspace*{.3cm}

\textbf{Première partie : Étudier les variations de $\mathbf{f}$} \\

$D_f = \R$. \\

$\forall x \in D_f, f'(x) = 2x - 2$ \\

\begin{tabular}{lll}
$f'(x) = 0$ & $\Longleftrightarrow$ & $2x - 2 = 0$ \\
& $\Longleftrightarrow$ & $x = 1$ \\
\end{tabular}

\vspace*{.3cm}

On peut en déduire le signe de $f'(x)$ et les variations de $f$ : \\

\variations
x & \mI  & & 1 & & \pI  \\
f'(x) & & - & \z & + & \\
f(x) & \h{\pI } & \d& \b{f(1) = -4} & \c & \h{\pI } \\
\fin

\vspace*{.3cm}

Le sommet de la parabole est le point $S(1 \; ; \; -4)$. \\ C'est un minimum absolu. \\

La courbe admet une tangente horizontale au point $S$ d'équation $y = 4$. \\

\textbf{Deuxième partie : Représentation graphique} \\

\begin{tikzpicture}[line cap=round,line join=round,>=triangle 45,x=1.0cm,y=1.0cm]
% ci-dessous la "grille" 
% \draw [color=lightgray,dash pattern=on 3pt off 3pt, xstep=1.0cm,ystep=1.0cm] (-2.8,-5.2) grid (6.28,2.52);
\draw[->] (-2.8,0) -- (6.28,0);
\foreach \x in {-2,-1,1,2,3,4,5,6}
\draw[shift={(\x,0)}] (0pt,2pt) -- (0pt,-2pt) node[below] {\footnotesize $\x$};
\draw[->] (0,-5.2) -- (0,2.52);
\foreach \y in {-5,-4,-3,-2,-1,1,2}
\draw[shift={(0,\y)}] (2pt,0pt) -- (-2pt,0pt) node[left] {\footnotesize $\y$};
\draw (0pt,-10pt) node[right] {\footnotesize $0$};
\clip(-2.8,-5.2) rectangle (6.28,2.52);
\draw [samples=50,rotate around={0:(1,-4)},xshift=1cm,yshift=-4cm,domain=-4.0:4.0)] plot (\x,{(\x)^2/2/0.5});
% Ci-dessous la tangente au choix
\draw [<->, color=black] (0.25,-4) -- (1.75,-4); % Plus petite
%\draw [very thin] (-0.5,-4) -- (2.5,-4);
\begin{scriptsize}
%\draw (0.46,-2.82) node {$c$};
\begin{pgfonlayer}{background}   
\draw[step=1mm,ultra thin,AntiqueWhite!10] (-2.8,-5.2) grid (6.28,2.52);
\draw[step=5mm,very thin,AntiqueWhite!30]  (-2.8,-5.2) grid (6.28,2.52);
\draw[step=1cm,very thin,AntiqueWhite!50]  (-2.8,-5.2) grid (6.28,2.52);
\draw[step=5cm,thin,AntiqueWhite]          (-2.8,-5.2) grid (6.28,2.52); 
\end{pgfonlayer}
\end{scriptsize}
\end{tikzpicture}

\newpage

\subsubsection{Exercice \no 2}

\begin{tabular}{llll}
Soit la fonction $f :$ & $\R$ & $\longrightarrow$ & $\R$ \\
& $x$ & $\longmapsto$ & $f(x) = x^3 - 6x^2 + 9x +1$ \\
\end{tabular}

\vspace*{.3cm}

\textbf{Première partie : Étudier les variations de $\mathbf{f}$} \\

$D_f = \R$. \\

$\forall x \in D_f, f'(x) = 3x^2 - 12x + 9$ \\

\begin{tabular}{lll}
$f'(x) = 0$ & $\Longleftrightarrow$ & $3x^2 - 12 + 9 = 0$ \\
& $\Longleftrightarrow$ & $x = 1$ ou $x = 3$ \\
\end{tabular}

\vspace*{.3cm}

On peut en déduire le signe de $f'(x)$ et les variations de $f$ : \\

\variations
x & \mI  & & 1 & & 3 & & \pI  \\
f'(x) & & + & \z & - & \z & + & \\
f(x) & \b{\mI}  & \c & \h{5} & \d & \b {1} & \c & \h \pI \\
\fin

\vspace*{.3cm}

On a : \\

\begin{itemize}
\item[•] $M\left(1 \; ; \; 5\right)$ est un maximum local, où passe une tangente horizontale.
\item[•] $m\left(3 \; ; \; 1\right)$ est un minimum local, où passe une tangente horizontale.
\end{itemize}

\vspace*{.5cm}

\textbf{Deuxième partie : Représentation graphique} \\

\begin{tikzpicture}[line cap=round,line join=round,>=triangle 45,x=1.0cm,y=1.0cm]
\draw[->] (-1.8,0) -- (7.28,0);
\foreach \x in {-1,1,2,3,4,5,6,7}
\draw[shift={(\x,0)}] (0pt,2pt) -- (0pt,-2pt) node[below] {\footnotesize $\x$};
\draw[->] (0,-1.4) -- (0,6.32);
\foreach \y in {-1,1,2,3,4,5,6}
\draw[shift={(0,\y)},color=black] (2pt,0pt) -- (-2pt,0pt) node[left] {\footnotesize $\y$};
\draw (0pt,-10pt) node[right] {\footnotesize $0$};
\clip(-1.8,-1.4) rectangle (7.28,6.32);
\draw[smooth,samples=100,domain=-1.8:7.3] plot(\x,{(\x)*(\x)*(\x)-6*(\x)*(\x)+9*(\x)+1});
\node at (1,5) [above] {$M$} ; \node at (3,1) [below] {$m$} ; 
% Ci-dessous les tangentes au choix
\draw [<->, color=black] (0.5,5) -- (1.5,5);
\draw [<->, color=black] (2.5,1) -- (3.5,1);
%\draw [very thin] (0.5,5) -- (1.5,5);
%\draw [very thin] (2,1) -- (4,1);
\begin{scriptsize}
%\draw (4,3) node {$f$};
\begin{pgfonlayer}{background}   
\draw[step=1mm,ultra thin,AntiqueWhite!10] (-1.8,-1.4) grid (7.28,6.32);
\draw[step=5mm,very thin,AntiqueWhite!30]  (-1.8,-1.4) grid (7.28,6.32);
\draw[step=1cm,very thin,AntiqueWhite!50]  (-1.8,-1.4) grid (7.28,6.32);
\draw[step=5cm,thin,AntiqueWhite]          (-1.8,-1.4) grid (7.28,6.32);
\end{pgfonlayer}
\end{scriptsize}
\end{tikzpicture}

\newpage

\textbf{Troisième partie : Un petit problème} \\

Déterminer le nombre de solution de l'équation $f'(x) = 0$, et donner un encadrement de la solution. \\

On résout graphiquement l'équation $f(x) = 0$, c'est-à-dire que l'on cherche le nombre de points d'intersection de la courbe $C_f$ avec l'axe des abscisses. \\

Il y a un point d'intersection entre $C_f$ et l'axe des abscisses, donc l'équation $f(x) = 0$ admet une et une seule solution $x_1$. \\

On a $-1 < x_1 < 0$. \\

Plus précisément, on a : $-0,2 < x_1 < -0,1$ \\

\begin{tabular}{ll}
Car & $f(-0,2) < 0$ \\
et & $f(-0,1) > 0$ \\
\end{tabular}

\vspace*{.3cm}

Encore plus précisément, on a même : $-0,11 < x_1 < -0,10$ \\

\begin{tabular}{ll}
Car & $f(-0,11) < 0$ \\
et & $f(-0,10) > 0$ \\
\end{tabular}

\vspace*{.3cm}

À $10^{-3}$ près, on a $x_1 \approx -0,105$. \\

\newpage

\subsubsection{Exercice \no 3}

\begin{tabular}{llll}
Soit la fonction $f :$ & $\R$ & $\longrightarrow$ & $\R$ \\
& $x$ & $\longmapsto$ & $f(x) = \dfrac{1}{4}x^4 - 2x^3 + 4x^2 + 1$ \\
\end{tabular}

\vspace*{.3cm}

\textbf{Première partie : Étudier les variations de $\mathbf{f}$.} \\

On a $D_f = \R$. \\

\begin{tabular}{llll}
$\forall x \in D_f$, & $f'(x)$ & $=$ & $x^3 - 6x^2 + 8x$ \\
$\forall x \in D_f$, & $f'(x)$ & $=$ & $x\left(x^2 - 6x + 8\right)$. \\
\end{tabular}

\vspace*{.3cm}

\begin{tabular}{lll}
$f'(x) = 0$ & $\Longleftrightarrow$ & $x = 0$ ou $x^2 - 6x + 8 = 0$ \\
& $\Longleftrightarrow$ & $x = 0$ ou $x = 2$ ou $x= 4$. \\
\end{tabular}

\vspace*{.3cm}

On en déduit le signe de $f'(x)$ et les variations de $f$ : \\

\variations
x & \mI  & & 0 & & 2 & & 4 & \pI  \\
x & & - & \z & + & \l & + & \l & + & \\
x^2 -6x + 8 & & + & \l & + & \z & - & \z & + & \\
f'(x) & & - & \z & + & \z & - & \z & + & \\
f(x) & \h{\pI } & \d& \b{1} & \c & \h{5} & \d& \b{1} & \c & \h{\pI } \\
\fin

\vspace*{.3cm}

\textbf{Deuxième partie : Représentation graphique}

\vspace*{.5cm}

\begin{tikzpicture}[line cap=round,line join=round,>=triangle 45,x=1.0cm,y=1.0cm]
\draw[->,color=black] (-1.8,0) -- (7.28,0);
\foreach \x in {-1,1,2,3,4,5,6,7}
\draw[shift={(\x,0)},color=black] (0pt,2pt) -- (0pt,-2pt) node[below] {\footnotesize $\x$};
\draw[->,color=black] (0,-1.4) -- (0,6.32);
\foreach \y in {-1,2,3,4,5,6}
\draw[shift={(0,\y)},color=black] (2pt,0pt) -- (-2pt,0pt) node[left] {\footnotesize $\y$};
\draw[color=black] (0pt,-10pt) node[right] {\footnotesize $0$};
\clip(-1.8,-1.4) rectangle (7.28,6.32);
\draw[smooth,samples=100,domain=-1.8:7.3] plot(\x,{0.25*(\x)*(\x)*(\x)*(\x)-2*(\x)*(\x)*(\x)+4*(\x)*(\x)+1});
\begin{scriptsize}
\node at (0.25,0.95) [below] {$m_1$};\node at (2,5) [above] {$M$};\node at (4,1) [below] {$m_2$}; 
\draw [<->, color=black] (-0.5,1) -- (0.5,1);
\draw [<->, color=black] (1.5,5) -- (2.5,5);
\draw [<->, color=black] (3.5,1) -- (4.5,1);
%\draw (-0.5,1)-- (0.5,1);
%\draw (1.5,5)-- (2.5,5);
%\draw (3.5,1)-- (4.5,1);
%\draw[color=black] (-1,4.62) node {$f$};
\begin{pgfonlayer}{background}   
\draw[step=1mm,ultra thin,AntiqueWhite!10] (-1.8,-1.4) grid (7.28,6.32);
\draw[step=5mm,very thin,AntiqueWhite!30]  (-1.8,-1.4) grid (7.28,6.32);
\draw[step=1cm,very thin,AntiqueWhite!50]  (-1.8,-1.4) grid (7.28,6.32);
\draw[step=5cm,thin,AntiqueWhite]          (-1.8,-1.4) grid (7.28,6.32);
\end{pgfonlayer}
\end{scriptsize}
\end{tikzpicture}

On a \\

\begin{itemize}
\item[•] $m_1(0 \; ; \; 1)$ est un minimum absolu, où passe une tangente horizontale.
\item[•] $M(2 \; ; \; 5)$ est un maximum local, où passe une tangente horizontale.
\item[•] $m_1(4 \; ; \; 1)$ est un minimum absolu, où passe une tangente horizontale.
\end{itemize}

\newpage

\textbf{Troisième partie : Un petit problème} \\

Soit $m \in \R$. \\

Discutez le nombre et le signe des solutions de l'équation $f(x) = m$, selon les valeurs de $m$. \\

\begin{itemize}
\item[•] $y = f(x)$ est l'équation de la courbe représentative de $f(x)$, notée $C_f$. \\
\item[•] $y = m$ est l'équation d'une droite parallèle à l'axe des abscisses. \\
\end{itemize}

On cherche donc le nombre de point d'intersection entre $C_f$ et les droites d'équation $y = m$, selon les valeurs de $m$. \\

\begin{tikzpicture}% [scale=0.5] scale impossible 
% lgt   = largeur colonne 1 (en cm)
% espcl = espace entre 2 valeurs (la meme pout tous) 
%
\tkzTabInit[lgt=1.9,espcl=4.6, nocadre,lw=0.0001pt]%
%\tkzTabInit[lgt=1.9,espcl=4.6,help]%
%
% Les élements de la première colonne entre {}
% sont séparés par des virgules
% ce sont des couples "blabla" / hauteur en cm
{$m$ / .8, $\genfrac{}{}{0pt}{0} {\text{Équation}} {f(x)=m}$ /2, $\,$ /1.5}
                          {$-\infty$, $1$, $5$, $+\infty$}
\tkzTabLine[]{, 
% \genfrac voir  http://www-sop.inria.fr/marelle/tralics/doc-g.html
                {\text{pas de solution}} , d,
              \genfrac{}{}{0pt}{0}      
              {\genfrac{}{}{0pt}{0}
                {\text{quatre solutions }} 
                {x_1, x_2, x_3 \text{ et } x_4 \text{ avec}}
              }  
              {\genfrac{}{}{0pt}{0}
                {x_1<0 \ x_2>0, x_3 > 0   \text{ et } x_4>0}
                  {} 
              }
                              , d,
          \genfrac{}{}{0pt}{0}
                {\text{deux solutions } x_1 et x_2}
                {\text{ avec } x_1 < 0 \text{ et } x_2 > 0 }
             }
\draw [decorate, line width=2pt, color=white] 
     (T13) -- (T12) -- (T02) -- (T03) -- (T23) -- (T22) -- (T12) ; 
\draw [decorate, line width=1, color=black] 
      (T02) -- (T00) -- (T20) -- (T22) -- (T21) -- (T01)  ;      
\draw [decorate, line width=1, color=black] (T10) -- (T12)  ;    
\draw [decorate, line width=1, color=black] (T20) -- (T22)  ;    
\draw [decorate, line width=1, color=black] (N21) -- (N22)  ; 
\draw [decorate, line width=1, color=black] (N31) -- (N32)  ;                     
\draw [decorate, line width=1, color=black] 
            (M13) -- (N22) -- (M23) -- (N32) -- (M33) ; 
\node  at (N23) {$\genfrac{}{}{0pt}{0}
                            {x_1=0, x_2=4} {\text{ double car minimum }}
                 $} ; 
\node  at (N33) {$\genfrac{}{}{0pt}{0} {x_1<0,} 
                   {\genfrac{}{}{0pt}{0}
         {x_2=2, \text{ et } x_3>0} {x_2 \text{ est une solution double }}}$ } ;              
\end{tikzpicture}  

\newpage

\subsubsection{Exercice \no 4}

\begin{tabular}{llll}
Soit la fonction $f :$ & $\R$ & $\longrightarrow$ & $\R$ \\
& $x$ & $\longmapsto$ & $f(x) = \dfrac{x-1}{x+1}$ \\
\end{tabular}

\vspace*{.3cm}

\textbf{Première partie : Étudier les variations de $\mathbf{f}$.} \\

Il ne faut pas que $x + 1 = 0 \Longleftrightarrow x = -1$. \\

Donc $D_f = \R \setminus \lb -1 \rb = \left]-\infty\; ; \; -1\right[\cup \left]-1 \; ; \; +\infty\right[$ \\

\begin{tabular}{ll}
On note & $u(x) = x - 1$ et $v(x) = x + 1$. \\
Donc & $u'(x) = 1$ et $v'(x) = 1$. \\
\end{tabular}

\vspace*{.3cm}

$f' = \dfrac{u'v - uv'}{v^2}$, on a donc : \\

\begin{tabular}{llll}
$\forall x \in D_f$, & $f'(x)$ & $=$ & $\dfrac{1 \times \left(x+1\right) - \left(x-1\right) \times 1}{\left(x+1\right)^2}$ \vspace*{.3cm} \\
& & $=$ & $\dfrac{x+1 - x +1}{\left(x+1\right)^2}$ \vspace*{.3cm} \\
$\forall x \in D_f,$ & $f'(x)$ & $=$ & $\dfrac{2}{\left(x+1\right)^2}$ \vspace*{.3cm} \\
\end{tabular}

On a : $\forall x \neq -1, f'(x) > 0$. \\

On peut en déduire les variations de $f$ : \\

\variations
x & \mI  & & & -1 & & & \pI  \\
f'(x) & & + & & \bb & & + & \\
f(x) & \b{1} & \cl & \h \pI  & \bb & \b{\mI}  & \cl & \h{1} \\
\fin

\vspace*{.3cm} 

On appelle \textbf{asymptote horizontale} à la courbe $C_f$ la droite d'équation $y = 1$. \\

On appelle \textbf{asymptote verticale} à la courbe $C_f$ la droite d'équation $x = -1$. 

\newpage

\textbf{Deuxième partie : Représentation graphique} \\

\begin{tikzpicture}[line cap=round,line join=round,>=triangle 45,x=1.0cm,y=1.0cm]
\draw[->] (-7,0) -- (8.24,0);
\foreach \x in {-7,-6,-5,-4,-3,-2,-1,1,2,3,4,5,6,7,8}
\draw[shift={(\x,0)}] (0pt,2pt) -- (0pt,-2pt) node[below] {\footnotesize $\x$};
\draw[->] (0,-5.13) -- (0,7.83);
\foreach \y in {-5,-4,-3,-2,-1,1,2,3,4,5,6,7}
\draw[shift={(0,\y)}] (2pt,0pt) -- (-2pt,0pt) node[left] {\footnotesize $\y$};
\draw (0pt,-10pt) node[right] {\footnotesize $0$};
\clip(-7,-5.13) rectangle (8.24,7.83);
\draw[smooth,samples=100,domain=-6.99:-1.01] plot(\x,{((\x)-1)/((\x)+1)});
\draw[smooth,samples=100,domain=-0.98:8.24] plot(\x,{((\x)-1)/((\x)+1)});
\draw [color=red,domain=-7:8.24] plot(\x,{(--1-0*\x)/1});
\draw [color=red] (-1,-5.13) -- (-1,7.83);
\begin{scriptsize}
%\draw(-4.25,2.12) node {$f$};
%\draw[color=red] (-6.73,1.55) node {$a$};
%\draw[color=red] (-0.65,7.59) node {$b$};
\end{scriptsize}
\begin{pgfonlayer}{background}   
\draw[step=1mm,ultra thin,AntiqueWhite!10] (-7,-5.13) grid (8.24,7.83);
\draw[step=5mm,very thin,AntiqueWhite!30]  (-7,-5.13) grid (8.24,7.83);
\draw[step=1cm,very thin,AntiqueWhite!50]  (-7,-5.13) grid (8.24,7.83);
\draw[step=5cm,thin,AntiqueWhite]          (-7,-5.13) grid (8.24,7.83);
\end{pgfonlayer}
\end{tikzpicture}

\newpage

\vspace*{-2cm}

\subsubsection{Exemple \no 5}

\begin{tabular}{llll}
Soit la fonction $f :$ & $\R$ & $\longrightarrow$ & $\R$ \\
& $x$ & $\longmapsto$ & $f(x) = \dfrac{2x^2 - 7x + 5}{x^2 - 5x + 7}$ \\
\end{tabular}

\vspace*{.3cm}

\textbf{Première partie : Étudier les variations de $\mathbf{f}$.} \\

Il ne faut pas que $x^2 - 5x + 7 = 0$. Or, $\Delta = b^2 - 4ac < 0$, donc $D_f = \R$. \\

N.B. : Il n'y a pas de valeurs interdites, donc pas d'asymptotes verticales. \\

On a $u(x) = 2x^2 - 7x +5$ et $v(x) = x^2 - 5x + 7$. \\
D'où $u'(x) = 4x - 7$ et $v'(x) = 2x - 5$. \\

\vspace*{.3cm}

$f' = \dfrac{u'v - uv'}{v^2}$, on a donc :

\begin{tabular}{llll}
$\forall x \in D_f$, & $f'(x)$ & $=$ & $\dfrac{\left(4x-7\right)\left(x^2 - 5x + 7\right)-\left(2x^2 - 7x + 5\right)\left(2x-5\right)}{\left(x^2 - 5x + 7\right)^2}$ \vspace*{.3cm} \\
& & $=$ & $\dfrac{\left(4x^3 - 20x^2 + 28x - 7x^2 + 35x - 49\right)-\left(4x^3 - 10x^2 - 14x^2 + 35x + 10x - 25\right)}{\left(x^2 - 5x + 7\right)^2}$ \vspace*{.3cm} \\
& & $=$ & $\dfrac{\left(4x^3 - 27x^2 +63x - 49\right)-\left(4x^3 - 24x^2 +45x - 25\right)}{\left(x^2 - 5x + 7\right)^2}$ \vspace*{.3cm} \\
& & $=$ & $\dfrac{4x^3 - 27x^2 +63x - 49 - 4x^3 + 24x^2 -45x + 25}{\left(x^2 - 5x + 7\right)^2}$ \vspace*{.3cm} \\
$\forall x \in D_f$, & $f'(x)$ & $=$ & $\dfrac{-3x^2 + 18x - 24}{\left(x^2 - 5x + 7\right)^2}$ \vspace*{.3cm} \\
\end{tabular}

On étudie le signe de $f'(x)$. \\

\begin{tabular}{lll}
$f'(x) = 0$ & $\Longleftrightarrow$ & $\dfrac{-3x^2 + 18x - 24}{\left(x^2 - 5x + 7\right)^2} = 0$ \vspace*{.3cm} \\
& $\Longleftrightarrow$ & $-3x^2 + 18x - 24 = 0$ \vspace*{.3cm} \\
& $\Longleftrightarrow$ & $x= 2$ ou $x = 4$ \vspace*{.3cm}\\
\end{tabular}

On peut ainsi en déduire le signe de $f'$ et les variations de $f$ : 

\vspace*{.2cm}

\variations
x & \mI  & & 2 & & \;4 & & \pI  \\
f'(x) & & - & \z & + & \z & - & \\
f(x) & \h{2} & \dl & \b{-1} & \cl & \h{3} & \dl& \b{2} \\
\fin

\vspace*{.3cm}

On a : \\

\begin{itemize}
\item[•] un minimum absolu au point $m(2 \; ; \; -1)$
\item[•] un maximum absolu au point $M(4 \; ; \; 3)$
\end{itemize}

\vspace*{.3cm}

On a aussi une asymptote horizontale : la droite d'équation $y = 2$. 

\vspace*{-5cm}

\newpage

\textbf{Deuxième partie : Représentation graphique} \\

\begin{tikzpicture}[line cap=round,line join=round,>=triangle 45,x=1.0cm,y=1.0cm]
\draw[->] (-6.73,0) -- (10.66,0);
\foreach \x in {-6,-5,-4,-3,-2,-1,1,2,3,4,5,6,7,8,9,10}
\draw[shift={(\x,0)}] (0pt,2pt) -- (0pt,-2pt) node[below] {\footnotesize $\x$};
\draw[->] (0,-2.98) -- (0,4.98);
\foreach \y in {-2,-1,1,2,3,4}
\draw[shift={(0,\y)},color=black] (2pt,0pt) -- (-2pt,0pt) node[left] {\footnotesize $\y$};
\draw (0pt,-10pt) node[right] {\footnotesize $0$};
\clip(-6.73,-2.98) rectangle (10.66,4.98);
\draw[smooth,samples=100,domain=-6.7:10.6] plot(\x,{(2*(\x)*(\x)-7*(\x)+5)/((\x)*(\x)-5*(\x)+7)});
\draw [line width=0.4pt,color=red,domain=-6.73:10.66] plot(\x,{(--2-0*\x)/1});
\draw [<->, color=black] (2.5,3) -- (5.5,3);
\draw [<->, color=black] (1,-1) -- (3,-1);
\begin{scriptsize}
%\draw (-6.46,1.45) node {$f$};
\end{scriptsize}
\begin{pgfonlayer}{background}   
\draw[step=1mm,ultra thin,AntiqueWhite!10] (-6.73,-2.98) grid (10.66,4.98);
\draw[step=5mm,very thin,AntiqueWhite!30]  (-6.73,-2.98) grid (10.66,4.98);
\draw[step=1cm,very thin,AntiqueWhite!50]  (-6.73,-2.98) grid (10.66,4.98);
\draw[step=5cm,thin,AntiqueWhite]          (-6.73,-2.98) grid (10.66,4.98);
\end{pgfonlayer}
\end{tikzpicture}

\vspace*{.3cm}

\textbf{\hbox{Troisième partie : Étudier l'intersection de l'asymptote horizontale et de la représentation graphique de $\mathbf{f}$.}} \\

On cherche le point d'intersection de $C_f$ et de l'asymptote horizontale. \\

\begin{tabular}{lll}
$f(x) = 2$ & $\Longleftrightarrow$ & $\dfrac{2x^2 - 7x +5}{x^2 - 5x + 7} = 2$ \vspace*{.3cm} \\
& $\Longleftrightarrow$ & $2x^2 - 7x + 5 = 2\left(x^2 - 5x + 7\right)$ \vspace*{.3cm} \\
& $\Longleftrightarrow$ & $2x^2 - 7x +5 = 2x^2 - 10x + 14$ \vspace*{.3cm} \\
& $\Longleftrightarrow$ & $-7x + 5 = -10x + 14$ \vspace*{.3cm} \\
& $\Longleftrightarrow$ & $3x = 9$ \vspace*{.3cm} \\
& $\Longleftrightarrow$ & $x = 3$ \vspace*{.3cm} \\
\end{tabular}

On a une solution unique, donc un seul point d'intersection, qui est le point $I(3 \; ; \; 2)$.

\newpage

\vspace*{-2.8cm}

\subsubsection{Exemple \no 6}

\begin{tabular}{llll}
Soit la fonction $f :$ & $\R$ & $\longrightarrow$ & $\R$ \\
& $x$ & $\longmapsto$ & $f(x) = \dfrac{2x^2 - 9x +4}{x^2 + x - 12}$ \\
\end{tabular}

\vspace*{.3cm}

\textbf{Première partie : Étudier les variations de $\mathbf{f}$.} 

Il ne faut pas que $x^2 + x - 12 = 0 \Longleftrightarrow x = -4$ ou $x = 3$. \\

D'où $D_f = \R \setminus \lb -4 \; ; \; 3 \rb = \left]-\infty  \; ; \; -4\right[\cup\left]-4 \; ; \; 3\right[\cup\left]3 \; ; \; +\infty\right[$ \\

N.B. : \hbox{On peut d'ores et déjà dire que l'on a 2 asymptotes verticales, les droites d'équations $x = -4$ et $x = 3$.} \\ 

On a $u(x) = 2x^2 - 9x +4$ et $v(x) = x^2 + x - 12$. \\
D'où $u'(x) = 4x - 9$ et $v'(x) = 2x +1 $. 

\vspace*{.3cm}

$f' = \dfrac{u'v - uv'}{v^2}$, on a donc :

\begin{tabular}{llll}
$\forall x \in D_f$, & $f'(x)$ & $=$ & $\dfrac{\left(4x - 9\right)\left(x^2 + x -12\right)-\left(2x^2 - 9x +4\right)\left(2x +1\right)}{\left(x^2 + x - 12\right)^2}$ \vspace*{.3cm} \\
& & $=$ & $\dfrac{\left(4x^3 + 4x^2 - 48x - 9x^2 - 9x + 108\right)-\left(4x^3 + 2x^2 - 18x^2 - 9x + 8x +4\right)}{\left(x^2 + x - 12\right)^2}$ \vspace*{.3cm} \\
& & $=$ & $\dfrac{\left(4x^3 - 5x^2 - 57x + 108\right)-\left(4x^3 - 16x^2 - x + 4\right)}{\left(x^2 + x - 12\right)^2}$ \vspace*{.3cm} \\
& & $=$ & $\dfrac{4x^3 - 5x^2 - 57x + 108 - 4x^3 + 16x^2 + x - 4}{\left(x^2 + x - 12\right)^2}$ \vspace*{.3cm} \\
$\forall x \in D_f$, & $f'(x)$ & $=$ & $\dfrac{11x^2 - 56x + 104}{\left(x^2 + x - 12\right)^2}$ \\
\end{tabular}

On étudie le signe de $f'(x)$. \\

\begin{tabular}{lll}
$f'(x) = 0$ & $\Longleftrightarrow$ & $\dfrac{11x^2 - 56x + 104}{\left(x^2 + x - 12\right)^2} = 0$ \vspace*{.3cm} \\
& $\Longleftrightarrow$ & $11x^2 - 56x + 104 = 0$ \vspace*{.3cm} \\
\end{tabular}

$\Delta < 0$, donc le trinôme est du signe de a pour tout réel x appartenant à $D_f$. \\
$ a = 11$, d'où $\forall x \in D_f, 11x^2 - 56x + 104 > 0$. \\

On peut ainsi en déduire les variations de $f$ : 

\vspace*{.2cm}

\variations
x & \mI  & & & -4 & & & & \;3 & & & \pI  \\
f'(x) & & + & & \bb & & + & & \bb & & + & \\
f(x) & \b{2} & \cl & \h\pI  & \bb & \b{\mI}  & \cl & \h  \pI  & \bb & \b{\mI}  & \cl & \h{2} \\
\fin

\vspace*{.3cm}

On a : 

\begin{itemize}
\item[*] 2 asymptotes verticales : 
\begin{itemize}
\item[•] la droite d'équation $x = -4$ 
\item[•] la droite d'équation $x = 3$ 
\end{itemize}
\item[*] 1 asymptote horizontale : 
\begin{itemize}
\item[•] la droite d'équation $ y = 2$ 
\end{itemize}
\end{itemize}

\vspace*{-10cm}

\newpage

\textbf{Deuxième partie : Représentation graphique} \\

\centerline{
\begin{tikzpicture}[line cap=round,line join=round,>=triangle 45,x=1.0cm,y=1.0cm,scale=0.7]
\draw[->] (-14.15,0) -- (14.99,0);
\foreach \x in {-14,-12,-10,-8,-6,-4,-2,2,4,6,8,10,12,14}
\draw[shift={(\x,0)}] (0pt,2pt) -- (0pt,-2pt) node[below] {\footnotesize $\x$};
\draw[->] (0,-4.69) -- (0,9.24);
\foreach \y in {-4,-2,2,4,6,8}
\draw[shift={(0,\y)}] (2pt,0pt) -- (-2pt,0pt) node[left] {\footnotesize $\y$};
\draw (0pt,-10pt) node[right] {\footnotesize $0$};
\clip(-14.15,-4.69) rectangle (14.99,9.24);
\draw[smooth,samples=100,domain=-14.149:-4.1] plot(\x,{(2*(\x)*(\x)-9*(\x)+4)/((\x)*(\x)+(\x)-12)});
\draw[smooth,samples=100,domain=-3.89:2.98] plot(\x,{(2*(\x)*(\x)-9*(\x)+4)/((\x)*(\x)+(\x)-12)});
\draw[smooth,samples=100,domain=3.05:14.9] plot(\x,{(2*(\x)*(\x)-9*(\x)+4)/((\x)*(\x)+(\x)-12)});
\draw [line width=0.4pt,color=red] (3,-4.69) -- (3,9.24);
\draw [line width=0.4pt,color=red,domain=-14.15:14.99] plot(\x,{(--2-0*\x)/1});
\draw [line width=0.4pt,color=red] (-4,-4.69) -- (-4,9.24);
\begin{scriptsize}
%\draw(-11.43,3.94) node {$f$};
\end{scriptsize}
\begin{pgfonlayer}{background}   
\draw[step=1mm,ultra thin,AntiqueWhite!10] (-14.15,-4.69) grid (14.99,9.24);
\draw[step=5mm,very thin,AntiqueWhite!30]  (-14.15,-4.69) grid (14.99,9.24);
\draw[step=1cm,very thin,AntiqueWhite!50]  (-14.15,-4.69) grid (14.99,9.24);
\draw[step=5cm,thin,AntiqueWhite]          (-14.15,-4.69) grid (14.99,9.24);
\end{pgfonlayer}
\end{tikzpicture}}

\vspace*{.3cm}

\textbf{\hbox{Troisième partie : Étudier l'intersection de l'asymptote horizontale et de la représentation graphique de $\mathbf{f}$.}} \\

On cherche le point d'intersection de $C_f$ et de l'asymptote horizontale. \\

\begin{tabular}{lll}
$f(x) = 2$ & $\Longleftrightarrow$ & $\dfrac{2x^2 - 9x +4}{x^2 + x - 12} = 2$ \vspace*{.3cm} \\
& $\Longleftrightarrow$ & $2x^2 - 9x + 4 = 2\left(x^2 + x - 12\right)$ \vspace*{.3cm} \\
& $\Longleftrightarrow$ & $2x^2 - 9x +4 = 2x^2 + 2x - 24$ \vspace*{.3cm} \\
& $\Longleftrightarrow$ & $-9x + 4 = 2x - 24$ \vspace*{.3cm} \\
&$\Longleftrightarrow$ & $-11x = -28$ \vspace*{.3cm} \\
&$\Longleftrightarrow$ & $x = \dfrac{28}{11}$ \\
\end{tabular}

On a une solution unique, donc un seul point d'intersection, qui est le point $I(\dfrac{28}{11} \; ; \; 2)$.

\newpage

\vspace*{-1.5cm}

\subsubsection{Exemple \no 7 : Exercice type bac}

\begin{tabular}{llll}
Soit la fonction $f :$ & $\R$ & $\longrightarrow$ & $\R$ \\
& $x$ & $\longmapsto$ & $f(x) = \dfrac{2x^2 + x -13}{x^2 - x - 2}$ \\
\end{tabular}

\vspace*{.3cm}

\textbf{Première partie : Étudier les variations de $\mathbf{f}$.} \\

Il ne faut pas que $x^2 - x - 2 = 0 \Longleftrightarrow x = -1$ ou $x = 2$. \\

N.B. : \hbox{On peut d'ores et déjà dire que l'on a 2 asymptotes verticales, les droites d'équations $x = -1$ et $x = 2$.} \\ 

On a $u(x) = 2x^2 + x -13$ et $v(x) = x^2 - x - 2$. \\
D'où $u'(x) = 4x +1 $ et $v'(x) = 2x -1 $. 

\vspace*{.3cm}

$f' = \dfrac{u'v - uv'}{v^2}$, on a donc :

\begin{tabular}{llll}
$\forall x \in D_f$, & $f'(x)$ & $=$ & $\dfrac{\left(4x + 1\right)\left(x^2 - x -2\right)-\left(2x^2 + x - 13\right)\left(2x - 1\right)}{\left(x^2 - x - 2\right)^2}$ \vspace*{.3cm} \\
& & $=$ & $\dfrac{\left(4x^3 - 4x^2 - 8x + x^2 - x - 2\right)-\left(4x^3 - 2x^2 + 2x^2 - x - 26x + 13\right)}{\left(x^2 - x - 2\right)^2}$ \vspace*{.3cm} \\
& & $=$ & $\dfrac{\left(4x^3 - 3x^2 - 9x - 2\right)-\left(4x^3 - 27x + 13\right)}{\left(x^2 - x - 2\right)^2}$ \vspace*{.3cm} \\
& & $=$ & $\dfrac{4x^3 - 3x^2 - 9x - 2 - 4x^3 + 27x - 13}{\left(x^2 - x - 2\right)^2}$ \vspace*{.3cm} \\
$\forall x \in D_f$, & $f'(x)$ & $=$ & $\dfrac{-3x^2 + 18x - 15}{\left(x^2 - x - 2\right)^2}$ \vspace*{.3cm} \\
\end{tabular}

On étudie le signe de $f'(x)$. \\

\begin{tabular}{lll}
$f'(x) = 0$ & $\Longleftrightarrow$ & $\dfrac{-3x^2 + 18x - 15}{\left(x^2 - x - 2\right)^2} = 0$ \vspace*{.3cm} \\
& $\Longleftrightarrow$ & $-3x^2 + 18x - 15 = 0$ \vspace*{.3cm} \\
& $\Longleftrightarrow$ & $x = 1$ ou $x = 5$ \vspace*{.3cm} \\
\end{tabular}

On peut ainsi en déduire le signe de $f'(x)$ les variations de $f$ : \\

\centerline{
\variations
x & \mI  & & & -1 & & & 1 & & & 2 & & & 5 & & \pI  \\
f'(x) & & - & & \bb & & - & \z & + & & \bb & & + & \z & - & \\
f(x) & \h 2 & \dl & \b{\mI}  & \bb & \h \pI  & \dl &  \b{5} & \cl & \h \pI  & \bb & \b{\mI}  & \cl & 
 \h {\stackrel{} {\dfrac{7}{3}}} & \dl &  \b{2} \\
\fin}

\vspace*{.3cm}



\begin{tabular}{ll}
\begin{minipage}{5cm}
On a :
\begin{itemize}
\item[•] $m\left(1 \; ; \; 5\right)$ est un minimum. \\
\item[•] $M\left(5 \; ; \; \dfrac{7}{3}\right)$ est un maximum.
\end{itemize}
\end{minipage}
\hspace*{3cm}&
\begin{minipage}{7cm}
On a aussi :
\begin{itemize}
\item[*] 2 asymptotes verticales : 
\begin{itemize}
\item[•] la droite d'équation $x = -1$ 
\item[•] la droite d'équation $x = 2$ 
\end{itemize}
\item[*] 1 asymptote horizontale : 
\begin{itemize}
\item[•] la droite d'équation $ y = 2$ 
\end{itemize}
\end{itemize}
\end{minipage}
\end{tabular}

\vspace*{-5cm}

\newpage

\textbf{Deuxième partie : Représentation graphique} \\

\centerline{
\begin{tikzpicture}[line cap=round,line join=round,>=triangle 45,x=1.0cm,y=1.0cm]
\draw[->] (-8.88,0) -- (9.15,0);
\foreach \x in {-8,-7,-6,-5,-4,-3,-2,-1,1,2,3,4,5,6,7,8}
\draw[shift={(\x,0)}] (0pt,2pt) -- (0pt,-2pt) node[below] {\footnotesize $\x$};
\draw[->] (0,-4.05) -- (0,9.88);
\foreach \y in {-4,-3,-2,-1,1,2,3,4,5,6,7,8,9}
\draw[shift={(0,\y)}] (2pt,0pt) -- (-2pt,0pt) node[left] {\footnotesize $\y$};
\draw (0pt,-10pt) node[right] {\footnotesize $0$};
\clip(-8.88,-4.05) rectangle (9.15,9.88);
\draw[smooth,samples=100,domain=-8.879:-1.05] plot(\x,{(2*(\x)*(\x)+(\x)-13)/((\x)*(\x)-(\x)-2)});
\draw[smooth,samples=100,domain=-0.95:1.95] plot(\x,{(2*(\x)*(\x)+(\x)-13)/((\x)*(\x)-(\x)-2)});
\draw[smooth,samples=100,domain=2.05:9.148] plot(\x,{(2*(\x)*(\x)+(\x)-13)/((\x)*(\x)-(\x)-2)});
\draw [line width=0.4pt,color=red,domain=-8.88:9.15] plot(\x,{(--2-0*\x)/1});
\draw [line width=0.4pt,color=red] (-1,-4.05) -- (-1,9.88);
\draw [line width=0.4pt,color=red] (2,-4.05) -- (2,9.88);
% ,line width=0.4pt pour extrémités plus discretes
\draw  (1,5)-- ++(-1.0pt,-1.0pt) -- ++(2.0pt,2.0pt) ++(-2.0pt,0) -- ++(2.0pt,-2.0pt);
\node at (1,4.8) {$m$} ;
\node at (5,2.5) {$M$} ;
\draw [<->,line width=0.4pt,>=latex] (0.5,5) -- (1.5,5);
\draw  (5,7/3)-- ++(-1.0pt,-1.0pt) -- ++(2.0pt,2.0pt) ++(-2.0pt,0) -- ++(2.0pt,-2.0pt);
\draw [<->,line width=0.4pt,>=latex] (4.5,7/3) -- (5.5,7/3);
% Une tangeante en (4,2) est impossible puisque y=2 est une asymptote 
\draw [line width=0.4pt,color=black,domain=-8.38:12.2] plot(\x,{1/4*(3*(\x)-1) });
%\draw  (4,2)-- ++(-1.0pt,-1.0pt) -- ++(2.0pt,2.0pt) ++(-2.0pt,0) -- ++(2.0pt,-2.0pt);
\begin{scriptsize}
%\draw (-5.92,1.05) node {$f$};
\end{scriptsize}
\begin{pgfonlayer}{background}   
\draw[step=1mm,ultra thin,AntiqueWhite!10] (-8.88,-4.05) grid (9.15,9.88);
\draw[step=5mm,very thin,AntiqueWhite!30]  (-8.88,-4.05) grid (9.15,9.88);
\draw[step=1cm,very thin,AntiqueWhite!50]  (-8.88,-4.05) grid (9.15,9.88);
\draw[step=5cm,thin,AntiqueWhite]          (-8.88,-4.05) grid (9.15,9.88);
\end{pgfonlayer}
\end{tikzpicture}}

\vspace*{.3cm}

\textbf{\hbox{Troisième partie : Étudier l'intersection de l'asymptote horizontale et de la représentation graphique de $\mathbf{f}$.}} \\

On cherche le point d'intersection de $C_f$ et de l'asymptote horizontale. \\

\begin{tabular}{lll}
$f(x) = 2$ & $\Longleftrightarrow$ & $\dfrac{2x^2 + x - 13}{x^2 - x - 2} = 2$ \vspace*{.3cm} \\
& $\Longleftrightarrow$ & $2x^2 + x - 13 = 2\left(x^2 - x - 2\right)$ \vspace*{.3cm} \\
&$\Longleftrightarrow$ & $2x^2 + x - 13 = 2x^2 - 2x - 4$ \vspace*{.3cm} \\
&$\Longleftrightarrow$ & $x - 13 = -2x - 4$ \vspace*{.3cm} \\
&$\Longleftrightarrow$ & $3x = 9$ \vspace*{.3cm} \\
&$\Longleftrightarrow$ & $x = 3$ \vspace*{.3cm} \\
\end{tabular}

On a une solution unique, donc un seul point d'intersection, qui est le point $I(3 \; ; \; 2)$.

\newpage

\textbf{Quatrième partie : Déterminer l'équation de la tangente à la courbe au point $\mathbf{I(3 \; ; \; 2)}$} \\

$y = f'(3)(x-3) + f(3)$ \\

$y = \dfrac{3}{4}(x - 3) + 2$ \\

$y = \dfrac{3}{4}x - \dfrac{9}{4} + 2$ \\

$ y = \dfrac{3}{4}x - \dfrac{1}{4}$ \\

\textbf{Cinquième partie : Discuter le nombre et le signe des solutions de l'équation $\mathbf{f(x) = m}$} \\

Soit $m \in \R$. \\

Discutez le nombre et le signe des solutions de l'équation $f(x) = m$, selon les valeurs de $m$. \\

\begin{itemize}
\item[•] $y = f(x)$ est l'équation de la courbe représentative de $f(x)$, notée $C_f$. \\
\item[•] $y = m$ est l'équation d'une droite parallèle à l'axe des abscisses. \\
\end{itemize}

On cherche donc le nombre de point d'intersection entre $C_f$ et les droites d'équation $y = m$, selon les valeurs de $m$. \\



\variations
   m      &  \mI   &   & 2  &  &   \frac{7}{3} & & 5 & & \pI  \\
\genfrac{}{}{0pt}{0} {\text{Équation}} {f(x)=m} 
 & &  \genfrac{}{}{0pt}{0} 
                  {} % Une ligne blache pour aérer
                  {  \genfrac{}{}{0pt}{0} 
                     {\genfrac{}{}{0pt}{0} 
                                  {\text{Deux solutions }}
                                 {x_1 \text{ et } x_2}
                   }  {} }\quad & \l &  
           \quad \genfrac{}{}{0pt}{0} 
                  {} % Une ligne blache pour aérer
                  {  \genfrac{}{}{0pt}{0} 
                     {\genfrac{}{}{0pt}{0} 
                                  {\text{Deux solutions }}
                                 {x_1 \text{ et } x_2}
                   }  {} }\quad & \l &  
           \quad   \genfrac{}{}{0pt}{0} 
                  {} {\text{Pas de  solutions}} \quad & \l & 
           \quad  \genfrac{}{}{0pt}{0} 
                  {}{\text{Deux solutions}} &\\
\fin

\vspace*{.3cm}

{\bfseries Et plus précisément}

\bigskip 
         

         
 \footnotesize
\begin{tikzpicture}
% lgt   = largeur colonne 1 (en cm)
% espcl = espace entre 2 valeurs (la meme pout tous) 
%
\tkzTabInit[lgt=1.9,espcl=2.3]%
%\tkzTabInit[lgt=1.9,espcl=2.3,help]%
%
% Les élements de la première colonne entre {}
% sont séparés par des virgules
% ce sont des couples "blabla" / hauteur en cm
{$x$ / .8, $f(x)=m$ /2, $\,$ /1.5}
                          {$-\infty$, $-1$, $\dfrac{7}{3}$,5, 
                           $\dfrac{13}{2}$, $+\infty$
                          }
                          % Garder la ligne blanche ci-dessous avant les obliques                         
\draw [decorate, line width=2pt, color=white] 
     (T13) -- (T12) -- (T02) -- (T03) -- (T23) -- (T22) -- (T12) ;      
\tkzTabLine[]{, 
% \genfrac voir  http://www-sop.inria.fr/marelle/tralics/doc-g.html
          \genfrac{}{}{0pt}{0}
                {\text{2 solutions }} 
                {x_1<0 , x_2 > 0}, d,                      
          \genfrac{}{}{0pt}{0}
                {\text{2 solutions }} 
                {x_1>0 , x_2 > 0}, d, 
                h, d,
           \genfrac{}{}{0pt}{0}
                {\text{2 solutions }} 
                {x_1 > 0 , x_2 > 0}, d,      
           \genfrac{}{}{0pt}{0}
                {\text{2 solutions }} 
                {x_1 < 0 , x_2 > 0}
             }         
\draw [decorate, line width=1, color=black] 
            (M13) -- (N22) -- (M23) -- (N32) -- (M33) 
            -- (N42) -- (M43) -- (N52) -- (M53) ;       
\fill [pattern=north west lines] (N32) -- (M33) --   (N42) -- cycle ;                                
\node [above] at (N23) {$\genfrac{}{}{0pt}{0}{x=3}{\text{Simple}} $} ;
\node [above] at (N33) {$\genfrac{}{}{0pt}{0}{x=5}{\text{Double}}$} ;
\node [above] at (N43) {$\genfrac{}{}{0pt}{0}{x=1}{\text{Double} }$} ;
\node [above] at (N53) {$\genfrac{}{}{0pt}{0} {x=0} {x_2>0} $} ;
    
\end{tikzpicture}\\

\newpage

\subsection{Supplément gratuit !}

\subsubsection{Exercice \no 1}

\begin{tabular}{llll}
Soit la fonction $f :$ & $\R$ & $\longrightarrow$ & $\R$ \\
& $x$ & $\longmapsto$ & $f(x) = \dfrac{x^2 + 3}{x - 1}$ \\
\end{tabular}

\vspace*{.3cm}

\textbf{Première partie : Étudier les variations de $\mathbf{f}$.} \\

Il ne faut pas que $x-1 = 0 \Longleftrightarrow x = 1$. \\

Donc $D_f = \R \setminus \lb 1 \rb = \left]-\infty\; ; \; 1\right[\cup\left]1\; ; \; +\infty\right[$. \\

On a $u(x) = x^2 + 3$ et $v(x) = x - 1$. \\
D'où $u'(x) = 2x$ et $v'(x) = 1$. \\

\vspace*{.3cm}

$f' = \dfrac{u'v - uv'}{v^2}$, on a donc :

\begin{tabular}{llll}
$\forall x \in D_f$, & $f'(x)$ & $=$ & $\dfrac{2x\left(x-1\right)-\left(x^2 + 3\right)\times 1}{\left(x-1\right)^2}$ \vspace*{.3cm} \\
& & $=$ & $\dfrac{2x^2 - 2x - x^2 - 3}{\left(x-1\right)^2} \vspace*{.3cm}$ \\
$\forall x \in D_f$, & $f'(x)$ & $=$ & $\dfrac{x^2 - 2x - 3}{\left(x-1\right)^2}$ \vspace*{.3cm} \\
\end{tabular}

On étudie le signe de $f'(x)$. \\

\begin{tabular}{lll}
$f'(x) = 0$ & $\Longleftrightarrow$ & $\dfrac{x^2 - 2x - 3}{\left(x-1\right)^2} = 0$ \vspace*{.3cm} \\
& $\Longleftrightarrow$ & $x^2 - 2x - 3 = 0$ \vspace*{.3cm} \\
& $\Longleftrightarrow$ & $x= -1$ ou $x = 3$ \vspace*{.3cm}\\
\end{tabular}

On peut ainsi en déduire le signe de $f'$ et les variations de $f$ : 

\vspace*{.2cm}

\variations
x & \mI  & & -1 & & & \;1 & & & 3 & & \pI  \\
f'(x) & & + & \z & - & & \bb & & - & \z & + & \\
f(x) & \b{\mI}  & \cl & \h{-2} & \dl & \b{\mI}  & \bb & \h \pI  & \dl & \b{6} & \cl & \h \pI  \\
\fin

\vspace*{.3cm}

On a :

\begin{itemize}
\item[•] un maximum local au point $M(-1 \; ; \; -2)$
\item[•] un minimum local au point $m(3 \; ; \; 6)$
\end{itemize}

\vspace*{.3cm}

On a aussi :

\begin{itemize}
\item[•] Une asymptote verticale : la droite d'équation $x = 1$.
\item[•] Aucun asymptote horizontale.
\item[•] Une \textbf{asymptote oblique} : la droite d'équation $y = x+1$.
\end{itemize}

\vspace*{-5cm}

\newpage

\textbf{Seconde partie : Représentation graphique} \\

\centerline{
\begin{tikzpicture}[line cap=round,line join=round,>=triangle 45,x=1.0cm,y=1.0cm]
\draw[->] (-8.38,0) -- (12.2,0);
\foreach \x in {-8,-7,-6,-5,-4,-3,-2,-1,1,2,3,4,5,6,7,8,9,10,11,12}
\draw[shift={(\x,0)}] (0pt,2pt) -- (0pt,-2pt) node[below] {\footnotesize $\x$};
\draw[->] (0,-9.9) -- (0,12.73);
\foreach \y in {-9,-8,-7,-6,-5,-4,-3,-2,-1,1,2,3,4,5,6,7,8,9,10,11,12}
\draw[shift={(0,\y)},color=black] (2pt,0pt) -- (-2pt,0pt) node[left] {\footnotesize $\y$};
\draw(0pt,-10pt) node[right] {\footnotesize $0$};
\clip(-8.38,-9.9) rectangle (12.2,12.73);
\draw[smooth,samples=100,domain=-8.375:0.95] plot(\x,{((\x)*(\x)+3)/((\x)-1)});
\draw[smooth,samples=100,domain=1.05:12.2] plot(\x,{((\x)*(\x)+3)/((\x)-1)});
\draw [line width=0.4pt,color=red] (1,-9.9) -- (1,12.73);
\draw [line width=0.4pt,color=red,domain=-8.38:12.2] plot(\x,{(\x)+1 });
\draw  (-1,-2)-- ++(-1.0pt,-1.0pt) -- ++(2.0pt,2.0pt) ++(-2.0pt,0) -- ++(2.0pt,-2.0pt) node [above] {$M$};
\draw [<->,line width=0.4pt,>=latex] (-1.5,-2) -- (-.5,-2);
\draw  (3,6)-- ++(-1.0pt,-1.0pt) -- ++(2.0pt,2.0pt) ++(-2.0pt,0) -- ++(2.0pt,-2.0pt) node [below] {$m$} ;
\draw [<->,line width=0.4pt,>=latex] (2.5,6) -- (3.5,6);
\begin{scriptsize}
\draw (-8.11,-7.88) node {$f$};
\end{scriptsize}
\begin{pgfonlayer}{background}   
\draw[step=1mm,ultra thin,AntiqueWhite!10] (-8.38,-9.9) grid (12.2,12.73);
\draw[step=5mm,very thin,AntiqueWhite!30]  (-8.38,-9.9) grid (12.2,12.73);
\draw[step=1cm,very thin,AntiqueWhite!50]  (-8.38,-9.9) grid (12.2,12.73);
\draw[step=5cm,thin,AntiqueWhite]          (-8.38,-9.9) grid (12.2,12.73);
\end{pgfonlayer}
\end{tikzpicture}}

\newpage

\subsubsection{Exercice \no 2}

\begin{tabular}{llll}
Soit la fonction $f :$ & $\R$ & $\longrightarrow$ & $\R$ \\
& $x$ & $\longmapsto$ & $f(x) = \dfrac{-x^2 + 5}{x + 1}$ \\
\end{tabular}

\vspace*{.3cm}

\textbf{Première partie : Étudier les variations de $\mathbf{f}$.} \\

Il ne faut pas que $x+1 = 0 \Longleftrightarrow x = -1$. \\

Donc $D_f = \R \setminus \lb -1 \rb = \left]-\infty\; ; \; -1\right[\cup\left]-1\; ; \; +\infty\right[$. \\

On a $u(x) = -x^2 + 5$ et $v(x) = x + 1$. \\
D'où $u'(x) = -2x$ et $v'(x) = 1$. \\

\vspace*{.3cm}

$f' = \dfrac{u'v - uv'}{v^2}$, on a donc :

\begin{tabular}{llll}
$\forall x \in D_f$, & $f'(x)$ & $=$ & $\dfrac{-2x\left(x+1\right)-\left(-x^2 + 5\right) \times 1}{\left(x+1\right)^2}$\vspace*{.3cm} \\
& & $=$ & $\dfrac{-2x^2 - 2x + x^5 - 5}{\left(x+1\right)^2}$ \vspace*{.3cm} \\
$\forall x \in D_f$, & $f'(x)$ & $=$ & $\dfrac{-x^2 - 2x - 5}{\left(x+1\right)^2}$ \vspace*{.3cm} \\
\end{tabular}

On étudie le signe de $f'(x)$. \\

\begin{tabular}{lll}
$f'(x) = 0$ & $\Longleftrightarrow$ & $\dfrac{-x^2 - 2x - 5}{\left(x+1\right)^2} = 0$ \vspace*{.3cm} \\
& $\Longleftrightarrow$ & $-x^2 - 2x - 5 = 0$ \vspace*{.3cm} \\
\end{tabular}

\vspace*{.3cm} 

$\Delta < 0$, donc le trinôme $-x^2 - 2x -5$ est négatif pour tout $x$ appartenant à $D_f$. \\

On peut ainsi en déduire le signe de $f'$ et les variations de $f$ : 

\vspace*{.2cm}

\variations
x & \mI  & & & -1 & & & \pI  \\
f'(x) & & - & & \bb & & - & \\
f(x) & \h \pI  & \dl & \b{\mI}  & \bb & \h \pI  & \dl & \b{\mI}  \\
\fin

\vspace*{.3cm}

On n'a aucun extremum. Cependant, on a : \\ 

\begin{itemize}
\item[•] Une asymptote verticale : la droite d'équation $x = -1$.
\item[•] Aucun asymptote horizontale.
\item[•] Une asymptote oblique : la droite d'équation $y = -x+1$.
\end{itemize}

\vspace*{-5cm}

\newpage

\textbf{Seconde partie : Représentation graphique} \\

\begin{tikzpicture}[line cap=round,line join=round,>=triangle 45,x=1.0cm,y=1.0cm,scale=0.5]
\draw[->] (-13.91,0) -- (11.78,0);
\foreach \x in {-13,-12,-11,-10,-9,-8,-7,-6,-5,-4,-3,-2,-1,1,2,3,4,5,6,7,8,9,10,11}
\draw[shift={(\x,0)}] (0pt,2pt) -- (0pt,-2pt) node[below] {\footnotesize $\x$};
\draw[->] (0,-11.59) -- (0,14.03);
\foreach \y in {-11,-10,-9,-8,-7,-6,-5,-4,-3,-2,-1,1,2,3,4,5,6,7,8,9,10,11,12,13}
\draw[shift={(0,\y)},color=black] (-2pt,0pt) node[left] {\footnotesize $\y$};
\draw(0pt,-10pt) node[right] {\footnotesize $0$};
\clip(-13.91,-11.59) rectangle (11.78,14.03);
\draw[smooth,samples=100,domain=-13.8:-1.05] plot(\x,{(-(\x)*(\x)+5)/((\x)+1)});
\draw[smooth,samples=100,domain=-0.95:11.67] plot(\x,{(-(\x)*(\x)+5)/((\x)+1)});
\draw [line width=0.4pt,color=red] (-1,-11.59) -- (-1,14.03);
\draw [line width=0.4pt,color=red,domain=-13.91:11.78] plot(\x,{(--1-1*\x)/1});
\begin{scriptsize}
%\draw (-10.45,9.93) node {$f$};
\end{scriptsize}
\begin{pgfonlayer}{background}   
\draw[step=1mm,ultra thin,AntiqueWhite!10] (-13.91,-11.59) grid (11.78,14.03);
\draw[step=5mm,very thin,AntiqueWhite!30]  (-13.91,-11.59) grid (11.78,14.03);
\draw[step=1cm,very thin,AntiqueWhite!50]  (-13.91,-11.59) grid (11.78,14.03);
\draw[step=5cm,thin,AntiqueWhite]          (-13.91,-11.59) grid (11.78,14.03);
\end{pgfonlayer}
\end{tikzpicture}

\newpage


\ifdefined\COMPLETE
\else
    \end{document}
\fi