\ifdefined\COMPLETE
\else
    \input{./preambule-sacha-utf8.ltx}
    \begin{document}
\fi


\subsection{Notion de limite en un point}

\subsubsection{Exemples}

\textbf{Exemple n°1}

\begin{tabular}{llll}
Soit la fonction $f$ : & $\R$ & $\longrightarrow$ & $\R$ \\
& $x$ & $\longmapsto$ & $f\left(x\right) = -x + 1$ \\
\end{tabular}

On a $D_f = \R$.

\begin{tikzpicture}[line cap=round,line join=round,>=triangle 45,x=1.0cm,y=1.0cm]
\draw[->,color=black] (-5.78,0) -- (6.56,0);
\foreach \x in {-5,-4,-3,-2,-1,1,2,3,4,5,6}
\draw[shift={(\x,0)},color=black] (0pt,2pt) -- (0pt,-2pt) node[below] {\footnotesize $\x$};
\draw[->,color=black] (0,-4.74) -- (0,7.98);
\foreach \y in {-4,-3,-2,-1,1,2,3,4,5,6,7}
\draw[shift={(0,\y)},color=black] (2pt,0pt) -- (-2pt,0pt) node[left] {\footnotesize $\y$};
\draw[color=black] (0pt,-10pt) node[right] {\footnotesize $0$};
\clip(-5.78,-4.74) rectangle (6.56,7.98);
\draw [domain=-5.78:6.56] plot(\x,{(--1-1*\x)/1});
\begin{scriptsize}
\draw[color=black] (-5.62,6.4) node {};
\end{scriptsize}
\begin{pgfonlayer}{background}   
\draw[step=1mm,ultra thin,AntiqueWhite!10] (-5.78,-4.74) grid (6.56,7.98);
\draw[step=5mm,very thin,AntiqueWhite!30]  (-5.78,-4.74) grid (6.56,7.98);
\draw[step=1cm,very thin,AntiqueWhite!50]  (-5.78,-4.74) grid (6.56,7.98);
\draw[step=5cm,thin,AntiqueWhite]          (-5.78,-4.74) grid (6.56,7.98);
\end{pgfonlayer} 
\end{tikzpicture}

\vspace*{.3cm}

$f$ est définie en $x_0 = 2$ et $f$ est continue en $x_0 = 2$. Donc, dans ce cas, la fonction admet une limite en $x_0 = 2$ et cette limite est $f(2) = -1$. \\

On écrit $\lim\limits_{x \to 2} f(x) = -1$. 

\newpage

\textbf{Exemple n°2}

\begin{tabular}{llllllll}
Soit la fonction $f$ : & $\R$ & $\longrightarrow$ & $\R$ & & & & \\
& $x$ & $\longmapsto$ & $f\left(x\right)$ & $ = $ & $ -x + 3$ & si & $x \in \left]-\infty \; ; \; 2\right[$ \\
& & & & $=$ & $x - 1$ &  si & $x \in \left]2 \; ; \; +\infty\right[$ \\
\end{tabular}

\vspace*{.3cm}

On a $D_f = \R \setminus \lb 2 \rb $ \\

\begin{tikzpicture}[line cap=round,line join=round,>=triangle 45,x=1.0cm,y=1.0cm]
\draw[->,color=black] (-3.76,0) -- (8.28,0);
\foreach \x in {-3,-2,-1,1,2,3,4,5,6,7,8}
\draw[shift={(\x,0)},color=black] (0pt,2pt) -- (0pt,-2pt) node[below] {\footnotesize $\x$};
\draw[->,color=black] (0,-1.52) -- (0,11.32);
\foreach \y in {-1,1,2,3,4,5,6,7,8,9,10,11}
\draw[shift={(0,\y)},color=black] (2pt,0pt) -- (-2pt,0pt) node[left] {\footnotesize $\y$};
\draw[color=black] (0pt,-10pt) node[right] {\footnotesize $0$};
\clip(-3.76,-1.52) rectangle (8.28,11.32);
\draw [domain=-3.7600000000000007:2.0] plot(\x,{(-6--2*\x)/-2});
\draw [domain=2.0:8.280000000000001] plot(\x,{(-2--2*\x)/2});
\begin{scriptsize}
\draw[color=black] (0.82,1.94) node {};
\draw[color=black] (3.28,1.94) node {};
\end{scriptsize}
\begin{pgfonlayer}{background}   
\draw[step=1mm,ultra thin,AntiqueWhite!10] (-3.76,-1.52) grid (8.28,11.32);
\draw[step=5mm,very thin,AntiqueWhite!30]  (-3.76,-1.52) grid (8.28,11.32);
\draw[step=1cm,very thin,AntiqueWhite!50]  (-3.76,-1.52) grid (8.28,11.32);
\draw[step=5cm,thin,AntiqueWhite]          (-3.76,-1.52) grid (8.28,11.32);
\end{pgfonlayer} 
\end{tikzpicture}

\vspace*{.3cm}

$f$ n'est pas définie en $x_0 = 2$, donc $f$ n'est pas continue en $x_0 = 2$. \\

Dans cet exemple, la fonction admet une limite en $x_0 = 2$ et cette limite est $1$. \\

On écrit $\lim\limits_{x \to 2} f(x) = 1$. 

\newpage

\textbf{Exemple n°3}

\begin{tabular}{llllllll}
Soit la fonction $f$ : & $\R$ & $\longrightarrow$ & $\R$ & & & & \\
& $x$ & $\longmapsto$ & $f\left(x\right)$ & $ = $ & $ -x+1$ & si & $x \in \left]-\infty \; ; \; 2\right[$ \\
& & & & $=$ & $x + 1$ &  si & $x \in \left]2 \; ; \; +\infty\right[$ \\
\end{tabular}

On a $D_f = \R \setminus \lb 2 \rb $ \\

\begin{tikzpicture}[line cap=round,line join=round,>=triangle 45,x=1.0cm,y=1.0cm]
\draw[->,color=black] (-3.21,0) -- (6.97,0);
\foreach \x in {-3,-2,-1,1,2,3,4,5,6}
\draw[shift={(\x,0)},color=black] (0pt,2pt) -- (0pt,-2pt) node[below] {\footnotesize $\x$};
\draw[->,color=black] (0,-3.4) -- (0,8.31);
\foreach \y in {-3,-2,-1,1,2,3,4,5,6,7,8}
\draw[shift={(0,\y)},color=black] (2pt,0pt) -- (-2pt,0pt) node[left] {\footnotesize $\y$};
\draw[color=black] (0pt,-10pt) node[right] {\footnotesize $0$};
\clip(-3.21,-3.4) rectangle (6.97,8.31);
\draw [domain=-3.2090934321912705:2.0] plot(\x,{(-2--2*\x)/-2});
\draw [domain=2.0:6.972718154808185] plot(\x,{(--1--1*\x)/1});
\begin{scriptsize}
\draw[color=black] (0.85,-0.06) node {};
\draw[color=black] (2.75,3.45) node {};
\end{scriptsize}
\begin{pgfonlayer}{background}   
\draw[step=1mm,ultra thin,AntiqueWhite!10] (-3.21,-3.4) grid (6.97,8.31);
\draw[step=5mm,very thin,AntiqueWhite!30]  (-3.21,-3.4) grid (6.97,8.31);
\draw[step=1cm,very thin,AntiqueWhite!50]  (-3.21,-3.4) grid (6.97,8.31);
\draw[step=5cm,thin,AntiqueWhite]          (-3.21,-3.4) grid (6.97,8.31);
\end{pgfonlayer} 
\end{tikzpicture}

\vspace*{.3cm}

$f$ n'est pas définie en $x_0 = 2$, $f$ n'est donc pas continue en $x_0 = 2$. \\

Dans cet exemple, la fonction $f$ n'admet pas une limite en $x_0 = 2$. \\

On note $\lim\limits_{x \to 2} f(x)$ n'existe pas. \\

Cependant : 

\begin{itemize}
\item[•] $f$ admet une limite à gauche en $x_0 = 2$, notée $\lim\limits_{\substack{x \to 2 \\ x<2}}f(x) = -1$. \\
\item[•] $f$ admet une limite à droite en $x_0 = 2$, notée $\lim\limits_{\substack{x \to 2 \\ x>2}} f(x) = 3$. \\
\end{itemize}

\textbf{Remarque :} Ici, $\lim\limits_{\substack{x \to 2 \\ x<2}}f(x) \neq \lim\limits_{\substack{x \to 2 \\ x>2}} f(x) $. \\

Néanmoins, si $\lim\limits_{\substack{x \to 2 \\ x<2}}f(x) = \lim\limits_{\substack{x \to 2 \\ x>2}} f(x)$, alors on écrit $\lim\limits_{x \to 2} f(x) = \lim\limits_{\substack{x \to 2 \\ x<2}}f(x) = \lim\limits_{\substack{x \to 2 \\ x>2}} f(x)$.

\newpage

\subsubsection{Exercice}

\begin{tabular}{llllllll}
Soit la fonction $f$ : & $\R$ & $\longrightarrow$ & $\R$ & & & & \\
& $x$ & $\longmapsto$ & $f\left(x\right)$ & $ = $ & $ x + 2$ & si & $x \in \left]-\infty \; ; \; -2\right[$ \\
& & & & $=$ & $-x - 2$ &  si & $x \in \left]-2 \; ; \; 2\right[$ \\
& & & & $=$ & $x + 2$ &  si & $x \in \left]2 \; ; \; +\infty\right[$ \\
\end{tabular}

\vspace*{.3cm}

On a $D_f = \R \setminus \lb -2 \; ; \; 2 \rb$. \\

\begin{tikzpicture}[line cap=round,line join=round,>=triangle 45,x=1.0cm,y=1.0cm]
\draw[->,color=black] (-6.67,0) -- (6.6,0);
\foreach \x in {-6,-5,-4,-3,-2,-1,1,2,3,4,5,6}
\draw[shift={(\x,0)},color=black] (0pt,2pt) -- (0pt,-2pt) node[below] {\footnotesize $\x$};
\draw[->,color=black] (0,-5.69) -- (0,8.25);
\foreach \y in {-5,-4,-3,-2,-1,1,2,3,4,5,6,7,8}
\draw[shift={(0,\y)},color=black] (2pt,0pt) -- (-2pt,0pt) node[left] {\footnotesize $\y$};
\draw[color=black] (0pt,-10pt) node[right] {\footnotesize $0$};
\clip(-6.67,-5.69) rectangle (6.6,8.25);
\draw [domain=-6.667491653245479:-2.0] plot(\x,{(-4-2*\x)/-2});
\draw (-2,0)-- (2,-4);
\draw [domain=2.0:6.596313424673678] plot(\x,{(--4--2*\x)/2});
\begin{scriptsize}
\draw[color=black] (-3.17,-0.64) node {};
\draw[color=black] (-0.16,-2.06) node {};
\draw[color=black] (3.27,4.94) node {};
\end{scriptsize}
\begin{pgfonlayer}{background}   
\draw[step=1mm,ultra thin,AntiqueWhite!10] (-6.67,-5.69) grid (6.6,8.25);
\draw[step=5mm,very thin,AntiqueWhite!30]  (-6.67,-5.69) grid (6.6,8.25);
\draw[step=1cm,very thin,AntiqueWhite!50]  (-6.67,-5.69) grid (6.6,8.25);
\draw[step=5cm,thin,AntiqueWhite]          (-6.67,-5.69) grid (6.6,8.25);
\end{pgfonlayer} 
\end{tikzpicture}

\begin{itemize}
\item[•] On a $\lim\limits_{x \to -2} f(x) = 0$. \\

\item[•] $\lim\limits_{x \to 2} f(x)$ n'existe pas. Cependant : \\
\end{itemize}

$\lim\limits_{\substack{x \to 2 \\ x<2}}f(x) = -4$. \\

$\lim\limits_{\substack{x \to 2 \\ x>2}} f(x) = 4$. \\

\newpage

\vspace*{-1.5cm}

\subsubsection{Deux exemples de calcul de limites}

\textbf{Exemple n°1}

\begin{tabular}{llll}
Soit la fonction $f$ & $\R$ & $\longrightarrow$ & $\R$ \\
& $x$ & $\longmapsto$ & $f(x) = \dfrac{2x^2 -x - 1}{x-1}$ \\
\end{tabular}

\vspace*{.3cm}

Il ne faut pas que $x-1 = 0 \Longleftrightarrow x = 1$. \\

$D_f = \left]-\infty \; ; \; 1 \right[\cup\left]1 \; ; \; +\infty\right[$ \\

On cherche $\lim\limits_{x \to 1} f(x)$ \\

$\lim\limits_{x \to 1} (2x^2 -x - 1) = 0$ et $\lim\limits_{x \to 1} (x-1) = 0$. \\

\vspace*{.3cm}

Cependant, $f(x) = \dfrac{(x-1)(2x+1)}{x-1} = 2x+1$ si $x \neq 1$. \\

Donc $\lim\limits_{x \to 1} f(x) = \lim\limits_{x \to 1} (2x  + 1) = 3$ \\

\textbf{Exemple n°2}

\begin{tabular}{llll}
Soit la fonction $f$ & $\R$ & $\longrightarrow$ & $\R$ \\
& $x$ & $\longmapsto$ & $f(x) = \dfrac{\sqrt{x+19} - 5}{x+6}$ \\
\end{tabular}

\vspace*{.3cm}

Il faut que $x + 19 \geqslant 0 \Longleftrightarrow x \geqslant -19$. \\
Il ne faut pas que $x - 6 = 0 \Longleftrightarrow x = 6$. \\

$D_f = \left[-19 \; ; \; 6 \right[\cup \left]6 \; ; \; +\infty\right[$. \\

On cherche $\lim\limits_{x \to 6} f(x)$ \\

$\lim\limits_{x \to 6} \left(\sqrt{x + 19} - 5\right) = 0$ et $\lim\limits_{x \to 6} (x-6) = 0$ \\

\begin{tabular}{llll}
Cependant, $f(x)$ & $=$ & $\dfrac{\sqrt{x + 19} - 5}{x -6} $ & \vspace*{.3cm} \\
& $=$ & $\dfrac{\left(\sqrt{x + 19} - 5\right)\left(\sqrt{x + 19} + 5\right)}{\left(x-6\right)\left(\sqrt{x+19} + 5\right)}$ & \vspace*{.3cm} \\
& $=$ & $\dfrac{x + 19 - 25}{\left(x-6\right)\left(\sqrt{x+19} + 5\right)}$ & \vspace*{.3cm} \\
& $=$ & $\dfrac{x + 19 - 25}{\left(x-6\right)\left(\sqrt{x+19} + 5\right)}$ & \\
\vspace*{.3cm} \\
& $=$ & $\dfrac{x - 6}{\left(x-6\right)\left(\sqrt{x+19} + 5\right)}$ & \vspace*{.3cm} \\
& $=$ & $\dfrac{1}{\sqrt{x+19} + 5}$ & si $x \neq 6 $ \vspace*{.3cm} \\
\end{tabular}

D'où $\lim\limits_{x \to 6} f(x) = \lim\limits_{x \to 6} \dfrac{1}{\sqrt{x+19} + 5} = \dfrac{1}{10}$


\ifdefined\COMPLETE
\else
    \end{document}
\fi