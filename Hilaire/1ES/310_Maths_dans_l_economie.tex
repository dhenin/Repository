\ifdefined\COMPLETE
\else
    \input{./preambule-sacha-utf8.ltx}
    \usepackage{variations}
    \begin{document}
\fi


\setcounter{section}{0} 

\part{Mathématiques appliquées à l'économie}

\section{Coût total, coût marginal et coût moyen}

\subsection{Première partie}

Le coût total de fabrication de $x$ objets est donnée par : $CT(x) = -0,01x^2 + 100x + 2000$. \\
$CT(x)$ est exprimé en euros. \\

\begin{itemize}
\item[1.] 
\begin{itemize}
\item[a)] Déterminer le coût total de fabrication de 1000 objets puis de 1001 objets. 
\item[b)] Déterminer \hbox{l'augmentation du coût total entraînée par la fabrication de cet objet supplémentaire.}
\end{itemize}
\item[2.]
\begin{itemize}
\item[a)] Exprimer en fonction de $x$ la différence $CT(x+1) - CT(x)$. \\ Ce nombre, noté $Cm(x)$, représente l'augmentation du coût total entraînée par la fabrication d'un objet supplémentaire lorsque l'on en a fabriqué $x$. Il est appelé \textbf{coût marginal} de $x$.
\item[b)] Déterminer $CT'(x)$. ($CT'$ est la fonction dérivée de $CT$). 
\end{itemize}
\item[3.] On suppose que $Cm(x) = CT'(x)$. 
\begin{itemize}
\item[a)] Quelle est l'erreur commise ?
\item[b)] Vérifier le résultat pour 1000 objets. 
\item[c)] Exprimer l'erreur commise en pourcentage relatif au coup marginal pour 1000 objets.
\end{itemize}
\end{itemize}

\vspace*{.3cm}

\textbf{Conclusion :} \\ Dans \hbox{la pratique, pourvu que $x$ soit assez grand, on assimile le \textbf{coût marginal} à la \textbf{dérivée du coût total}.}

\vspace*{.3cm}

\begin{itemize}
\item[1.] 
\begin{itemize}
\item[a)] $CT(1000) = 92 000$ € et $CT(1001) = 92 079,99$ €. \\
\item[b)] $CT(1001) - CT(1000) = \dfrac{7999}{100} = 79,99$ €. \\ Donc produire un objet de plus coûte 79,99 €. \\
\end{itemize}
\item[2.] 
\begin{itemize}
\item[a)] $CT(x+1) - CT(x) = -0,02x + 99,99$. \\ Donc $Cm(x) = -0,02x + 99,99$. \\
\item[b)] On a $CT(x) = -0,01x^2 + 100x + 2000$. \\ Donc $CT'(x) = -0,02x + 100$. \\
\end{itemize}
\item[3.] 
\begin{itemize}
\item[a)] L'erreur est donc de $0,01$ € $ = 1$ centime \\
\item[b)] $CT'(1000) = 80$ et $Cm'(1000) = 79,99$. \\  
\item[c)] $\dfrac{CT'(1000) - Cm(1000)}{Cm(1000)} \times 100 = \dfrac{80 - 79,99}{79,99} \times 100 \approx 0,0125$ \%. \vspace*{.3cm} \\ L'erreur commise est donc de $0,0125$ \% du coût marginal. 
\end{itemize}
\end{itemize}

\newpage

\subsection{Deuxième partie}

\textbf{Coût total $\mathbf{CT(x)}$} \\

Le coût total de fabrication de $x$ centaines d'objets est donné par : $CT(x) = x^3 - 12x^2 + 48x$. \\
$CT(x)$ est exprimé en milliers d'euros. \\

Étudier les variations de la fonction $CT$ sur l'intervalle $\left[0\; ; \; 8\right]$. \\

Représenter graphiquement la fonction $CT$ sur l'intervalle $\left[0\; ; \; 8\right]$ dans un repère $\left(O \; ; \; \overrightarrow{i} \; ; \; \overrightarrow{j}\right)$ \\

Définition du repère $\left(O \; ; \; \overrightarrow{i} \; ; \; \overrightarrow{j}\right)$ : \\

$\left(O \; ; \; \overrightarrow{i} \; ; \; \overrightarrow{j}\right)$ est un repère orthogonal défini par : sur l'axe des abscisses, 1 cm correspond à 1 et sur l'axe des ordonnées, 1 cm correspond à 10. \\

\textbf{Coût marginal $\mathbf{Cm(x)}$}. \\

La production étant importante, on assimile le coup marginal à la dérivée du coup total. \\

\begin{itemize}
\item[1.] Déterminer $Cm(x)$. \\
\item[2.] Étudier les variations de la fonction $Cm$ sur l'intervalle $\left[0\; ; \; 8\right]$. \\

Représenter graphiquement la fonction $CT$ sur l'intervalle $\left[0\; ; \; 8\right]$ dans un repère $\left(O \; ; \; \overrightarrow{i} \; ; \; \overrightarrow{j}\right)$ \\

$\left(O \; ; \; \overrightarrow{i} \; ; \; \overrightarrow{j}\right)$ est un repère orthogonal défini par : sur l'axe des abscisses, 1 cm correspond à 1 et sur l'axe des ordonnées, 1 cm correspond à 10. \\
\end{itemize}

\textbf{Coût moyen $\mathbf{CM(x)}$} \\

$CM(x) = \dfrac{CT(x)}{x}$ pour tout $x \neq 0$. \\

\begin{itemize}
\item[1.] Déterminer $CM(x)$. \\
\item[2.] Étudier les variations de la fonction $CM$ sur l'intervalle $\left]0\; ; \; 8\right]$. \\

Représenter graphiquement la fonction $CT$ sur l'intervalle $\left]0\; ; \; 8\right]$ dans un repère $\left(O \; ; \; \overrightarrow{i} \; ; \; \overrightarrow{j}\right)$. On utilisera le même repère que pour la fonction $Cm$ \\
\end{itemize}

Pour quelle valeur $x_0$ de $x$ le coût moyen est-il minimum.  ? \\ Comparer $CM(x_0)$ et $Cm(x_0)$. Que remarque-t-on ? \\ Généraliser ce résultat. \\

\textbf{Conclusion} \\

Le \textbf{coût moyen} est \textbf{minimum} lorsqu'il est \textbf{égal} au \textbf{coût marginal}. 

\newpage

\textbf{Coût total $\mathbf{CT(x)}$} \\

$CT(x) = x^3 - 12x^2 + 48x$. \\

La fonction $CT$ est définie sur $\R$, donc sur $\left[0\ ; ;\; 8\right]$ \\

$CT'(x) = 3x^2 - 24x + 48$. \\

\begin{tabular}{lll}
$CT'(x) = 0$ & $\Longleftrightarrow$ & $3x^2 - 24x + 48 = 0$ \\
& $\Longleftrightarrow$ & $x = 4$. \\
\end{tabular}

\vspace*{.3cm}

On peut en déduire le signe de $CT'(x)$ et les variations de $CT$ : \\

\variations
x & -\infty & & 0 & & & 4 & & & 8 & & +\infty \\
CT'(x) & \ha & \ha & \bg & & + & \z & \; \; \; \; \; \; \; \; \;  + & & \bd & \ha & \ha \\
CT(x) & \hv & \hv &\bg & \b{0} & \tcb & 64 & \ch & \h{128} & \bd & \hv & \hv \\
\fin

\vspace*{.3cm}

Une tangente horizontale passe au point $I\left(4 \; ; \; 64\right)$. \\

\begin{tikzpicture}[line cap=round,line join=round,>=triangle 45,x=1.0cm,y=0.10cm,scale=.8]
\draw[->] (-0.91,0) -- (10.76,0);
\foreach \x in {,2,4,6,8,10}
\draw[shift={(\x,0)}] (0pt,2pt) -- (0pt,-2pt) node[below] {\footnotesize $\x$};
\draw[->] (0,-1) -- (0,150);
\foreach \y in {,20,40,50,60,70,80,100,120,140}
\draw[shift={(0,\y)}] (2pt,0pt) -- (-2pt,0pt) node[left] {\footnotesize $\y$};
\draw(0pt,-10pt) node[right] {\footnotesize $0$};
\clip(-0.91,-1) rectangle (10.76,150);
\draw[smooth,samples=100,domain=0:10] plot(\x,{(\x)*(\x)*(\x)-12*(\x)*(\x)+48*(\x)});
\draw [color=red,domain=0:10.76] plot(\x,{(-0--35*\x)/1});
\draw [<->,color=blue] (3.5,64) -- (4.5,64) ; 
\node [color=blue] at (4,64) [above]  {$I$} ;
\draw [color=red,domain=0:10.76] plot(\x,{(-0--18.33*\x)/1});
\draw [color=green,domain=0:10.76] plot(\x,{(-0--12*\x)/1});
\begin{scriptsize}
\node at (8,120) [right] {$CT$};
\draw [color=red] (6,72)-- ++(-1.5pt,-1.5pt) -- ++(3.0pt,3.0pt) ++(-3.0pt,0) -- ++(3.0pt,-3.0pt);
\draw[color=red] (6,72) node [left]{$T$};
\end{scriptsize}
\begin{pgfonlayer}{background}   
\draw[step=1mm,ultra thin,AntiqueWhite!10] (-0.91,-1) grid (10.76,150);
\draw[step=5mm,very thin,AntiqueWhite!30]  (-0.91,-1) grid (10.76,150);
\draw[step=1cm,very thin,AntiqueWhite!50]  (-0.91,-1) grid (10.76,150);
\draw[step=5cm,thin,AntiqueWhite]          (-0.91,-1) grid (10.76,150);
\end{pgfonlayer}
\end{tikzpicture}

\newpage

\textbf{Coût marginal $\mathbf{Cm(x)}$}. \\

$Cm(x) = CT'(x) = 3x^2 - 24x + 48$. \\

La fonction $Cm$ est définie sur $\R$, donc sur $\left[0\; ; \; 8\right]$. \\

$Cm'(x) = 6x - 24$. \\

\begin{tabular}{lll}
$Cm'(x) = 0$ & $\Longleftrightarrow$ & $6x - 24 = 0$ \\
& $\Longleftrightarrow$ & $x = 4$ \\
\end{tabular}

\vspace*{.3cm}

On peut en déduire le signe de $Cm'(x)$ et la variations de $Cm$ : \\

\variations
x & -\infty & & 0 & & & 4 & & & 8 & & +\infty \\
Cm'(x) & \ha & \ha & \bg & & - & \z & \; \; \; \; \; \; \; \; \;  + & & \bd & \ha & \ha \\
Cm(x) & \hv & \hv &\bg & \h{48} & \dl & \b{0} & \cl & \h{48} & \bd & \hv & \hv \\
\fin

\vspace*{.3cm}

Le point $m(4 \; ; \; 0)$ est un minimum absolu par lequel asse une tangente horizontale. \\

\begin{tikzpicture}[line cap=round,line join=round,>=triangle 45,x=1.0cm,y=.10cm]
\draw[->] (-0.66,0) -- (8.48,0);
\foreach \x in {,1,2,3,4,5,6,7,8}
\draw[shift={(\x,0)}] (0pt,2pt) -- (0pt,-2pt) node[below] {\footnotesize $\x$};
\draw[->] (0,-5) -- (0,60);
\foreach \y in {,10,20,30,40,50,60}
\draw[shift={(0,\y)}] (2pt,0pt) -- (-2pt,0pt) node[left] {\footnotesize $\y$};
\draw(0pt,-10pt) node[right] {\footnotesize $0$};
\clip(-0.66,-12.6) rectangle (9,65);
%\draw[smooth,samples=100,domain=-0:8] plot(\x,{(\x)*(\x)-12*(\x)+48});
\draw[smooth,samples=100,domain=-0:8] plot(\x,{3*(\x)*(\x)-24*(\x)+48});
\draw (7.06,25.8) node[anchor=north west] {C marginal};
%\draw (7.08,11) node[anchor=north west] {C moyen};
\begin{scriptsize}
%\draw [fill=blue] (4,0) circle (1.5pt);
%\draw[<->, >=latex, color=blue] (5.5,12) -- (6.5,12) ;
%\draw[color=blue] (6,12) node [above] {$M$};
\draw[<->, >=latex, color=blue] (3.5,0) -- (4.5,0) ;
\draw[color=blue] (4,0) node [above] {$m$};
\end{scriptsize}
\begin{pgfonlayer}{background}   
\draw[step=1mm,ultra thin,AntiqueWhite!10] (-0.66,-12.6) grid (9,65);
\draw[step=5mm,very thin,AntiqueWhite!30]  (-0.66,-12.6) grid (9,65);
\draw[step=1cm,very thin,AntiqueWhite!50]  (-0.66,-12.6) grid (9,65);
\draw[step=5cm,thin,AntiqueWhite]          (-0.66,-12.6) grid (9,65);
\end{pgfonlayer}
\end{tikzpicture}

Pour $x_0 = 6$, on a : $CM(6) = 12$ et $Cm(6) = 12$. \\ 

D'où $CM(6) = Cm(6)$. \\

\newpage

\textbf{Coût moyen $\mathbf{CM(x)}$} \\

$CM(x) = \dfrac{CT(x)}{x}$ pour tout $x \neq 0$. \\

$CM(x) = \dfrac{CT(x)}{x} = \dfrac{x^3 - 12x^2 + 48x}{x} = \dfrac{x\left(x^2 - 12x + 48\right)}{x} = x^2 - 12x + 48$ pour tout $x \neq 0$. \\

La fonction $CM$ est définie sur $\R^* = \R \setminus \lb 0 \rb$, donc sur $\left]0 \; ; \; 8\right]$. 

$CM'(x) = 2x - 12$. \\

\begin{tabular}{lll}
$CM'(x) = 0$ & $\Longleftrightarrow$ & $2x - 12 = 0$ \\
& $\Longleftrightarrow$ & $2x = 12$ \\
& $\Longleftrightarrow$ & $x = 6$ \\ 
\end{tabular}

\vspace*{.3cm}

\variations
x & -\infty & & 0 & & & 6 & & & 8 & & +\infty \\
CM'(x) & \ha & \ha & \bg & & - & \z & \; \; \; \; \; \; \; \; \;  + & & \bd & \ha & \ha \\
CM(x) & \hv & \hv &\bg & \h{48} & \dl & \b{12} & \cl & \h{16} & \bd & \hv & \hv \\
\fin

\vspace*{.3cm}

$M(6 \; ; \; 12)$ est un minimum absolu, par lequel passe une tangente horizontale. 

\begin{tikzpicture}[line cap=round,line join=round,>=triangle 45,x=1.0cm,y=.10cm]
\draw[->] (-0.66,0) -- (8.48,0);
\foreach \x in {,1,2,3,4,5,6,7,8}
\draw[shift={(\x,0)}] (0pt,2pt) -- (0pt,-2pt) node[below] {\footnotesize $\x$};
\draw[->] (0,-5) -- (0,60);
\foreach \y in {,10,20,30,40,50,60}
\draw[shift={(0,\y)}] (2pt,0pt) -- (-2pt,0pt) node[left] {\footnotesize $\y$};
\draw(0pt,-10pt) node[right] {\footnotesize $0$};
\clip(-0.66,-12.6) rectangle (9,65);
\draw[smooth,samples=100,domain=-0:8] plot(\x,{(\x)*(\x)-12*(\x)+48});
\draw[smooth,samples=100,domain=-0:8] plot(\x,{3*(\x)*(\x)-24*(\x)+48});
\draw (7.06,25.8) node[anchor=north west] {C marginal};
\draw (7.08,11) node[anchor=north west] {C Moyen};
\begin{scriptsize}
%\draw [fill=blue] (4,0) circle (1.5pt);
\draw[<->, >=latex, color=blue] (5.5,12) -- (6.5,12) ;
\draw[color=blue] (6,12) node [above] {$M$};
\draw[<->, >=latex, color=blue] (3.5,0) -- (4.5,0) ;
\draw[color=blue] (4,0) node [above] {$m$};
\end{scriptsize}
\begin{pgfonlayer}{background}   
\draw[step=1mm,ultra thin,AntiqueWhite!10] (-0.66,-12.6) grid (9,65);
\draw[step=5mm,very thin,AntiqueWhite!30]  (-0.66,-12.6) grid (9,65);
\draw[step=1cm,very thin,AntiqueWhite!50]  (-0.66,-12.6) grid (9,65);
\draw[step=5cm,thin,AntiqueWhite]          (-0.66,-12.6) grid (9,65);
\end{pgfonlayer}
\end{tikzpicture}

\newpage

On a aussi : \\

\begin{tabular}{lll}
$CM(x)$ & $=$ & $\dfrac{CT(x)}{x}$ \vspace*{.3cm} \\
$CM'(x)$ & $=$ & $\dfrac{CT'(x)x - CT(x)}{x^2}$ \\
\end{tabular}

\vspace*{.3cm}

\begin{tabular}{lll}
$CM'(x) = 0$ & $\Longleftrightarrow$ & $\dfrac{CT'(x)x - CT(x)}{x^2} = 0$ \vspace*{.3cm} \\
& $\Longleftrightarrow$ & $CT'(x)x - CT(x) = 0$ \vspace*{.3cm} \\
& $\Longleftrightarrow$ & $CT'(x)x = CT(x)$ \vspace*{.3cm} \\
& $\Longleftrightarrow$ & $CT'(x) = \dfrac{CT(x)}{x}$ \vspace*{.3cm} \\
& $\Longleftrightarrow$ & $Cm(x) = CM(x)$ \vspace*{.3cm} \\
\end{tabular}

\textbf{Conclusion} 

Le coût moyen est minimum lorsqu'il est égal au coût marginal. 

\newpage

\subsection{Troisième partie}

Soit $M$ un point d'abscisse $x$ de la représentation graphique de $CT$. \\

Montrer que le coefficient directeur de la droite $\left(OM\right)$ est $CM(x)$. \\

Montrer \hbox{que le coefficient directeur de la tangente en $M$ à la représentation graphique de $CT$ est $Cm(x)$.} \\

Déterminer l'équation de la tangente à la représentation graphique de la fonction $CT$ au point \\ d'abscisse $x_0$. \\

Tracer cette tangente. Que remarque-t-on ? \\

Généraliser ce résultat. \\

Montrer que l'on peut déterminer graphiquement la valeur de $x_0$ de $x$ pour laquelle le coût moyen est minimum. \\

\textbf{Conclusion}

Le \textbf{coût moyen} est \textbf{minimum} lorsque la \textbf{sécante $ \; \mathbf{\left(OM\right)}$} est \textbf{tangente} à la représentation \\ graphique de la fonction $CT$.

\vspace*{.6cm}

\hbox{Équation de la droite tangente au point de la représentation graphique de la fonction $CT$ d'abscisse $x_0 = 6$,} donc au point $I(6 \; ; \; 72)$. \\

$y = CT'(x_0)(x-x_0)+CT(x_0)$ \\
$y = CT'(6)(x-6)+CT(6)$ \\
$y = 12\left(x-6\right)+72$ \\
$ y = 12x$ \\

Généralisation : \\

Soit $M\left(x \; ; \; CT(x)\right)$ appartenant à la représentation graphique de la fonction $CT$. On cherche l'équation de la droite $\left(OM\right)$. \\

La droite $\left(OM\right)$ passe par l'origine, son équation est donc de la forme : $y = ax$. \\

On cherche l'expression du coefficient directeur de la droite : \\

$a = \dfrac{y_M - y_0}{x_M - x_0} = \dfrac{y_M}{x_M} = \dfrac{CT(x)}{x}$ \\

Donc la droite $\left(OM\right)$ a pour équation $ y = \dfrac{CT(x)}{x}x \Longleftrightarrow y = CM(x)x$ \\

\newpage

\section{Problème}

Une entreprise fabrique un certain objet. \\
Le coût total de fabrication est donné par : $CT(x) = 2x^3 -33x^2 + 181,5x$ \\
$x$ exprime une quantité en centaines d'objets et $CT(x)$ un prix en milliers d'euros. \\

\begin{itemize}
\item[1.] Étudier la fonction $CT$ sur l'intervalle $\left[0\; ; \; 10\right]$. La représenter graphiquement dans un repère $\left(O \; ;\; \overrightarrow{i} \; ; \; \overrightarrow{j}\right)$ défini par : en abscisses, 1 cm pour 1 et en ordonnées 1 cm pour 50. \\
\item[2.] Le coût moyen de fabrication est désigné par $CM(x)$. $CM(x) = \dfrac{CT(x)}{x}$ pour tout $x \neq 0$. \\

Déterminer graphiquement le nombre d'objets à fabriquer pour que le coût moyen soit minimum. \\

\item[3.] Étudier la fonction $CM$. \\ Déterminer par le calcul le nombre d'objets à fabriquer pour que le coût moyen soit minimum. \\

\item[4.] Le coût marginal pour $x$ centaines d'objets par désigné par $Cm(x)$. \\ La production étant importante, le coût marginal est assimilé à la dérivée du coût total. \\ Résoudre l'équation $CM(x) = Cm(x)$. \\ 

\item[5.] Montrer que le coût moyen est minimum lorsqu'il est égal au coût marginal. \\ Que vaut ce coût moyen minimum ? \\

\item[6.] Soit $p(x) = -83,1x + 1270,5$. \\ Lorsque $x$ centaines d'objets sont vendues, chacune l'est au prix $p(x)$. \\ Déterminer la recette totale $RT(x)$ pour la vente de $x$ centaines d'objets. \\

\item[7.] La recette marginale pour $x$ centaines d'objets est désignée par $Rm(x)$. \\ La recette marginale est assimilée à la dérivée de la recette totale. \\ Résoudre l'équation $Rm(x) = Cm(x)$. \\

\item[8.] Déterminer le bénéfice $B(x)$ pour la production et la vente de $x$ centaines d'objets. \\

\item[9.] Étudier la fonction $B$. \\ Déterminer le nombre d'objets à fabriquer et à vendre pour que le bénéfice soit maximum. \\

\item[10.] Montrer que le bénéfice est maximum lorsque la recette marginale est égale au coût marginal. \\ Que vaut ce bénéfice maximum ?
\end{itemize}

\newpage

\vspace*{-1.2cm}

$CT(x) = 2x^3 -33x^2 + 181,5x$ 

$CT(x)$ est définie sur $\R$, donc sur $\left[0 \; ; \; 10\right]$. 

\begin{itemize}
\item[1.] $CT'(x) = 6x^2 - 66x + 181,5$. \\
\end{itemize}

\begin{tabular}{lll}
$CT'(x) = 0$ & $\Longleftrightarrow$ & $6x^2 - 66x + 181,5 = 0$ \\
& $\Longleftrightarrow$ & $x = 5,5$ \\
\end{tabular}

On peut en déduire le signe de $CT'(x)$ et les variations de $CT$ : 

\vspace*{.1cm}

\variations
x & -\infty & & 0 & & & 5,5 & & & 10 & & +\infty \\
CT'(x) & \ha & \ha & \bg & & + & \z & \; \; \; \; \; \; \; \; \;  + & & \bd & \ha & \ha \\
CT(x) & \hv & \hv &\bg & \b{0} & \tcb & 332,75 & \ch & \h{515} & \bd & \hv & \hv \\
\fin

\vspace*{.1cm}

Une tangente horizontale passe au point $I\left(5,5 \; ; \; 332,75\right)$. \\

\begin{tikzpicture}[line cap=round,line join=round,>=triangle 45,x=1.0cm,y=.02cm,scale=.6]
\draw[->] (-0.9,0) -- (11.26,0);
\foreach \x in {,2,4,6,8,10}
\draw[shift={(\x,0)}] (0pt,2pt) -- (0pt,-2pt) node[below] {\footnotesize $\x$};
\draw[->] (0,-43.33) -- (0,577.11);
\foreach \y in {,100,200,300,400,500}
\draw[shift={(0,\y)}] (2pt,0pt) -- (-2pt,0pt) node[left] {\footnotesize $\y$};
\draw[color=black] (0pt,-10pt) node[right] {\footnotesize $0$};
\clip(-1,-50) rectangle (11,575);
\draw[smooth,samples=100,domain=-0:11] plot(\x,{2*(\x)*(\x)*(\x)-33*(\x)*(\x)+181.5*(\x)});
\draw [color=red,domain=0:11] plot(\x,{(-0--45.38*\x)/1});
\begin{scriptsize}
\draw [fill=blue] (5.5,332.75) circle (1.5pt) node [above] {$I$} ;
\draw[<->, >=latex, color=blue] (5,332.75) -- (6,332.75) ;
\node [red] at (9.5,400) {$CM$} ; 
\end{scriptsize}
\begin{pgfonlayer}{background}   
\draw[step=1mm,ultra thin,AntiqueWhite!10] (-1,-50) grid (11,575);
\draw[step=5mm,very thin,AntiqueWhite!30]  (-1,-50) grid (11,575);
\draw[step=1cm,very thin,AntiqueWhite!50]  (-1,-50) grid (11,575);
\draw[step=5cm,thin,AntiqueWhite]          (-1,-50) grid (11,575);
\end{pgfonlayer}
\end{tikzpicture}

\begin{itemize}
\item[2.] Tangente à $C_{CT}$ passant par l'origine. 

Graphiquement, on lit que le coût marginal est minimum pour la fabrication de 825 objets. \\

\item[3.] $CM(x) = \dfrac{CT(x)}{x} = 2x^2 -33x + 181,5$. \\

Donc $CM'(x) = 4x - 33$ \\
\end{itemize}

\vspace*{-.3cm}

\begin{tabular}{lll}
$CM'(x) = 0$ & $\Longleftrightarrow$ & $4x - 33 = 0$ \\
& $\Longleftrightarrow$ & $4x = 33$ \\
& $\Longleftrightarrow$ & $x = \dfrac{33}{3}$ \\
& $\Longleftrightarrow$ & $x = 8,25$ \\
\end{tabular}

\vspace*{.3cm}

\variations
x & -\infty & & 0 & & & 4 & & & 8 & & +\infty \\
CM'(x) & \ha & \ha & \bg & & - & \z & \; \; \; \; \; \; \; \; \;  + & & \bd & \ha & \ha \\
CM(x) & \hv & \hv &\bg & \h{181,5} & \dl & \b{45,375} & \cl & \h{51,5} & \bd & \hv & \hv \\
\fin

\vspace*{.3cm}

$m(8,25 \; ; \; 45,375)$ est un minimum absolu. \\

Donc le coût moyen est minimum pour la fabrication de $8,25$ centaines d'objets, donc pour la fabrication de 825 objets. Ce coût moyen minimum est de $45 375$ euros. 

\vspace*{-5cm}

\newpage

\begin{itemize}
\item[4.] $Cm(x) \approx CT'(x) \Longleftrightarrow Cm(x) \approx 6x^2 -66x + 181,5$. \\
\end{itemize}

\vspace*{.3cm}

\begin{tabular}{lll}
$CM(x) = Cm(x)$ & $\Longleftrightarrow$ & $2x^2 - 33x + 181,5 = 6x^2 - 66x + 181,5$ \\
& $\Longleftrightarrow$ & $-4x^2 + 33x = 0$ \\
& $\Longleftrightarrow$ & $x\left(-4x + 33\right) = 0$ \\
& $\Longleftrightarrow$ & $x = 0$ ou $-4x + 33 = 0$ \\
& $\Longleftrightarrow$ & $x = 0$ ou $ -4x = -33$ \\
& $\Longleftrightarrow$ & $x = 0$ ou $ x = 8,25$ \\
\end{tabular}

\vspace*{.3cm}

Or, $x \neq 0$, donc $x = 8,25$. \\

\begin{itemize}
\item[5.] Ainsi, le coût moyen est minimum lorsqu'il est égal au coût marginal. Ici, on a $x = 8,25$. \\

\item[6.] $p(x) = -83,1x + 1270,5$ \\

$RT(x) = xp(x) = x\left(-83,1x + 1270,5\right) = -83,1x^2 + 1270,5x$ \\

\item[7.] On a $Rm(x) = RT'(x) \Longleftrightarrow Rm(x) = -166,2x 1270,5$ \\
\end{itemize}

\begin{tabular}{lll}
$Rm(x) = Cm(x) $ & $\Longleftrightarrow$ & $-166,2x 1270,5 = 6x^2 -66x + 181,5$ \\
& $\Longleftrightarrow$ & $-6x^2 - 100,2x + 1089 = 0$ \\
& $\Longleftrightarrow$ & $x = -94,2$ ou $x = 7,5$. \\
\end{tabular}

\vspace*{.3cm}

\begin{itemize}
\item[8.] Déterminer le bénéfice $B$. \\
\end{itemize}

\vspace*{.3cm}

\begin{tabular}{lll}
$B(x)$ & $ = $ & $RT(x) - CT(x)$ \\
& $=$ & $\left(-83,1x^2 + 1270x\right) -\left(2x^3 -33x^2 + 181,5x\right)$ \\
& $=$ & $-2x^3 -50,1x^2 + 1099x$ \\
\end{tabular}

\vspace*{.3cm}

\begin{itemize}
\item[9.] $B'(x) = -6x^2 - 100,2x + 1089$ \\
\end{itemize}

\begin{tabular}{lll}
$B'(x) = 0$ & $\Longleftrightarrow$ & $-6x^2 - 100,2x + 1089 = 0$ \\
& $\Longleftrightarrow$ & $x = -24,2$ ou $x = 7,5$. \\
\end{tabular}

\vspace*{.3cm}

\variations
x & -\infty & & -24,2 & & & 0 & & & 7,5 & & & 10 & & +\infty \\
6x^2 - 100,2x + 1089 & &  - & \z & \; \; \; + & & \bg & & + & \z & \; \; \; \; \; \; \; \; - & & \bd & \; \; \; \; \; \; \; \; \; \; \; \; \;  -  \\
B'(x) & \ha & \ha & \ha & \ha & \ha & \bg & & + & \z & \; \; \; \; \; \; \; \; \; - & & \bd & \ha & \ha \\
B(x) & \hv & \hv & \hv & \hv & \hv & \bg & \b{0} & \cl & \h{4\; 505 \; 625} & \dl & \b{3880} & \bd & \hv & \hv  \\
\fin

\vspace*{.3cm}

\begin{itemize}
\item[10.] Le bénéfice est donc maximum pour la fabrication et la vente de 750 objets. Ce bénéfice est alors  de $4 \; 505 \; 625$ euros. \\ Le bénéfice est maximum lorsque la recette marginale est égale au coût marginal. Ici, $x = 7,5$.
\end{itemize}

\ifdefined\COMPLETE
\else
    \end{document}
\fi