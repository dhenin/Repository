\ifdefined\COMPLETE
\else
    \input{./preambule-sacha-utf8.ltx}
    \begin{document}
\fi


\setcounter{section}{0} 

\part{Pourcentages}

\section{Exercice \no 0}

Remplir les cases vides... \\

\begin{tabular}{c|c|c|c} 
Prix HT en € & TVA en \% & Prix TT en € & Coefficient multiplicateur \\
\hline
140 & 5,5 & $\textcolor{red}{147,70}$ & $\textcolor{red}{\times 1,055}$ \\
$\textcolor{red}{1500}$& 19,6 & 1794 & $\textcolor{red}{: 1,196}$ \\
1450 & $\textcolor{red}{33}$ & 1928,50 & $\textcolor{red}{1,33}$ \\
\end{tabular} \\

Même consigne... \\

\begin{tabular}{c|c|c|c} 
Ancien prix en € & réduction en \% & Nouveau prix en € & Coefficient multiplicateur \\
\hline
220 & 15 & $\textcolor{red}{187}$ & $\textcolor{red}{\times 0,85}$ \\
$\textcolor{red}{1790}$& 35 & 1163,50 & $\textcolor{red}{: 0,65}$ \\
12300 & $\textcolor{red}{0,1}$ & 12287,70 & $\textcolor{red}{0,999}$ \\
\end{tabular} \\

Ainsi, on peut dire que : 

\begin{enumerate}
\item[*] Ajouter 12 \% à $p$ revient à dire que : $ p + \dfrac{12}{100}p = p + 0,12p = p \left(1 + 0,12\right) = 1,12 p$.
\item[*] Retrancher 12 \% à $p$ revient à dire que : $ p - \dfrac{12}{100}p = p - 0,12p = p \left(1 - 0,12\right) = 0,88 p$.
\end{enumerate}

Enfin, le pourcentage d'évolution se calcule ainsi : \\

Soient $V_D$ la valeur de départ et $V_A$ la valeur d'arrivée. \\

$V_A = V_D \left(1 + \dfrac{t}{100} \right) $ \\

$ 1 + \dfrac{t}{100} = \dfrac{V_A}{V_D} $ \\

$ \dfrac{t}{100} = \dfrac{V_A}{V_D} - 1 $ \\

$ \dfrac{t}{100} = \dfrac{V_A - V_D}{V_D} $ \\

$ t = \dfrac{V_A - V_D}{V_D} \times 100 $ \\

En reprenant le premier tableau, on peut dire que : \\

$ t = \dfrac{1928,50 - 1450}{1450} \times 100 = 33 $ \\

Le prix a donc augmenté de 33 \%. \\

Puis, avec le second tableau :  \\

$ t = \dfrac{12287,70 - 12300}{12300} \times 100 = -0,1 $ \\

Le prix a donc diminué de 0,1 \%.

\newpage

\section{Exercice \no 1}

\subsection*{Première partie}

Un prix $p$ augmente de $20$ \% puis de $10$ \%. \\

A-t-il augenté de $30$ \% ? \\

\begin{enumerate}
\item[*] Prix initial : $p$
\item[*] Prix intermédiaire : $1,2p$
\item[*] Prix final : $1,1 \times 1,2p = 1,32 p $. 
\end{enumerate} 

\vspace{.3cm}

$p$ a donc augmenté de 32 \%.

\subsection*{Deuxième partie}

Un prix $p$ diminue de $20$ \% puis de $10$ \%. \\

A-t-il diminué de $30$ \% ? 

A-t-il diminué de $32$ \% ? \\

\begin{enumerate}
\item[*] Prix initial : $p$
\item[*] Prix intermédiaire : $0,8p$
\item[*] Prix final : $0,9 \times 0,8p = 0,72 p $. 
\end{enumerate} 

\vspace{.3cm}

$p$ a donc diminué de 28 \%.

\subsection*{Troisième Partie}

Un prix $p$ augmente de $10$ \% puis diminue de $10$ \%. \\

$p$ est-il revenu au prix initial ? \\

\begin{enumerate}
\item[*] Prix initial : $p$
\item[*] Prix intermédiaire : $1,1p$
\item[*] Prix final : $0,9 \times 1,1p = 0,99 p $. 
\end{enumerate} 

\vspace{.3cm}

$p$ a donc diminué de $1$ \%.

\newpage

\subsection*{Amusette}

Vous achetez un objet (de valeur) chez un commerçant. Celui-ci propose :

\begin{enumerate}
\item[*] Une réduction de $10$ \% sur le prix HT puis l'application de la TVA à $19,6$ \%.
\item[*] Une réduction de $10$ \% sur le prix TTC.
\end{enumerate}

Quelle option doit-on choisir ? \\

\begin{tabular}{l|c|c}
& Option A & Option B \\
\hline
Prix initial & $p$ & $p$ \\
Prix intermédiaire & $0,9p$ & $1,196p$ \\
Prix final & $1,196 \times 0,9p = 1,0764p $ & $1,196 \times 0,9p = 1,0764p $ \\
\end{tabular} 

\vspace*{.3cm}

Le choix est indifférent pour le client. \\

Cependant, le choix n'est pas indifférent au commerçant : \\

\textbf{Exemple} \\

Un objet à 10 000 € : \\

\begin{tabular}{l|c|c}
& Option A & Option B \\
\hline
Client & $10764$ & $10764$ \\
Commerçant & $9000$ & $8804$ \\
État & $1764$ & $1960$ \\
\end{tabular} 

\vspace*{.3cm}

Le commerçant préfèrera donc l'option A.

\newpage

\section{Exercice \no 2}

Un produit A coûte 25 \% plus cher qu'un produit B. \\

Le produit B coûte-t-il 25 \% moins cher que le produit A ? \\

Soient $p_A$ le prix du produit A et $p_B$ le prix du produit B. \\

$p_A = p_B \times 1,25 $ \\

$ p_B = \dfrac{p_A}{1,25} $ \\

$ p_B = p_A \times \dfrac{1}{1,25} $ \\

$ p_B = p_A \times 0,8 $ \\

Donc, le produit B coûte 20 \% moins cher que le pruduit A.

\section{Exercice \no 3}

Le taux de TVA dans la restauration était de 19,6 \%. Ce taux a été baissé à 5,5 \%. \\

Les prix ont-ils donc baissé de 14,1 \% ? \\

Soient $p$ le prix HT. Le prix avec l'ancienne TVA est $p_A = 1,196p$ et le prix avec la nouvelle TVA est $p_N = 1,055p$. \\

%Il faut introduire l'accolade en dessous.

On a donc 

$p_A = 1,196p$
$ p_N = 1,055 p$ \\

Ainsi : \\

$p_N = p \times 1,055$ \\

$ p_N = \dfrac{p_A}{1,196} \times 1,055 $ \\

$ p_N = p_A \times \dfrac{1,055}{1,196} $ \\

$ p_N \approx p_A  \times 0,882 $ \\

Les prix auraient donc dû baisser de 11,8 \% \\

\textbf{Vérification} 

On dit que $p = 10$€ \\

$ t = \dfrac{10,55 - 11,96}{11,96} \times 100 = - 11,8$ \%

\newpage

\section{Exercice \no 4}

\subsection*{Première partie}

On place de l'argent à un taux de 1 \% par mois, le taux est-il de 12 \% par an ? \\

Soient $p$ la somme d'argent initiale, $p'$ la somme d'argent au bout d'un mois, et $p''$ la somme d'argent au bout de 2 mois. \\

\begin{tabular}{ll}
Placement initial & $p$ \\
Placement au bout d'un mois & $p' = 1,01p$ \\
Placement au bout de deux mois & $p'' = p' \times 1,01 = 1,01 \times 1,01 \times p = 1,01^2 \times p $ \\
... & \\
Placement au bout de 12 mois & $ p \times 1,01^{12} = 1,1268p $ \\
\end{tabular}

\vspace{.3cm}

Le placement a donc augmenté de 12,68 \%.

\subsection*{Seconde partie}

Un placement au taux de 12 \% par an revient-il à un placement de 1 \% par mois ? \\

Soient $p$ la somme d'argent initiale, $p'$ la somme d'argent au bout d'un an. \\

\begin{tabular}{ll}
Placement initial & $p$ \\
Placement au bout d'un an & $p' = 1,12p$ \\
\end{tabular}

\vspace{.3cm}

\textbf{Une idée géniale}

On peut dire que pour tout $a \geqslant 0 $, $\sqrt{a} = a^{\frac{1}{2}}$. \\

En effet, $\left(\sqrt{a}\right)^2 = a$, et donc $\left(a^{\frac{1}{2}}\right)^2 = a^{\frac{1}{2} \times 2} = a$. \\

Par exemple : $\sqrt{9} = 9^{\frac{1}{2}} = 3$ \\

On peut dire ici que la placement au bout d'un mois est : $p \times 1,12^{\frac{1}{12}} = p \times 1,0095 $. \\

Le placement revient donc à 0,95 \% par an.

\newpage

\section{Exercice \no 5 : Moyenne de pourcentages}

\subsection{Première partie}

Le prix $p$ d'un objet augmente de 20 \% la première année puis de 10 \% la deuxième année. \\ $p$ a-t-il augmenté en moyenne de 15 \% ? \\

Soit $t$ le pourcentage recherché. \\

\begin{tabular}{l|l}
Réalité & Fiction : en mathématiques \\
\hline
Prix initial : p & Prix initial : p \\
Au bout d'un an : $p \times 1,20$ & $p \left(1 + \dfrac{t}{100}\right)$ \\
Au bout de deux ans : $\left(p \times 1,20 \right) \times 1,1 = p \times 1,32$ & $\left[p\left(1 + \dfrac{t}{100}\right)\right]\left(1 + \dfrac{t}{100}\right) = p\left(1 + \dfrac{t}{100}\right)^2$ \\
\end{tabular}

\vspace*{.3cm}

Il vient : $p\left(1 + \dfrac{t}{100}\right)^2 = p \times 1,32$ \\

D'où $\left(1 + \dfrac{t}{100}\right)^2 = 1,32$ \\

$ 1 + \dfrac{2t}{100} + \dfrac{t^2}{10 000} = 1,32$ \\

$\dfrac{t^2}{10 000} + \dfrac{2t}{100} - 0,32 = 0$ \\

$ t^2 + 200t - 3200 = 0$ \\

$t = -214,89$ ou $t = 14,89$ \\

Or, $t = -214,89$ ne convient pas, car $t$ est un pourcentage d'augmentation, donc $t \in \left[0 \; ; \; 100 \right]$. \\

Ainsi, une augmentation de 20 \% suivie d'une augmentation de 10 \% est à peu près équivalente à une augmentation de 14,89 \% suivie d'une autre augmentation de 14,89 \%.

\newpage

\subsection{Seconde partie}

Le prix $p$ d'un objet baisse de 20 \% la première année puis de 10 \% la deuxième année. \\ $p$ a-t-il diminué en moyenne de 15 \% ? \\ $p$ a-t-il diminué en moyenne de 14,89 \% ?

Soit $t$ le pourcentage recherché. \\

\begin{tabular}{l|l}
Réalité & Fiction : en mathématiques \\
\hline
Prix initial : p & Prix initial : p \\
Au bout d'un an : $p \times 0,80$ & $p \left(1 + \dfrac{t}{100}\right)$ \\
Au bout de deux ans : $\left(p \times 0,80 \right) \times 0,90 = p \times 0,72$ & $\left[p\left(1 + \dfrac{t}{100}\right)\right]\left(1 + \dfrac{t}{100}\right) = p\left(1 + \dfrac{t}{100}\right)^2$ \\
\end{tabular}

\vspace*{.3cm}

Il vient : $p\left(1 + \dfrac{t}{100}\right)^2 = p \times 0,72$ \\

D'où $\left(1 + \dfrac{t}{100}\right)^2 = 0,72$ \\

$ 1 + \dfrac{2t}{100} + \dfrac{t^2}{10 000} = 0,72$ \\

$\dfrac{t^2}{10 000} + \dfrac{2t}{100} + 0,28 = 0$ \\

$ t^2 + 200t + 2800 = 0$ \\

$t = 184,85$ ou $t = -15,15$ \\

Or, $t = 184,85$ ne convient pas, car $t$ est un pourcentage de diminution, donc $t \in \left[-100 \; ; \; 0 \right]$. \\

Ainsi, une diminution de 20 \% suivie d'une diminution de 10 \% est à peu près équivalente à une diminution de 15,15 \% suivie d'une autre diminution de 15,15 \%. \\

\newpage

\vspace*{-2cm}

\section{Exercice \no 6}

Un véhicule dont le prix initial étaient de $12 000$ € ) subi une première baisse de $t$ \% puis une seconde baisse de $t$ \%. Son prix est maintenant de $10 267,50$ \%. \\ Déterminer $t$. \\

\begin{tabular}{ll}
Prix initial : & $12 000$ \\
Prix intermédiaire : & $12 000 \left( 1 + \dfrac{t}{100}\right)$ \\
Prix final : & $12 000 \left( 1 + \dfrac{t}{100}\right)^2$ \\
\end{tabular}

\vspace*{.3cm}

Il vient $ 12 000 \left(1 + \dfrac{t}{100}\right)^2 = 10 267,50$ \\ 

$12 000 \left(1 + \dfrac{2t}{100} + \dfrac{t^2}{10 000}\right) = 10 267,50$ \\

$12 000 + \dfrac{24 000t}{100} + \dfrac{12 000 t^2}{10 000} = 10 267,50$ \\

$\dfrac{12 000 t^2}{10 000} + \dfrac{24 000t}{100} + 1732,50 = 0$ \\

$12 000t^2 + 2 400 000t + 17 325 000 = 0$ \\

$ t = \dfrac{-385}{2}$ ou $t = -\dfrac{15}{2}$ \\

Or, $t = \dfrac{-385}{2}$ ne convient pas, car $t$ est un pourcentage de diminution, donc $t \in \left[-100 \; ; \; 0 \right]$. \\

D'où $t = -7,5$. \\

Le prix de la voiture a donc baissé de 7,5 \%.

\newpage

\section{Exercice \no 7}

Sylvain place de l'argent sur un compte rémunéré à $t$ \% pendant deux ans. La première année, il place 3 000 €, puis la seconde, il place 2 000 €. \\
Le capital de Sylvain s'élève à $5 407,50$ € au bout de ces 2 années. \\ 
Déterminer $t$. \\ 

On veut résoudre l'équation : $3000 \left(1 + \dfrac{t}{100}\right)^2 + 2000\left(1 + \dfrac{t}{100}\right) = 5407,5$ \\

$3000 \left(1 + \dfrac{2t}{100} + \dfrac{t^2}{10000}\right) + 2000 + \dfrac{2000t}{100} = 5407,5$ \\

$3000 + 60t + 0,3t^2 + 2000 + 20t - 5407,5 = 0$ \\

$ 0,3t^2 + 80t - 407,5 = 0$ \\

$ t = 5 $ ou $t = -\dfrac{815}{3}$ \\

Or, $t$ est un pourcentage de placement, donc un pourcentage d'augmentation. \\ Donc $t = -\dfrac{815}{3}$ ne convient pas, car $t \in \left[0 \; ; \; 100 \right]$. \\

Ainsi, le taux de placement sur le compte de Sylvain est de 5 \%. 


\ifdefined\COMPLETE
\else
    \end{document}
\fi