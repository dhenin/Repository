\setcounter{section}{0}

\part{Suites}

\section{Introduction}

\subsection{Définition}

On appelle « suite numérique » une fonction de $\N$ (ou d'une partie de $\N$) dans $\R$ : \\

\begin{tabular}{lllll}
Soit une fonction $f$ : & $\N$ & $\longrightarrow$ & $\R$ \\
& $n$ & $\longrightarrow$ & $f\left(n\right)$ \\ 
\end{tabular}

\vspace*{.3cm}

Tout élément de $\N$ (ou d'une partie de $\N$) a au plus une image dans $\R$.

\textbf{Remarque :} au plus une image : soit une image, soit aucune. 

\subsection{Notation}

\begin{itemize}
\item[•] La suite $f$ est notée $\left(u_n\right)_{n \in \N}$ ou plus simplement $\left(u_n\right)$. \\

\item[•] Le nombre réel $f\left(n\right)$ est noté $u_n$ et est un \textbf{terme général} de la suite $\left(u_n\right)_{n \in \N}$ \\

\textbf{Attention :} ce n'est pas nécessairement le $n^\mathrm{ième}$ terme de la suite. 
\end{itemize}

\subsection{Les différents types de suite}

\subsubsection{Suite définie par son terme général}

Soit $\left(u_n\right)_{n \in \N}$ la suite définie par $\forall n \in \N, u_n = 2n - 3$ \\

Calculer les termes suivants : \\

\begin{itemize}
\item[•] $u_0 = -3$ ; c'est le premier terme de la suite.
\item[•] $u_1 = -1$ ; c'est le deuxième terme de la suite.
\item[•] $u_2 = 1$ ; c'est le troisième terme de la suite.
\item[•] $u_3 = 3$ ; c'est le quatrième terme de la suite.
\item[•] $u_{1000} = 1997$ ; c'est le mille-et-unième terme de la suite.
\end{itemize}

\vspace*{.3cm}

\textbf{Attention : } $u_{n+1} \neq u_n + 1$ \\

En effet, $u_{n+1} = 2\left(n+1\right) -3 = 2n + 2 - 3 = 2n - 1 $ \\

$u_n + 1 = \left(2n - 3\right) + 1 = 2n - 2$

\newpage

\subsubsection{Suite définie par une relation de récurrence}

Soit $\left(u_n\right)_{n \in \N}$ la suite définie par :$ \; \; \; \begin{cases}
u_0 = 2 \\
\forall n \in \N, u_{n+1} = u_n + 3 \\
\end{cases}$ \\

Calculer les termes suivants : 

\begin{itemize}
\item[•] $u_0 = 2$
\item[•] $ u_1 = u_0 + 3 = 2 + 3 = 5$ 
\item[•] $ u_2 = u_1 + 3 = 5 + 3 = 8$ 
\item[•] $ u_3 = u_2 + 3 = 8 + 3 = 11$
\end{itemize}

\vspace*{.3cm}

On conjecture que : $\forall n \in \N, u_n = 3n + 2$

\subsection{Notion de raisonnement par récurrence}

On cherche à montrer que la formule conjecturée, appelée \textbf{hypothèse de récurrence} est vraie pour tout $n$.

\subsubsection{Montrer que la formule conjecturée est vraie pour la première valeur de $\mathbf{n}$}

$u_0 = 3 \times 0 + 2 = 2$ 

\subsubsection{On démontre que, si la formule conjecturée est vraie pour le terme de rang $\mathbf{n}$, alors elle est vraie pour le terme de rang $\mathbf{n + 1}$.}

Ainsi, $u_{n+1} = 3\left(n+1\right) + 2 = 3n + 5$ \\

L'hypothèse de récurrence est : $u_n = 3n + 2$ \\

\begin{tabular}{lll}
$u_{n+1}$ & $=$ & $u_n + 3$ \\
& $=$ & $\left(3n + 2\right) + 3$ \\
& $=$ & $3n + 5$ \\ 
\end{tabular}

\vspace*{.3cm}

Donc l'hypothèse est vérifiée, et la suite peut se définir par l'expression de son terme général $u_n = 3n + 2$.

\newpage

\vspace*{-2cm}

\subsection{Exercices}

\subsubsection{Exercice \no 1}

Soit $\left(u_n\right)_{n \in \N^*}$ la suite définie par :$ \; \; \; \begin{cases}
u_1 = 0 \\
\forall n \in \N^*, u_{n+1} = \dfrac{1}{2 - u_n} \\
\end{cases}$ \\

Calculer les termes suivants : \\

\begin{itemize}
\item[•] $u_2 = \dfrac{1}{2 - u_1} = \dfrac{1}{2 - 0} = \dfrac{1}{2}$ \vspace*{.3cm} \\
\item[•] $ u_3 = \dfrac{1}{2 - u_2} = \dfrac{1}{2 - \dfrac{1}{2}} = \dfrac{1}{\dfrac{3}{2}} = \dfrac{2}{3}$ \vspace*{.3cm} \\
\item[•] $u_4 = \dfrac{1}{2 - u_3} = \dfrac{1}{2 - \dfrac{2}{3}} = \dfrac{1}{\dfrac{4}{3}} = \dfrac{3}{4}$ \vspace*{.3cm} \\
\item[•] $u_5 = \dfrac{1}{2 - u_4} = \dfrac{1}{2 - \dfrac{3}{4}} = \dfrac{1}{\dfrac{5}{4}} = \dfrac{4}{5}$ \vspace*{.3cm} \\
\end{itemize}

\vspace*{.3cm}

On conjecture que : $\forall n \in \N^*, u_n = \dfrac{n - 1}{n}$ \\

On vérifie que la formule conjecturée est vraie pour la première valeur de $n$, c'est-à-dire $n = 1$ : $u_1 = \dfrac{1 - 1}{1} = 0$ \\

On démontre que si la formule conjecturée est vraie pour le terme de rang $n$, alors elle l'est aussi pour le terme de rang $n + 1$ : \\

Si $u_n = \dfrac{n-1}{n}$, alors $ u_{n+1} = \dfrac{\left(n+1\right) - 1}{n + 1} = \dfrac{n}{n+1}$ \\

D'après l'énoncé, on a : $u_{n+1} = \dfrac{1}{2 - u_n}$. \\


\begin{tabular}{lll}
D'où $u_{n+1}$ & $ = $ & $ \dfrac{1}{2 - \dfrac{n-1}{n}}$ \vspace*{.3cm}  \\
& $ = $ & $\dfrac{1}{\dfrac{2n - \left(n-1\right)}{n}}$ \vspace*{.3cm} \\
& $=$ & $\dfrac{1}{\dfrac{2n - n + 1}{n}}$ \vspace*{.3cm} \\
& $=$ & $\dfrac{1}{\dfrac{n+1}{n}}$ \vspace*{.3cm} \\
& $=$ & $\dfrac{n}{n+1}$ \vspace*{.3cm} \\
\end{tabular}

\vspace*{.3cm}

L'hypothèse est vérifiée, et la suite peut se caractériser par son terme général $u_n = \dfrac{n-1}{n}$.

\newpage
 
\subsubsection{Amusette}

Soit $\left(u_n\right)_{n \in \N}$ la suite définie par :$ \; \; \; \begin{cases}
u_0 = 1 \\
\forall n \in \N, u_{n+1} = 10u_n + 1 - 9n \\
\end{cases}$ \\

\begin{itemize}
\item[1.] Déterminer $u_1$, $u_2$, $u_3$, $u_4$ \\
\item[2.] Conjecturer le terme général de la suite $\left(u_n\right)_{n \in \N}$ \\
\item[3.] Démontrer la conjecture \\
\end{itemize}

\begin{itemize}
\item[•] $u_1 = 10u_0 + 1 - 9 \times 0 = 10 \times 1 + 1 = 11$ \\
\item[•] $u_2 = 10u_1 + 1 - 9 \times 1 = 10 \times 11 + 1 - 9 = 111 - 9 = 102$ \\
\item[•] $u_3 = 10u_2 + 1 - 9 \times 2 = 10 \times 102 + 1 - 18 = 1020 + 1 - 18 = 1021 - 18 = 1003 $ \\
\item[•] $u_4 = 10u_3 + 1 - 9 \times 3 = 10 \times 1003 + 1 - 27 = 10031 - 27 = 10004$ \\
\end{itemize}

2) On conjecture que $u_n = 10^n + n$ \\

3) On vérifie que la valeur est vraie pour la première valeur de $n$, c'est-à-dire $n = 0$ \\

$u_0 = 10^0 + 0 = 1 + 0 = 1 $ \\

On démontre que la formule conjecturée est vraie pour $n + 1$, c'est-à-dire si $u_n = 10^n + n$, alors $u_{n+1} = 10^{n+1} + \left(n+1\right)$ \\

\begin{tabular}{lll}
$u_{n+1}$ & $=$ & $10u_n + 1 - 9n$ \\
& $=$ & $10\left(10^n + n\right) + 1 - 9n$ \\
& $=$ & $10^{n+1} + 10n + 1 - 9n $ \\
& $=$ & $10^{n+1} + n + 1$ \\
\end{tabular}

\subsubsection{À la calculatrice...}

\textbf{Mode d'emploi : }

\begin{itemize}
\item[•] Mode : Normal
\item[•] Suite
\item[•] Non-relié
\end{itemize}

\vspace*{.3cm}

On peut étudier les premiers termes des suites suivantes, en appuyant sur 2nde, puis Table. \\

\begin{tabular}{lll}

\hspace*{-2cm}

Soit $\left(u_n\right)_{n \in \N}$ la suite définie par &

\hspace*{-.6cm}

Soit $\left(u_n\right)_{n \in \N^*}$ la suite définie par &

\hspace*{-.3cm}

Soit $\left(u_n\right)_{n \in \N}$ la suite définie par \\

\hspace*{-2cm}

définie par :$ \; \; \; \begin{cases}
u_0 = 2 \\
\forall n \in \N, u_{n+1} = u_n + 3 \\
\end{cases}$ 

&

\hspace*{-.6cm}

définie par :$ \; \; \; \begin{cases}
u_1 = 0 \\
\forall n \in \N, u_{n+1} = \dfrac{1}{2 - u_n} \\
\end{cases}$ 

&

\hspace*{-.3cm}

définie par :$ \; \; \; \begin{cases}
u_0 = 1 \\
\forall n \in \N, u_{n+1} = 10u_n -9n + 1 \\
\end{cases}$ 

\vspace*{.5cm}

\\

\hspace*{-2cm}

On a :

$n_{min} = 0$ & 

\hspace*{-.6cm}

On a :

$n_{min} = 1$ (car $n \; \in \; \N^*$)

& 

\hspace*{-.3cm}

On a :

$n_{min} = 0$ \\

\hspace*{-1cm}

$u_n = u_{n-1} +3$ & 

\hspace*{.4cm}

$u_n = \dfrac{1}{2 - u_{n-1}}$ & 

\hspace*{.7cm}

$u_n = 10u_{n-1} - 9\left(n-1\right) + 1$  \\

\hspace*{-1cm}

$u_{n_{min}} = 2$ & 

\hspace*{.4cm}

$u_{n_{min}} = 0$ & 

\hspace*{.7cm}

$u_{n_{min}} = 1$ \\

\end{tabular}

\newpage

\subsection{Représentations graphiques de suites}

\subsubsection{Suite définie par son terme général}

Soit la suite $\left(u_n\right)_{n \; \in \; \N}$ définie par $u_n = \dfrac{2}{3}n + 1$ \\

\definecolor{yqyqyq}{rgb}{0.5,0.5,0.5}

\definecolor{xdxdff}{rgb}{0.49,0.49,1}

\definecolor{ffqqqq}{rgb}{1,0,0}

\definecolor{cqcqcq}{rgb}{0.75,0.75,0.75}


\begin{tikzpicture}[line cap=round,line join=round,>=triangle 45,x=1cm,y=1cm,scale=1.2]

\draw[->] (-0.7,0) -- (12.75,0);
\foreach \x in {,1,2,3,4,5,6,7,8,9,10,11,12}
\draw[shift={(\x,0)}] (0pt,2pt) -- (0pt,-2pt) node[below] {\footnotesize $\x$};
\draw[->] (0,-0.7) -- (0,10.5);
\draw[color=black] (0pt,-8pt) node[left] {\footnotesize $0$};
\clip(-0.7,-0.7) rectangle (12.75,9.5);
\draw[dotted,color=cqcqcq, smooth,samples=100,domain=0.0:12.75] plot(\x,{(2*(\x))/3+1});
\draw [dash pattern=on 3pt off 3pt,color=yqyqyq] (1,0)-- (1,1.66);
\draw [dash pattern=on 3pt off 3pt,color=yqyqyq] (1,1.66)-- (0,1.66);
\draw [dash pattern=on 3pt off 3pt,color=yqyqyq] (2,0)-- (2,2.33);
\draw [dash pattern=on 3pt off 3pt,color=yqyqyq] (2,2.33)-- (0,2.33);
\draw [dash pattern=on 3pt off 3pt,color=yqyqyq] (3,0)-- (3,3);
\draw [dash pattern=on 3pt off 3pt,color=yqyqyq] (3,3)-- (0,3);
\draw [dash pattern=on 3pt off 3pt,color=yqyqyq] (4,0)-- (4,3.66);
\draw [dash pattern=on 3pt off 3pt,color=yqyqyq] (4,3.66)-- (0,3.66);
\draw [dash pattern=on 3pt off 3pt,color=yqyqyq] (5,0)-- (5,4.33);
\draw [dash pattern=on 3pt off 3pt,color=yqyqyq] (5,4.33)-- (0,4.33);
\draw [dash pattern=on 3pt off 3pt,color=yqyqyq] (6,0)-- (6,5);
\draw [dash pattern=on 3pt off 3pt,color=yqyqyq] (6,5)-- (0,5);
\draw [dash pattern=on 3pt off 3pt,color=yqyqyq] (7,0)-- (7,5.66);
\draw [dash pattern=on 3pt off 3pt,color=yqyqyq] (7,5.66)-- (0,5.66);
\draw [dash pattern=on 3pt off 3pt,color=yqyqyq] (8,0)-- (8,6.33);
\draw [dash pattern=on 3pt off 3pt,color=yqyqyq] (8,6.33)-- (0,6.33);
\draw [dash pattern=on 3pt off 3pt,color=yqyqyq] (10,0)-- (10,7.66);
\draw [dash pattern=on 3pt off 3pt,color=yqyqyq] (10,7.66)-- (0,7.66);
\draw [dash pattern=on 3pt off 3pt,color=yqyqyq] (9,0)-- (9,7);
\draw [dash pattern=on 3pt off 3pt,color=yqyqyq] (9,7)-- (0,7);
\draw [dash pattern=on 3pt off 3pt,color=yqyqyq] (11,0)-- (11,8.33);
\draw [dash pattern=on 3pt off 3pt,color=yqyqyq] (11,8.33)-- (0,8.33);
\draw [dash pattern=on 3pt off 3pt,color=yqyqyq] (12,0)-- (12,9);

\draw [dash pattern=on 3pt off 3pt,color=yqyqyq] (12,9)-- (0,9);

\draw [dash pattern=on 3pt off 3pt,color=yqyqyq] (13,0)-- (13,9.66);

\draw [dash pattern=on 3pt off 3pt,color=yqyqyq] (13,9.66)-- (0,9.66);


\draw [color=ffqqqq] (0,1)-- ++(-1.0pt,-1.0pt) -- ++(2.0pt,2.0pt) ++(-2.0pt,0) -- ++(2.0pt,-2.0pt);

\draw [color=ffqqqq] (0.98,1.65)-- ++(-1.0pt,-1.0pt) -- ++(2.0pt,2.0pt) ++(-2.0pt,0) -- ++(2.0pt,-2.0pt);

\draw [color=ffqqqq] (2,2.33)-- ++(-1.0pt,-1.0pt) -- ++(2.0pt,2.0pt) ++(-2.0pt,0) -- ++(2.0pt,-2.0pt);

\draw [color=ffqqqq] (3,3)-- ++(-1.0pt,-1.0pt) -- ++(2.0pt,2.0pt) ++(-2.0pt,0) -- ++(2.0pt,-2.0pt);

\draw [color=ffqqqq] (4.02,3.68)-- ++(-1.0pt,-1.0pt) -- ++(2.0pt,2.0pt) ++(-2.0pt,0) -- ++(2.0pt,-2.0pt);

\draw [color=ffqqqq] (5.02,4.35)-- ++(-1.0pt,-1.0pt) -- ++(2.0pt,2.0pt) ++(-2.0pt,0) -- ++(2.0pt,-2.0pt);

\draw [color=ffqqqq] (6,5)-- ++(-1.0pt,-1.0pt) -- ++(2.0pt,2.0pt) ++(-2.0pt,0) -- ++(2.0pt,-2.0pt);

\draw [color=ffqqqq] (7.02,5.68)-- ++(-1.0pt,-1.0pt) -- ++(2.0pt,2.0pt) ++(-2.0pt,0) -- ++(2.0pt,-2.0pt);

\draw [color=ffqqqq] (8.04,6.36)-- ++(-1.0pt,-1.0pt) -- ++(2.0pt,2.0pt) ++(-2.0pt,0) -- ++(2.0pt,-2.0pt);

\draw [color=ffqqqq] (9,7)-- ++(-1.0pt,-1.0pt) -- ++(2.0pt,2.0pt) ++(-2.0pt,0) -- ++(2.0pt,-2.0pt);

\draw [color=ffqqqq] (10.04,7.69)-- ++(-1.0pt,-1.0pt) -- ++(2.0pt,2.0pt) ++(-2.0pt,0) -- ++(2.0pt,-2.0pt);

\draw [color=ffqqqq] (10.96,8.31)-- ++(-1.0pt,-1.0pt) -- ++(2.0pt,2.0pt) ++(-2.0pt,0) -- ++(2.0pt,-2.0pt);

\draw [color=ffqqqq] (12,9)-- ++(-1.0pt,-1.0pt) -- ++(2.0pt,2.0pt) ++(-2.0pt,0) -- ++(2.0pt,-2.0pt);

\draw [color=ffqqqq] (13,9.67)-- ++(-1.0pt,-1.0pt) -- ++(2.0pt,2.0pt) ++(-2.0pt,0) -- ++(2.0pt,-2.0pt);


% \fill [color=xdxdff] (0,1.66) circle (1.5pt);

\draw (0,1) node [left] {$U_0$};

\draw (0,1.66) node [left] {$U_1$};

\draw (0,2.33) node [left] {$U_2$};

\draw (0,3) node [left] {$U_3$};

\draw (0,3.66) node [left] {$U_4$};

\draw (0,4.33) node [left] {$U_5$};

\draw (0,5) node [left] {$U_6$};

\draw (0,5.66) node [left] {$U_7$};

\draw (0,6.33) node [left] {$U_8$};

\draw (0,7) node [left] {$U_9$};

\draw (0,7.66) node [left] {$U_{10}$};

\draw (0,8.33) node [left] {$U_{11}$};

\draw (0,9) node [left] {$U_{12}$};


\begin{pgfonlayer}{background}   


\draw[step=1mm,ultra thin,AntiqueWhite!10] (-0.7,-0.7) grid (12.75,10.5);

\draw[step=5mm,very thin,AntiqueWhite!30]  (-0.7,-0.7) grid (12.75,10.5);

\draw[step=1cm,very thin,AntiqueWhite!50](-0.7,-0.7) grid (12.75,10.5);

\draw[step=5cm,thin,AntiqueWhite]         (-0.7,-0.7) grid (12.75,10.5);


\end{pgfonlayer}

\end{tikzpicture}

\newpage

\subsubsection{Suite définie par une relation de récurrence}

\textbf{Exemple n°1} \\

Soit $\left(u_n\right)_{n \; \in \N}$ une suite définie par :$ \; \; \; \begin{cases}
u_0 = 11 \\
\forall n \in \N, u_{n+1} = \dfrac{3}{4}u_n + 1 \\
\end{cases}$ 

\vspace*{.3cm}

On appelle $\Delta$ la droite d'équation $y = \dfrac{3}{4}x + 1$. \\

On a $u_1 = \dfrac{3}{4}u_0 + 1$. \\

On trace également la droite d'équation $ y = x$

Cette droite est la représentation graphique de la fonction identité, définie par $f(x) = x$. Elle est appelée \textbf{la première bissectrice du plan}. \\

\begin{texgraph}[file,name=suite_02,export=tkz]
Include "papiers.mac";
Graph image = [
Fenetre(-.9+12*i, 12-.9*i, 1+i),Marges(0,0,0,0),
papier(milli,-1-i,12+12*i,
[subsubgridcolor :=beige,
subgridcolor:=antiquewhite,
gridcolor :=bisque]
),
  Arrows:=1,
Axes(0,1 +i),
  Arrows:=0,
  u0:=11,nb:=15, Width:=6,
Color:=darkseagreen, Droite(1,-1,0), 
LabelAngle:=0, Label (2+i, "$y=x$") ,
Color:=red,  tMin:=-1, tMax:=12, Width:=8, 
              Cartesienne((3/4)*x+1),
LabelAngle:=0, Label (6+4.3*i, "$y=\frac{3x}{4}+1$"),              
 Width:=6, Color:=gray,
  suite((3/4)*x+1, u0,nb),
Color:=black,LabelAngle:=0, 
Label(8.5+9.25*i, "$U_1$", 
      7.5+7.93*i, "$U_2$",  
      6.6+6.95*i, "$U_3$",   
      5.7+6.21*i, "$U_4$",   
      5.3+5.66*i, "$U_5$")   
];
\end{texgraph}

\vspace*{.5cm}

On conjecture que :

\begin{itemize}
\item[•] $\left(u_n\right)_{n \in \N}$ est décroissante.
\item[•] $\forall n \in \N, 4 < u_n \leqslant 11$.
\item[•] La suite est convergente vers 4. On écrit alors $\lim\limits_{n \to +\infty} u_n = 4$. 
\end{itemize}

\vspace*{.3cm}

\newpage

\textbf{Exercice n°2} \\

Soit $\left(u_n\right)_{n \; \in \N}$ une suite définie par :$ \; \; \; \begin{cases}
u_0 = -2 \\
\forall n \in \N, u_{n+1} = -\dfrac{3}{4}u_n + 7 \\
\end{cases}$ 

\vspace*{.3cm}

On a $\Delta : y = -\dfrac{3}{4}x + 7$. On trace également $y = x$, la première bissectrice du plan. \\

\begin{texgraph}[file,name=Suite03,export=pgf]
Include "papiers.mac";
Graph image = [
Fenetre(-3+12*i, 12-.9*i, 1+i),Marges(0,0,0,0),
papier(milli,-3-i,12+12*i,
          [subsubgridcolor :=beige,
              subgridcolor :=antiquewhite,
                 gridcolor :=bisque]
      ),
  Arrows:=1,
  Axes(0,1 +i),
  Arrows:=0,
  u0:=-2,nb:=15, Width:=6,
Color:=darkseagreen, Droite(1,-1,0), 
 Label (2+i, "$y=x$"),
Color:=red,  tMin:=-3, tMax:=11, Width:=8, 
              Cartesienne(-1*(3/4)*x+7),
 Label (-2+9.6*i, "$y=\frac{-3x}{4}+7$"),              
 Width:=6, Color:=gray,
             suite(-1*(3/4)*x+7, u0,nb),
Color:=black,LabelAngle:=0, 
Label(-5/2+8.5*i, "$U_1$", 
        9+.7*i,   "$U_2$",  
        .8+6.8*i, "$U_3$",   
         7+2.2*i,   "$U_4$",   
       2.5+5.66*i, "$U_5$"
)   
];
\end{texgraph}

\vspace*{.5cm}

On conjecture que  :
\begin{itemize}
\item[•] $\left(u_n\right)_{n \in \N}$ non monotone.
\item[•] $\forall n \in \N, -2 < u_n \leqslant \dfrac{17}{2}$. (on a $u_1 = \dfrac{17}{2}$ )
\item[•] La suite est convergente vers 4. On écrit alors $\lim\limits_{n \to +\infty} u_n = 4$. 
\end{itemize}

\newpage

\section{Suites arithmétiques}

\subsection{Définition}

\subsubsection{Expression et vocabulaire}

Soit $\left(u_n\right)_{n \in \N}$ une suite. \\

$\left(u_n\right)_{n \in \N}$ est une suite arithmétique si, et seulement si, il existe un réel $r$ tel que : \\
$\forall n \in \N, u_{n+1} = u_n + r$. \\

$r$ est appelé la \textbf{raison de la suite arithmétique}. 

\subsubsection{Exemple}
Soit $\left(u_n\right)_{n \in \N}$ une suite définie par $u_n = 3n - 5$. \\

On a : 

\begin{itemize}
\item[•] $u_0 = -5$
\item[•] $u_1 = -2$
\item[•] $u_2 = 1$
\item[•] $u_3 = 4$
\item[•] $u_4 = 7$
\end{itemize}

\vspace*{.3cm}

\begin{tabular}{lll}
On a $u_{n+1}$ & $ = $ & $ 3\left(n+1\right)-5$ \\
& $=$ & $3n + 3 - 5$ \\
& $=$ & $3n - 5 + 3$ \\
& $=$ & $u_n + 3$ \\
\end{tabular}

\vspace*{.3cm}

D'où $\forall n \in \R, u_{n+1} = u_n + 3$. \\

Donc $\left(u_n\right)_{n \in \N}$ est une suite arithmétique de 1$^{\mathrm{er}}$ terme $u_0 = 5$ et de raison $r = 3$. 

\subsection{Terme général d'une suite arithmétique}

Soit $\left(u_n\right)_{n \in \N}$ une suite arithmétique de premier terme $u_0$ et de raison $r$. \\

On a :

\begin{itemize}
\item[•] $u_1 = u_0 + r$
\item[•] $u_2 = u_1 + r = u_0 + r + r = u_0 + 2r$
\item[•] $u_3 = u_2 + r = u_0 + 2r + r = u_0 + 3r$
\item[•] $u_4 = u_3 + r = u_0 + 3r + r = u_0 + 4r$
\end{itemize}

\vspace*{.3cm}

On conjecture que : $\forall n \in \N, u_n = u_0 + nr$ \\

\newpage

Montrons que $\forall n \in \N, u_n = u_0 + nr$ \\

On vérifie que la formule conjecturée est vraie pour la première valeur de $n$, c'est-à-dire $n = 0$. \\

$u_0 = u_0 + 0r$. 

Puis, on démontre que, si la formule conjecturée est vraie pour $n$, alors elle est vraie pour $n+1$, c'est-à-dire que si $u_n = u_0 + nr$, alors $u_{n+1} = u_0 + \left(n+1\right)r$ \\

Hypothèse de récurrence : $u_n = u_0 + nr$. \\

On calcule $u_{n+1}$. \\

\begin{tabular}{lll}
$u_{n+1}$ & $ = $ & $u_n + r$ \\
& $=$ & $u_0 + nr + r$ \\
& $=$ & $u_0 + r\left(n+1\right)$ \\ 
\end{tabular}

Donc $\forall n \in \N, u_n = u_0 + nr$ \\

\textbf{Attention !}

Soit $\left(u_n\right)_{n \in \N^*}$ une suite arithmétique de premier terme $u_1$ et de raison $r$. \\

On a :

\begin{itemize}
\item[•] $u_2 = u_1 + r$
\item[•] $u_3 = u_2 + r = u_1 + r + r = u_1 + 2r$
\item[•] $u_4 = u_3 + r = u_1 + 2r + r = u_1 + 3r$
\item[•] $u_5 = u_4 + r = u_1 + 3r + r = u_1 + 4r$
\end{itemize}

\vspace*{.3cm}

On a ici : $\forall n \in \N^*, u_n = u_1 + \left(n-1\right)r$. 

\newpage

\subsection{Somme des termes d'une suite arithmétique}

\subsubsection{Formule et démonstration}


Soit $\left(u_n\right)_{n \in \N}$ une suite arithmétique de premier terme $u_0$ et de raison $r$. \\

$ S = \underbrace{u_0 + u_1 + u_2 + u_3 + ... + u_n}_{n+1 \; \mathrm{termes}}$ \vspace*{.5cm} \\

\centerline{\footnotesize
\begin{tabular}{llrlllllllllllll}
On peut écrire : & & $S$ & $=$ &$u_0$ & $+$ &$u_1$ & $+$ &$u_2$ & $+$ & $...$ & $+$ &$u_{n-1}$ & $+$ &$u_n$ \\
& $+$ & $S$ & $=$ &$u_n$ & $+$ &$u_{n-1}$ & $+$ &$u_{n-2}$ & $+$ & $...$ & $+$ &$u_1$ & $+$ &$u_0$ \\
\hline
En posant l'addition, on a : & & $2S$ & $=$ & $\left(u_0 + u_n\right)$& $+$ & $\left(u_1 + u_{n-1}\right)$ & $+$ & $\left(u_2 + u_{n-2}\right)$ & $+$ & $...$ & $+$ & $\left(u_{n-1} + u_1\right)$ & $+$ & $\left(u_n + u_0\right)$ \\ 
\end{tabular}}

\vspace*{.3cm}

On observe, grâce à la propriété démontrée au paragraphe précédent, que :  \\

\begin{itemize}
\item[•]$u_0 + u_n = u_0 + \left(u_0 + nr\right) = 2u_0 + nr$ \\
\item[•] $u_1 + u_{n-1} = \left(u_0 + r\right) + \left[u_0+\left(n-1\right)r\right] = u_0 + r + u_0 + nr - r = 2u_0 + nr$ \\ 
\item[•] $u_2 + u_{n-2} = \left(u_0 + 2r\right) + \left[u_0 + \left(n-2\right)r\right] = u_0 + 2r + u_0 + nr -2r = 2u_0 + nr$ \\
\end{itemize}

\vspace*{.3cm}

On cherche à montrer que $u_p + u_{n-p} = 2u_0 + nr$ \\

Toujours d'après la propriété du paragraphe 2.2, on a : \\

\begin{tabular}{lll}
$u_p + u_{n-p}$ & $ = $ & $ \left(u_0 + pr\right) + \left[u_0 + \left(n-p\right)r\right]$ \\
& $=$ & $u_0 + pr + u_0 + nr - pr$ \\
& $=$ & $2u_0 + nr$. \\
\end{tabular}

\vspace*{.3cm}

Or, $2u_0 + nr = u_0 + u_n$. \\ 

\begin{tabular}{rll}
On a donc $2S$ & $=$ & $\left(u_0 + u_n\right) + \left(u_0 + u_n\right) + \left(u_0 + u_n\right) + ... + \left(u_0 + u_n\right)$ \vspace*{.3cm} \\
$2S$ & $ = $ & $ \left(n+1\right)\left(u_0 + u_n\right)$ \vspace*{.3cm} \\
$S$ & $=$ & $\left(n+1\right)\dfrac{\left(u_0 + u_n\right)}{2}$ \vspace*{.3cm} \\
\end{tabular}

\vspace*{.3cm} 

Ainsi, on a $u_0 + u_1 + u_2 + u_3 + ... + u_n = \left(n+1\right) \dfrac{u_0 + u_n}{2}$ \\

\textbf{Attention !} \\

Soit $\left(u_n\right)_{n\in \N^*}$ une suite arithmétique de premier terme $u_1$ et de raison $r$. \\

$u_1 + u_2 + u_3 + ... + u_n = n\dfrac{u_1 + u_n}{2}$

\newpage

\subsubsection{Exercice \no 1 : la suite de Gauß}

Soit $S = 1 + 2 + 3 + 4 + ... + n$. Donner la valeur de $S$. \\

Soit $\left(u_n\right)_{n \in \N^*}$ une suite arithmétique de premier terme $u_1 = 1$ et de raison $1$. \\

On peut écrire $ S = u_1 + u_2 + u_3 + u_4 + ... + u_n$ \\

Donc $S = n\dfrac{1+n}{2}$ 

Ainsi, $1 + 2+ 3 + 4 + ... + n = \dfrac{n\left(n+1\right)}{2}$ \\

Vérification : $1 + 2 + 3 + 4 + 5 = \dfrac{5 \times 6}{2} = 15$ \\

$1 + 2 + 3 + 4 + ... + 1000 = \dfrac{1000 \times 1001}{2} = 500\; 500$.

\subsubsection{Exercice \no 2}

Soit $S = 7 + 18 + 29 + 40 + ... + 2856$. Donner la valeur de $S$. \\

Soit $\left(u_n\right)_{n \in \N^*}$ une suite arithmétique de premier terme $u_1 = 7$ et de raison $r = 11$. \\

La somme des termes de la suite est donnée par $S = n\dfrac{u_1+u_n}{2}$. \\

On cherche à connaître le rang $n$ du terme $u_n = 1003$. \\

\begin{tabular}{lll}
$u_n$ & $=$ & $ u_1 + \left(n-1\right)r$ \\
& $=$ & $7 + 11\left(n-1\right)$ \\
& $=$ & $7 + 11n - 11$ \\
& $=$ & $11n - 4$ \\
\end{tabular}

\vspace*{.3cm}

\begin{tabular}{rll}
On a donc $11n - 4$ & $=$ & $2856$ \\
$11n$ & $=$ &$2860$ \\
$n$ & $=$ & $260$ \\ 
\end{tabular}

\vspace*{.3cm}

\begin{tabular}{lll}
D'où $S$ & $=$ & $n\dfrac{u_1+u_n}{2}$ \vspace*{.3cm} \\
& $=$ & $260 \times \dfrac{7 + 2856}{2}$ \vspace*{.3cm} \\
& $=$ & $372 190$ \\
\end{tabular}

\vspace*{.3cm}

D'où $S = 101 103$.

\newpage

\vspace*{-1.5cm} 

\subsubsection{Exercice \no 3}

Soit $S = 3 + 8 + 13 + 18 + ... + 1003$. Donner la valeur de $S$. \\

Soit $\left(u_n\right)_{n \in \N^*}$ une suite arithmétique de premier terme $u_1 = 3$ et de raison $r = 5$. \\

La somme des termes de la suite est donnée par $S = n\dfrac{u_1+u_n}{2}$. \\

On cherche à connaître le rang $n$ du terme $u_n = 1003$. \\


\begin{tabular}{lll}
$u_n$ & $=$ & $ u_1 + \left(n-1\right)r$ \\
& $=$ & $3 + 5\left(n-1\right)$ \\
& $=$ & $3 + 5n - 5$ \\
& $=$ & $5n - 2$ \\
\end{tabular}

\vspace*{.3cm}

\begin{tabular}{rll}
On a donc $5n - 2$ & $=$ & $1003$ \\
$5n$ & $=$ &$1005$ \\
$n$ & $=$ & $201$ \\ 
\end{tabular}

\vspace*{.3cm}

\begin{tabular}{lll}
D'où $S$ & $=$ & $n\dfrac{u_1+u_n}{2}$ \vspace*{.3cm} \\
& $=$ & $201 \times \dfrac{3 + 1003}{2}$ \vspace*{.3cm} \\
& $=$ & $101 103$ \\
\end{tabular}

\vspace*{.3cm}

D'où $S = 101 103$.

\subsection{Sens de variation d'une suite arithmétique}

Soit $\left(u_n\right)_{n \in \N}$ une suite arithmétique de premier terme $u_0$ et de raison $r \neq 0$. \\

On a donc : $\forall n \in \N, u_{n+1} = u_n + 1$. \\

On étudie le signe de $u_{n+1} - u_{n}$ \\

On a  $u_{n+1} - u_{n} = r$. \\

Donc :

\begin{itemize}
\item[•] Si $r > 0$, alors $\left(u_n\right)_{n \in \N}$ est strictement croissante. 
\item[•] Si $r < 0$, alors $\left(u_n\right)_{n \in \N}$ est strictement décroissante. 
\end{itemize}

\subsection{Notion de limite d'une suite arithmétique}

Soit $\left(u_n\right)_{n \in \N}$ une suite arithmétique de premier terme $u_0$ et de raison $r \neq 0$. \\

On a donc : $\forall n \in \N, u_{n} = u_0 + nr$. \\

On étudie la limite de la suite arithmétique, ce qui s'écrit $\lim\limits_{h \to +\infty} u_n$. \\

On a :

\begin{itemize}
\item[•] Si $r > 0$, alors $\lim\limits_{h \to +\infty} u_n = +\infty$ 
\item[•] Si $r < 0$, alors $\lim\limits_{h \to +\infty} u_n = -\infty$ 
\end{itemize}

\vspace*{-5cm}

\newpage

\vspace*{-1cm}

\subsection{Un superbe exercice}

1) Soit $\left(u_n\right)_{u \in \N}$ la suite arithmétique décroissante définie par :$ \; \; \; \begin{cases}
u_0 + u_1 + u_2 = 270 \\
u_0 \times u_1 \times u_2 = 720 000 \\
\end{cases}$ \\

Déterminer $u_0$, $u_1$ et $u_2$ \\

On peut dire que $u_1 = u_0 + r$ et $u_2 = u_0 + 2r$. \\ On peut aussi dire $u_0 = u_1 - r$ et $u_2 = u_1 + r$. \\

On a alors $\left(u_1 - r\right) + u_1 + \left(u_1 + r\right) = 270$ \\

Ainsi, $3u_1 = 270$ et $u_1 = 90$. \\

On sait que $u_0 \times u_1 \times u_2 = 720 000$ \\

Il vient que $\left(90 - r\right) \times 90 \times \left(90+r\right) = 720 000$. \\

$\left(90-r\right)\left(90+r\right) = \dfrac{720 000}{90}$ \\

$8100 - r^2 = 8000$ 

$-r^2 = -100$ 

$r^2 = 100$ 

Donc $r = 10$ ou $r = -10$. \\

Cependant, on sait que la suite $\left(u_n\right)_{n \in \N}$ est décroissante. Donc $r < 0$. \\

Ainsi, $r = -10$. \\

On a donc $u_0 = 100$, $u_1 = 90$ et $u_2 = 80$. \\

2) Soit $S = u_0 + u_1 + u_2 + ... + u_n$ 

Déterminer $n$ tel que $S = 450$. \\

On cherche à trouver $n$ tel que $\left(n+1\right)\dfrac{u_0 + u_n}{2} = 450$ \\

$\left(u_n\right)_{n \in \N}$ est une suite arithmétique de premier terme $u_0 = 100$ et de raison $r = -10$. \\

On a $u_n = u_0 + nr = 100 - 10n$ \\

Donc on résout $\left(n+1\right)\dfrac{100 + \left(100 -10n\right)}{2} = 450$ \\

$\left(n+1\right)\left(100 - 5n\right) = 450$ 

$100n -5n^2 + 100 - 5n = 450$ 

$-5n^2 + 95n - 350 = 0$ 

$ n = 5$ ou $n = 14$ \\

Et en effet : \\

\centerline{\footnotesize
\begin{tabular}{rllllllllllllllllllllllllllll}
$u_0$ & \hspace*{-.3cm} $+$ \hspace*{-.3cm} & \hspace*{-.3cm} $u_1$ \hspace*{-.3cm} & \hspace*{-.3cm} $+$ \hspace*{-.3cm} & \hspace*{-.3cm}$u_2$ \hspace*{-.3cm} & \hspace*{-.3cm} $+$ \hspace*{-.3cm} & \hspace*{-.3cm}$u_3$ \hspace*{-.3cm} & \hspace*{-.3cm} $+$ \hspace*{-.3cm} &  \hspace*{-.3cm} $u_4$ \hspace*{-.3cm} & \hspace*{-.3cm} $+$ \hspace*{-.3cm} & \hspace*{-.3cm} $u_5$ \hspace*{-.3cm} & \hspace*{-.3cm} $+$ \hspace*{-.3cm} & \hspace*{-.3cm} $u_6$ \hspace*{-.3cm} & \hspace*{-.3cm} $+$ \hspace*{-.3cm} & \hspace*{-.3cm} $u_7$ \hspace*{-.3cm} & \hspace*{-.3cm} $+$ \hspace*{-.3cm} & \hspace*{-.3cm} $u_8$ \hspace*{-.3cm} & \hspace*{-.3cm} $+$ \hspace*{-.3cm} & \hspace*{-.3cm} $u_9$\hspace*{-.3cm} &  \hspace*{-.3cm}$+$ \hspace*{-.3cm} &  \hspace*{-.3cm}$u_{10}$ \hspace*{-.3cm} & \hspace*{-.3cm} $+$ \hspace*{-.3cm} & \hspace*{-.3cm} $u_{11}$ \hspace*{-.3cm} &  \hspace*{-.3cm} $+$ \hspace*{-.3cm} & \hspace*{-.3cm} $u_{12}$ \hspace*{-.3cm} & \hspace*{-.3cm} $+$ \hspace*{-.3cm} & \hspace*{-.3cm} $u_{13}$ \hspace*{-.3cm} &  \hspace*{-.3cm} $+$ \hspace*{-.3cm} & \hspace*{-.3cm} $u_{14}$ \\
$100$ & \hspace*{-.3cm} $+$ & \hspace*{-.3cm} $90$  & \hspace*{-.3cm} $+$ & \hspace*{-.3cm} $80$ & \hspace*{-.3cm} $+$ & \hspace*{-.3cm} $70$ & \hspace*{-.3cm} $+$ & \hspace*{-.3cm} $60$ & \hspace*{-.3cm} $+$ & \hspace*{-.3cm} $50$ & \hspace*{-.3cm} $+$ & \hspace*{-.3cm} $40$ & \hspace*{-.3cm} $+$ & \hspace*{-.3cm} $30$ & \hspace*{-.3cm} $+$ & \hspace*{-.3cm} $20$ & \hspace*{-.3cm} $+$ & \hspace*{-.3cm} $10$ & \hspace*{-.3cm} $+$ & \hspace*{-.3cm} $0$ & \hspace*{-.3cm} $-$ & \hspace*{-.3cm} $10$ & \hspace*{-.3cm} $-$ & \hspace*{-.3cm} $20$ & \hspace*{-.3cm} $-$ & \hspace*{-.3cm} $30$ & \hspace*{-.3cm} $-$ & \hspace*{-.3cm} $40$ \\
\end{tabular}}

\newpage

\newpage

\section{Suites géométriques}

\subsection{Définition}

\subsubsection{Expression et vocabulaire}

Soit $\left(u_n\right)_{n \in \N}$ une suite. \\

$\left(u_n\right)_{n \in \N}$ est une suite géométrique si, et seulement si, il existe un réel $q$ \textbf{non nul} tel que : \\
$\forall n \in \N, u_{n+1} = u_n \times q$. \\

$q$ est appelé la \textbf{raison de la suite géométrique}. 

\subsubsection{Exemple}
Soit $\left(u_n\right)_{n \in \N}$ une suite définie par $u_n = 2 \times 3^n$. \\

On a : 

\begin{itemize}
\item[•] $u_0 = 2 \times 3^0 = 2 \times 1 = 2$
\item[•] $u_1 = 2 \times 3^1 = 2 \times 3 = 6$
\item[•] $u_2 = 2 \times 3^2 = 2 \times 9 = 18$
\item[•] $u_3 = 2 \times 3^3 = 2 \times 27 = 54$
\item[•] $u_4 = 2 \times 3^4 = 2 \times 81 = 162$
\end{itemize}

\vspace*{.3cm}

\begin{tabular}{lll}
On a $u_{n+1}$ & $ = $ & $ 2 \times 3^{n+1}$ \\
& $=$ & $2 \times \left(3^n \times 3\right)$ \\
& $=$ & $\left(2 \times 3^n\right) \times 3$ \\
& $=$ & $u_n \times 3$ \\
\end{tabular}

\vspace*{.3cm}

D'où $\forall n \in \R, u_{n+1} = u_n \times 3$. \\

Donc $\left(u_n\right)_{n \in \N}$ est une suite arithmétique de 1$^{\mathrm{er}}$ terme $u_0 = 2$ et de raison $q = 3$. 

\subsection{Terme général d'une suite géométrique}

Soit $\left(u_n\right)_{n \in \N}$ une suite géométrique de premier terme $u_0$ et de raison $q$. \\

On a :

\begin{itemize}
\item[•] $u_1 = u_0 \times q$
\item[•] $u_2 = u_1 \times q = u_0 \times q \times q = u_0 \times q^2$
\item[•] $u_3 = u_2 \times q = u_0 \times q^2 \times q = u_0 \times q^3$
\item[•] $u_4 = u_3 \times q = u_0 \times q^3 \times q = u_0 \times q^4$
\end{itemize}

\vspace*{.3cm}

On conjecture que : $\forall n \in \N, u_n = u_0 \times q^n$ \\

\newpage

Montrons que $\forall n \in \N, u_n = u_0 \times q^n$ \\

On vérifie que la formule conjecturée est vraie pour la première valeur de $n$, c'est-à-dire $n = 0$. \\

$u_0 = u_0 \times q^0 = u_0$. 

Puis, on démontre que, si la formule conjecturée est vraie pour $n$, alors elle est vraie pour $n+1$, c'est-à-dire que si $u_n = u_0 \times q^n$, alors $u_{n+1} = u_0 \times q^{n+1}$ \\

Hypothèse de récurrence : $u_n = u_0 \times q^n$. \\

On calcule $u_{n+1}$. \\

\begin{tabular}{lll}
$u_{n+1}$ & $ = $ & $u_n \times q$ \\
& $=$ & $u_0 \times q^n \times q$ \\
& $=$ & $u_0 \times q^{n+1}$ \\ 
\end{tabular}

Donc $\forall n \in \N, u_n = u_0 + nr$ \\

\textbf{Attention !}

Soit $\left(u_n\right)_{n \in \N^*}$ une suite géométrique de premier terme $u_1$ et de raison $r$. \\

On a :

\begin{itemize}
\item[•] $u_2 = u_1 \times q$
\item[•] $u_3 = u_2 \times q = u_1 \times q \times q = u_1 \times q^2$
\item[•] $u_4 = u_3 \times q = u_1 \times q^2 \times q = u_1 + \times q^3$
\item[•] $u_5 = u_4 \times q = u_1 \times q^3 \times q = u_1 \times q^4$
\end{itemize}

\vspace*{.3cm}

On a ici : $\forall n \in \N^*, u_n = u_1 \times q^{n-1}$. 

\newpage

\subsection{Somme des termes d'une suite géométrique}

\subsubsection{Formule et démonstration}

Soit $\left(u_n\right)_{n \in \N}$ une suite géométrique de premier terme $u_0$ et de raison $q$. \\

$ S = \underbrace{u_0 + u_1 + u_2 + u_3 + ... + u_n}_{n+1 \; \mathrm{termes}}$ \vspace*{.5cm} \\

\centerline{\footnotesize
\begin{tabular}{llrlllllllllllll}
On peut écrire : & & $S$ & $=$ &$u_0$ & $+$ &$u_1$ & $+$ &$u_2$ & $+$ & $...$ & $+$ &$u_{n-1}$ & $+$ &$u_n$ \\
& et & $S$ & $=$ &$u_0$ & $+$ &$u_0 \times q$ & $+$ &$u_0 \times q^2$ & $+$ & $...$ & $+$ &$u_0 \times \times q^{n-1}$ & $+$ &$u_0 \times q^n$ \\
En multipliant par $q$ chaque \\
membre de l'égalité, on a : & & $qS$ & $=$ & $u_0 \times q$& $+$ & $u_0 \times q^2$ & $+$ & $u_0 \times q^3$ & $+$ & $...$ & $+$ & $u_0 \times q^n$ & $+$ & $u_0 \times q^{n+1}$ \\ 
\end{tabular}}

\vspace*{.3cm}

Ainsi, on peut dire que : \\

\begin{itemize}
\item[•]$ S - qS = u_0 - u_0 \times q^{n+1}$ \\
\item[•] $\left(1 \times q\right)S = u_0 \left(1 - q^{n+1}\right)$ \\
\item[•] $S = u_0 \times \dfrac{1 - q^{n+1}}{1 - q}$ \\
\end{itemize}

\vspace*{.3cm}

Donc, on a $u_0 + u_1 + u_2 + u_3 + ... + u_n = u_0 \dfrac{1 - q^{n+1}}{1 - q}$ \\

\textbf{Attention !} \\

Soit $\left(u_n\right)_{n\in \N^*}$ une suite géométrique de premier terme $u_1$ et de raison $q$. \\

$u_1 + u_2 + u_3 + ... + u_n = u_1 \dfrac{1 - q^n}{1 - q}$

\newpage

\vspace*{-1.5cm}

\subsubsection{Exemple fondamental : somme des puissances de 2}

$S = 2^0 + 2^1 + 2^2 + 2^3 + 2^4 + ... + 2^n$. Calculer $S$. \\

$S = u_0 + u_1 + u_2 + u_3 + u_4 + ... + u_n$. \\

Soit $\left(u_n\right)_{n \in \N^*}$ une suite géométrique de premier terme $u_0 = 1$ et de raison $q = 2$. \\

\textbf{Attention !}  \\

\textbf{$\mathbf{S}$ n'est pas une suites géométriques mais la sommes des termes d'une suites géométrique.} \\

La somme des termes de la suite est donnée par $S = u_0\dfrac{1 - q^{n+1}}{1 - q}$. \\

$S = 1 \times \dfrac{1 - 2^{n+1}}{1-2}$ \\

$ S = -1 \times \left(1 - 2^{n+1}\right)$ \\

$ S = -1 + 2^{n+1}$ \\

Donc $2^0 + 2^1 + 2^2 + 2^3 + ... + 2^n = -1 + 2^{n+1}$ \\

\textbf{N.B. : Tout nombre de la forme $\mathbf{2^p - 1}$ est appelé \underline{nombre de Mersenne}. Les nombres premiers de Mersenne ont des propriétés \hbox{très particulières pour trouver rapidement de grands nombres parfaits}. }

\subsubsection{Exemple \no 2}

Soit $S = 5 + 15 + 45 + 135 + ... + 295 245$. Calculer $S$. \\

$ S = u_1 + u_2 + u_3 + u_4 + ... + u_n$ \\

Soit $\left(u_n\right)_{n \in \N^*}$ une suite géométrique de premier terme $u_1 = 5$ et de raison $q = 3$. \\

\textbf{Attention !}  \\

\textbf{$\mathbf{S}$ n'est pas une suites géométriques mais la sommes des termes d'une suites géométrique.} \\

La somme des termes de la suite est donnée par $S = u_1\dfrac{1 - q^n}{1 - q}$. \\

On cherche à connaître le rang $n$ du terme $u_n = 295 245 $. \\

\begin{tabular}{lll}
$u_n$ & $=$ & $295 245$ \\
$u_1q^{n-1}$& $=$ & $295 245$ \\
$5 \times 3^{n-1}$ & $=$ & $295 245$ \\
$3^{n-1}$& $=$ & $59049$ \\
\end{tabular}

\vspace*{.3cm}

On a $n - 1 = 10$ car $3^{10} = 59049$. \\

D'où $n = 11$. \\

Enfin, $S = 5 \times \dfrac{1 - 3^{11}}{1 - 3} = 442 865$ 

\vspace*{-5cm}

\newpage

\vspace*{-1.4cm}

\subsection{Sens de variation d'une suite géométrique}

Soit $\left(u_n\right)_{n \in \N}$ une suite arithmétique de premier terme $u_0 \neq 0$ et de raison $q \neq 1$. \\

On a donc : $\forall n \in \N, u_{n} = u_0 \times q^n$. \\

On étudie le signe de $u_{n+1} - u_{n}$ \\

\begin{tabular}{lll}
On a  $u_{n+1} - u_{n}$ & $ = $ &$ u_0 \times q^{n+1} - u_0 \times q^{n}$ \\
& $=$ & $u_0q^n \left(q-1\right)$
\end{tabular}

\vspace*{.3cm}

On a donc trois cas possibles : \\
\begin{itemize}
\item[•] Si $q > 1$, $q^n > 0$ et $q - 1 > 0$ alors on a :
\begin{itemize}
\item[*] Si $u_0 > 0$, alors $u_{n+1} - u_n > 0$ et $\left(u_n\right)_{n \in \N}$ est strictement croissante. \\
\item[*] Si $u_0 < 0$, alors $u_{n+1} - u_n < 0$ et $\left(u_n\right)_{n \in \N}$ est strictement décroissante. \\
\end{itemize} 
\item[•] Si $q > 1$, $q^n > 0$ et $q - 1 < 0$ alors on a :
\begin{itemize}
\item[*] Si $u_0 > 0$, alors $u_{n+1} - u_n < 0$ et $\left(u_n\right)_{n \in \N}$ est strictement décroissante. \\
\item[*] Si $u_0 < 0$, alors $u_{n+1} - u_n > 0$ et $\left(u_n\right)_{n \in \N}$ est strictement croissante. \\
\end{itemize} 
\item[•] Si $q < 0$, alors le signe de $q^n$ dépend de $n$ et $\left(u_n\right)_{n \in \N}$ est non monotone.
\end{itemize}

\subsection{Notion de limite d'une suite géométrique}

Soit $\left(u_n\right)_{n \in \N}$ une suite géométrique de premier terme $u_0$ et de raison $q$. \\

On peut étudier ici deux limites : \\

\begin{itemize}
\item[1.] Étudions la limites de $q^n$ selon les valeurs de $q$ :
\begin{itemize}
\item[•] Si $q > 1$, alors $\lim\limits_{n \to +\infty} q^n = + \infty$ \\ 
\item[•] Si $0 < q < 1$, alors $\lim\limits_{n \to +\infty} q^n = 0$. Exemple : Si $q = \dfrac{1}{2}$, alors $\lim\limits_{n \to +\infty} \left(\dfrac{1}{2}\right)^n = 0$ \\ 
\item[•] Si $-1 < q < 0$, alors $\lim\limits_{n \to +\infty} q^n = 0$. Exemple : Si $q = -\dfrac{1}{2}$, alors $\lim\limits_{n \to +\infty} \left(-\dfrac{1}{2}\right)^n = 0$ \\ 
\item[•] Si $q < -1$, alors $\lim\limits_{n \to +\infty} q^n$ n'existe pas. Exemple : Si $q = -2$, alors $\lim\limits_{n \to +\infty} \left(-2\right)^n$ n'existe pas. \\ 
\end{itemize}
\item[2.] Étudions la limite de $u_n$ selon les valeurs de $q$ et les valeurs de $u_0$, en rappelant que  : $u_n = u_0q^n$ :
\begin{itemize}
\item[•] Si $q > 1$, alors on a deux cas possibles :
\begin{itemize}
\item[*] Si $u_0 > 0$, alors $\lim\limits_{n \to +\infty} u_n = +\infty$ \\
\item[*] Si $u_0 < 0$, alors $\lim\limits_{n \to +\infty} u_n = -\infty$ \\
\end{itemize}
\item[•] Si $0 < q < 1$, alors $\lim\limits_{n \to +\infty} u_n = 0$ \\
\item[•] Si $-1 < q < 0$, alors $\lim\limits_{n \to +\infty} u_n = 0$ \\
\item[•] Si $q < -1$, alors $\lim\limits_{n \to +\infty} u_n$ n'existe pas. 
\end{itemize}
\end{itemize}

\vspace*{-5cm}

\newpage

\subsection{Un superbe exercice}

1) Soit $\left(u_n\right)_{u \in \N}$ la suite géométrique décroissante définie par :$ \; \; \; \begin{cases}
u_0 + u_1 + u_2 = 999 \\
u_0 \times u_1 \times u_2 = 729 000 \\
\end{cases}$ \\

Déterminer $u_0$, $u_1$ et $u_2$ \\

2) Soit $S = u_0 + u_1 + u_2 + ... + u_n$ \\

Déterminer $n$ tel que $S = 999 999$. \\

1) On peut dire que $u_1 = u_0 \times q $ et $u_2 = u_0 + \times q^2$. \\ On peut aussi dire $u_0 = \dfrac{u_1}{q}$ et $u_2 = u_1 \times q$. \\

On a alors $\dfrac{u_1}{q} \times u_1 \times \left(u_1 \times q\right) = 729 000$ \\

Ainsi, $u_1^3 = 729 000$ et $u_1 = 729 000^{\dfrac{1}{3}}$. \\

D'où $u_1 = 90$ \\

On sait que $u_0 + u_1 + u_2 = 999$ \\

Il vient que $\dfrac{90}{q} + 90 + 90q = 999$. \\

$\dfrac{90}{q} + 90q = 909$ \\

$\dfrac{90 + 90q^2}{q} = 909$ \\

$90q^2 + 90 = 909q$ \\

$90q^2 - 909 + 90 = 0$ \\ 

Donc $q = \dfrac{1}{10}$ ou $q = 1$. \\

Cependant, on sait que la suite $\left(u_n\right)_{n \in \N}$ est décroissante. Donc $q \neq 10$. \\

Ainsi, $q = \dfrac{1}{10}$. \\

On a donc $u_0 = \dfrac{u_1}{q} = \dfrac{90}{\dfrac{1}{10}} = 900$, $u_1 = 90$ et $u_2 = u_1q = 90 \times \dfrac{1}{10} = 9$. \\

\newpage

2) On cherche à trouver $n$ tel que $u_0\dfrac{1-q^{n+1}}{1-q} = 999 999$ \vspace*{.3cm} \\

$900\dfrac{1- \left(\dfrac{1}{10}\right)^{n+1}}{1 - \dfrac{1}{10}} = 999 999$ \vspace*{.3cm} \\

$1000 \left[1 - \left(\dfrac{1}{10}\right)^{n+1}\right] = 999 999$ \vspace*{.3cm} \\

$1 - \left(\dfrac{1}{10}\right)^{n+1} = 0,999999$ \vspace*{.3cm} \\

$ - \left(\dfrac{1}{10}\right)^{n+1} = -0,000001$ \vspace*{.3cm} \\

$\left(\dfrac{1}{10}\right)^{n+1} = 0,000001$ \vspace*{.3cm} \\

$ n + 1 = 6$ \\

$ n = 5$. 