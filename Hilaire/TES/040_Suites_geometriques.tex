\section{Suites géométriques}

\subsection{Définition}

\subsubsection{Expression et vocabulaire}

Soit $\left(u_n\right)_{n \in \N}$ une suite. \\

$\left(u_n\right)_{n \in \N}$ est une suite géométrique si, et seulement si, il existe un réel $q$ \textbf{non nul} tel que : \\
$\forall n \in \N, u_{n+1} = u_n \times q$. \\

$q$ est appelé la \textbf{raison de la suite géométrique}. 

\subsubsection{Exemple}
Soit $\left(u_n\right)_{n \in \N}$ une suite définie par $u_n = 2 \times 3^n$. \\

On a : 

\begin{itemize}
\item[•] $u_0 = 2 \times 3^0 = 2 \times 1 = 2$
\item[•] $u_1 = 2 \times 3^1 = 2 \times 3 = 6$
\item[•] $u_2 = 2 \times 3^2 = 2 \times 9 = 18$
\item[•] $u_3 = 2 \times 3^3 = 2 \times 27 = 54$
\item[•] $u_4 = 2 \times 3^4 = 2 \times 81 = 162$
\end{itemize}

\vspace*{.3cm}

\begin{tabular}{lll}
On a $u_{n+1}$ & $ = $ & $ 2 \times 3^{n+1}$ \\
& $=$ & $2 \times \left(3^n \times 3\right)$ \\
& $=$ & $\left(2 \times 3^n\right) \times 3$ \\
& $=$ & $u_n \times 3$ \\
\end{tabular}

\vspace*{.3cm}

D'où $\forall n \in \R, u_{n+1} = u_n \times 3$. \\

Donc $\left(u_n\right)_{n \in \N}$ est une suite arithmétique de 1$^{\mathrm{er}}$ terme $u_0 = 2$ et de raison $q = 3$. 

\subsection{Terme général d'une suite géométrique}

Soit $\left(u_n\right)_{n \in \N}$ une suite géométrique de premier terme $u_0$ et de raison $q$. \\

On a :

\begin{itemize}
\item[•] $u_1 = u_0 \times q$
\item[•] $u_2 = u_1 \times q = u_0 \times q \times q = u_0 \times q^2$
\item[•] $u_3 = u_2 \times q = u_0 \times q^2 \times q = u_0 \times q^3$
\item[•] $u_4 = u_3 \times q = u_0 \times q^3 \times q = u_0 \times q^4$
\end{itemize}

\vspace*{.3cm}

On conjecture que : $\forall n \in \N, u_n = u_0 \times q^n$ \\

\newpage

Montrons que $\forall n \in \N, u_n = u_0 \times q^n$ \\

On vérifie que la formule conjecturée est vraie pour la première valeur de $n$, c'est-à-dire $n = 0$. \\

$u_0 = u_0 \times q^0 = u_0$. 

Puis, on démontre que, si la formule conjecturée est vraie pour $n$, alors elle est vraie pour $n+1$, c'est-à-dire que si $u_n = u_0 \times q^n$, alors $u_{n+1} = u_0 \times q^{n+1}$ \\

Hypothèse de récurrence : $u_n = u_0 \times q^n$. \\

On calcule $u_{n+1}$. \\

\begin{tabular}{lll}
$u_{n+1}$ & $ = $ & $u_n \times q$ \\
& $=$ & $u_0 \times q^n \times q$ \\
& $=$ & $u_0 \times q^{n+1}$ \\ 
\end{tabular}

Donc $\forall n \in \N, u_n = u_0 + nr$ \\

\textbf{Attention !}

Soit $\left(u_n\right)_{n \in \N^*}$ une suite géométrique de premier terme $u_1$ et de raison $r$. \\

On a :

\begin{itemize}
\item[•] $u_2 = u_1 \times q$
\item[•] $u_3 = u_2 \times q = u_1 \times q \times q = u_1 \times q^2$
\item[•] $u_4 = u_3 \times q = u_1 \times q^2 \times q = u_1 + \times q^3$
\item[•] $u_5 = u_4 \times q = u_1 \times q^3 \times q = u_1 \times q^4$
\end{itemize}

\vspace*{.3cm}

On a ici : $\forall n \in \N^*, u_n = u_1 \times q^{n-1}$. 

\newpage

\subsection{Somme des termes d'une suite géométrique}

\subsubsection{Formule et démonstration}

Soit $\left(u_n\right)_{n \in \N}$ une suite géométrique de premier terme $u_0$ et de raison $q$. \\

$ S = \underbrace{u_0 + u_1 + u_2 + u_3 + ... + u_n}_{n+1 \; \mathrm{termes}}$ \vspace*{.5cm} \\

\centerline{\footnotesize
\begin{tabular}{llrlllllllllllll}
On peut écrire : & & $S$ & $=$ &$u_0$ & $+$ &$u_1$ & $+$ &$u_2$ & $+$ & $...$ & $+$ &$u_{n-1}$ & $+$ &$u_n$ \\
& et & $S$ & $=$ &$u_0$ & $+$ &$u_0 \times q$ & $+$ &$u_0 \times q^2$ & $+$ & $...$ & $+$ &$u_0 \times \times q^{n-1}$ & $+$ &$u_0 \times q^n$ \\
En multipliant par $q$ chaque \\
membre de l'égalité, on a : & & $qS$ & $=$ & $u_0 \times q$& $+$ & $u_0 \times q^2$ & $+$ & $u_0 \times q^3$ & $+$ & $...$ & $+$ & $u_0 \times q^n$ & $+$ & $u_0 \times q^{n+1}$ \\ 
\end{tabular}}

\vspace*{.3cm}

Ainsi, on peut dire que : \\

\begin{itemize}
\item[•]$ S - qS = u_0 - u_0 \times q^{n+1}$ \\
\item[•] $\left(1 \times q\right)S = u_0 \left(1 - q^{n+1}\right)$ \\
\item[•] $S = u_0 \times \dfrac{1 - q^{n+1}}{1 - q}$ \\
\end{itemize}

\vspace*{.3cm}

Donc, on a $u_0 + u_1 + u_2 + u_3 + ... + u_n = u_0 \dfrac{1 - q^{n+1}}{1 - q}$ \\

\textbf{Attention !} \\

Soit $\left(u_n\right)_{n\in \N^*}$ une suite géométrique de premier terme $u_1$ et de raison $q$. \\

$u_1 + u_2 + u_3 + ... + u_n = u_1 \dfrac{1 - q^n}{1 - q}$

\newpage

\vspace*{-1.5cm}

\subsubsection{Exemple fondamental : somme des puissances de 2}

$S = 2^0 + 2^1 + 2^2 + 2^3 + 2^4 + ... + 2^n$. Calculer $S$. \\

$S = u_0 + u_1 + u_2 + u_3 + u_4 + ... + u_n$. \\

Soit $\left(u_n\right)_{n \in \N^*}$ une suite géométrique de premier terme $u_0 = 1$ et de raison $q = 2$. \\

\textbf{Attention !}  \\

\textbf{$\mathbf{S}$ n'est pas une suites géométriques mais la sommes des termes d'une suites géométrique.} \\

La somme des termes de la suite est donnée par $S = u_0\dfrac{1 - q^{n+1}}{1 - q}$. \\

$S = 1 \times \dfrac{1 - 2^{n+1}}{1-2}$ \\

$ S = -1 \times \left(1 - 2^{n+1}\right)$ \\

$ S = -1 + 2^{n+1}$ \\

Donc $2^0 + 2^1 + 2^2 + 2^3 + ... + 2^n = -1 + 2^{n+1}$ \\

\textbf{N.B. : Tout nombre de la forme $\mathbf{2^p - 1}$ est appelé \underline{nombre de Mersenne}. Les nombres premiers de Mersenne ont des propriétés \hbox{très particulières pour trouver rapidement de grands nombres parfaits}. }

\subsubsection{Exemple \no 2}

Soit $S = 5 + 15 + 45 + 135 + ... + 295 245$. Calculer $S$. \\

$ S = u_1 + u_2 + u_3 + u_4 + ... + u_n$ \\

Soit $\left(u_n\right)_{n \in \N^*}$ une suite géométrique de premier terme $u_1 = 5$ et de raison $q = 3$. \\

\textbf{Attention !}  \\

\textbf{$\mathbf{S}$ n'est pas une suites géométriques mais la sommes des termes d'une suites géométrique.} \\

La somme des termes de la suite est donnée par $S = u_1\dfrac{1 - q^n}{1 - q}$. \\

On cherche à connaître le rang $n$ du terme $u_n = 295 245 $. \\

\begin{tabular}{lll}
$u_n$ & $=$ & $295 245$ \\
$u_1q^{n-1}$& $=$ & $295 245$ \\
$5 \times 3^{n-1}$ & $=$ & $295 245$ \\
$3^{n-1}$& $=$ & $59049$ \\
\end{tabular}

\vspace*{.3cm}

On a $n - 1 = 10$ car $3^{10} = 59049$. \\

D'où $n = 11$. \\

Enfin, $S = 5 \times \dfrac{1 - 3^{11}}{1 - 3} = 442 865$ 

\vspace*{-5cm}

\newpage

\vspace*{-1.4cm}

\subsection{Sens de variation d'une suite géométrique}

Soit $\left(u_n\right)_{n \in \N}$ une suite arithmétique de premier terme $u_0 \neq 0$ et de raison $q \neq 1$. \\

On a donc : $\forall n \in \N, u_{n} = u_0 \times q^n$. \\

On étudie le signe de $u_{n+1} - u_{n}$ \\

\begin{tabular}{lll}
On a  $u_{n+1} - u_{n}$ & $ = $ &$ u_0 \times q^{n+1} - u_0 \times q^{n}$ \\
& $=$ & $u_0q^n \left(q-1\right)$
\end{tabular}

\vspace*{.3cm}

On a donc trois cas possibles : \\
\begin{itemize}
\item[•] Si $q > 1$, $q^n > 0$ et $q - 1 > 0$ alors on a :
\begin{itemize}
\item[*] Si $u_0 > 0$, alors $u_{n+1} - u_n > 0$ et $\left(u_n\right)_{n \in \N}$ est strictement croissante. \\
\item[*] Si $u_0 < 0$, alors $u_{n+1} - u_n < 0$ et $\left(u_n\right)_{n \in \N}$ est strictement décroissante. \\
\end{itemize} 
\item[•] Si $q > 1$, $q^n > 0$ et $q - 1 < 0$ alors on a :
\begin{itemize}
\item[*] Si $u_0 > 0$, alors $u_{n+1} - u_n < 0$ et $\left(u_n\right)_{n \in \N}$ est strictement décroissante. \\
\item[*] Si $u_0 < 0$, alors $u_{n+1} - u_n > 0$ et $\left(u_n\right)_{n \in \N}$ est strictement croissante. \\
\end{itemize} 
\item[•] Si $q < 0$, alors le signe de $q^n$ dépend de $n$ et $\left(u_n\right)_{n \in \N}$ est non monotone.
\end{itemize}

\subsection{Notion de limite d'une suite géométrique}

Soit $\left(u_n\right)_{n \in \N}$ une suite géométrique de premier terme $u_0$ et de raison $q$. \\

On peut étudier ici deux limites : \\

\begin{itemize}
\item[1.] Étudions la limites de $q^n$ selon les valeurs de $q$ :
\begin{itemize}
\item[•] Si $q > 1$, alors $\lim\limits_{n \to +\infty} q^n = + \infty$ \\ 
\item[•] Si $0 < q < 1$, alors $\lim\limits_{n \to +\infty} q^n = 0$. Exemple : Si $q = \dfrac{1}{2}$, alors $\lim\limits_{n \to +\infty} \left(\dfrac{1}{2}\right)^n = 0$ \\ 
\item[•] Si $-1 < q < 0$, alors $\lim\limits_{n \to +\infty} q^n = 0$. Exemple : Si $q = -\dfrac{1}{2}$, alors $\lim\limits_{n \to +\infty} \left(-\dfrac{1}{2}\right)^n = 0$ \\ 
\item[•] Si $q < -1$, alors $\lim\limits_{n \to +\infty} q^n$ n'existe pas. Exemple : Si $q = -2$, alors $\lim\limits_{n \to +\infty} \left(-2\right)^n$ n'existe pas. \\ 
\end{itemize}
\item[2.] Étudions la limite de $u_n$ selon les valeurs de $q$ et les valeurs de $u_0$, en rappelant que  : $u_n = u_0q^n$ :
\begin{itemize}
\item[•] Si $q > 1$, alors on a deux cas possibles :
\begin{itemize}
\item[*] Si $u_0 > 0$, alors $\lim\limits_{n \to +\infty} u_n = +\infty$ \\
\item[*] Si $u_0 < 0$, alors $\lim\limits_{n \to +\infty} u_n = -\infty$ \\
\end{itemize}
\item[•] Si $0 < q < 1$, alors $\lim\limits_{n \to +\infty} u_n = 0$ \\
\item[•] Si $-1 < q < 0$, alors $\lim\limits_{n \to +\infty} u_n = 0$ \\
\item[•] Si $q < -1$, alors $\lim\limits_{n \to +\infty} u_n$ n'existe pas. 
\end{itemize}
\end{itemize}

\vspace*{-5cm}

\newpage

\subsection{Un superbe exercice}

1) Soit $\left(u_n\right)_{u \in \N}$ la suite géométrique décroissante définie par :$ \; \; \; \begin{cases}
u_0 + u_1 + u_2 = 999 \\
u_0 \times u_1 \times u_2 = 729 000 \\
\end{cases}$ \\

Déterminer $u_0$, $u_1$ et $u_2$ \\

2) Soit $S = u_0 + u_1 + u_2 + ... + u_n$ \\

Déterminer $n$ tel que $S = 999 999$. \\

1) On peut dire que $u_1 = u_0 \times q $ et $u_2 = u_0 + \times q^2$. \\ On peut aussi dire $u_0 = \dfrac{u_1}{q}$ et $u_2 = u_1 \times q$. \\

On a alors $\dfrac{u_1}{q} \times u_1 \times \left(u_1 \times q\right) = 729 000$ \\

Ainsi, $u_1^3 = 729 000$ et $u_1 = 729 000^{\dfrac{1}{3}}$. \\

D'où $u_1 = 90$ \\

On sait que $u_0 + u_1 + u_2 = 999$ \\

Il vient que $\dfrac{90}{q} + 90 + 90q = 999$. \\

$\dfrac{90}{q} + 90q = 909$ \\

$\dfrac{90 + 90q^2}{q} = 909$ \\

$90q^2 + 90 = 909q$ \\

$90q^2 - 909 + 90 = 0$ \\ 

Donc $q = \dfrac{1}{10}$ ou $q = 1$. \\

Cependant, on sait que la suite $\left(u_n\right)_{n \in \N}$ est décroissante. Donc $q \neq 10$. \\

Ainsi, $q = \dfrac{1}{10}$. \\

On a donc $u_0 = \dfrac{u_1}{q} = \dfrac{90}{\dfrac{1}{10}} = 900$, $u_1 = 90$ et $u_2 = u_1q = 90 \times \dfrac{1}{10} = 9$. \\

\newpage

2) On cherche à trouver $n$ tel que $u_0\dfrac{1-q^{n+1}}{1-q} = 999 999$ \vspace*{.3cm} \\

$900\dfrac{1- \left(\dfrac{1}{10}\right)^{n+1}}{1 - \dfrac{1}{10}} = 999 999$ \vspace*{.3cm} \\

$1000 \left[1 - \left(\dfrac{1}{10}\right)^{n+1}\right] = 999 999$ \vspace*{.3cm} \\

$1 - \left(\dfrac{1}{10}\right)^{n+1} = 0,999999$ \vspace*{.3cm} \\

$ - \left(\dfrac{1}{10}\right)^{n+1} = -0,000001$ \vspace*{.3cm} \\

$\left(\dfrac{1}{10}\right)^{n+1} = 0,000001$ \vspace*{.3cm} \\

$ n + 1 = 6$ \\

$ n = 5$. 