\subsection{Loi normale centrée réduite $N\left(0 \; ; \; 1\right)$}

\subsubsection{Définition}

Soit $X$ une variable aléatoire continue sur $\R$. \\
Soit $f$ la fonction de densité de la loi de probabilité de $X$. \\

On dit que \underline{$X$ suit la loi normale centrée réduite sur $\R$}, noté $N\left(0 \; ; \; 1\right)$, si et seulement si : \\
pour tout $x \in \R, f\left(x\right) = \dfrac{1}{\sqrt{2\pi}}e^{-\frac{1}{2}x^2}$.

\subsubsection{Étude et représentation graphique de $f$}

\begin{tabular}{llll}
\hspace*{-.3cm} Soit la fonction $f:$ & $\R$ & $\longrightarrow$ & $\R$ \\
& $x$ & $\longmapsto$ & $f(x) = \dfrac{1}{\sqrt{2\pi}}e^{-\frac{1}{2}x^2}$
\end{tabular}. \\

\begin{itemize}
\item[•] On a $D_{f} = \R$. 

\item[•] On étudie les variations de $f$. \\

\vspace*{.3cm}

\begin{tabular}{lll}
\hspace*{-.3cm} $\forall x \in \R, f'(x)$ & $=$ & $\dfrac{1}{\sqrt{2\pi}} \times \left(-\dfrac{1}{2}\right) \times 2xe^{-\frac{1}{2}x^2}$ \vspace*{.3cm} \\
& $=$ & $\dfrac{-x}{\sqrt{2\pi}}e^{-\frac{1}{2}x^2}$.
\end{tabular}

\vspace*{.3cm}

\begin{tabular}{lll}
\hspace{-.3cm} On a $f'(x) = 0$ & $\Longleftrightarrow$ & $\dfrac{-x}{\sqrt{2\pi}}e^{-\frac{1}{2}x^2} = 0$ \vspace*{.3cm} \\
& $\Longleftrightarrow$ & $\dfrac{-x}{\sqrt{2\pi}} = 0$ ou $e^{-\frac{1}{2}x^2} = 0$ \vspace*{.3cm} \\
& $\Longleftrightarrow$ & $-x = 0$ ou Impossible \vspace*{.3cm} \\
& $\Longleftrightarrow$ & $x=0$ \\
\end{tabular}

\vspace*{.3cm}

Le signe de $f'(x)$ est le signe de $-x$ car pour tout $x \in D_f, e^{-\frac{1}{2}x^2}>0$. \\

\vspace*{.3cm}

On a : $ \displaystyle {\lim_{x \rightarrow -\infty}} \; f(x) = 0$ \\

De plus, $ \displaystyle {\lim_{x \rightarrow +\infty}} \; f(x) = 0$

\vspace*{.3cm}

Enfin, $f(0) = \dfrac{1}{\sqrt{2\pi}}e^0 = \dfrac{1}{\sqrt{2\pi}} \approx 0,4$. \\

\vspace*{.3cm}

On en déduit le tableau de signes et de variations suivant : \\ 

\variations
x & -\infty & & 0 & & +\infty \\
f'(x) & & + & \z & - & \\
f(x) & \b{0} & \cl & \h{\dfrac{1}{\sqrt{2\pi}}} & \dl & \b{0} \\
\fin

\begin{tikzpicture}[line cap=round,line join=round,>=triangle 45,x=1cm,y=10cm,scale=1.3]

\draw[red] (-5.5,0.001) -- (5.5,0.001);
\draw[->] (-5.9,0) -- (5.9,0);
\foreach \x in {-5,...,-1,1,2,3,4,5}
\draw[shift={(\x,0)}] (0pt,2pt) -- (0pt,-2pt) node[below] {\footnotesize $\x$};
\draw[->] (0,-.05) -- (0,.57);
\foreach \y in {0.1,0.2,0.3,0.4,0.5}
\draw[shift={(0,\y)}] (2pt,0pt) -- (-2pt,0pt) node[left] {\footnotesize $\y$};
\draw (0,-8pt) node[left] {\footnotesize $0$};

\draw [domain=-5:5,smooth,samples=100] plot (\x,{gauss(0,1)}) ; 
\clip (-6,-0.1) rectangle (6,0.6) ;

\begin{pgfonlayer}{background}   
\draw[step=1mm,ultra thin,AntiqueWhite!10] (-6,-0.1) grid (6,0.6) ;
\draw[step=5mm,very thin,AntiqueWhite!30]  (-6,-0.1) grid (6,0.6) ;
\draw[step=1cm,very thin,AntiqueWhite!50]  (-6,-0.1) grid (6,0.6) ;
\draw[step=5cm,thin,AntiqueWhite]          (-6,-0.1) grid (6,0.6) ;
\end{pgfonlayer}

\end{tikzpicture}

\vspace*{.3cm}

\textbf{Remarque :} Cette courbe est appelée « Courbe en cloche de Gauss ». Elle est symétrique par rapport à l'ace des abscisses. \\

\item[•] Montrons que $f$ est une fonction de densité sur $\R$. \\

\begin{itemize}
\item[*] $f$ est définie, continue et positive sur $\R$. \\
\item[*] On a $\displaystyle{\int_{-\infty}^{+\infty} f(x) \; \diff x} = 1$. \\

Vérification à la calculatrice : $\displaystyle{\int_{-1000}^{1000} f(x) \; \diff x} = 1$.
\end{itemize}

\end{itemize}

\newpage

\vspace*{-1cm}

\subsubsection{Probabilité d'un événement}

\vspace*{.3cm}

\begin{minipage}{25cm}

Soit $X$ une variable aléatoire continue sur $\R$ suivant la loi normale contrée réduite. \\

Soient $a\in \R$ et $b \in \R$, avec $a < b$. On a $p\left(a \leqslant X  \leqslant b\right) = \displaystyle{\int_a^b \dfrac{1}{\sqrt{2\pi}} e^{-\frac{1}{2}x^2} \; \diff x}$. \vspace*{.3cm} \\

\begin{tikzpicture}[line cap=round,line join=round,>=triangle 45,x=1cm,y=10cm,scale=1]

\draw[->] (-1.5,0) -- (9.9,0);
\foreach \x in {-5,...,-1,1,2,3,4,5}
\draw[shift={(4+\x,0)}] (0pt,2pt) -- (0pt,-2pt) node[below] {\footnotesize $\x$};
\draw[->] (4,-.05) -- (4,.57);
\foreach \y in {0.1,0.2,0.3,0.5}
\draw[shift={(4,\y)}] (2pt,0pt) -- (-2pt,0pt) node[left] {\footnotesize $\y$};
\draw (4,0.42) node [left] {\footnotesize $0.4$} ;
\draw (4,-8pt) node[left] {\footnotesize $0$};

\draw [domain=0:10,smooth,samples=100] plot (\x,{gauss(4,1)}) ; 
\draw [<->,DarkGreen] (3.5, 0.4) -- (4.5, 0.4) ; 
 
\draw [DarkGreen, pattern color=DarkGreen, pattern=north east lines, smooth, samples=100,domain=3:6] (3,0)  -- plot (\x,{gauss(4,1)}) -- (6,0) -- cycle ;

\draw (3,-10pt) node [below] {\footnotesize $a$} ;  \draw (6,-10pt) node [below] {\footnotesize $b$} ; 

\draw [DarkGreen,pattern color=DarkGreen, pattern=north east lines] (7,.3) node[anchor=south east] {\footnotesize Aire} rectangle (7.5,.35) ; 


\begin{pgfonlayer}{background}   
\draw[step=1mm,ultra thin,AntiqueWhite!10] (-2,-0.1) grid (10.5,0.6) ;
\draw[step=5mm,very thin,AntiqueWhite!30]  (-2,-0.1) grid (10.5,0.6) ;
\draw[step=1cm,very thin,AntiqueWhite!50]  (-2,-0.1) grid (10.5,0.6) ;
\draw[step=5cm,thin,AntiqueWhite]          (-2,-0.1) grid (10.5,0.6) ;
\end{pgfonlayer}

\end{tikzpicture}

\vspace*{.3cm}

\begin{minipage}{10cm}

\textbf{Quelques propriétés :} L'aire totale sous la courbe est $1$. 

\vspace*{.3cm}

\begin{tikzpicture}[line cap=round,line join=round,>=triangle 45,x=1cm,y=10cm,scale=1]

\draw[->] (-1.5,0) -- (9.9,0);
\draw[->] (4,-.05) -- (4,.47);

\draw [domain=0:10,smooth,samples=100] plot (\x,{gauss(4,1)}) ; 
\draw [<->,DarkGreen] (3.5, 0.4) -- (4.5, 0.4) ; 
 
\draw [red, pattern color=red, pattern=north east lines, smooth, samples=100,domain=0:4] (0,0)  -- plot (\x,{gauss(4,1)}) -- (4,0) -- cycle ;
 
\draw [DarkGreen, pattern color=DarkGreen, pattern=north east lines, smooth, samples=100,domain=4:7] (4,0)  -- plot (\x,{gauss(4,1)}) -- (7,0) -- cycle ;

 
\node [blue] (g) at (0,.2) {$p(x \leqslant 0) = 0.5$} ;
\node [black] at (1,.41) {$p(x \leqslant 0) = p(x \geqslant 0) = 0,5$} ; 
\node (n) at (2,.22) {} ; 
\node (G) at (3,.2) {} ;
\draw[->,red] (g.east) parabola bend (n) (G);

\node [blue] (h) at (8,.2) { $p(x \geqslant 0) =0.5$ } ; 
\node (m) at (6,.22) {} ; 
\node (H) at (5,.2) {} ;
\draw[->,DarkGreen] (h.west) parabola bend (m) (H);

\begin{pgfonlayer}{background}   
\draw[step=1mm,ultra thin,AntiqueWhite!10] (-2,-0.1) grid (10.5,0.5) ;
\draw[step=5mm,very thin,AntiqueWhite!30]  (-2,-0.1) grid (10.5,0.5) ;
\draw[step=1cm,very thin,AntiqueWhite!50]  (-2,-0.1) grid (10.5,0.5) ;
\draw[step=5cm,thin,AntiqueWhite]          (-2,-0.1) grid (10.5,0.5) ;
\end{pgfonlayer}
\end{tikzpicture}

\vspace*{.3cm}


\begin{tikzpicture}[line cap=round,line join=round,>=triangle 45,x=1cm,y=10cm,scale=1]

\draw[->] (-1.5,0) -- (9.9,0);
\draw[->] (4,-.05) -- (4,.57);


\draw [domain=0:10,smooth,samples=100] plot (\x,{gauss(4,1)}) ; 
\draw [<->,DarkGreen] (3.5, 0.4) -- (4.5, 0.4) ; 
 
\draw [DarkGreen, pattern color=DarkGreen, pattern=north east lines, smooth, samples=100,domain=0:2.5] (0,0)  -- plot (\x,{gauss(4,1)}) -- (2.5,0) -- cycle ;
 
\draw [DarkGreen, pattern color=DarkGreen, pattern=north east lines, smooth, samples=100,domain=5.5:8] (5.5,0)  -- plot (\x,{gauss(4,1)}) -- (8,0) -- cycle ;

\node [black] at (1,.41) {\begin{tabular}{l}
                              Soit $a>0$ \\
                           $p(x \leqslant -a) = p(x \geqslant a) $
                          \end{tabular}} ; 
                          
\node [DarkGreen] at (0,.05) {$p(x \leqslant -a) = p(x \geqslant a)$} ; 

\node [DarkGreen] at (8.3,.05) {$p(x \leqslant -a) = p(x \geqslant a)$} ; 



\node (G) at (.2,-1) {} ;

\draw (2.5,0) node [below] {\footnotesize $-a$} ;  \draw (5.5,0) node [below] {\footnotesize $a$} ; 




\begin{pgfonlayer}{background}   
\draw[step=1mm,ultra thin,AntiqueWhite!10] (-2,-0.1) grid (10.5,0.6) ;
\draw[step=5mm,very thin,AntiqueWhite!30]  (-2,-0.1) grid (10.5,0.6) ;
\draw[step=1cm,very thin,AntiqueWhite!50]  (-2,-0.1) grid (10.5,0.6) ;
\draw[step=5cm,thin,AntiqueWhite]          (-2,-0.1) grid (10.5,0.6) ;
\end{pgfonlayer}

\end{tikzpicture}

\end{minipage}
\end{minipage}

\newpage

\vspace*{-2cm}

\subsubsection{Exemples}

Soit $X$ une variable aléatoire continue sur $\R$, suivant la loi normale centrée réduit. \\

\begin{itemize}
\item[1)] Calculons tout d'abord $p\left(1 \leqslant X \leqslant 2\right)$. On a deux manière de procéder :

\begin{itemize}
\item[*] Calculatrice : $\approx 0,136$. 
\item[*] $\mathrm{Distrib/normalFrép(1;2)} \approx 0,136$. \\
\end{itemize} 

\item[2)] Calculons $p\left(X \leqslant 1\right)$. 

\begin{itemize}
\item[*] Approximativement : $p\left(-1000 \leqslant X \leqslant 1\right) \approx 0,84$. 
\item[*] Beaucoup mieux : $p\left(X \leqslant 1\right) = 0,5 + p\left(0 \leqslant X \leqslant 1\right)$, d'après le dessin ci-dessous. \\

\begin{tikzpicture}[line cap=round,line join=round,>=triangle 45,x=1cm,y=10cm,scale=.5]

\draw[->] (-1.5,0) -- (9.9,0);
% \foreach \x in {-5,...,-1,1,2,3,4,5}
% \draw[shift={(4+\x,0)}] (0pt,2pt) -- (0pt,-2pt) node[below] {\footnotesize $\x$};
\draw[->] (4,-.05) -- (4,.57);
% \foreach \y in {0.1,0.2,0.3,0.5}
% \draw[shift={(4,\y)}] (2pt,0pt) -- (-2pt,0pt) node[left] {\footnotesize $\y$};
% \draw (4,0.42) node [left] {\footnotesize $0.4$} ;
% \draw (4,-8pt) node[left] {\footnotesize $0$};

\draw [domain=0:10,smooth,samples=100] plot (\x,{gauss(4,1)}) ; 
% \draw [<->,DarkGreen] (3.5, 0.4) -- (4.5, 0.4) ; 
 
\draw [DarkGreen, pattern color=DarkGreen, pattern=north east lines, smooth, samples=100,domain=0:6] (0,0)  -- plot (\x,{gauss(4,1)}) -- (6,0) -- cycle ;


\draw (6,0) node [below] {\footnotesize $1$} ; 




\begin{pgfonlayer}{background}   
\draw[step=1mm,ultra thin,AntiqueWhite!10] (-2,-0.1) grid (10.5,0.6) ;
\draw[step=5mm,very thin,AntiqueWhite!30]  (-2,-0.1) grid (10.5,0.6) ;
\draw[step=1cm,very thin,AntiqueWhite!50]  (-2,-0.1) grid (10.5,0.6) ;
\draw[step=5cm,thin,AntiqueWhite]          (-2,-0.1) grid (10.5,0.6) ;
\end{pgfonlayer}

\end{tikzpicture}

Donc $p\left( X\leqslant 1\right) = 0,841$. \\
\end{itemize} 

\item[3)] Calculons $p\left(X \geqslant 0,5\right)$. 

\begin{itemize}
\item[*] Approximativement : $p\left(0,5 \leqslant X \leqslant 1000\right) \approx 0,309$. 
\item[*] Beaucoup mieux : $p\left(X \geqslant 0,5\right) = p\left(X \leqslant -0,5\right) = 0,5 - p\left(-0,5 \leqslant X \leqslant 0\right) = 0,309$.
\end{itemize}
\end{itemize}

\subsubsection{Intervalles particuliers}

\begin{tabular}{ll}
\begin{minipage}{8cm}
1. $p(-1 \leqslant X \leqslant 1) = 0.68$. \\

\begin{tikzpicture}[line cap=round,line join=round,>=triangle 45,x=1cm,y=10cm,scale=.8]

\draw[->] (0,0) -- (7.9,0);
\foreach \x in {-3,...,-1,1,2,3}
\draw[shift={(4+\x,0)}] (0pt,2pt) -- (0pt,-2pt) node[below] {\footnotesize $\x$};
\draw[->] (4,-.05) -- (4,.57);
\foreach \y in {0.1,0.2,0.3,0.5}
\draw[shift={(4,\y)}] (2pt,0pt) -- (-2pt,0pt) node[left] {\footnotesize $\y$};
\draw (4,0.42) node [left] {\footnotesize $0.4$} ;
\draw (4,-8pt) node[left] {\footnotesize $0$};

\draw [domain=0:8,smooth,samples=100] plot (\x,{gauss(4,1)}) ; 
\draw [<->,DarkGreen] (3.5, 0.4) -- (4.5, 0.4) ; 
 
\draw [DarkGreen, pattern color=DarkGreen, pattern=north east lines, smooth, samples=100,domain=3:5] (3,0)  -- plot (\x,{gauss(4,1)}) -- (5,0) -- cycle ;

\begin{pgfonlayer}{background}   
\draw[step=1mm,ultra thin,AntiqueWhite!10] (-0,-0.1) grid (8.5,0.6) ;
\draw[step=5mm,very thin,AntiqueWhite!30]  (-0,-0.1) grid (8.5,0.6) ;
\draw[step=1cm,very thin,AntiqueWhite!50]  (-0,-0.1) grid (8.5,0.6) ;
\draw[step=5cm,thin,AntiqueWhite]          (-0,-0.1) grid (8.5,0.6) ;
\end{pgfonlayer}

\end{tikzpicture} 
\end{minipage}
&
\begin{minipage}{5cm}
2. $p(-2 \leqslant X \leqslant 2) = 0.954$. \\

\begin{tikzpicture}[line cap=round,line join=round,>=triangle 45,x=1cm,y=10cm,scale=.8]

\draw[->] (0,0) -- (7.9,0);
\foreach \x in {-3,...,-1,1,2,3}
\draw[shift={(4+\x,0)}] (0pt,2pt) -- (0pt,-2pt) node[below] {\footnotesize $\x$};
\draw[->] (4,-.05) -- (4,.57);
\foreach \y in {0.1,0.2,0.3,0.5}
\draw[shift={(4,\y)}] (2pt,0pt) -- (-2pt,0pt) node[left] {\footnotesize $\y$};
\draw (4,0.42) node [left] {\footnotesize $0.4$} ;
\draw (4,-8pt) node[left] {\footnotesize $0$};

\draw [domain=0:8,smooth,samples=100] plot (\x,{gauss(4,1)}) ; 
\draw [<->,DarkGreen] (3.5, 0.4) -- (4.5, 0.4) ; 
 
\draw [DarkGreen, pattern color=DarkGreen, pattern=north east lines, smooth, samples=100,domain=2:6] (2,0)  -- plot (\x,{gauss(4,1)}) -- (6,0) -- cycle ;

\begin{pgfonlayer}{background}   
\draw[step=1mm,ultra thin,AntiqueWhite!10] (-0,-0.1) grid (8.5,0.6) ;
\draw[step=5mm,very thin,AntiqueWhite!30]  (-0,-0.1) grid (8.5,0.6) ;
\draw[step=1cm,very thin,AntiqueWhite!50]  (-0,-0.1) grid (8.5,0.6) ;
\draw[step=5cm,thin,AntiqueWhite]          (-0,-0.1) grid (8.5,0.6) ;
\end{pgfonlayer}

\end{tikzpicture}
\end{minipage}
\vspace*{.6cm} \\

\begin{minipage}{5cm}

3. $p(-3 \leqslant X \leqslant 3) = 0.997$. \\

\begin{tikzpicture}[line cap=round,line join=round,>=triangle 45,x=1cm,y=10cm,scale=.8]

\draw[->] (0,0) -- (7.9,0);
\foreach \x in {-3,...,-1,1,2,3}
\draw[shift={(4+\x,0)}] (0pt,2pt) -- (0pt,-2pt) node[below] {\footnotesize $\x$};
\draw[->] (4,-.05) -- (4,.57);
\foreach \y in {0.1,0.2,0.3,0.5}
\draw[shift={(4,\y)}] (2pt,0pt) -- (-2pt,0pt) node[left] {\footnotesize $\y$};
\draw (4,0.42) node [left] {\footnotesize $0.4$} ;
\draw (4,-8pt) node[left] {\footnotesize $0$};

\draw [domain=0:8,smooth,samples=100] plot (\x,{gauss(4,1)}) ; 
\draw [<->,DarkGreen] (3.5, 0.4) -- (4.5, 0.4) ; 
 
\draw [DarkGreen, pattern color=DarkGreen, pattern=north east lines, smooth, samples=100,domain=1:7] (1,0)  -- plot (\x,{gauss(4,1)}) -- (7,0) -- cycle ;

\begin{pgfonlayer}{background}   
\draw[step=1mm,ultra thin,AntiqueWhite!10] (-0,-0.1) grid (8.5,0.6) ;
\draw[step=5mm,very thin,AntiqueWhite!30]  (-0,-0.1) grid (8.5,0.6) ;
\draw[step=1cm,very thin,AntiqueWhite!50]  (-0,-0.1) grid (8.5,0.6) ;
\draw[step=5cm,thin,AntiqueWhite]          (-0,-0.1) grid (8.5,0.6) ;
\end{pgfonlayer}

\end{tikzpicture}
\end{minipage}
&
\begin{minipage}{5cm}
4. $p(-1.96 \leqslant X \leqslant 1.96) = 0.95$. \\

\begin{tikzpicture}[line cap=round,line join=round,>=triangle 45,x=1cm,y=10cm,scale=.8]

\draw[->] (0,0) -- (7.9,0);
\foreach \x in {-3,...,-1,1,2,3}
\draw[shift={(4+\x,0)}] (0pt,2pt) -- (0pt,-2pt) node[below] {\footnotesize $\x$};
\draw[->] (4,-.05) -- (4,.57);
\foreach \y in {0.1,0.2,0.3,0.5}
\draw[shift={(4,\y)}] (2pt,0pt) -- (-2pt,0pt) node[left] {\footnotesize $\y$};
\draw (4,0.42) node [left] {\footnotesize $0.4$} ;
\draw (4,-8pt) node[left] {\footnotesize $0$};

\draw [domain=0:8,smooth,samples=100] plot (\x,{gauss(4,1)}) ; 
\draw [<->,DarkGreen] (3.5, 0.4) -- (4.5, 0.4) ; 
 
\draw [DarkGreen, pattern color=DarkGreen, pattern=north east lines, smooth, samples=100,domain=2.1:5.9] (2.1,0)  -- plot (\x,{gauss(4,1)}) -- (5.9,0) -- cycle ;

\begin{pgfonlayer}{background}   
\draw[step=1mm,ultra thin,AntiqueWhite!10] (-0,-0.1) grid (8.5,0.6) ;
\draw[step=5mm,very thin,AntiqueWhite!30]  (-0,-0.1) grid (8.5,0.6) ;
\draw[step=1cm,very thin,AntiqueWhite!50]  (-0,-0.1) grid (8.5,0.6) ;
\draw[step=5cm,thin,AntiqueWhite]          (-0,-0.1) grid (8.5,0.6) ;
\end{pgfonlayer}

\end{tikzpicture}
\end{minipage}
\end{tabular}

\vspace*{-5cm}

\newpage

\vspace*{-2cm}

\subsubsection{Exercice : résolution d'équations avec loi loi normale centrée réduite}

Soit $X$ une variable aléatoire continue suivant la loi normale centrée réduite. \\

\begin{itemize}
\item[1.] Déterminer le nombre réel $a$ tel que : $p\left(X \leqslant a\right) = 0,4856$. 

\begin{itemize}
\item[•] À la calculatrice : $\mathrm{Distrib/FracNormale} \Longrightarrow -0,0361$. 
\item[•] Vérification : $\mathrm{Distrib/normaleFrép} \Longrightarrow \left(-1000 \; ; \; -0,0361\right) = 0,4856$. 
\end{itemize}

\vspace*{.3cm}

\item[2.] Déterminer le réel $a$ tel que $p\left(X \geqslant a\right) = 0,2347$. \\

On a $p\left(X \geqslant a\right) = p\left(X \leqslant -a\right)$ (voir la représentation graphique ci-dessous). \\


\begin{tikzpicture}[line cap=round,line join=round,>=triangle 45,x=1cm,y=10cm,scale=.8]

\draw[->] (1.5,0) -- (8,0);
\draw[->] (4,-.05) -- (4,.47);


\draw [domain=2:8,smooth,samples=100] plot (\x,{gauss(4,1)}) ; 
% \draw [<->,DarkGreen] (3.5, 0.4) -- (4.5, 0.4) ; 
 
\draw [DarkGreen, pattern color=DarkGreen, pattern=north east lines, smooth, samples=100,domain=0:2.5] (0,0)  -- plot (\x,{gauss(4,1)}) -- (2.5,0) -- cycle ;
 
\draw [DarkGreen, pattern color=DarkGreen, pattern=north east lines, smooth, samples=100,domain=5.5:8] (5.5,0)  -- plot (\x,{gauss(4,1)}) -- (8,0) -- cycle ;


\draw (2.5,0) node [below] {\footnotesize $-a$} ;  \draw (5.5,0) node [below] {\footnotesize $a$} ; 

\draw[<->,DarkGreen] (1.5,-.03) parabola bend (4,-.08)  (6.5,-.03);
\draw  (4,-.08) node [below] {$=$} ; 

\begin{pgfonlayer}{background}   
\draw[step=1mm,ultra thin,AntiqueWhite!10] (-0,-0.1) grid (8.5,0.5) ;
\draw[step=5mm,very thin,AntiqueWhite!30]  (-0,-0.1) grid (8.5,0.5) ;
\draw[step=1cm,very thin,AntiqueWhite!50]  (-0,-0.1) grid (8.5,0.5) ;
\draw[step=5cm,thin,AntiqueWhite]          (-0,-0.1) grid (8.5,0.5) ;
\end{pgfonlayer}
\end{tikzpicture}

\vspace*{.3cm}

\begin{itemize}
\item[•] À la calculatrice : $\mathrm{Distrib/FracNormale} \Longrightarrow -a = -0,7235$. Donc $a = 0,7235$. 
\item[•] Vérification : $\mathrm{Distrib/normaleFrép} \Longrightarrow \left(0,7235 \; ; \; 1000\right) = 0,2347$. 
\end{itemize}

\vspace*{.3cm}

\samepage

\item[3.] Déterminer le nombre réel $a$ tel que $p\left(-a \leqslant X \leqslant a\right) = 0,95$.

On pose $\varphi\left(a\right) = p\left(X \leqslant a\right)$. Ainsi $\varphi\left(-a\right) = p\left(X \leqslant -a\right) = p\left(X \geqslant a\right)$, d'après la figure précédente. \\

On a alors $\varphi\left(a\right) + \varphi\left(-a\right) = p\left(X \leqslant a\right) + p\left(X \geqslant a\right) = 1$ (voir la figure 1. ci-dessous). \\ D'où $\varphi\left(-a\right) = 1 - \varphi\left(a\right)$. \\

On a donc $p\left(-a \leqslant X \leqslant a\right) = p\left(X \leqslant a\right) - p\left(X \leqslant -a\right)$, comme le montre la figure 2. ci-dessous. \\
\end{itemize}

\vspace*{-.2cm}

\begin{tabular}{cc}
\begin{minipage}{8cm}

\begin{tikzpicture}[line cap=round,line join=round,>=triangle 45,x=1cm,y=10cm,scale=.8]

\draw[->] (1.5,0) -- (8,0);
\draw[->] (4,-.05) -- (4,.47);


\draw [domain=0:8,smooth,samples=100] plot (\x,{gauss(4,1)}) ; 
 
\draw [DarkGreen, pattern color=DarkGreen, pattern=north east lines, smooth, samples=100,domain=0:5.5] (0,0)  -- plot (\x,{gauss(4,1)}) -- (5.5,0) -- cycle ;
\draw [DarkGreen, pattern color=DarkGreen!30, pattern=north west lines, smooth, samples=100,domain=5.5:8] (5.5,0)  -- plot (\x,{gauss(4,1)}) -- (8,0) -- cycle ;
 
% \draw [DarkGreen, pattern color=DarkGreen, pattern=north east lines, smooth, samples=100,domain=5.5:8] (5.5,0)  -- plot (\x,{gauss(4,1)}) -- (8,0) -- cycle ;


% \draw (2.5,0) node [below] {\footnotesize $-a$} ; 
\draw (5.5,0) node [below] {\footnotesize $a$} ; 
 

\begin{pgfonlayer}{background}   
\draw[step=1mm,ultra thin,AntiqueWhite!10] (-0,-0.1) grid (8.5,0.5) ;
\draw[step=5mm,very thin,AntiqueWhite!30]  (-0,-0.1) grid (8.5,0.5) ;
\draw[step=1cm,very thin,AntiqueWhite!50]  (-0,-0.1) grid (8.5,0.5) ;
\draw[step=5cm,thin,AntiqueWhite]          (-0,-0.1) grid (8.5,0.5) ;
\end{pgfonlayer}

\end{tikzpicture}
\end{minipage}
&
\begin{minipage}{8cm}

\begin{tikzpicture}[line cap=round,line join=round,>=triangle 45,x=1cm,y=10cm,scale=.8]

\draw[->] (0.5,0) -- (8,0);
\draw[->] (4,-.05) -- (4,.47);


\draw [domain=0:8,smooth,samples=100] plot (\x,{gauss(4,1)}) ; 
 
\draw [DarkGreen, pattern color=DarkGreen, pattern=north east lines, smooth, samples=100,domain=2.5:5.5] (2.5,0)  -- plot (\x,{gauss(4,1)}) -- (5.5,0) -- cycle ;
\draw [DarkGreen, pattern color=DarkGreen!30, pattern=north west lines, smooth, samples=100,domain=5.5:8] (5.5,0)  -- plot (\x,{gauss(4,1)}) -- (8,0) -- cycle ;
\draw [DarkGreen, pattern color=DarkGreen!30, pattern=north west lines, smooth, samples=100,domain=0:5.5] (0,0)  -- plot (\x,{gauss(4,1)}) -- (5.5,0) -- cycle ;

\draw (2.5,0) node [below] {\footnotesize $-a$} ; 
\draw (5.5,0) node [below] {\footnotesize $a$} ; 
 

\begin{pgfonlayer}{background}   
\draw[step=1mm,ultra thin,AntiqueWhite!10] (-0,-0.1) grid (8.5,0.5) ;
\draw[step=5mm,very thin,AntiqueWhite!30]  (-0,-0.1) grid (8.5,0.5) ;
\draw[step=1cm,very thin,AntiqueWhite!50]  (-0,-0.1) grid (8.5,0.5) ;
\draw[step=5cm,thin,AntiqueWhite]          (-0,-0.1) grid (8.5,0.5) ;
\end{pgfonlayer}

\end{tikzpicture}
\end{minipage}
\\
Figure 1. & Figure 2. \\
\end{tabular}

\vspace*{.3cm}

\begin{tabular}{lll}
D'où $p\left(-a \leqslant X \leqslant a\right)$ & $=$ & $\varphi\left(a\right) - \varphi\left(-a\right)$ \\
& $=$ & $\varphi\left(a\right) - \left[1 - \varphi\left(a\right) \right]$ \\
& $=$ & $2 \varphi\left(a\right) - 1$. \\
\end{tabular}

\vspace*{.3cm}

\begin{tabular}{lll}
On cherche $a$ tel que $p\left(-a\leqslant X \leqslant a\right) = 0,95$ & $\Longleftrightarrow$ & $2 \varphi\left(a\right) - 1 = 0,95$ \\
& $\Longleftrightarrow$ & $2\varphi\left(a\right) = 1,95$ \\
& $\Longleftrightarrow$ & $\varphi\left(a\right) = 0,975$ \\
& $\Longleftrightarrow$ & $p\left(X\leqslant a\right) = 0,975$ \\
\end{tabular}

\vspace*{.3cm}

\hspace{.15cm} Grâce à la calculatrice, on trouve : $ a \approx 1,96$. 

\vspace*{-5cm}

\newpage

\subsection{Loi normale $N\left(\mu \; ; \; \sigma^2\right)$, avec $\mu$ la moyenne et $\sigma$ l'écart-type de la variable aléatoire}

\subsubsection{Introduction et rappel : la loi binomiale}

Soit $X$ une variable aléatoire suivant la loi binomiale de paramètres $n$ et $p$. \\

On a $p\left(X = k\right) = \begin{pmatrix}
n \\
k \\
\end{pmatrix} p^k \left(1 - p\right)^{n-k}$, avec $k \leqslant n$. \\

On a aussi $E\left(x\right) = np$ et $\sigma\left(x\right) = \sqrt{np\left(1-p\right)}$.

\subsubsection{Exercice}

Une entreprise fabrique des parapluies. La production hebdomadaire est fidèle aux précisions avec une probabilité de 0,7. \\
O, suppose que les productions hebdomadaires sont indépendantes les unes des autres. \\
On fait une étude sur 84 semaines. \\
Soit $X$ la variable aléatoire égale au nombre de semaines où la production est fidèle aux précisions. \\

\begin{itemize}
\item[1.] Quelle est la loi de probabilité suivie par $X$ ? \\
Déterminer une espérance mathématique $\mu$. \\ Déterminer son écart-type $\sigma$. \\

$X$ suit la loi binomiale de paramètres $n = 84$ et $p = 0,7$. \\

On a $\mu = np$, donc $\mu = 84 \times 0,7 = 58,8$. \\
On a aussi $\sigma = \sqrt{np\left(1-p\right)}$, donc $\sigma = \sqrt{84 \times 0,7 \times \left(1-0,7\right)} = 4,2$.

\item[2.] On pose $T = \dfrac{X - \mu}{\sigma}$. On a donc $T = \dfrac{X - 58,8}{4,2}$. \\

On admet que $T$ suit la loi normale centrée réduite. \\

Déterminer un intervalle $\left[a \; ; \; b\right]$ d'amplitude minimale tel que $p\left(a \leqslant X \leqslant b\right) = 0,95$. \\ Interpréter. \\

On sait que $p\left(-1,96 \leqslant T \leqslant 1,96 \right) = 0,95$. \\

\begin{tabular}{lll}
\hspace{-.3cm} Donc $p\left(-1,96 \leqslant \dfrac{X-58,8}{4,2} \leqslant 1,96\right) = 0,95$ & $\Longleftrightarrow$ & $p\left(-8,232 \leqslant X - 58,8 \leqslant 8,232\right) = 0,95$ \vspace*{.3cm} \\
& $\Longleftrightarrow$ & $p\left(50,568 \leqslant X \leqslant 67,032\right) = 0,95$ \\
\end{tabular}

\vspace*{.3cm}

On a : \\

La probabilité d'avoir entre $51$ et $67$ semaines est de $0,95$.

\end{itemize}


\newpage

\subsubsection{Loi normale d'espérance $\mu$ et d'écart-type $\sigma$ : définition}

Soit $X$ une variable aléatoire continue sur $\R$. On note $E\left(x\right) = \mu$ et $\sigma\left(x\right) = \sigma$. \\

On dit que $X$ suite la loi normale, notée $N\left(\mu \; ; \; \sigma^2\right)$ si et seulement si la variable aléatoire $T = \dfrac{X - \mu}{\sigma}$ suit la loi normale centré réduite $N\left(0 \; ; \; 1\right)$.

\subsubsection{Influence des paramètres}

$f$, une fonction de densité associée à une variable aléatoire suivant la loi normale $N\left(\mu \; ; \; \sigma^2\right)$, est une fonction similaire à la fonction de densité de densité associée à une variable aléatoire suivant la loi normale centrée réduite. \\

\begin{tabular}{llll}
\hspace{-.3cm} On a $f:$ & $\R$ & $\longrightarrow$ & $\R$ \\
& $x$ & $\longmapsto$ & $f(x) = \dfrac{1}{\sigma\sqrt{2\pi}}e^{-\frac{1}{2}\left(\frac{x-\mu}{\sigma}\right)^2}$. \\
\end{tabular}

\vspace*{.3cm} 

Pour le cas particulier $\mu = 0$ et $\sigma = 1$, on a : \\

\begin{tabular}{llll}
$f:$ & $\R$ & $\longrightarrow$ & $\R$ \\
& $x$ & $\longmapsto$ & $f(x) = \dfrac{1}{\sqrt{2\pi}}e^{-\frac{1}{2}x^2}$. \\
\end{tabular}

\vspace*{.5cm}

On a les représentations graphiques suivantes : \\

\begin{tabular}{cc}
Figure $F_1$ & Figure $F_2$ \vspace*{.3cm} \\
\begin{minipage}{8cm}

\begin{tikzpicture}[line cap=round,line join=round,>=triangle 45,x=1cm,y=10cm,scale=.85]
\draw[->] (-2,0) -- (6.9,0);
\draw[->] (0,-.05) -- (0,.57);
\draw (0,-8pt) node[left] {\footnotesize $0$};

\draw [domain=-2:6,smooth,samples=100] plot (\x,{gauss(2,1)}) ; 
\clip (-2,-0.1) rectangle (7,0.6) ;

\draw [DarkGreen, pattern color=DarkGreen, pattern=north east lines, smooth, samples=100,domain=.5:3.5] (.5,0)  -- plot (\x,{gauss(2,1)}) -- (3.5,0) -- cycle ;

\draw [DarkGreen] (.5,0) node [below] {\footnotesize $\mu - \sigma $} ; 
\draw [dashed] (2,0) node [below] {\footnotesize $\mu$} -- (2,.4) ; 
\draw [DarkGreen] (3.5,0) node [below] {\footnotesize $\mu + \sigma$} ; 

\draw[<-] (2.3,.3) parabola bend (2.8,.35)  (3.5,.4) node [right] {$p \simeq 0.68$} ;

\begin{pgfonlayer}{background}   
\draw[step=1mm,ultra thin,AntiqueWhite!10] (-2,-0.1) grid (7,0.6) ;
\draw[step=5mm,very thin,AntiqueWhite!30]  (-2,-0.1) grid (7,0.6) ;
\draw[step=1cm,very thin,AntiqueWhite!50]  (-2,-0.1) grid (7,0.6) ;
\draw[step=5cm,thin,AntiqueWhite]          (-2,-0.1) grid (7,0.6) ;
\end{pgfonlayer}

\end{tikzpicture}
\end{minipage}
&
\begin{minipage}{5cm}

\begin{tikzpicture}[line cap=round,line join=round,>=triangle 45,x=1cm,y=10cm,scale=.85]
\draw[->] (-2,0) -- (6.9,0);
\draw[->] (0,-.05) -- (0,.57);
\draw (0,-8pt) node[left] {\footnotesize $0$};

\draw [domain=-2:6,smooth,samples=100] plot (\x,{gauss(2,2)}) ; 
\clip (-2,-0.1) rectangle (7,0.6) ;

\draw [DarkGreen, pattern color=DarkGreen, pattern=north east lines, smooth, samples=100,domain=-1:5] (-1,0)  -- plot (\x,{gauss(2,2)}) -- (5,0) -- cycle ;

\draw [DarkGreen] (-1,0) node [below] {\footnotesize $\mu - \sigma $} ; 
\draw [dashed] (2,0) node [below] {\footnotesize $\mu$} -- (2,.2) ; 
\draw [DarkGreen] (5,0) node [below] {\footnotesize $\mu + \sigma$} ; 

\draw[<-] (2.3,.15) parabola bend (2.8,.22)  (3.5,.2) node [right] {$p \simeq 0.68$} ;

\begin{pgfonlayer}{background}   
\draw[step=1mm,ultra thin,AntiqueWhite!10] (-2,-0.1) grid (7,0.6) ;
\draw[step=5mm,very thin,AntiqueWhite!30]  (-2,-0.1) grid (7,0.6) ;
\draw[step=1cm,very thin,AntiqueWhite!50]  (-2,-0.1) grid (7,0.6) ;
\draw[step=5cm,thin,AntiqueWhite]          (-2,-0.1) grid (7,0.6) ;
\end{pgfonlayer}

\end{tikzpicture}
\end{minipage}
\\
\end{tabular}

\vspace*{.3cm}

Les courbes sont symétriques par rapport à la droite d'équation $x = \mu$. \\
$\sigma$ est une mesure de dispersion par rapport à la moyenne. \\

Sur la figure 1, on peut dire que $\sigma$ est « petit ». \\
Sur la figure 2, on peut dire que $\sigma$ est « grand ». \\
En effet, la courbe de la figure $F_1$ est plus petite que la courbe de la figure $F_2$ car la bourbe de la figure $F_2$ est moins haute et donc compense en largeur.

\newpage
\subsubsection{Intervalle « un sigma », « deux sigmas », « trois sigmas »}

Soit $X$ une variable aléatoire continue suivant la loi normale $N\left(\mu \; ; \; \sigma^2\right)$. Soit $T = \dfrac{X - \mu}{\sigma}$. \\

On sait que $T$ suit la loi normale centrée réduite $N\left(0 \; ; \; 1\right)$. \\

\begin{tabular}{ll}
\hspace{-.3cm} On sait que : & • $p\left(-1 \leqslant T \leqslant 1\right) = 0,68$ \\
& • $p\left(-2 \leqslant T \leqslant 2\right) = 0,954$ \\
& • $p\left(-3 \leqslant T \leqslant 3\right) = 0,997$ \\
& • $p\left(-1,96 \leqslant T \leqslant 1,96\right) = 0,95$ \\
\end{tabular}

\vspace*{.5cm}

\begin{tabular}{lll}
\hspace{-.3cm} Puisque $T = \dfrac{X - \mu}{\sigma}$, on a $p\left(-1 \leqslant T \leqslant 1\right) = 0,68$ & $\Longleftrightarrow$ & $p\left(-1 \leqslant \dfrac{X-\mu}{\sigma} \leqslant 1\right) = 0,68$ \vspace*{.3cm} \\
& $\Longleftrightarrow$ & $p\left(-\sigma \leqslant X - \mu \leqslant \sigma\right) = 0,68$ \vspace*{.3cm} \\
& $\Longleftrightarrow$ & $p\left(\mu - \sigma \leqslant X \leqslant \mu + \sigma\right) = 0,68$ \\
\end{tabular}

\vspace*{.5cm}

\begin{tabular}{lll}
\hspace{-.3cm} On a de plus $p\left(-2 \leqslant T \leqslant 2\right) = 0,954$ & $\Longleftrightarrow$ & $p\left(-2 \leqslant \dfrac{X-\mu}{\sigma} \leqslant 2\right) = 0,954$ \vspace*{.3cm} \\
& $\Longleftrightarrow$ & $p\left(-2\sigma \leqslant X - \mu \leqslant 2\sigma\right) = 0,954$ \vspace*{.3cm} \\
& $\Longleftrightarrow$ & $p\left(\mu - 2\sigma \leqslant X \leqslant \mu + 2\sigma\right) = 0,954$ \\
\end{tabular}

\vspace*{.5cm}

\begin{tabular}{lll}
\hspace{-.3cm} On a enfin $p\left(-3 \leqslant T \leqslant 3\right) = 0,997$ & $\Longleftrightarrow$ & $p\left(-3 \leqslant \dfrac{X-\mu}{\sigma} \leqslant 3\right) = 0,997$ \vspace*{.3cm} \\
& $\Longleftrightarrow$ & $p\left(-3\sigma \leqslant X - \mu \leqslant 3\sigma\right) = 0,997$ \vspace*{.3cm} \\
& $\Longleftrightarrow$ & $p\left(\mu - 3\sigma \leqslant X \leqslant \mu + 3\sigma\right) = 0,997$ \\
\end{tabular}

\vspace*{.5cm}

On peut en déduire la représentation graphique suivante : \\

\begin{tikzpicture}[line cap=round,line join=round,>=triangle 45,x=1cm,y=10cm,scale=1]

\draw[->] (-2,0) -- (6.9,0);
\draw[->] (0,-.05) -- (0,.57);

\clip (-2,-0.1) rectangle (7,0.6) ;
\draw [domain=-2:6,smooth,samples=100] plot (\x,{gauss(2,1)}) ; 

\draw [blue, pattern color=blue!40, 
       pattern=dots, 
       smooth, samples=100,domain=-1:5] 
       (-1,0)  -- plot (\x,{gauss(2,1)}) -- (5,0) -- cycle ;  
\draw [blue] (-1,0) node [below] {\tiny $\mu - \sigma $} ; 
\draw [blue] (5,0) node [below] {\tiny $\mu + \sigma$} ;        

\draw [red, pattern color=white, 
       pattern=north east lines, 
       smooth, samples=100,domain=0:4] 
       (0,0)  -- plot (\x,{gauss(2,1)}) -- (4,0) -- cycle ; 

\draw [red, pattern color=red!70, 
      pattern=north east lines, smooth, samples=100,domain=0:4] 
      (0,0)  -- plot (\x,{gauss(2,1)}) -- (4,0) -- cycle ;
\draw [red] (0,0) node [below] {\tiny $\mu - \sigma $} ; 
\draw [red] (4,0) node [below] {\tiny $\mu + \sigma$} ;

\draw [DarkGreen,pattern color=white,
       pattern=horizontal lines,
       smooth, samples=100,domain=1:3] 
       (1,0)  -- plot (\x,{gauss(2,1)}) -- (3,0) -- cycle ;
\draw [DarkGreen,
       pattern color=DarkGreen,
       pattern=horizontal lines,
       smooth, samples=100,domain=1:3] 
       (1,0)  -- plot (\x,{gauss(2,1)}) -- (3,0) -- cycle ;

\draw [DarkGreen] (1,0) node [below] {\tiny $\mu - \sigma $} ; 
\draw [DarkGreen] (3,0) node [below] {\tiny $\mu + \sigma$} ; 
\draw [dashed,thick] (2,0) node [below] {\footnotesize $\mu$} -- (2,.4) ; 

\draw [DarkGreen,pattern color=DarkGreen, pattern=horizontal lines] (4.5,.35)  rectangle (5,.4) node[anchor=north west] {\footnotesize $68\%$} ; 

\draw [red,pattern color=red, pattern=north east lines] (4.5,.25)  rectangle (5,.3) node[anchor=north west] {\footnotesize $95.4\%$} ; 

\draw [blue,pattern color=blue, pattern=dots] (4.5,.15) rectangle (5,.2) node[anchor=north west] {\footnotesize $99.7\%$} ; 

\begin{pgfonlayer}{background}   
\draw[step=1mm,ultra thin,AntiqueWhite!10] (-2,-0.1) grid (7,0.6) ;
\draw[step=5mm,very thin,AntiqueWhite!30]  (-2,-0.1) grid (7,0.6) ;
\draw[step=1cm,very thin,AntiqueWhite!50]  (-2,-0.1) grid (7,0.6) ;
\draw[step=5cm,thin,AntiqueWhite]          (-2,-0.1) grid (7,0.6) ;
\end{pgfonlayer}

\end{tikzpicture}

\vspace*{.3cm}

\begin{tabular}{ll}
\hspace{-.3cm} \textbf{Remarques :} & • On a aussi $p\left(\mu -1,96\sigma \leqslant X \leqslant \mu + 1,96 \sigma\right) = 0,95 = 95 \; $ \%. \vspace*{.3cm} \\
& • Ainsi, les probabilités ne dépendent ni de $\mu$ ni de $\sigma$. 
\end{tabular}

\newpage

\vspace*{-1.7cm}

\subsubsection{Exercices}

\textbf{Exercice n°1} 

Les \hbox{températures du mois de juillet autour du lac Léman suivent la loi normale d'espérance $18,2^{\circ}$ C} et d'écart-type $3,6^{\circ}$ C. \\

Quelle est la probabilité que la température un jour du mois de juillet soit : \\

\begin{itemize}
\item[1.] comprise entre $20^{\circ}$ C et $24,5^{\circ}$. 
\item[2.] inférieure à $16^{\circ}$ C. 
\item[3.] Supérieure à $21^{\circ}$ C. 
\end{itemize}

\vspace*{.3cm}

\textbf{Solution} 

Soit $X$ la variable aléatoire continue correspondant à la température. \\

\begin{itemize}
\item[1.] On cherche $p\left(20 \leqslant X \leqslant 24,5\right)$. 

D'après la calculatrice, en tapant $\mathrm{distrib}/2/\left(20 \; ; \; 24,5 \; ; \; 18,2 \; ; \; 3,6\right)$, on trouve $0,27$. 

Il y a donc $27 \;$ \% de chance que la température soit comprise entre $20$ et $24,5^{\circ}$ C. \\

\item[2.] On cherche $p\left(X \leqslant 16\right)$. 

D'après la calculatrice, en tapant $\mathrm{distrib}/2/\left(-1000 \; ; \; 16 \; ; \; 18,2 \; ; \; 3,6\right)$, on trouve $0,271$. \\

\textbf{Autre méthode :} Beaucoup plus élégant : On a $p\left(X \leqslant 16\right) = 0,5 - p\left(16 \leqslant X \leqslant 18,2\right)$. (on peut vérifier avec le dessin ci-dessous). \\

\begin{tabular}{ll}
\begin{minipage}{8cm}
\begin{tikzpicture}[line cap=round,line join=round,>=triangle 45,x=1cm,y=10cm,scale=.8]

\draw[->] (-2,0) -- (6.9,0);
\draw[->] (0,-.05) -- (0,.57);

\clip (-2,-0.1) rectangle (7,0.6) ;
\draw [domain=-2:6,smooth,samples=100] plot (\x,{gauss(2,1)}) ; 

\draw [DarkGreen, pattern color=DarkGreen, pattern=north east lines, smooth, samples=100,domain=-2:1] (-2,0)  -- plot (\x,{gauss(2,1)}) -- (1,0) -- cycle ;

\draw [blue, pattern color=blue, pattern=north west lines, smooth, samples=100,domain=1:2] (1,0)  -- plot (\x,{gauss(2,1)}) -- (2,0) -- cycle ;

\draw [red, pattern color=red, pattern=north east lines, smooth, samples=100,domain=2:6] (2,0)  -- plot (\x,{gauss(2,1)}) -- (6,0) -- cycle ;

\draw [DarkGreen] (1,0) node [below] {\footnotesize $16$} ; 
\draw  (2,0) node [below] {\footnotesize $\mu = 18.2$} -- (2,.4) ; 

\newcommand{\TextSoulign}[3]{\emph{\textcolor{#1}{\underline{#2\textcolor{black}{#3}}}}}

\TextSoulign{red}{phrase soulignée en rouge : début de phrase rouge }{ et fin de phrase noire}

\draw [blue] (3.5,.45) node  {$p(X\leqslant 16) = {\textcolor{red} {\underline{\textcolor{blue}{0.5}}}} -p \underline{(16\leqslant X \leqslant 18.2)}$} ;

\draw[<-,red] (2.3,.3) parabola bend (2.8,.38)  (3,.435) ;
\draw[<-,blue] (1.7,.2) parabola bend (4,.3)  (5,.435) ;

\begin{pgfonlayer}{background}   
\draw[step=1mm,ultra thin,AntiqueWhite!10] (-2,-0.1) grid (7,0.6) ;
\draw[step=5mm,very thin,AntiqueWhite!30]  (-2,-0.1) grid (7,0.6) ;
\draw[step=1cm,very thin,AntiqueWhite!50]  (-2,-0.1) grid (7,0.6) ;
\draw[step=5cm,thin,AntiqueWhite]          (-2,-0.1) grid (7,0.6) ;
\end{pgfonlayer}

\end{tikzpicture}
\end{minipage}
&
\begin{minipage}{8cm}
D'après la calculatrice, \\ en tapant $\mathrm{distrib}/2/\left(16 \; ; \; 18,2 \; ; \; 18,2 \; ; \; 3,6\right)$, \\ on trouve $0,229$. \\

D'où $p\left(X \leqslant 16\right) = 0,5 - 0,229= 0,271$. \\

Les résultats sont cohérents. 
\end{minipage}
\\
\end{tabular}

\vspace*{.3cm}

\item[3.] On cherche $p\left(X \geqslant 21\right)$, ce que l'on ne sais pas faire avec la calculatrice. 

On sait que $p\left(X \leqslant -21\right) + p\left(X \geqslant 21\right) = 1 \Longleftrightarrow p\left(X \geqslant 21\right) = 1 - p\left(X \leqslant 21\right)$. \\

\begin{tabular}{ll}
\begin{minipage}{8cm}
\begin{tikzpicture}[line cap=round,line join=round,>=triangle 45,x=1cm,y=10cm,scale=.8]
\draw[->] (-2,0) -- (6.9,0);
\draw[->] (0,-.05) -- (0,.57);

\draw [domain=-2:6,smooth,samples=100] plot (\x,{gauss(2,1)}) ; 
\clip (-2,-0.1) rectangle (7,0.6) ;

% \draw [red, pattern color=red, pattern=north east lines, smooth, samples=100,domain=3.5,6] (3.5,0)  -- plot (\x,{gauss(2,1)}) -- (6,0) -- cycle ;

\draw [red, pattern color=red, 
       pattern=north east lines, 
       smooth, samples=100,domain=3.5:6] 
       (3.5,0)  -- plot (\x,{gauss(2,1)}) -- (6,0) -- cycle ;        
       
\draw [dashed] (2,0) node [below,blue] { $\begin{array}{c} 18.2\\ \mu \end{array}$} -- (2,.4) ; 
\draw [blue] (3.5,0) node [below] { $21$} ; 

\begin{pgfonlayer}{background}   
\draw[step=1mm,ultra thin,AntiqueWhite!10] (-2,-0.1) grid (7,0.6) ;
\draw[step=5mm,very thin,AntiqueWhite!30]  (-2,-0.1) grid (7,0.6) ;
\draw[step=1cm,very thin,AntiqueWhite!50]  (-2,-0.1) grid (7,0.6) ;
\draw[step=5cm,thin,AntiqueWhite]          (-2,-0.1) grid (7,0.6) ;
\end{pgfonlayer}

\end{tikzpicture}
\end{minipage}
&
\begin{minipage}{8cm}
D'après la calculatrice, en tapant $\mathrm{distrib}/2/\left(-1000 \; ; \; 21 \; ; \; 18,2 \; ; \; 3,6\right)$, on trouve $0,781$. \\

D'où $p\left(X \geqslant 21\right) = 1 - 0,781 = 0,219$. \\

\textbf{Autre méthode} : beaucoup plus élégant : on a $p\left(X \geqslant 21\right) = 1 - \left[0,5 + p\left(18,2 \leqslant X \leqslant 21\right)\right] = 0,5 - p\left(18,2 \geqslant 21\right)$. \\

On trouve $p\left(X \leqslant 21\right) = 0,219$. \\ Les résultats sont cohérents. 
\end{minipage}
\\
\end{tabular}
\end{itemize}

\vspace*{-5cm}

\newpage

\vspace*{-1cm}

\textbf{Exercice n°2} \\

Le test de C.A.T.E.L. est un test destinée à mesurer le coefficient intellectuel d'un individu. Les résultats suivent la loi normale d'espérance $100$ et d'écart-type  $24$. \\

\begin{itemize}
\item[1.] Quel pourcentage de la population : \\
\begin{itemize}
\item[a)] a un Q.I. compris entre $76$ et $124$ ?
\item[b)] a un Q.I. inférieur à $70$ ? \\
\end{itemize}

\item[2.] L'organisation M.E.N.S.A. est ouverte aux personnes dont le Q.I. est supérieur à celui de $98 \;$ \% de la population. L'actrice Sharon Stone a prétendu faire partie de la M.E.N.S.A. \\
Si tel est le cas, que peut-on en déduire concernant son Q.I. ?
\end{itemize}

\vspace*{.3cm}

\textbf{Solution} \\

Soit $X$ a variable aléatoire continue correspondant au Q.I. de la population. \\

\begin{itemize}
\item[1.] 
\begin{itemize}
\item[a)] On a $p\left(76 \leqslant X \leqslant 124\right) = p\left(100 - 24 \leqslant X \leqslant 100 + 24 \right)$. \\

On sait que $p\left(\mu - \sigma \leqslant X \leqslant \mu + \sigma\right) \approx 0,68$. \\

Donc $68 \;$ \% de la population a un Q.I. compris entre $76$ et $124$. \\ 

\item[b)] On a $p\left(X \leqslant 70\right) = 0,5 - p\left(70 \leqslant X \leqslant 100\right)$, comme le montre la représentation graphique suivante : \\

\begin{tikzpicture}[line cap=round,line join=round,>=triangle 45,x=1cm,y=10cm,scale=.8]
\draw[->] (-2,0) -- (6.9,0);
\draw[->] (0,-.05) -- (0,.57);

\draw [domain=-2:6,smooth,samples=100] plot (\x,{gauss(2,1)}) ; 
\clip (-2,-0.1) rectangle (7,0.6) ;

\draw [red, pattern color=red, 
       pattern=north east lines, 
       smooth, samples=100,domain=-6:1] 
       (6,0)  -- plot (\x,{gauss(2,1)}) -- (1,0) -- cycle ;        
       
\draw [dashed] (2,0) node [below,blue] { $\begin{array}{c} 100\\ \mu \end{array}$} -- (2,.4) ; 
\draw [blue] (1,0) node [below] { $70$} ; 

\begin{pgfonlayer}{background}   
\draw[step=1mm,ultra thin,AntiqueWhite!10] (-2,-0.1) grid (7,0.6) ;
\draw[step=5mm,very thin,AntiqueWhite!30]  (-2,-0.1) grid (7,0.6) ;
\draw[step=1cm,very thin,AntiqueWhite!50]  (-2,-0.1) grid (7,0.6) ;
\draw[step=5cm,thin,AntiqueWhite]          (-2,-0.1) grid (7,0.6) ;
\end{pgfonlayer}

\end{tikzpicture}

\vspace*{.3cm}

D'après la calculatrice, en tapant $\mathrm{distrib}/2/\left(70 \; ; \; 100 \; ; \; 100 \; ; \; 24\right)$, on a $p\left(70 \leqslant X \leqslant 100\right) = 0,39$. \\

Donc $p\left(X \leqslant 70\right) = 0,5 - 0,39 = 0,11$. \\

Donc $11 \;$ \% de la population a un Q.I. inférieur à $70$. \\ 
\end{itemize}
\item[2.] Soit $a$ le Q.I. de Sharon Stone. \\

On cherche $a$ tel que $p\left(X \leqslant a\right) = 0,98$. \\

D'après la calculatrice, en tapant $\mathrm{distrib}/3/\mathrm{fracnormale}\left(0,98 \; ; \; 100 \; ; \; 24\right)$, on trouve $a = 149$. \\

Donc le Q.I. de Sharon Stone est supérieur ou égal à $149$.

\vspace*{-50cm}

\end{itemize}

\vspace*{-50cm}

\newpage

\subsubsection{Résolution d'équations avec la loi normale}

Soit $X$ une variable aléatoire continue suivant la loi normale $N\left(\mu \; ; \; \sigma^2\right)$, avec $\mu = 90$ et $\sigma = 20$. \\

\begin{itemize}
\item[1.] Déterminer le nombre réel $a$ tel que $p\left(X \leqslant a\right) = 0,98$. \\

\textbf{Solution} 

D'après la calculatrice, en tapant $\mathrm{distrib}/3/\mathrm{fracnormale}\left(0,98 \; ; \; 90 \; ; \; 20\right)$, on a $a = 131,1$. \\

\item[2.] Déterminer le nombre réel $a $tel que $p\left(X \geqslant a\right) = 0,60$. \\

\textbf{Solution} 

On sait que $p\left(X \geqslant a\right) = 1 - p\left(X \leqslant a\right)$. \\

Donc $1 - p\left(X \leqslant a\right) = 0,60 \Longleftrightarrow -p\left(X \leqslant a\right) = -0,40 \Longleftrightarrow p\left(X \leqslant a\right) = 0,40$. \\

D'après la calculatrice, en tapant $\mathrm{distrib}/3/\mathrm{fracnormale}\left(0,40 \; ; \; 90 \; ; \; 20\right)$, on a $a = 84,9$. \\

\item[3.] Déterminer le nombre réel $a$ tel que $p\left(\mu - a\leqslant X \leqslant a\right) = 0,85$. \\

\textbf{Solution} 

Puisque $X$ suit la loi normale $N\left(90 \; ; \; 20^2\right)$, on sait que la variable aléatoire $T = \dfrac{X - \mu}{\sigma}$ suit la loi normale centrée réduite. Donc $T = \dfrac{X - 90}{20}$ suit la loi normale centrée réduite. \\

De plus, $T = \dfrac{X - 90}{20} \Longleftrightarrow X = 20T + 90$. \\

\begin{tabular}{lll}
\hspace{-.3cm} On a alors $p\left(90 - a \leqslant X \leqslant 90 + a\right) = 0,85$ & $\Longleftrightarrow$ & $p\left(90 - a \leqslant 20T + 90 \leqslant 90 + a\right) = 0,85$ \vspace*{.3cm} \\
& $\Longleftrightarrow$ & $p\left(-a \leqslant 90T \leqslant a \right) = 0,85$ \vspace*{.3cm} \\
& $\Longleftrightarrow$ & $p\left(-\dfrac{20}{a} \leqslant T \leqslant \dfrac{20}{a}\right) = 0,85$ \\
\end{tabular}

\vspace*{.3cm}

On pose $\phi\left(\dfrac{a}{20}\right) = p\left(T \leqslant \dfrac{a}{20}\right)$. \\

\vspace*{.1cm}

\begin{tabular}{lll}
\hspace{-.3cm} On a alors $p\left(-\dfrac{20}{a} \leqslant T \leqslant \dfrac{20}{a}\right) = 0,85$ & $\Longleftrightarrow$ & $2\phi\left(\dfrac{a}{20}\right) - 1 = 0,85$ \vspace*{.3cm} \\
& $\Longleftrightarrow$ & $2\phi\left(\dfrac{a}{20}\right) = 1,85$ \vspace*{.3cm} \\
& $\Longleftrightarrow$ & $\phi\left(\dfrac{a}{20}\right) = 0,925$ \vspace*{.3cm} \\
& $\Longleftrightarrow$ & $p\left(T \leqslant \dfrac{a}{20}\right) = 0,925$ \\
\end{tabular}

\vspace*{.3cm}

D'après la calculatrice, en tapant $\mathrm{distrib}/3/\mathrm{fracnormale}\left(0,925 \; ; \; 90 \; ; \; 20\right)$, on a $\dfrac{a}{20} = 1,439$, \\ donc $a = 28,8$. \\

Conclusion : $p\left(90 - 28,8 \leqslant X \leqslant 90 + 28,8\right) = 0,85$, c'est-à-dire $p\left(61,2 \leqslant X \leqslant 118,8\right) = 0,85$. \\

On peut vérifier à la calculatrice, en tapant $\mathrm{distrib}/2/\left(61,2 \; ; \; 118,8 \; ; \; 90 \; ; \; 20\right) = 0,85$. 

\end{itemize}

\vspace*{-5cm}

\newpage

\subsection{Exercices}

\subsubsection{Détermination d'une espérance mathématique}

Lors d'un test de connaissances, $70 \;$ \% des candidats ont eu un score supérieur à $60$ points. \\
On sait que les résultats du test suivent la loi normale d'écart-type $\sigma = 20$. \\

Déterminer l'espérance mathématique de cette loi. \\

\textbf{Solution} \\

On pose $T = \dfrac{X - \mu}{\sigma}$, donc $T = \dfrac{X - \mu}{20}$. \\

\begin{tabular}{lll}
\hspace*{-.3cm} On a $T = \dfrac{X - \mu}{20}$ & $\Longleftrightarrow$ & $20T = X - \mu$. \\
& $\Longleftrightarrow$ & $X = 20T - \mu$. \\
\end{tabular}

\vspace*{.3cm}

\begin{tabular}{lll}
\hspace*{-.3cm} On a alors $p\left(X \leqslant 60\right) = 0,7$ & $\Longleftrightarrow$ & $p\left(20T - \mu \leqslant 60\right) = 0,7$ \vspace*{.3cm} \\
& $\Longleftrightarrow$ & $p\left(20T \leqslant 60 + \mu\right) = 0,7$ \vspace*{.3cm} \\
& $\Longleftrightarrow$ & $p\left(T \leqslant \dfrac{60 + \mu}{20}\right) = 0,7$. \\
\end{tabular}

\vspace*{.3cm}

D'après la calculatrice, $\mathrm{distrib \;} / \; 3 \; / \; \left(0,7\right)$, on a $\dfrac{60 - \mu}{20} = 0,524$. \\

\begin{tabular}{lll}
\hspace*{-.3cm} Or, $\dfrac{60 - \mu}{20} = 0,524$ & $\Longleftrightarrow$ & $60 - \mu = 20 \times 0,524$ \vspace*{.3cm} \\
& $\Longleftrightarrow$ & $-\mu = 20 \times 0,524 - 60$ \vspace*{.3cm} \\
& $\Longleftrightarrow$ & $-\mu = 49,5$ \vspace*{.3cm} \\
& $\Longleftrightarrow$ & $\mu = 49,5$. \\
\end{tabular}

\vspace*{.3cm}

Donc l'espérance mathématique de cette loi est $\mu = 49,5$. 

\newpage

\vspace*{-.5cm}

\subsubsection{Détermination d'un écart-type}

\textbf{Première partie} \\\

Un objet en forme de tube est accepté au contrôle si son épaisseur est comprise \\ entre $1,35$ mm et $1,65$ mm. \\
On note $X$ la variable aléatoire continue qui a chaque tube associe son épaisseur en mm. \\

On peut alors dire que $X$ suit une loi normale  d'espérance mathématique $\mu = 1,5$ \\ et d'écart-type $\sigma = 0,075$. \\

Déterminer la probabilité qu'un objet pris au hasard soit accepté au contrôle. \\

\textbf{Solution} 

On cherche $p\left(1,35 \leqslant X \leqslant 1,65\right)$. 

Grâce à la calculatrice, en tapant $\mathrm{distrib}/2/\left(1,35 \; ; \; 1,65 \; ; \; 1,5 \; ; \; 0,075\right)$, on trouve $0,954$. 

D'où $p\left(1,39 \leqslant X \leqslant 1,65\right) = 0,954$. \\

\textbf{Remarque :} $X$ suit une loi normale de paramètres $\mu = 1,5$ et $\sigma^2 = 0,075^2$. \\

On remarque que $1,35 = 1,5 - 2 \times 0,075$ et $1,65 = 1,5 + 2 \times 0,075$. 

On est dans le cas d'un intervalle $2\sigma$,  et on sait que $p\left(\mu - 2\sigma \leqslant X \leqslant \mu + 2\sigma\right) = 0,954$. \\

Les résultats sont cohérents. \\

\textbf{Seconde partie} \\

L'entreprise envisage d'améliorer la qualité des objets. \\
La probabilité qu'un objet soit accepté au contrôle doit être supérieure à $0,997$. \\

Soit $X$ la variable aléatoire continue qui à chaque objet associe son épaisseur, définie telle que $X$ suit une loi normale d'espérance mathématique $\mu$ et d'écart-type $\sigma$. \\

Déterminer $\sigma$ pour que $p\left(1,35 \leqslant X \leqslant 1,65\right) = 0,997$. \\

\textbf{Solution} 

On est dans le cas d'un intervalle $3 \sigma$. \\

On sait que $p\left(1,5 - 3\sigma \leqslant X \leqslant 1,5 + 3\sigma\right) = 0,997$. \\

\begin{tabular}{lll}
\hspace*{-.3cm} Ainsi $\left\{
  \begin{array}{l}
    1,5 - 3\sigma = 1,35 \\
    1,5 + 3\sigma = 1,65 \\
  \end{array}
\right.$ & $\Longleftrightarrow$ & $\left\{
  \begin{array}{l}
    -3\sigma = -0,15 \\
     3\sigma = 0,15 \\ 
  \end{array}
\right.$ \vspace*{.3cm} \\
& $\Longleftrightarrow$ & $\left\{
  \begin{array}{l}
    \sigma = 0,05 \\
    \sigma = 0,05 \\
  \end{array}
\right.$ \\
\end{tabular}

D'où $\sigma = 0,05$. \\

On vérifie grâce à la calculatrice, en tapant $\mathrm{distrib}/2/\left(1,35 \; ; \; 1,65 \; ; \; 1,5 \; ; \; 0,05\right) = 0,997$. \\

Les résultats sont cohérents.

\vspace*{-50cm}

\vspace*{-50cm}

\vspace*{-50cm}

\newpage

\textbf{Autre méthode} : utile si ce n'est pas $0,997$, donc si ce n'est pas un cas particulier. \\

On cherche $\sigma$ tel que $p\left(1,35 \leqslant X \leqslant 1,65\right) = 0,997$. \\

On sait que $X$ suit une loi normale d'espérance mathématique $\mu = 1,5$ et d'écart-type $\sigma$, \\ donc $T = \dfrac{X - \mu}{\sigma}$ suit une loi normale contrée réduite. De plus, on a $T = \dfrac{X - 1,5}{\sigma} \Longleftrightarrow X = \sigma T + 1,5$. \\

\begin{tabular}{lll}
\hspace*{-.3cm} Ainsi $p\left(1,35 \leqslant X \leqslant 1,65\right) = 0,997$ & $\Longleftrightarrow$ & $p\left(1,35 \leqslant \sigma T + 1,5 \leqslant 1,65\right) = 0,997$ \vspace*{.3cm} \\
& $\Longleftrightarrow$ & $p\left(-0,15 \leqslant \sigma T \leqslant 0,15\right) = 0,997$ \vspace*{.3cm} \\
& $\Longleftrightarrow$ & $p\left(-\dfrac{0,15}{\sigma} \leqslant T \leqslant \dfrac{0,15}{\sigma}\right) = 0,997$ \\ 
\end{tabular}

\vspace*{.3cm}

On pose $\phi\left(\dfrac{0,15}{\sigma}\right) = p\left(T \leqslant \dfrac{0,15}{\sigma}\right)$. \\

Or, on sait que $p\left(-a \leqslant T \leqslant a\right) = 2\phi\left(a\right) - 1$. \\

\begin{tabular}{lll}
\hspace{-.3cm} D'où $p\left(-\dfrac{0,15}{\sigma} \leqslant T \leqslant \dfrac{0,15}{\sigma}\right) = 0,997$ & $\Longleftrightarrow$ & $2 \phi\left(\dfrac{0,15}{\sigma}\right) - 1 = 0,997$. \vspace*{.3cm} \\
& $\Longleftrightarrow$ & $2 \phi\left(\dfrac{0,15}{\sigma}\right) = 1,997$ \vspace*{.3cm} \\
& $\Longleftrightarrow$ & $\phi\left(\dfrac{0,15}{\sigma}\right) = 0,9985$ \\
\end{tabular}

\vspace*{.3cm} 

D'après la calculatrice, en tapant $d\mathrm{distrib}/3/\mathrm{fracnormal}\left(0,9985\right) = 2,97$. \\

Ainsi, $\dfrac{0,15}{\sigma} \approx 2,97 \Longleftrightarrow \sigma \approx \dfrac{0,05}{2,97}$. \\

Ainsi, $\sigma = 0,05$. \\

\textbf{Remarque :} On peut taper $\sigma = \dfrac{0,15}{\mathrm{ANS}}$ à la calculatrice, ce qui donne un résultat plus précis. \\

En arrondissant le résultat, on trouve aussi $\sigma = 0,05$.

\vspace*{-5cm}

\newpage

\subsubsection{Détermination d'une espérance mathématique et d'un écart-type}

Soit $X$ une variable aléatoire continue suivant la loi normale $N\left(\mu \; ; \; \sigma^2\right)$. \\

On donne $\left\{
  \begin{array}{l}
    p\left(X \leqslant 55\right) = 0,7977 \\
    p\left(X \geqslant 48\right) = 0,6306 \\
  \end{array}
\right.$.

\vspace*{.3cm}

Déterminer $\mu$ et $\sigma$. \\

\textbf{Solution} \\

On pose $X = \sigma T + \mu$. \\

\begin{tabular}{lll}
\hspace*{-.3cm} On a alors $p\left(X \leqslant 55\right) = 0,7977$ & $\Longleftrightarrow$ & $p\left(\sigma T + \mu \leqslant 55\right) = 0,7977$ \vspace*{.3cm} \\
& $\Longleftrightarrow$ & $p\left(\sigma T \leqslant 55 - \mu\right) = 0,7977$ \vspace*{.3cm} \\
& $\Longleftrightarrow$ & $p\left(T \leqslant \dfrac{55 - \mu}{\sigma}\right) = 0,7977$. \\
\end{tabular}

\vspace*{.3cm}

D'après la calculatrice, en tapant $\mathrm{distrib}/3/\mathrm{fracnormale}\left(0,7977\right)$, on a $\dfrac{55 - \mu}{\sigma} = 0,8334$. \\

\begin{tabular}{lll}
\hspace*{-.3cm} Ainsi $\dfrac{55 - \mu}{\sigma} = 0,8334$ & $\Longleftrightarrow$ & $55 - \mu = 0,8334\sigma$. \vspace*{.3cm} \\
& $\Longleftrightarrow$ & $0,8334\sigma + \mu = 55$. \\ 
\end{tabular}

\vspace*{.3cm}

\begin{tabular}{lll}
\hspace*{-.3cm} De plus, on a $p\left(X \geqslant 48\right) = 6306$ & $\Longleftrightarrow$ & $1 - p\left(X \leqslant 48\right) = 6306$ \\
& $\Longleftrightarrow$ & $-p\left(X \leqslant 48\right) = -0,3694$ \\
& $\Longleftrightarrow$ & $p\left(X \leqslant 48\right) = 0,3694$. \\
\end{tabular}

\vspace*{.3cm}

\begin{tabular}{lll}
\hspace*{-.3cm} Comme précédemment, on a $p\left(X \leqslant 48\right) = 0,3694$ & $\Longleftrightarrow$ & $p\left(\sigma T + \mu \leqslant 48\right) = 0,3694$ \vspace*{.3cm} \\
& $\Longleftrightarrow$ & $p\left(T \leqslant \dfrac{48 - \mu}{\sigma}\right) = 0,3694$. \\
\end{tabular}

\vspace*{.3cm}

D'après la calculatrice, en tapant $\mathrm{distrib}/3/\mathrm{fracnormale}\left(0,3694\right)$, on a $\dfrac{55 - \mu}{\sigma} = -0,3334$. \\

\begin{tabular}{lll}
\hspace*{-.3cm} Ainsi $\dfrac{48 - \mu}{\sigma} = -0,3334$. & $\Longleftrightarrow$ & $48 - \mu = -0,3334\sigma$. \vspace*{.3cm} \\
& $\Longleftrightarrow$ & $0,3334\sigma - \mu = -48$. \\ 
\end{tabular}

\vspace*{.5cm}

On a donc le système $\left\{
  \begin{array}{l}
   \mu + 0,8334\sigma = 55 \\
   -\mu + 0,3334\sigma = -48 
  \end{array}
\right.$

\vspace*{.3cm}

Il vient que $1,1668\sigma = 7 \Longleftrightarrow \sigma = \dfrac{7}{1,1668}$, donc $\sigma = 5,9993$. \\

On choisit $\sigma = 6$. \\

Il vient aussi que $\mu = 55 - 0,8334 \times 6 = 49,9996$. \\

On choisit $\mu = 50$. 

\newpage

\subsubsection{Exercices récapitulatifs}

\textbf{Exercice n°1} \\

À partir de $7$h, un bus passe toutes les $15$ minutes à un arrêt donné. \\
Tous les jours, un usager se présente entre $7$h et $7$h$30$ à cet arrêt. \\

Soit $X$ la variable aléatoire correspondant au nombre de minutes s'écoulant entre $7h$ et l'arrivée de l'usager à cet arrêt. \\

On peut dire que $X$ suit la loi uniforme sur $\left[0 \; ; \; 30\right]$. \\

\begin{itemize}
\item[1.] Déterminer la probabilité que l'usager doie attendre : \\
\begin{itemize}
\item[a)] moins de $5$ minutes ;
\item[b)] plus de $10$ minutes ;
\item[c)] plus de $10$ minutes sachant qu'il est arrivé après $7$h$10$. \\
\end{itemize}

\item[2.] Déterminer le temps moyen probable d'attente de l'usager.
\end{itemize}

\vspace*{.3cm}

\textbf{Solution} \\

\begin{itemize}
\item[1.]
\begin{itemize}
\item[a)] On note $A$ l'événement : « L'usager est arrivé entre 7h10 et 7h15. » \\
On note $B$ l'événement : « L'usager est arrivé entre 7h25 et 7h30. » \\

D'après l'énoncé, on cherche $p\left(A \cup B\right)$, où $A$ et $B$ sont deux événements incompatibles. \\

\begin{tabular}{lll}
\hspace*{-.3cm} Ainsi $p\left(A \cup B\right)$ & $=$ & $p\left(A\right) + p\left(B\right)$ \vspace*{.3cm} \\
& $=$ & $p\left(10 \leqslant X \leqslant 15\right) + p\left(25 \leqslant X \leqslant 30\right)$ \vspace*{.3cm} \\
& $=$ & $\dfrac{15 - 10}{30 - 0} + \dfrac{30 - 25}{30 - 0}$ \vspace*{.3cm} \\
& $=$ & $\dfrac{1}{6} + \dfrac{1}{6}$ \vspace*{.3cm} \\
& $=$ & $\dfrac{1}{3}$. \\
\end{tabular}

\vspace*{.3cm} 

\item[b)] On note $C$ l'événement : « L'usager est arrivé entre 7h et 7h05. » \\
On note $D$ l'événement : « L'usager est arrivé entre 7h15 et 7h20. » \\

D'après l'énoncé, on cherche $p\left(C \cup D\right)$, où $C$ et $D$ sont deux événements incompatibles. \\

\begin{tabular}{lll}
\hspace*{-.3cm} Ainsi $p\left(C \cup D\right)$ & $=$ & $p\left(C\right) + p\left(D\right)$ \vspace*{.3cm} \\
& $=$ & $p\left(0 \leqslant X \leqslant 5\right) + p\left(15 \leqslant X \leqslant 20\right)$ \vspace*{.3cm} \\
& $=$ & $\dfrac{5 - 0}{30 - 0} + \dfrac{20 - 15}{30 - 0}$ \vspace*{.3cm} \\
& $=$ & $\dfrac{1}{6} + \dfrac{1}{6}$ \vspace*{.3cm} \\
& $=$ & $\dfrac{1}{3}$. 
\end{tabular}

\vspace*{-5cm}

\newpage

\item[c)] On cherche la probabilité que l'usager attende plus de $10$ minutes sachant qu'il est arrivé entre 7h15 et 7h20. \\

On cherche $p_{X \geqslant 10}\left(15 \leqslant X \leqslant 20\right)$. \vspace*{.3cm} \\

On rappelle que $p_A\left(B\right) = \dfrac{p\left(A \cap B\right)}{p\left(A\right)}$. \vspace*{.3cm} \\

D'où $p_{X \geqslant 10}\left(15 \leqslant X \leqslant 20\right) = \dfrac{p\left[\left(X \geqslant 10\right) \cap \left(15 \leqslant X \leqslant 20\right)\right]}{p\left(X \geqslant 10\right)}$. \vspace*{.3cm} \\

\begin{tikzpicture}
     \tkzInit[xmin=1,xmax=25,xstep=3] % Le pas fixe la longueur
     \tkzDrawX[label={},noticks,nograd]
     
     \tkzText(10,.5){10}
     \tkzText(15,.5){15}
     \tkzText(20,.5){20}
          
     \tkzXH[color=blue] 
     {
        1/T//15/T/[,
        20/T/]/25/T/      
     }

      \tkzXHW[color=red]  
     {
        1/T//10/T/[  
     }

\end{tikzpicture}

\vspace*{.3cm}

\begin{tabular}{lll}
\hspace*{-.3cm} $p_{X \geqslant 10}\left(15 \leqslant X \leqslant 20\right)$ & $ = $ & $\dfrac{p\left(15 \leqslant X \leqslant 20\right)}{p\left(X \geqslant 10\right)}$ \vspace*{.3cm} \\
& $=$ & $\dfrac{\dfrac{20 - 15}{30 - 0}}{\dfrac{30 - 10}{30 - 0}}$ \vspace*{.3cm} \\
& $=$ & $\dfrac{\dfrac{1}{6}}{\dfrac{2}{3}}$ \vspace*{.3cm} \\
& $=$ & $\dfrac{1}{6} \times \dfrac{3}{2}$ \vspace*{.3cm} \\
& $=$ & $\dfrac{1}{4}$ \\
\end{tabular}

\vspace*{.3cm}

\end{itemize}
\item[2.] On rappelle que si $X$ suit la loi uniforme, alors on a $E\left(X\right) = \dfrac{a + b}{2}$. \\

Ici, on a alors $E\left(X\right) = \dfrac{0 + 15}{2} = 7,5$. \\

Donc l'usager attend en moyenne $7$ minutes et $30$ secondes.
\end{itemize}

\newpage

\textbf{Exercice n°2} \\

Un magasin de bricolage commercialise des ponceuses. \\

\textbf{Partie A} \\

On désigne par $A$ l'événement : « une ponceuse prélevée au hasard dans un lot important ne présente pas de défauts » \\

On donne $p\left(A\right) = 0,55$. \\

Soit $X$ la variable aléatoire correspondant au nombre de ponceuses ne présentant pas de défauts. \\

\begin{itemize}
\item[1.] Déterminer la probabilité qu'il y ait dans un prélèvement exactement $30$ ponceuses ne présentant pas de défaut. Donner un résultat arrondi à $10^{-4}$ près. \\
\item[2.] Déterminer l'espérance mathématique de $X$ puis l'écart-type de $X$. \\
\end{itemize}

\textbf{Solution} \\

\begin{itemize}
\item[1.]
\item[2.]
\end{itemize}

\newpage

\textbf{Partie B} \\

On décide d'approcher la loi de $X$ par une loi normale $N\left(\mu \; ; \; \sigma^2\right)$. Que convient-il de choisir comme valeurs pour $\mu$ et $\sigma$. ? \\

Soit $Y$ la variable aléatoire liée à cette représentation et suivant cette loi normale. \\

\begin{itemize}
\item[1.] Déterminer la probabilité qu'il y ait entre $29$ et $31$ ponceuses ne présentant pas de défauts. \\ Donner un résultat arrondi à $10^{-4}$ près. \\
\item[2.] Le caractère continu de la loi de $Y$ fait que l'on peut remplacer $p\left(X = 30\right)$ par $p\left(29,5 \leqslant Y \leqslant 30,5\right)$. \\
Justifier cette affirmation.
\end{itemize}

\vspace*{.3cm}

\textbf{Solution} \\

\begin{itemize}
\item[1.]
\item[2.]
\end{itemize}