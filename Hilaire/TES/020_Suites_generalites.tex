
\ifdefined\COMPLETE
\else
    \input{./preambule-sacha-utf8.ltx}
    \begin{document}
\fi



%\setcounter{section}{0}

\section{Généralités sur les suites}

\subsection{Définition}

On appelle « suite numérique » une \textbf{fonction} de $\N$ (ou d'une partie de $\N$) dans $\R$ : \\

\begin{tabular}{lllll}
Soit une fonction $f$ : & $\N$ & $\longrightarrow$ & $\R$ \\
& $n$ & $\longrightarrow$ & $f\left(n\right)$ \\ 
\end{tabular}

\vspace*{.3cm}

Tout élément de $\N$ (ou d'une partie de $\N$) a au plus une image dans $\R$.

\textbf{Remarque :} au plus une image : soit une image, soit aucune. 

\subsection{Notation}

\begin{itemize}
\item[•] La suite $f$ est notée $\left(u_n\right)_{n \in \N}$ ou plus simplement $\left(u_n\right)$. \\

\item[•] Le nombre réel $f\left(n\right)$ est noté $u_n$ et est un \textbf{terme général} de la suite $\left(u_n\right)_{n \in \N}$ \\

\textbf{Attention :} ce n'est pas nécessairement le $n^\mathrm{ième}$ terme de la suite. 
\end{itemize}

\subsection{Les différents types de suite}

\subsubsection{Suite définie par son terme général}

Soit $\left(u_n\right)_{n \in \N}$ la suite définie par $\forall n \in \N, u_n = 2n - 3$ \\

Calculer les termes suivants : \\

\begin{itemize}
\item[•] $u_0 = -3$ ; c'est le premier terme de la suite.
\item[•] $u_1 = -1$ ; c'est le deuxième terme de la suite.
\item[•] $u_2 = 1$ ; c'est le troisième terme de la suite.
\item[•] $u_3 = 3$ ; c'est le quatrième terme de la suite.
\item[•] $u_{1000} = 1997$ ; c'est le mille-et-unième terme de la suite.
\end{itemize}

\vspace*{.3cm}

\textbf{Attention : } $u_{n+1} \neq u_n + 1$ \\

En effet, $u_{n+1} = 2\left(n+1\right) -3 = 2n + 2 - 3 = 2n - 1 $ \\

$u_n + 1 = \left(2n - 3\right) + 1 = 2n - 2$

\newpage

\subsubsection{Suite définie par une relation de récurrence}

Soit $\left(u_n\right)_{n \in \N}$ la suite définie par :$ \; \; \; \begin{cases}
u_0 = 2 \\
\forall n \in \N, u_{n+1} = u_n + 3 \\
\end{cases}$ \\

Calculer les termes suivants : 

\begin{itemize}
\item[•] $u_0 = 2$
\item[•] $ u_1 = u_0 + 3 = 2 + 3 = 5$ 
\item[•] $ u_2 = u_1 + 3 = 5 + 3 = 8$ 
\item[•] $ u_3 = u_2 + 3 = 8 + 3 = 11$
\end{itemize}

\vspace*{.3cm}

On conjecture que : $\forall n \in \N, u_n = 3n + 2$

\subsection{Notion de raisonnement par récurrence}

On cherche à montrer \hbox{que la formule conjecturée, appelée \textbf{hypothèse de récurrence} est vraie pour tout $n$.}

\subsubsection{Montrer que la formule conjecturée est vraie pour la première valeur de $\mathbf{n}$}

$u_0 = 3 \times 0 + 2 = 2$. L'initialisation du raisonnement par récurrence est donc vérifiée. 

\subsubsection{On démontre que, si la formule conjecturée est vraie pour le terme de rang $\mathbf{n}$, alors elle est vraie pour le terme de rang $\mathbf{n + 1}$.}

Ainsi, on cherche à montrer que $u_{n+1} = 3\left(n+1\right) + 2 = 3n + 5$. \\

L'hypothèse de récurrence est : $u_n = 3n + 2$. On la suppose vraie. \\

\begin{tabular}{lll}
$u_{n+1}$ & $=$ & $u_n + 3$ \\
& $=$ & $\left(3n + 2\right) + 3$ \\
& $=$ & $3n + 5$ \\ 
\end{tabular}

\vspace*{.3cm}

Donc l'hypothèse est vérifiée, et la suite peut se définir par l'expression de son terme général  : \\ $\forall n \in \N, u_n = 3n + 2$.

\newpage

\vspace*{-2cm}

\subsection{Exercices}

\subsubsection{Exercice \no 1}

Soit $\left(u_n\right)_{n \in \N^*}$ la suite définie par :$ \; \; \; \begin{cases}
u_1 = 0 \\
\forall n \in \N^*, u_{n+1} = \dfrac{1}{2 - u_n} \\
\end{cases}$ \\

Calculer les termes suivants : \\

\begin{itemize}
\item[•] $u_2 = \dfrac{1}{2 - u_1} = \dfrac{1}{2 - 0} = \dfrac{1}{2}$ \vspace*{.3cm} \\
\item[•] $ u_3 = \dfrac{1}{2 - u_2} = \dfrac{1}{2 - \dfrac{1}{2}} = \dfrac{1}{\dfrac{3}{2}} = \dfrac{2}{3}$ \vspace*{.3cm} \\
\item[•] $u_4 = \dfrac{1}{2 - u_3} = \dfrac{1}{2 - \dfrac{2}{3}} = \dfrac{1}{\dfrac{4}{3}} = \dfrac{3}{4}$ \vspace*{.3cm} \\
\item[•] $u_5 = \dfrac{1}{2 - u_4} = \dfrac{1}{2 - \dfrac{3}{4}} = \dfrac{1}{\dfrac{5}{4}} = \dfrac{4}{5}$ \vspace*{.3cm} \\
\end{itemize}

\vspace*{.3cm}

On conjecture que : $\forall n \in \N^*, u_n = \dfrac{n - 1}{n}$ \\

On vérifie que la formule conjecturée est vraie pour la première valeur de $n$, c'est-à-dire $n = 1$ : $u_1 = \dfrac{1 - 1}{1} = 0$ \\

On démontre que si la formule conjecturée est vraie pour le terme de rang $n$, alors elle l'est aussi pour le terme de rang $n + 1$ : \\

Si $u_n = \dfrac{n-1}{n}$, alors $ u_{n+1} = \dfrac{\left(n+1\right) - 1}{n + 1} = \dfrac{n}{n+1}$ \\

D'après l'énoncé, on a : $u_{n+1} = \dfrac{1}{2 - u_n}$. \\


\begin{tabular}{lll}
D'où $u_{n+1}$ & $ = $ & $ \dfrac{1}{2 - \dfrac{n-1}{n}}$ \vspace*{.3cm}  \\
& $ = $ & $\dfrac{1}{\dfrac{2n - \left(n-1\right)}{n}}$ \vspace*{.3cm} \\
& $=$ & $\dfrac{1}{\dfrac{2n - n + 1}{n}}$ \vspace*{.3cm} \\
& $=$ & $\dfrac{1}{\dfrac{n+1}{n}}$ \vspace*{.3cm} \\
& $=$ & $\dfrac{n}{n+1}$ \vspace*{.3cm} \\
\end{tabular}

\vspace*{.3cm}

L'hypothèse \hbox{est vérifiée, et la suite peut se caractériser par son terme général $u_n = \dfrac{n-1}{n}$ pour tout $n \in \N^*$.}

\newpage
 
\subsubsection{Amusette}

Soit $\left(u_n\right)_{n \in \N}$ la suite définie par :$ \; \; \; \begin{cases}
u_0 = 1 \\
\forall n \in \N, u_{n+1} = 10u_n + 1 - 9n \\
\end{cases}$ \\

\begin{itemize}
\item[1.] Déterminer $u_1$, $u_2$, $u_3$, $u_4$. \\
\item[2.] Conjecturer le terme général de la suite $\left(u_n\right)_{n \in \N}$. \\
\item[3.] Démontrer la conjecture. \\
\end{itemize}

\begin{itemize}
\item[•] $u_1 = 10u_0 + 1 - 9 \times 0 = 10 \times 1 + 1 = 11$ \\
\item[•] $u_2 = 10u_1 + 1 - 9 \times 1 = 10 \times 11 + 1 - 9 = 111 - 9 = 102$ \\
\item[•] $u_3 = 10u_2 + 1 - 9 \times 2 = 10 \times 102 + 1 - 18 = 1020 + 1 - 18 = 1021 - 18 = 1003 $ \\
\item[•] $u_4 = 10u_3 + 1 - 9 \times 3 = 10 \times 1003 + 1 - 27 = 10031 - 27 = 10004$ \\
\end{itemize}

2) On conjecture que $u_n = 10^n + n$. \\

3) On vérifie que la valeur est vraie pour la première valeur de $n$, c'est-à-dire $n = 0$ : \\

$u_0 = 10^0 + 0 = 1 + 0 = 1 $ \\

On démontre que la formule conjecturée est vraie pour $n + 1$, c'est-à-dire si $u_n = 10^n + n$, alors $u_{n+1} = 10^{n+1} + \left(n+1\right)$ \\

\begin{tabular}{lll}
$u_{n+1}$ & $=$ & $10u_n + 1 - 9n$ \\
& $=$ & $10\left(10^n + n\right) + 1 - 9n$ \\
& $=$ & $10^{n+1} + 10n + 1 - 9n $ \\
& $=$ & $10^{n+1} + n + 1$ \\
\end{tabular}

\subsubsection{À la calculatrice...}

\textbf{Mode d'emploi : }

\begin{itemize}
\item[•] Mode : Normal
\item[•] Suite
\item[•] Non-relié
\end{itemize}

\vspace*{.3cm}

On peut étudier les premiers termes des suites suivantes, en appuyant sur 2nde, puis Table. \\

\begin{tabular}{lll}

\hspace*{-2cm}

Soit $\left(u_n\right)_{n \in \N}$ la suite définie par &

\hspace*{-.6cm}

Soit $\left(u_n\right)_{n \in \N^*}$ la suite définie par &

\hspace*{-.3cm}

Soit $\left(u_n\right)_{n \in \N}$ la suite définie par \\

\hspace*{-2cm}

définie par :$ \; \; \; \begin{cases}
u_0 = 2 \\
\forall n \in \N, u_{n+1} = u_n + 3 \\
\end{cases}$ 

&

\hspace*{-.6cm}

définie par :$ \; \; \; \begin{cases}
u_1 = 0 \\
\forall n \in \N, u_{n+1} = \dfrac{1}{2 - u_n} \\
\end{cases}$ 

&

\hspace*{-.3cm}

définie par :$ \; \; \; \begin{cases}
u_0 = 1 \\
\forall n \in \N, u_{n+1} = 10u_n -9n + 1 \\
\end{cases}$ 

\vspace*{.5cm}

\\

\hspace*{-2cm}

On a :

$n_{min} = 0$ & 

\hspace*{-.6cm}

On a :

$n_{min} = 1$ (car $n \; \in \; \N^*$)

& 

\hspace*{-.3cm}

On a :

$n_{min} = 0$ \\

\hspace*{-1cm}

$u_n = u_{n-1} +3$ & 

\hspace*{.4cm}

$u_n = \dfrac{1}{2 - u_{n-1}}$ & 

\hspace*{.7cm}

$u_n = 10u_{n-1} - 9\left(n-1\right) + 1$  \\

\hspace*{-1cm}

$u_{n_{min}} = 2$ & 

\hspace*{.4cm}

$u_{n_{min}} = 0$ & 

\hspace*{.7cm}

$u_{n_{min}} = 1$ \\

\end{tabular}

\newpage

\subsection{Représentations graphiques de suites}

\subsubsection{Suite définie par son terme général}

Soit la suite $\left(u_n\right)_{n \; \in \; \N}$ définie par $u_n = \dfrac{2}{3}n + 1$. \\

\definecolor{yqyqyq}{rgb}{0.5,0.5,0.5}

\definecolor{xdxdff}{rgb}{0.49,0.49,1}

\definecolor{ffqqqq}{rgb}{1,0,0}

\definecolor{cqcqcq}{rgb}{0.75,0.75,0.75}


\begin{tikzpicture}[line cap=round,line join=round,>=triangle 45,x=1cm,y=1cm,scale=1.2]

\draw[->] (-0.7,0) -- (12.75,0);
\foreach \x in {,1,2,3,4,5,6,7,8,9,10,11,12}
\draw[shift={(\x,0)}] (0pt,2pt) -- (0pt,-2pt) node[below] {\footnotesize $\x$};
\draw[->] (0,-0.7) -- (0,10.5);
\draw[color=black] (0pt,-8pt) node[left] {\footnotesize $0$};
\clip(-0.7,-0.7) rectangle (12.75,9.5);
\draw[dotted,color=cqcqcq, smooth,samples=100,domain=0.0:12.75] plot(\x,{(2*(\x))/3+1});
\draw [dash pattern=on 3pt off 3pt,color=yqyqyq] (1,0)-- (1,1.66);
\draw [dash pattern=on 3pt off 3pt,color=yqyqyq] (1,1.66)-- (0,1.66);
\draw [dash pattern=on 3pt off 3pt,color=yqyqyq] (2,0)-- (2,2.33);
\draw [dash pattern=on 3pt off 3pt,color=yqyqyq] (2,2.33)-- (0,2.33);
\draw [dash pattern=on 3pt off 3pt,color=yqyqyq] (3,0)-- (3,3);
\draw [dash pattern=on 3pt off 3pt,color=yqyqyq] (3,3)-- (0,3);
\draw [dash pattern=on 3pt off 3pt,color=yqyqyq] (4,0)-- (4,3.66);
\draw [dash pattern=on 3pt off 3pt,color=yqyqyq] (4,3.66)-- (0,3.66);
\draw [dash pattern=on 3pt off 3pt,color=yqyqyq] (5,0)-- (5,4.33);
\draw [dash pattern=on 3pt off 3pt,color=yqyqyq] (5,4.33)-- (0,4.33);
\draw [dash pattern=on 3pt off 3pt,color=yqyqyq] (6,0)-- (6,5);
\draw [dash pattern=on 3pt off 3pt,color=yqyqyq] (6,5)-- (0,5);
\draw [dash pattern=on 3pt off 3pt,color=yqyqyq] (7,0)-- (7,5.66);
\draw [dash pattern=on 3pt off 3pt,color=yqyqyq] (7,5.66)-- (0,5.66);
\draw [dash pattern=on 3pt off 3pt,color=yqyqyq] (8,0)-- (8,6.33);
\draw [dash pattern=on 3pt off 3pt,color=yqyqyq] (8,6.33)-- (0,6.33);
\draw [dash pattern=on 3pt off 3pt,color=yqyqyq] (10,0)-- (10,7.66);
\draw [dash pattern=on 3pt off 3pt,color=yqyqyq] (10,7.66)-- (0,7.66);
\draw [dash pattern=on 3pt off 3pt,color=yqyqyq] (9,0)-- (9,7);
\draw [dash pattern=on 3pt off 3pt,color=yqyqyq] (9,7)-- (0,7);
\draw [dash pattern=on 3pt off 3pt,color=yqyqyq] (11,0)-- (11,8.33);
\draw [dash pattern=on 3pt off 3pt,color=yqyqyq] (11,8.33)-- (0,8.33);
\draw [dash pattern=on 3pt off 3pt,color=yqyqyq] (12,0)-- (12,9);

\draw [dash pattern=on 3pt off 3pt,color=yqyqyq] (12,9)-- (0,9);

\draw [dash pattern=on 3pt off 3pt,color=yqyqyq] (13,0)-- (13,9.66);

\draw [dash pattern=on 3pt off 3pt,color=yqyqyq] (13,9.66)-- (0,9.66);


\draw [color=ffqqqq] (0,1)-- ++(-1.0pt,-1.0pt) -- ++(2.0pt,2.0pt) ++(-2.0pt,0) -- ++(2.0pt,-2.0pt);

\draw [color=ffqqqq] (0.98,1.65)-- ++(-1.0pt,-1.0pt) -- ++(2.0pt,2.0pt) ++(-2.0pt,0) -- ++(2.0pt,-2.0pt);

\draw [color=ffqqqq] (2,2.33)-- ++(-1.0pt,-1.0pt) -- ++(2.0pt,2.0pt) ++(-2.0pt,0) -- ++(2.0pt,-2.0pt);

\draw [color=ffqqqq] (3,3)-- ++(-1.0pt,-1.0pt) -- ++(2.0pt,2.0pt) ++(-2.0pt,0) -- ++(2.0pt,-2.0pt);

\draw [color=ffqqqq] (4.02,3.68)-- ++(-1.0pt,-1.0pt) -- ++(2.0pt,2.0pt) ++(-2.0pt,0) -- ++(2.0pt,-2.0pt);

\draw [color=ffqqqq] (5.02,4.35)-- ++(-1.0pt,-1.0pt) -- ++(2.0pt,2.0pt) ++(-2.0pt,0) -- ++(2.0pt,-2.0pt);

\draw [color=ffqqqq] (6,5)-- ++(-1.0pt,-1.0pt) -- ++(2.0pt,2.0pt) ++(-2.0pt,0) -- ++(2.0pt,-2.0pt);

\draw [color=ffqqqq] (7.02,5.68)-- ++(-1.0pt,-1.0pt) -- ++(2.0pt,2.0pt) ++(-2.0pt,0) -- ++(2.0pt,-2.0pt);

\draw [color=ffqqqq] (8.04,6.36)-- ++(-1.0pt,-1.0pt) -- ++(2.0pt,2.0pt) ++(-2.0pt,0) -- ++(2.0pt,-2.0pt);

\draw [color=ffqqqq] (9,7)-- ++(-1.0pt,-1.0pt) -- ++(2.0pt,2.0pt) ++(-2.0pt,0) -- ++(2.0pt,-2.0pt);

\draw [color=ffqqqq] (10.04,7.69)-- ++(-1.0pt,-1.0pt) -- ++(2.0pt,2.0pt) ++(-2.0pt,0) -- ++(2.0pt,-2.0pt);

\draw [color=ffqqqq] (10.96,8.31)-- ++(-1.0pt,-1.0pt) -- ++(2.0pt,2.0pt) ++(-2.0pt,0) -- ++(2.0pt,-2.0pt);

\draw [color=ffqqqq] (12,9)-- ++(-1.0pt,-1.0pt) -- ++(2.0pt,2.0pt) ++(-2.0pt,0) -- ++(2.0pt,-2.0pt);

\draw [color=ffqqqq] (13,9.67)-- ++(-1.0pt,-1.0pt) -- ++(2.0pt,2.0pt) ++(-2.0pt,0) -- ++(2.0pt,-2.0pt);


% \fill [color=xdxdff] (0,1.66) circle (1.5pt);

\draw (0,1) node [left] {$u_0$};

\draw (0,1.66) node [left] {$u_1$};

\draw (0,2.33) node [left] {$u_2$};

\draw (0,3) node [left] {$u_3$};

\draw (0,3.66) node [left] {$u_4$};

\draw (0,4.33) node [left] {$u_5$};

\draw (0,5) node [left] {$u_6$};

\draw (0,5.66) node [left] {$u_7$};

\draw (0,6.33) node [left] {$u_8$};

\draw (0,7) node [left] {$u_9$};

\draw (0,7.66) node [left] {$u_{10}$};

\draw (0,8.33) node [left] {$u_{11}$};

\draw (0,9) node [left] {$u_{12}$};


\begin{pgfonlayer}{background}   


\draw[step=1mm,ultra thin,AntiqueWhite!10] (-0.7,-0.7) grid (12.75,10.5);

\draw[step=5mm,very thin,AntiqueWhite!30]  (-0.7,-0.7) grid (12.75,10.5);

\draw[step=1cm,very thin,AntiqueWhite!50](-0.7,-0.7) grid (12.75,10.5);

\draw[step=5cm,thin,AntiqueWhite]         (-0.7,-0.7) grid (12.75,10.5);


\end{pgfonlayer}

\end{tikzpicture}

\newpage

\subsubsection{Suite définie par une relation de récurrence}

\textbf{Exemple n°1} \\

Soit $\left(u_n\right)_{n \; \in \N}$ une suite définie par :$ \; \; \; \begin{cases}
u_0 = 11 \\
\forall n \in \N, u_{n+1} = \dfrac{3}{4}u_n + 1 \\
\end{cases}$ 

\vspace*{.3cm}

On appelle $\Delta$ la droite d'équation $y = \dfrac{3}{4}x + 1$. \\

On a $u_1 = \dfrac{3}{4}u_0 + 1$. \\

On trace également la droite d'équation $ y = x$.

Cette droite est la représentation graphique de la fonction identité, définie par $f(x) = x$. \\ Elle est appelée \textbf{la première bissectrice du plan}. \\

\begin{texgraph}[file,name=suite_02,export=tkz]
Include "papiers.mac";
Graph image = [
Fenetre(-.9+12*i, 12-.9*i, 1+i),Marges(0,0,0,0),
papier(milli,-1-i,12+12*i,
[subsubgridcolor :=beige,
subgridcolor:=antiquewhite,
gridcolor :=bisque]
),
  Arrows:=1,
Axes(0,1 +i),
  Arrows:=0,
  u0:=11,nb:=15, Width:=6,
Color:=darkseagreen, Droite(1,-1,0), 
LabelAngle:=0, Label (2+i, "$y=x$") ,
Color:=red,  tMin:=-1, tMax:=12, Width:=8, 
              Cartesienne((3/4)*x+1),
LabelAngle:=0, Label (6+4.3*i, "$y=\frac{3x}{4}+1$"),              
 Width:=6, Color:=gray,
  suite((3/4)*x+1, u0,nb),
Color:=black,LabelAngle:=0, 
Label(8.5+9.25*i, "$u_1$", 
      7.5+7.93*i, "$u_2$",  
      6.6+6.95*i, "$u_3$",   
      5.7+6.21*i, "$u_4$",   
      5.3+5.66*i, "$u_5$")   
];
\end{texgraph}

\vspace*{.5cm}

On conjecture que :

\begin{itemize}
\item[•] $\left(u_n\right)_{n \in \N}$ est décroissante.
\item[•] $\forall n \in \N, 4 < u_n \leq 11$.
\item[•] La suite est convergente vers 4. On écrit alors $\lim\limits_{n \to +\infty} u_n = 4$. 
\end{itemize}

\vspace*{.3cm}

\newpage

\textbf{Exercice n°2} \\

Soit $\left(u_n\right)_{n \; \in \N}$ une suite définie par :$ \; \; \; \begin{cases}
u_0 = -2 \\
\forall n \in \N, u_{n+1} = -\dfrac{3}{4}u_n + 7 \\
\end{cases}$ 

\vspace*{.3cm}

On a $\Delta : y = -\dfrac{3}{4}x + 7$. On trace également $y = x$, la première bissectrice du plan. \\

\begin{texgraph}[file,name=Suite03,export=pgf]
Include "papiers.mac";
Graph image = [
Fenetre(-3+12*i, 12-.9*i, 1+i),Marges(0,0,0,0),
papier(milli,-3-i,12+12*i,
          [subsubgridcolor :=beige,
              subgridcolor :=antiquewhite,
                 gridcolor :=bisque]
      ),
  Arrows:=1,
  Axes(0,1 +i),
  Arrows:=0,
  u0:=-2,nb:=15, Width:=6,
Color:=darkseagreen, Droite(1,-1,0), 
 Label (2+i, "$y=x$"),
Color:=red,  tMin:=-3, tMax:=11, Width:=8, 
              Cartesienne(-1*(3/4)*x+7),
 Label (-2+9.6*i, "$y=\frac{-3x}{4}+7$"),              
 Width:=6, Color:=gray,
             suite(-1*(3/4)*x+7, u0,nb),
Color:=black,LabelAngle:=0, 
Label(-5/2+8.5*i, "$u_1$", 
        9+.7*i,   "$u_2$",  
        .8+6.8*i, "$u_3$",   
         7+2.2*i, "$u_4$",   
       2.5+5.66*i,"$u_5$"
)   
];
\end{texgraph}

\vspace*{.5cm}

On conjecture que  :
\begin{itemize}
\item[•] $\left(u_n\right)_{n \in \N}$ est non monotone.
\item[•] $\forall n \in \N, -2 \leqslant u_n \leq \dfrac{17}{2}$. (on a $u_0 = -2$ et $u_1 = \dfrac{17}{2}$ )
\item[•] La suite est convergente vers 4. On écrit alors $\lim\limits_{n \to +\infty} u_n = 4$. 
\end{itemize}


\ifdefined\COMPLETE
\else
    \end{document}
\fi