\vspace*{-1cm}

\part{Suites numériques}

\section{Introduction : le raisonnement par récurrence}

\subsection{Exemples}

\subsubsection{Exemple \no 1}

Montrer que : $\forall n \in \N, 1 + 2 + 3 + 4 + ... + n = \dfrac{n\left(n+1\right)}{2}$. \\

On raisonnera par récurrence :

\begin{itemize}

\item[*] On vérifie que la formule est vraie pour la première valeur de $n$, c'est-à-dire ici $n = 1$ : \vspace{.3cm} \\ $\dfrac{1\left(1+1\right)}{2} = \dfrac{1\times 2}{2} = 1$. \\

\item[*] On démontre que si la formule est vraie pour $n$, alors elle est vraie pour $n+1$, c'est-à-dire que si \\ $1 + 2 + 3 + 4 + ... + n = \dfrac{n\left(n+1\right)}{2}$ est vraie, alors $1 + 2 + 3 + 4 + ... + n + \left(n + 1\right) = \dfrac{\left(n+1\right)\left(n+2\right)}{2}$. \vspace{.3cm} \\ On pose une \textbf{hypothèse de récurrence} : $1 + 2 + 3 + 4 + ... + n = \dfrac{n\left(n+1\right)}{2}$.
\end{itemize}

\vspace*{.3cm}

\begin{tabular}{lll}
$1 + 2 + 3 + 4 + ... + n + \left(n+1\right)$ & $ = $ & $ \dfrac{n\left(n+1\right)}{2} + \left(n+1\right)$ \vspace*{.3cm} \\
& $=$ & $\dfrac{n\left(n+1\right) + 2\left(n+1\right)}{2}$ \vspace*{.3cm} \\
& $=$ & $\dfrac{\left(n+1\right)\left(n+2\right)}{2}$ \vspace*{.3cm} \\
\end{tabular}

\vspace*{-.6cm}

\subsubsection{Exemple \no 2}

Montrer que : $\forall n \in \N^*, 1^3 + 3^3 + 5^3 + 7^3 + ... + \left(2n - 1\right)^3 = 2n^4 - n^2$. \\ %Faire une accolade en dessous pour écrire qu'il y a n termes. Cf cahier

On raisonnera par récurrence : \\

\begin{itemize}

\item[*] On vérifie que la formule est vraie pour la première valeur de $n$, c'est-à-dire ici $n = 1$ :  \vspace{.3cm} \\ $1^3 = 1$ et $2 \times 2^4 - 1^2 = 2 - 1 = 1$. \\

\item[*] On démontre que si la formule est vraie pour $n$, alors elle est vraie pour $n+1$, c'est-à-dire que si $1^3 + 3^3 + ... + \left(2n-1\right)^3 = 2n^4 - n^2$ est vraie, alors $1^3 + 3^3 + ... + \left(2n-1\right)^3 + \left(2n+1\right)^3 = 2\left(n+1\right)^4 - \left(n+1\right)^2$. \vspace{.3cm} \\ On pose une \textbf{hypothèse de récurrence} : $1^3 + 3^3 + 5^3 + 7^3 + ... + \left(2n-1\right)^3 = 2n^4 - n^2$.
\end{itemize}

\vspace*{.3cm}

Pour montrer que $1^3 + 3^3 + 5^3 + 7^3 + ... + \left(2n-1\right)^3 + \left(2n+1\right) = 2\left(n+1\right)^4 - \left(n+1\right)^2$, on va calculer chaque membre de l'égalité séparément. \\

Or, d'après l'hypothèse de récurrence, $1^3 + 3^3 + 5^3 + 7^3 + ... + \left(2n-1\right)^3 + \left(2n+1\right) = 2n^4 - n^2 + \left(2n+1\right)^3$. \\

On calcule $2n^4 - n^2 + \left(2n+1\right)^3$ et $2\left(n+1\right)^4 - \left(n+1\right)^2$. \\

\vspace*{-5cm}

\newpage

\begin{tabular}{lll}
$2n^4 - n^2 + \left(2n+1\right)^3$ & $=$ & $2n^4 - n^2 + \left(2n+1\right)^2 \times \left(2n+1\right)$. \\
& $=$ & $2n^4 -n^2 + \left(2n+1\right)\left(4n^2 + 4n + 1\right)$ \\
& $=$ & $2n^4 - n^2 + 8n^3 + 8n^2 + 2n + 4n^2 + 4n + 1$ \\
& $=$ & $2n^4 + 8n^3 + 11n^2 + 6n + 1$ \\
\end{tabular}

\vspace*{.3cm}

\begin{tabular}{lll}
On a aussi $2\left(n+1\right)^4 - \left(n+1\right)^2$ & $ =$ &$ 2\left(n+1\right)^2\times\left(n+1\right)^2 - \left(n+1\right)^2$ \\
& $=$ & $2\left(n^2 + 2n + 1\right)\left(n^2 + 2n + 1\right) - \left(n^2 + 2n + 1\right)$ \\
& $=$ & $2\left(n^4 + 2n^3 + n^2 + 2n^3 + 4n^2 + 2n + n^2 + 2n + 1\right)-\left(n^2 + 2n + 1\right)$ \\
& $=$ & $2\left(n^4 + 4n^3 + 6n^2 + 4n + 1\right)-\left(n^2 + 2n + 1\right)$ \\
& $=$ & $2n^4 + 8n^3 + 12n^2 + 8n + 2 - n^2 - 2n - 1$ \\
& $=$ & $2n^4 + 8n^3 + 11n^2 + 6n + 1$ \\
\end{tabular}

\vspace*{.3cm}

On en conclut que $2n^4 - n^2 + \left(2n+1\right)^3 = 2\left(n+1\right)^4 - \left(n+1\right)^2$

\subsection{Conjecture et récurrence}

\subsubsection{Exemple \no 1}

Soit $n \in \N^*$. \\

Exprimer en fonction de $n$ la somme $S_n = 1 + 3 + 5 + 7 + ... + \left(2n -1\right)$. \\

On calcule les premiers termes afin de pouvoir conjecturer l'hypothèse à démontrer : \\

$S_1 = 1$. \\

$S_2 = 1 + 3 = 4$. \\

$S_3 = 1 + 3 + 5 = 9$ \\

$S_4 = 1 + 3 + 5 + 7 = 16$ \\

On conjecture que : $\forall n \in \N^*, S_n = n^2$. \\

Montrons que : $\forall n \in \N^*, 1 + 3 + 5 + 7 + ... + \left(2n-1\right) = n^2$. \\

On raisonnera par récurrence : \\

\begin{itemize}

\item[*] On vérifie que la formule est vraie pour la première valeur de $n$, c'est-à-dire ici $n = 1$ : \vspace{.3cm} \\ $S_1 = 1$ et $1^2 = 1$. \\

\item[*] On démontre que si la formule est vraie pour $n$, alors elle est vraie pour $n+1$, c'est-à-dire que si \\ $1 + 3 + 5 + 7 + ... + \left(2n-1\right) = n^2$ est vraie, alors $1 + 3 + 5 + 7 + ... + \left(2n-1\right) + \left(2n +1\right) = \left(n+1\right)^2.$ \vspace{.3cm} \\ On pose une \textbf{hypothèse de récurrence} : $1 + 3 + 5 + 7 + ... + \left(2n-1\right) = n^2$.
\end{itemize}

\vspace*{.3cm}

\begin{tabular}{lll}
$1 + 3 + 5 + 7 + ... + \left(2n-1\right) + \left(2n+1\right)$ & $ = $ & $ n^2 + \left(2n+1\right)$ \vspace*{.3cm} \\
& $=$ & $\left(n+1\right)^2$ \\
\end{tabular}

\newpage

\subsubsection{Exemple \no 2}

Soit $n \in \N^*$. \\

Exprimer en fonction de $n$ la somme $S_n = \dfrac{1}{3} + \dfrac{1}{15} + \dfrac{1}{35} + ... + \dfrac{1}{4n^2 - 1}$. \\

On calcule les premiers termes afin de pouvoir conjecturer l'hypothèse à démontrer : \\

$S_1 = \dfrac{1}{3}$. \\

$S_2 = \dfrac{6}{15} = \dfrac{2}{5}$. \\

$S_3 = \dfrac{35}{105} + \dfrac{7}{105} + \dfrac{3}{105} = \dfrac{45}{105} = \dfrac{3}{7}$ \\

$S_4 = \dfrac{1}{3} + \dfrac{1}{15} + \dfrac{1}{35} + \dfrac{1}{63} = \dfrac{315}{945} + \dfrac{63}{945} + \dfrac{27}{945} + \dfrac{15}{945} = \dfrac{420}{945} = \dfrac{4}{9}$ \\

On conjecture que : $\forall n \in \N^*, S_n = \dfrac{n}{2n+1}$. \\

Montrons que : $\forall n \in \N^*, \dfrac{1}{3} + \dfrac{1}{15} + \dfrac{1}{35} + ... + \dfrac{1}{4n^2 - 1} = \dfrac{n}{2n+1}$. \\

On raisonnera par récurrence : \\

\begin{itemize}

\item[*] On vérifie que la formule est vraie pour la première valeur de $n$, c'est-à-dire ici $n = 1$ : \vspace{.3cm} \\ $S_1 = \dfrac{1}{3}$ et $\dfrac{1}{1+ 1 + 1} = \dfrac{1}{3}$. \\

\item[*] On démontre que si la formule est vraie pour $n$, alors elle est vraie pour $n+1$, c'est-à-dire que si \\ $\dfrac{1}{3} + \dfrac{1}{15} + \dfrac{1}{35} + ... + \dfrac{1}{4n^2 - 1} = \dfrac{n}{2n+1}$ est vraie, alors $ \dfrac{1}{3} + \dfrac{1}{5} + \dfrac{1}{15} + ... + \dfrac{1}{4n^2 - 1} + \dfrac{1}{4\left(n+1\right)^2 - 1} = \dfrac{n+1}{2n + 3}$ \vspace{.3cm} \\ On pose une \textbf{hypothèse de récurrence} : $\dfrac{1}{3} + \dfrac{1}{15} + \dfrac{1}{35} + ... + \dfrac{1}{4n^2 - 1} = \dfrac{n}{2n+1}$.
\end{itemize}

\vspace*{.3cm}

\begin{tabular}{lll}
$\dfrac{1}{3} + \dfrac{1}{5} + \dfrac{1}{15} + ... + \dfrac{1}{4n^2 - 1} + \dfrac{1}{4\left(n+1\right)^2 - 1}$ & $=$ & $\dfrac{n}{2n + 1} + \dfrac{1}{4\left(n+1\right)^2 - 1}$ \vspace*{.3cm} \\
& $=$ & $\dfrac{n}{2n +1} + \dfrac{1}{\left[2\left(n+1\right)+1\right]
\left[2\left(n+1\right)-1\right]}$ \vspace*{.3cm} \\
& $=$ & $\dfrac{n}{2n + 1} + \dfrac{1}{\left(2n + 2 + 1\right)\left(2n + 2 - 1\right)}$ \\
& $=$ & $\dfrac{n}{2n + 1} + \dfrac{1}{\left(2n + 3\right)\left(2n + 1\right)}$ \vspace*{.3cm} \\ 
& $=$ & $\dfrac{n\left(2n+3\right)+1}{\left(2n+3\right)\left(2n+1\right)}$ \vspace*{.3cm} \\
& $=$ & $\dfrac{2n^2 + 3n + 1}{\left(2n+3\right)\left(2n+1\right)}$ \vspace*{.3cm} \\
& $=$ & $\dfrac{\left(2n+1\right)\left(n+1\right)}{\left(2n+3\right)\left(2n+1\right)}$ \vspace*{.3cm} \\
& $=$ & $\dfrac{n+1}{2n+3}$ \\
\end{tabular}