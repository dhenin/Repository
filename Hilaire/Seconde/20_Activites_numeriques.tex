

\section{Activités Numériques}

\subsection{Fractions}

\subsubsection{Rappels}



Écrire sous la forme d'une fraction irréductible :\\

$ A = ${$\dfrac{7}{9} - \dfrac{1}{9} \times \dfrac{3}{2} $ }\\

$ A = ${  $\dfrac{7}{9} - \dfrac{3}{18} $}\\

$ A = ${  $\dfrac{14-3}{18} $ }\\

$ A = ${  $\dfrac{11}{18} $ }\\

\vspace{1cm}

$ B = ${  $\left(\dfrac{2}{3}\right)^2 - \dfrac{3}{2} $ }\\

$ B = ${  $\dfrac{4}{9} - \dfrac{3}{2} $ }\\

$ B = ${  $\dfrac{8}{18} - \dfrac{27}{18} $} \\

$ B = - ${  $\dfrac{19}{18} $ }\\

\vspace{1cm}

$ C = ${  $\dfrac{A}{B} + \dfrac{11}{19} $ }\\

\vspace{.1cm}

$ C = { \Large \tfrac{\dfrac{11}{18}}{-\dfrac{19}{18}}} + \dfrac{11}{19}$ \\

\vspace{.3cm}

$ C = ${  $\dfrac{11}{18} \times \dfrac{18}{-19} + \dfrac{11}{19} $} \\

$ C = - ${  $\dfrac{11}{19} + \dfrac{11}{19} $ }\\

$ C = 0 $ \\

\newpage

\subsubsection{Un peu plus dur...}

\begin{minipage}{0.29\textwidth}
\vspace*{\stretch{1}}
$ A = ${  $\dfrac{9}{8} - \dfrac{\dfrac{7}{6}}{5} + \dfrac{4}{\dfrac{3}{2}} $ }\\

\vspace{.5cm}

$ A = ${  $\dfrac{9}{8} - \dfrac{7}{6} \times \dfrac{1}{5} + 4 \times \dfrac{2}{3} $} \\

\vspace{.5cm}

$ A = ${  $\dfrac{9}{8} - \dfrac{7}{30} + \dfrac{8}{3} $ }\\

\vspace{.5cm}

$ A = ${  $\dfrac{135}{120} - \dfrac{28}{120} + \dfrac{320}{120} $} \\

\vspace{.5cm}

$ A = ${  $\dfrac{107}{120} + \dfrac{320}{120} $ }\\

\vspace{.5cm}

$ A = ${  $\dfrac{427}{120} $ }
\vspace*{\stretch{2}}
\end{minipage}
\begin{minipage}{0.29\textwidth}
\vspace*{\stretch{1}}
$ B = ${$ \dfrac {1}{2-\dfrac{1}{3-\dfrac{1}{4-\dfrac{1}{5}}}} $} \\


$ B = ${$\dfrac{1}{2-\dfrac{1}{3-\dfrac{1}{\dfrac{19}{5}}}} $ }\\


$ B = ${  $\dfrac{1}{2-\dfrac{1}{3-\dfrac{5}{19}}} $ }\\


$ B = ${  $\dfrac{1}{2-\dfrac{1}{\dfrac{52}{19}}} $ }\\


$ B = ${  $\dfrac{1}{2-\dfrac{19}{52}} $ }\\

\vspace{.5cm}

$ B = ${  $\dfrac{1}{\dfrac{85}{52}} $ }\\

\vspace{.5cm}

$ B = ${  $\dfrac{52}{85} $ }\\

\vspace*{\stretch{2}}
\end{minipage}
\begin{minipage}{0.29\textwidth}
\vspace*{\stretch{1}}
$ C = \dfrac{1248}{7259} \times \dfrac{A}{B} $ \\

\vspace{.5cm}

$ C = \dfrac{1248}{7259} \times \dfrac{\dfrac{427}{120}}{\dfrac{52}{85}} $ \\

\vspace{.5cm}

$ C = \dfrac{1248}{7259} \times \dfrac{427}{120} \times \dfrac{85}{52} $ \\

\vspace{.5cm}

$ C = \dfrac{24}{17} \times \dfrac{85}{120} $ \\

\vspace{.5cm}

$ C = \dfrac{24}{17} \times \dfrac{17}{24} $ \\

$ C = 1 $ \\
\vspace*{\stretch{2}}
\end{minipage}

%\end{multicols}

\newpage

\subsection{Puissances}

\textbf{Exemple \no 0}

Écrire sous la forme d'une fraction irréductible. \\

$ A = \dfrac{\left(10^{-3}\right)^5 \times 10^8}{5 \times 10^{-6}} $ \\

$ A = \dfrac{10^{-15} \times 10^8}{5	\times10^{-6}} $ \\

$ A = \dfrac{10^{-7}}{10^{-6}} \times \dfrac{1}{5} $ \\

$ A = \dfrac{1}{5} \times 10^{-1} $ \\

$ A = \dfrac{1}{50} $ \\

\subsubsection{Rappels}



Soit a un nombre réel tel que $ a \neq 0 $. 

Soient n et p des nombres entiers relatifs. \\

\begin{itemize}

\item[*] $ a^n \times a^p = a^{n+p} $ \\

\item[*] $ \dfrac{a^n}{a^p} = a^{n-p} $ \\

\item[*] $ \left(a^n\right)^p = a^{np} $ \\

\end{itemize}

Soient $ a \in \R $ et $ b \in \R $ avec $ b \neq 0 $ \\

\begin{itemize}

\item[*] $ \left(ab\right)^n = a^nb^n $ \\

\item[*] $ \left(\dfrac{a}{b}\right)^n = \dfrac{a^n}{b^n} $ \\

\end{itemize}

\textbf{Et aussi...}

\begin{itemize}

\item[*] $ a^0 = 1 $ si et seulement si $ a \neq 0 $ \\

\item[*] $ a^{-n} = \dfrac{1}{a^n} $ si et seulement si $ a \neq 0 $ \\

\end{itemize}

\newpage
\subsubsection{Un peu plus dur...}

$ A = \dfrac{189}{2\left(-5\right)^{-2}-5\left(-2\right)^{-5}} $ \\

... \\

$ A = 800 $ \\

Soit $ B(n) = \dfrac{9^{n+1} + 9^n}{3^{2n+1} - 3^{2n}} $ \\

Calculer $ B(0) $, $ B(1) $, $ B(2)$ et $ B(3) $. Que remarque-t-on ? Justifiez. \\

Pour calculer $ B(1)$, $B(2)$ ou $ B(3)$, on remplace $n$ par $ 1 $, $2$ ou $3$

$ B(0) = 5 $ 

$ B(1) = 5 $

$ B(2) = 5 $

$ B(3) = 5 $

Pour justifier, on calculer $ B(n) $ : \\

$ B(n) = \dfrac{9^{n+1} + 9^n}{3^{2n+1} - 3^{2n}} $ \\

$ B(n) = \dfrac{\left(3^2\right)^{n+1} + \left(3^2\right)^n}{3^{2n+1} - 3^{2n}} $

$ B(n) = \dfrac{3^{2n+2} + 3^{2n}}{3^{2n+1} - 3^{2n}} $ \\

$ B(n) = \dfrac{3^{2n} \times 3^2 + 3^{2n}}{3^{2n} \times 3 - 3^{2n}} $ \\

$ B(n) = \dfrac{3^{2n}\left(3^{2} + 1\right)}{3^{2n}\left(3-1\right)} $ \\

$ B(n) = \dfrac{10}{2} $ \\

$ B(n) = 5 $ \\

$ C = \dfrac{8 + 2\sqrt{28} - \sqrt{252}}{3 ± 2\sqrt{63} - \sqrt{343}} $ \\

... \\

$ C = 5 + \sqrt{7} $

$ D = \dfrac{\dfrac{1}{\dfrac{\sqrt{3}}{\sqrt{11}}}-\dfrac{\dfrac{1}{\sqrt{3}}}{\sqrt{11}}}{\dfrac{1}{\sqrt{33}}} $ \\

...

$ D = 10 $

\newpage

\subsection{Racines carrées}

\subsubsection{Écrire sous la forme $ \mathbf{a\sqrt{b}} $}

Soient $ a \in \N $ et $ b \in \N $ et $ b $ le plus petit possible. \\

$ A = 2\sqrt{8} - 3\sqrt{32} + 2\sqrt{98} $ \\

$ A = 4\sqrt{2} - 12 \sqrt{2} + 14\sqrt{2} $ \\

$ A = \left(4-12+14)\right)\sqrt{2} $ \\

$ A = 6\sqrt{2} $ \\

$ A_{bis} = 3\sqrt{1183} - \sqrt{3703} - 2\sqrt{11767} $ \\

$ A_{bis} = 39\sqrt{7} - 23\sqrt{7} - 82\sqrt{7} $ \\

$ A_{bis} = \left(39-23-82\right)\sqrt{7} $ \\

$ A_{bis} = -66\sqrt{7} $ \\

$ B = 3\sqrt{5} \times 5\sqrt{2} \times 2\sqrt{15} $ \\

$ B = 3\sqrt{5} \times 5\sqrt{2} \times 2\sqrt{3 \times 5} $ \\

$ B = 3\sqrt{5}^2 \times 5\sqrt{2} \times 2\sqrt{3} $ \\

$ B = 15 \times 5\sqrt{2} \times 2\sqrt{3} $ \\

$ B = 150\sqrt{6} $ \\

$ B_{bis} = 4\sqrt{7} \times 11\sqrt{14} 5\sqrt{6} $ \\

$ B_{bis} =4\sqrt{7} \times 11\sqrt{2\times7} \times 5\sqrt{2\times3} $ \\

$ B_{bis} = 4 \times 11 \times 2 \times 5 \times 7\sqrt{3} $ \\

$ B_{bis} = 3080\sqrt{3} $

\subsubsection*{Rappels}

$ \sqrt{ab} = \sqrt{a}\sqrt{b} $ avec $ a \geqslant 0 $ et $ b \geqslant 0 $ \\

$ \sqrt{\dfrac{a}{b}} = \dfrac{\sqrt{a}}{\sqrt{b}} $ avec $ a \geqslant 0 $ et $ b > 0 $ \\

$ \sqrt{a + b} = $ Rien !

\newpage

\subsubsection{Racines carrées au dénominateur}

$ A = \dfrac{1}{\sqrt{5}} = \dfrac{\sqrt{5}}{\sqrt{5}^2} = \dfrac{\sqrt{5}}{5} $ \\

\vspace{0,2cm}

$ A_{bis} = \dfrac{15\sqrt{2}}{\sqrt{5}} = \dfrac{15\sqrt{10}}{5} = 3\sqrt{10} $ \\

$ B = \dfrac{4}{3-\sqrt{5}} $ \\

\vspace{1cm}

\textbf{1$^{re}$ idée :} \\

$ \dfrac{4\sqrt{5}}{\sqrt{5}\left(3-\sqrt{5}\right)} = \dfrac{4\sqrt{5}}{3\sqrt{5}-5} \Longrightarrow $ \textbf{NON !} \\
 
 
\vspace{1cm}


 \textbf{2$^{e}$ idée :}\\
 
$ \dfrac{4\left(3-\sqrt{5}\right)}{\left(3-\sqrt{5}\right)^2} = \dfrac{12 - 4\sqrt{5}}{14-6\sqrt{5}} \Longrightarrow $ \textbf{NON !} \\


\vspace{1cm}

\textbf{Idée géniale :} \\

$ \dfrac{4\left(3+\sqrt{5}\right)}{\left(3-\sqrt{5}\right) \left(3+\sqrt{5}\right)} = \dfrac{12+ 4\sqrt{5}}{9 - 5} = \dfrac{4\left(3+\sqrt{5}\right)}{4} = 3 + \sqrt{5} $ \\
 
$ 3 + \sqrt{5} $ est le \textbf{conjugué} de $ 3 - \sqrt{5} $. \\

$ B_{bis} = \dfrac{44}{3\sqrt{5} + 1} $ \\

$ B_{bis} = \dfrac{44\left(3\sqrt{5}-1\right)}{\left(3\sqrt{5} + 1\right)\left(3\sqrt{5}-1\right)} $ \\

$ B_{bis} = \dfrac{44\left(3\sqrt{5}-1\right)}{45-1} $ \\

$ B_{bis} = \dfrac{44\left(3\sqrt{5}-1\right)}{44} $ \\

$ B_{bis} = 3\sqrt{5} -1 $ \\

\newpage

\subsection{Exercices}

\textbf{Simplifier}

$ A = \left(\dfrac{\sqrt{17-2\sqrt{7}}}{5}\right)^2 + \left(\dfrac{1 + \sqrt{7}}{5}\right)^2 $ \\

... \\

$ A = 1 $ \\

$ B = \left(\sqrt{11+4\sqrt{7}} - \sqrt{11-4\sqrt{7}}\right)^2 $ \\

... \\

$ B = 16 $ \\

D'où $ \sqrt{11+4\sqrt{7}} - \sqrt{11-4\sqrt{7}} = 4 $ \\

$ C = \left(\sqrt{37-12\sqrt{7}} - \sqrt{37 + 12\sqrt{7}} \right)^2 $

... \\

$ C = 36 $ \\

Ainsi $ \sqrt{37-12\sqrt{7}} - \sqrt{37+12\sqrt{7}} = -6 $ car $ \sqrt{37-12\sqrt{7}} - \sqrt{37} +\sqrt{12\sqrt{7}} < 0 $ \\

\underline{Amusette :} \\

$ \sqrt{37-12\sqrt{7}} = \sqrt{\left(3-2\sqrt{7}\right)^2} = -3 + 2\sqrt{7} $ car $ 3^2 < \left(2\sqrt{7}\right)^2 $ \\

$ \sqrt{37 + 12\sqrt{7}} = \sqrt{\left(3 + 2\sqrt{7}\right)^2} = 3+2\sqrt{7} $ \\

 Ainsi $ \sqrt{37-12\sqrt{7}} - \sqrt{37+12\sqrt{7}} = \left(-3 + 2\sqrt{7}\right) -\left(3+2\sqrt{7}\right) = -3 = 2\sqrt{7} -3 -2\sqrt{7} = -6 $ \\
 
\newpage

\subsection{L'apothéose :}

On donne $ \varphi = \dfrac{1+\sqrt{5}}{2} $. Ce nombre s'appelle le nombre d'or et a des propriétés bien particulières. \\

\subsubsection{Montrer que $\mathbf{\varphi^2 = \varphi + 1 }$}

$ A = \left(\dfrac{1+\sqrt{5}}{2}\right)^2 $ \\

$ A = \dfrac{6 + 2\sqrt{5}}{4} $ \\

$ A = \dfrac{2\left(3+\sqrt{5}\right)}{2\times 2} $ \\

$ A = \dfrac{3+\sqrt{5}}{2} $ \\

$ B = \dfrac{1 + \sqrt{5}}{2} + 1 $ \\

$ B = \dfrac{1 + \sqrt{5} + 2}{2} $ \\

$ B = \dfrac{3 + \sqrt{5}}{2} $ \\

\subsubsection{Calculez d'un seul coup}

$ C = \sqrt{1+\sqrt{1+\sqrt{1+\varphi}}} $ \\

$ C = \sqrt{1+\sqrt{1+\sqrt{1+\dfrac{1 + \sqrt{5}}{2}}}} $ \\

$ C = \dfrac{1 + \sqrt{5}}{2} $ \\


