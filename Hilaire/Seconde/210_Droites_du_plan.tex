\section{Droites du plan}
\subsection{Définitions}

Une droite $(d)$ est définie :

\begin{itemize}
    \item Soit par la donnée d'un point $M_{0}$ et d'un vecteur non nul         
          $\vec{u}$ \\ 

           \vspace{.5cm}
 \begin{center}
       
\begin{tikzpicture}[scale=0.8]
      \coordinate (M) at (0, 0) ; 

        % Le coef directeur est 1/3 et le vecteur directeur (3 ; 1)     

         \draw[thick] (-4,-1.5)  -- (8,2.5) ;
         \draw (0,0) node {$\times$} ; \draw (0,0) node [above] {$M_{0}$} ;
         \draw[thick,darkgreen, ->] (0,-0.1)  -- (3,0.9) ;
         \draw[thick,darkgreen, ->] (-1,2)  -- node[midway,above, darkgreen]
              {$\overrightarrow{u}$} (2,3) ;
         \tkzDefPoint(8,2.5){D}
        \tkzLabelPoint[color=black,below](D){$(d)$}
\end{tikzpicture}
   
 
\end{center} 

\centerline{\textcolor{darkgreen}
           {\bf $\mathbf{\overrightarrow{u}}$est le vecteur directeur de
            la droite $\mathbf{(d)}$.}
            } 

\vspace{.5cm}

\begin{center}
     \fcolorbox{black}  {white}{
      \hbox{
        $M\in(d)\Longleftrightarrow\overrightarrow{M_{0}M}$ 
            et     $\overrightarrow{u}$ sont colinéaires.}}
\end{center}

\vspace{.5cm}

\item Soit par la donnée de 2 points distincts $A$ et $B$.

\begin{center}
\begin{tikzpicture}[scale=0.8]
% Le coef directeur est 1/5 et le vecteur directeur (5 ; 1)     
      \tkzDefPoint(5,1){A}
      \tkzText(A){\textcolor{black}{$\times$}}
      \tkzLabelPoint[color=black,below](A){$A$}
      \tkzDefPoint(10,2){B}
      \tkzText(B){\textcolor{black}{$\times$}}
      \tkzLabelPoint[color=black,below](B){$B$}
      \tkzDefPoint(15,3){D}
      \tkzLabelPoint[color=black,below](D){$(d)$}
      \draw [thick] (0,0) -- (15,3) ; 
\end{tikzpicture}
\end{center}

\vspace{.5cm}

\begin{center}
    \fcolorbox{black}  {white}{
     \hbox{
       $M\in(d)\Longleftrightarrow\overrightarrow{AM}$ 
           et $\overrightarrow{AB}$ sont colinéaires.}}
\end{center}

\end{itemize}

\newpage 


\subsection{Équation cartésienne d'une droite}
\vspace{-.4cm}\hspace{3cm}$\llcorner$ {\footnotesize de René Descartes}

\subsubsection{Exemple \no 1}
Soit $(O, \vec{i}, \vec{j})$ un repère. \\
Soient $M_{0}(-3,2) $ et $\vec{u}\left(\begin{array}{c}
                                    1\\
                                    -2
                               \end{array}\right)$.
                 
Soit $(d)$ la droite définie par $M_{0}$ et $\vec{u}$.

\begin{center}
\begin{tikzpicture}[scale=.8]
     \tkzInit[xmin=-4.5,xmax=2.5, ymin=-5.5, ymax=5.5]
     \tkzRep[xlabel=$\vec{i}$, ylabel=$\vec{j}$]
     \tkzDrawXY [noticks]%, label={}]
     \tkzGrid [color=bistre,line width=0.01] 

     \tkzLabelPoint[color=black,below](-3,0){$-3$} 
     \tkzLabelPoint[color=black,left](0,2){$2$}
     \draw [color=violet, thick, ->] (0,0)   -- +(1,-2) 
               node[midway,below, violet]{$ \overrightarrow{u}$}; 
     \draw [color=violet, thick, ->] (-3.05,2)   -- +(1,-2) 
               node[midway,below, violet]{$ \overrightarrow{u}$};
     \tkzDefPoint(-3,2){M0}
     \tkzText(M0){\textcolor{violet}{$\times$}}
     \tkzLabelPoint[color=violet,below](M0){$M_{0}$}
     \draw [color=black, thick] (-5, 6 ) -- +(6, -12) ; 
     \tkzLabelPoint[color=black,above](-5,6){$(d)$}
     \tkzDefPoint(-1,-2){M1}
     \tkzText(M1){\textcolor{black}{$\times$}}
     \tkzLabelPoint[color=black,below](M1){$M_{1}$}
\end{tikzpicture}
\end{center}
\vspace{.5cm}

\begin{align*}
    M(x,y)\in(d) 
        &\Longleftrightarrow   \overrightarrow{M_{0}M} 
             \text { et }      \vec{u} \text{ sont colinéaires} \\ 
        &\Longleftrightarrow   \text{Det}(\overrightarrow{M_{0}M},
                                          \vec{u})        = 0 \\
        & \Longleftrightarrow \left| \begin{array}{cc}
                                          x + 3 & 1\\
                                          y - 2 & -2
                                     \end{array} \right|  =  0 \\
       & \Longleftrightarrow      -2(x+3)-(y-2)   =   0\\
       & \Longleftrightarrow          -2x-6-y+2   =   0\\
       & \Longleftrightarrow          -2x-y-4     =   0\\
       & \Longleftrightarrow           2x+y+4     =   0\\
\end{align*}

Équation cartésienne de (d) : $2x+y+4=0$ 

\vspace{.5cm}


\begin{tabular}{lll}
\textbf{Remarque :} 
         & $2x+y+4 = 0$     &      \\
         &   $ y = -2x -4 $ & Cette équation est l'équation réduite de la droite $(d)$ \\
\end{tabular}

\newpage

\subsubsection{Exemple \no 2}

Soient $A(-1, -3) $ et$  B(2, -1)$ \\
Soit $(d)$ la droite définie par $A$ et $B$ \\



\begin{center}
\begin{tikzpicture}[scale=.8]
    \tkzInit[xmin=-4.5,xmax=6.5, ymin=-5.5, ymax=3.5]
    \tkzRep[xlabel=$\vec{i}$, ylabel=$\vec{j}$]
    \tkzDrawXY [noticks]%, label={}]
    \tkzGrid [color=bistre,line width=0.01] 
    \tkzLabelPoint[color=black,below](-1,0){$-1$}
    \tkzLabelPoint[color=black,below](2,0){$2$}
    \tkzLabelPoint[color=black,left](0,-1){$-1$}
    \tkzLabelPoint[color=black,right](0,-3){$-3$}
    \tkzDefPoint(-1,-3){A}
    \tkzText(A){\textcolor{red}{$\times$}}
    \tkzLabelPoint[color=red,below](A){$A$}
    \tkzDefPoint(2,-1){B}
    \tkzText(B){\textcolor{red}{$\times$}}
    \tkzLabelPoint[color=red,below](B){$B$}
    \draw [color=black, thick] (-4, -5 ) -- +(12, 8) ; 
    \tkzDefPoint(5,1){M1}
    \tkzText(M1){\textcolor{black}{$\times$}}
    \tkzLabelPoint[color=black,above](M1){$M_{1}$}
    \tkzDefPoint(1,1){M2}
    \tkzText(M2){\textcolor{black}{$\times$}}
    \tkzLabelPoint[color=black,above](M2){$M_{2}$}
\end{tikzpicture}
\end{center}

\vspace{.5cm}

\begin{align*}
    M(x,y)\in(d) 
        & \Longleftrightarrow \overrightarrow{M_{0}M} \text { et } \vec{u} \text{ sont colinéaires} \\ 
        & \Longleftrightarrow \text{Det}(\overrightarrow{AM},\overrightarrow{AB}) = 0 \\
        & \Longleftrightarrow \left|
                   \begin{array}{cc}
                     x+1 & 3\\
                     y+3 & 2
                   \end{array} \right|        =    0 \\
        & \Longleftrightarrow 2(x+1)-3(y+3)   =    0 \\
        & \Longleftrightarrow  2x+2-3y-9      =    0 \\
        & \Longleftrightarrow  2x-3y-7        =    0 \\
\end{align*}


\vspace{.5cm}

% A reprendre l'espacement des 2 lignes de l'équation réduite
{\renewcommand{\arraystretch }{1.75}
\begin{tabular}{rrl}
	Équation cartésienne de (d) : & $  2x-3y-7 $ & $ = 0 $ \\
	Équation réduite de (d) :     & $ -3y      $ & $ = -2x +7 $\\
                                      & $   y      $ & $ = \dfrac{2}{3}x -\dfrac{7}{3}$ \\
\end{tabular}}
\renewcommand{\arraystretch }{1}



\vspace{.5cm}

Par exemple : \\

Pour  $M_{1}(5,1)$ : $10 -3 -7 = 0 $ 
donc, $M_{1} \in (d)$. \\

Pour  $M_{2}(1,1)$ : $2-3-7\neq 0$
donc, $M_{1} \notin (d)$

\newpage

\subsubsection{Conclusion}

\textbf{En résumé} Soit $\mathbf{ (d) }$ une droite,\\

les équations cartésiennes de (d) sont de la forme :\\
            $ax+by+c=0$ \\
           \textcolor{red}{avec $a$ et $b$ non simultanément nuls.
	  } 

\vspace{.5cm}

{\bf Réciproquement} : \begin{quote}
    Tout équation de la forme $ax+by+c=0$ avec $a$ et $b$ simultanément non nuls est l'équation cartésienne d'une droite.
\end{quote} 

\subsubsection{Exercice \no 1}

Soit la droite $(d)$ d'équation cartésienne $ x-3y+4=0$ \\

\begin{center}
\begin{tikzpicture}[scale=.7]
    \tkzInit[xmin=-6.5,xmax=7.5, ymin=-2.5, ymax=4.5]
    \tkzRep[xlabel=$\vec{i}$, ylabel=$\vec{j}$]
    \tkzDrawXY [noticks]%, label={}]
    \tkzGrid [color=bistre,line width=0.01] 
    \tkzDefPoint(-4,-0){A}
    \tkzText(A){\textcolor{red}{$\times$}}
    \tkzLabelPoint[color=red,above](A){$A$}
    \tkzDefPoint(-1,1){B}
    \tkzText(B){\textcolor{red}{$\times$}}
    \tkzLabelPoint[color=red,above](B){$B$}
    \tkzDefPoint(2,2){C}
    \tkzText(C){\textcolor{red}{$\times$}}
    \tkzLabelPoint[color=red,above](C){$C$}
    \tkzDefPoint(5,3){D}
    \tkzText(D){\textcolor{red}{$\times$}}
    \tkzLabelPoint[color=red,above](D){$B$}
    \draw [color=red, thick] (-7, -1 ) -- +(15, 5) ; 
\end{tikzpicture}
\end{center}

On cherche graphiquement des points de $(d)$.

    $A(-4,0)$\\
    $B(-1,1)$\\
    $C(2,2)$\\
    $D(5,3)$\\

On remarque que :
%Dans la liste des vecteurs, il faut aligner le vecteur AB avec AC et AD.

$\overrightarrow{AB}
		          \left(\begin{array}{c}
                                3\\
                                1
                           \end{array}\right) 
			        \overrightarrow{AB} = \overrightarrow{BC} = \overrightarrow{CD}$ \\

$\overrightarrow{AC}
	                 \left(\begin{array}{c}
                               6\\
                               2
                         \end{array}\right) \overrightarrow{AC} = 2\overrightarrow{AB}$ \\

$\overrightarrow{AD}
	                \left(\begin{array}{c}
                              9\\
                              3
                        \end{array}\right) \overrightarrow{AD} = 3\overrightarrow{AB}$ \\
                        
Ainsi, l'incrément de $x$ est : $+ 3$ \\ 
et l'incrément de $y$ est : $ + 1$. \\

$\overrightarrow{AB}$ est donc un vecteur directeur de $(d)$. \\

De manière générale, $(d) : ax+by+c = 0$  est caractérisée par le vecteur directeur $\vec{u} 
                        \left(\begin{array}{c}
                              -b\\
                              a
                        \end{array}\right)$


\newpage

\subsubsection{Exercice \no 2}

Soit $D : 10x-7y+8 = 0$ \\

L'équation réduite de $(D)$ est : 

\begin{tabular}{r@{$\;$}c@{$\;$}l}
$-7y$ & = & $-10x-8 $\\
$y$ &  = & $ \dfrac{10}{7} x + \dfrac{8}{7} $ \\
\end{tabular}        \\

On cherche 2 points appartenant à $(D)$ :
        
$A(-5, -6)$ et
$B(2,4) $

\subsubsection{Exercice \no 3}

Soit $(D): 13x +12y+57 = 0$. \\

L'équation réduite de $(D)$ est : 

\begin{tabular}{r@{$\;$}c@{$\;$}l}
$12y$ &= & $-13x-57 $\\
$y$ & =& $ -\dfrac{13}{12}x-\dfrac{57}{12} $ \\
\end{tabular}        \\

On cherche 2 points appartenant à $(D)$ : 

$A(-9, 5)$ et
$B(3,-8) $

\newpage

\subsubsection{Exercice \no 4}

Soit $D_{1} : 10x -7y -2 = 0$.

L'équation réduite de $(D_{1})$ est :

\begin{tabular}{r@{$\;$}c@{$\;$}l}
$-7y$ & = & $ -10x +2 $ \\
$y$ & =& $ \dfrac{10}{7}x-\dfrac{2}{7} $ \\
\end{tabular} \\

On trouve les points $A_{1}(-4, -6) $ et $  B_{1}(3,4) $ qui appartiennent à $(D_{1})$.

\vspace{.5cm}

Soit $D_{2} : 5x +9y -76 = 0$ 

L'équation réduite de $(D_{2})$ est :

\begin{tabular}{r@{$\;$}c@{$\;$}l}
$9y$ & =& $ -5x +76 $ \\
$y$ & =& $ -\dfrac{5}{9}x+\dfrac{76}{9}$\\
\end{tabular} \\

On trouve les points  $A_{2}(-1, 9) $ et $ B_{2}(8,4) $ qui appartiennent à $(D_{2})$.


\vspace{.5cm}

\begin{center}
\begin{tikzpicture}[scale=.6]
\tkzInit[xmin=-4.5,xmax=10.5, ymin=-6.5, ymax=10.5]
\tkzRep[xlabel=$\vec{i}$, ylabel=$\vec{j}$]
\tkzDrawXY [noticks]%, label={}]
\tkzGrid [color=bistre,line width=0.01] 
\clip (-5,11) rectangle (11,-7);

\tkzDefPoint(-4,-6){A1}
\tkzText(A1){\textcolor{black}{$\times$}}
\tkzLabelPoint[color=black,below](A1){$A_{1}$}
\tkzDefPoint(3,4){B1}
\tkzText(B1){\textcolor{black}{$\times$}}
\tkzLabelPoint[color=black,below](B1){$B_{1}$}

\tkzDefPoint(-1,9){A2}
\tkzText(A2){\textcolor{black}{$\times$}}
\tkzLabelPoint[color=black,above](A2){$A_{2}$}
\tkzDefPoint(8,4){B2}
\tkzText(B2){\textcolor{black}{$\times$}}
\tkzLabelPoint[color=black,above](B2){$B_{2}$}

\tkzDefPoint(22/5, 6){I}
\tkzText(I){\textcolor{black}{$\times$}}
\tkzLabelPoint[color=black,below](I){$I$}

\draw [thick] (-5, -52/7 ) -- +(14, 20) ; 
\draw [thick] (-5, 101/9 ) -- +(18, -10) ; 
\draw [thick, dashed] (0,6 ) -- +(4.4,0) ; 
\draw [thick, dashed] (4.4,0 ) -- +(0,6) ; 

\tkzLabelPoint[color=black,below](4.4,6){$I$}

\tkzText(-4,0){\textcolor{black}{$\mid$}}
\tkzLabelPoint[color=black,below](-4,0){$-4$}
\tkzText(-1,0){\textcolor{black}{$\mid$}}
\tkzLabelPoint[color=black,below](-1,0){$-1$}
\tkzText(3,0){\textcolor{black}{$\mid$}}
\tkzLabelPoint[color=black,below](3,0){$3$}
\tkzText(4.4,0){\textcolor{black}{$\mid$}}
\tkzLabelPoint[color=black,below](4.4,0){$4.4$}
\tkzText(8,0){\textcolor{black}{$\mid$}}
\tkzLabelPoint[color=black,below](8,0){$8$}
\tkzText(0,9){\textcolor{black}{\_}}
\tkzLabelPoint[color=black,right](0,9){$9$}
\tkzText(0,8){\textcolor{black}{\_}}
\tkzLabelPoint[color=black,left](0,8){$8$}
\tkzText(0,6){\textcolor{black}{\_}}
\tkzLabelPoint[color=black,left](0,6){$6$}
\tkzText(0,4){\textcolor{black}{\_}}
\tkzLabelPoint[color=black,left](0,4){$4$}
\tkzText(0,-6){\textcolor{black}{\_}}
\tkzLabelPoint[color=black,left](0,-6){$-6$}

% \tkzLabelPoint[color=black,above](-7,-1){$(d)$}
\end{tikzpicture}
\end{center}


\vspace{.5cm}

Soit $I$ le point d'intersection de $D_{1}$ et $D_{2}$. \\
 On cherche le point qui associe à $x$ la même ordonnée $y$ :

\begin{tabular}{l|l|l|l}
\multicolumn{1}{c|}{$x$} & \multicolumn{1}{c|}{$y_{1}$}  & \multicolumn{1}{c|}{$y_{2}$} & \\
\hline
$4$ & $5,4286$ & $6,2222$ & \\
$4,2$ & $5,7143$ & $6,1111$ & \\
$4,4$ & $6$ & $6$ & $\longleftarrow$ OUI \\
\end{tabular}

Ainsi, le point d'intersection des deux droites $(D_{1})$ et $ (D_{2}) $ est le point $ I\left(\dfrac{22}{5}, 6\right) $

\newpage
\subsection{Droites parallèles et droites sécantes}
\subsubsection{Conditions de parallélisme de deux droites}

Soit $(O,\vec{i},\vec{j})$ un repère.

Soit $D$ : $ax +by + c = 0 \qquad $ avec comme vecteur directeur $ \vec{u}
\left(\begin{array}{c}
-b\\
a
\end{array}\right)$


Soit $D'$ : $a'x +b'y + c' = 0 \quad $ avec comme vecteur directeur $ \overrightarrow{u'}
\left(\begin{array}{c}
-b'\\
a'
\end{array}\right)$

% \sslash necessite \usepackage{stmaryrd}


\begin{align*}
     D \sslash D' &\Longleftrightarrow   \overrightarrow{u} \text { et }\overrightarrow{u'}               \text{ sont colinéaires} \\ 
     &\Longleftrightarrow \text{det}(\overrightarrow{u},\overrightarrow{u'})=0 \\
     & \Longleftrightarrow \left| \begin{array}{cc}
                                    -b & -b'\\
                                     a & a'
                                  \end{array} \right|=0 \\
    & \Longleftrightarrow -ba'-a \times (-b')=0\\
    & \Longleftrightarrow  -a'b+ab'=0\\
    & \Longleftrightarrow ab' - a'b=0\\
    & \Longleftrightarrow \left| \begin{array}{cc}
                                    a & b\\
                                   a' & b'
                                 \end{array} \right|=0 \\
\end{align*}

\fcolorbox{black}  {white}{\hbox
{\begin{tabular}{rrl}
$D$ :& $ax +by + c =$&$0$\\
      $D'$ : & $a'x +b'y + c' =$ &$0$\\ 
\multicolumn{3}{c}{$D \sslash D' \Longleftrightarrow \left| \begin{array}{cc}
-b & -b'\\
a & a'
\end{array} \right|=0 $}           
\end{tabular}       
}}

\vspace{.5cm}

\subsubsection{Exemple}

%\begin{tabular}{llll|l}
%            &     &  &    & NB :   \\
%{\bf Ex : } & $D$ & :   & $x -3y +4 =0$ & $D$ : $x -3y +4 = 0 $ \\
 %           & $D'$ & :  & $3x -9y -14 = 0 $ & $D'$ : $3x -9y +12 = 0 $ \\
%            
\begin{tabular}{lll}
$D$ & :   & $x -3y +4 =0$ \\
$D'$ & :  & $3x -9y -14 = 0 $ \\
\end{tabular} \\

\begin{tabular}{ll}
$ \left| \begin{array}{cc}
1 & -3 \\
3 & -9  \end{array} \right| = -9 +9 = 0$
\end{tabular}

Ainsi $ D \sslash D'  $. \\

\textbf{Remarque}

Si les droites sont parallèles, alors, dans le déterminant, les coefficients $a$ ,$a'$ et $b$, $b'$ sont proportionnels deux à deux. \\

Si même $c$ et $c'$ sont dans la même proportion, alors les droites sont confondues :

\begin{tabular}{lll}
$D$ & :   & $x -3y +4 =0$ \\
$D'$ & :  & $3x -9y + 12 = 0 $ \\
\end{tabular} \\

\begin{tabular}{ll}
$ \left| \begin{array}{cc}
1 & -3 \\
3 & -9  \end{array} \right| = -9 +9 = 0$ \\
\end{tabular} \\

Mais on peut aussi simplifier $D'$ :  $3x -9y + 12 = 0 $ en $D'$ :   $x -3y +4 =0$ \\

Ainsi $D = D'$.

\newpage 

\subsection{Intersection de 2 droites non parallèles}

Soit $(O,\vec{i},\vec{j})$ un repère.

\vspace{.5cm}


Soit $D_{1}$ : $10x -7y -2 = 0 $

Soit $D_{2}$ : $5x +9y -76 = 0 $

\vspace{.5cm}


$\left| \begin{array}{cc}
10 & -7\\
5 & 9    \end{array} \right|= 90 +35 = 125 \neq 0 $

\vspace{.5cm}

Donc $D_{1}$ et $D_{2}$ sont sécantes en un point $I$.

\vspace{.5cm}

\begin{tabular}{lcl}
$\left\{ \begin{array}{ll|l} 
10x -7y &=2 &9 \\
  5x +9y &=76 & 7
\end{array} \right. $ & \hspace{1cm} & $\left\{ \begin{array}{ll|l} 
10x -7y &=2 & \\
  5x +9y &=76 & -2
\end{array} \right. $ \\
   & & \\
$\left\{ \begin{array}{ll} 
90x -63y &=18  \\
  35x +63y &=532 
\end{array} \right. $ & \hspace{1cm} & $\left\{ \begin{array}{ll} 
10x -7y &=2  \\
  -10x -18y &=-152
\end{array} \right. $ \\
\cline{1-1} \cline{3-3}\\
$125x = 550$ & & $-25y=-150 $\\
$ x = 4,4$   & & $25y=150$ \\
$x = \frac{22}{5}$ & & $y=6$ \\
\end{tabular}

\vspace{.5cm}

$I\left(\dfrac{22}{5}; 6\right)$ 

\vspace{.5cm}

\textbf{Remarque :}

Après avoir procédé à la méthode de combinaison linéaire, on trouve une équation à une inconnue. Le coefficient de l'inconnue est alors toujours un diviseur du déterminant.

\newpage

\subsection*{Un superbe exercice : $\star \star \star$ } 

Soit $(O,\vec{i},\vec{j})$ un repère.

\vspace{.1cm}

\begin{enumerate}
\item Soient $A(-2,1)$ et $B(1, -7) $\\
Déterminer l'équation cartésienne de la droite $(AB)$

\item Construire avec précision : 

\begin{tabular}{lccl}
$D_{1}$ & : & $2x -15y +61$ & $= 0 $ \\
$D_{2}$ & : & $14x -9y -21$ & $ = 0 $ \\
\end{tabular}

\item Déterminer une équation cartésienne de la droite $\Delta$ qui passe par $C(-6, -1)$ et qui est parallèle à   $D_{2}$ 

\item Montrer que $(AB)$, $D_{1}$ et $\Delta$ sont concourantes en un point $I$ dont on déterminera les coordonnées.
\end{enumerate}

% -------------------- Vérification faite, c'est bien -13 ---------- 
% --- D'ailleurs, à la question 4 on a 8x +3y = -13
% ------  J'ai bien des difficultés à distinguer 3 et 9 dans tes écrits.
% -- C'est important de réctifier dès maintenant 


1. \raisebox{-13.75ex}{\parbox{8cm}{
\begin{align*}
M(x,y)\in(d) &\Longleftrightarrow   \overrightarrow{AM} \text { et } \overrightarrow{AB}\text{ sont colinéaires} \\ 
            &\Longleftrightarrow \text{Det}(\overrightarrow{AM},\overrightarrow{AB})=0 \\
            & \Longleftrightarrow \left| \begin{array}{cc}
                                             x+2 & 3\\
                                             y-1 & -8
                                         \end{array} \right|=0 \\
            & \Longleftrightarrow -8(x+2)-3(y-1)=0\\
            & \Longleftrightarrow  -8x -16 -3y -3 = 0 \\
            & \Longleftrightarrow  -8x -3y -13 = 0 \\
            & \Longleftrightarrow  8x +3y +13 = 0 \\
\end{align*}}
}

2. \definecolor{xdxdff}{rgb}{0.49,0.49,1}
\definecolor{uuuuuu}{rgb}{0.27,0.27,0.27}
\definecolor{qqqqff}{rgb}{0,0,1}
\begin{center}
\begin{tikzpicture}[line cap=round,line join=round,>=triangle 45,x=1.0cm,y=1.0cm,scale=0.6]
\clip(-9,-9) rectangle (8,8);
\draw[->,color=black] (-10,0) -- (9,0);
\foreach \x in {-8,-7,-6,-5,-4,-3,-2,-1,1,2,3,4,5,6,7,8}
\draw[shift={(\x,0)},color=black] (0pt,2pt) -- (0pt,-2pt);
\draw[->,color=black] (0,-12.84) -- (0,11.41);
\foreach \y in {-8,-7,-6,-5,-4,-3,-2,-1,1,2,3,4,5,6,7}
\draw[shift={(0,\y)},color=black] (2pt,0pt) -- (-2pt,0pt);

\draw [domain=-10:9] plot(\x,{(-13-8*\x)/3});
\draw [domain=-10:9] plot(\x,{(--69--3*\x)/15});
\draw [domain=-10:9] plot(\x,{(-26.98-4.98*\x)/-2.88});
\draw [domain=-10:9] plot(\x,{(--26.98--4.98*\x)/2.88});
\draw (-6.5,4.5) node[anchor=north west] {$ (D_1) $};
\draw (-8,-2.5) node[right] {$(\Delta)$};
\draw [domain=-10:9] plot(\x,{(-7--14*\x)/9});
\draw (7,0) node[below] {$ 7 $};
\draw [->] (0,0) -- (1,0);
\draw [->] (0,0) -- (0,1);
\draw (0,-7) node[left] {$ -7 $};
\draw (-8,0) node[below] {$ -8 $};
\draw (0,7) node[left ] {$ 5 $};
\draw (1.5,-7.24) node[anchor=north west] {$(AB)$};
\draw (0,0) node[anchor=north east] {$ O $};
\begin{scriptsize}
\fill [color=qqqqff] (-8,3) circle (1.5pt);
\draw[color=qqqqff] (-8,3) node[below] {$A1$};
\fill [color=qqqqff] (7,6) circle (1.5pt);
\draw[color=qqqqff] (7,6) node [below]{$B1$};
\fill [color=qqqqff] (-2,1) circle (1.5pt);
\draw[color=qqqqff] (-2,1) node[right] {$A$};
\fill [color=qqqqff] (1,-7) circle (1.5pt);
\draw[color=qqqqff] (1,-7) node [right]{$B$};
\fill [color=uuuuuu] (-3.12,3.98) circle (1.5pt);
\draw[color=uuuuuu] (-3,3.7) node [right]{$I$};
\fill [color=qqqqff] (-6,-1) circle (1.5pt);
\draw[color=qqqqff] (-6,-1) node [right]{$C$};
\fill [color=qqqqff] (-4,-7) circle (1.5pt);
\draw[color=qqqqff] (-4,-7) node[right] {$A2$};
\fill [color=qqqqff] (5,7) circle (1.5pt);
\draw[color=qqqqff] (5,7) node [right] {$B2$};
\fill [color=xdxdff] (1,0) circle (1.5pt);
\draw[color=black] (0.5,0) node [below] {$i$};
\fill [color=xdxdff] (0,1) circle (1.5pt);
\draw[color=black] (0,0.5) node [left] {$j$};
\end{scriptsize}
\end{tikzpicture}\\
\end{center}

\newpage

\begin{tabular}{llll}
      &   \begin{minipage}{3cm}
            \begin{equation*} \begin{aligned}
 D_1 :     -15y    &= -2x -61\\
                   y    &=\dfrac{2}{15}x +\dfrac{61}{15}\\
             \end{aligned} \end{equation*}
              \end{minipage} 
                       & \hspace*{2cm} 
                          & $A_1(-8,3) \qquad B_1(7,5) $\\
  &   & &  \\       
& \begin{minipage}{3cm} 
          \begin{equation*} \begin{aligned}
D_2 :     -9y   &= -14x +211\\
                y   &=\dfrac{14}{9}x +\dfrac{7}{9}\\
          \end{aligned} \end{equation*}
          \end{minipage} 
                   & \hspace*{2cm} 
                      & $A_2(-4,-7) \qquad B_2(5,7) $ \\
\end{tabular}\\

3. Si les deux droites sont parallèles, alors les vecteurs directeurs sont colinéaires.\\

$D_2$ a pour vecteur directeur $\vec{u} \left( \begin{array}{c}
            9\\
            14
        \end{array} \right) $\\

Donc $\Delta$ est définie par $C(-6,-1)$ et         
$ \vec{u} \left( \begin{array}{c}
            9\\
            14
        \end{array} \right)$
     
\raisebox{4.7ex}{\begin{minipage}[t]{5cm}
\begin{equation*}  \begin{aligned}
M(x,y) \in \Delta  &\Longleftrightarrow \overrightarrow{CM} \textrm{ et } \vec{u} \textrm{ sont colinéaires} \\
   & \Longleftrightarrow \textrm{Det }  (\overrightarrow{CM},\vec{u}) = 0 \\
   & \Longleftrightarrow \left|\begin{array}{ll}
x+6 & 9\\
y+1 & 14
\end{array}\right| = 0 \\
    & \Longleftrightarrow 14(x+6) -9 (y+1) = 0 \\
    & \Longleftrightarrow 14x +84 -9y -9 = 0 \\
    & \Longleftrightarrow 14x -9y +75 = 0 \\
 \end{aligned}  \end{equation*}
\end{minipage}}  \\

Donc $\Delta : 14x -9y +75 = 0$ \\

  4.  \raisebox{-16 ex}{\parbox{7.5cm}{
        \begin{tabular}{ll}
\begin{minipage}[t]{4cm}
   \begin{equation*} \left\lbrace \begin{aligned}
              8x +3y   &= -13  \quad\mid 1 \\
              2x -15y   &= -61  \quad\mid -4 \\
    \end{aligned} \right.  \end{equation*}
\end{minipage} & \begin{minipage}[t]{3cm} 
             \begin{equation*} \left\lbrace\begin{aligned}
              8x +3y  &= -13 \quad\mid 5\\
              2x -15y &= -61  \quad\mid 1\\
                  \end{aligned}\right.  \end{equation*}               
\end{minipage}\\
\begin{minipage}[t]{4cm}
   \begin{equation*} \left\lbrace \begin{aligned}
              8x +3y   &= -13 \\
              -8x +60y &= 244 \\
    \end{aligned} \right.  \end{equation*}
\end{minipage} & \begin{minipage}[t]{3cm} 
             \begin{equation*} \left\lbrace\begin{aligned}
              40x +15y  &= -65 \\
              2x -15y &= -61\\
                  \end{aligned}\right.  \end{equation*}               
\end{minipage}\\
      &      \\
\begin{minipage}[b]{4cm}
   \begin{equation*}  \begin{aligned}
              63y   &= 231 \\
                    &       \\
              y  &= \dfrac{231}{63} \\ 
                    &     \\
              y  &= \dfrac{11}{3} \\      
   \end{aligned} \end{equation*}
\end{minipage} & \begin{minipage}[b]{3cm} 
             \begin{equation*} \begin{aligned}
              42x &= -126\\
                  &      \\
              x  &= \dfrac{-126}{42} \\ 
                 &      \\
              x  &= -3 \\   
                  \end{aligned} \end{equation*}               
\end{minipage}\\
\end{tabular} \\  
}}\\

Vérifions que $I \in \Delta$\\

$14 \times (-3) -9 \times \dfrac{11}{3} +75 = 42 -33 +75=0$\\
Donc $I(-3, \dfrac{11}{3}) $

\newpage

         
\subsection{Droites remarquables}         

Soit $(O, \vec{i}, \vec{j})$ un repère.

\subsubsection{Droites parallèles à l'axe des abscisses}

Soit $D$ une droite parallèle à l'axe des abscisses et $M_0(x_0,y_0)$\\
Un vecteur directeur de $D$ est 
$\vec{i} \left( \begin{array}{c}
                      1\\
                      0
               \end{array} 
         \right)$
         
\begin{minipage}[t]{6cm}
\begin{equation*}  \begin{aligned}
M(x,y) \in \Delta  &\Longleftrightarrow \overrightarrow{M_0M} \textrm{ et } \vec{i} \textrm{ sont colinéaires} \\
   & \Longleftrightarrow \textrm{Det }  (\overrightarrow{M_0M},\vec{i}) = 0 \\
   & \Longleftrightarrow \left|\begin{array}{ll}
x-x_0 & 1\\
y-y_0 & 0
\end{array}\right| = 0 \\
    & \Longleftrightarrow 0(x-x_0) - 1(y-y_0) = 0 \\
    & \Longleftrightarrow -y+y_0 = 0 \\
    & \Longleftrightarrow y=y_0 \\
 \end{aligned}  \end{equation*}
 
\begin{center}
\fcolorbox{black}  {white}{
\vbox{
        $y=y_0$  et $y-y_0=0$\\
\vspace*{.01cm}        
       Forme : $ax+by+c=0$     
\begin{flushleft} avec \end{flushleft} \vspace*{-.75cm}         
             \begin{equation*} \begin{aligned}
              a  &= 0\\
              b  &= 1 \\ 
              c  &= -y_0 \\   
             \end{aligned} \end{equation*}           
   }
}
\end{center}
\end{minipage}

\subsubsection{Droites parallèles à l'axe des ordonnées}

Soit $D$ une droite parallèle à l'axe des ordonnées
et  $M_0(x_0,y_0)$. \\
Un vecteur directeur de $D$ est 
$\vec{j} \left( \begin{array}{c}
                      0\\
                      1
               \end{array} 
         \right)$
              
\begin{minipage}[t]{6cm}
\begin{equation*}  \begin{aligned}
M(x,y) \in \Delta  &\Longleftrightarrow \overrightarrow{M_0M} \textrm{ et } \vec{j} \textrm{ sont colinéaires} \\
   & \Longleftrightarrow \textrm{Det }  (\overrightarrow{M_0M},\vec{j}) = 0 \\
   & \Longleftrightarrow \left|\begin{array}{ll}
x-x_0 & 0\\
y-y_0 & 1
\end{array}\right| = 0 \\
    & \Longleftrightarrow 1(x-x_0) - 0(y-y_0) = 0 \\
    & \Longleftrightarrow x-x_0 = 0 \\
    & \Longleftrightarrow x=x_0 \\
 \end{aligned}  \end{equation*}
 
\begin{center}
\fcolorbox{black}  {white}{
\vbox{
        $x=x_0$  et $x-x_0=0$\\
\vspace*{.01cm}        
       Forme : $ax+by+c=0$\\     
\begin{flushleft} avec \end{flushleft} \vspace*{-.75cm}         
             \begin{equation*} \begin{aligned}
              a  &= 1\\
              b  &= 0 \\ 
              c  &= -x_0 \\   
             \end{aligned} \end{equation*}           
   }
}
\end{center}
\end{minipage}
\newpage 

\vspace*{-1.5cm}

\subsubsection{Droites \underline{ non } parallèles à l'un des axes}
Soit $D$ une droite \underline{ non }   parallèles à l'un des axes.

\smallskip
$D$  : \raisebox{4.5ex}{\begin{minipage}[t]{4cm}
                           \begin{equation*} 
                                 \begin{aligned}
                                ax +by +c &= 0 \\
                                 by       &= -ax -c \\ 
                             y        &=  -\dfrac{a}{b}x -\frac{c}{b} \\   
                                 \end{aligned} 
                           \end{equation*}
                      \end{minipage}}\\
             
\fcolorbox{black}  {white}{
\vbox{\hsize=8.5cm \begin{center}
\smallskip        
       $y=mx+p$ est l'équation réduite de la droite $D$\\     
             \begin{equation*} \begin{aligned}
            \mathrm{avec\;\; } &  m  &= -\dfrac{a}{b}\\
            \mathrm{et\;\;}   &  p  &= -\dfrac{c}{b}\\  
             \end{aligned} \end{equation*}  
             
             $m =$ le coefficient directeur de $D$\\
             $p =$ l'ordonnée à l'origine de $D$         
   \end{center}}}   
   
\smallskip 
%\medskip 
%\bigskip

\underline{Réciproquement :}
\smallskip 

Toute équation de la forme $y=mx+p$, avec $a \neq 0 $ et $b \neq 0$ 
est celle d'une droite $D$  non parallèle à l'un des axes.
\begin{minipage}[t]{4.5cm}
   \begin{equation*} \begin{aligned}
            y        &= mx+p\\
            mx -y +p &= 0 \\  
             \end{aligned} \end{equation*} 

\fcolorbox{black}  {white}{
\vbox{\hsize=5cm \begin{center}
\vspace*{.01cm}        
       Forme : $ax+by+c=0$      
\begin{flushleft} avec \end{flushleft} \vspace*{-.75cm}         
             \begin{equation*} \begin{aligned}
              a  &= m\\
              b  &= -1 \\ 
              c  &= p \\   
             \end{aligned} \end{equation*}                 
   \end{center}}}
\end{minipage}
         
\smallskip 
%\medskip 
%\bigskip 
Un vecteur directeur de $D$ est $\vec{u} \left( \begin{array}{c}
                      1\\
                      m
               \end{array} 
         \right)$\\

     
\begin{tabular}{r@{$\;$}cl}    
Det  & $(\vec{u},\vec{j}) \left|\begin{array}{ll}
1 & 0\\
m & 1
\end{array}\right| = 1 \neq 0$   & \begin{minipage}[t]{10cm}
$\vec{u}$ et $\vec{j}$ ne sont pas colinéaires,\\
                       donc $D$ n'est pas parallèle
                       à l'axe des ordonnées.
\end{minipage}
\end{tabular}  \\

%\smallskip 
%\medskip 
%\bigskip 
              
\begin{minipage}[t]{6cm}
\begin{tabular}{rl}
\underline{Remarque} \hspace{1cm}
  Soit & \parbox[t]{6cm}
{
           $D \mathrm{ : } y=mx+p \qquad
               \vec{u}\left(           
                      \begin{array}{c}1\\m
                      \end{array}\right)$\\ 
                                          
           $D'\mathrm{ : } y=m'x+p' \quad 
   \overrightarrow{u'}\left(
                      \begin{array}{c} 1\\m'
                      \end{array} \right)$
}\\                                  
\end{tabular}
\end{minipage}

\begin{tabular}{lcc}
\parbox{5cm}{
\begin{equation*}  \begin{aligned}
D\; /\!\!/ \;D' &\Longleftrightarrow \vec{u} \textrm{ et } \vec{u'} \textrm{ sont colinéaires} \\
   & \Longleftrightarrow \textrm{Det }  (\vec{u}, \vec{u'}) = 0 \\
   & \Longleftrightarrow \left|\begin{array}{ll}
1 & 1\\
m & m'
\end{array}\right| = 0 \\
    & \Longleftrightarrow m' - m = 0 \\
    & \Longleftrightarrow m = m' \\
 \end{aligned}  \end{equation*} 
 } & 
        \hspace*{2cm}
                          &\parbox{5cm}{
\fcolorbox{black}  {white}{
\vbox{
       $D\; /\!\!/ \;D'$ 
                  
\begin{flushleft} Donc \end{flushleft} \vspace*{-.75cm}         
             \begin{equation*} \begin{aligned}
             D   &: mx+p\\
             D'  &=m'x +p' \\ 
D\; /\!\!/ \;D' &\Longleftrightarrow m=m' 
             \end{aligned} \end{equation*}           
   }
}}\\
\end{tabular}

\samepage

\newpage
\vspace{-1cm}
\subsection{Un soupçon d'algorithmique}
\subsubsection{Équation cartésienne d'une droite (AB)}

Soit $(O, \vec{i}, \vec{j})$ un repère.
\smallskip
Soient $A(x_A, y_A)$  et$  B(x_B, y_B)$ avec $ A \neq B $.     

\vspace*{-.2cm}

\begin{minipage}[t]{6cm}
\begin{equation*}  \begin{aligned}
M(x,y) \in (AB)  &\Longleftrightarrow \overrightarrow{AM} \textrm{ et } \overrightarrow{AB} \textrm{ sont colinéaires} \\
   & \Longleftrightarrow\textrm{Det } (\overrightarrow{AB},\overrightarrow{AM}) = 0 \\
   & \Longleftrightarrow \left|\begin{array}{ll}
x-x_A & x_B-x_A\\
y-y_A & y_B-y_A
\end{array}\right| = 0 \\
    & \Longleftrightarrow (x-x_A) (y_B-y_A)- (y-y_A)(x_B-x_A) = 0 \\
    & \Longleftrightarrow xy_B-xy_A -x_Ay_B+x_Ay_A - (yx_B-yx_A-x_By_A+x_Ay_A)= 0 \\
    & \Longleftrightarrow xy_B-xy_A-x_Ay_B+x_Ay_A-yx_B+yx_A+x_By_A-x_Ay_A= 0 \\
    & \Longleftrightarrow xy_B-xy_A-yx_B+yx_A-x_Ay_B+x_By_A 
                         +\cancel{x_AyA}-\cancel{x_Ay_A}= 0 \\
   & \Longleftrightarrow xy_B-xy_A-yx_B+yx_A-x_Ay_B+x_By_A \\  
   & \Longleftrightarrow x\underset{a}{\underbrace{(y_B-y_A)}} +y \underset{b}{\underbrace{(x_A -x_B)}}    -\underset{c}{\underbrace{x_Ay_B+x_By_A}} \\
 \end{aligned}  \end{equation*}
 
\begin{tabular}{rl}
 Forme carthésienne & $ax+by+c=0$ \\
  avec & \parbox[t]{3cm}{ $a=y_B - y_A$\\
                       $b=x_A - x_B $ \\
                       $c=x_By_A - x_Ay_B$\\                        
          } \\
\end{tabular}
 
\begin{tabular}{l|l}
Algorithme                        &  Programme calculatrice \\
\parbox{6cm}{\vspace*{-.75cm}
\ding{43} \underline{entrées}\\ %
    \begin{minipage}{0.5\columnwidth}%          
              \begin{minipage}[t]{\columnwidth}%
                  \hspace{1.5cm}$x_{A},y_{A},x_{B},y_{B}$
             \end{minipage}%
     \end{minipage} \\
             }   & 
\begin{minipage}{0.8\columnwidth}
%    \vspace*{-1cm}      
\fcolorbox{ecranTI}{ecranTI}{\parbox{3cm}
{ \small
\texttt{PROGRAM:EQLINE}\\
\texttt{:Input "XA : ",X}\\
\texttt{:Input "YA : ",Y}\\
\texttt{:Input "XB : ",Z}\\
\texttt{:Input "YB : ",T}
}}
\smallskip
  \end{minipage} \\
\hline
\parbox{6cm}{\medskip
\ding{43} \underline{Traitement}\\ %
    \begin{minipage}{\columnwidth}%    
              %\vspace*{-1cm}       
              \begin{minipage}[t]{\columnwidth}%
              \begin{tabular}{rl}
                 $y_{B}-y_{A}$ & donne la valeur $a$\\%
                 $x_{A}-x_{B}$ & donne la valeur $b$\\%
                 $x_{B}y_{A}-x_Ay_B$ & donne la valeur $c$\\%
              \end{tabular}  
     \end{minipage}
          \end{minipage} \\
            }    & 
\begin{minipage}{0.8\columnwidth}
    %\vspace*{-2cm}      
\fcolorbox{ecranTI}{ecranTI}{\parbox{3cm}
{ \small
\texttt{PROGRAM:EQLINE}\\
\texttt{:T-Y$\rightarrow$A}\\
\texttt{:X-Z$\rightarrow$B}\\
\texttt{:Z*Y-X*T$\rightarrow$C}
}}
\smallskip
  \end{minipage} \\
\hline
\parbox{6cm}{\medskip
\ding{43} \underline{Sorties}\\ %
    \begin{minipage}{\columnwidth}%    
              %\vspace*{-1cm}       
              \begin{minipage}[t]{\columnwidth}%
              \begin{tabular}{ll}
                 Afficher "\texttt{ax + by + c = 0}"\\
                 Afficher "\texttt{a, b, c}"\\
              \end{tabular}  
     \end{minipage}
          \end{minipage} \\
            }    & 
\begin{minipage}{0.8\columnwidth}    
\fcolorbox{ecranTI}{ecranTI}{\parbox{3cm}
{ \small
\texttt{:Disp "AX+BY+C=0"}\\
\texttt{:Disp A,B,C}
}}
  \end{minipage} \\
\end{tabular} \\

\medskip

\begin{tabular}{lr}
\textbf{Exemple n°1} :& $A(-1, -3)$ et $B(2,-1)$\\
Éq. cart. & $2x-3y-7=0$\\
\end{tabular}
\medskip

\begin{tabular}{lrl}
\textbf{Exemple n°2} :& $A(-3, 5)$ et $B(3,-8)$&\\
Équ. cart. &$-13x-12y-57=0$&\\
&$13x+12y+57=0$&\reflectbox{\ding{43}}\\
\end{tabular}
\medskip

\begin{tabular}{lrl}
\textbf{Exemple n°3} :& $A(-3, -10)$ et $B(5,-4)$&\\
Équ. cart. &$6x-8y-62=0$&\\
&$3x-4y-31=0$&\reflectbox{\ding{43}}\\
\end{tabular}
\end{minipage}

\newpage

\subsubsection{Équation réduite d'une droite (AB)}

\begin{tabular}{lc}
$A(x_A, y_A)$ & $A\neq B$ \\
$B(x_B, y_B)$ & et $x_A \neq x_B$\\
\end{tabular}\\

\textbf{Remarque :}

Si $x_A = x_B $, alors la droite $(AB)$ est parallèle à l'axe des ordonnées.

Soit la droite $D$ définie par l'équation réduite $y=mx+p$\\

on a : $\begin{cases}
y_A\!\!\!\!\!\!\!\!&= mx_A + p\\
y_B\!\!\!\!\!\!\!\!&= mx_B +p\\
      \end{cases}$\\
      
\begin{minipage}{6cm}      
  \begin{equation*} 
    \begin{aligned}
      y_B-y_A &= mx_B-mxA\\
      y_B-y_A &= m(x_B-xA)\\
            m &= \begin{tabular}{rl}
                    $y_B-y_A$ & $\longrightarrow$ différences des ordonnées \\
                    \cline{1-1}
                    $x_B-x_A$ & $\longrightarrow$ différences des abscisses \\
                 \end{tabular} \\                     
    \end{aligned} 
 \end{equation*}
\end{minipage}\\

$p=y_A - mx_A$\\

$p=y_A - \dfrac{y_B-y_A}{x_B-x_A} x_A$\\

\begin{tabular}{l|l}
Algorithme                        &  Programme calculatrice \\
\parbox{6cm}{\vspace*{-.75cm}
\ding{43} \underline{entrées}\\ %
    \begin{minipage}{0.5\columnwidth}%          
              \begin{minipage}[t]{\columnwidth}%
                  \hspace{1.5cm}$x_{A},y_{A},x_{B},y_{B}$
             \end{minipage}%
     \end{minipage} \\
             }   & 
\begin{minipage}{0.8\columnwidth}
%    \vspace*{-1cm}      
\fcolorbox{ecranTI}{ecranTI}{\parbox{3cm}
{ \small
\texttt{PROGRAM:EQLINE}\\
\texttt{:Input "XA : ",X}\\
\texttt{:Input "YA : ",Y}\\
\texttt{:Input "XB : ",Z}\\
\texttt{:Input "YB : ",T}
}}
\smallskip
  \end{minipage} \\
\hline
\parbox{6cm}{\medskip
\ding{43} \underline{Traitement}\\ %
    \begin{minipage}{\columnwidth}%       
        \begin{minipage}[t]{\columnwidth}%
        $\dfrac{y_B-y_A}{x_B-x_A}$ donne la valeur de $m$ \\
        
        $yA - \dfrac{y_B-y_A}{x_B-x_A} x_A$ donne la valeur de $p$
        \end{minipage}
     \end{minipage} \\
            }    & 
\begin{minipage}{0.8\columnwidth}
    %\vspace*{-2cm}      
\fcolorbox{ecranTI}{ecranTI}{\parbox{3cm}
{ \small
\texttt{PROGRAM:EQLINE}\\
\texttt{:(T-Y)/(Z-X)$\rightarrow$M}\\
\texttt{:Y-(M)*X$\rightarrow$P}\\
}}
\smallskip
  \end{minipage} \\
\hline
\parbox{6cm}{\medskip
\ding{43} \underline{Sorties}\\ %
    \begin{minipage}{\columnwidth}%    
              %\vspace*{-1cm}       
              \begin{minipage}[t]{\columnwidth}%
                 Afficher "\texttt{y = mx + p}"\\
                 Afficher "\texttt{m} et \texttt{p} "
     \end{minipage}
          \end{minipage} \\
            }    & 
\begin{minipage}{0.8\columnwidth}    
\fcolorbox{ecranTI}{ecranTI}{\parbox{3cm}
{ \small
\texttt{:Disp "Y=MX+P"}\\
\texttt{:Disp M,P}
}}
  \end{minipage} \\
\end{tabular} \\

\newpage



\textbf{Exemple n°1} : $A(-1,2)$ et $B(3,2)$\\

$(AB)$ est parallèle à l'axe des abscisses. Donc :\\

$a=0$ 
$b=-4$
$c=8$ 

Ainsi : 

$0x-4y+8 =  0 $ \\
$-4y+8   =  0 $ \\
$-4y     = -8 $ \\
$y       =  2 $ \\


\begin{tabular}{lr}
Equation réduite : & $m=0$\\
           & $p=2$\\
\end{tabular} \\           

$y=0x+2$\\
$y=2$\\


\textbf{Exemple n°2} : $A(3,1)$ et $B(3,4)$\\

$(AB)$ est parallèle à l'axe des ordonnées. Donc : \\

$a=3$
$b=0$
$c=-9$

Ainsi :

$3x+0y-9 =0$ \\
$-x-9=0$ \\
$3x=9$ \\
$x=3$ \\

Pas d'équation réduite sous la forme habituelle. 

$y=mx+p\qquad \longleftarrow$ ne convient pas. On écrit : $x=3$ \\


\textbf{Équation réduite donnée}

\textbf{Exercice n°1} \\

$y=2x+3$ \\

$A(1,5)$\\
$B(2,7)$\\
$m=2$\\
$p=3$\\

\textbf{Exercice n°2} : $y=\dfrac{2}{3}x-\frac{7}{3}$\\

$A(-1, -3)$ et $B(2,1)$\\

$m=0,66666667 \qquad m=\dfrac{2}{3}$\\
$p=-2,3333334 \qquad p=-\dfrac{7}{3}$\\

\textbf{Exercice \no 2 Un superbe exercice : Étude d'une famille de droites} \\

Soit $(O, \vec{i}, \vec{j})$\\

Soit $n \in \Re$\\  

Soit $D_n $ la droite d'équation cartésienne : $(m - 5)x + (m - 3)y + m + 1 = 0 $ \\

$m$ est un paramètre et de  forme  $ax + by +c = 0$\\

$ \begin{array}{r@{$\;$}l}
a &= m-5\\
b &= m-3\\
c &= m+1\\
\end{array}$

\begin{enumerate}
\item Déterminer les équations de $D_2$ et de $D_4$ ; 
\item Déterminer l'équation réduite de $D_m$ qui est parallèle à l'axe des abscisses ;
\item Déterminer l'équation réduite de $D_m$ qui est parallèle à l'axe des ordonnées ;  
\item Montrer que toutes les droites $D_M$ passent par un point fixe $I$ dont in déterminera les coordonnées.
\end{enumerate}

\begin{enumerate}
\item $D_2 :$\raisebox{-3ex}{$ \begin{array}{r@{$\;$}r@{$\;$}r@{$\;$}l}
             (2-5)x &+ (2-3) y & +2 +1 &= 0 \\
               -3x  &     -y   &    +3 &= 0 \\
                3x  &      +y  &    +3 &= 0 \\
             \end{array}$}\\
$D_4 :$\raisebox{-3ex}{$ \begin{array}{r@{$\;$}r@{$\;$}r@{$\;$}l}
             (4-5)x &+ (4-3) y & +4 +1 &= 0 \\
                -x  &     +y   &    +5 &= 0 \\
                 x  &      -y  &    -5 &= 0 \\
             \end{array}$}\\
             
\item Si $D_m$ est paralèle à l'axe des abscisses, alors on a $ax+by +c =0$ avec $a=0$\\

pour que $a = 0$ on a $m = 5$ \\

\begin{minipage}{9.5cm}
$D_5 :$\raisebox{-1.5ex}{$ \begin{array}{r@{$\;$}r@{$\;$}r@{$\;$}l}
             (5-5)x &+ (5-3) y & +5 +1 &= 0 \\
                    &    +2y   &    +6 &= 0 \\
             \end{array}$}
\begin{center}
\fcolorbox{black}  {white}{$y=3$}
\end{center}
\end{minipage}\\

         
Si $D_m$ est paralèle à l'axe des ordonnées, alors on a $ax+by +c =0$ avec $b=0$\\

pour que $b = 0$ on a $m = 3$ \\

\begin{minipage}{9.5cm}
$D_3 :$\raisebox{-3ex}{$ \begin{array}{r@{$\;$}r@{$\;$}r@{$\;$}l}
             (3-5)x &+ (3-3) y & +3 +1 &= 0 \\
              -2x   &          &    +4 &= 0 \\
              -2x   &         &        &= -4 \\
             \end{array}$}           
\begin{center}
\fcolorbox{black}  {white}{$x=2$}
\end{center}
\end{minipage}\\

\newpage

\begin{tabular}{l}
{\begin{tikzpicture}[line cap=round,line join=round,>=triangle 45,scale=1.02]
\draw[->,color=black] (-6,0) -- (10,0);
\foreach \x in {-6,-5,-4,-3,-2,-1,1,2,3,4,5,6,7,8,9}
\draw[shift={(\x,0)},color=black] (0pt,2pt) -- (0pt,-2pt);
\draw[->,color=black] (0,-7.33) -- (0,7.33);
\foreach \y in {-7,-6,-5,-4,-3,-2,-1,1,2,3,4,5,6,7}
\draw[shift={(0,\y)},color=black] (2pt,0pt) -- (-2pt,0pt);
\clip(-6,-7.33) rectangle (10,7.33);
\draw (2,-7.33) -- (2,7.33);
\draw [domain=-6:10] plot(\x,{(-3-0*\x)/1});
\draw [domain=-6:10] plot(\x,{(--3-3*\x)/1});
\draw [domain=-6:10] plot(\x,{(--5-1*\x)/-1});
\draw [->] (0,0) -- (1,0);
\draw [->] (0,0) -- (0,1);
\draw (3.5,-5.3) node {$(D_2)$};
\draw (1.5,4) node{$(D_3)$};
\draw (7,-2.5) node {$(D_5)$};
\draw (8,4) node{$(D_4)$};
\draw (-0.2,-0.3) node {$O$};

\draw[color=black] (0.5,-0.3) node {$\vec{i}$};
\draw[color=black] (-0.3,0.5) node {$\vec{j}$};
\fill [color=black] (3,-6) circle (1.5pt);
\draw[color=black] (3,-6) node [left]{$B_2$};
\fill [color=black] (3,-2) circle (1.5pt);
\draw[color=black] (3,-2) node [right] {$B_4$};
\fill [color=black] (0,3) circle (1.5pt);
\draw[color=black] (0,3) node [left]{$A_2$};
\fill [color=black] (0,-5) circle (1.5pt);
\draw[color=black] (0,-5) node [left]{$A_4$};

\end{tikzpicture}} \\
\end{tabular}

\newpage

\item \raisebox{-2.5ex}{\begin{minipage}{7cm}
\begin{align*}
I(x,y) \in D_n  &\Longleftrightarrow  ax+by+c=0 \\ 
&\Longleftrightarrow (m-5)x+(m-3)y+m+1=0\\
\end{align*}
\end{minipage}}\\

        \begin{enumerate}
          \item [1{)}] \raisebox{-3ex} { \begin{minipage}{11.2cm}
$D_2 :$\raisebox{-3ex}{$ \begin{array}{r@{$\;$}r@{$\;$}r@{$\;$}l}
             (2-5)x &+ (2-3) y & +2 +1 &= 0 \\
              -3x   &    -y   &    +3 &= 0 \\
               3x   &    +y   &    -3  &= 0 \\
             \end{array}$}           
        \end{minipage}}
        
        \begin{minipage}{11.2cm}   
$D_4 :$\raisebox{-3ex}{$ \begin{array}{r@{$\;$}r@{$\;$}r@{$\;$}l}
             (4-5)x &+ (4-3) y & +4 +1 &= 0 \\
              -x    &    +y   &    +5 &= 0 \\
               x   &    -y   &    -5  &= 0 \\
             \end{array}$}           
\end{minipage}\\
        
\underline{Eq red : } \raisebox{-3ex}{
\parbox{10cm}{
$y  = -3x + 3  \textrm { pour } D_2 \quad A_2(0,3) \textrm { et } B_2(2,-6)$\\

$ y = x -  3  \textrm { pour } D_4 \quad A_4(0,-5) \textrm { et } B_2(3,-2)$            
           } }\\          

\item [2{)}]  Si $D_m$ est parallèle à l'axe des abscisses, alors on a $ax+by+c=0$ avec $a=0$ 

Pour que $a=0$ il faut $m=5$\\

\begin{minipage}{9.8cm}
$D_5 :$\raisebox{-4.5ex}{$ \begin{array}{r@{$\;$}r@{$\;$}r@{$\;$}l}
             (5-5)x &+ (5-3) y & +5 +1 &= 0 \\
                &    2y   &    +6 &= 0 \\
                &    2y   &     &= -6 \\               
             \end{array}$}    
\begin{center}
\fcolorbox{black}  {white}{$y=-3$}
\end{center}       
        \end{minipage}\\
        
Si $D_m$ est parallèle à l'axe des ordonnées, alors on a $ax+by+c=0$ avec $b=0$ 

Pour que $b=0$ il faut $m=3$ \\

\begin{minipage}{9.6cm}
$D_3 :$\raisebox{-4.5ex}{$ \begin{array}{r@{$\;$}r@{$\;$}r@{$\;$}l}
             (3-5)x &+ (3-3) y & +3 +1 &= 0 \\
              -2x  &      &    +4 &= 0 \\
                -2x &     &     &= -4 \\               
                  x &       &     &= 2 \\
             \end{array}$}  
\begin{center}             
\fcolorbox{black}  {white}{$x=2$}
\end{center}                          
        \end{minipage}        
\end{enumerate} 

\item $D_3$ et $D_5$ sont sécantes au point $I(2,-3)$

Montrons que $I\in$ à toutes les droites $D_m$ \\

$(m-5)\times 2 + (m-3) \times -3 + m +1 $\\
$= 2m -10 -3m +9 +m +1 $\\
$= 0$ \\ 
     
\end{enumerate}
