\ifdefined\COMPLETE
\else
    \input{./preambule-sacha-utf8.ltx}
    \begin{document}
\fi

\section{Fluctuation d'échantillonnage}

\subsection{Dans les statistiques}

On conçoit que si on lance un dé à 6 faces, non truqué, chacun des 6 numéros $1, 2, 3, 4, 5$  et $ 6$,  a environ une chance sur 6 de sortir.

Dans cette activité, nous nous proposons de comparer les résultats fournis par plusieurs séries de lancers, puis de voir si, lorsque le nombre de lancers est « grand », la fréquence d'apparition de chacun de ces 6 nombres est approximativement égale à $\dfrac{1}{6}$. 

\subsubsection*{Première série}

Effectuez 50 lancers d'un dé à 6 faces et remplissez le tableau ci-dessous : \\

\begin{tabular}{|l|ccc|ccc|ccc|ccc|ccc|ccc|ccc|}
\hline
Valeurs & 1 & & & 2 & & & 3 & & & 4 & & & 5 & & & 6 & & & $\Sigma$ & &  \\
\hline
Effectifs & 6 & & & 8 & & & 11 & & & 8 & & & 8 & & & 9 & & & 50 & & \\
\hline
Fréquences & 0,12 & & & 0,16 & & & 0,22 & & & 0,16 & & & 0,16 & & & 0,18 & & & 1 & & \\
\hline
\end{tabular}

\vspace*{.3cm}

Par exemple, si le 1 est sorti 8 fois, on dira que la fréquance est égale à $\dfrac{8}{50}$, c'est-à-dire 0,16 ou encore, en pourcentage, 16 \%. \\

Construisez le diagramme en bâtons des fréquences. On choisira un centimètre pour 10 \% sur l'axe des ordonnées.

%Inclure le premier dessin.

\begin{center}

\begin{tikzpicture}[scale=.84]
    \tkzInit[xmax=6,ymax=.24, ystep=.02]

    \tkzGrid[color=AntiqueWhite]
    \tkzDrawX[label=$~$]
    \tkzLabelX[orig=false, color=black ]
   \tkzDrawY[label=$~$]
%     \tkzLabelY[orig=false]
    \tkzLabelY
    \tkzBardiagram[wd = 0.1,
                   pos = {below,outer sep = 5pt},
                   sp = 1,
                   noval]           
      { /.12, /.16, /.22, /.16, /.16, /.18}
     \draw (3,14) node{\large Diagramme en bâtons des fréquences} ;  
\end{tikzpicture}

\end{center}

\newpage

\subsubsection*{Deuxième série}

Effectuez une autre série de 50 lancers, en utilisant votre calculatrice. \\

Avec la calculatrice, on obtient un nombre aléatoire supérieur ou égal à 0, et strictement inférieur à 1, en appuyant sur les touches suivantes : 

\begin{itemize}
\item[*] Sur CASIO : (OPT PROB) \textbf{Ran\#}
\item[*] Sur TI : (MATH PRB) \textbf{rand} (ou rand0 sur TI 89-90).
\end{itemize}

\vspace*{.3cm}

Et la partie entière d'un nombre est donnée par la fonction :

\begin{itemize}
\item[*] Sur CASIO : (OPT NUM) \textbf{Int}
\item[*] Sur TI : (MATH NUM) \textbf{rint}.
\end{itemize}

\vspace*{.3cm}

On a ainsi un moyen d'obtenir un nombre aléatoire entier de 1 à 6 avec la calculatrice, en tapant : \textbf{Int(6rand) + 1}. \\

Maintenant, effectuez une série de lancers de dé avec votre calculatrice, et complétez le tableau suivant :

\begin{tabular}{|l|ccc|ccc|ccc|ccc|ccc|ccc|ccc|}
\hline
Valeurs & 1 & & & 2 & & & 3 & & & 4 & & & 5 & & & 6 & & & $\Sigma$ & &  \\
\hline
Effectifs & 12 & & & 11 & & & 6 & & & 3 & & & 5 & & & 13 & & & 50 & & \\
\hline
Fréquences & 0,24 & & & 0,22 & & & 0,12 & & & 0,06 & & & 0,10 & & & 0,26 & & & 1 & & \\
\hline
\end{tabular}

\vspace*{.3cm}

Construisez le diagramme en bâtons des fréquences.

\begin{center}

\begin{tikzpicture}[scale=.77]
    \tkzInit[xmax=7,ymax=.28, ystep=.02]

    \tkzGrid[color=AntiqueWhite]
    \tkzDrawX[label=$~$]
    \tkzLabelX[orig=false, color=black ]
   \tkzDrawY[label=$~$]
%     \tkzLabelY[orig=false]
    \tkzLabelY
    \tkzBardiagram[wd = 0.1,
                   pos = {below,outer sep = 5pt},
                   sp = 1,
                   noval]           
      { /.24, /.22, /.12, /.06, /.1, /.26}
     \draw (3,16) node{\large Diagramme en bâtons des fréquences} ;  
\end{tikzpicture}

\end{center}

\newpage

\subsubsection*{Troisième série}

Regroupez les résultats des deux séries de lancers, puis complétez le tableau associé à cette nouvelle série de 100 lancers, puis le diagramme en bâtons associé.\\

\begin{tabular}{|l|ccc|ccc|ccc|ccc|ccc|ccc|ccc|}
\hline
Valeurs & 1 & & & 2 & & & 3 & & & 4 & & & 5 & & & 6 & & & $\Sigma$ & &  \\
\hline
Effectifs & 18 & & & 19 & & & 17 & & & 11 & & & 13 & & & 22 & & & 100 & & \\
\hline
Fréquences & 0,18 & & & 0,19 & & & 0,17 & & & 0,11 & & & 0,13 & & & 0,22 & & & 1 & & \\
\hline
\end{tabular}

\begin{center}

\begin{tikzpicture}[scale=1]
    \tkzInit[xmax=7,ymax=.22, ystep=.02]

    \tkzGrid[color=AntiqueWhite]
    \tkzDrawX[label=$~$]
    \tkzLabelX[orig=false, color=black ]
   \tkzDrawY[label=$~$]
%     \tkzLabelY[orig=false]
    \tkzLabelY
    \tkzBardiagram[wd = 0.1,
                   pos = {below,outer sep = 5pt},
                   sp = 1,
                   noval]           
      { /.18, /.19, /.17, /.11, /.13, /.22}
     \draw (3,13) node{\large Diagramme en bâtons des fréquences} ;  
\end{tikzpicture}

\end{center}

\newpage

\subsubsection*{Conclusion}

Les fréquences obtenues lors de ce trois séries de lancers sont différentes. Vous observez que pour chacun des nombres la distribution des fréquences \textbf{fluctue}. \\Ce phénomène est appelé \underline{fluctuation d'échantillonnage}. \\

Regroupons les résultats obtenus par groupe de 4 échantillons de la classe : \\

\centerline{
\begin{tabular}{|l|ccc|ccc|ccc|ccc|ccc|ccc|ccc|}
\hline
Valeurs & 1 & & & 2 & & & 3 & & & 4 & & & 5 & & & 6 & & & $\Sigma$ & &  \\
\hline
Fréquences de l'échantillon 1 & 140 & & & 108 & & & 109 & & & 118 & & & 102 & & & 123 & & & 700 & & \\
\hline
Fréquences de l'échantillon 2 & 135 & & & 131 & & & 106 & & & 102 & & & 98 & & & 128 & & & 700 & & \\
\hline
Fréquences de l'échantillon 3 & 122 & & & 131 & & & 119 & & & 107 & & & 103 & & & 118 & & & 700 & & \\
\hline
Fréquences de l'échantillon 4 & 119 & & & 129 & & & 117 & & & 109 & & & 103 & & & 123 & & & 700 & & \\
\hline
Totaux & 516 & & & 499 & & & 451 & & & 436 & & & 406 & & & 492 & & & 2800 & & \\
\hline
Fréquences & 0,18 & & & 0,18 & & & 0,16 & & & 0,16 & & & 0,15 & & & 0,18 & & & 1,01 & & \\
\hline
\end{tabular}}

\vspace*{.3cm}

On constate que $p \approx \dfrac{1}{6} \approx 0,17$. \\

\textbf{Remarque :} $\Sigma_{f_i} = 1$. Ici, il y a un problème d'arrondi. 

\begin{center}
\begin{tikzpicture}[scale=1]
    \tkzInit[xmax=7,ymax=.20, ystep=.02]

    \tkzGrid[color=AntiqueWhite]
    \tkzDrawX[label=$~$]
    \tkzLabelX[orig=false, color=black ]
   \tkzDrawY[label=$~$]
%     \tkzLabelY[orig=false]
    \tkzLabelY
    \tkzBardiagram[wd = 0.1,
                   pos = {below,outer sep = 5pt},
                   sp = 1,
                   noval]           
      { /.18, /.18, /.16, /.16, /.15, /.18}
     \draw (3,11) node{\large Diagramme en bâtons des fréquences} ;  
\end{tikzpicture}
\end{center}

\newpage

\subsection{Intervalle de fluctuation d'une fréquence au seuil de 95  \%}

Soit un caractère dont la proportion dans une population donnée est $p$.  

On considère un échantillon de taille $n$. \\

Si $0,2\leq p \leq 0,8$ et si $n \geq 25$, alors dans au moins 95  \% des cas, \\
la fréquence appartient à l'intervalle $I = \left[p - \dfrac{1}{\sqrt{n}} ; p + \dfrac{1}{\sqrt{n}}\right]$.

\subsubsection{Exemple \no 1}

On considère dans une rivière une population de truites. \\ La proportion mâles-femelles est de 0,5 pour chaque sexe. On prélève 100 truites. \\
On constate que la proportion de femelles dans l'échantillon est de 0,64. \\
Au seuil de 95  \%, peut-on suspecter une anomalie ? \\

On note $p$ la proportion de femelles dans la rivière. $p = 0,5$, on a bien $0,2\leq p \leq 0,8$. \\

On note $n$ la taille de l'échantillon prélevé. $n = 100$, et on a bien $n \geq 25$. \\

$ p - \dfrac{1}{\sqrt{n}} = 0, 5- \dfrac{1}{\sqrt{100}} = 0,4 $ \\

$ p + \dfrac{1}{\sqrt{n}} = 0, 5 + \dfrac{1}{\sqrt{100}} = 0,6 $ \\

Ainsi, $ I \left[0,4 ; 0,6 \right]$ \\

Donc $f = 0,64$, et $f \notin I $ \\

La fréquence de 0,64 n'est pas due à une fluctuation. On peut suspecter une anomalie, par exemple une pollution. \\

N.B. : Le nombre total de truites dans la rivière est inconnu, mais nous est indifférent. \\

\newpage

\subsubsection{Exemple \no 2}

Dans une commune de 50 000 habitants, il y a 25 500 hommes. \\ Au conseil municipal, composé de 43 élus, il y a 17 femmes. \\ Au seuil de 95  \%, peut-on considérer que la parité hommes / femmes est respectée ? \\

On note $p$ la proportion de femmes dans la commune. On a $p = \dfrac{25500}{50 000} = 0,51$. On a bien $0,2\leq p \leq 0,8$. \\

On note $n$ la taille de l'échantillon (conseil municipal). Avec $n = 43$, on a bien $n \geq 25$. \\

$ p - \dfrac{1}{\sqrt{n}} = 0, 51- \dfrac{1}{\sqrt{43}} $ \\

$ p + \dfrac{1}{\sqrt{n}} = 0, 51 + \dfrac{1}{\sqrt{43}}  $ \\

$ f = \dfrac{17}{43} $. \\

On a $ 0,51 - \dfrac{1}{\sqrt{43}} \leq \dfrac{17}{48} \leq 0,51 + \dfrac{1}{\sqrt{43}} $, car : \\

\begin{itemize}
\item[*] $ \left(0,51 + \dfrac{1}{\sqrt{43}}\right) - \dfrac{17}{43} > 0  $
\item[*] $ \left(0,51 - \dfrac{1}{\sqrt{43}}\right) - \dfrac{17}{43} < 0 $
\end{itemize}

\vspace{.3cm}

Ainsi, On a $ 0,51 - \dfrac{1}{\sqrt{43}} \leq f \leq 0,51 + \dfrac{1}{\sqrt{43}} $. \\

Au seuil de 95  \%, on peut considérer qu'une fréquence égale à $\dfrac{17}{43}$ est due à une fluctuation d'échantillonnage. La parité au conseil général est donc respectée. 

\newpage 

\subsubsection{Exemple \no 3}

Lors d'une élection, un sondage portant sur 1 000 personnes donne 400 votants pour le candidat A. \\ Avec un risque d'erreur de 5  \%, quelles informations peut-on obtenir sur la proportion réelle de votants pour A ? \\

Cette fois, on cherche $p$. \\

On a $n = 1000 $, avec $ n \geq 25 $. \\

$ f = \dfrac{400}{1000} = 0,4 $ \\

On doit avoir $ p - \dfrac{1}{\sqrt{n}} \leq f \leq p + \dfrac{1}{\sqrt{n}} $.

\vspace*{.3cm}

\begin{tabular}{ll}
$ p - \dfrac{1}{\sqrt{n}} \leq f $ & $ p + \dfrac{1}{\sqrt{n}} \geq f $ \\
$ p \leq f + \dfrac{1}{\sqrt{n}}$ & $ p \geq f - \dfrac{1}{\sqrt{n}}$ \\
\end{tabular}

\vspace{.3cm}

Donc $ f - \dfrac{1}{\sqrt{n}} \leq p \leq f + \dfrac{1}{\sqrt{n}} $ \\

Ainsi : $ 0,4 - \dfrac{1}{\sqrt{1000}} \leq p \leq 0,4 + \dfrac{1}{\sqrt{1000}} $ \\

$ 0,4 - \dfrac{1}{\sqrt{1000}} \approx 0,3684 $ \\

$ 0,4 + \dfrac{1}{\sqrt{1000}} \approx 0,4316 $ \\

$ 0,3683 \leq p \leq 0,4316 $ \\

A va obtenir entre 37  \% et 43  \% des voies. Il ne sera donc pas élu.

\newpage 

\subsubsection{Exemple \no 4}

Lors d'une élection, un sondage portant sur 900 personnes donne 459 votants pour le candidat A. \\ Avec un risque d'erreur de 5  \%, quelles informations peut-on obtenir sur la proportions réelle de votants pour A ? \\

On cherche $p$. \\

On a $n = 900 $, avec $ n \geq 25 $. \\

$ f = \dfrac{459}{900} = 0,51 $ \\

On doit avoir $ p - \dfrac{1}{\sqrt{n}} \leq f \leq p + \dfrac{1}{\sqrt{n}} $.

\vspace*{.3cm}

\begin{tabular}{ll}
$ p - \dfrac{1}{\sqrt{n}} \leq f $ & $ p + \dfrac{1}{\sqrt{n}} \geq f $ \\
$ p \leq f + \dfrac{1}{\sqrt{n}}$ & $ p \geq f - \dfrac{1}{\sqrt{n}}$ \\
\end{tabular}

\vspace{.3cm}

Donc $ f - \dfrac{1}{\sqrt{n}} \leq p \leq f + \dfrac{1}{\sqrt{n}} $ \\

Ainsi : $ 0,51 - \dfrac{1}{\sqrt{900}} \leq p \leq 0,51 + \dfrac{1}{\sqrt{900}} $ \\

$ 0,51 - \dfrac{1}{\sqrt{900}} \approx 0,4767 $ \\

$ 0,51 + \dfrac{1}{\sqrt{900}} \approx 0,5433 $ \\

$ 0,0,4767 \leq p \leq 0,5433 $ \\

A va obtenir entre 47  \% et 54  \% des voies. On ne peut pas savoir si A sera élu.

\ifdefined\COMPLETE
\else
    \end{document}
\fi