\ifdefined\COMPLETE
\else
    \input{./preambule-sacha-utf8.ltx}
    \begin{document}
\fi

\section{Pourcentages}

\subsection{Exercice \no 0}

Remplir les cases vides... \\

\begin{tabular}{c|c|c|c} 
Prix HT en € & TVA en \% & Prix TT en € & Coefficient multiplicateur \\
\hline
140 & 5,5 & $\textcolor{red}{147,70}$ & $\textcolor{red}{1,055}$ \\
$\textcolor{red}{1500}$& 19,6 & 1794 & $\textcolor{red}{ 1,196}$ \\
1450 & $\textcolor{red}{33}$ & 1928,50 & $\textcolor{red}{1,33}$ \\
\end{tabular} \\

Même consigne... \\

\begin{tabular}{c|c|c|c} 
Ancien prix en € & réduction en \% & Nouveau prix en € & Coefficient multiplicateur \\
\hline
220 & 15 & $\textcolor{red}{187}$ & $\textcolor{red}{0,85}$ \\
$\textcolor{red}{1790}$& 35 & 1163,50 & $\textcolor{red}{0,65}$ \\
12300 & $\textcolor{red}{0,1}$ & 12287,70 & $\textcolor{red}{0,999}$ \\
\end{tabular} \\

Ainsi, on peut dire que : 

\begin{enumerate}
\item[*] Ajouter 12 \% à $p$ revient à dire que : $ p + \dfrac{12}{100}p = p + 0,12p = p \left(1 + 0,12\right) = 1,12 p$.
\item[*] Retrancher 12 \% à $p$ revient à dire que : $ p - \dfrac{12}{100}p = p - 0,12p = p \left(1 - 0,12\right) = 0,88 p$.
\end{enumerate}

Enfin, le pourcentage d'évolution se calcule ainsi : \\

Soient $V_D$ la valeur de départ et $V_A$ la valeur d'arrivée. \\

$V_A = V_D \left(1 + \dfrac{t}{100} \right) $ \\

$ 1 + \dfrac{t}{100} = \dfrac{V_A}{V_D} $ \\

$ \dfrac{t}{100} = \dfrac{V_A}{V_D} - 1 $ \\

$ \dfrac{t}{100} = \dfrac{V_A - V_D}{V_D} $ \\

$ t = \dfrac{V_A - V_D}{V_D} \times 100 $ \\

En reprenant le premier tableau, on peut dire que : \\

$ t = \dfrac{1928,50 - 1450}{1450} \times 100 = 33 $ \\

Le prix a donc augmenté de 33 \%. \\

Puis, avec le second tableau :  \\

$ t = \dfrac{12287,70 - 12300}{12300} \times 100 = -0,1 $ \\

Le prix a donc diminué de 0,1 \%.

\newpage

\subsection{Exercice \no 1}

\subsubsection*{Première partie}

Un prix $p$ augmente de $20$ \% puis de $10$ \%. \\

A-t-il augenté de $30$ \% ? \\

\begin{enumerate}
\item[*] Prix initial : $p$
\item[*] Prix intermédiaire : $1,2p$
\item[*] Prix final : $1,1 \times 1,2p = 1,32 p $. 
\end{enumerate} 

\vspace{.3cm}

$p$ a donc augmenté de 32 \%.

\subsubsection*{Deuxième partie}

Un prix $p$ diminue de $20$ \% puis de $10$ \%. \\

A-t-il diminué de $30$ \% ? 

A-t-il diminué de $32$ \% ? \\

\begin{enumerate}
\item[*] Prix initial : $p$
\item[*] Prix intermédiaire : $0,8p$
\item[*] Prix final : $0,9 \times 0,8p = 0,72 p $. 
\end{enumerate} 

\vspace{.3cm}

$p$ a donc diminué de 28 \%.

\subsubsection*{Troisième Partie}

Un prix $p$ augmente de $10$ \% puis diminue de $10$ \%. \\

$p$ est-il revenu au prix initial ? \\

\begin{enumerate}
\item[*] Prix initial : $p$
\item[*] Prix intermédiaire : $1,1p$
\item[*] Prix final : $0,9 \times 1,1p = 0,99 p $. 
\end{enumerate} 

\vspace{.3cm}

$p$ a donc diminué de $1$ \%.

\newpage

\subsubsection*{Amusette}

Vous achetez un objet (de valeur) chez un commerçant. Celui-ci propose :

\begin{enumerate}
\item[*] Une réduction de $10$ \% sur le prix HT puis l'application de la TVA à $19,6$ \%.
\item[*] Une réduction de $10$ \% sur le prix TTC.
\end{enumerate}

Quelle option doit-on choisir ? \\

\begin{tabular}{l|c|c}
& Option A & Option B \\
\hline
Prix initial & $p$ & $p$ \\
Prix intermédiaire & $0,9p$ & $1,196p$ \\
Prix final & $1,196 \times 0,9p = 1,0764p $ & $1,196 \times 0,9p = 1,0764p $ \\
\end{tabular} 

\vspace*{.3cm}

Le choix est indifférent pour le client. \\

Cependant, le choix n'est pas indifférent au commerçant : \\

\textbf{Exemple}

\begin{tabular}{l|c|c}
& Option A & Option B \\
\hline
Client & $10764$ & $10764$ \\
Commerçant & $9000$ & $8804$ \\
État & $1764$ & $1960$ \\
\end{tabular} 

\vspace*{.3cm}

Le commerçant préfèrera donc l'option A.

\newpage

\subsection{Exercice \no 2}

Un produit A coûte 25 \% plus cher qu'un produit B. \\

Le produit B coûte-t-il 25 \% moins cher que le produit A ? \\

Soient $p_A$ le prix du produit A et $p_B$ le prix du produit B. \\

$p_A = p_B \times 1,25 $ \\

$ p_B = \dfrac{p_A}{1,25} $ \\

$ p_B = p_A \times \dfrac{1}{1,25} $ \\

$ p_B = p_A \times 0,8 $ \\

Donc, le produit B coûte 20 \% moins cher que le pruduit A.

\subsection{Exercice \no 3}

Le taux de TVA dans la restauration était de 19,6 \%. Ce taux a été baissé à 5,5 \%. \\

Les prix ont-ils donc baissé de 14,1 \% ? \\

Soient $p$ le prix HT. Le prix avec l'ancienne TVA est $p_A = 1,196p$ et le prix avec la nouvelle TVA est $p_N = 1,055p$. \\

%Il faut introduire l'accolade en dessous.

On a donc 

$p_A = 1,196p$
$ p_N = 1,055 p$ \\

Ainsi : \\

$p_N = p \times 1,055$ \\

$ p_N = \dfrac{p_A}{1,196} \times 1,055 $ \\

$ p_N = p_A \times \dfrac{1,055}{1,196} $ \\

$ p_N \approx p_A  \times 0,882 $ \\

Les prix auraient donc dû baisser de 11,8 \% \\

\textbf{Vérification} 

On dit que $p = 10$€ \\

$ t = \dfrac{10,55 - 11,96}{11,96} \times 100 = - 11,8$ \%

\newpage

\subsection{Exercice \no 4}

\subsubsection*{Première partie}

On place de l'argent à un taux de 1 \% par mois, le taux est-il de 12 \% par an ? \\

Soient $p$ la somme d'argent initiale, $p'$ la somme d'argent au bout d'un mois, et $p''$ la somme d'argent au bout de 2 mois. \\

\begin{tabular}{ll}
Placement initial & $p$ \\
Placement au bout d'un mois & $p' = 1,01p$ \\
Placement au bout de deux mois & $p'' = p' \times 1,01 = 1,01 \times 1,01 \times p = 1,01^2 \times p $ \\
... & \\
Placement au bout de 12 mois & $ p \times 1,01^{12} = 1,1268p $ \\
\end{tabular}

\vspace{.3cm}

Le placement a donc augmenté de 12,68 \%.

\subsubsection*{Seconde partie}

Un placement au taux de 12 \% par an revient-il à un placement de 1 \% par mois ? \\

Soient $p$ la somme d'argent initiale, $p'$ la somme d'argent au bout d'un an. \\

\begin{tabular}{ll}
Placement initial & $p$ \\
Placement au bout d'un an & $p' = 1,12p$ \\
\end{tabular}

\vspace{.3cm}

\textbf{Une idée géniale}

On peut dire que pour tout $a \geq 0 $, $\sqrt{a} = a^{\frac{1}{2}}$. \\

En effet, $\left(\sqrt{a}\right)^2 = a$, et donc $\left(a^{\frac{1}{2}}\right)^2 = a^{\frac{1}{2} \times 2} = a$. \\

Par exemple : $\sqrt{9} = 9^{\frac{1}{2}} = 3$ \\

On peut dire ici que la placement au bout d'un mois est : $p \times 1,12^{\frac{1}{12}} = p \times 1,0095 $. \\

Le placement revient donc à 0,95 \% par an.

\newpage

\subsection{Exercice \no 5}

\subsubsection*{Première partie}

Le prix d'un objet augmente de 20 \% la première année puis de 10 \% la deuxième. \\

Le prix a-t-il augmenté en moyenne de 15 \% ? \\

Soit $p$ le prix initial. On cherche $t$ tel que : \\

$p\left(1 + \dfrac{t}{100} \right)^2 = p \times 1,32 $ \\

$\left(1 + \dfrac{t}{100} \right)^2 = 1,32 $ \\

\begin{tabular}{lcl}
$1 + \dfrac{t}{100} = \sqrt{1,32} $ & ou & $1 + \dfrac{t}{100} = -\sqrt{1,32} $ \\
\end{tabular}

\vspace{.3cm}

On sait que : \\

$ 0 \leq t \leq 100 $ \\

$ 0 \leq \dfrac{t}{100} \leq 1 $ \\

$ 1 \leq 1 + \dfrac{t}{100} \leq 2 $ \\

Ainsi : \\

$1 + \dfrac{t}{100} = \sqrt{1,32} $ \\

$\dfrac{t}{100} = \sqrt{1,32} - 1 $ \\

$ t = \left(\sqrt{1,32} - 1 \right) \times 100 $ \\

$ t = 14,89 $ \\

Donc, une augmentation de 14,89 \% suivie d'une augmentation de 14,89 \% revient à une augmentation de 20 \% suivie d'une augmentation de 10 \%. 

\newpage

\subsubsection*{Seconde Partie}

Le prix d'un objet diminue de 20 \% la première année puis de 10 \% la deuxième. \\

Le prix a-t-il diminué en moyenne de 15 \% ? 

Le prix a-t-il diminué en moyenne de 14,89 \% ? \\

Soit $p$ le prix initial. On cherche $t$ tel que : \\

$p\left(1 + \dfrac{t}{100} \right)^2 = p \times 0,72 $ \\

$\left(1 + \dfrac{t}{100} \right)^2 = 0,72 $ \\

\begin{tabular}{lcl}
$1 + \dfrac{t}{100} = \sqrt{0,72} $ & ou & $1 + \dfrac{t}{100} = -\sqrt{0,72} $ \\
\end{tabular}

\vspace{.3cm}

On sait que : \\

$ 0 \leq t \leq 100 $ \\

$ 0 \leq \dfrac{t}{100} \leq 1 $ \\

$ 1 \leq 1 + \dfrac{t}{100} \leq 2 $ \\

Ainsi : \\

$1 + \dfrac{t}{100} = \sqrt{0,72} $ \\

$\dfrac{t}{100} = \sqrt{0,72} - 1 $ \\

$ t = \left(\sqrt{1,32} - 1 \right) \times 100 $ \\

$ t = -15,15 $ \\

Donc, une diminution de 15,15 \% suivie d'une diminution de 15,15 \% revient à une diminution de 20~\% suivie d'une diminution de 10 \%. 

\ifdefined\COMPLETE
\else
    \end{document}
\fi
