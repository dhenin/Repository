\ifdefined\COMPLETE
\else
    \input{./preambule-sacha-utf8.ltx}
    \begin{document}
\fi


\begin{tikzpicture}[line cap=round,line join=round,>=triangle 45,x=1.0cm,y=1.0cm]
\draw[->] (-1.54,0) -- (3.94,0);
\foreach \x in {-1,1,2,3}
\draw[shift={(\x,0)}] (0pt,2pt) -- (0pt,-2pt);
\draw[->] (0,-2.14) -- (0,3.37);
\foreach \y in {-2,-1,1,2,3}
\draw[shift={(0,\y)}] (2pt,0pt) -- (-2pt,0pt);
\clip(-1.54,-2.14) rectangle (3.94,3.37);
\begin{pgfonlayer}{background}  
% Attention l'ordre de ces lignes est important 
% Ne pas le modifier   
\draw[step=1mm,ultra thin,AntiqueWhite!10](-1.54,-2.14)  grid (3.94,3.37);
\draw[step=5mm,very thin,AntiqueWhite!30] (-1.54,-2.14)  grid (3.94,3.37);
\draw[step=1cm,very thin,AntiqueWhite!50] (-1.54,-2.14)  grid (3.94,3.37);
\draw[step=5cm,thin,AntiqueWhite]         (-1.54,-2.14)  grid (3.94,3.37);

\end{pgfonlayer} 
\draw (0,0 ) node[below left] {O};
\draw[smooth,samples=100,domain=-0.9999990589764385:2.9999994338555847] plot(\x,{sqrt(-(\x)^2+2*(\x)+3)});
\draw [dash pattern=on 3pt off 3pt] (1,-2.14) -- (1,3.37);



\fill  (-1,0) circle (1.5pt);
\fill  (-0.75,0.97) circle (1.5pt);
\fill  (-0.5,1.32) circle (1.5pt);
\fill  (-0.25,1.56) circle (1.5pt);
\fill  (0,1.73) circle (1.5pt);
\fill  (0.25,1.85) circle (1.5pt);
\fill  (0.5,1.94) circle (1.5pt);
\fill  (0.75,1.98) circle (1.5pt);
\fill  (1,2) circle (1.5pt);
\fill  (1.25,1.98) circle (1.5pt);
\fill  (1.5,1.94) circle (1.5pt);
\fill  (1.75,1.85) circle (1.5pt);
\fill  (2,1.73) circle (1.5pt);
\fill  (2.25,1.56) circle (1.5pt);
\fill  (2.5,1.32) circle (1.5pt);
\fill  (2.75,0.97) circle (1.5pt);
\fill  (3,0) circle (1.5pt);
% \draw  (-1.07,-0.49)-- ++(-1.5pt,-1.5pt) -- ++(3.0pt,3.0pt) ++(-3.0pt,0) -- ++(3.0pt,-3.0pt);
\draw (-1,0) node [below] {$m_1$};
% \draw  (2.97,-0.32)-- ++(-1.5pt,-1.5pt) -- ++(3.0pt,3.0pt) ++(-3.0pt,0) -- ++(3.0pt,-3.0pt);
\draw (3,0) node [below] {$m_2$};
% \draw  (0.92,2.38)-- ++(-1.5pt,-1.5pt) -- ++(3.0pt,3.0pt) ++(-3.0pt,0) -- ++(3.0pt,-3.0pt);
\draw (1,2) node [above] {$M$};


\end{tikzpicture}

%    \usepackage{variations}

\ifdefined\COMPLETE
\else
    \end{document}
\fi