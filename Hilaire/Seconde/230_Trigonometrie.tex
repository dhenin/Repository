\ifdefined\COMPLETE
\else
    \input{./preambule-sacha-utf8.ltx}
    \begin{document}
\fi


\section{Trigonométrie}
\subsection{Cercle trigonométrique}
On appelle cercle trigonométrique tout cercle de rayon $R=1$ sur lequel on a choisi une origine $I$ et un sens de parcours appelé « sens trigonométrique ». 

%------------- Cerlcle trigonométrique ------------------

\hspace*{2cm}
\begin{tikzpicture}[line cap=round,line join=round,>=triangle 45,x=1.0cm,y=1.0cm,scale=.8]
\clip(-3,-3) rectangle (3,3);
\draw [color=gray] (0,0) circle (2cm);
\draw (0,0)-- (2,0);
\draw [shift={(0,0)}] plot[domain=0.17:1.12,variable=\t]({1*2.35*cos(\t r)+0*2.35*sin(\t r)},{0*2.35*cos(\t r)+1*2.35*sin(\t r)});
\draw [->] (1.05,2.1) -- (1.03,2.11);
\fill (0,0) circle (1.5pt);
\draw (0,0) node [below]{$O$};
\fill (2,0) circle (1.5pt);
\draw(2,0) node [right] {$I$};
% \draw (-1/2, sqrt(3)/2) node [left] {$I'$};
\end{tikzpicture}\raisebox{10ex}{$R=OI=1$}
 
\subsection{Image d'un nombre réel sur le cercle trigonométrique}

Soit $\mathcal{C}$ un cercle trigonométrique.

$\mathcal{P} = 2{\pi}R = 2\pi \text{ puisque } R=1$ \\ 

\begin{itemize}
\item [\ding{87}] À tout nombre réel $x$ correspond un point et un seul de $\mathcal{C}$ appelé « image de $x$ sur $\mathcal{C}$ 
noté $M(x)$.\\

$M(\pi) = I'$ 
\item [\ding{87}] Tout point $M$ de  $\mathcal{C}$ est l'image d'une infinité de nombres réels.

\begin{minipage}{9cm}
\begin{tabbing}
Si \= $M$ est $\qquad\quad$ \= l'image de $x$ sur  $\mathcal{C}$\\
   \>  $M$ est aussi \> l'image de $x+2\pi$ \\
   \>  $M$ est aussi \> l'image de $x+4\pi$ \\  
   \>  $M$ est aussi \> l'image de $x+6\pi$\ldots  \\   
   \>  $M$ est aussi \> l'image de $x+2k\pi \quad k\in \mathbb{Z}$ \\    
\end{tabbing}
\hspace*{5cm}
\end{minipage}
%
%------- Enroulement des réels sur le cercle trigo --------
% Pour tracer les courbes de Bésier : 
% ajouter, 
% 1 - dans les préférences de TexMaker, 
%     dans  la ligne  pdflatex 
%     "--shell-escape  -enable-write18 " avant "%.tex"
% 2 - un lien vers gnuplot dans le répertoire  texbin 
%     cd /usr/texbin/
%     sudo ln -s /opt/local/bin/gnuplot ./gnuplot
%
\raisebox{-30ex}{\begin{tikzpicture}[line cap=round,line join=round,>=triangle 45,x=1.0cm,y=1.0cm, scale=.55]
\draw[->,color=black] (0,-7.1) -- (0,11.85);
\foreach \y in {-6,-4,-2,2,4,6,8,10}
\draw[shift={(0,\y)},color=black] (2pt,0pt) -- (-2pt,0pt);
\clip(-4.19,-7.1) rectangle (3.6,11.85);
\draw(-1,0) circle (1cm);
\draw (-2,0)-- (0,0);
\draw [shift={(1.63,-0.1)}] plot[domain=2.04:3.11,variable=\t]({1*3.63*cos(\t r)+0*3.63*sin(\t r)},{0*3.63*cos(\t r)+1*3.63*sin(\t r)});
\draw[smooth,samples=100,domain=0.0:1.0] plot[parametric] function{(1-t)**3*0.01+3*(1-t)**2*t*(-2.81)+3*(1-t)*t**2*(-2.81)+t**3*(-1),(1-t)**3*4.72+3*(1-t)**2*t*2.77+3*(1-t)*t**2*(-1.53)+t**3*(-1)};
\draw[smooth,samples=100,domain=0.0:1.0] plot[parametric] function{(1-t)**3*0+3*(1-t)**2*t*(-7)+3*(1-t)*t**2*2.23+t**3*0,(1-t)**3*6.28+3*(1-t)**2*t*(-2)+3*(1-t)*t**2*(-4.14)+t**3*0};
\draw [shift={(-0.13,-0.64)}] plot[domain=3.54:4.85,variable=\t]({1*0.94*cos(\t r)+0*0.94*sin(\t r)},{0*0.94*cos(\t r)+1*0.94*sin(\t r)});
\draw [shift={(1.64,5.56)}] plot[domain=1.97:2.71,variable=\t]({1*4.2*cos(\t r)+0*4.2*sin(\t r)},{0*4.2*cos(\t r)+1*4.2*sin(\t r)});
\draw [shift={(0.64,7.45)}] plot[domain=1.75:2.8,variable=\t]({1*3.6*cos(\t r)+0*3.6*sin(\t r)},{0*3.6*cos(\t r)+1*3.6*sin(\t r)});
\draw[smooth,samples=100,domain=0.0:1.0] plot[parametric] function{(1-t)**3*0+3*(1-t)**2*t*(-9.62)+3*(1-t)*t**2*7.32+t**3*(-1),(1-t)**3*7.85+3*(1-t)**2*t*(-8.49)+3*(1-t)*t**2*(-1.24)+t**3*1};
\draw [shift={(0.02,0.37)}] plot[domain=1.59:2.59,variable=\t]({1*1.2*cos(\t r)+0*1.2*sin(\t r)},{0*1.2*cos(\t r)+1*1.2*sin(\t r)});
\draw [shift={(-1.27,-0.26)}] plot[domain=4.29:5.13,variable=\t]({1*3.15*cos(\t r)+0*3.15*sin(\t r)},{0*3.15*cos(\t r)+1*3.15*sin(\t r)});
\draw [shift={(-0.52,-0.36)}] plot[domain=4.13:4.83,variable=\t]({1*4.39*cos(\t r)+0*4.39*sin(\t r)},{0*4.39*cos(\t r)+1*4.39*sin(\t r)});
\draw [shift={(0.18,-1.34)}] plot[domain=3.85:4.68,variable=\t]({1*4.95*cos(\t r)+0*4.95*sin(\t r)},{0*4.95*cos(\t r)+1*4.95*sin(\t r)});
\draw (-1,-1) -- (-1,1) ; 
\draw [->] (-2.73,8.71) -- (-2.75,8.66);
\draw [->] (-2.12,7.42) -- (-2.17,7.31);
\draw [->] (-2.4,-3.2) -- (-2.56,-3.14);
\draw [->] (-2.85,-4.07) -- (-2.92,-4.02);
\draw [->] (-3.51,-4.64) -- (-3.57,-4.57);
\draw [->] (.01,-.02) -- (0,0);
\draw [->] (-0.99,1.01) -- (-1,1);
\draw [->] (-0.98,.995) -- (-1,1);
\draw [->] (-0.99,-1.02) -- (-1,-1);
\draw [->] (-1.01,-1) -- (-1,-1);
\draw [->] (-2,.01) -- (-2,0);
\fill  (0,1.57) circle (1.5pt);
\draw (0,1.57) node [right] {${\pi}/2$};
\fill  (0,3.14) circle (1.5pt);
\draw (0,3.14) node [right]{$\pi$};
\fill  (0,9.42) circle (1.5pt);
\draw (0,9.42) node [right]{$3\pi$};
\fill  (0,11) circle (1.5pt);
\draw (0,11) node [right]{$\dfrac{7\pi}{2}$};
\fill  (0,-3.14) circle (1.5pt);
\draw (0,-3.14) node [right]{$-\pi$};
\fill  (0,-4.71) circle (1.5pt);
\draw (0,-4.71) node [right]{$-\dfrac{3\pi}{2}$};
\fill  (0,-1.57) circle (1.5pt);
\draw (0,-1.57) node [right]{$-\dfrac{\pi}{2}$};
\fill  (0,-6.28) circle (1.5pt);
\draw (0,-6.28) node [right]{$-2\pi$};
\fill  (-1,0) circle (1.5pt);
\draw (-1.28,-0.16) node {$O$};
\fill  (0,4.72) circle (1.5pt);
\draw (0,4.72) node [right]{$\dfrac{3\pi}{2}$};
\fill  (-1,-1) circle (1.5pt);
\draw (-1.1,-1) node [below] {$J'$};
\fill  (0,6.28) circle (1.5pt);
\draw (0,6.28) node [right]{$2\pi$};
\fill  (0,0) circle (1.5pt);
\draw (0,0) node [right]{$I$};
\fill  (-2,0) circle (1.5pt);
\draw (-1.7,0.27) node {$I'$};
\fill  (0,7.85) circle (1.5pt);
\draw (0,7.85) node [right]{$\dfrac{5\pi}{2}$};
\fill  (-1,1) circle (1.5pt);
\draw (-0.87,1.28) node {$J$};
\end{tikzpicture}}

\end{itemize}
\newpage

% ---------- Points remarquable du cercle trigo -----------

\begin{tikzpicture}[line cap=round,line join=round,>=triangle 45,x=1.0cm,y=1.0cm, scale=4]
\clip(-1.4,-1.4) rectangle (1.5,1.2);
\draw(0,0) circle (1cm);
\draw [dash pattern=on 1pt off 1pt,color=gray] (-1,1)-- (-1,-1);
\draw [dash pattern=on 1pt off 1pt,color=gray] (-1,-1)-- (1,-1);
\draw [dash pattern=on 1pt off 1pt,color=gray] (1,-1)-- (1,1);
\draw [dash pattern=on 1pt off 1pt,color=gray] (1,1)-- (-1,1);
\draw (-1,1)-- (1,-1);
\draw (1,1)-- (-1,-1);
\draw (0,1) node[above left] {$J$};
\draw [line width=1.2pt] (0,1)-- (0,-1);
\draw [color=violet](1,0) node[above right] {$M(0)$};
\draw [color=violet](1,0) node[below right] {$M(2\pi)\ldots$};
\draw [color=violet](-1,0) node[below left] {$ M(\pi) $};
\draw [dash pattern=on 1pt off 1pt,color=gray] (-0.5,0.87)-- (0.5,0.87);
\draw [dash pattern=on 1pt off 1pt,color=gray] (0.5,0.87)-- (0.5,-0.87);
\draw [dash pattern=on 1pt off 1pt,color=gray] (0.5,-0.87)-- (-0.5,-0.87);
\draw [dash pattern=on 1pt off 1pt,color=gray] (-0.5,-0.87)-- (-0.5,0.87);
\draw [dash pattern=on 1pt off 1pt,color=gray] (-0.87,0.5)-- (0.87,0.5);
\draw [dash pattern=on 1pt off 1pt,color=gray] (0.87,0.5)-- (0.87,-0.5);
\draw [dash pattern=on 1pt off 1pt,color=gray] (0.87,-0.5)-- (-0.87,-0.5);
\draw [dash pattern=on 1pt off 1pt,color=gray] (-0.87,-0.5)-- (-0.87,0.5);
\draw [line width=1.2pt] (-1,0)-- (1,0);
\begin{scriptsize}
\fill  (0,0) circle (.5pt);
\draw (0,0) node [below left]{$O$};
\fill [color=blue] (0.71,0.71) circle (.5pt);
\draw[color=blue] (0.71,0.71) node[right] {$M(\dfrac{\pi}{4})$};
\fill [color=blue] (-0.71,0.71) circle (.5pt);
\draw[color=blue] (-0.71,0.71) node [left]{$M(\dfrac{3\pi}{4})$};
\fill [color=VertClair] (0.87,0.5) circle (.5pt);
\draw[color=VertClair] (0.87,0.5) node [right] {$M(\dfrac{\pi}{6})$};
\fill [color=darkgreen] (0.5,0.87) circle (.5pt);
\draw[color=darkgreen] (0.5,0.87) node [right]{$M(\dfrac{\pi}{3})$};
\fill [color=violet] (0,1) circle (.5pt);
\draw[color=violet] (0,1) node [above right]{$M(\dfrac{\pi}{2})$};
\fill [color=darkgreen] (-0.5,0.87) circle (.5pt);
\draw[color=darkgreen] (-0.5,0.87) node [left]{$M(\dfrac{2\pi}{3)})$};
\fill [color=VertClair] (-0.87,0.5) circle (.5pt);
\draw[color=VertClair] (-0.87,0.5) node [left]{$M(\dfrac{5\pi}{6})$};
\fill [color=blue] (-1,0) circle (.5pt);
\draw[color=blue] (-1,0) node [above left]{$I'$};
\fill [color=blue] (1,0) circle (.5pt);
\draw[color=blue] (1,0) node [above left]{$I$};
\fill [color=blue] (0,-1) circle (.5pt);
\draw[color=blue] (0,-1) node[below left] {$J'$};
\fill [color=VertClair] (-0.87,-0.5) circle (.5pt);
\draw[color=VertClair] (-.87,-0.5) node[left]{$M(\dfrac{7\pi}{6})$};
\fill [color=blue] (-0.71,-0.71) circle (.5pt);
\draw[color=blue] (-0.71,-0.71) node [left]{$M(\dfrac{5\pi}{4})$};
\fill [color=darkgreen] (-0.5,-0.87) circle (.5pt);
\draw[color=darkgreen] (-0.5,-0.87) node [left]{$M(\dfrac{4\pi}{3})$};
\fill [color=violet] (0,-1) circle (.5pt);
\draw[color=violet] (0,-1) node [below right]{$M(\dfrac{3\pi}{2})$};
\fill [color=darkgreen] (0.5,-0.87) circle (.5pt);
\draw[color=darkgreen] (0.5,-0.87) node [right] {$(\dfrac{5\pi}{3})$};
\fill [color=blue] (0.71,-0.71) circle (.5pt);
\draw[color=blue] (0.71,-0.71) node [right] {$M(\dfrac{7\pi}{4})$};
\fill [color=VertClair] (0.87,-0.5) circle (.5pt);
\draw[color=VertClair](0.87,-0.5)node[right]{$M(\dfrac{11\pi}{6})$};
\end{scriptsize}
\end{tikzpicture}\\

Notion de congruence 

$\dfrac{5\pi}{2} \equiv \dfrac{\pi}{2} \left[2\pi\right] \text{ car } \dfrac{5\pi}{2}-\dfrac{\pi}{2} = 2\pi \longleftarrow \text{ Un tour de cercle.}$\\

$\dfrac{9\pi}{2} \equiv \dfrac{\pi}{2} \left[2\pi\right] \text{ car } \dfrac{9\pi}{2}-\dfrac{\pi}{2} = 4\pi \longleftarrow \text{ Deux tours de cercle.}$\\

$\dfrac{15\pi}{2} \cancel{\equiv} \dfrac{\pi}{2} \left[2\pi\right] \text{ car } \dfrac{15\pi}{2}-\dfrac{\pi}{2} = 7\pi \longleftarrow \dfrac{7}{2}\text{ tours de cercle.}$\\

Exercice : Placer sur un cercle trigonométrique les images de : \\
$\dfrac{223\pi}{6},\quad  \dfrac{252\pi}{4}, \quad \dfrac{431\pi}{3},\quad  \dfrac{1035\pi}{4}, \quad \dfrac{1702\pi}{3}, \quad \dfrac{2015\pi}{6}$ \\

\bigskip 

\begin{tabular}{c@{}c@{}c@{}c@{}c@{}l}
\ding{43} 
   $\qquad\dfrac{223\pi}{6}  $ & 
        $ \equiv \dfrac{7\pi}{6}   $ &
           $ \left[2\pi\right] $ & 
               $\text{ car } \dfrac{223\pi}{6}-\dfrac{7\pi}{6}  $ & 
                  $ = 36\pi   $ &
                    $ \longleftarrow \text{ 18 tours de cercle.}$\\
&&&\\
\ding{43}$\qquad\dfrac{291\pi}{4}   $ & $ \equiv \dfrac{3\pi}{4}   $ & $ \left[2\pi\right] $ & $\text{ car } \dfrac{291\pi}{4}-\dfrac{3\pi}{4}  $ & $ = 72\pi   $ & $ \longleftarrow \text{ 36 tours de cercle.}$\\
&&&\\
\ding{43}$\qquad\dfrac{431\pi}{3}   $ & $ \equiv \dfrac{5\pi}{3}   $ & $ \left[2\pi\right] $ & $\text{ car } \dfrac{231\pi}{3}-\dfrac{5\pi}{3}  $ & $ = 142\pi   $ & $ \longleftarrow \text{ 71 tours de cercle.}$\\
&&&\\
\ding{43}$\qquad\dfrac{1035\pi}{4}   $ & $ \equiv \dfrac{3\pi}{4}   $ & $ \left[2\pi\right] $ & $\text{ car } \dfrac{1035\pi}{4}-\dfrac{3\pi}{4}  $ & $ = 258\pi   $ & $ \longleftarrow \text{ 129 tours de cercle.}$\\
&&&\\
\ding{43}$\qquad\dfrac{1702\pi}{3}   $ & $ \equiv \dfrac{4\pi}{3}   $ & $ \left[2\pi\right] $ & $\text{ car } \dfrac{1702\pi}{3}-\dfrac{4\pi}{3}  $ & $ = 566\pi   $ & $ \longleftarrow \text{ 283 tours de cercle.}$\\
&&&\\
\ding{43}$\qquad\dfrac{2015\pi}{6}   $ & $ \equiv \dfrac{11\pi}{6}   $ & $ \left[2\pi\right] $ & $\text{ car } \dfrac{2015\pi}{6}-\dfrac{11\pi}{6}  $ & $ = 334\pi   $ & $ \longleftarrow \text{ 167 tours de cercle.}$\\
\end{tabular}

\newpage

\subsection{Cosinus et sinus d'un nombre réel}

Soit $\mathcal{C}$ un cercle trigonométrique.\\
Soit $M(x)$ l'image de $x$ sur $\mathcal{C}$.\\
Soit $H$ le projeté orthogonal de $M$ sur $(OI)$.\\
Soit $K$ le projeté orthogonal de $M$ sur $(OJ)$.\\

% ------•---- Projetés des points du cercle trigo -----------


\begin{tikzpicture}[line cap=round,line join=round,>=triangle 45,x=1.0cm,y=1.0cm,scale=.8]
\clip(0.3,1) rectangle (14,12.56);
\draw[line width=0.4pt,color=red] (3,11.44) -- (3.21,11.44) -- (3.21,11.65) -- (3,11.65) -- cycle; 
\draw[line width=0.4pt,color=red] (4.13,10.21) -- (3.92,10.21) -- (3.92,10) -- (4.13,10) -- cycle; 
\draw[line width=0.4pt,color=red] (9.79,11) -- (9.79,10.79) -- (10,10.79) -- (10,11) -- cycle; 
\draw[line width=0.4pt,color=red] (8.48,10) -- (8.48,10.21) -- (8.27,10.21) -- (8.27,10) -- cycle; 
\draw[line width=0.4pt,color=red] (1.99,3.79) -- (2.2,3.79) -- (2.2,4) -- (1.99,4) -- cycle; 
\draw[line width=0.4pt,color=red] (3,2.49) -- (2.79,2.49) -- (2.79,2.27) -- (3,2.27) -- cycle; 
\draw[line width=0.4pt,color=red] (11.51,4) -- (11.51,3.79) -- (11.72,3.79) -- (11.72,4) -- cycle; 
\draw[line width=0.4pt,color=red] (10.21,2.98) -- (10.21,3.19) -- (10,3.19) -- (10,2.98) -- cycle; 
\draw(3,4) circle (2cm);
\draw(10,4) circle (2cm);
\draw(3,10) circle (2cm);
\draw(10,10) circle (2cm);
\draw (1,10)-- (5,10);
\draw (3,12)-- (3,8);
\draw [dash pattern=on 5pt off 5pt,color=red] (4.13,11.65)-- (3,11.65);
\draw [dash pattern=on 5pt off 5pt,color=red] (4.13,11.65)-- (4.13,10);
\draw (4.6,8.66) node[anchor=north west] {\parbox{2.12 cm}{${\color{blue}\cos x=OH}\\{\color{Green}\sin x = OK}$}};
\draw [line width=1.6pt,very thick,color=Green] (3,11.65)-- (3,10);
\draw [line width=1.6pt,very thick,color=blue] (3,10)-- (4.13,10);
\draw (10,12)-- (10,8);
\draw [dash pattern=on 5pt off 5pt,color=red] (8.27,10)-- (8.27,11);
\draw (12,10)-- (8,10);
\draw [dash pattern=on 5pt off 5pt,color=red] (8.27,11)-- (10,11);
\draw (10,6)-- (10,2);
\draw (8,4)-- (12,4);
\draw [dash pattern=on 5pt off 5pt,color=red] (11.72,4)-- (11.72,2.98);
\draw [dash pattern=on 5pt off 5pt,color=red] (11.72,2.98)-- (10,2.98);
\draw (1,4)-- (5,4);
\draw (3,6)-- (3,2);
\draw [dash pattern=on 5pt off 5pt,color=red] (3,2.27)-- (1.99,2.27);
\draw [dash pattern=on 5pt off 5pt,color=red] (1.99,2.27)-- (1.99,4);
\draw (0,0.24) node[anchor=north west] {\parbox{2.12 cm}{${\color{blue}\cos x=OH}\\{\color{Green}\sin x = OK}$}};
\draw (11.68,8.68) node[anchor=north west] {\parbox{2.2 cm}{${\color{blue}\cos x=-OH}\\{\color{Green}\sin x = OK}$}};
\draw (4.62,2.7) node[anchor=north west] {\parbox{2.2 cm}{${\color{blue}\cos x= -OH}\\{\color{Green}\sin x = -OK}$}};
\draw (11.62,2.6) node[anchor=north west] {\parbox{2.16 cm}{${\color{blue}\cos x=OH}\\{\color{Green}\sin x = -OK}$}};
\draw [very thick, color=blue] (8.27,10)-- (10,10);
\draw [very thick,color=Green] (10,10)-- (10,11);
\draw [very thick, color=blue] (3,4)-- (1.99,4);
\draw [very thick,color=Green] (3,4)-- (3,2.27);
\draw [very thick, color=blue] (10,4)-- (11.72,4);
\draw [very thick,color=Green] (10,4)-- (10,2.98);

\fill (3,4) circle (1.5pt);
\draw(3,4) node [below left]{$O$};
\fill (10,4) circle (1.5pt);
\draw(10,4) node [below left]{$O$};
\fill (3,10) circle (1.5pt);
\draw(3,10) node [below left]{$O$};
\fill (10,10) circle (1.5pt);
\draw(10,10) node [below left]{$O$};
\fill (1,10) circle (1.5pt);
\draw(1,10) node [left] {$I'$};
\fill (5,10) circle (1.5pt);
\draw(5.32,10.1) node {$I$};
\fill (3,12) circle (1.5pt);
\draw(3,12) node [above]{$J$};
\fill (3,8) circle (1.5pt);
\draw(3,8) node [below] {$J'$};
\fill [color=red] (4.13,11.65) circle (1.5pt);
\draw[color=red] (4.13,11.65) node [right] {$M(x)$};
\fill [color=red] (3,11.65) circle (1.5pt);
\draw[color=red] (3,11.65) node [left] {$K$};
\fill [color=red] (4.13,10) circle (1.5pt);
\draw[color=red] (4.13,10) node [below] {$H$};
\fill [color=red] (8.27,11) circle (1.5pt);
\draw[color=red] (8.27,11) node [left]{$M(x)$};
\fill [color=red] (1.99,2.27) circle (1.5pt);
\draw[color=red] (1.99,2.27) node [left]{$M(x)$};
\fill [color=red] (11.72,2.98) circle (1.5pt);
\draw[color=red] (11.72,2.98) node [right]{$M(x)$};
\fill [color=red] (10,11) circle (1.5pt);
\draw[color=red] (10,11) node [right]{$K$};
\fill [color=red] (8.27,10) circle (1.5pt);
\draw[color=red] (8.27,10) node [below]{$H$};
\fill [color=red] (10,2.98) circle (1.5pt);
\draw[color=red] (10,2.98) node [left]{$K$};
\fill [color=red] (11.72,4) circle (1.5pt);
\draw[color=red] (11.72,4) node [above]{$H$};
\fill [color=red] (3,2.27) circle (1.5pt);
\draw[color=red] (3,2.27) node [right]{$K$};
\fill [color=red] (1.99,4) circle (1.5pt);
\draw[color=red] (1.99,4) node [above]{$H$};
\fill (8,10) circle (1.5pt);
\draw(8,10)  node [left] {$I'$};
\fill (12,10) circle (1.5pt);
\draw(12,10) node [right] {$I$};
\fill (1,4) circle (1.5pt);
\draw(1,4)  node [left] {$I'$};
\fill (5,4) circle (1.5pt);
\draw(5,4) node [right] {$I$};
\fill (8,4) circle (1.5pt);
\draw(8,4)  node [left] {$I'$};
\fill (12,4) circle (1.5pt);
\draw(12,4) node [right] {$I$};
\fill (10,12) circle (1.5pt);
\draw(10,12) node [above] {$J$};
\fill (10,8) circle (1.5pt);
\draw(10,8) node [below] {$J'$};
\fill (10,6) circle (1.5pt);
\draw(10,6)  node [above] {$J$};
\fill (10,2) circle (1.5pt);
\draw(10,2) node [below] {$J'$};
\fill (3,6) circle (1.5pt);
\draw(3,6)  node [above] {$J$};
\fill (3,2) circle (1.5pt);
\draw(3,2) node [below] {$J'$};

\end{tikzpicture}


% ------•---- Les quatres cadrans --------------------------

Quatre quadrants.

\parbox{5cm}{\begin{tikzpicture}[line cap=round,line join=round,>=triangle 45,x=1.0cm,y=1.0cm, scale=.8]
\clip (0.38,7.29) rectangle (6.04,12.92);
\draw (3,10) circle (2cm);
\draw (1,10)-- (5,10);
\draw (3,12)-- (3,8);
\draw (3.75,10.75)    node {\Huge \ding{172}};
\draw (2.25,10.75)       node {\Huge  \ding{173}};
\draw (2.25,9)     node {\Huge  \ding{174}};
\draw (3.75,9)  node {\Huge  \ding{175}};
\fill (3,10) circle (1.5pt);
\draw (3,10) node [below left]{$O$};
\fill (1,10) circle (1.5pt);
\draw (1,10) node [left] {$I'$};
\fill (5,10) circle (1.5pt);
\draw (5,10) node [right] {$I$};
\fill (3,12) circle (1.5pt);
\draw (3,12) node [above]{$J$};
\fill (3,8) circle (1.5pt);
\draw (3,8) node [below]{$J'$};
\end{tikzpicture}}
\parbox{10cm}{
   \ding{172} $\longrightarrow$\raisebox{-1ex}{$ \begin{array}{c}  
\cos (x) > 0 \\ 
\sin (x) >0 \\ 
   \end{array} 
        \qquad$}  
 \ding{173}
        $\longrightarrow$\raisebox{-1ex}{$\begin{array}{c}  
                 \cos (x) < 0 \\ 
                 \sin (x) > 0 \\ 
               \end{array}$}\\
               
               
\ding{174} 
    $\longrightarrow$\raisebox{-1ex}{$ \begin{array}{c}  
\cos (x) < 0 \\ 
\sin (x) < 0 \\ 
   \end{array} 
        \qquad$}  \ding{175}$\longrightarrow$\raisebox{-1ex}{ $\begin{array}{c}  
                 \cos (x) > 0 \\ 
                 \sin (x) < 0 \\ 
               \end{array}$}\\               
} \\
\renewcommand{\arraystretch }{1.75}
En particulier : 
\begin{quote}
{
\begin{tabular}{l@{$\;$}c@{$\qquad$}l@{$\;$}c}
$\bullet$  & $\cos 0 = 1 $ 
         & $\bullet$ & $\cos \dfrac{\pi}{2} = 0 $ \\  
          & $\sin 0 = 0 $ 
         &  & $\sin \dfrac{\pi}{2} = 1 $ \\  
& & & \\
$\bullet$  & $\cos \pi = -1 $ 
         & $\bullet$ & $\cos \dfrac{3\pi}{2} = 0 $ \\  
  & $\sin \pi = 0 $ 
         & & $\sin \dfrac{3\pi}{2} = -1 $ \\                             
\end{tabular}              
}
\end{quote}
\renewcommand{\arraystretch }{1}
\newpage 

Propriétés fondamentales pour $k \in \mathbb{Z}$  

\begin{quote}
\begin{tabular}{l@{$\,$}cc@{$\,$}l}
\ding{81} & Pour tout $x\in \Re$ & $\bullet$ & $-1
                       \leqslant \cos x  \leqslant 1 $ \\
          &          & $\bullet$ & $-1
                       \leqslant \sin x  \leqslant 1 $ \\
 & & & \\ 
 \ding{81} & Pour tout $x\in \Re$ & $\bullet$ 
                     & $ \cos (x+2k\pi) = \cos x $ \\
       &   & $\bullet$ & $ \sin x  (x+2k\pi) = \sin x $ \\
                        & & & \\ 
 \ding{81} & Pour tout $x\in \Re$ & $\bullet$ 
         & $ \cos^2 (x+2k\pi) + \sin^2 (x+2k\pi) = 1 $ \\

\end{tabular}\\

\bigskip

$(O, \overrightarrow{OI},  \overrightarrow{OJ})$ un repère orthonormal.\\

$M(\cos x, \sin x)$ dans $(O, \overrightarrow{OI},  \overrightarrow{OJ})$. \\

\begin{tabular}{r@{\,}l}
$ \cos^2 (x+2k\pi) + \sin^2 x $ &$= OH^2 + OK^2$\\
 &$= OH^2 + HM^2 $\\
 &$= OM^2$\\
 &$= 1 \qquad \qquad \text{ car }\; OM = 1 $\\ 
\end{tabular}\\

{\renewcommand{\arraystretch }{1.75}
\begin{tabular}{ll}
 
\begin{tikzpicture}[line cap=round,line join=round,>=triangle 45,x=1.0cm,y=1.0cm,scale=1]
\clip(-2.25,-2.38) rectangle (2.37,2.39);
\draw[line width=0.4pt] (0,1.32) -- (0.1,1.32) -- (0.1,1.42) -- (0,1.42) -- cycle; 
\draw[line width=0.4pt] (1.41,0.1) -- (1.31,0.1) -- (1.31,0) -- (1.41,0) -- cycle; 
\draw(0,0) circle (2cm);
\draw (-2,0)-- (2,0);
\draw (0,2)-- (0,-2);
\draw [dash pattern=on 2pt off 2pt] (1.41,1.42)-- (0,1.42);
\draw [dash pattern=on 2pt off 2pt] (1.41,1.42)-- (1.41,0);
\draw [line width=1.6pt] (0,1.42)-- (0,0);
\draw [line width=1.6pt] (0,0)-- (1.41,0);
\draw [line width=0.4pt] (0,2)-- (0,0);
\draw [line width=0.4pt] (0,0)-- (2,0);
\draw [line width=0.4pt] (2,0)-- (2,2);
\draw [line width=0.4pt] (2,2)-- (0,2);
\draw (0,0)-- (2,2);

\fill  (0,0) circle (1.5pt);
\draw(0,0) node [below left] {$O$};
\fill (-2,0) circle (1.5pt);
\draw(-2,0) node [left] {$I'$};
\fill (2,0) circle (1.5pt);
\draw(2,0) node [right] {$I$};
\fill (0,2) circle (1.5pt);
\draw(0,2) node [above] {$J$};
\fill (0,-2) circle (1.5pt);
\draw(0,-2) node [below] {$J'$};
\fill (1.41,1.42) circle (1.5pt);
\draw(1.41,1.42) node [above] {$M$};
\fill (0,1.42) circle (1.5pt);
\draw(0,1.42) node [left]{$K$};
\fill (1.41,0) circle (1.5pt);
\draw(1.41,0) node [below] {$H$};

\end{tikzpicture}


         &  \raisebox{27ex}{ \ding{81}}
            \raisebox{16.5ex}{              
         \parbox{.5\textwidth}{ 
        Cosinus et sinus de $\dfrac{\pi}{4}$ \\
        
        $ \begin{array}{r@{\;}l}
                 \cos^2 x + \sin^2 x              &= 1 \\
\cos^2 (\dfrac{\pi}{4}) + \sin^2 (\dfrac{\pi}{4}) &= 1 \\
                             2 sin^2 \dfrac{\pi}{4}      &= 1 \\ 
                         cos^2    \dfrac{\pi}{4} =     sin^2  \dfrac{\pi}{4} &= 
   \dfrac{1}{2} \quad \text { donc }= \dfrac{\sqrt{2}}{2} \text { ou } \cancel {-\dfrac{\sqrt{2}}{2}}\\  
        \end{array}$
         }}\\
\end{tabular}\\
\vspace*{-1cm}
\begin{center}
\fcolorbox{black}  {white}{
\hbox{
$\cos \dfrac{\pi}{4} = \dfrac{\sqrt{2}}{2} \quad \text{ et } \quad \sin \dfrac{\pi}{4} =  \dfrac{\sqrt{2}}{2} $ 
}}
\end{center}

\bigskip 

\begin{tabular}{ll}
 
% \definecolor{uuuuuu}{rgb}{0.27,0.27,0.27}
% \definecolor{qqffqq}{rgb}{0,1,0}
% \definecolor{ffqqqq}{rgb}{1,0,0}
% \definecolor{qqqqff}{rgb}{0,0,1}
\begin{tikzpicture}[line cap=round,line join=round,>=triangle 45,x=1.0cm,y=1.0cm,scale=1]
\clip(-2.25,-2.38) rectangle (2.37,2.39);
\draw[line width=0.4pt] (0,1.64) -- (0.1,1.64) -- (0.1,1.73) -- (0,1.73) -- cycle; 
\draw[line width=0.4pt] (1,0.1) -- (0.9,0.1) -- (0.9,0) -- (1,0) -- cycle; 
\draw(0,0) circle (2cm);
\draw (-2,0)-- (2,0);
\draw (0,2)-- (0,-2);
\draw [dash pattern=on 2pt off 2pt] (1,1.73)-- (0,1.73);
\draw [dash pattern=on 2pt off 2pt] (1,1.73)-- (1,0);
\draw [line width=1.6pt] (0,1.73)-- (0,0);
\draw [line width=1.6pt] (0,0)-- (1,0);
\draw  (0,0)-- (2,0);
\draw  (2,0)-- (1,1.73);
\draw  (1,1.73)-- (0,0);

\fill  (0,0) circle (1.5pt);
\draw(0,0) node [below left] {$O$};
\fill (-2,0) circle (1.5pt);
\draw(-2,0) node [left] {$I'$};
\fill (2,0) circle (1.5pt);
\draw(2,0) node [right] {$I$};
\fill (0,2) circle (1.5pt);
\draw(0,2) node [above] {$J$};
\fill (0,-2) circle (1.5pt);
\draw(0,-2) node [below] {$J'$};
\fill  (1,1.73) circle (1.5pt);
\draw (1.1,1.73) node [right]{$M$};
\fill  (0,1.73) circle (1.5pt);
\draw (0,1.73) node [left] {$K$};
\fill  (1,0) circle (1.5pt);
\draw (1,0) node [below] {$H$};
\end{tikzpicture}

         &  \raisebox{31ex}{ \ding{81}}
            \raisebox{10ex}{              
         \parbox{.5\textwidth}{ 
        Cosinus et sinus de $\dfrac{\pi}{3}$ \\
        
        $ \begin{array}{r@{\;}l}
                 \cos^2 x + \sin^2 x              &= 1 \\
\cos^2 (\dfrac{\pi}{3}) + \sin^2 (\dfrac{\pi}{3}) &= 1 \\
                                       & \qquad \qquad 0H=\dfrac{1}{2} \\
                     \dfrac{1}{4} + \sin^2 (\dfrac{\pi}{3})    &= 1 \\
                       \sin^2 (\dfrac{\pi}{3}) &= 1 - \dfrac{1}{4} \\
                          \sin^2 (\dfrac{\pi}{3}) -\dfrac{3}{4}  &= 0 \\ 
 \left( \sin (\dfrac{\pi}{3} - \sqrt{\dfrac{3}{4}}\right)  \left( \sin (\dfrac{\pi}{3}) +\sqrt{\dfrac{3}{4}}\right) &= 0 \\       
                                   \sin (\dfrac{\pi}{3}) &= 
\dfrac{\sqrt{3}}{2} \text { ou } \cancel {-\dfrac{\sqrt{3}}{2}}\\  
        \end{array}$
         }}\\         
\end{tabular}\\
}
\renewcommand{\arraystretch }{1}
\begin{center}
\fcolorbox{black}  {white}{
\hbox{
$\cos \dfrac{\pi}{3} = \dfrac{1}{2} \quad \text{ et } \quad \sin \dfrac{\pi}{3} =  \dfrac{\sqrt{3}}{2} $ 
}}
\end{center}

\newpage

{\renewcommand{\arraystretch }{1.75}
\begin{tabular}{ll}

\begin{tikzpicture}[line cap=round,line join=round,>=triangle 45,x=1.0cm,y=1.0cm]
\clip(-2.4,-2.38) rectangle (2.8,2.39);
\draw(0,0) circle (2cm);
\draw (-2,0)-- (2,0);
\draw (0,2)-- (0,-2);
\draw [dash pattern=on 2pt off 2pt] (1.73,1)-- (0,1);
\draw [dash pattern=on 2pt off 2pt] (1.73,1)-- (1.73,0);
\draw [line width=1.6pt] (0,1)-- (0,0);
\draw [line width=1.6pt] (0,0)-- (1.73,0);
\draw [line width=2pt](0,0)-- (1.73,1);
\draw [line width=2pt](1.73,1)-- (0,2);
\draw [line width=2pt](0,0)-- (0,2);
\draw (0,2)-- (0,0);
\fill  (0,0) circle (1.5pt);
\draw(0,0) node [below left] {$O$};
\fill (-2,0) circle (1.5pt);
\draw(-2,0) node [left] {$I'$};
\fill (2,0) circle (1.5pt);
\draw(2,0) node [right] {$I$};
\fill (0,2) circle (1.5pt);
\draw(0,2) node [above] {$J$};
\fill (0,-2) circle (1.5pt);
\draw(0,-2) node [below] {$J'$};
\fill  (1.73,1) circle (1.5pt);
\draw (1.73,1) node [right] {$M(\dfrac{\pi}{6})$};
\fill  (0,1) circle (1.5pt);
\draw (0,1) node [left]{$K$};
\fill  (1.73,0) circle (1.5pt);
\draw (1.73,0) node [below] {$H$};
\end{tikzpicture}
         &  \raisebox{31ex}{ \ding{81}}
            \raisebox{10ex}{              
         \parbox{.5\textwidth}{ 
        Cosinus et sinus de $\dfrac{\pi}{6}$ \\
        
        $ \begin{array}{r@{\;}l}
                 \cos^2 x + \sin^2 x              &= 1 \\
\cos^2 (\dfrac{\pi}{6}) + \sin^2 (\dfrac{\pi}{6}) &= 1 \\
                                       & \qquad \qquad OK=\dfrac{1}{2} \\
                      \cos^2 (\dfrac{\pi}{6})  +\dfrac{1}{4}   &= 1 \\
                       \cos^2 (\dfrac{\pi}{6}) &= 1 - \dfrac{1}{4} \\
                          \cos^2 (\dfrac{\pi}{6}) -\dfrac{3}{4}  &= 0 \\ 
 \left( \cos^2 (\dfrac{\pi}{6}) - \sqrt{\dfrac{3}{4}}\right)  \left( \cos^2 (\dfrac{\pi}{6})) +\sqrt{\dfrac{3}{4}}\right) &= 0 \\       
                                   \cos (\dfrac{\pi}{6}) &= 
\dfrac{\sqrt{3}}{2} \text { ou } \cancel {-\dfrac{\sqrt{3}}{2}}\\  
        \end{array}$
         }}\\         
\end{tabular}\\
}
\renewcommand{\arraystretch }{1}
\begin{center}
\fcolorbox{black}  {white}{
\hbox{
$\cos \dfrac{\pi}{6} = \dfrac{\sqrt{3}}{2} \quad \text{ et } \quad \sin \dfrac{\pi}{6} =  \dfrac{1}{2} $ 
}}
\end{center}
\end{quote}

\vspace{2cm}

\underline{Récapitulation} : \\

\bigskip 

\hspace*{2cm}
{\renewcommand{\arraystretch }{2.3}
\begin{tabular}{|c||c|c|c|c|c|}
\hline
$x$ & $0$ & $\dfrac{\pi}{6}$ & $\dfrac{\pi}{4}$ & $\dfrac{\pi}{3}$ & $\dfrac{\pi}{2}$ \\
\hline
\hline
$\cos x$ & $1$ & $\dfrac{\sqrt{3}}{2}$ & $\dfrac{\sqrt{2}}{2}$ & $\dfrac{1}{2}$ & $0$ \\
\hline
$\sin x$ & $0$ & $\dfrac{1}{2}$ & $\dfrac{\sqrt{2}}{2}$ & $\dfrac{\sqrt{3}}{2}$ & $1$ \\
\hline
\end{tabular}
}

\vspace{2cm}


\centerline{
 \renewcommand{\arraystretch }{2.3}
\begin{tabular}{|c||c|c|c|c|c|c|c|c|c|c|c|c|c|c|c|c|}
\hline
$x$ & $0$ & $\dfrac{\pi}{6}$ & $\dfrac{\pi}{4}$ & $\dfrac{\pi}{3}$ & $\dfrac{\pi}{2}$ & $\dfrac{2\pi}{3}$ & $\dfrac{3\pi}{4}$ & $\dfrac{5\pi}{6}$ & $\pi$ & $\dfrac{7\pi}{6}$ & $\dfrac{5\pi}{4}$ & $\dfrac{4\pi}{3}$ & $\dfrac{3\pi}{2}$ & $\dfrac{5\pi}{3}$ & $\dfrac{7\pi}{4}$ & $\dfrac{11\pi}{6}$ \\
\hline
\hline
$\cos x$ & $1$ & $\dfrac{\sqrt{3}}{2}$ & $\dfrac{\sqrt{2}}{2}$ & $\dfrac{1}{2}$ & $0$
         & $-\dfrac{1}{2}$ & $-\dfrac{\sqrt{2}}{2}$ & $- \dfrac{\sqrt{3}}{2}$ & $-1$ 
         & $-\dfrac{\sqrt{3}}{2}$ & $-\dfrac{\sqrt{2}}{2}$ & $-\dfrac{1}{2}$ & $0$ 
         & $\dfrac{1}{2}$ & $\dfrac{\sqrt{2}}{2}$ & $\dfrac{\sqrt{3}}{2}$\\
\hline
$\sin x$ & $0$ &  $\dfrac{1}{2}$ & $\dfrac{\sqrt{2}}{2}$ & $\dfrac{\sqrt{3}}{2}$ & $1$
         & $\dfrac{\sqrt{3}}{2}$ & $\dfrac{\sqrt{2}}{2}$ & $\dfrac{1}{2}$ & $0$
         & $-\dfrac{1}{2}$ & $-\dfrac{\sqrt{2}}{2}$ & $- \dfrac{\sqrt{3}}{2}$ & $-1$ 
         & $-\dfrac{\sqrt{3}}{2}$ & $-\dfrac{\sqrt{2}}{2}$ & $-\dfrac{1}{2}$\\
\hline
\end{tabular}
}



\newpage

\subsection*{Formules de transposition}

\definecolor{uuuuuu}{rgb}{0.27,0.27,0.27}
\begin{tikzpicture}[line cap=round,line join=round,>=triangle 45,x=1.0cm,y=1.0cm,scale=.8]
\clip(-2.52,-2.76) rectangle (16.6,2.76);
\draw[line width=0.4pt,color=red] (0,1.44) -- (0.21,1.44) -- (0.21,1.65) -- (0,1.65) -- cycle; 
\draw[line width=0.4pt,color=red] (1.13,0.21) -- (0.91,0.21) -- (0.91,0) -- (1.13,0) -- cycle; 
\draw[line width=0.4pt,color=red] (0.21,-1.65) -- (0.21,-1.44) -- (0,-1.44) -- (0,-1.65) -- cycle; 
\draw[line width=0.4pt,color=red] (6.98,1.46) -- (7.19,1.46) -- (7.19,1.68) -- (6.98,1.68) -- cycle; 
\draw[line width=0.4pt,color=red] (8.11,0.24) -- (7.89,0.24) -- (7.89,0.03) -- (8.11,0.03) -- cycle; 
\draw[line width=0.4pt,color=red] (14,1.44) -- (14.21,1.44) -- (14.21,1.65) -- (14,1.65) -- cycle; 
\draw[line width=0.4pt,color=red] (15.13,0.21) -- (14.91,0.21) -- (14.91,0) -- (15.13,0) -- cycle; 
\draw[line width=0.4pt,color=red] (6.07,0.03) -- (6.07,0.24) -- (5.85,0.24) -- (5.85,0.03) -- cycle; 
\draw[line width=0.4pt,color=red] (12.87,-0.21) -- (13.09,-0.21) -- (13.09,0) -- (12.87,0) -- cycle; 
\draw[line width=0.4pt,color=red] (14,-1.44) -- (13.79,-1.44) -- (13.79,-1.65) -- (14,-1.65) -- cycle; 
\draw(0,0) circle (2cm);
\draw (-2,0)-- (2,0);
\draw (0,2)-- (0,-2);
\draw [dash pattern=on 5pt off 5pt,color=red] (1.13,1.65)-- (0,1.65);
\draw [dash pattern=on 5pt off 5pt,color=red] (1.13,1.65)-- (1.13,0);
\draw [line width=1.6pt,color=Green] (0,1.65)-- (0,0);
\draw [line width=1.6pt,color=blue] (0,0)-- (1.13,0);
\draw [dash pattern=on 5pt off 5pt,color=red] (1.13,0)-- (1.13,-1.65);
\draw [dash pattern=on 5pt off 5pt,color=red] (1.13,-1.65)-- (0,-1.65);
\draw [line width=1.6pt,color=Green] (0,-1.65)-- (0,0);
\draw(6.98,0.03) circle (2cm);
\draw (4.98,0.03)-- (8.98,0.03);
\draw (6.98,2.03)-- (6.98,-1.97);
\draw [dash pattern=on 5pt off 5pt,color=red] (8.11,1.68)-- (6.98,1.68);
\draw [dash pattern=on 5pt off 5pt,color=red] (8.11,1.68)-- (8.11,0.03);
\draw [line width=1.6pt,color=Green] (6.98,1.68)-- (6.98,0.03);
\draw [line width=1.6pt,color=blue] (6.98,0.03)-- (8.11,0.03);
\draw(14,0) circle (2cm);
\draw (12,0)-- (16,0);
\draw (14,2)-- (14,-2);
\draw [dash pattern=on 5pt off 5pt,color=red] (15.13,1.65)-- (14,1.65);
\draw [dash pattern=on 5pt off 5pt,color=red] (15.13,1.65)-- (15.13,0);
\draw [line width=1.6pt,color=Green] (14,1.65)-- (14,0);
\draw [line width=1.6pt,color=blue] (14,0)-- (15.13,0);
\draw [dash pattern=on 5pt off 5pt,color=red] (5.85,1.68)-- (6.98,1.68);
\draw [dash pattern=on 5pt off 5pt,color=red] (5.85,1.68)-- (5.85,0.03);
\draw [dash pattern=on 5pt off 5pt,color=red] (12.87,-1.65)-- (15.13,1.65);
\draw [dash pattern=on 5pt off 5pt,color=red] (12.87,-1.65)-- (14,-1.65);
\draw [dash pattern=on 5pt off 5pt,color=red] (12.87,-1.65)-- (12.87,0);
\draw [line width=1.6pt,color=Green] (14,0)-- (14,-1.65);
\draw [line width=1.2pt,color=blue] (14,0)-- (12.87,0);
\fill (0,0) circle (1.5pt);
\draw(0,0) node [below left] {$O$};
\fill (-2,0) circle (1.5pt);
\draw(-2,0) node [left] {$I'$};
\fill (2,0) circle (1.5pt);
\draw(2,0) node [right] {$I$};
\fill (0,2) circle (1.5pt);
\draw(0.01,2.47) node {$J$};
\fill (0,-2) circle (1.5pt);
\draw(-0.09,-2.18) node [below] {$J'$};
\fill [color=red] (1.13,1.65) circle (1.5pt);
\draw[color=red] (1.3,1.91) node {$M(x)$};
\fill [color=red] (0,1.65) circle (1.5pt);
\draw[color=red] (0,1.65) node [left]{$K$};
\fill [color=red] (1.13,0) circle (1.5pt);
\draw[color=red] (1.13,0) node [below left]{$H$};
\fill [color=red] (1.13,-1.65) circle (1.5pt);
\draw[color=red] (1.13,-1.65) node [right]{$M'(x)$};
\fill [color=red] (0,-1.65) circle (1.5pt);
\draw[color=red] (0,-1.65) node [left] {$K'$};
\fill (7,0) circle (1.5pt);
\draw(7,0) node [below left]{$O$};
\fill (5,0) circle (1.5pt);
\draw(5,0) node[below left] {$I'$};
\fill (9,0) circle (1.5pt);
\draw(9,0) node [right] {$I$};
\fill (7,2.5) circle (1.5pt);
\draw(7,2.5) node {$J$};
\fill (7,-2) circle (1.5pt);
\draw(7,-2) node [below] {$J'$};
\fill [color=red] (8.11,1.68) circle (1.5pt);
\draw[color=red] (8.11,1.68) node [right]{$M(x)$};
\fill [color=red] (7,1.68) circle (1.5pt);
\draw[color=red] (7,1.68) node [below left] {$K$};
\fill [color=red] (8.11,0) circle (1.5pt);
\draw[color=red] (8.11,0) node [below]{$H$};
\fill (14,0) circle (1.5pt);
\draw(14,0) node [above left]{$O$};
\fill (12,0) circle (1.5pt);
\draw(12,0) node [left]{$I'$};
\fill (16,0) circle (1.5pt);
\draw(16,0) node [right] {$I$};
\fill (14,2) circle (1.5pt);
\draw(14,2) node [above]{$J$};
\fill (14,-2) circle (1.5pt);
\draw(14,-2) node [below]{$J'$};
\fill [color=red] (15.13,1.65) circle (1.5pt);
\draw[color=red] (15.13,1.65) node [right]{$M(x)$};
\fill [color=red] (14,1.65) circle (1.5pt);
\draw[color=red] (14,1.65) node [left]{$K$};
\fill [color=red] (15.13,0) circle (1.5pt);
\draw[color=red] (15,0) node [below]{$H$};
\fill [color=red] (5.85,1.68) circle (1.5pt);
\draw[color=red] (5.85,1.68) node [left]{$M'(x)$};
\fill (5.85,0) circle (1.5pt);
\draw [color=red] (5.85,0) node [below] {$H'$};
\fill [color=red] (12.87,-1.65) circle (1.5pt);
\draw[color=red] (12.87,-1.65) node [below left]{$M'(x)$};
\fill (12.87,0) circle (1.5pt);
\draw [color=red] (12.87,0) node [above]{$H'$};
\fill (14,-1.65) circle (1.5pt);
\draw [color=red](14,-1.65) node [right] {$K'$};
\end{tikzpicture}


\bigskip 

\centerline{
\begin{tabular}{c@{$\,$}l@{\hspace{3cm}}c@{$\,$}l@{\hspace{3cm}}c@{$\,$}l}
% $\bullet$ & $M'(x) \text{ image de } x $ & 
%      $\bullet$ & $M'(\pi - x) \text{ image de } \pi - x $& 
%           $\bullet$ & $M'(\pi + x) \text{ image de } \pi + x $ \\
&$\cos (-x) = \cos (x) $& &$\cos (\pi -x) = -\cos (x)$&&$\cos (\pi+x) = -\cos (x)$\\
&$\sin(-x) = -\sin (x) $& &$\sin (\pi -x) = \sin (x)$&&$\sin (\pi+x) = \sin (x)$\\
\end{tabular}
}

\centerline{
\begin{tikzpicture}[line cap=round,line join=round,>=triangle 45,x=1.0cm,y=1.0cm, scale=.8]
\clip(-2.33,-2.76) rectangle (10.12,2.59);
\draw[line width=0.4pt,color=red] (0,0.81) -- (0.19,0.81) -- (0.19,1) -- (0,1) -- cycle; 
\draw[line width=0.4pt,color=red] (1.73,0.19) -- (1.54,0.19) -- (1.54,0) -- (1.73,0) -- cycle; 
\draw[line width=0.4pt,color=red] (6.98,1.65) -- (7.18,1.65) -- (7.18,1.45) -- (6.98,1.45) -- cycle; 
\draw[line width=0.4pt,color=red] (8.13,0.19) -- (7.94,0.19) -- (7.94,0) -- (8.13,0) -- cycle; 
\draw[line width=0.4pt,color=red] (0.16,0.09) -- (0.07,0.26) -- (-0.09,0.16) -- (0,0) -- cycle; 
\draw[line width=0.4pt,color=red] (-0.19,1.73) -- (-0.19,1.54) -- (0,1.54) -- (0,1.73) -- cycle; 
\draw[line width=0.4pt,color=red] (-0.81,0) -- (-0.81,0.19) -- (-1,0.19) -- (-1,0) -- cycle; 
\draw(0,0) circle (2cm);
\draw (-2,0)-- (2,0);
\draw (0,2)-- (0,-2);
\draw [dash pattern=on 5pt off 5pt,color=red] (1.73,1)-- (0,1);
\draw [dash pattern=on 5pt off 5pt,color=red] (1.73,1)-- (1.73,0);
\draw [line width=1.6pt,color=Green] (0,1)-- (0,0);
\draw [line width=1.6pt,color=blue] (0,0)-- (1.73,0);
\draw(7,0) circle (2cm);
\draw (5,0)-- (9,0);
\draw (6.98,2.03)-- (6.98,-1.97);
\draw [dash pattern=on 5pt off 5pt,color=red] (8.13,1.65)-- (6.98,1.65);
\draw [dash pattern=on 5pt off 5pt,color=red] (8.13,1.65)-- (8.13,0);
\draw [line width=1.6pt,color=Green] (6.98,1.65)-- (7,0);
\draw [line width=1.6pt,color=blue] (7,0)-- (8.13,0);
\draw [dash pattern=on 2pt off 2pt,color=red] (0,0)-- (-1,1.73);
\draw [dash pattern=on 2pt off 2pt,color=red] (0,0)-- (1.73,1);
\draw [dash pattern=on 5pt off 5pt,color=red] (-1,1.73)-- (0,1.73);
\draw [dash pattern=on 5pt off 5pt,color=red] (-1,1.73)-- (-1,0);
\draw [dotted,color=red] (7,0)-- (8.87,0.7);
\draw [dotted,color=red] (7,0)-- (8.13,1.65);
\draw [dash pattern=on 5pt off 5pt,color=red] (8.87,0)-- (8.87,0.7);
\draw [dash pattern=on 5pt off 5pt,color=red] (8.87,0.7)-- (6.98,0.7);
\fill  (0,0) circle (1.5pt);
\draw (0,0) node [below left]{$O$};
\fill  (-2,0) circle (1.5pt);
\draw (-2,0) node [left]{$I'$};
\fill  (2,0) circle (1.5pt);
\draw (2,0) node [right] {$I$};
\fill  (0,2) circle (1.5pt);
\draw (0,2) node [above] {$J$};
\fill  (0,-2) circle (1.5pt);
\draw (0,-2) node [below] {$J'$};
\fill [color=red] (1.73,1) circle (1.5pt);
\draw[color=red] (1.73,1) node [right] {$M(x)$};
\fill [color=red] (0,1) circle (1.5pt);
\draw[color=red] (0,1) node [left] {$K$};
\fill [color=red] (1.73,0) circle (1.5pt);
\draw[color=red] (1.73,0) node [below] {$H$};
\fill  (7,0) circle (1.5pt);
\draw (7,0) node [below left]{$O$};
\fill  (5,0) circle (1.5pt);
\draw (5,0) node [left] {$I'$};
\fill  (9,0) circle (1.5pt);
\draw (9,0) node [right] {$I$};
\fill  (7,2) circle (1.5pt);
\draw (7,2) node [above] {$J$};
\fill  (7,-2) circle (1.5pt);
\draw (7,-2) node [below] {$J'$};
\fill [color=red] (8.13,1.65) circle (1.5pt);
\draw[color=red] (8.13,1.65) node [right]{$M(x)$};
\fill [color=red] (6.98,1.65) circle (1.5pt);
\draw[color=red] (6.98,1.65) node [left] {$K$};
\fill [color=red] (8.13,0) circle (1.5pt);
\draw[color=red] (8.13,0) node [below] {$H$};
\fill [color=red] (-1,1.73) circle (1.5pt);
\draw[color=red] (-1,1.73) node [left] {$M'(x)$};
\fill [color=red] (0,1.73) circle (1.5pt);
\draw[color=red] (0,1.73) node [right] {$K'$};
\fill [color=red] (-1,0) circle (1.5pt);
\draw[color=red] (-1,0) node [below]{$H'$};
\fill [color=red] (8.87,0.7) circle (1.5pt);
\draw[color=red] (8.87,0.7) node [right] {$M'(x)$};
\fill [color=red] (6.98,0.7) circle (1.5pt);
\draw[color=red] (6.98,0.7) node [left] {$K'$};
\fill [color=red] (8.87,0) circle (1.5pt);
\draw[color=red] (8.87,0) node [below] {$H'$};
\end{tikzpicture}
}

\centerline{\renewcommand{\arraystretch }{2}
\begin{tabular}{c@{$\,$}l@{\hspace{2.5cm}}%
                           c@{$\,$}l}
% $\bullet$ & $M'\left( x+\dfrac{\pi}{2} \right) \text{ est l'image ddu réel  } x $ & 
%      $\bullet$ & $M'\left(\dfrac{\pi}{2} - x\right) \text{ image de } x $ \\
           &$\cos \left(x+\dfrac{\pi}{2}\right) = -\sin (x) $& 
               &$\cos \left(\dfrac{\pi}{2} -x\right) = \sin (x)$\\
          &$\sin \left(x+\dfrac{\pi}{2}\right) = \cos (x) $& 
            &$\sin \left(\dfrac{\pi}{2} -x\right) = \cos (x)$\\
\end{tabular}}

\vspace*{.3cm}

Un superbe exercice : 

Soit $x\in \mathbb{R}$, simplifier : 

\begin{enumerate}

\item \raisebox{-2.8ex}{\renewcommand{\arraystretch }{1}
\begin{tabular}{r@{}l}
$A$&$=\cos x +\cos \left(\dfrac{\pi}{2}-x\right) +\cos \left(x +\dfrac{\pi}{2}\right)+\cos (\pi -x) $\\ 
   &$=\cos x + \sin x - \sin x - \cos x $ \\
   & $= 0$ \\
\end{tabular}}

\item \raisebox{-7ex}{\renewcommand{\arraystretch }{1.6}
\begin{tabular}{r@{$\,$}l}
$A'$&$=\cos \left(\dfrac{\pi}{8}\right) +\cos \left(\dfrac{3\pi}{8}\right)+ \cos \left(\dfrac{5\pi}{8}\right) +\cos \left(\dfrac{7\pi}{8}\right)$\\ 
    &$=\cos \left(\dfrac{\pi}{8}\right) +\cos \left(\dfrac{\pi}{2}-\dfrac{\pi}{8}\right)+ \cos \left(\dfrac{\pi}{2}+\dfrac{\pi}{8}\right) +\cos \left(\pi - \dfrac{\pi}{8}\right)$\\ 
   &$=\cos \dfrac{\pi}{8}+ \sin \dfrac{\pi}{8} -  \sin \dfrac{\pi}{8} - \cos \dfrac{\pi}{8}$  \\
   & $=0$ \\
\end{tabular}}

\item \raisebox{-3ex}{\renewcommand{\arraystretch }{1}
\begin{tabular}{r@{$\,$}l}
 $B$ & $=\sin^2 x 
         + \sin^2\left(\dfrac{\pi}{2} - x\right) 
         + \sin^2\left(x +\dfrac{\pi}{2}\right)
         +\sin^2 (\pi - x) $\\ 
   & $=\sin^2 x + \cos^2 x + \cos^2 x + \sin^2 x $  \\
   & $= 1 + 1 = 2 $ \\
\end{tabular}}

\item \raisebox{-4.5ex}{\renewcommand{\arraystretch }{1.6}
\begin{tabular}{r@{$\,$}l}
$B'$&$= \sin^2 \left(\dfrac{\pi}{8}\right) 
      + \sin^2 \left(\dfrac{\pi}{2} - \dfrac{\pi}{8}\right)) 
      + \sin^2 \left(\dfrac{\pi}{2}+\dfrac{\pi}{8}\right)) +  \sin^2 (\pi - \dfrac{\pi}{8}) $\\ 
    &$=\sin^2 \left(\dfrac{\pi}{8}\right) 
       + \cos^2 \left(\dfrac{\pi}{8}\right) 
       + \cos^2 \left(\dfrac{\pi}{8}\right) 
       + \sin^2 \left(\dfrac{\pi}{8}\right) $  \\
   & $= 1 + 1 = 2 $ \\
\end{tabular}}

\end{enumerate}

\samepage

\newpage 

\renewcommand{\arraystretch }{1}

\subsection{Représentations graphiques}

\subsubsection{Représentation graphique de la fonction sinus}

\begin{tabular}{rl}
$\sin\quad : $ & $\mathbb{R} \longrightarrow \mathbb{R} $ \\
               &  $x \longmapsto \sin x $ \\
               & \\
               & $ {\Large \mathcal{D}_{\sin}} = \mathbb{R} $ \\
\end{tabular}\\

% \definecolor{ttzzqq}{rgb}{0.2,0.6,0} % Vert agreable
% \definecolor{qqzzqq}{rgb}{0,0.6,0}   % Vert agreable
% \definecolor{ffefdv}{rgb}{1,0.94,0.84} % AntiqueWhite pour grid
\begin{tikzpicture}[line cap=round,line join=round,>=triangle 45,x=1.0cm,y=1.0cm,scale=.8]

% \draw [color=AntiqueWhite,, xstep=0.1cm,ystep=0.1cm] (-7.1,-1.54) grid (13.29,2.53);


\draw[->,color=black] (-7.1,0) -- (13.29,0);
\foreach \x in {-6,-4,-2,2,4,6,8,10,12}
\draw[shift={(\x,0)},color=black] (0pt,2pt) -- (0pt,-2pt);
\draw[->] (0,-1.54) -- (0,2.53);
\foreach \y in {-1,1,2}
\draw[shift={(0,\y)},color=black] (2pt,0pt) -- (-2pt,0pt);
\clip(-7.1,-1.54) rectangle (13.29,2.53);
\draw[color=Green, smooth,samples=100,domain=-7.097377544417069:13.285348399587015] plot(\x,{sin(((\x))*180/pi)});

% Délimite les périodes 
\draw (-6.28,2)-- (-6.28,-2);
% \draw (0,2.39)-- (0,-2.27);
\draw (12.57,2)-- (12.57,-2);
\draw (6.28,2)-- (6.28,-2);

\draw (0,2) -- node[below, midway] {Période $2\pi$}  (6.28,2);

\begin{tiny}
\draw  (1.05,0)-- ++(-1.0pt,0 pt) -- ++(2.0pt,0 pt) ++(-1.0pt,-1.0pt) -- ++(0 pt,2.0pt);
\draw (pi/3,0) node [below]{$\dfrac{\pi}{3}$};
\draw  (1.57,0)-- ++(-1.0pt,0 pt) -- ++(2.0pt,0 pt) ++(-1.0pt,-1.0pt) -- ++(0 pt,2.0pt);
\draw (pi/2,0 ) node [below] {$\dfrac{\pi}{2}$};
\draw  (3.67,0)-- ++(-1.0pt,0 pt) -- ++(2.0pt,0 pt) ++(-1.0pt,-1.0pt) -- ++(0 pt,2.0pt);
\draw (2*pi/3,0) node [below]{$\dfrac{2\pi}{3}$};
\draw  (4.71,0)-- ++(-1.0pt,0 pt) -- ++(2.0pt,0 pt) ++(-1.0pt,-1.0pt) -- ++(0 pt,2.0pt);
\draw (3*pi/2,0) node [below] {$\dfrac{3\pi}{2}$};
\draw  (5.76,0)-- ++(-1.0pt,0 pt) -- ++(2.0pt,0 pt) ++(-1.0pt,-1.0pt) -- ++(0 pt,2.0pt);
\draw (5*pi/6,0) node [below] {$\dfrac{5\pi}{6}$};
\draw [color=Green] (6.28,0)-- ++(-1.0pt,-1.0pt) -- ++(2.0pt,2.0pt) ++(-2.0pt,0) -- ++(2.0pt,-2.0pt);
\draw  (2.62,0)-- ++(-1.0pt,0 pt) -- ++(2.0pt,0 pt) ++(-1.0pt,-1.0pt) -- ++(0 pt,2.0pt);
\draw (11*pi/6,0) node [above]{$\dfrac{11\pi}{6}$};
\draw (7*pi/6,0) node [above]{$\dfrac{7\pi}{6}$};
\draw  (0.57,0)-- ++(-1.0pt,0 pt) -- ++(2.0pt,0 pt) ++(-1.0pt,-1.0pt) -- ++(0 pt,2.0pt);
\draw (pi/6,0) node [below] {$\dfrac{\pi}{6}$};
\draw (0,0.5) node [left] {$\dfrac{1}{2}$};
\draw  (0,-0.5)-- ++(-1.0pt,0 pt) -- ++(2.0pt,0 pt) ++(-1.0pt,-1.0pt) -- ++(0 pt,2.0pt);
\draw (0,-0.5) node [left]{$-\dfrac{1}{2}$};
\end{tiny}
\begin{scriptsize}
\draw (pi,0) node [above]{$\pi$};
\draw (2*pi,0) node [below right]{$2\pi$};
\draw (0,1) node[right] {$1$};
\draw  (0,-1)-- ++(-1.0pt,0 pt) -- ++(2.0pt,0 pt) ++(-1.0pt,-1.0pt) -- ++(0 pt,2.0pt);
\draw (0,-1) node [left]{$-1$};
\end{scriptsize}
\fill [color=black,shift={(0.1,2)},rotate=90] (0,0) ++(0 pt,2.25pt) -- ++(1.95pt,-3.375pt)--++(-3.9pt,0 pt) -- ++(1.95pt,3.375pt);
\fill [color=black,shift={(6.18,2)},rotate=270] (0,0) ++(0 pt,2.25pt) -- ++(1.95pt,-3.375pt)--++(-3.9pt,0 pt) -- ++(1.95pt,3.375pt);
\draw [color=Green] (0.52,0.5)-- ++(-1.0pt,-1.0pt) -- ++(2.0pt,2.0pt) ++(-2.0pt,0) -- ++(2.0pt,-2.0pt);
\draw [color=Green] (-6.28,0)-- ++(-1.0pt,-1.0pt) -- ++(2.0pt,2.0pt) ++(-2.0pt,0) -- ++(2.0pt,-2.0pt);
\draw [color=Green] (-5.76,0.5)-- ++(-1.0pt,-1.0pt) -- ++(2.0pt,2.0pt) ++(-2.0pt,0) -- ++(2.0pt,-2.0pt);
\draw [color=Green] (-4.71,1)-- ++(-1.0pt,-1.0pt) -- ++(2.0pt,2.0pt) ++(-2.0pt,0) -- ++(2.0pt,-2.0pt);
\draw [color=Green] (-3.67,0.5)-- ++(-1.0pt,-1.0pt) -- ++(2.0pt,2.0pt) ++(-2.0pt,0) -- ++(2.0pt,-2.0pt);
\draw [color=Green] (-2.62,-0.5)-- ++(-1.0pt,-1.0pt) -- ++(2.0pt,2.0pt) ++(-2.0pt,0) -- ++(2.0pt,-2.0pt);
\draw [color=Green] (-3.14,0)-- ++(-1.0pt,-1.0pt) -- ++(2.0pt,2.0pt) ++(-2.0pt,0) -- ++(2.0pt,-2.0pt);
\draw [color=Green] (-1.57,-1)-- ++(-1.0pt,-1.0pt) -- ++(2.0pt,2.0pt) ++(-2.0pt,0) -- ++(2.0pt,-2.0pt);
\draw [color=Green] (-0.52,-0.5)-- ++(-1.0pt,-1.0pt) -- ++(2.0pt,2.0pt) ++(-2.0pt,0) -- ++(2.0pt,-2.0pt);
\draw [color=Green] (1.05,0.87)-- ++(-1.0pt,-1.0pt) -- ++(2.0pt,2.0pt) ++(-2.0pt,0) -- ++(2.0pt,-2.0pt);
\draw [color=Green] (1.57,1)-- ++(-1.0pt,-1.0pt) -- ++(2.0pt,2.0pt) ++(-2.0pt,0) -- ++(2.0pt,-2.0pt);
\draw [color=Green] (2.09,0.87)-- ++(-1.0pt,-1.0pt) -- ++(2.0pt,2.0pt) ++(-2.0pt,0) -- ++(2.0pt,-2.0pt);
\draw [color=Green] (2.62,0.5)-- ++(-1.0pt,-1.0pt) -- ++(2.0pt,2.0pt) ++(-2.0pt,0) -- ++(2.0pt,-2.0pt);
\draw [color=Green] (3.14,0)-- ++(-1.0pt,-1.0pt) -- ++(2.0pt,2.0pt) ++(-2.0pt,0) -- ++(2.0pt,-2.0pt);
\draw [color=Green] (3.67,-0.5)-- ++(-1.0pt,-1.0pt) -- ++(2.0pt,2.0pt) ++(-2.0pt,0) -- ++(2.0pt,-2.0pt);
\draw [color=Green] (4.71,-1)-- ++(-1.0pt,-1.0pt) -- ++(2.0pt,2.0pt) ++(-2.0pt,0) -- ++(2.0pt,-2.0pt);
\draw [color=Green] (5.76,-0.5)-- ++(-1.0pt,-1.0pt) -- ++(2.0pt,2.0pt) ++(-2.0pt,0) -- ++(2.0pt,-2.0pt);
\draw  (0,0.5)-- ++(-1.0pt,0 pt) -- ++(2.0pt,0 pt) ++(-1.0pt,-1.0pt) -- ++(0 pt,2.0pt);

\begin{pgfonlayer}{background}  
% Attention l'ordre de ces lignes est important 
% Ne pas le modifier   
\draw[step=1mm,ultra thin,AntiqueWhite!10](-7,-2)  grid (14,2.6);
\draw[step=5mm,very thin,AntiqueWhite!30] (-7,-2)  grid (14,2.6);
\draw[step=1cm,very thin,AntiqueWhite!50] (-7,-2)  grid (14,2.6);
\draw[step=5cm,thin,AntiqueWhite]         (-7,-2)  grid (14,2.6);

\end{pgfonlayer} 

\end{tikzpicture}

\bigskip 

La fonction sinus est périodique de période $2\pi$. \\

La fonction est dite \underline{sinusoïde}. 


\bigskip 

\subsubsection{Représentation graphique de la fonction cosinus}

\begin{tabular}{rl}
$\cos\quad : $ & $\mathbb{R} \longrightarrow \mathbb{R} $ \\
               &  $x \longmapsto \cos x $ \\
               & \\
               & $ {\Large \mathcal{D}_{\cos}} = \mathbb{R} $ \\
\end{tabular}\\

% \definecolor{ttzzqq}{rgb}{0.2,0.6,0} % Vert agreable
% \definecolor{qqzzqq}{rgb}{0,0.6,0}   % Vert agreable
% \definecolor{ffefdv}{rgb}{1,0.94,0.84} % AntiqueWhite pour grid
\begin{tikzpicture}[line cap=round,line join=round,>=triangle 45,x=1.0cm,y=1.0cm,scale=.8]
% \draw [color=AntiqueWhite,, xstep=0.1cm,ystep=0.1cm] (-7.1,-1.54) grid (13.29,2.53);
\draw[->,color=black] (-7.1,0) -- (13.29,0);
\foreach \x in {-6,-4,-2,2,4,6,8,10,12}
\draw[shift={(\x,0)},color=black] (0pt,2pt) -- (0pt,-2pt);
\draw[->,color=black] (0,-1.54) -- (0,2.53);
\foreach \y in {-1,1,2}
\draw[shift={(0,\y)},color=black] (2pt,0pt) -- (-2pt,0pt);
\clip(-7.1,-1.54) rectangle (13.29,2.53);
\draw[color=Green, smooth,samples=100,domain=-7.097377544417069:13.285348399587015] plot(\x,{cos(((\x))*180/pi)});

% \draw (-6.28,1.5)-- (-6.28,-2);
% \draw (0,2.39)-- (0,-2.27);
% \draw (12.57,1.5)-- (12.57,-2);
% \draw (6.28,2)-- (6.28,-2);


% Délimite les périodes 
\draw (-6.28,2)-- (-6.28,-2);
% \draw (0,2.39)-- (0,-2.27);
\draw (12.57,2)-- (12.57,-2);
\draw (6.28,2)-- (6.28,-2);

\draw (0,2) -- node[below, midway] {Période $2\pi$}  (6.28,2);
% \draw (1.53,1.78) node] {...};
\begin{tiny}
\draw  (1.05,0)-- ++(-1.0pt,0 pt) -- ++(2.0pt,0 pt) ++(-1.0pt,-1.0pt) -- ++(0 pt,2.0pt);
\draw (pi/3,0) node [below]{$\dfrac{\pi}{3}$};
\draw  (1.57,0)-- ++(-1.0pt,0 pt) -- ++(2.0pt,0 pt) ++(-1.0pt,-1.0pt) -- ++(0 pt,2.0pt);
\draw (pi/2,0 ) node [below] {$\dfrac{\pi}{2}$};
\draw  (3.67,0)-- ++(-1.0pt,0 pt) -- ++(2.0pt,0 pt) ++(-1.0pt,-1.0pt) -- ++(0 pt,2.0pt);
\draw (2*pi/3,0) node [below]{$\dfrac{2\pi}{3}$};
\draw  (4.71,0)-- ++(-1.0pt,0 pt) -- ++(2.0pt,0 pt) ++(-1.0pt,-1.0pt) -- ++(0 pt,2.0pt);
\draw (3*pi/2,0) node [below] {$\dfrac{3\pi}{2}$};
\draw  (5.76,0)-- ++(-1.0pt,0 pt) -- ++(2.0pt,0 pt) ++(-1.0pt,-1.0pt) -- ++(0 pt,2.0pt);
\draw (5*pi/6,0) node [below] {$\dfrac{5\pi}{6}$};
\draw [color=Green] (6.28,0)-- ++(-1.0pt,-1.0pt) -- ++(2.0pt,2.0pt) ++(-2.0pt,0) -- ++(2.0pt,-2.0pt);
\draw  (2.62,0)-- ++(-1.0pt,0 pt) -- ++(2.0pt,0 pt) ++(-1.0pt,-1.0pt) -- ++(0 pt,2.0pt);
\draw (11*pi/6,0) node [above]{$\dfrac{11\pi}{6}$};
\draw (7*pi/6,0) node [above]{$\dfrac{7\pi}{6}$};
\draw  (0.57,0)-- ++(-1.0pt,0 pt) -- ++(2.0pt,0 pt) ++(-1.0pt,-1.0pt) -- ++(0 pt,2.0pt);
\draw (pi/6,0) node [below] {$\dfrac{\pi}{6}$};
\draw (0,0.5) node [left] {$\dfrac{1}{2}$};
\draw  (0,-0.5)-- ++(-1.0pt,0 pt) -- ++(2.0pt,0 pt) ++(-1.0pt,-1.0pt) -- ++(0 pt,2.0pt);
\draw (0,-0.5) node [left]{$-\dfrac{1}{2}$};
\end{tiny}
\begin{scriptsize}
\draw (pi,0) node [above]{$\pi$};
\draw (2*pi,0) node [below right]{$2\pi$};
\draw (0,1) node[right] {$1$};
\draw  (0,-1)-- ++(-1.0pt,0 pt) -- ++(2.0pt,0 pt) ++(-1.0pt,-1.0pt) -- ++(0 pt,2.0pt);
\draw (0,-1) node [left]{$-1$};
\end{scriptsize}
\fill [color=black,shift={(0.1,2)},rotate=90] (0,0) ++(0 pt,2.25pt) -- ++(1.95pt,-3.375pt)--++(-3.9pt,0 pt) -- ++(1.95pt,3.375pt);
\fill [color=black,shift={(6.18,2)},rotate=270] (0,0) ++(0 pt,2.25pt) -- ++(1.95pt,-3.375pt)--++(-3.9pt,0 pt) -- ++(1.95pt,3.375pt);


\draw [color=Green] (0.52-pi/2,0.5)-- ++(-1.0pt,-1.0pt) -- ++(2.0pt,2.0pt) ++(-2.0pt,0) -- ++(2.0pt,-2.0pt);
\draw [color=Green] (-6.28-pi/2,0)-- ++(-1.0pt,-1.0pt) -- ++(2.0pt,2.0pt) ++(-2.0pt,0) -- ++(2.0pt,-2.0pt);
\draw [color=Green] (-5.76-pi/2,0.5)-- ++(-1.0pt,-1.0pt) -- ++(2.0pt,2.0pt) ++(-2.0pt,0) -- ++(2.0pt,-2.0pt);
\draw [color=Green] (-4.71-pi/2,1)-- ++(-1.0pt,-1.0pt) -- ++(2.0pt,2.0pt) ++(-2.0pt,0) -- ++(2.0pt,-2.0pt);
\draw [color=Green] (-3.67-pi/2,0.5)-- ++(-1.0pt,-1.0pt) -- ++(2.0pt,2.0pt) ++(-2.0pt,0) -- ++(2.0pt,-2.0pt);
\draw [color=Green] (-2.62-pi/2,-0.5)-- ++(-1.0pt,-1.0pt) -- ++(2.0pt,2.0pt) ++(-2.0pt,0) -- ++(2.0pt,-2.0pt);
\draw [color=Green] (-3.14-pi/2,0)-- ++(-1.0pt,-1.0pt) -- ++(2.0pt,2.0pt) ++(-2.0pt,0) -- ++(2.0pt,-2.0pt);
\draw [color=Green] (-1.57-pi/2,-1)-- ++(-1.0pt,-1.0pt) -- ++(2.0pt,2.0pt) ++(-2.0pt,0) -- ++(2.0pt,-2.0pt);
\draw [color=Green] (-0.52-pi/2,-0.5)-- ++(-1.0pt,-1.0pt) -- ++(2.0pt,2.0pt) ++(-2.0pt,0) -- ++(2.0pt,-2.0pt);
\draw [color=Green] (1.05-pi/2,0.87)-- ++(-1.0pt,-1.0pt) -- ++(2.0pt,2.0pt) ++(-2.0pt,0) -- ++(2.0pt,-2.0pt);
\draw [color=Green] (1.57-pi/2,1)-- ++(-1.0pt,-1.0pt) -- ++(2.0pt,2.0pt) ++(-2.0pt,0) -- ++(2.0pt,-2.0pt);
\draw [color=Green] (2.09-pi/2,0.87)-- ++(-1.0pt,-1.0pt) -- ++(2.0pt,2.0pt) ++(-2.0pt,0) -- ++(2.0pt,-2.0pt);
\draw [color=Green] (2.62-pi/2,0.5)-- ++(-1.0pt,-1.0pt) -- ++(2.0pt,2.0pt) ++(-2.0pt,0) -- ++(2.0pt,-2.0pt);
\draw [color=Green] (3.14-pi/2,0)-- ++(-1.0pt,-1.0pt) -- ++(2.0pt,2.0pt) ++(-2.0pt,0) -- ++(2.0pt,-2.0pt);
\draw [color=Green] (3.67-pi/2,-0.5)-- ++(-1.0pt,-1.0pt) -- ++(2.0pt,2.0pt) ++(-2.0pt,0) -- ++(2.0pt,-2.0pt);
\draw [color=Green] (4.71-pi/2,-1)-- ++(-1.0pt,-1.0pt) -- ++(2.0pt,2.0pt) ++(-2.0pt,0) -- ++(2.0pt,-2.0pt);
\draw [color=Green] (5.76-pi/2,-0.5)-- ++(-1.0pt,-1.0pt) -- ++(2.0pt,2.0pt) ++(-2.0pt,0) -- ++(2.0pt,-2.0pt);
\draw  (0,0.5)-- ++(-1.0pt,0 pt) -- ++(2.0pt,0 pt) ++(-1.0pt,-1.0pt) -- ++(0 pt,2.0pt);

\begin{pgfonlayer}{background}  
% Attention l'ordre de ces lignes est important 
% Ne pas le modifier   
\draw[step=1mm,ultra thin,AntiqueWhite!10](-7,-2)  grid (14,2.6);
\draw[step=5mm,very thin,AntiqueWhite!30] (-7,-2)  grid (14,2.6);
\draw[step=1cm,very thin,AntiqueWhite!50] (-7,-2)  grid (14,2.6);
\draw[step=5cm,thin,AntiqueWhite]         (-7,-2)  grid (14,2.6);

\end{pgfonlayer} 

\end{tikzpicture}

\bigskip 

La fonction cosinus est périodique de période $2\pi$. \\

La fonction est dite \underline{sinusoïde}. 

\newpage 

\subsubsection{Comparaison des représentations graphiques des fonctions sinus et cosinus}

\centerline{\begin{tabular}{r@{\hspace*{3cm}}l}
    \begin{tabular}{rl}
    \multicolumn{2}{c}{\textcolor{Green} {Cosinus tracé en vert}} \\
    $\cos\quad : $ & $\mathbb{R} \longrightarrow \mathbb{R} $ \\
                   &  $x \longmapsto \cos x $ \\
                   & \\
                   & $ {\Large \mathcal{D}_{\cos}} = \mathbb{R} $ \\
    \end{tabular} & 
                   \begin{tabular}{rl}
                       \multicolumn{2}{c}{\textcolor{Red} {Sinus tracé en rouge}} \\
                    $\sin\quad : $ & $\mathbb{R} \longrightarrow \mathbb{R} $ \\
                                &  $x \longmapsto \sin x $ \\
                                & \\
                                & $ {\Large \mathcal{D}_{\cos}} = \mathbb{R} $ \\
                     \end{tabular}\\
\end{tabular}}

\bigskip 


% \definecolor{ttzzqq}{rgb}{0.2,0.6,0} % Vert agreable
% \definecolor{qqzzqq}{rgb}{0,0.6,0}   % Vert agreable
% \definecolor{ffefdv}{rgb}{1,0.94,0.84} % AntiqueWhite pour grid
\begin{tikzpicture}[line cap=round,line join=round,>=triangle 45,x=1.0cm,y=1.0cm,scale=.8]
% \draw [color=AntiqueWhite,, xstep=0.1cm,ystep=0.1cm] (-7.1,-1.54) grid (13.29,2.53);
\draw[->,color=black] (-7.1,0) -- (13.29,0);
\foreach \x in {-6,-4,-2,2,4,6,8,10,12}
\draw[shift={(\x,0)},color=black] (0pt,2pt) -- (0pt,-2pt);
\draw[->,color=black] (0,-1.54) -- (0,2.53);
\foreach \y in {-1,1,2}
\draw[shift={(0,\y)},color=black] (2pt,0pt) -- (-2pt,0pt);
\clip(-7.1,-1.54) rectangle (13.29,2.53);
\draw[color=Green, smooth,samples=100,domain=-7.097377544417069:13.285348399587015] plot(\x,{cos(((\x))*180/pi)});
\draw[color=red, smooth,samples=100,domain=-7.097377544417069:13.285348399587015] plot(\x,{sin(((\x))*180/pi)});

% \draw (-6.28,1.5)-- (-6.28,-2);
% \draw (0,2.39)-- (0,-2.27);
% \draw (12.57,1.5)-- (12.57,-2);
% \draw (6.28,2)-- (6.28,-2);


% Délimite les périodes 
\draw (-6.28,2)-- (-6.28,-2);
% \draw (0,2.39)-- (0,-2.27);
\draw (12.57,2)-- (12.57,-2);
\draw (6.28,2)-- (6.28,-2);

\draw (0,2) -- node[below, midway] {Période $2\pi$}  (6.28,2);

\begin{tiny}
% \draw  (1.05,0)-- ++(-1.0pt,0 pt) -- ++(2.0pt,0 pt) ++(-1.0pt,-1.0pt) -- ++(0 pt,2.0pt);
%\draw (pi/3,0) node [below]{$\dfrac{\pi}{3}$};
%\draw  (1.57,0)-- ++(-1.0pt,0 pt) -- ++(2.0pt,0 pt) ++(-1.0pt,-1.0pt) -- ++(0 pt,2.0pt);
%\draw (pi/2,0 ) node [below] {$\dfrac{\pi}{2}$};
%\draw  (3.67,0)-- ++(-1.0pt,0 pt) -- ++(2.0pt,0 pt) ++(-1.0pt,-1.0pt) -- ++(0 pt,2.0pt);
%\draw (2*pi/3,0) node [below]{$\dfrac{2\pi}{3}$};
%\draw  (4.71,0)-- ++(-1.0pt,0 pt) -- ++(2.0pt,0 pt) ++(-1.0pt,-1.0pt) -- ++(0 pt,2.0pt);
%\draw (3*pi/2,0) node [below] {$\dfrac{3\pi}{2}$};
%\draw  (5.76,0)-- ++(-1.0pt,0 pt) -- ++(2.0pt,0 pt) ++(-1.0pt,-1.0pt) -- ++(0 pt,2.0pt);
%\draw (5*pi/6,0) node [below] {$\dfrac{5\pi}{6}$};
%\draw [color=Green] (6.28,0)-- ++(-1.0pt,-1.0pt) -- ++(2.0pt,2.0pt) ++(-2.0pt,0) -- ++(2.0pt,-2.0pt);
%\draw  (2.62,0)-- ++(-1.0pt,0 pt) -- ++(2.0pt,0 pt) ++(-1.0pt,-1.0pt) -- ++(0 pt,2.0pt);
%\draw (11*pi/6,0) node [above]{$\dfrac{11\pi}{6}$};
%\draw (7*pi/6,0) node [above]{$\dfrac{7\pi}{6}$};
%\draw  (0.57,0)-- ++(-1.0pt,0 pt) -- ++(2.0pt,0 pt) ++(-1.0pt,-1.0pt) -- ++(0 pt,2.0pt);
%\draw (pi/6,0) node [below] {$\dfrac{\pi}{6}$};

\draw (0,0.5) node [left] {$\dfrac{1}{2}$};
\draw  (0,-0.5)-- ++(-1.0pt,0 pt) -- ++(2.0pt,0 pt) ++(-1.0pt,-1.0pt) -- ++(0 pt,2.0pt);
\draw (0,-0.5) node [left]{$-\dfrac{1}{2}$};
\end{tiny}
\begin{scriptsize}
\draw (pi,0) node [above]{$\pi$};
\draw (2*pi,0) node [below right]{$2\pi$};
\draw (0,1) node[right] {$1$};
\draw  (0,-1)-- ++(-1.0pt,0 pt) -- ++(2.0pt,0 pt) ++(-1.0pt,-1.0pt) -- ++(0 pt,2.0pt);
\draw (0,-1) node [left]{$-1$};
\end{scriptsize}
\fill [color=black,shift={(0.1,2)},rotate=90] (0,0) ++(0 pt,2.25pt) -- ++(1.95pt,-3.375pt)--++(-3.9pt,0 pt) -- ++(1.95pt,3.375pt);
\fill [color=black,shift={(6.18,2)},rotate=270] (0,0) ++(0 pt,2.25pt) -- ++(1.95pt,-3.375pt)--++(-3.9pt,0 pt) -- ++(1.95pt,3.375pt);


\draw [color=Green] (0.52-pi/2,0.5)-- ++(-1.0pt,-1.0pt) -- ++(2.0pt,2.0pt) ++(-2.0pt,0) -- ++(2.0pt,-2.0pt);


\begin{pgfonlayer}{background}  
% Attention l'ordre de ces lignes est important 
% Ne pas le modifier   
\draw[step=1mm,ultra thin,AntiqueWhite!10](-7,-2)  grid (14,2.6);
\draw[step=5mm,very thin,AntiqueWhite!30] (-7,-2)  grid (14,2.6);
\draw[step=1cm,very thin,AntiqueWhite!50] (-7,-2)  grid (14,2.6);
\draw[step=5cm,thin,AntiqueWhite]         (-7,-2)  grid (14,2.6);

\end{pgfonlayer} 

% \clip(-7.1,-1.54) rectangle (13.29,2.53);
% \draw [color=AntiqueWhite,, xstep=0.1cm,ystep=0.1cm] (-7.1,-1.54) grid (13.29,2.53);
%\draw [color=Green] (-6.28-pi/2,0)-- ++(-1.0pt,-1.0pt) -- ++(2.0pt,2.0pt) ++(-2.0pt,0) -- ++(2.0pt,-2.0pt);
%\draw [color=Green] (-5.76-pi/2,0.5)-- ++(-1.0pt,-1.0pt) -- ++(2.0pt,2.0pt) ++(-2.0pt,0) -- ++(2.0pt,-2.0pt);
%\draw [color=Green] (-4.71-pi/2,1)-- ++(-1.0pt,-1.0pt) -- ++(2.0pt,2.0pt) ++(-2.0pt,0) -- ++(2.0pt,-2.0pt);
%\draw [color=Green] (-3.67-pi/2,0.5)-- ++(-1.0pt,-1.0pt) -- ++(2.0pt,2.0pt) ++(-2.0pt,0) -- ++(2.0pt,-2.0pt);
%\draw [color=Green] (-2.62-pi/2,-0.5)-- ++(-1.0pt,-1.0pt) -- ++(2.0pt,2.0pt) ++(-2.0pt,0) -- ++(2.0pt,-2.0pt);
%\draw [color=Green] (-3.14-pi/2,0)-- ++(-1.0pt,-1.0pt) -- ++(2.0pt,2.0pt) ++(-2.0pt,0) -- ++(2.0pt,-2.0pt);
%\draw [color=Green] (-1.57-pi/2,-1)-- ++(-1.0pt,-1.0pt) -- ++(2.0pt,2.0pt) ++(-2.0pt,0) -- ++(2.0pt,-2.0pt);
%\draw [color=Green] (-0.52-pi/2,-0.5)-- ++(-1.0pt,-1.0pt) -- ++(2.0pt,2.0pt) ++(-2.0pt,0) -- ++(2.0pt,-2.0pt);
%\draw [color=Green] (1.05-pi/2,0.87)-- ++(-1.0pt,-1.0pt) -- ++(2.0pt,2.0pt) ++(-2.0pt,0) -- ++(2.0pt,-2.0pt);
%\draw [color=Green] (1.57-pi/2,1)-- ++(-1.0pt,-1.0pt) -- ++(2.0pt,2.0pt) ++(-2.0pt,0) -- ++(2.0pt,-2.0pt);
%\draw [color=Green] (2.09-pi/2,0.87)-- ++(-1.0pt,-1.0pt) -- ++(2.0pt,2.0pt) ++(-2.0pt,0) -- ++(2.0pt,-2.0pt);
%\draw [color=Green] (2.62-pi/2,0.5)-- ++(-1.0pt,-1.0pt) -- ++(2.0pt,2.0pt) ++(-2.0pt,0) -- ++(2.0pt,-2.0pt);
%\draw [color=Green] (3.14-pi/2,0)-- ++(-1.0pt,-1.0pt) -- ++(2.0pt,2.0pt) ++(-2.0pt,0) -- ++(2.0pt,-2.0pt);
%\draw [color=Green] (3.67-pi/2,-0.5)-- ++(-1.0pt,-1.0pt) -- ++(2.0pt,2.0pt) ++(-2.0pt,0) -- ++(2.0pt,-2.0pt);
%\draw [color=Green] (4.71-pi/2,-1)-- ++(-1.0pt,-1.0pt) -- ++(2.0pt,2.0pt) ++(-2.0pt,0) -- ++(2.0pt,-2.0pt);
%\draw [color=Green] (5.76-pi/2,-0.5)-- ++(-1.0pt,-1.0pt) -- ++(2.0pt,2.0pt) ++(-2.0pt,0) -- ++(2.0pt,-2.0pt);
%\draw  (0,0.5)-- ++(-1.0pt,0 pt) -- ++(2.0pt,0 pt) ++(-1.0pt,-1.0pt) -- ++(0 pt,2.0pt);
\end{tikzpicture}

\bigskip 

La représentation graphique de la fonction cosinus est obtenue par translation de la fonction sinus avec un décalage de $\dfrac{\pi}{2}$. \\

En effet, pour tout $x \in \mathbb{R}$ \\
\begin{quote}
$\cos \left(\dfrac{\pi}{2}-x\right) = \sin x$ \\

\smallskip 
$\sin \left(\dfrac{\pi}{2}-x\right) = \cos x$ \\
\end{quote}

\newpage

\subsubsection{Exercice}

\begin{tabular}{r@{\hspace*{3cm}}l}
    \begin{tabular}{rl}
    \multicolumn{2}{c}{\textcolor{Green} {$f$ tracé en vert}} \\
    $f \quad : $ & $\mathbb{R} \longrightarrow \mathbb{R} $ \\
                   &  $x \longmapsto f(x) = \cos x $ \\
                   & $f$ est périodique de période $2\pi$ \\ 
                   & \\
                   & $ {\Large \mathcal{D}_{f}} = \mathbb{R} $ \\
    \end{tabular} & 
                   \begin{tabular}{rl}
                     \multicolumn{2}{c}{\textcolor{Red} {$g$ tracé en rouge}} \\
                    $g \quad : $ & $\mathbb{R} \longrightarrow \mathbb{R} $ \\
                        &  $x \longmapsto \sin \left(\dfrac{1}{2} x\right) $ \\
                                & $g$ est périodique de période $\pi$ \\ 
                                & \\
                                & $ {\Large \mathcal{D}_{g}} = \mathbb{R} $ \\
                     \end{tabular}\\
\end{tabular}

% \definecolor{ttzzqq}{rgb}{0.2,0.6,0} % Vert agreable
% \definecolor{qqzzqq}{rgb}{0,0.6,0}   % Vert agreable
% \definecolor{ffefdv}{rgb}{1,0.94,0.84} % AntiqueWhite pour grid
\begin{tikzpicture}[line cap=round,line join=round,>=triangle 45,x=1.0cm,y=1.0cm,scale=.8]
% \draw [color=AntiqueWhite,, xstep=0.1cm,ystep=0.1cm] (-7.1,-1.54) grid (13.29,2.53);
\draw[->,color=black] (-7.1,0) -- (13.29,0);
\foreach \x in {-6,-4,-2,2,4,6,8,10,12}
\draw[shift={(\x,0)},color=black] (0pt,2pt) -- (0pt,-2pt);
\draw[->,color=black] (0,-1.54) -- (0,2.53);
\foreach \y in {-1,1,2}
\draw[shift={(0,\y)}] (2pt,0pt) -- (-2pt,0pt);
\clip(-7.1,-1.54) rectangle (13.29,2.80);
\draw[color=Green, smooth,samples=100,domain=-7.097377544417069:13.285348399587015] plot(\x,{cos(((\x))*180/pi)});
\draw[color=red, smooth,samples=100,domain=-7.097377544417069:13.285348399587015] plot(\x,{sin(((\x/2))*180/pi)});

% Délimite les périodes 
\draw (-6.28,2)-- (-6.28,-2);
% \draw (0,2.39)-- (0,-2.27);
\draw (12.57,2)-- (12.57,-2);
\draw (6.28,2)-- (6.28,-2);

\draw (0,2) -- node[above, midway] {La période de $f \text{ est } 2\pi$}  (6.28,2);
\draw [<->] (-2*pi,1.5) -- node[below, midway] {La période de $g \text{ est } 4\pi$}  (6.28,1.5);

\begin{tiny}
% \draw  (1.05,0)-- ++(-1.0pt,0 pt) -- ++(2.0pt,0 pt) ++(-1.0pt,-1.0pt) -- ++(0 pt,2.0pt);
% \draw (pi/3,0) node [below]{$\dfrac{\pi}{3}$};
% \draw  (1.57,0)-- ++(-1.0pt,0 pt) -- ++(2.0pt,0 pt) ++(-1.0pt,-1.0pt) -- ++(0 pt,2.0pt);
% \draw (pi/2,0 ) node [below] {$\dfrac{\pi}{2}$};
% \draw  (3.67,0)-- ++(-1.0pt,0 pt) -- ++(2.0pt,0 pt) ++(-1.0pt,-1.0pt) -- ++(0 % pt,2.0pt);
% \draw (2*pi/3,0) node [below]{$\dfrac{2\pi}{3}$};
% \draw  (4.71,0)-- ++(-1.0pt,0 pt) -- ++(2.0pt,0 pt) ++(-1.0pt,-1.0pt) -- ++(0 pt,2.0pt);
% \draw (3*pi/2,0) node [below] {$\dfrac{3\pi}{2}$};
% \draw  (5.76,0)-- ++(-1.0pt,0 pt) -- ++(2.0pt,0 pt) ++(-1.0pt,-1.0pt) -- ++(0 pt,2.0pt);
% \draw (5*pi/6,0) node [below] {$\dfrac{5\pi}{6}$};
% \draw [color=Green] (6.28,0)-- ++(-1.0pt,-1.0pt) -- ++(2.0pt,2.0pt) ++(-2.0pt,0) -- ++(2.0pt,-2.0pt);
% \draw  (2.62,0)-- ++(-1.0pt,0 pt) -- ++(2.0pt,0 pt) ++(-1.0pt,-1.0pt) -- ++(0 pt,2.0pt);
% \draw (11*pi/6,0) node [above]{$\dfrac{11\pi}{6}$};
% \draw (7*pi/6,0) node [above]{$\dfrac{7\pi}{6}$};
% \draw  (0.57,0)-- ++(-1.0pt,0 pt) -- ++(2.0pt,0 pt) ++(-1.0pt,-1.0pt) -- ++(0 pt,2.0pt);
% \draw (pi/6,0) node [below] {$\dfrac{\pi}{6}$};

\draw (0,0.5) node [left] {$\dfrac{1}{2}$};
\draw  (0,-0.5)-- ++(-1.0pt,0 pt) -- ++(2.0pt,0 pt) ++(-1.0pt,-1.0pt) -- ++(0 pt,2.0pt);
\draw (0,-0.5) node [left]{$-\dfrac{1}{2}$};
\end{tiny}
\begin{scriptsize}
\draw (pi,0) node [above]{$\pi$};
\draw (2*pi,0) node [below right]{$2\pi$};
\draw (0,1) node[right] {$1$};
\draw  (0,-1)-- ++(-1.0pt,0 pt) -- ++(2.0pt,0 pt) ++(-1.0pt,-1.0pt) -- ++(0 pt,2.0pt);
\draw (0,-1) node [left]{$-1$};
\end{scriptsize}

\fill [color=black,shift={(0.1,2)},rotate=90] (0,0) ++(0 pt,2.25pt) -- ++(1.95pt,-3.375pt)--++(-3.9pt,0 pt) -- ++(1.95pt,3.375pt);
\fill [color=black,shift={(6.18,2)},rotate=270] (0,0) ++(0 pt,2.25pt) -- ++(1.95pt,-3.375pt)--++(-3.9pt,0 pt) -- ++(1.95pt,3.375pt);


\draw [color=Green] (0.52-pi/2,0.5)-- ++(-1.0pt,-1.0pt) -- ++(2.0pt,2.0pt) ++(-2.0pt,0) -- ++(2.0pt,-2.0pt);
\begin{pgfonlayer}{background}  
% Attention l'ordre de ces lignes est important 
% Ne pas le modifier   
\draw[step=1mm,ultra thin,AntiqueWhite!10](-7,-2)  grid (14,2.6);
\draw[step=5mm,very thin,AntiqueWhite!30] (-7,-2)  grid (14,2.6);
\draw[step=1cm,very thin,AntiqueWhite!50] (-7,-2)  grid (14,2.6);
\draw[step=5cm,thin,AntiqueWhite]         (-7,-2)  grid (14,2.6);

\end{pgfonlayer} 

%\draw [color=Green] (-6.28-pi/2,0)-- ++(-1.0pt,-1.0pt) -- ++(2.0pt,2.0pt) ++(-2.0pt,0) -- ++(2.0pt,-2.0pt);
%\draw [color=Green] (-5.76-pi/2,0.5)-- ++(-1.0pt,-1.0pt) -- ++(2.0pt,2.0pt) ++(-2.0pt,0) -- ++(2.0pt,-2.0pt);
%\draw [color=Green] (-4.71-pi/2,1)-- ++(-1.0pt,-1.0pt) -- ++(2.0pt,2.0pt) ++(-2.0pt,0) -- ++(2.0pt,-2.0pt);
%\draw [color=Green] (-3.67-pi/2,0.5)-- ++(-1.0pt,-1.0pt) -- ++(2.0pt,2.0pt) ++(-2.0pt,0) -- ++(2.0pt,-2.0pt);
%\draw [color=Green] (-2.62-pi/2,-0.5)-- ++(-1.0pt,-1.0pt) -- ++(2.0pt,2.0pt) ++(-2.0pt,0) -- ++(2.0pt,-2.0pt);
%\draw [color=Green] (-3.14-pi/2,0)-- ++(-1.0pt,-1.0pt) -- ++(2.0pt,2.0pt) ++(-2.0pt,0) -- ++(2.0pt,-2.0pt);
%\draw [color=Green] (-1.57-pi/2,-1)-- ++(-1.0pt,-1.0pt) -- ++(2.0pt,2.0pt) ++(-2.0pt,0) -- ++(2.0pt,-2.0pt);
%\draw [color=Green] (-0.52-pi/2,-0.5)-- ++(-1.0pt,-1.0pt) -- ++(2.0pt,2.0pt) ++(-2.0pt,0) -- ++(2.0pt,-2.0pt);
%\draw [color=Green] (1.05-pi/2,0.87)-- ++(-1.0pt,-1.0pt) -- ++(2.0pt,2.0pt) ++(-2.0pt,0) -- ++(2.0pt,-2.0pt);
%\draw [color=Green] (1.57-pi/2,1)-- ++(-1.0pt,-1.0pt) -- ++(2.0pt,2.0pt) ++(-2.0pt,0) -- ++(2.0pt,-2.0pt);
%\draw [color=Green] (2.09-pi/2,0.87)-- ++(-1.0pt,-1.0pt) -- ++(2.0pt,2.0pt) ++(-2.0pt,0) -- ++(2.0pt,-2.0pt);
%\draw [color=Green] (2.62-pi/2,0.5)-- ++(-1.0pt,-1.0pt) -- ++(2.0pt,2.0pt) ++(-2.0pt,0) -- ++(2.0pt,-2.0pt);
%\draw [color=Green] (3.14-pi/2,0)-- ++(-1.0pt,-1.0pt) -- ++(2.0pt,2.0pt) ++(-2.0pt,0) -- ++(2.0pt,-2.0pt);
%\draw [color=Green] (3.67-pi/2,-0.5)-- ++(-1.0pt,-1.0pt) -- ++(2.0pt,2.0pt) ++(-2.0pt,0) -- ++(2.0pt,-2.0pt);
%\draw [color=Green] (4.71-pi/2,-1)-- ++(-1.0pt,-1.0pt) -- ++(2.0pt,2.0pt) ++(-2.0pt,0) -- ++(2.0pt,-2.0pt);
%\draw [color=Green] (5.76-pi/2,-0.5)-- ++(-1.0pt,-1.0pt) -- ++(2.0pt,2.0pt) ++(-2.0pt,0) -- ++(2.0pt,-2.0pt);
%\draw  (0,0.5)-- ++(-1.0pt,0 pt) -- ++(2.0pt,0 pt) ++(-1.0pt,-1.0pt) -- ++(0 pt,2.0pt);
\end{tikzpicture}

{\renewcommand{\arraystretch }{2}
\begin{tabular}{c@{$\, \longrightarrow\,$}c@{$\, \longrightarrow \,$}c
@{\hspace*{3cm}}lc@{$ \; = \; $}c}
$2\pi$ & $\pi $  & $0$ &$x=0$ & $\sin 0 $ & $0$ \\
$\pi$ & $ \dfrac{ \pi}{2} $  & $1$ &$x=\dfrac{\pi}{3}$ & $\sin \dfrac{\pi}{6} $ & $\dfrac{1}{2}$ \\
$\dfrac{\pi}{2}$ & $\dfrac{\pi}{4}$  & $\dfrac{\sqrt{2}}{2}$ &$x=\pi$ & $\sin \dfrac{\pi}{2} $ & $1$ \\
$\dfrac{\pi}{3}$ & $\dfrac{\pi}{6}$  & $\dfrac{1}{2}$ &$x=\dfrac{5\pi}{3}$ & $\sin \dfrac{5\pi}{6} $ & $\dfrac{1}{2}$ \\
\multicolumn{3}{c}{}&$x=2\pi$ & $\sin \pi $ & $0$ \\
\end{tabular}
}\renewcommand{\arraystretch }{1}


\ifdefined\COMPLETE
\else
    \end{document}
\fi